Chapter 1 discusses the space $ac$ and its subspace $ac_0$, gives a criterion for almost convergence to zero
(i.e., belonging to the space $ac_0$) for a nonnegative sequence.

\reflecttheorem{thm:M_j_ac0_inf_lim}
	Let $n_i$ be a strictly increasing sequence of natural numbers,
	\begin{equation}
		\label{eq:definition_M_j}
		M(j) = \liminf_{i\to\infty} n_{i+j} - n_i,
	\end{equation}
	\begin{equation}
		x_k = \left\{\begin{array}{ll}
			1, & \mbox{~if~} k = n_i
			\\
			0  & \mbox{~otherwise~}
		\end{array}\right.
	\end{equation}
	Then the following conditions are equivalent:
	\\
	(i)   $x \in ac_0$;
	\\\\
	(ii)  $\lim\limits_{j \to \infty} \dfrac{M(j)}{j} = +\infty$;
	\\\\
	(iii) $\inf\limits_{j \in \N}     \dfrac{M(j)}{j} = +\infty$.


\reflecttheorem{thm:crit_ac0_Mj_lambda}
	Let $x\in\ell_\infty$, $x \geq 0$, $\lambda>0$.
	Denote by $n^{(\lambda)}_i$ an increasing sequence
	of indices of elements of $x$ such that $x_k \geq \lambda$ if and only if
	$k=n^{(\lambda)}_i$ for some $i$.
	Denote
	\begin{equation}
		M^{(\lambda)}(j) = \liminf_{i\to\infty} n^{(\lambda)}_{i+j} - n^{(\lambda)}_i
		.
	\end{equation}


	Then for $x\in ac_0$, it is necessary and sufficient that
	for any $\lambda>0$ one has
	\begin{equation}
		\lim_{j \to \infty} \frac{M^{(\lambda)}(j)}{j} = +\infty
		.
	\end{equation}

\reflecttheorem{thm:rho_x_c_leq_alpha_t_s_x_united}
	For any $x\in ac$
	\begin{equation}
		\frac{1}{2} \alpha(x) \leq \rho(x,c)\leq \lim_{s\to\infty} \alpha(T^s x)
		.
	\end{equation}

\reflectcorollary{cor:rho_x_c0_leq_alpha_t_s_x_united}
	For any $x\in ac_0$
	\begin{equation}
		\frac{1}{2} \alpha(x) \leq \rho(x,c_0)\leq \lim_{s\to\infty} \alpha(T^s x)
		.
	\end{equation}

\reflecttheorem{thm:Connor_generalized}
	The measure of the set $F=\{x\in\Omega : q(x) = 0 \wedge p(x)= 1\}$,
	where $p(x)$ and $q(x)$ are the upper and lower Sucheston functionals respectively,
	is equal to 1.
