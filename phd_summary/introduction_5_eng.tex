Chapter 5 is devoted to the connection between the multiplicative properties of the support of a 0-1-sequence
and the values that the Sucheston functionals can take on such a sequence.

\reflectcorollary{cor:ac0_powers_finite_set_of_numbers}
	Let $\{p_1, ..., p_k\} \subset \N$,
	\begin{equation}
		x_k = \begin{cases}
			1, &\mbox{if~} k = p_1^{j_1}\cdot p_2^{j_2}\cdot ... \cdot p_k^{j_k} \mbox{~for some~} j_1,...,j_k\in\N,
			\\
			0  &\mbox{otherwise}.
		\end{cases}
	\end{equation}
	Then $x\in ac_0$.

\reflectdefinition{def:P-property}
	The set $A\subset\N$ has the $P$-property if for any $n\in\N$ there exists a set of pairwise coprime numbers
	\begin{equation}
		\{a_{n,1}, a_{n,2}, ..., a_{n,n}  \} \subset A
		.
	\end{equation}

\reflecttheorem{thm:p_x_infinite_multiples}
	Let $A\subset \N\setminus\{1\}$.
	Then the following conditions are equivalent:
	\begin{enumerate}[label=(\roman*)]
		\item
			$A$ has the $P$-property
		\item
			There exists an infinite subset of pairwise coprime numbers in $A$
		\item
			$p(\chi\mathscr{M}A)=1$.
	\end{enumerate}

\reflectcorollary{cor:ac0_primes_p_psi_A_prod}
	Let $A = \{a_1, a_2, ..., a_n,...\}$ be an infinite set of pairwise coprime numbers
	with $a_{n+1}>a_1\cdot...\cdot a_n$.
	Then
	\begin{equation}
		q(\chi\mathscr{M}A) = 1-\prod_{j=1}^\infty \left(1-\frac{1}{a_j}\right)
		.
	\end{equation}

\reflectlemma{lem:q_x_infinite_Euler}
	Let $\varepsilon \in  (0; 1{]}$.
	There exists an infinite family of pairwise disjoint subsets of prime numbers
	$A_i$ such that $q(\chi\mathscr{M}A_i)\geq\varepsilon$ for each $i\in\N$.
