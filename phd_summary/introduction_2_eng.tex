
В главе 2
%TODO: \ref ???
изучается $\alpha$--функция, введённая в~\cite{our-vzms-2018}:
\begin{equation}
	\alpha(x) = \varlimsup_{i\to\infty} \max_{i<j\leqslant 2i} |x_i - x_j|
	.
\end{equation}
%
%TODO: ссылка на статью Семенова!
%
Поскольку $\alpha(c)=0$,
то $\alpha$--функцию также можно считать <<мерой несходимости>> последовательности;
равенство $\alpha(x) = 0$, однако, вовсе не гарантирует сходимость.

Устанавливается, что $\alpha$--функция не инвариантна относительно оператора сдвига $T$,
и даётся оценка на $\alpha(T^n x)$.
С другой стороны, $\alpha$--функция, в отличие от некоторых банаховых пределов
\cite{Semenov2010invariant,Semenov2011dan},
инвариантна относительно операторов растяжения $\sigma_n$.
Затем выявляется связь между $\alpha$--функцией, расстоянием от заданнной последовательности до пространства $c$
и почти сходимостью.
Рассмотрены и другие свойства $\alpha$--функции.
Приведём основные результаты.

\reflectcorollary{thm:est_alpha_Tn_x_full}
	Для любых $x\in\ell_\infty$ и $n \in \N$
	\begin{equation}\label{est_alpha_Tn_x}
		\frac{1}{2}\alpha(x) \leq \alpha(T^n x) \leq \alpha(x)
		.
	\end{equation}

\reflecttheorem{thm:alpha_beta_T_seq}
	Пусть $\beta_k$~--- монотонная невозрастающая последовательность,
	$\beta_k \to \beta$, $\beta\in\left[\frac{1}{2}; 1\right]$, $\beta_1 \leq 1$.
	Тогда существует такой $x\in\ell_\infty$, что для любого натурального $n$
	\begin{equation}
		\frac{\alpha(T^n x)}{\alpha(x)} = \beta_n.
	\end{equation}

\reflecttheorem{thm:alpha_sigma_n}
	Для любого $x\in\ell_\infty$ и для любого натурального $n$ верно равенство
	\begin{equation}
		\alpha(\sigma_n x) = \alpha(x)
		.
	\end{equation}

\reflecttheorem{thm:alpha_sigma_1_n}
	Для любого $n\in\N$ и любого $x\in\ell_\infty$ выполнено
	\begin{equation}
		\alpha(\sigma_{1/n} x) \leq \left( 2- \frac{1}{n} \right) \alpha(x)
		.
	\end{equation}

\reflecttheorem{thm:alpha_Cx_no_gamma}
	Имеет место равенство
	\begin{equation}
		\sup_{x\in\ell_\infty, \alpha(x)\neq 0} \frac{\alpha(Cx)}{\alpha(x)}=1
		.
	\end{equation}

\reflecttheorem{thm:alpha_xy}
	Пусть $(x\cdot y)_k = x_k\cdot y_k$.
	Тогда
	$\alpha(x\cdot y)\leq \alpha(x)\cdot \|y\|_* + \alpha(y)\cdot \|x\|_*$,
	где
	\begin{equation}
		\|x\|_* = \limsup_{k\to\infty} |x_k|
	\end{equation}
	есть  фактор-норма по $c_0$ на пространстве $\ell_\infty$.

В параграфе~\ref{sec:space_A0} исследуется пространство $A_0 = \{x: \alpha(x) = 0\}$.
Это пространство несепарабельно, замкнуто относительно покоординатного умножения,
операторов левого и правого сдвигов, оператора Чезаро,
операторов растяжения $\sigma_n$ и усредняющего сжатия $\sigma_{1/n}$.

В параграфе~\ref{sec:noncomplementarity} устанавливается, что в цепочке вложений
\begin{equation}
	c_0 \subset A_0 \subset \ell_\infty
\end{equation}
оба подпространства недополняемы.
