Convergent sequences, i.e., sequences that have a limit in the sense of classical mathematical analysis,
are studied sufficiently well.
In particular, any convergent sequence is bounded.
We will denote the space of bounded sequences, following the classics \cite{wojtaszczyk1996banach,lindenstrauss1973classical},
by $\ell_\infty$ and equip it with the norm
\begin{equation*}
	\|x\| = \sup_{n\in\mathbb{N}}|x_n|
	.
\end{equation*}

However, in applications, bounded sequences often arise
that are not convergent.
In such a case, it is natural to ask:
how to measure the <<lack of convergence>>?
<<How much does the sequence diverge?>>

The most obvious seems to be computing the distance $\rho(x,c)$ from a given element $x\in\ell_\infty$
to the space of convergent sequences $c$
(which equals half the difference of the upper and lower limits of the sequence).
However, it turns out that other approaches exist.

It is easy to notice that the operation of taking the classical limit on the space of convergent sequences
is a continuous (in the $\ell_\infty$ norm) linear functional.
In 1929, S. Mazur announced~\cite{Mazur_eng},
that this functional can be continuously extended to the entire space $\ell_\infty$
(the proof is given in S. Banach's book~\cite{banach1932theorie_eng}).
Based on this idea, Banach limits
(sometimes also called Banach--Mazur limits \cite{alekhno2012superposition,alekhno2015banach})
are defined as follows.

A Banach limit is a functional $B\in \ell_\infty^*$ such that:
\begin{enumerate}
	\item
		$B \geqslant 0$
	\item
		$B\one = 1$
	\item
		$B=BT$
\end{enumerate}

Basic properties:
\begin{itemize}
	\item
		$\|B\|_{\ell_\infty^*} = 1$
	\item
		$Bx = \lim\limits_{n\to\infty} x_n$ for any $x=(x_1, x_2, ...) \in c$.
\end{itemize}

Thus,
the Banach limit is indeed a natural generalization of the concept of the limit of a convergent sequence
to all bounded sequences.

The set of Banach limits is usually denoted by $\mathfrak{B}$
(rarely by $BM$~--- see, for example~\cite{alekhno2012superposition,alekhno2015banach}).

Lorentz \cite{lorentz1948contribution} established that there exists a subspace $\ell_\infty$,
on which all Banach limits take the same value.
This space is called the space of almost convergent sequences and is usually denoted by $ac$
(from <<almost convergent>>).
The embedding $c \subset ac$ is proper, i.e., $ac \setminus c \neq \varnothing$.
Moreover, Lorentz proved that for given $t\in\R$ and $x\in\ell_\infty$, the equality $Bx=t$ holds for all $B\in\mathfrak{B}$
if and only if
\begin{equation}
	\label{eq:crit_Lorentz}
	\lim_{n\to\infty} \frac{1}{n} \sum_{k=m+1}^{m+n} x_k = t
\end{equation}
uniformly in $m\in\N$.
This statement is called the Lorentz criterion.


Generalizing the Lorentz criterion, Sucheston \cite{sucheston1967banach} proved that for any $x\in\ell_\infty$
and any $B\in\mathfrak{B}$
\begin{equation*}
	q(x) =
	\lim_{n\to\infty} \sup_{m\in\mathbb{N}} \frac{1}{n} \sum_{k=m+1}^{m+n} x_k
	\leq
	Bx
	\leq
	\lim_{n\to\infty} \inf_{m\in\mathbb{N}} \frac{1}{n} \sum_{k=m+1}^{m+n} x_k
	= p(x)
\end{equation*}
and, moreover,
\begin{equation*}
	\mathfrak{B}x = [q(x), p(x)]
	.
\end{equation*}


For a more detailed review of early research on Banach limits,
we refer the reader to~\cite{greenleaf1969invariant,day1973normed,kangro1976theory}.
Shortly after Sucheston's works, J. Kurtz extended the notion of Banach limits
to vector sequences~\cite{kurtz1970almost},
and then to sequences in arbitrary Banach spaces~\cite{kurtz1972almost}.
For a discussion of Banach limits in vector spaces, we refer the reader
to~\cite{deeds1968summability,hajdukovic1975almost,armario2013vector,garcia2015extremal,garcia2016fundamental}.
In the recent work~\cite{chen2007characterizations}, C.-P. Chen and M.-K. Kuo study generalizations of Banach limits
to arbitrary Hilbert spaces and to spaces of summable functions $L_p$.
Other generalizations of Banach limits are discussed in
\cite{hajdukovic1975functionals,koga2016generalization}.


Another fairly fruitful generalization of Banach limits turned out to be their analogues on double sequences~\cite{robison1926divergent},
introduced by J. D. Hill in~\cite{hill1965almost}.
For further results in this direction, we refer the reader
to~\cite{moricz1988almost,bacsarir1995strong,mursaleen2003almost,edely2004almost,mursaleen2004almost}.
Among recent works, the paper by M. Mursaleen and S. A. Mohiuddine~\cite{mursaleen2012banach}
is worth noting separately,
in which using the concept of almost convergence in the space of bounded double sequences
introduces a number of new interesting subspaces.

Finally, if we exclude the requirement of shift-invariance from the definition of a Banach limit,
then we get an object called a generalized limit,
studied in detail by M. Jerison in~\cite{jerison1957set} and many other works.


Thus, the question: <<How much does the sequence diverge?>> %~---
can be answered in terms of almost convergence, i.e., belonging to the space $ac$,
and the question: <<How much does the sequence not almost converge?>>~---
by naming the length of the interval $[q(x), p(x)]$.
Further, the space of almost convergent sequences has repeatedly become the subject of
various studies
\cite{semenov2006space,usachev2008transformations}.
In particular, in~\cite{connor1990almost}, it is proved
that almost all 0-1-sequences are not in the space $ac$.
This fact demonstrates that almost convergent sequences are <<quite rare>>.


Banach limits have also found their application in applications
\cite{semenov2015banachtraces,semenov2009fourier,strukova2015spectres}.

In the present work, some questions of asymptotic characteristics of bounded sequences are considered,
including Banach limits.
The numbers of the theorems, lemmas, definitions, and corollaries presented below corresponds with their numbers in the thesis.
