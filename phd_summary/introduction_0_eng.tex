Сходящиеся последовательности, т.е. последовательности, имеющие предел в смысле классического математического анализа,
изучены достаточно хорошо.
В частности, любая сходящаяся последовательность является ограниченной.
Пространство ограниченных последовательностей будем, вслед за классиками \cite{wojtaszczyk1996banach,lindenstrauss1973classical},
обозначать через $\ell_\infty$ и снабжать его нормой
\begin{equation*}
	\|x\| = \sup_{n\in\mathbb{N}}|x_n|
	.
\end{equation*}

Однако в приложениях часто возникают ограниченные последовательности,
которые не являются сходящимися.
В таком случае возникает закономерный вопрос:
как измерить <<недостаток сходимости>>?
<<насколько не сходится>> последовательность?

Наиболее очевидным кажется вычисление расстояния $\rho(x,c)$ от заданного элемента $x\in\ell_\infty$
до пространства сходящихся последовательностей $c$
(которое равно половине разности верхнего и нижнего пределов последовательности).
Однако выясняется, что имеют место быть и другие подходы.

Нетрудно заметить, что операция взятия классического предела на пространстве сходящихся последовательностей
является непрерывным (в норме $\ell_\infty$) линейным функционалом.
В 1929 г. С. Мазур анонсировал~\cite{Mazur},
что этот функционал может быть непрерывно продолжен на всё пространство $\ell_\infty$
(доказательство приведено в книге С.~Банаха~\cite{banach2001theory_rus}).
На основе этой идеи были определены банаховы пределы
(иногда также называемые пределами Банаха--Мазура \cite{alekhno2012superposition,alekhno2015banach})
следующим образом.

Банаховым пределом называется функционал $B\in \ell_\infty^*$ такой, что:
\begin{enumerate}
	\item
		$B \geqslant 0$
	\item
		$B\one = 1$
	\item
		$B=BT$
\end{enumerate}

Простейшие свойства:
\begin{itemize}
	\item
		$\|B\|_{\ell_\infty^*} = 1$
	\item
		$Bx = \lim\limits_{n\to\infty} x_n$ для любого $x=(x_1, x_2, ...) \in c$.

		Таким образом,
		банахов предел~--- действительно естественное обобщение понятия предела сходящейся последовательности
		на все ограниченные последовательности.
\end{itemize}

Множество банаховых пределов обычно обозначают через $\mathfrak{B}$
(реже через $BM$~--- см., например~\cite{alekhno2012superposition,alekhno2015banach}).

Лоренц \cite{lorentz1948contribution} установил, что существует подпространство $\ell_\infty$,
на котором все банаховы пределы принимают одинаковое значение.
Это пространство названо пространством почти сходящихся последовательностей и обычно обозначается через $ac$
(от англ. <<almost convergent>>).
Включение $c \subset ac$ собственное, т.е. $ac \setminus c \neq \varnothing$.
Более того, Лоренц доказал, что для заданных $t\in\R$ и $x\in\ell_\infty$ равенство $Bx=t$ выполнено для всех $B\in\mathfrak{B}$
тогда и только тогда, когда
\begin{equation}
	\label{eq:crit_Lorentz}
	\lim_{n\to\infty} \frac{1}{n} \sum_{k=m+1}^{m+n} x_k = t
\end{equation}
равномерно по $m\in\N$.
Это утверждение называют критерием Лоренца.


Обобщая критерий Лоренца, Сачестон \cite{sucheston1967banach} доказал, что для любого $x\in\ell_\infty$
и любого $B\in\mathfrak{B}$
\begin{equation*}
	q(x) =
	\lim_{n\to\infty} \sup_{m\in\mathbb{N}} \frac{1}{n} \sum_{k=m+1}^{m+n} x_k
	\leq
	Bx
	\leq
	\lim_{n\to\infty} \inf_{m\in\mathbb{N}} \frac{1}{n} \sum_{k=m+1}^{m+n} x_k
	= p(x)
\end{equation*}
и, более того,
\begin{equation*}
	\mathfrak{B}x = [q(x), p(x)]
	.
\end{equation*}


За более подробным обзором ранних исследований банаховых пределов отсылаем читателя к~\cite{greenleaf1969invariant,day1973normed,kangro1976theory}.
%Source: https://encyclopediaofmath.org/wiki/Banach_limit
Вскоре после работ Сачестона Дж. Куртц распространил понятие банаховых пределов
на векторные последовательности~\cite{kurtz1970almost},
а затем и на последовательности в произвольных банаховых пространствах~\cite{kurtz1972almost}.
За обсуждением банаховых пределов в векторных пространствах отсылаем читателя
к~\cite{deeds1968summability,hajdukovic1975almost,armario2013vectorvalued_rus,garcia2015extremal,garcia2016fundamental_rus}.
%Тут есть ещё ссылки: https://www.mathnet.ru/php/archive.phtml?wshow=paper&jrnid=faa&paperid=3146&option_lang=rus
%TODO2: В том числе и про то, где применяется.
В недавней работе~\cite{chen2007characterizations} Ч.~Чен и М.~Куо изучают обобщения банаховых пределов
на произвольные гильбертовы пространства и на пространства суммируемых функций $L_p$.
Другим обобщениям банаховых пределов посвящены работы
\cite{hajdukovic1975functionals,koga2016generalization}.
%tanaka2018banach - иероглифы, несовместимые с библиографией

Ещё одним достаточно плодотворным обобщением банаховых пределов оказались их аналоги на двойных последовательностях~\cite{robison1926divergent}, введённые Дж.~Д.~Хиллом в~\cite{hill1965almost}.
За дальнейшими результатами в этом направлении отсылаем читателя
к~\cite{moricz1988almost,bacsarir1995strong,mursaleen2003almost,edely2004almost,mursaleen2004almost}.
Из недавних работ стоит отдельно отметить статью М. Мурсалена и С.А. Мухиддина~\cite{mursaleen2012banach},
в которой с помощью понятия почти сходимости в пространстве ограниченных двойных последовательностей вводится ряд новых интересных подпространств.

Наконец, если исключить из определения банахова предела требование трансляционной инвариантности,
то мы получим объект, называемый обобщённым пределом,
подробно изучавшийся М.~Джерисоном в~\cite{jerison1957set} и многих других работах.


Таким образом, на вопрос: <<Насколько не сходится последовательность?>> %~---
можно дать ответ в терминах почти сходимости, т.е. принадлежности пространству $ac$,
а на вопрос: <<Насколько почти не сходится последовательность?>>~---
назвать длину отрезка $[q(x), p(x)]$.
В дальнейшем пространство почти сходящихся последовательностей неоднократно становилось предметом
различных исследований
\cite{semenov2006ac,usachev2008transforms}.
В частности, в работе~\cite{connor1990almost} доказано,
что последовательность из нулей и единиц почти наверное не принадлежит пространству $ac$.
Этот факт демонстрирует, что почти сходящиеся последовательности <<достаточно редки>>.

%TODO: ссылки! Хватит или ещё?

Банаховы пределы также нашли своё применение в приложениях
\cite{semenov2015banachtraces,SU,strukova2015spectres}.
%Известны обобщения банаховых пределов на двойные последовательности
%\cite{edely2004almost}.

%TODO: ссылки! Хватит или ещё?


В настоящей работе рассматриваются некоторые вопросы асимптотических характеристик ограниченных последовательностей,
в том числе банаховых пределов.
Нумерация приводимых ниже теорем, лемм, определений и следствий совпадает с их нумерацией в диссертации.
