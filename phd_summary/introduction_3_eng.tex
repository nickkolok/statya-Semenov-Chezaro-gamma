As stated above, Banach limits (like the ordinary limit on the space $c$)
are invariant with respect to the shift operator by their definition.
Then, an essential question arises: can we require a Banach limit to preserve its value
when composed with some other operators on $\ell_\infty$?
This problem was investigated by W. Eberlein in 1950 \cite{Eberlein},
i.e., two years after the classical work of G. G. Lorentz \cite{lorentz1948contribution}.
Eberlein established that there exist such linear operators $A : \ell_\infty\to \ell_\infty$,
for which $BAx = Bx$ independently of the choice of $x$ and for Banach limits of a special type.

We will say that $B\in\mathfrak B(A)$, $A : \ell_\infty\to \ell_\infty$, if for any $x\in \ell_\infty$
the equality $BAx = Bx$ holds.
Such a Banach limit $B$ is called invariant with respect to the operator $A$.

Can we highlight some special properties of the shift operator
that are necessary or sufficient for an operator so that all or some Banach limits are invariant with respect to it?
It is clear that if the operator $A$ is such that for any $x\in\ell_\infty$ between $Ax$ and $x$
there exists a (finite) Damerau–Levenshtein distance \cite{damerau1964technique}
(i.e., the minimum number of insertion, deletion, substitution,
and transposition of two neighboring elements operations necessary to transform $x$ into $Ax$,
and for different $x\in\ell_\infty$ these operations are not necessarily the same),
then any Banach limit is invariant with respect to this operator.
A similar statement holds if $Ax -x \in c_0$ for any $x\in \ell_\infty$.

The next most natural action (after shift and replacement of a finite number of elements)
that preserves the convergence of a sequence is repetition of sequence elements (also known as sequence dilation),
for example, the operator
\begin{equation}
	\sigma_2(x_1,x_2,x_3,...) = (x_1,x_1, \; x_2, x_2, \; x_3, x_3, \; ...)
	.
\end{equation}
However, with respect to such an operator, only some Banach limits are invariant.
Thus, in~\cite[theorem 14]{alekhno2017order} it is shown that
\begin{equation}
	\B(\sigma_n) \cap \ext \B = \varnothing  \mbox{~~for any~~} n\in\N_2
	.
\end{equation}
Note that if we consider the nonuniform dilation operator
\begin{equation}
	\sigma_{1,2}(x_1,x_2,x_3,x_4,x_5,...) = (x_1, \; x_2, x_2, \;  x_3, \; x_4, x_4, \; x_5, ...)
	,
\end{equation}
then we see that for the periodic sequence $y_n = (-1)^n$, $y\in ac_0$, the operator $\sigma_{1,2}$
transforms it into the periodic sequence
\begin{equation}
	(-1, 1, 1, \; -1, 1, 1, \; ...) \in ac_{1/3}
	,
\end{equation}
since on any periodic sequence, any Banach limit takes the value equal to the period average.
Thus, there are no Banach limits invariant with respect to the operator $\sigma_{1,2}$.

In Chapter 3, some examples of operators are presented, and the sets of Banach limits
invariant with respect to these operators are found.
Then the following classes of linear operators $H:\ell_\infty \to \ell_\infty$ are considered:

-- semi-Eberlein: such that $B_1 H \in\B$ for some $B_1\in \B$;

-- Eberlein: such that $B_1 H = B_1$ for some $B_1\in \B$;

-- B-regular: such that $B_1 H \in \B$ for any $B_1\in \B$;

-- essentially Eberlein: such that $B_1 H \in \B$ for any $B_1\in \B$ and $B_2 H \ne B_2$ for some $B_2\in \B$.

It is established (see theorems~\ref{thm:amiable_but_not_Eberlein_exists} and~\ref{thm:Eberlein_but_not_B-regular_exists}),
that each subsequent class is embedded in the previous one and does not coincide with it.
In addition, a number of related results are proved; in particular, the inverse invariance problem is solved.

\reflecttheorem{thm:generated_operator_G_B}
	For each $B\in \B$ there exists an operator $G_B:\ell_\infty \to \ell_\infty$
	such that $\B(G_B) = \{B\}$.
