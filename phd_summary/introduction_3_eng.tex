
Как сказано выше, банаховы пределы по определению (как и обычный предел на пространстве $c$) инвариантны относительно оператора сдвига.
Возникает закономерный вопрос: можно ли потребовать от банахова предела сохранять своё значение
при суперпозиции с некоторыми другими операторами на $\ell_\infty$?
Эту проблему исследовал У. Эберлейн в 1950 г. \cite{Eberlein},
т.е. через два года после классической работы Г. Г. Лоренца~\cite{lorentz1948contribution}.
Эберлейн установил, что существуют такие линейные операторы  $A : \ell_\infty\to \ell_\infty$,
для которых $BAx = Bx$ независимо от выбора $x$ и для банаховых пределов специального вида.

Будем говорить, что $B\in\mathfrak B(A)$, $A : \ell_\infty\to \ell_\infty$, если для любого $x\in \ell_\infty$
выполнено равенство $BAx = Bx$.
Такой банахов предел $B$ называют инвариантным относительно оператора $A$.

Можно ли выделить какие-то особые свойства оператора сдвига,
которые необходимы или достаточны оператору, чтобы относительно него были инвариантны все или некоторые банаховы пределы?
Понятно, что если оператор $A$ таков, что для любого $x\in\ell_\infty$ между $Ax$ и $x$
существует (конечное) расстояние Дамерау--Левенштейна \cite{damerau1964technique} (т.е. минимальное количество операций вставки, удаления, замены и перестановки двух соседних элементов последовательности, необходимых для перевода $x$ в $Ax$, причём для разных $x\in\ell_\infty$ эти операции, вообще говоря, не обязаны быть одинаковыми), то относительно данного оператора инвариантен любой банахов предел. Аналогичное утверждение справедливо и в случае, если $Ax -x \in c_0$ для любого $x\in \ell_\infty$.

Следующим по естественности (после сдвига и замены конечного числа элементов) действием, сохраняющем сходимость последовательности, является повторение элементов последовательности, например, оператор
\begin{equation}
	\sigma_2(x_1,x_2,x_3,...) = (x_1,x_1, \; x_2, x_2, \; x_3, x_3, \; ...)
	.
\end{equation}
Однако относительно такого оператора инвариантны не все, а только некоторые банаховы пределы.
Так, в~\cite[теорема 14]{ASSU2} показано, что
\begin{equation}
	\B(\sigma_n) \cap \ext \B = \varnothing  \mbox{~~для любого~~} n\in\N_2
	.
\end{equation}
Заметим, что если мы рассмотрим оператор неравномерного растяжения
\begin{equation}
	\sigma_{1,2}(x_1,x_2,x_3,x_4,x_5,...) = (x_1, \; x_2, x_2, \;  x_3, \; x_4, x_4, \; x_5, ...)
	,
\end{equation}
то увидим, что периодическую последовательность $y_n = (-1)^n$, $y\in ac_0$ оператор $\sigma_{1,2}$
переводит в периодическую последовательность
\begin{equation}
	(-1, 1, 1, \; -1, 1, 1, \; ...) \in ac_{1/3}
	,
\end{equation}
поскольку на периодической последовательности любой банахов предел принимает значение, равное среднему по периоду.
Таким образом, не существует банаховых пределов, инвариантных относительно оператора $\sigma_{1,2}$.

В главе 3 изложены некоторые примеры операторов и найдены множества банаховых пределов,
инвариантных относительно этих операторов.
Затем рассматриваются следующие классы линейных операторов $H:\ell_\infty \to \ell_\infty$:

-- полуэберлейновы: такие, что $B_1 H \in\B$ для некоторого $B_1\in \B$;

-- эберлейновы: такие, что $B_1 H = B_1$ для некоторого $B_1\in \B$;

-- В-регулярные: такие, что $B_1 H \in \B$ для любого   $B_1\in \B$;

-- существенно эберлейновы: такие, что $B_1 H \in \B$ для любого $B_1\in \B$ и $B_2 H \ne B_2$ для некоторого $B_2\in \B$.

Устанавливается (см. теоремы~\ref{thm:amiable_but_not_Eberlein_exists} и~\ref{thm:Eberlein_but_not_B-regular_exists}),
что каждый следующий из этих классов вложен в предыдущий и не совпадает с ним.
Кроме того, доказывается ещё ряд смежных результатов, в частности, решается обратная задача об инвариантности.

\reflecttheorem{thm:generated_operator_G_B}
	Для каждого $B\in \B$ существует такой оператор $G_B:\ell_\infty \to \ell_\infty$,
	что $\B(G_B) = \{B\}$.
