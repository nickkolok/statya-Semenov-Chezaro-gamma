\paragraph{Relevance of the subject and its current state.}

One of the founders of functional analysis as a field of modern mathematics, S. Banach,
in 1932 introduced the set of continuous linear functionals on the space of bounded sequences
that coincide with the usual limit on all convergent sequences.
These functionals were later called Banach limits;
their study was carried out by G.G. Lorentz, L. Sucheston, G. Das, W.F. Eberlein, and other mathematicians.

In 1948, G.G. Lorentz used Banach limits to define the concept of an almost convergent sequence "---  
a sequence on which the value of a Banach limit does not depend on the choice of this Banach limit.
In 1967, L. Sucheston constructed analogues of the upper and lower limits for Banach limits "---  
(nonlinear) Sucheston functionals.
The study and generalization of almost convergence has been carried out in the works of
R.A. Raimi, M. Mursaleen, G. Bennett, N.J. Kalton, D. Hadjicovalis, E.A. Alekhno, D. Zanin, and others.
Banach limits that are invariant relative to some linear operators are an object of special interest.

Banach limits are closely connected to other areas of mathematics.
Thus, in the research of S. Lord, J. Phillips, A.L. Keri, P.G. Dodds, E.M. Semenov, B. de Pagter,
A.A. Sedaev, A.S. Usachev, F.A. Sukochev Banach limits are applied to the study of Dixmier traces;
Dixmier traces, in turn, are widely used in Connes' noncommutative geometry.
The connections of Banach limits with ergodic theory are dedicated to the works of L. Sucheston and others.

In this thesis, the space of almost convergent sequences
(and its subspace of sequences almost converging to zero)
and Sucheston functionals, as well as invariant Banach limits, are investigated.

\paragraph{The purpose and objectives of the study.}
The goal of the work is to study Banach limits and their invariance properties.

The tasks of the work:
\begin{itemize}
	\item
		investigation of the structure of the set of Banach limits;
	\item
		investigation of subspaces of the space of bounded sequences
		defined using Banach limits;
	\item
		investigation of the properties of the composition of linear operators and Banach limits.
\end{itemize}


\paragraph{Scientific Novelty.}
All main results of the thesis are new.

New criteria for almost convergence of a bounded sequence have been obtained.
A seminorm on the space of bounded sequences has been studied,
which in the works of other scholars arose naturally
(during the study of Banach limits invariant relative to the Cesàro operator),
but only in an auxiliary role,
and had not previously become an independent object of research.
New properties of continuous linear operators on the space of bounded sequences,
defined using Banach limits, have been revealed.
New examples of sets separating Banach limits and possessing special properties have been constructed.
The connection of Sucheston functionals with sets of multiples has been revealed.

\paragraph{Theoretical and practical significance of the results.}
The work is theoretical in nature.
The results of the thesis can be used in the educational process, special courses, and scientific research
conducted at Voronezh, Rostov, Samara, Yaroslavl State Universities, and others.
Several conjectures have been put forward in the work;
they can be used as tasks for students and graduates of bachelor's, specialist's, or master's programs
to complete course or final qualifying works.


\paragraph{Methodology and methods of the study.}
For the study of Banach limits and related mathematical objects, concepts, methods, and approaches of modern functional analysis
have been applied, as well as special elements and facts from topology and number theory.

\paragraph{Main provisions.}
The following main results are put forward for defense:
\begin{enumerate}
	\item
		special criteria for almost convergence of a bounded sequence have been obtained;
	\item
		a new asymptotic characteristic of a bounded sequence has been investigated,
		which allows revealing additional (compared to Banach limits) properties
		of such sequences;
	\item
		a hierarchical system of classes of bounded linear operators
		on the space $\ell_\infty$ has been constructed depending on the properties of their superposition with Banach limits;
	\item
		an example of a set of 0-1-sequences separating Banach limits
		and having zero measure induced by the Lebesgue measure on the interval $[0;1]$ has been constructed;
	\item
		the connection between the multiplicative structure of the support of a 0-1-sequence
		and the values that the upper and lower Sucheston functionals can take on such a sequence has been investigated.
\end{enumerate}


\paragraph{Verification and approbation of the results.}

All results included in the thesis have been proven
in accordance with modern standards of reliability of mathematical proofs.

The results of the thesis were presented and discussed:
\begin{itemize}
	\item
		at the International Conference ``Voronezh Winter Mathematical School of S.G.~Krein'' in 2018, 2022, 2025;
	\item
		at the International Conference ``Pontryagin Readings — XXIX,'' dedicated to the 90th anniversary of Vladimir Alexandrovich Il'in (Voronezh, 2018);
	\item
		at the International Youth Scientific School ``Current Directions in Mathematical Analysis and Related Issues'' (Voronezh, 2018);
	\item
		at the competition of research works of students and graduate students of Russian universities
		``Science of the Future --- Science of the Young'' in the section ``Information Technologies and Mathematics''
		(awarded 2nd place among graduate students) in November 2021;
	\item
		at the International (53rd All-Russian) Youth School-Conference
		``Modern Problems of Mathematics and Its Applications''
		(Yekaterinburg, 2022);
	\item
		at the scientific session of VSU in 2020, 2021, 2022, 2024, 2025;
	\item
		October 16, 2024, at the seminar at the Steklov Mathematical Institute under the guidance of Corresponding Member of the RAS O.V. Besov;
	\item
		October 15, 2024, at the seminar at MSU under the guidance of Professor of the RAS P.A. Borodin;
	\item
		November 13, 2024, at the seminar under the guidance of Prof. S.V. Astashkin (Samara);
	\item
		November 19, 2024, at the seminar under the guidance of Prof. A.L. Skubachevsky (Moscow);
	\item
		January 29, 2025, at the OTDE seminar under the guidance of Prof. A.G. Kusraev and M.A. Pliev (Vladikavkaz);
	\item
		at the international (56th All-Russian) youth school-conference
		``Modern Problems of Mathematics and Its Applications''
		(Yekaterinburg, February 2025);
	\item
		at the All-Russian Youth Scientific Conference ``Path to Science. Mathematics''
		(Yaroslavl, May 2025; the report was recognized as one of the best and awarded a diploma).
\end{itemize}

The research work of the author on the topic of the thesis was supported by grants:
\begin{itemize}
	\item
		Russian Science Foundation (grant No.~19-11-00197);
	\item
		Russian Science Foundation (grant No.~24-21-00220);
	\item
		Theoretical Physics and Mathematics Advancement Foundation ``BASIS'' (project No.~22-7-2-27-3).
\end{itemize}


\paragraph{Publications based on the research results.}
The main results of the thesis have been published in the works~\selfcite{
%Solid articles
our-mz2019ac0,
our-mz2019measure,
our-mz2021linearhulls,
avdeed2021AandA,
avdeev2021vestnik,
avdeev2023vmzprimes_eng,
avdeev2024decomposition,
avdeev2024set_DAN_eng,
%Theses
our-vvmsh-2018,
our-vzms-2018,
our-ped-2018-inf-dim-ker,
our-ped-2018-alpha-Tx,
vzms2022setsofmultiples,
avdeev2022measure,
vzms2025linear,
%TODO: if there are more works
}.
From the joint works~\selfcite{our-mz2019measure,avdeed2021AandA,avdeev2024decomposition,avdeev2024set_DAN_eng,our-vzms-2018}
only the results belonging personally to the author have been included in the thesis.

\paragraph{Structure and volume of the thesis.}
The thesis consists of an introduction, five chapters divided into 43 paragraphs,
and a reference list of 111 entries.
The total volume of the thesis is 121 pages.
%TODO: and paragraphs?
