\paragraph{Relevance of the subject and its current state.}

Один из основателей функционального анализа как области современной математики С. Банах
в 1932 году ввёл в рассмотрение множество непрерывных линейных функционалов на пространстве ограниченных последовательностей,
которые совпадают с обычным пределом на всех сходящихся последовательностях.
Эти функционалы в дальнейшем и были названы банаховыми пределами;
их изучением занимались Г.Г. Лоренц, Л. Сачестон, Г. Дас, У.Ф. Эберлейн и другие математики.

В 1948 году Г.Г. Лоренц, используя банаховы пределы, ввёл понятие почти сходящейся последовательности "---
последовательности, на которой значение банахова предела не зависит от выбора этого банахова предела.
В 1967 году Л. Сачестон построил аналоги верхнего и нижнего пределов для банаховых пределов "---
(нелинейные) функционалы Сачестона.
Изучению и обобщению понятия почти сходимости посвящены работы
Р.А. Раими, М. Мурсалена, Г. Беннета, Н.Дж. Калтона, Д. Хаджуковича, Е.А. Алехно, Д. Занина и др.
Отдельный интерес представляет вопрос о банаховых пределах, инвариантных относительно некоторых линейных операторов.

Банаховы пределы тесно связаны с другими областями математики.
Так, в исследованиях С. Лорда, Дж. Филлипса, А.Л. Кери, П.Г. Доддса, Е.М. Семёнова, Б. де Пагтера,
А.А.~Седаева, А.С. Усачёва, Ф.А. Сукочева банаховы пределы применяются к изучению следов Диксмье;
следы же Диксмье, в свою очередь, широко применяются в некоммутативной геометрии А. Конна.
Связи банаховых пределов с эргодической теорией посвящены работы Л. Сачестона и др.

В настоящей диссертации исследуются пространство почти сходящихся последовательностей
(и его подпространство последовательностей, почти сходящихся к нулю),
функционалы Сачестона и инвариантные банаховы пределы.

\paragraph{The purpose and objectives of the study.}
Целью работы является изучение банаховых пределов и их свойств инвариантности.

Задачи работы:
\begin{itemize}
	\item
		исследование структуры множества банаховых пределов;
	\item
		исследование подпространств пространства ограниченных последовательностей,
		определяемых с помощью банаховых пределов;
	\item
		исследование свойств композиции линейных операторов и банаховых пределов.
\end{itemize}


\paragraph{Scientific Novelty.}
Все основные результаты диссертации являются новыми.

Получены новые критерии почти сходимости ограниченной последовательности.
Изучена полунорма на пространстве ограниченных последовательностей,
которая в работах других учёных возникала естественным образом
(при изучении банаховых пределов, инвариантных относительно оператора Чезаро),
но лишь во вспомогательной роли,
и самостоятельным объектом исследования ранее не становилась.
Выявлены новые свойства непрерывных линейных операторов на пространстве ограниченных последовательностей,
определяемые с помощью банаховых пределов.
Построены новые примеры множеств, разделяющих банаховы пределы и обладающих специальными свойствами.
Выявлена связь функционалов Сачестона с множествами кратных.


\paragraph{Theoretical and practical significance of the results.}
Работа носит теоретический характер.
Результаты диссертации могут быть использованы в учебном процессе, спецкурсах и научных исследованиях,
проводимых в Воронежском, Ростовском, Самарском, Ярославском государственных университетах и др.
В работе сформулированы гипотезы,
которые могут быть использованы в составе заданий для выполнения
студентами и выпускниками бакалавриата, специалитета или магистратуры
курсовых или выпускных квалификационных работ.





\paragraph{Methodology and methods of the study.}
Для исследования банаховых пределов и связанных с ними математических объектов применяются
понятия, методы и подходы современного функционального анализа,
а также отдельные элементы и факты топологии и теории чисел.

\paragraph{Main provisions.}
На защиту выносятся следующие основные результаты:
\begin{enumerate}
	\item
		сформулированы и доказаны специальные критерии почти сходимости
		ограниченной последовательности;
	\item
		исследована новая асимптотическая характеристика ограниченной последовательности,
		позволяющая выявлять дополнительные (по сравнению с банаховыми пределами)
		свойства таких последовательностей;
	\item
		построена иерархическая система классов ограниченных линейных операторов
		на пространстве $\ell_\infty$ в зависимости от свойств их суперпозиции с банаховыми пределами;
	\item
		построен пример множества последовательностей из нулей и единиц, разделяющего банаховы пределы
		и имеющего нулевую меру, индуцированную мерой Лебега на отрезке $[0;1]$;
	\item
		исследована связь мультипликативной структуры носителя последовательности из нулей и единиц
		и значений, которые на такой последовательности могут принимать верхний и нижний функционалы Сачестона.
\end{enumerate}



\paragraph{Verification and approbation of the results.}

Все включенные в диссертацию результаты доказаны
в соответствии с современными стандартами достоверности математических доказательств.

The results of the thesis were presented and discussed:
\begin{itemize}
	\item
		at the International Conference ``Voronezh Winter Mathematical School of S.G.~Krein'' in 2018, 2022, 2025;
	\item
		at the International Conference ``Pontryagin Readings — XXIX,'' dedicated to the 90th anniversary of Vladimir Alexandrovich Il'in (Voronezh, 2018);
	\item
		at the International Youth Scientific School ``Current Directions in Mathematical Analysis and Related Issues'' (Voronezh, 2018);
	\item
		at the competition of research works of students and graduate students of Russian universities
		``Science of the Future --- Science of the Young'' in the section ``Information Technologies and Mathematics''
		(awarded 2nd place among graduate students) in November 2021;
	\item
		at the International (53rd All-Russian) Youth School-Conference
		``Modern Problems of Mathematics and Its Applications''
		(Yekaterinburg, 2022);
	\item
		at the scientific session of VSU in 2020, 2021, 2022, 2024, 2025;
	\item
		October 16, 2024, at the seminar at the Steklov Mathematical Institute under the guidance of Corresponding Member of the RAS O.V. Besov;
	\item
		October 15, 2024, at the seminar at MSU under the guidance of Professor of the RAS P.A. Borodin;
	\item
		November 13, 2024, at the seminar under the guidance of Prof. S.V. Astashkin (Samara);
	\item
		November 19, 2024, at the seminar under the guidance of Prof. A.L. Skubachevsky (Moscow);
	\item
		January 29, 2025, at the OTDE seminar under the guidance of Prof. A.G. Kusraev and M.A. Pliev (Vladikavkaz);
	\item
		at the international (56th All-Russian) youth school-conference
		``Modern Problems of Mathematics and Its Applications''
		(Yekaterinburg, February 2025);
	\item
		at the All-Russian Youth Scientific Conference ``Path to Science. Mathematics''
		(Yaroslavl, May 2025; the report was recognized as one of the best and awarded a diploma).
\end{itemize}

The research work of the author on the topic of the thesis was supported by grants:
\begin{itemize}
	\item
		Russian Science Foundation (grant No.~19-11-00197);
	\item
		Russian Science Foundation (grant No.~24-21-00220);
	\item
		Theoretical Physics and Mathematics Advancement Foundation ``BASIS'' (project No.~22-7-2-27-3).
\end{itemize}


\paragraph{Publications based on the research results.}
Основные результаты диссертации опубликованы в работах~\selfcite{
%Солидные статьи
our-mz2019ac0,
our-mz2019measure,
our-mz2021linearhulls,
avdeed2021AandA,
avdeev2021vestnik,
avdeev2021vmzprimes,
avdeev2024decomposition,
avdeev2024set_DAN_rus,
%Тезисы
our-vvmsh-2018,
our-vzms-2018,
our-ped-2018-inf-dim-ker,
our-ped-2018-alpha-Tx,
vzms2022setsofmultiples,
avdeev2022measure,
vzms2025linear,
%TODO: если будут ещё работы
}.
Из совместных работ~\selfcite{our-mz2019measure,avdeed2021AandA,avdeev2024decomposition,avdeev2024set_DAN_rus,our-vzms-2018}
в диссертацию вошли только результаты, принадлежащие лично диссертанту.

\paragraph{Структура и объём диссертации.}
Диссертация состоит из введения, пяти глав, разбитых на 43 параграфа,
и списка литературы, включающего 111 источников.
Общий объём диссертации 121 страница.
%TODO: а подпараграфы?
