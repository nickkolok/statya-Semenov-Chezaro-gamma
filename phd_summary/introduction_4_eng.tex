
Главы 4 и 5 посвящены верхнему и нижнему функционалам Сачестона $p(x)$ и $q(x)$"--- аналогам верхнего и нижнего пределов последовательности.
В главе 4 изучаются разделяющие множества.

Обозначим через $\Omega$ множество всех последовательностей, состоящих из нулей и единиц.

\reflecttheorem{thm:Lin_Omega_Sucheston}
	Пусть
	$1 \geq a > b \geq 0$ и
	$\Omega^a_b = \{x\in\Omega : p(x) = a, q(x) = b\}$,
	где $p(x)$ и $q(x)$~--- верхний и нижний функционалы Сачестона~\cite{sucheston1967banach} соответственно.
	Тогда $\Omega \subset \operatorname{Lin} \Omega^a_b$.


\reflectcorollary{crl:Lin_Omega_Sucheston}
	Множество $\Omega^a_b$ является разделяющим.
	Т.к. при $a\neq 1$ или $b\neq 0$ множество $\Omega^a_b$ имеет меру нуль~\cite{semenov2010characteristic,connor1990almost},
	то оно является разделяющим множеством нулевой меры.


Пусть $X^a_b = \{x\in\ell_\infty : p(x) = a,~ q(x) = b\}$, $Y^a_b = \{x\in A_0 : p(x) = a, q(x) = b\}$, где $a>b$.
\reflecttheorem{thm:A_0_c_infty_lin}
	Пусть $a\neq -b$.
	Тогда справедливо равенство $\operatorname{Lin} Y^a_b = A_0$.

\reflecttheorem{thm:Lin_ell_infty}
	Справедливо равенство $\operatorname{Lin} X^a_b = \ell_\infty$.

Через $\dim_H E$ будем обозначать хаусдорфову размерность множества $E$.

\reflecttheorem{thm:Hausdorf_measure_1_n}
	Пусть $n\in\N$.
	Тогда существует разделяющее множество $E\subset\Omega$ такое,
	что $\dim_H E = 1/n$.

%Конец главы 4
