Chapters 4 and 5 are devoted to the upper and lower Sucheston functionals $p(x)$ and $q(x)$ —
analogues of the upper and lower limits of a sequence.
In Chapter 4, separating sets are discussed.

Denote by $\Omega$ the set of all 0-1-sequences.

\reflecttheorem{thm:Lin_Omega_Sucheston}
	Let
	$1 \geq a > b \geq 0$ and
	$\Omega^a_b = \{x\in\Omega : p(x) = a, q(x) = b\}$,
	where $p(x)$ and $q(x)$ are the upper and lower Sucheston functionals~\cite{sucheston1967banach} respectively.
	Then $\Omega \subset \operatorname{Lin} \Omega^a_b$.

\reflectcorollary{crl:Lin_Omega_Sucheston}
	The set $\Omega^a_b$ is separating.
	Since for $a\neq 1$ or $b\neq 0$ the set $\Omega^a_b$ has measure zero~\cite{semenov2010characteristic_eng,connor1990almost},
	it is a separating set of zero measure.

Let $X^a_b = \{x\in\ell_\infty : p(x) = a,~ q(x) = b\}$, $Y^a_b = \{x\in A_0 : p(x) = a, q(x) = b\}$, where $a>b$.
\reflecttheorem{thm:A_0_c_infty_lin}
	Let $a\neq -b$.
	Then the equality $\operatorname{Lin} Y^a_b = A_0$ holds.

\reflecttheorem{thm:Lin_ell_infty}
	The equality $\operatorname{Lin} X^a_b = \ell_\infty$ holds.

We denote by $\dim_H E$ the Hausdorff dimension of the set $E$.

\reflecttheorem{thm:Hausdorf_measure_1_n}
	Let $n\in\N$.
	Then there exists a separating set $E\subset\Omega$ such that
	$\dim_H E = 1/n$.
