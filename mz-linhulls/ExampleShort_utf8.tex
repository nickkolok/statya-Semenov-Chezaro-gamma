% Уважаемые Авторы!
%Для подготовки pdf-файлы Вашей работы необходимо 
%запускать компиляцию с помощью команды latex 
%(в программе WinEdt это делается нажатием 
%клавиш Shift + Ctrl + L), после этого нужно 
%из получившегося dvi файла сделать pdf 
%с помощью команды dvi2pdf (в программе WinEdt для 
%этого есть специальная кнопка в верхней панели).
%Не нужно использовать команду pdflatex, 
%так как она препятствует команде \mag, 
%которая производит увеличение страниц 
%статьи для удобства редактирования.
%Просим не использовать в тексте рукописи
%нестандартные макроопределения.

\documentclass{article}


\usepackage[T2A]{fontenc}
\usepackage[utf8]{inputenc}
\usepackage[croatian,french,german,english,russian]{babel}
\usepackage[tbtags]{amsmath}
\usepackage{amsfonts,amssymb}

\hoffset -10mm
\voffset -7mm


\usepackage{mathrsfs}
\usepackage[short]{mz2ewin.utf8}

\theoremstyle{plain}

\newtheorem{condition}{Условие}
\newtheorem{lemma}{Лемма}
\newtheorem{theorem}{Теорема}
\newtheorem{proposition}{Предложение}
\newtheorem{corollary}{Следствие}
\newtheorem{notation}{Обозначения}
\newtheorem{definition}{Определение}

\theoremstyle{definition}
\newtheorem{proof}{Доказательство}\def\theproof{}
\newtheorem{remark}{Замечание}






\overfullrule5pt
\mag1300
\begin{document}

\udk{517.982.276} % 	Пространства последовательностей и матриц

\date{}

\author{Н.\,Н.~Авдеев}
\address{
Воронежский госуниверситет,
г.~Воронеж}
\email{nickkolok@mail.ru}




\title{
	О подмножествах пространства ограниченных последовательностей
}



\maketitle
\begin{fulltext}

%\begin{abstract}

%\end{abstract}

\begin{keywords}
Ограниченная последовательность,
почти сходящаяся последовательность,
функционал Сачестона,
банахов предел,
линейная оболочка,
разделяющее множество.
\end{keywords}


\footnotetext{
	Работа выполнена в Воронежском университете при поддержке РНФ, грант 19-11-00197.
}



\section{Введение}

Рассмотрим пространство ограниченных последовательностей $\ell_\infty$ с обычной нормой
\begin{equation*}
	\|x\| = \sup_{k\in\mathbb{N}} |x_k|
	.
\end{equation*}
и обычной полуупорядоченностью, где $\mathbb{N}$ "--- множество натуральных чисел.
Через $c$ будем обозначать пространство сходящихся последовательностей,
через $c_0$ "---  пространство последовательностей, сходящихся к нулю.


\begin{definition}
	Линейный функционал $B\in l_\infty^*$ называется банаховым пределом,
	если
	\begin{enumerate}
		\item
			$B\geq0$, т.~е. $Bx \geq 0$ для $x \geq 0$,
		\item
			$B\mathbb{I}=1$, где $\mathbb{I} =(1,1,\ldots)$,
		\item
			$B(Tx)=B(x)$ для всех $x\in l_\infty$, где $T$~---
		оператор сдвига, т.~е. $T(x_1,x_2,\ldots)=(x_2,x_3,\ldots)$.
	\end{enumerate}
\end{definition}
Множество всех банаховых пределов обозначим через $\mathfrak{B}$.
Существование банаховых пределов было анонсировано С. Мазуром \cite{Mazur} и позднее доказано в книге С.~Банаха~\cite{B}. 


Сачестон~\cite{sucheston1967banach} установил, что
для любых $x\in l_\infty$ и $B\in\mathfrak{B}$
\begin{equation}\label{Sucheston}
	q(x) \leqslant Bx \leqslant p(x)
	,
\end{equation}
где
\begin{equation*}
	q(x) = \lim_{n\to\infty} \inf_{m\in\mathbb{N}}  \frac{1}{n} \sum_{k=m+1}^{m+n} x_k
	\mbox{~~~~и~~~~}
	p(x) = \lim_{n\to\infty} \sup_{m\in\mathbb{N}}  \frac{1}{n} \sum_{k=m+1}^{m+n} x_k
\end{equation*}
называют нижним и верхним функционалом Сачестона соотвественно.
Заметим, что $p(x) = -q(-x)$.
Неравенства \eqref{Sucheston} точны:
для данного $x$ для любого $r\in[q(x); p(x)]$ найдётся банахов предел
$B\in\mathfrak{B}$ такой, что $Bx = r$.

Множество таких $x\in\ell_\infty$, что $p(x)=q(x)$,
образует подпространство почти сходящихся последовательностей $ac$~~\cite{lorentz1948contribution}.
На почти сходящейся последовательности все банаховы пределы принимают одинаковые значения.

При исследовании банаховых пределов особый интерес представляют разделяющие множества~\cite[\S 3]{Semenov2014geomprops}.
Множество $Q\in\ell_\infty$ называют разделяющим, если
для любых неравных $B_1, B_2\in\mathfrak{B}$ существует такая последовательность $x\in Q$,
что $B_1 x \neq B_2 x$.
В частности, разделяющим является~\cite{semenov2010characteristic} множество всех последовательностей из 0 и 1,
которое в дальнейшем мы будем обозначать через $\Omega$
(иногда в литературе встречается также обозначение $\{0;1\}^\mathbb{N}$).

Каждой последовательности $(x_1, x_2, \dots)\in \Omega$ можно поставить в соответствие число
\begin{equation}\label{eq:bijection_omega_0_1}
	\sum_{k=1}^\infty 2^{-k} x_k \in [0,1]
	.
\end{equation}
С точностью до счётного множества это соответствие взаимно однозначно и определяет на множестве $\Omega$ меру,
которую мы будем отождествлять с мерой Лебега на $[0,1]$.
%
Оказывается, что из $\Omega$ можно выделить некоторые подмножества, которые также будут разделяющими,
например \cite[\S 3, Теорема 11]{Semenov2014geomprops},
\begin{equation}
	U = \{ x\in\Omega: q(x) = 0, p(x) = 1 \}
	.
\end{equation}
Однако множество $U$ имеет меру 1~\cite{semenov2010characteristic}.
В настоящей статье строится пример разделяющего множества,
являющегося подмножеством $\Omega$ и имеющего меру нуль.
Для построения такого множества используется следующий факт.
%
\begin{lemma}[{\cite[\S 3, замечание 6]{Semenov2014geomprops}}]
	Пусть $X$~--- разделяющее множество и $X \subset \operatorname{Lin} Y$,
	где $\operatorname{Lin} Y$ обозначает линейную оболочку $Y$.
	Тогда $Y$ также является разделяющим множеством.
\end{lemma}
Затем в статье обсуждаются свойства линейных оболочек множеств, определённых с помощью функционалов Сачестона.

\section{Вспомогательные построения}

В данном параграфе вводятся некоторые вспомогательные объекты,
которые потребуются далее при доказательстве теоремы~\ref{thm:Lin_Omega_Sucheston}.

%\subsection{Двоичные приближения}	

\begin{definition}
	$k$-м двоичным приближением к произвольному числу $d\in[0;1]$
	называется такое число $d_{(k)}\in\mathbb{N}\cup\{0\}$,
	что
	\begin{equation}
		\label{eq:binary_approximations_for_number}
		\frac{d_{(k)}}{2^k} < d \leq \frac{d_{(k)}+1}{2^k}
		.
	\end{equation}
\end{definition}

\begin{remark}
	Очевидно, что $d_{(k+1)}\in\{2d_{(k)},2d_{(k)}+1\}$.
\end{remark}

%\subsection{Последовательности-<<блоки>>}

Пусть $n$ зафиксировано.
Пусть
$
	K = \{k\in\mathbb{N} : k \geq n\} = \{n, n+1, n+2, ...\}
	.
$
%
Рекурсивно определим функцию $\operatorname{Br}:K\times [0;1] \to \ell_\infty$,
генерирующую <<блоки>> из нулей и единиц,
соответствующие приближению $d_{(k)}$ к числу $d\in[0;1]$ для $k \geq n$.
Сначала определим $\operatorname{Br}(k,d)$ для $k=n$ по следующему правилу:
\begin{equation}
	(\operatorname{Br}(n,d))_j = \begin{cases}
		1, & \mbox{~если~} 2^n - d_{(n)} < j \leq 2^n,
		\\
		0  & \mbox{~для остальных~} j
		.
	\end{cases}
\end{equation}
Заметим, что все элементы $\operatorname{Br}(n,d)$, начиная с $(2^n+1)$-го, равны нулю;
кроме того, в $\operatorname{Br}(n,d)$ ровно $d_{(n)}$ единиц.
%
Через $e_j$ обозначим $j$-й орт и
для каждого $k \geq n$ положим
\begin{equation}
	\label{eq:Br(k+1,d)}
	\operatorname{Br}(k+1,d) = \operatorname{Br}(k,d) + T^{2^k}\operatorname{Br}(k,d) + (d_{(k+1)}-2d_{(k)})e_{2^k+2^n-d_{(n)}}
	.
\end{equation}
%
\begin{proposition}
	\label{prop:Br_k_c_0_1}
	Среди элементов последовательности $\operatorname{Br}(k,d)$ ровно $d_{(k)}$ единиц;
	остальные "--- нули.
\end{proposition}
%
\begin{remark}
	Выполнено включение $\operatorname{supp}\operatorname{Br}(k,d) \subset \operatorname{supp}\operatorname{Br}(k+1,d)$
	и, более того,
	\begin{equation}
		(\operatorname{Br}(k,d))_j = \begin{cases}
			(\operatorname{Br}(k+1,d))_j, & \mbox{~если~}  j \leq 2^k,
			\\
			0  & \mbox{~для остальных~} j
			.
		\end{cases}
	\end{equation}
\end{remark}
%
\begin{lemma}
	\label{lem:sum_Br_k_c}
	Для любых таких $m$, $k$ и $i$, что $n \leq m \leq k$ и  $ i + 2^m - 1 \leq 2^k$,
	выполнено
	\begin{equation}
		d_{(m)} \leq \sum_{j=i}^{i+2^m-1} (\operatorname{Br}(k,d))_j \leq d_{(m)}+1
		.
	\end{equation}
\end{lemma}
%
Из~\eqref{eq:binary_approximations_for_number} непосредственно вытекает следующий факт.
\begin{proposition}
	\label{prop:Br_has_nulls}
	Пусть $d<1-3/2^n$.
	Тогда
	$%\begin{equation}
		(\operatorname{Br}(d,k))_j = 0 $ для $j = m\cdot 2^n + 1, m\in\mathbb{N} \cup\{0\}
		.
	$%\end{equation}
\end{proposition}

%\subsection{Частичный предел в функционале Сачестона}
%
\begin{proposition}
	\label{prop:Sucheston_partial_limit}
	Выполнено равенство
	\begin{equation*}
		p(x) = \lim_{n\to\infty} \sup_{m\in\mathbb{N}}  \frac{1}{n} \sum_{k=m+1}^{m+n} x_k
		= \lim_{n\to\infty} \sup_{m\in\mathbb{N}}  \frac{1}{2^n} \sum_{k=m+1}^{m+2^n} x_k
		.
	\end{equation*}
\end{proposition}
Аналогичное равенство выполнено и для функционала $q(x)$.

\section{Разделяющее множество нулевой меры}

\begin{theorem}
	\label{thm:Lin_Omega_Sucheston}
	Пусть
	$1 \geq a > b \geq 0$ и
	$\Omega^a_b = \{x\in\Omega : p(x) = a, q(x) = b\}$.
	Тогда $\Omega \subset \operatorname{Lin} \Omega^a_b$.
\end{theorem}

\begin{proof}
	Выберем $n\in\mathbb{N}$ таким образом, что
	\begin{equation}
		\label{eq:Omega_a_b_gap}
		a - b > \frac{3}{2^n}
	\end{equation}
	и $n$ чётно.
	Очевидно, что существует разложение
	\begin{equation}
		x = \sum_{i=0}^{k-1} T^i x_i, \quad x_i \in \Omega
		,
	\end{equation}
	где $k\in\mathbb{N}$ и все элементы последовательностей $x_i$,
	кроме имеющих индексы $km+1$, $m\in\mathbb{N}_0$, являются нулевыми.
	Пусть $k=2^n$; зафиксируем $i$ и в дальнейшем для удобства записи положим $w=x_i$.
	Наша задача~--- построить конечную линейную комбинацию элементов из $\Omega^a_b$, равную $w$.
	%
	Положим
	\begin{equation}
		v_j = \begin{cases}
			0,  & \mbox{~если~} j \leq 2^n,
			\\
			(\operatorname{Br}(2^{2k  },a))_{j-2^{2k  }},  & \mbox{~если~} 2^{2k  } < j \leq 2^{2k+1}, 2k   \geq n,
			\\
			(\operatorname{Br}(2^{2k+1},b))_{j-2^{2k+1}},  & \mbox{~если~} 2^{2k+1} < j \leq 2^{2k+2}, 2k+1 \geq n
			.
		\end{cases}
	\end{equation}
	Положим далее
	\begin{equation}
		u_j = \begin{cases}
			v_j + w_j,  & \mbox{~если~} j \leq 2^n
			\\
			            & \mbox{~или~} 2^{4k+3} < j \leq 2^{4k+4} \mbox{~и~} 4k + 3 \geq n,
			\\
			v_j         & \mbox{~для остальных~} j
			.
		\end{cases}
	\end{equation}
	В силу предложений~\ref{prop:Br_has_nulls} и~\ref{prop:Sucheston_partial_limit}, а также леммы~\ref{lem:sum_Br_k_c}
	получаем $u,v\in\Omega^a_b$.
	Заметим теперь, что
	\begin{equation}
		(u-v)_j = \begin{cases}
			w_j,  & \mbox{~если~} j \leq 2^n,
			\\
			0,  & \mbox{~если~} 2^{2k  } < j \leq 2^{2k+1}, 2k    \geq n,
			\\
			0,  & \mbox{~если~} 2^{4k+1} < j \leq 2^{4k+2}, 4k + 1 \geq n,
			\\
			w_j,  & \mbox{~если~} 2^{4k+3} < j \leq 2^{4k+4}, 4k + 3 \geq n
			.
		\end{cases}
	\end{equation}

	Аналогично строятся пары элементов, разность которых равна $w_j$ на $2^{4k+i} < j \leq 2^{4k+i+1}, 4k + i \geq n$ для $i=0,1,2$
	(требуется только обнулить первые $2^n$ элементов).
	Складывая полученные таким образом $4\cdot 2^n$ разностей элементов из $\Omega^a_b$, получаем требуемый элемент $x$.

\end{proof}

\begin{corollary}
	Множество $\Omega^a_b$ является разделяющим.
	Т.к. при $a\neq 1$ или $b\neq 0$ множество $\Omega^a_b$ имеет меру нуль~\cite{semenov2010characteristic,connor1990almost},
	то оно является разделяющим множеством нулевой меры.
\end{corollary}



\section{Линейные оболочки множеств, определяемых функционалами Сачестона}

Итак, $\Omega \subset \operatorname{Lin}\{x\in\Omega : p(x) = a,\ q(x) = b\}$, где $1\geq a>b\geq 0$.
Оказывается, что аналогичным свойством обладает и ещё одно подмножество пространства $\ell_\infty$: подпространство
$A_0 = \{ x \in \ell_\infty : \alpha(x) =0 \}$,
где~\cite{our-vzms-2018}
\begin{equation*}
	\alpha(x) = \varlimsup_{i\to\infty} \max_{i<j\leqslant 2i} |x_i - x_j|
	.
\end{equation*}
В~\cite{our-ped-2018-alpha-Tx} показано, что, хотя сама функция $\alpha(x)$ не инвариантна относительно оператора сдвига $T$,
подпространство $A_0$ такой инвариантностью обладает.
%Кроме того, пространство $A_0$ замкнуто, несепарабельно,
%инвариантно относительно операторов растяжения $\sigma_n$,
%усредняющего сжатия $\sigma_{1/n}$ и Чезаро $C$.
%Пространство $A_0$ замкнуто относительно операции поэлементного умножения
%(в отличие от пространства $ac$),
%хотя и не образует по ней идеал в $\ell_\infty$.

Пусть $Y^a_b = \{x\in A_0 : p(x) = a, q(x) = b\}$, где $a>b$.
Определим линейные операторы $S, M:\ell_\infty \to \ell_\infty$ следующим образом:
\begin{equation*}\label{operator_S}
	(Sy)_k = y_{i+2}, \mbox{ где } 2^i < k \leq 2^i+1
	,
\end{equation*}
\begin{multline*}
	M\omega=\left(
		0, 1\omega_1,
		0, \frac{1}{2}\omega_2, 1\omega_2, \frac{1}{2}\omega_2,
		0, \frac{1}{3}\omega_3, \frac{2}{3}\omega_3, 1\omega_3, \frac{2}{3}\omega_3, \frac{1}{3}\omega_3,
		0, ...,
	\right. \\ \left.
		0, \frac{1}{p}\omega_p, \frac{2}{p}\omega_p, ..., \frac{p-1}{p}\omega_p, 1\omega_p,
			\frac{p-1}{p}\omega_p, ..., \frac{2}{p}\omega_p, \frac{1}{p}\omega_p,
		0, \frac{1}{p+1}\omega_{p+1}, ...
	\right)
	.
\end{multline*}
Заметим, что $SM: \ell_\infty \to A_0$.

\begin{lemma}[{\cite{our-vzms-2018}}]
	Для любого $x\in \ell_\infty$ выполнено равенство
	\begin{equation*}
		\alpha(Sx) = \varlimsup_{k\to\infty} |x_{k+1} - x_{k}|
		.
	\end{equation*}
\end{lemma}

\begin{lemma}
	\label{lem:const_Lin_alpha_0}
	Пусть $a\neq -b$.
	Тогда справедливо включение
	$\mathbb{I}\in \operatorname{Lin} Y^a_b$.
\end{lemma}

\begin{proof}
	Не теряя общности, положим $a>0$.
	Пусть $y = M \mathbb{I}$,
	тогда $Sy\in A_0$ и $\mathbb{I}-Sy = S(\mathbb{I}-y)\in A_0$.
	%
	Пусть далее $x=(a-b)Sy+b\mathbb{I}$, $z=(a-b)(\mathbb{I}-Sy)+b\mathbb{I}$.
	Тогда $p(x)=p(z)=a$, $q(x)=q(z)=b$ и, следовательно, $x,z\in Y^a_b$.
	Кроме того, заметим, что
	$
		x+z = (a+b)\mathbb{I}
		,
	$
	откуда и следует, что $\mathbb{I}\in Y^a_b$.
\end{proof}


\begin{lemma}
	\label{lem:const_Lin_alpha_0_a_eq_-b}
	Справедливо включение
	$\mathbb{I}\in \operatorname{Lin} Y^a_{-a}$.
\end{lemma}

\begin{proof}
	Положим $x=aS(2M(\mathbb{I})-\mathbb{I})$,
	$y=-aS(2M(1,0,1,0,1,...)-\mathbb{I})$,
	$z=-aS(2M(0,1,0,1,0,...)-\mathbb{I})$.
	
	Очевидно, что каждая из последовательностей $x,y,z$ содержит отрезки сколь угодно большой длины,
	состоящие из $a$ (равно как и из $-a$), при этом $-a \leq x,y,z \leq a$.
	Следовательно, $p(x)=p(y)=p(z) = a$ и $q(x)=q(y)=q(z) = -a$,
	откуда $x,y,z \in Y^a_{-a}$.
	%
	Заметим теперь, что $x + y + z = a\mathbb{I}$,
	откуда $\mathbb{I} \in \operatorname{Lin} Y^a_{-a}$.
\end{proof}

\begin{theorem}
	\label{thm:A_0_c_infty_lin}
	Cправедливо равенство $\operatorname{Lin} Y^a_b = A_0$.
\end{theorem}

\begin{proof}
	Зафиксируем $x \in A_0$.

	Пусть сначала $p(x) = q(x)$.
	Тогда, согласно~\cite[следствие 2]{our-mz2019ac0}, $x\in c$
	и $x$ может быть представлен в виде суммы константы и последовательности из $c_0$.
	Утверждение теоремы следует из лемм~\ref{lem:const_Lin_alpha_0} и ~\ref{lem:const_Lin_alpha_0_a_eq_-b},
	а также включения $c_0 \subset Y^a_b$.

	Пусть теперь $p(x) > q(x)$.
	Положим $y=k\cdot x + C\cdot\mathbb{I}$,
	где $k=({a-b})/({p(x)-q(x)})$, $C=({bp(x)-aq(x)})/({p(x)-q(x)})$.
	Тогда, очевидно,
	\begin{equation}
		\label{eq:x_representation}
		x=\frac{y-C\cdot\mathbb{I}}{k}
		.
	\end{equation}
	Представление~\eqref{eq:x_representation} искомое.
	Действительно, в силу лемм~\ref{lem:const_Lin_alpha_0}~и~\ref{lem:const_Lin_alpha_0_a_eq_-b} выполнено
	$C\cdot\mathbb{I}\in Y^a_b$; кроме того, $p(y) = k\cdot p(x) + C = a$ и $q(y) = k\cdot q(x) + C = b$.
	%\begin{equation}
	%	p(y) = k\cdot p(x) + C
	%	=
	%	%\\=
	%	\frac{ap(x)-bp(x)+bp(x)-aq(x)}{p(x)-q(x)}
	%	=
	%	a
	%	,
	%\end{equation}
	%\begin{equation}
	%	q(y) = k\cdot q(x) + C
	%	=
	%	%\\=
	%	\frac{aq(x)-bq(x)+bp(x)-aq(x)}{p(x)-q(x)}
	%	=
	%	b
	%	.
	%\end{equation}
\end{proof}

Факт, аналогичный теоремам~\ref{thm:Lin_Omega_Sucheston} и~\ref{thm:A_0_c_infty_lin}, верен и для
всего пространство $\ell_\infty$:
$\ell_\infty\subset \operatorname{Lin} X^a_b$, где
$X^a_b = \{x\in\ell_\infty : p(x) = a,\ q(x) = b\}$, $a>b$.



\begin{theorem}
	\label{thm:Lin_ell_infty}
	Справедливо равенство $\operatorname{Lin} X^a_b = \ell_\infty$.
\end{theorem}

\begin{proof}
	Зафиксируем $x \in \ell_\infty$ и представим его в виде линейной комбинации последовательностей из $X^a_b$.
	Не теряя общности, положим $x\geq 0$.
	Если $p(x) = q(x)$, то возьмём некоторый $y\in\ell_\infty$,
	такой, что $p(y) > p(x) = q(x)  \geq q(y) \geq 0$.
	Тогда в силу выпуклости функционала $p$ имеем
	$
		q(x+y) < p(x+y)
	$,
	и задача сведена к отысканию представлений для $y$ и $x+y$.
	
	Таким образом, можно рассматривать только такие $x$, что $p(x) > q(x)$.
	Определим $y$ как в доказательстве теоремы~\ref{thm:A_0_c_infty_lin}.
	Дальнейшее доказательство переносится дословно.
\end{proof}

\section{Заключительные замечания}
Подмножество $\Omega$ нулевой меры, не являющееся разделяющим, сконструировать очень легко
(например, можно взять конечное множество или множество ортов).
Однако пока неясно, существует ли измеримое подмножество $\Omega$ ненулевой меры,
не являющееся разделяющим множеством.
Остаётся открытым и вопрос о том, какие множества (кроме $\Omega$, $A_0$ и самого $\ell_\infty$)
обладают свойством, аналогичным установленному в теоремах~\ref{thm:Lin_Omega_Sucheston},~\ref{thm:A_0_c_infty_lin}~и~\ref{thm:Lin_ell_infty}.

Автор выражает сердечную благодарность проф. Е.М. Семёнову за ценные cоветы и плодотворное обсуждение.
\end{fulltext}



\begin{thebibliography}{99}
	%
	\RBibitem{Mazur}
	\by S.~Mazur,
	\paper O metodach sumomalności
	\jour Ann. Soc. Polon. Math. (Supplement)
	\yr 1929
	\pages 102--107
	%
	\RBibitem{banach1993theorie}
	\by S.~Banach,
	\book Théorie des opérations linéaires,
	\publaddr Sceaux
	\publ Éditions Jacques Gabay
	\year 1993
	\pages iv+128
	%\isbn 2-87647-148-5
	\miscnote Reprint of the 1932 original
	%
	\RBibitem{sucheston1967banach}
	\by L.~Sucheston,
	\paper Banach limits
	\jour Amer. Math. Monthly
	\yr 1967
	\vol 74
	\pages 308--311
	%\issn 0002-9890
	%
	\RBibitem{lorentz1948contribution}\by G.\,G.~Lorentz,
	\paper A contribution to the theory of divergent sequences
	\jour Acta Mathematica
	\yr 1948
	\vol 80
	\issue 1
	\pages 167--190
	%\issn 0001-5962
	%
	\RBibitem{Semenov2014geomprops}
	\by Е.\,М.~Семёнов, Ф.\,А.~Сукочев, А.\,С.~Усачев,
	\paper Геометрические свойства множества банаховых пределов
	\jour Известия Российской академии наук. Серия математическая
	\yr 2014
	\vol 78
	\issue 3
	\pages 177--204
	%
	\RBibitem{semenov2010characteristic}
	\by Е.\,М.~Семенов, Ф.\,А.~Сукочев,
	\paper Характеристические функции банаховых пределов
	\jour Сибирский математический журнал
	\yr 2010
	\vol 51
	\issue 4
	\pages 904--910
	%
	\RBibitem{connor1990almost}
	\by J.~Connor,
	\paper Almost none of the sequences of 0’s and 1’s are almost convergent
	\jour International Journal of Mathematics and Mathematical Sciences
	\yr 1990
	\vol 13
	\issue 4
	\pages 775--777
	%
	\RBibitem{our-vzms-2018}
	\by Н.\,Н.~Авдеев, Е.\,М.~Семенов,
	\paper Об асимптотических свойствах оператора Чезаро
	\jour Воронежская Зимняя Математическая школа С.Г. Крейна -- 2018.
	Материалы Международной конференции. Под ред. В.А. Костина.
	\yr 2018
	\pages 107--109
	%
	\RBibitem{our-ped-2018-alpha-Tx}
	\by Н.\,Н.~Авдеев,
	\paper О суперпозиции оператора сдвига и одной функции на пространстве
	ограниченных последовательностей
	\jour Некоторые вопросы анализа, алгебры, геометрии и математического
	образования.
	\yr 2018\pages 20--21
	%
	\RBibitem{our-mz2019ac0}
	\by Н.\,Н.~Авдеев,
	\paper О пространстве почти сходящихся последовательностей
	\jour Математические заметки
	\yr 2019
	\vol 105
	\issue 3
	\pages 462--466
	%\elink http://mi.mathnet.ru/rus/mz/v105/i3/p462
\end{thebibliography}


\end{document}
