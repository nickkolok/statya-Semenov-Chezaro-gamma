\documentclass[a4paper,openbib]{report}
\usepackage{amsmath}
\usepackage[utf8]{inputenc}
\usepackage[english,russian]{babel}
\usepackage{amsfonts,amssymb}
\usepackage{latexsym}
\usepackage{euscript}
\usepackage{enumerate}
\usepackage{graphics}
\usepackage[dvips]{graphicx}
\usepackage{geometry}
\usepackage{wrapfig}
\usepackage[colorlinks=true,allcolors=black]{hyperref}
\usepackage{mathrsfs}

\usepackage{bbm}


\righthyphenmin=2

\usepackage[14pt]{extsizes}

\geometry{left=3cm}% левое поле
\geometry{right=1cm}% правое поле
\geometry{top=2cm}% верхнее поле
\geometry{bottom=2cm}% нижнее поле

\renewcommand{\baselinestretch}{1.3}

\renewcommand{\leq}{\leqslant}
\renewcommand{\geq}{\geqslant} % И делись оно всё нулём!

\newcommand{\longcomment}[1]{}

\usepackage[backend=biber,style=gost-numeric,sorting=none]{biblatex}
\addbibresource{../bib/general_monographies.bib}
\addbibresource{../bib/ext.bib}
\addbibresource{../bib/my.bib}
\addbibresource{../bib/Semenov.bib}
\addbibresource{../bib/Bibliography_from_Usachev.bib}
\addbibresource{../bib/classic.bib}

\addbibresource{../../sct-statya/common/my.bib}


\input{../bib/ext.hyphens.bib}

\usepackage{amsthm}
\theoremstyle{definition}
\newtheorem{lemma}{Лемма}%[section]
\newtheorem{theorem}[lemma]{Теорема}
\newtheorem{example}[lemma]{Пример}
\newtheorem{property}[lemma]{Свойство}
\newtheorem{corollary}{Следствие}[lemma]
\newtheorem{definition}[lemma]{Определение}
\newtheorem{remark}[lemma]{Замечание}

%Only referenced equations are numbered
\usepackage{mathtools}
\mathtoolsset{showonlyrefs}

%\mathtoolsset{showonlyrefs=false}
% (an equation/multline to be force-numbered)
%\mathtoolsset{showonlyrefs=true}


\begin{document}
\clubpenalty=10000
\widowpenalty=10000

\setcounter{page}{2}

\begin{center}
	\LARGE
	Исследования в области банаховых пределов
\end{center}
\vspace{7em}


Одним из основных объектов научно-исследовательской работы, выполняемой при поддержке гранта,
является пространство ограниченных последовательностей $\ell_\infty$ с обычной нормой
\begin{equation*}
	\|x\| = \sup_{k\in\mathbb{N}} |x_k|
	.
\end{equation*}
и обычной полуупорядоченностью, где $\mathbb{N}$ "--- множество натуральных чисел.
Естественным обобщением предела с пространства сходящихся последовательностей $c$ на пространство $\ell_\infty$
является понятие банахова предела.


\begin{definition}
	Линейный функционал $B\in \ell_\infty^*$ называется банаховым пределом,
	если
	\begin{enumerate}
		\item
			$B\geq0$, т.~е. $Bx \geq 0$ для $x \geq 0$,
		\item
			$B\mathbbm{1}=1$, где $\mathbbm{1} =(1,1,\ldots)$,
		\item
			$B(Tx)=B(x)$ для всех $x\in \ell_\infty$, где $T$~---
		оператор сдвига, т.~е. $T(x_1,x_2,\ldots)=(x_2,x_3,\ldots)$.
	\end{enumerate}
\end{definition}
Множество всех банаховых пределов обозначим через $\mathfrak{B}$.
Существование банаховых пределов было анонсировано С. Мазуром \cite{Mazur} и позднее доказано в книге С.~Банаха~\cite{B}.


Сачестон~\cite{sucheston1967banach} установил, что
для любых $x\in l_\infty$ и $B\in\mathfrak{B}$
\begin{equation}\label{Sucheston}
	q(x) \leqslant Bx \leqslant p(x)
	,
\end{equation}
где
\begin{equation*}
	q(x) = \lim_{n\to\infty} \inf_{m\in\mathbb{N}}  \frac{1}{n} \sum_{k=m+1}^{m+n} x_k
	~~~~\mbox{и}~~~~
	p(x) = \lim_{n\to\infty} \sup_{m\in\mathbb{N}}  \frac{1}{n} \sum_{k=m+1}^{m+n} x_k
	.
\end{equation*}
называют нижним и верхним функционалом Сачестона соотвественно.
Заметим, что $p(x) = -q(-x)$.
Неравенства \eqref{Sucheston} точны:
для данного $x$ для любого $r\in[q(x); p(x)]$ найдётся банахов предел
$B\in\mathfrak{B}$ такой, что $Bx = r$.

Множество таких $x\in\ell_\infty$, что $p(x)=q(x)$,
образует подпространство почти сходящихся последовательностей $ac$~~\cite{lorentz1948contribution}.
На почти сходящейся последовательности все банаховы пределы принимают одинаковые значения.

Представляет интерес вопрос об объекте, в некотором смысле двойственном пространству почти сходящихся последовательностей:
насколько малым можно взять подмножество пространства $\ell_\infty$,
чтобы для любых двух несовпадающих банаховых пределов в этом подмножестве всё ещё находился элемент,
на котором эти бааховы пределы будут различаться?
Так возникает понятие разделяющего множества~\cite[\S 3]{Semenov2014geomprops}.

Множество $Q\in\ell_\infty$ называют разделяющим, если
для любых неравных $B_1, B_2\in\mathfrak{B}$ существует такая последовательность $x\in Q$,
что $B_1 x \neq B_2 x$.
В частности, разделяющим является~\cite{semenov2010characteristic} множество всех последовательностей из 0 и 1,
которое в дальнейшем мы будем обозначать через $\Omega$
(иногда в литературе встречается также обозначение $\{0;1\}^\mathbb{N}$).

Множество $\Omega$, вообще говоря, интуитивно воспринимается малым по сравнению со всем пространством $\ell_\infty$.
Однако можно ли найти ещё меньшее подмножество? Критерием малости теперь можно взять классическую меру.

Каждой последовательности $(x_1, x_2, \dots)\in \Omega$ можно поставить в соответствие число
\begin{equation}\label{eq:bijection_omega_0_1}
	\sum_{k=1}^\infty 2^{-k} x_k \in [0,1]
	.
\end{equation}
С точностью до счётного множества это соответствие взаимно однозначно и определяет на множестве $\Omega$ меру,
которую мы будем отождествлять с мерой Лебега на $[0,1]$.

Оказывается, что из $\Omega$ можно выделить некоторые подмножества, которые также будут разделяющими,
например \cite[\S 3, Теорема 11]{Semenov2014geomprops},
\begin{equation}
	U = \{ x\in\Omega: q(x) = 0, p(x) = 1 \}
	.
\end{equation}

Однако множество $U$ имеет меру 1~\cite{semenov2010characteristic}.

В ходе выполнения работ по гранту был построен и опубликован в статье~\cite{avdeev2021vestnik} пример разделяющего множества,
являющегося подмножеством $\Omega$ и имеющего меру нуль.
Для построения такого множества используется следующий факт.

\begin{lemma}[{\cite[\S 3, замечание 6]{Semenov2014geomprops}}]
	Пусть $X$~--- разделяющее множество и $X \subset \operatorname{Lin} Y$,
	где $\operatorname{Lin} Y$ обозначает линейную оболочку $Y$.
	Тогда $Y$ также является разделяющим множеством.
\end{lemma}

Затем в этой же статье обсуждаются свойства линейных оболочек множеств, определённых с помощью функционалов Сачестона.
В частности, доказывается,
что наряду с использованным при построении разделяющего множества меры нуль включением
\begin{equation}
	\Omega \subset \operatorname{Lin}\{x\in\Omega : p(x) = a,~ q(x) = b\}
\end{equation}
для любых $0\leq b < a \leq 1$,
имеет место равенство
\begin{equation}
	\ell_\infty = \operatorname{Lin}\{x\in\ell_\infty : p(x) = a,~ q(x) = b\}
\end{equation}
для любых $a>b$.

Возникает закономерный вопрос: для каких ещё подмножеств пространства $\ell_\infty$
верны аналогичные соотношения?

Оказывается, что аналогичным свойством обладает и ещё одно подмножество пространства $\ell_\infty$: подпространство
$A_0 = \{ x \in \ell_\infty : \alpha(x) =0 \}$,
где~\cite{our-vzms-2018}
\begin{equation*}
	\alpha(x) = \varlimsup_{i\to\infty} \max_{i<j\leqslant 2i} |x_i - x_j|
	.
\end{equation*}

Пространство $A_0$ обладает рядом интересных свойств.


\begin{theorem}[{\cite[следствие 2]{our-mz2019ac0}}]
	\label{thm:alpha_c_ac_c}
	Пусть $x\in ac$, т.е. $p(x) = q(x)$.
	Тогда $x\in c$ если и только если $\alpha(x) = 0$.
\end{theorem}
Таким образом, $c = ac \cap A_0$.
Включение $c\subset A_0$ собственное.

В~\cite{our-ped-2018-alpha-Tx} показано, что, хотя сама функция $\alpha(x)$ не инвариантна относительно оператора сдвига $T$,
подпространство $A_0$ такой инвариантностью обладает;
из доказанного в~\cite{SSUZ2} соотношения
\begin{equation}
	\alpha(Cx) \leq \alpha(x)
	,
\end{equation}
где $Cx$ есть оператор Чезаро
\begin{equation}
	(Cx)_n = \frac{1}{n} \sum_{k=1}^n x_k
	,
\end{equation}
следует инвариантность пространства $A_0$ относительно оператора Чезаро.

В статье~\cite{avdeev2021vestnik} доказывается,
что пространство $A_0$ инвариантно относительно операторов растяжения $\sigma_n$
и усредняющего сжатия $\sigma_{1/n}$.






Дальнейшим ослаблением понятия сходимости является сходимость по Чезаро (сходимость в среднем).
Говорят, что последовательность $\{x_n\}\in\ell_\infty$ сходится по Чезаро к $t$, если
\begin{equation}
	\lim_{n\to\infty}\frac1{n}\sum_{i=1}^n x_i = t
	.
\end{equation}
Легко заметить, что обсуждаемые обобщения верхнего и нижнего пределов удовлетворяют соотношению
\begin{multline}
	\label{eq:generalization_of_limits}
	\liminf_{n\to\infty} x_n \leq q(x) \leq \liminf_{n\to\infty}\frac1{n}\sum_{i=1}^n x_i
	\leq
	\\ \leq
	\limsup_{n\to\infty}\frac1{n}\sum_{i=1}^n x_i
	\leq p(x)
	\leq \limsup_{n\to\infty} x_n
	.
\end{multline}


Понятно, что каждый $x\in \Omega$ можно отождествить с подмножеством множества натуральных чисел
$\operatorname{supp} x \subset \mathbb{N}$.
Вслед за~\cite{hall1992behrend} будем обозначать через $\mathscr{M}A$ множество всех чисел,
кратных элементам множества $A\subset\mathbb{N}$, т.е.
\begin{equation}
	\mathscr{M}A = \{ka: k\in\mathbb{N}, a\in A\}
	,
\end{equation}
через $\chi A$ "--- характеристическую функцию множества $A$.

Так, например,
\begin{gather}
	\chi \mathscr{M}\!A(\{2\}) = \chi \mathscr{M}\!A(\{2, 4\}) = \chi \mathscr{M}\!A(\{2,4,8,16,...\})
	= (0,1,0,1,0,1,0,1,0,...),
\\
	\chi \mathscr{M}\!A(\{3\}) = \chi \mathscr{M}\!A(\{3,9,27,...\}) = (0,0,1,\;0,0,1,\;0,0,1,\;0,0,1,\;0,0,1,\;...),
\\
	\chi \mathscr{M}\!A(\{2,3\}) = \chi \mathscr{M}\!A(\{2,3,6\}) = (0,1,1,1,0,1,\;0,1,1,1,0,1,\;0,1,1,1,0,1,...).
\end{gather}

Возникает закономерный вопрос о взаимосвязи структуры множества $A$
и значений, которые принимают обобщения верхнего и нижнего пределов~\eqref{eq:generalization_of_limits}
на последовательности $\chi \mathscr{M}\!A$.
Так, в работах~\cite{davenport1936sequences,davenport1951sequences} доказано, что для любого
$A=\{a_1,a_2,...\}\subset\mathbb{N}$
выполнено
\begin{equation}
	\liminf_{n\to\infty}\frac1{n}\sum_{i=1}^n (\chi\mathscr{M}A)_i =
	\lim_{j\to\infty}\lim_{n\to\infty}\frac1{n}\sum_{i=1}^n (\chi\mathscr{M}\{a_1,a_2,...,a_j\})_i
	.
\end{equation}
В работе~\cite[\S 7]{besicovitch1935density} построено такое множество $A\subset\mathbb{N}$, что
\begin{equation}
	\liminf_{n\to\infty}\frac1{n}\sum_{i=1}^n (\chi\mathscr{M}A)_i \neq
	\limsup_{n\to\infty}\frac1{n}\sum_{i=1}^n (\chi\mathscr{M}A)_i
	.
\end{equation}
За более подробной информацией о множествах типа $\mathscr{M}A$ отсылаем читателя к монографии~\cite{hall1996multiples}.

В статье~\cite{avdeev2021vmzprimes} изучаена зависимость значений, которые могут принимать функционалы Сачестона
на последовательностях $\chi\mathscr{M}A$, от свойств множества $A$.
Основной теоремой статьи~\cite{avdeev2021vmzprimes} является следующая:

\begin{theorem}
	Пусть $A\subset \mathbb{N}\setminus\{1\}$.
	Тогда следующие условия эквивалентны:
	\begin{enumerate}%[label=(\roman*)]
		\item
			Для любого $n\in\mathbb{N}$ найдётся набор попарно взаимно простых чисел
			$
				\{a_{n,1}, a_{n,2}, ..., a_{n,n}  \} \subset A
				.
			$
		\item
			В $A$ существует бесконечное подмножество попарно взаимно простых чисел
		\item
			$p(\chi\mathscr{M}A)=1$.
	\end{enumerate}
\end{theorem}








Таким образом, запланированные работы по гранту выполнены в полном объёме.




\addcontentsline{toc}{chapter}{Список литературы}
\small
\printbibliography{}

\end{document}
