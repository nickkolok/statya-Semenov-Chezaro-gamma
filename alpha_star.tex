Как мы выяснили, $\alpha$--функция не инвариантна относительно сдвига.
Чтобы избавиться от этого недостатка, введём функционалы
\begin{equation}
	\alpha^* = \lim_{n\to\infty} U^n x
\end{equation}
и
\begin{equation}
	\alpha_* = \lim_{n\to\infty} T^n x
\end{equation}
(оба пределы существуют как пределы монотонных ограниченных последовательностей).
Тогда верны следующие соотношения.
\begin{lemma}
	\begin{equation}
		\alpha^*(x) = \alpha^*(\sigma_n x) = \alpha^*(Tx) = \alpha^*(Ux)
		,
		%TODO: \sigma_{1/n}
	\end{equation}
	\begin{equation}
		\alpha_*(x) = \alpha_*(\sigma_n x) = \alpha_*(Tx) = \alpha_*(Ux)
		.
		%TODO: \sigma_{1/n}
	\end{equation}
\end{lemma}

\begin{lemma}
	\begin{equation}
		\frac{1}{2} \alpha(x) \leq \alpha_*(x) \leq \alpha(x) \leq \alpha^*(x) \leq 2 \alpha(x)
		.
	\end{equation}
\end{lemma}

Кроме того, для учёта обоих функционалов и сохранения однородности предлагается ввести функционал
\begin{equation}
	\alpha_*^*(x) = \sqrt{\alpha_*(x)\alpha^*(x)}
	.
\end{equation}

\begin{hypothesis}
	Функционал $\alpha_*^*(x)$ удовлетворяет неравенству треугольника.
\end{hypothesis}

\begin{hypothesis}
	Аналог теоремы~\ref{thm:alpha_Cx_no_gamma} верен для функционалов
	$\alpha^*(x)$, $\alpha_*(x)$ и $\alpha_*^*(x)$
\end{hypothesis}
