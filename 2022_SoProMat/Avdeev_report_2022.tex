\documentclass[10pt,pdf,hyperref={unicode},aspectratio=169,color={usenames, dvipsnames}]{beamer}\usepackage{amsmath}
\usepackage[utf8]{inputenc}
\usepackage[english,russian]{babel}
\usepackage{amsfonts}
\usepackage{amsfonts,amssymb}
\usepackage{amssymb}
\usepackage{latexsym}
\usepackage{euscript}
\usepackage{enumerate}
\usepackage{graphics}
\usepackage{graphicx}
\usepackage{geometry}
\usepackage{wrapfig}

\usepackage{bbm}
\usepackage{mathrsfs}


\usepackage{varwidth}

\usepackage{amsthm}
\theoremstyle{definition}
\newtheorem{llemma}{Лемма}
\newtheorem{ttheorem}[llemma]{Теорема}
\newtheorem{eexample}[llemma]{Пример}
\newtheorem{property}[llemma]{Свойство}
\newtheorem{remark}[llemma]{Замечание}
\newtheorem{ccorollary}{Следствие}[llemma]
\newtheorem{hhypothesis}{Гипотеза}[llemma]
\newtheorem{ddefinition}{Определение}[llemma]

\righthyphenmin=2

%\usetheme{Antibes}
\usefonttheme{professionalfonts} % using non standard fonts for beamer
\usefonttheme[onlymath]{serif} % default family is serif
%\usepackage{fontspec}
%\setmainfont{Liberation Serif}
\begin{document}


\title{
	О мере множеств, разделяющих банаховы пределы
}
\author{Н.Н.~Авдеев}
\institute{ Воронежский Государственный Университет \\
		{Работа выполнена  при поддержке РНФ, грант 19-11-00197.
	}}
\date{Воронеж 2022}

\maketitle

\setbeamertemplate{footline}[frame number]
\setbeamertemplate{navigation symbols}{\large{}}

\begin{frame}\frametitle{Обобщения понятия сходимости в $\ell_\infty$}
	%\hspace{1.468em}
	\begin{varwidth}[t]{\linewidth}
		\centering
		Верхний и нижний
		\\
		пределы:
		\\
		$\displaystyle \limsup_{n\to\infty} x_n$
		,~
		$\displaystyle \liminf_{n\to\infty} x_n$
		\\~\\
		\emph{
			не характеризуют
			\\
			распределение элементов
		}
	\end{varwidth}
	\hfill
	\begin{varwidth}[t]{\linewidth}
		\centering
		Сходимость в среднем
		\\
		(по Чезаро):
		$\displaystyle \lim_{n\to\infty} (Cx)_n$
		\\
		$\displaystyle (Cx)_n = \frac1n\sum_{i=1}^n x_i$
		\\~\\
		\emph{слишком универсальна}
	\end{varwidth}
	\hfill
	\begin{varwidth}[t]{\linewidth}
		\centering
		Банаховы пределы $B\in \mathfrak{B}$:
		\\
		т. Хана--Банаха к $\lim: c\to\mathbb R$
		\\
			$B \geqslant 0$
		\\
			$B\mathbbm{1} = 1$
		\\
			$B=BT$
		\\
		\emph{почти сходимость}
	\end{varwidth}
	\\~\\~\\
	\begin{varwidth}[t]{\linewidth}
		\centering
		\emph{Критерий Лоренца:}
		$\displaystyle
			x\in ac_t
			\Leftrightarrow
			\forall(B\in\mathfrak{B})[Bx = t]
			\Leftrightarrow
			\lim_{n\to\infty}  \sum_{k=m+1}^{m+n} \frac{x_k}n = t
		$
		равном. по $m\in\mathbb{N}$.
	\end{varwidth}
	\\~\\
	\vspace{0.8em}
	\begin{varwidth}[t]{\linewidth}
		$\displaystyle ac = \mathop{\cup}\limits_{t\in\mathbb R} ac_t$ "--- пространство почти сходящихся последовательностей
	\end{varwidth}
\end{frame}


\begin{frame}\frametitle{Пространство $\ell_\infty$}
	$\ell_\infty$~--- пространство всех ограниченных последовательностей
	$x=(x_1, x_2, ..., x_n, ...)$
	с нормой
	$$
		\|x\|_{\ell_\infty} = \sup_{k\in\mathbb{N}} |x_k|
	$$
	{Свойства:}

	\begin{itemize}
		\item
			$\ell_\infty$~--- линейное пространство над полем $\mathbb{R}$
		\item
			$\ell_\infty$  несепарабельно
		\item
			$\ell_1 \subset \ell_2 \subset \dots \subset \ell_\infty$
		\item
			$\ell_1^* = \ell_\infty$
		\item
			$\ell_\infty^* \neq \ell_1$
	\end{itemize}
\end{frame}

\begin{frame}\frametitle{Банаховы пределы}
	Банаховым пределом называется функционал $B\in \ell_\infty^*$ такой, что:
	\begin{enumerate}
		\item
			$B \geqslant 0$
		\item
			$B\mathbbm{1} = 1$
		\item
			$B=BT$
	\end{enumerate}
	Здесь $\mathbbm{1}=(1,1,1,1,1,...)$,
	$T$~--- оператор сдвига: $T(x_1, x_2, x_3, ...) = (x_2, x_3, ...)$.

	Простейшие свойства:
	\begin{itemize}
		\item
			$\|B\|_{\ell_\infty^*} = 1$
		\item
			$Bx = \lim_{n\to\infty} x_n$ для любого $x=(x_1, x_2, ...) \in c$,
			\\
			где $c$~--- множество сходящихся последовательностей.

			Таким образом,
			банахов предел~--- естественное обобщение понятия предела сходящейся последовательности
			на все ограниченные последовательности.
	\end{itemize}
\end{frame}

\begin{frame}\frametitle{Свойства множества банаховых пределов $\mathfrak{B}$}
	\begin{enumerate}
		\item
			$\mathfrak{B}$~--- замкнутое выпуклое подмножество $ S_{\ell_\infty^*}$~---
			единичной сферы пространства $\ell_\infty^*$
		\item
			$d(\mathfrak{B}) = 2$
		\item
			$\mathfrak{B}$ слабо *-компактно
	\end{enumerate}
\end{frame}

\begin{frame}\frametitle{Теорема Лоренца}
	Для заданного $r\in\mathbb{R}$ равенство $Bx=r$ выполнено для всех $B\in\mathfrak{B}$
	тогда и только тогда, когда
	\begin{equation*}
		\lim_{n\to\infty} \frac{1}{n} \sum_{k=m+1}^{m+n} x_k = r
	\end{equation*}
	сходится равномерно по всем $m\in\mathbb{N}$.

	Множество всех таких $x \in \ell_\infty$ обозначается $ac$.
\end{frame}

\begin{frame}\frametitle{Теорема Сачестона}
	является уточнением теоремы Лоренца.
	Пусть
	\begin{equation*}
		q(x) = \lim_{n\to\infty} \inf_{m\in\mathbb{N}}  \frac{1}{n} \sum_{k=m+1}^{m+n} x_k,
		~~~~~~~~
		p(x) = \lim_{n\to\infty} \sup_{m\in\mathbb{N}}  \frac{1}{n} \sum_{k=m+1}^{m+n} x_k.
	\end{equation*}
	Тогда для любых $x\in \ell_\infty$ и $B\in\mathfrak{B}$
	\begin{equation}\label{Sucheston}
		q(x) \leqslant Bx \leqslant p(x)
	\end{equation}
	Неравенства (\ref{Sucheston}) точны:
	для данного $x$ для любого $r\in[q(x); p(x)]$ найдётся банахов предел
	$B\in\mathfrak{B}$ такой, что $Bx = r$.
\end{frame}

\begin{frame}\frametitle{Что такое почти сходимость?}
	\begin{varwidth}[t]{\linewidth}
		\hspace{3em}
		Пример $x\in ac_0$, но $\displaystyle0=\liminf_{n\to\infty}x_n < \limsup_{n\to\infty}x_n=1$:~~
		$$\displaystyle
			x_k=\begin{cases}
				1, & k=2^j,~j\in\mathbb N
				\\
				0 & \mbox{для прочих~} k
			\end{cases}
		$$
		$x=(0,1,0,1,\;0,0,0,1,\;\;0,0,0,0,\;0,0,0,1,\;\;0,0,0,0,\;0,0,0,0,\;\;0,0,0,0,\;0,0,0,1,\;\;0,0,0,0,\;...)$
	\end{varwidth}
	\\
	\vspace{2em}
	\begin{varwidth}[t]{\linewidth}
		Пример сходящейся по Чезаро к нулю последовательности $x\notin ac$:~~
		$$\displaystyle
			x_k=\begin{cases}
				1, & 2^j-j < k \leq 2^j,~j\in\mathbb N
				\\
				0 & \mbox{для прочих~} k
			\end{cases}
		$$
		$x=(0,1,1,1,\;0,1,1,1,\;\;0,0,0,0,\;1,1,1,1,$\\
		$\phantom{x=(}0,0,0,0,\;0,0,0,0,\;\;0,0,0,1,\;1,1,1,1,$\\
		$\phantom{x=(}0,0,0,0,\;0,0,0,0,\;\;0,0,0,0,\;0,0,0,0,$\\
		$\phantom{x=(}0,0,0,0,\;0,0,0,0,\;\;0,0,1,1,\;1,1,1,1,...)$
	\end{varwidth}
\end{frame}

\begin{frame}\frametitle{Разделяющие множества}
	Объект, <<противоположный>> пространству $ac$.

	Насколько малым можно взять подмножество пространства $\ell_\infty$,
	чтобы для любых двух несовпадающих банаховых пределов в этом подмножестве всё ещё находился элемент,
	на котором эти банаховы пределы будут различаться?

	Множество $Q\in\ell_\infty$ "--- разделяющее, если
	для любых неравных $B_1, B_2\in\mathfrak{B}$ существует такая последовательность $x\in Q$,
	что $B_1 x \neq B_2 x$.
	В частности, разделяющим является множество $\Omega = \{0;1\}^\mathbb{N}$.

	Интуитивно: $\Omega$ малопо сравнению с $\ell_\infty$. Можно ли <<ещё меньше>>?
\end{frame}

\begin{frame}\frametitle{Мера разделяющих множеств}

	\begin{equation}\label{eq:bijection_omega_0_1}
		[0,1] \ni \sum_{k=1}^\infty 2^{-k} x_k
		\quad
		\stackrel{\text{п.б.}}{\leftrightarrow}
		\quad
		(x_1, x_2, \dots)\in \Omega
	\end{equation}

	$U = \{ x\in\Omega: q(x) = 0, p(x) = 1 \} \subset\Omega$ "--- разделяющее,
	но $\mu \Omega = 1$.

	\begin{llemma}
		Пусть $X$~--- разделяющее множество и $X \subset \operatorname{Lin} Y$,
		где $\operatorname{Lin} Y$ "--- линейная оболочка $Y$.
		Тогда $Y$ также является разделяющим множеством.
	\end{llemma}

	\begin{llemma}
		\begin{equation}
			\Omega \subset X_b^a\operatorname{Lin}\{x\in\Omega : p(x) = a,~ q(x) = b\}
		\end{equation}
		для любых $0\leq b < a \leq 1$; $\mu X_b^a = 0$.
	\end{llemma}


\end{frame}

\begin{frame}\frametitle{Линейные оболочки и функционалы Сачестона}

	\begin{llemma}
		\begin{equation}
			\ell_\infty = \operatorname{Lin}\{x\in\ell_\infty : p(x) = a,~ q(x) = b\}
		\end{equation}
		для любых $a>b$.
	\end{llemma}
	\vspace{1em}

	Для каких ещё подмножеств пространства $\ell_\infty$ ..?

	$A_0 = \{ x \in \ell_\infty : \alpha(x) =0 \}$,
	где~\cite{our-vzms-2018}
	\begin{equation*}
		\alpha(x) = \varlimsup_{i\to\infty} \max_{i<j\leqslant 2i} |x_i - x_j|
		.
	\end{equation*}


\end{frame}

\begin{frame}\frametitle{Свойства пространства $A_0$ }
	\begin{ttheorem}[{\cite[следствие 2]{avdeev2019space}}]
		\label{thm:alpha_c_ac_c}
		Пусть $x\in ac$, т.е. $p(x) = q(x)$.
		Тогда $x\in c$ если и только если $\alpha(x) = 0$.
	\end{ttheorem}
	Таким образом, $c = ac \cap A_0$.
	Включение $c\subset A_0$ собственное.


	Функция $\alpha(x)$ не инвариантна относительно оператора сдвига $T$,
	но подпространство $A_0$ инвариантно относительно сдвига и оператора Чезаро
	\begin{equation}
		(Cx)_n = \frac{1}{n} \sum_{k=1}^n x_k
		,
	\end{equation}

	а также относительно операторов растяжения $\sigma_n$
	и усредняющего сжатия $\sigma_{1/n}$.

\end{frame}


\def\rinc{}
\def\vak{}
\def\mzm{}

\setbeamertemplate{bibliography item}{\insertbiblabel\small}

\begin{frame}
	\frametitle{Основные публикации с участием докладчика}
	\vspace{-1.2em}
	\hfill
	\begin{varwidth}[t]{0.95\linewidth}
	\setlength\itemsep{0pt}
	\begin{thebibliography}{99}
		\addtobeamertemplate{block begin}{\vspace*{-30pt}}{}
		\addtobeamertemplate{block end}{}{\vspace*{-30pt}}
		\setlength{\parskip}{0pt}%
		\setlength{\itemsep}{0pt}
		{}
		\bibitem{avdeev2021subsets}\small
			\emph{Avdeev N.} On Subsets of the Space of Bounded Sequences // Mathematical Notes. — 2021. — т. 109, No 1. — с. 150—154.
			\mzm
		{}
		\vspace{-0.5em}
		\bibitem{avdeev2019banach}\small
			\emph{Avdeev} \emph{N.}, \emph{Semenov} \emph{E.}, \emph{Usachev} \emph{A.} Banach Limits and a Measure on the Set of 0-1-Sequences //
			Mathematical Notes. — 2019. — т. 106, No 5/6. — с. 833—836.
			\mzm
		{}
		\vspace{-0.5em}
		\bibitem{avdeev2019space}\small
			\emph{Avdeev} \emph{N.} On the Space of Almost Convergent Sequences // Mathematical Notes. — 2019. — т. 105, No 3/4. — с. 464—468.
			%— DOI: \href
			%{https://doi.org/10.1134/S0001434619030179} {\nolinkurl {10.1134/S0001434619030179}}. — URL: \url
			%{https://www.scopus.com/record/display.uri?origin=inward&eid=2-s2.0-85065492318}.
			\mzm
		{}
		\vspace{-0.5em}
		\bibitem{avdeed2021AandA}\small
			\emph{Авдеев} \emph{Н. Н.}, \emph{Семёнов} \emph{Е. М.}, \emph{Усачев} \emph{А. С.}
			Банаховы пределы: экстремальные свойства,
			инвариантность и теорема Фубини // Алгебра и анализ. — 2021. — т. 33, No 4. — с. 32—48.
			\vak
		{}
		\vspace{-0.5em}
		\bibitem{our-ped-2018-alpha-Tx}\footnotesize
			\emph{Авдеев} \emph{Н. Н.} О суперпозиции оператора сдвига и одной функции на пространстве ограниченных последовательностей //
			Некоторые вопросы анализа, алгебры, геометрии и математического образования. — 2018. — с. 20—21.
			\rinc
		{}
		\vspace{-0.6em}
		\bibitem{our-vzms-2018}\footnotesize
			\emph{Авдеев} \emph{Н. Н.}, \emph{Семенов} \emph{Е. М.}
			Об асимптотических свойствах оператора Чезаро \\// Материалы Воронежской Зимней
			Математической\\школы С.Г. Крейна – 2018.  Под ред.\\В.А. Костина. — 2018. — с. 107—109.
			\rinc
	\end{thebibliography}
	\end{varwidth}
	\vspace{10em}
\end{frame}

\setbeamertemplate{navigation symbols}{}

\begin{frame}
	{
		\huge\centering
		~\\~\\~\\~\\
		Спасибо за внимание
	}
	~\\
	\vspace{6.28em}
	nickkolok@mail.ru, avdeev@math.vsu.ru
	\\
	github.com/nickkolok
	\\
	arxiv.org/a/avdeev\_n\_1.html
\end{frame}

\end{document}
