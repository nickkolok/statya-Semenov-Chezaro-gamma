\documentclass[a4paper,14pt]{article}
\usepackage[cp1251]{inputenc}
%\usepackage{floatflt}
\usepackage{float}
\usepackage{cite}
\usepackage[english,russian]{babel}
\usepackage{amssymb,amsmath,amsfonts}
\usepackage{color}
\usepackage{enumerate}
\usepackage[dvips]{graphicx}
\usepackage{setspace}
\usepackage{xcolor}
\usepackage{fancyhdr}

\textheight=220mm
\textwidth=160mm
\oddsidemargin=0.1in
\evensidemargin=0.1in




%\usepackage[backend=biber,style=ajc2020unofficial,sorting=none]{biblatex}
%\addbibresource{../bib/general_monographies.bib}
%\addbibresource{../bib/ext.bib}
%\addbibresource{../bib/my.bib}
%\addbibresource{../bib/Semenov.bib}
%\addbibresource{../bib/Bibliography_from_Usachev.bib}
%\addbibresource{../bib/classic.bib}

%\addbibresource{../../sct-statya/common/my.bib}


%\input{../bib/ext.hyphens.bib}


%\def\nolinkurl{}
%\usepackage{../../../biblatex2bibitem/biblatex2bibitem}












\begin{document}

\pagestyle{fancy}
\fancyhead{}
\fancyhead[L]{\footnotesize{Современные проблемы математики и ее приложений}}
\fancyhead[R]{\footnotesize{Тезисы конференции}}
\fancyfoot{}
\fancyfoot[L]{\footnotesize{Екатеринбург, Россия}}
\fancyfoot[R]{\footnotesize{31 января -- 4 февраля, 2022}}
\renewcommand{\footrulewidth}{0.1 mm}

\begin{center}
\textbf{О мере множеств, разделяющих банаховы пределы}%
\footnote{Исследование выполнено за счет гранта Российского научного фонда (проект № 19-11-00197).}\\
\vspace{\baselineskip}
Авдеев Н.Н.\\
\emph{Воронежский госуниверситет, Воронеж, Россия}\\nickkolok@mail.ru, avdeev@math.vsu.ru
\vspace{\baselineskip}\\
\end{center}
\vspace{\baselineskip}


Обозначим через $\ell_\infty$ пространство ограниченных последовательностей с обычной нормой
$
	\|x\| = \sup_{k\in\mathbb{N}} |x_k|
	,
$
где $\mathbb{N}$ "--- множество натуральных чисел, и обычной полуупорядоченностью.
Естественным обобщением предела с пространства сходящихся последовательностей $c$ на $\ell_\infty$
является понятие банахова предела.

\textbf{Определение.}
	Линейный функционал $B\in \ell_\infty^*$ называется банаховым пределом,
	если
	\begin{enumerate}
		\item
			$B\geq0$, т.~е. $Bx \geq 0$ для $x \geq 0$,
		\item
			$B\mathbb{I}=1$, где $\mathbb{I} =(1,1,\ldots)$,
		\item
			$B(Tx)=B(x)$ для всех $x\in \ell_\infty$, где $T$~---
			оператор сдвига, т.~е. $T(x_1,x_2,\ldots)=(x_2,x_3,\ldots)$.
	\end{enumerate}

Кратко пишем: $B \in \mathfrak{B}$.
Существование банаховых пределов было анонсировано С. Мазуром в~1929~г. и доказано С.~Банахом~\cite{banach1993theorie}.
Сачестон установил~\cite{sucheston1967banach}, что
для любых $x\in \ell_\infty$ и $B\in\mathfrak{B}$
\begin{equation*}\label{Sucheston}
	q(x) \leqslant Bx \leqslant p(x)
	,
	\quad\mbox{~~где~~}\quad
	q(x) = \lim_{n\to\infty} \inf_{m\in\mathbb{N}}  \frac{1}{n} \sum_{k=m+1}^{m+n} x_k
	~~~~\mbox{и}~~~~
	p(x) = \lim_{n\to\infty} \sup_{m\in\mathbb{N}}  \frac{1}{n} \sum_{k=m+1}^{m+n} x_k
\end{equation*}
суть нижний и верхний функционалы Сачестона соотв.
%Неравенства \eqref{Sucheston} точны:
%для данного $x$ для любого $r\in[q(x); p(x)]$ найдётся банахов предел
%$B\in\mathfrak{B}$ такой, что $Bx = r$.
%
Множество таких $x\in\ell_\infty$, что $p(x)=q(x)$,
образует~\cite{lorentz1948contribution} пространство почти сходящихся последовательностей $ac$.
На каждом $x\in ac$ все $B\in \mathfrak{B}$ принимают одинаковые значения.


Представляет интерес вопрос об объекте, в некотором смысле двойственном пространству почти сходящихся последовательностей:
насколько малым можно взять подмножество пространства $\ell_\infty$,
чтобы для любых двух несовпадающих банаховых пределов в этом подмножестве всё ещё находился элемент,
на котором эти банаховы пределы будут различаться?
Так возникает понятие разделяющего множества~\cite[\S 3]{Semenov2014geomprops}.

Множество $Q\in\ell_\infty$ называют разделяющим, если
для любых неравных $B_1, B_2\in\mathfrak{B}$ существует такая последовательность $x\in Q$,
что $B_1 x \neq B_2 x$.
В частности, разделяющим является~\cite{semenov2010characteristic} множество всех последовательностей из 0 и 1,
которое в дальнейшем мы будем обозначать через $\Omega$.
Множество $\Omega$, вообще говоря, интуитивно воспринимается малым по сравнению со всем пространством $\ell_\infty$.
Однако можно ли найти ещё меньшее подмножество? Критерием малости теперь можно взять классическую меру.
Каждой последовательности $(x_1, x_2, \dots)\in \Omega$ поставим в соответствие число
\begin{equation}\label{eq:bijection_omega_0_1}
	\sum_{k=1}^\infty 2^{-k} x_k \in [0,1]
	.
\end{equation}
С точностью до счётного множества это соответствие взаимно однозначно и определяет на множестве $\Omega$ меру,
которую мы будем отождествлять с мерой Лебега на $[0,1]$.
Оказывается, что из $\Omega$ можно выделить некоторые подмножества, которые также будут разделяющими,
например \cite[\S 3, Теорема 11]{Semenov2014geomprops},
$
	U = \{ x\in\Omega: q(x) = 0, p(x) = 1 \}
	.
$
Однако множество $U$ имеет меру 1~\cite{semenov2010characteristic}.


В статье~\cite{avdeev2021vestnik} пример разделяющего множества,
являющегося подмножеством $\Omega$ и имеющего меру нуль,
а именно
\begin{equation}
	\{x\in\ell_\infty : p(x) = a,~ q(x) = b\}\quad 0 < b < a < 1
	.
\end{equation}
Для построения такого множества используется следующий факт.

\textbf{Лемма}{\cite[\S 3, замечание 6]{Semenov2014geomprops}}{\sl
	Пусть $X$~--- разделяющее множество и $X \subset \operatorname{Lin} Y$,
	где $\operatorname{Lin} Y$ обозначает линейную оболочку $Y$.
	Тогда $Y$ также является разделяющим множеством.
}

Наряду с использованным при построении разделяющего множества меры нуль включением
\begin{equation}
	\Omega \subset \operatorname{Lin}\{x\in\Omega : p(x) = a,~ q(x) = b\}
\end{equation}
для любых $0\leq b < a \leq 1$,
имеет место равенство
\begin{equation}
	\ell_\infty = \operatorname{Lin}\{x\in\ell_\infty : p(x) = a,~ q(x) = b\}
\end{equation}
для любых $a>b$.
Возникает закономерный вопрос: для каких ещё подмножеств пространства $\ell_\infty$
верны аналогичные соотношения?
Оказывается, что этим свойством обладает и ещё одно подмножество пространства $\ell_\infty$: подпространство
$A_0 = \{ x \in \ell_\infty : \alpha(x) =0 \}$,
где~\cite{our-vzms-2018}
\begin{equation*}
	\alpha(x) = \varlimsup_{i\to\infty} \max_{i<j\leqslant 2i} |x_i - x_j|
	.
\end{equation*}

Пространство $A_0$ обладает и другими интересными свойствами.


\textbf{Теорема.}{\cite[следствие 2]{our-mz2019ac0}}{\sl
	\label{thm:alpha_c_ac_c}
	Пусть $x\in ac$, т.е. $p(x) = q(x)$.
	Тогда $x\in c$ если и только если $\alpha(x) = 0$.
}
Таким образом, $c = ac \cap A_0$.
Включение $c\subset A_0$ собственное.

В~\cite{our-ped-2018-alpha-Tx} показано, что, хотя сама функция $\alpha(x)$ не инвариантна относительно оператора сдвига $T$,
подпространство $A_0$ такой инвариантностью обладает;
из доказанного в~\cite{SSUZ2} соотношения
$
	\alpha(Cx) \leq \alpha(x)
	,
$
где $Cx$ есть оператор Чезаро
$
	(Cx)_n = \frac{1}{n} \sum_{k=1}^n x_k
	,
$
следует инвариантность пространства $A_0$ относительно оператора Чезаро.
В статье~\cite{avdeev2021vestnik} доказывается,
что пространство $A_0$ инвариантно относительно операторов растяжения $\sigma_n$
и усредняющего сжатия $\sigma_{1/n}$.



%\printbibitembibliography
%\end{document}

\begin{thebibliography}{99}\small{
{}
\bibitem{banach1993theorie}
S. Banach, Th\'eorie des op\'erations lin\'eaires, Reprint of the 1932 original, Sceaux: \'Editions Jacques Gabay,
1993, iv+128, \textsc{isbn}: 2-87647-148-5.
{}
\bibitem{sucheston1967banach}
L. Sucheston, Banach limits, \emph{Amer. Math. Monthly} 74 (1967), 308--311, \textsc{issn}: 0002-9890,
\textsc{doi}: {10.2307/2316038}.
{}
\bibitem{lorentz1948contribution}
G. G. Lorentz, A contribution to the theory of divergent sequences, \emph{Acta Math.} 80.\textbf{1}
(1948), 167--190, \textsc{issn}: 0001-5962, \textsc{doi} {10.1007/BF02393648}.
{}
\bibitem{Semenov2014geomprops}
Е. М. Семёнов, Ф. А. Сукочев и А. С. Усачев, Геометрические свойства множества банаховых пределов,
\emph{Известия Российской академии наук. Серия математическая} 78.\textbf{3} (2014), 177--204.
{}
\bibitem{semenov2010characteristic}
Е. М. Семенов и Ф. А. Сукочев, Характеристические функции банаховых пределов, \emph{Сибирский
математический журнал} 51.\textbf{4} (2010), 904--910.
{}
\bibitem{avdeev2021vestnik}
Н. Н. Авдеев, О разделяющих множествах меры нуль и функционалах Сачестона, \emph{Вестн. ВГУ. Серия:
Физика. Математика} 4 (2021), 38--50.
{}
\bibitem{our-vzms-2018}
Н. Н. Авдеев и Е. М. Семенов, Об асимптотических свойствах оператора Чезаро, \emph{Воронежская
Зимняя Математическая школа С.Г. Крейна --- 2018. Материалы Международной конференции. Под ред. В.А.
Костина.} (2018), 107--109.{}
\bibitem{our-mz2019ac0}
Н. Н. Авдеев, О пространстве почти сходящихся последовательностей, \emph{Математические заметки}
105.\textbf{3} (2019), 462--466, \textsc{doi}: {10.4213/mzm12298}.
{}
\bibitem{our-ped-2018-alpha-Tx}
Н. Н. Авдеев, О суперпозиции оператора сдвига и одной функции на пространстве ограниченных
последовательностей, \emph{Некоторые вопросы анализа, алгебры, геометрии и математического
образования.} (2018), 20--21.
{}
\bibitem{SSUZ2}
E. Semenov и др., Invariant Banach limits and applications to noncommutative geometry, \emph{Pacific J. Math.}
306.\textbf{1} (2020), 357--373.
}\end{thebibliography}


\end{document}




%\textbf{Теорема/Следствие/Лемма/Замечание.} {\sl Формулировка теоремы/следствия/леммы/ замечания.}

%Пример ссылки \cite{Pi1,Pi2,Pi3,Pi4,Pi5}.
%\\
\begin{thebibliography}{99}

\bibitem{Pi1}
\small{M.~Hagie, The prime graph of a sporadic simple group. {\it Comm. Algebra}, {\bf 31}:~9 (2003), 4405--4424.}

\bibitem{Pi2}
\small{V.~D.~Mazurov, W.~J.~Shi, A note to the characterization of sporadic simple groups. {\it Algebra Colloq.},
{\bf 5}:~3 (1998), 285--288.}

\bibitem{Pi3}
\small{P.~Poto\v{c}nik, P.~Spiga, G.~Verret, Cubic vertex--transitive graphs on up to 1280 vertices. arXiv:1201.5317v1 [math.CO].}

\bibitem{Pi4}
\small{J.~H.~Conway, R.~T.~Curtis, S.~P.~Norton, R.~A.~Parker, R.~A.~Wilson, Atlas of finite groups.
Oxford: Clarendon Press, 1985.}

\bibitem{Pi5}
\small{А.~С.~Кондратьев, Н.~А.~Минигулов, О конечных неразрешимых $4$-примарных $3'$-группах. {\it Тезисы международной конференции "Алгебра, теория чисел и математическое моделирование динамических систем"\,, посвященной 70-летию А.Х. Журтова.} Нальчик: издательство КБГУ, 2019, 56--57.}
\end{thebibliography}
\end{document}
