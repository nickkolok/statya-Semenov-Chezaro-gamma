\documentclass[a4paper,14pt]{article} %размер бумаги устанавливаем А4, шрифт 12пунктов
\usepackage[T2A]{fontenc}
\usepackage[utf8]{inputenc}
\usepackage[english,russian]{babel} %используем русский и английский языки с переносами
\usepackage{amssymb,amsfonts,amsmath,mathtext,cite,enumerate,float,amsthm} %подключаем нужные пакеты расширений
\usepackage[unicode,colorlinks=true,citecolor=black,linkcolor=black]{hyperref}
%\usepackage[pdftex,unicode,colorlinks=true,linkcolor=blue]{hyperref}
\usepackage{indentfirst} % включить отступ у первого абзаца
\usepackage[dvips]{graphicx} %хотим вставлять рисунки?
\graphicspath{{illustr/}}%путь к рисункам

\makeatletter
\renewcommand{\@biblabel}[1]{#1.} % Заменяем библиографию с квадратных скобок на точку:
\makeatother %Смысл этих трёх строчек мне непонятен, но поверим "Запискам дебианщика"

\usepackage{geometry} % Меняем поля страницы.
\geometry{left=2cm}% левое поле
\geometry{right=1cm}% правое поле
\geometry{top=2cm}% верхнее поле
\geometry{bottom=2cm}% нижнее поле

\renewcommand{\theenumi}{\arabic{enumi}}% Меняем везде перечисления на цифра.цифра
\renewcommand{\labelenumi}{\arabic{enumi}}% Меняем везде перечисления на цифра.цифра
\renewcommand{\theenumii}{.\arabic{enumii}}% Меняем везде перечисления на цифра.цифра
\renewcommand{\labelenumii}{\arabic{enumi}.\arabic{enumii}.}% Меняем везде перечисления на цифра.цифра
\renewcommand{\theenumiii}{.\arabic{enumiii}}% Меняем везде перечисления на цифра.цифра
\renewcommand{\labelenumiii}{\arabic{enumi}.\arabic{enumii}.\arabic{enumiii}.}% Меняем везде перечисления на цифра.цифра

\sloppy


\renewcommand\normalsize{\fontsize{14}{25.2pt}\selectfont}

\begin{document}
% !!!
% Здесь начинается реальный ТеХ-код
% Всё, что выше - беллетристика

Дальнейшее излагается неакадемическим (а местами - прямо-таки антиакадемическим языком)
с целью упрощения как написания, так и чтения текста.
Кроме того, в тексте много пометок и комментариев.

\section{Вспомогательные построения}

\subsection{}
\begin{equation}\label{summa_drobey}
	\sum_{i=0}^{p-1} \frac{i \cdot 2^i}{p} = \frac{2^p(p-2) + 2}{p}
\end{equation}

Как это посчитать вручную --- не знаю.
Мне это считал компьютер.
В принципе, вполне доказывавается индукцией,
но, скорее всего, тут можно поставить ссылку на какой-нибудь уважаемый задачник типа Демидовича.
Или справочник по суммам и рядам.


\subsection{Вспомогательный оператор $S$}
Пусть $y\in l_\infty$.
Определим оператор $S:l_\infty \to l_\infty$ следующим образом:
\begin{equation}\label{operator_S}
	(Sy)_k = y_{i+2}, \mbox{ где } 2^i < k \leq 2^i+1
\end{equation}
Этот оператор вводится исключительно для упрощения изложения конструкции
(а она достаточно изощрённая).
Никакая высокая тёмная магия типа теоремы Хана-Банаха не используется.
Даже линейность не используется.

\paragraph{Пример.}
$$
	S(\{1,2,3,4,5,6, ...\}) = \{1,2,3,3,4,4,4,4,5,5,5,5,5,5,5,5,6...\}
$$

Возможно, кто-то где-то уже вводил такой оператор и исследовал его свойства,
но где такое искать --- я не знаю.


Теперь нам потребуются некоторые свойства оператора $S$.

\paragraph{Утверждение.}
\begin{equation}\label{alpha_S}
	\alpha(Sx) = \varlimsup_{k\to\infty} |x_{k+1} - x_{k}|
\end{equation}

\paragraph{Доказательство.}
Возможно, это очевидно и не стоит переводить бумагу, но пусть пока будет.

\begin{equation*}
	\alpha(Sx) =
	\varlimsup_{i\to\infty} \sup_{i < j \leq 2i} | (Sx)_i - (Sx)_j | = ...
\end{equation*}
Положим для каждого $i$ число $m_i$ так,
что $m_i = 2^{k_i}$, $i \leq m_i < 2i$
(очевидно, это всегда можно сделать и есть даже формула с логарифмом).
\begin{equation*}
	... =
	\varlimsup_{i\to\infty} \max \left\{
		\max_{i   < j \leq m_i} | (Sx)_i - (Sx)_j |,
		\max_{m_i < j \leq 2i } | (Sx)_i - (Sx)_j |
	\right\} =
	...
\end{equation*}
Но при $2^{k_i - 1} < i < j \leq m_i = 2^{k_i}$
имеем $(Sx)_i = (Sx)_j$, и первый модуль обнуляется.

\begin{equation*}
	... =
	\varlimsup_{i\to\infty}
		\max_{m_i < j \leq 2i } | (Sx)_i - (Sx)_j |
	=
	...
\end{equation*}
Но при $2^{k_i - 1} < i \leq m_i = 2^{k_i} < j \leq 2^{k_i+1}$
имеем $(Sx)_i = x_{k_i+1}$, $(Sx)_j = x_{k_i+2}$, откуда
\begin{equation}\label{alpha_S_sosedi}
	... =
	\varlimsup_{k\to\infty}
		| x_{k+1} - x_k |
\end{equation}

Утверждение доказано.

\paragraph{Утверждение.}
\begin{equation}\label{summa_S_less}
	\sum_{k=2}^{2^p} (Sy)_k =
	\sum_{i=0}^{p-1} 2^i y_{i+2}
\end{equation}

\paragraph{Доказательство.}

\begin{equation*}
	\sum_{k=2}^{2^p} (Sy)_k =
	\sum_{i=0}^{p-1} \sum_{k=2^i+1}^{2^{i+1}} (Sy)_k =
	\sum_{i=0}^{p-1} \sum_{k=2^i+1}^{2^{i+1}} y_{i+2} =
	\sum_{i=0}^{p-1} 2^i y_{i+2}
\end{equation*}

Утверждение доказано.


\paragraph{Утверждение.}
\begin{equation}\label{summa_S}
	\sum_{k=2^i+1}^{2^{i+j+1}} (Sx)_k =
	2^i\sum_{k=2}^{2^{j+1}} (ST^ix)_k
\end{equation}

Здесь и далее $(Tx)_n = x_{n+1}$.

\paragraph{Доказательство.}

\begin{multline*}
	\sum_{k=2^i+1}^{2^{i+j+1}} (Sx)_k =
	\sum_{m = i}^{i+j}\sum_{k=2^m+1}^{2^{m+1}} (Sx)_k =
	\sum_{m = i}^{i+j}2^m \cdot x_{m+2} =
	\\=
	2^i \cdot \sum_{n = 0}^{j}2^n \cdot x_{n+2+i} =
	2^i \cdot \sum_{n = 0}^{j}2^n (T^i x)_{n+2} =
	2^i \cdot \sum_{k=2}^{2^{j+1}} (ST^i x)_k
\end{multline*}

Утверждение доказано.

\subsection{Вспомогательная функция $k_b$}

Введём функцию
\begin{equation}\label{def_k_b}
	k_b(x) = \frac{1}{2b}\left|
		\sum_{k=1}^{b}x_k - \sum_{k=b+1}^{2b}x_k
	\right|
\end{equation}

\paragraph{Утверждение.}
\begin{equation}\label{alpha_greater_k_b}
	\alpha (Cx) \geq \varlimsup_{i\to \infty} k_i(x)
\end{equation}

\paragraph{Доказательство.}

\begin{multline*}
	\alpha (Cx) \mathop{=}\limits^{def}
	\varlimsup_{i\to \infty} \sup_{i<j\leq 2i} |(Cx)_i - (Cx)_j| \geq
	\varlimsup_{i\to \infty} |(Cx)_i - (Cx)_{2i}| =
	\\ =
	\varlimsup_{i\to \infty} \left|\frac{1}{i}\sum_{k=1}^i  - \frac{1}{2i}\sum_{k=1}^{2i} \right| =
	\varlimsup_{i\to \infty} \left|\frac{1}{i}\sum_{k=1}^i  - \frac{1}{2i}\sum_{k=1}^{i}- \frac{1}{2i}\sum_{k=i+1}^{2i}\right| =
	\\=
	\varlimsup_{i\to \infty} \left|\frac{1}{2i}\sum_{k=1}^i - \frac{1}{2i}\sum_{k=i+1}^{2i}\right| =
	\varlimsup_{i\to \infty} k_i(x)
\end{multline*}

\paragraph{Примечание.}
Введение функции $k_b(x)$ позволит нам в дальнейшем лихо уйти от работы с оператором Чезаро
и заняться тихим небуйным расписыванием сумм.



\section{Основные построения}

Построим вектор $y\in l_\infty$ следующим образом:

\begin{equation}\label{y_construction}
	y = \left\{
		0, 0, \frac{1}{p}, \frac{2}{p}, \frac{3}{p},
		...,
		\frac{p-1}{p}, 1, \frac{p-1}{p},
		...,
		\frac{1}{p},
		0, 0, 0, ..., 0,
		\frac{1}{p}, ...
	\right\}
\end{equation}
так, что
\begin{equation}\label{T_y}
	T^{5p}y = y
\end{equation}
(0 повторяется $3p+1$ раз или около того, надо будет ещё очень аккуратно пересчитать).

Вообще, конечно, здесь надо красивое равенство со скобочкой, но на это заморочимся,
если доказательство верно.

Положим $x = Sy$.
Тогда с учётом (\ref{alpha_S})
\begin{equation}\label{alpha_x}
	\alpha (x) = \alpha (Sy) = \frac{1}{p}
\end{equation}


Оценим $\alpha(Cx)$:

\begin{multline*}
	\alpha (Cx) \mathop{\geq}^{(\ref{alpha_greater_k_b})}
	\varlimsup_{b\to \infty} k_b(x) =
	\varlimsup_{b\to \infty}\frac{1}{2b}\left|
		\sum_{k=1}^{b}x_k - \sum_{k=b+1}^{2b}x_k
	\right| \geq
	\\ \geq
	\varlimsup_{
		i\to \infty,~
		b=2^i~
	}\frac{1}{2^{i+1}}\left|
		\sum_{k=1}^{2^i}(Sy)_k - \sum_{k=2^i+1}^{2^{i+1}}(Sy)_k
	\right| =
	\\=
	\varlimsup_{i\to \infty}\frac{1}{2^{i+1}}\left|
		\sum_{k=1}^{2^i}(Sy)_k - 2^i y_{i+2}
	\right| =
	\varlimsup_{i\to \infty}\left|
		\frac{1}{2^{i+1}}\sum_{k=1}^{2^i}(Sy)_k - \frac{y_{i+2}}{2}
	\right| \geq
	\\ \geq
	\varlimsup_{
		m\to \infty,~
		i=5pm+p~
	}\left|
		\frac{1}{2^{5pm+p+1}}\sum_{k=1}^{2^{5pm+p}}(Sy)_k - \frac{y_{5pm+p+2}}{2}
	\right| =
	\\=
	\varlimsup_{m\to \infty}\left|
		\frac{1}{2^{5pm+p+1}}\sum_{k=1}^{2^{5pm+p}}(Sy)_k - \frac{1}{2}
	\right| =
	\\=
	\varlimsup_{m\to \infty}\left|
		\frac{1}{2^{5pm+p+1}}\sum_{k=1}^{2^{5pm}}(Sy)_k
		+
		\frac{1}{2^{5pm+p+1}}\sum_{k=2^{5pm}+1}^{2^{5pm+p}}(Sy)_k
		- \frac{1}{2}
	\right|
	\mathop{=}^{(\ref{summa_S})}
	\\=
	\varlimsup_{m\to \infty}\left|
		\frac{1}{2^{5pm+p+1}}\sum_{k=1}^{2^{5pm}}(Sy)_k
		+
		\frac{1}{2^{5pm+p+1}} \cdot 2^{5pm} \cdot \sum_{k=2}^{2^p}(ST^{5pm}y)_k
		- \frac{1}{2}
	\right|
	\mathop{=}^{(\ref{T_y})}
	\\=
	\varlimsup_{m\to \infty}\left|
		\frac{1}{2^{5pm+p+1}}\sum_{k=1}^{2^{5pm}}(Sy)_k
		+
		\frac{1}{2^{5pm+p+1}} \cdot 2^{5pm} \cdot \sum_{k=2}^{2^p}(Sy)_k
		- \frac{1}{2}
	\right| =
\end{multline*}
\begin{multline*}
	\\=
	\varlimsup_{m\to \infty}\left|
		\frac{1}{2^{5pm+p+1}}\sum_{k=1}^{2^{5pm}}(Sy)_k
		+
		\frac{1}{2^{p+1}} \sum_{k=2}^{2^p}(Sy)_k
		- \frac{1}{2}
	\right|
	\mathop{=}^{(\ref{summa_S_less})}
	\\=
	\varlimsup_{m\to \infty}\left|
		\frac{1}{2^{5pm+p+1}}\sum_{k=1}^{2^{5pm}}(Sy)_k
		+
		\frac{1}{2^{p+1}} \sum_{i=0}^{p-1}2^i y_{i+2}
		- \frac{1}{2}
	\right| =
	\\=
	\varlimsup_{m\to \infty}\left|
		\frac{1}{2^{5pm+p+1}}\sum_{k=1}^{2^{5pm}}(Sy)_k
		+
		\frac{1}{2^{p+1}} \sum_{i=0}^{p-1}2^i \cdot \frac{i}{p}
		- \frac{1}{2}
	\right|
	\mathop{=}^{(\ref{summa_drobey})}
	\\=
	\varlimsup_{m\to \infty}\left|
		\frac{1}{2^{5pm+p+1}}\sum_{k=1}^{2^{5pm}}(Sy)_k
		+
		\frac{1}{2^{p+1}} \cdot \frac{2^p(p-2)+2}{p}
		- \frac{1}{2}
	\right| =
	\\=
	\varlimsup_{m\to \infty}\left|
		\frac{1}{2^{5pm+p+1}}\sum_{k=1}^{2^{5pm}}(Sy)_k
		+
		\frac{1}{2} \cdot \frac{p-2}{p} + \frac{1}{p 2^p}
		- \frac{1}{2}
	\right| =
	\\=
	\varlimsup_{m\to \infty}\left|
		\frac{1}{2^{5pm+p+1}}\sum_{k=1}^{2^{5pm}}(Sy)_k
		-
		\frac{1}{p} + \frac{1}{p 2^p}
	\right| =
	\\=
	\varlimsup_{m\to \infty}\left|
		\frac{1}{2^{5pm+p+1}}\sum_{k=1}^{2^{5pm-2p}}(Sy)_k
		+
		\frac{1}{2^{5pm+p+1}}\sum_{k=2^{5pm-2p}+1}^{2^{5pm}}(Sy)_k
		-\frac{1}{p} + \frac{1}{p 2^p}
	\right| =
	\\\mbox{но во второй сумме все $(Sy)_k$ --- нули по построению (доказать?)}
	\\=
	\varlimsup_{m\to \infty}\left|
		\frac{1}{2^{5pm+p+1}}\sum_{k=1}^{2^{5pm-2p}}(Sy)_k
		-\frac{1}{p} + \frac{1}{p 2^p}
	\right| = h
\end{multline*}

Но $0 \leq (Sy)_k \leq 1$,
значит,
$$
	\frac{1}{2^{5pm+p+1}}\sum_{k=1}^{2^{5pm-2p}}(Sy)_k
	\leq
	\frac{1}{2^{5pm+p+1}} \cdot 2^{5pm-2p}
	=
	\frac{1}{2^{3p+1}}
$$
Модуль раскрываем со знаком ``-''

\begin{multline*}
	h=
	\varlimsup_{m\to \infty} \left(
		\frac{1}{p} (1-2^{-p})
		- \frac{1}{2^{3p+1}}
	\right) =
	\frac{1}{p} (1-2^{-p})
	- \frac{1}{2^{3p+1}}
	= \\ =
	\frac{1}{p} (1-2^{-p})
	- \frac{1}{2^{2p+1}} \cdot 2^{-p}
	>
	\frac{1}{p} (1-2^{-p})
	- \frac{1}{p} \cdot 2^{-p}
	=
	\frac{1}{p} (1-2^{-p+1})
\end{multline*}


Таким образом,
$$
	\frac{\alpha(Cx)}{\alpha(x)} \geq
	\frac{	\frac{1}{p} (1-2^{-p+1}) }{\frac{1}{p}} =
	1-2^{-p+1}
$$

Рассматривая $x$ как функцию от $p$, имеем:
$$
	\sup_{p\in\mathbb{N}} \frac{\alpha(Cx(p))}{\alpha(x(p))} \geq
	\sup_{p\in\mathbb{N}} (1-2^{-p+1}) \geq
	1
$$
\end{document}
