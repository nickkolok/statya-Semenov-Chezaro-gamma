При изучении инвариантности банаховых пределов относительно различных непрерывных линейных операторов
возникает закономерный вопрос о классификации этих операторов.

\begin{definition}
	Будем говорить, что оператор $H : \ell_\infty \to \ell_\infty$ \emph{эберлейнов},
	если $\B (H) \ne \varnothing$.
\end{definition}

Выбор именования этого класса операторов обусловлен тем, что именно Эберлейн в работе~\cite{Eberlein}
впервые сколь-либо системно изучил инвариантные банаховы пределы
(хотя отдельные шаги в этом направлении были сделаны ещё в работе~\cite{agnew1938extensions}).

В работе~\cite{alekhno2018invariant} вводится следующее

\begin{definition}
	Оператор $H : \ell_\infty \to \ell_\infty$ называется \emph{B-регулярным},
	если $H*\B \subseteq \B$.
	(Или, что то же самое, $HB\in\B$ для любого $B\in\B$.)
\end{definition}

В той же работе с помощью теоремы о неподвижной точке доказывается,
что любой B-регулярный оператор "--- эберлейнов, приводится следующее необходимое и достаточное условия В-регулярности.

\begin{theorem}
	Оператор $H:\ell_\infty \to \ell_\infty$ является B-регулярным тогда и только тогда, когда:
	\\(i) $H\mathbbm{1} \in ac_1$;
	\\(ii) $q(Hx)\geq 0$ для любого $x\geq 0$;
	\\(iii) $H ac_0 \subseteq ac_0$.
\end{theorem}
Легко заметить, что эти условия являются более слабыми, чем достаточные условия эберлейновости.
%TODO: ссылка


\begin{hypothesis}
	Существует эберлейнов оператор, не являющийся B-регулярным.
\end{hypothesis}

Эту классификацию можно надстроить в обе стороны (по включению) следующими двумя определениями:

\begin{definition}
	Будем называть оператор $A:\ell_\infty \to \ell_\infty$ \emph{дружелюбным (amiable)}, если $BA\in\mathfrak B$ для некоторого $B\in\mathfrak B$.
\end{definition}

\begin{definition}
	Будем называть оператор $A:\ell_\infty \to \ell_\infty$ \emph{существенно дружелюбным}, если $BA\in\mathfrak B$ для любого $B\in\mathfrak B$ и $BA\ne B$ для некоторого $B\in\mathfrak B$.
\end{definition}

Эти четыре класса операторов (существенно дружелюбные, В-регулярные, эберлейновы, дружелюбные "--- в порядке включения)
получены последовательным ослаблением условий, естественных для <<достаточно хороших>> операторов:
$\sigma_k$, $C$
и образуют иерархию по включению.
Возникает закономерный вопрос о совпадении классов.
Ясно, что оператор сдвига $T$ является В-регулярным, но не является существенно дружелюбным.

\begin{hypothesis}
	Существует дружелюбный оператор, не являющийся эберлейновым.
\end{hypothesis}


Далее возникает вопрос о свойствах классов операторов.

Из определения В-регулярности незамедлительно следует

\begin{lemma}
	Множество В-регулярных операторов замкнуто относительно суперпозиции.
\end{lemma}
