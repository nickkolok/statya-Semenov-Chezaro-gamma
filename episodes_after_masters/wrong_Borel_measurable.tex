
Следующее доказательство неверно: я в нашёл ошибку в доказательстве уже тогда, когда набрал.

\begin{theorem}
	Любое измеримое по Борелю подмножество $A\subset \Omega$ ненулевой меры является разделяющим.
\end{theorem}

\begin{proof}
	Так как $A$ измеримо по Борелю, то оно представимо в виде
	\begin{equation}
		A = \bigcup_{i=1}^\infty  \bigcap_{j=1}^\infty \omega_{ij}
		,
	\end{equation}
	(здесь ошибка!!!)

	где $\omega_{ij}$~--- цилиндры (множества с одной фиксированной координатой).

	Выберем такое $i$, что для
	\begin{equation}
		A_i = \bigcap_{j=1}^\infty \omega_{ij}
	\end{equation}
	выполнено $\mu A_i > 0$.
	Тогда $A_i$~--- пересечение лишь конечного числа цилиндров;
	пусть $n$~--- номер последней координаты, зафиксированной этими цилиндрами.
	Тогда $T^n A_i = \Omega$,
	откуда в силу инвариантности банаховых пределов относительно оператора сдвига
\end{proof}

Конец неверного доказательства.
