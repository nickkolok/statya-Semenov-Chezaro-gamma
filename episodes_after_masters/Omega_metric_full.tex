На множестве всех последовательностей из нулей и единиц определим метрику
$$
	\rho(x,y)=
	\begin{cases}
		0, & \mbox{~если~} x=y, \\
		2^{-k}, k = \min\{n:x_n \neq y_n\} - 1, & \mbox{~если~} x\neq y. %x,y\in\{0,1\}^\N, \\
	\end{cases}
$$
Как показано в \cite[Утверждение 2.1.8]{Edgar}, пара $(\{0,1\}^\mathbb{N}, \rho)$ действительно является метрическим пространством.
Усилим этот результат.

\begin{theorem}
	\label{thm:metric_complete}
	Метрическое пространство $(\{0,1\}^\mathbb{N}, \rho)$ полно.
\end{theorem}

\begin{proof}
	Пусть $\{x_i\}$ "--- фундаментальная последовательность элементов из $\{0,1\}^\mathbb{N}$,
	$x_i=\left((x_i)_1, (x_i)_2, (x_i)_3,...\right)$.
	Тогда, согласно определению фундаментальной последовательности,
	\begin{equation}
		\forall(k\in\mathbb{N})
		\exists(n_k \in \mathbb{N})
		\forall(m,n \geq n_k)
		[\rho(x_m,x_n)< 2^{-k}]
		,
	\end{equation}
	в частности,
	\begin{equation}
		\label{eq:metric_n_k}
		\forall(k\in\mathbb{N})
		\exists(n_k \in \mathbb{N})
		\forall(m > n_k)
		\forall(s \leq k)
		\left[
			\left(x_m\right)_s = \left(x_{n_k}\right)_s
		\right]
		.
	\end{equation}
	Пусть $y\in\{0,1\}^\mathbb{N}$. Для каждого $k\in\mathbb{N}$ положим $y_k = \left(x_{n_k}\right)_k$,
	где $n_k$ выбрано согласно~\eqref{eq:metric_n_k}.
	Тогда
	\begin{equation}
		\forall(k\in\mathbb{N})
		\exists(n_k \in \mathbb{N})
		\forall(m > n_k)
		\forall(s \leq k)
		\left[
			\left(x_m\right)_s = y_s
		\right]
		,
	\end{equation}
	т.е.
	\begin{equation}
		\label{eq:metric_n_k_convergency}
		\forall(k\in\mathbb{N})
		\exists(n_k \in \mathbb{N})
		\forall(m > n_k)
		\left[
			\rho(x_m,y)< 2^{-k}
		\right]
		.
	\end{equation}
	Однако~\eqref{eq:metric_n_k_convergency} и означает,
	что последовательность $\{x_i\}$ сходится к последовательности $y$ по метрике $\rho$.
\end{proof}

Заметим, что теорема~\ref{thm:metric_complete} переносится дословно на пару $(\ell_\infty,\rho)$.
