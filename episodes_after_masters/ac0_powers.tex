Хорошо известен следующий факт
[TODO: ссылка? Где это можно посмотреть?]

\begin{lemma}
	\label{lem:simple_powers_in_ac0}
	Пусть $x_n = m^n$, $m\in\mathbb{N}$, $m\geq 2$.
	Тогда 
	\begin{equation}
		\chi_x \in ac_0
		.
	\end{equation} 
\end{lemma}

Обобщим теперь утверждение леммы~\ref{lem:simple_powers_in_ac0}.

\begin{lemma}
	Пусть $y_n$ --- строго возрастающая последовательность,
	$\chi_y\in\{0;1\}^\mathbb{N} \cap ac_0$.
	Пусть $m \in \mathbb{N}$.
	Тогда почти сходится к нулю последовательность $x_k$, определённая соотношением
	\begin{equation}
		x_k = \begin{cases}
			1, &\mbox{~если~} k = y_i \cdot m^j \mbox{~для некоторых~} i,j\in\mathbb{N},
			\\
			0  &\mbox{~иначе}
			.
		\end{cases}
	\end{equation}
\end{lemma}

\begin{proof}
	Зафиксируем некоторое $K \in \mathbb{N}$ и покажем, что $p(x) < m^{-K}$.
	Действительно, представим $x$ в виде суммы
	\begin{equation}
		\label{eq:ac0_powers_x_as_sum}
		x \leq z_1 + z_2 + \dots + z_K + z'_{K+1}
		,
	\end{equation}
	где
	\begin{equation}
		(z_j)_k = \begin{cases}
			1, &\mbox{~если~} k = y_i \cdot m^j \mbox{~для некоторого~} i\in\mathbb{N},
			\\
			0  &\mbox{~иначе}
			,
		\end{cases}
	\end{equation}
	\begin{equation}
		\label{eq:ac0_powers_residue}
		(z'_j)_k = \begin{cases}
			1, &\mbox{~если~} k = y_i \cdot m^{j'} \mbox{~для некоторых~} i,j'\in\mathbb{N},~~ j' > j
			\\ &\mbox{~и ни для какого~} j' \leq j \mbox{~не выполнено~} k = y_i \cdot m^{j'},
			\\
			0  &\mbox{~иначе}
			.
		\end{cases}
	\end{equation}
	(Знак неравенства в~\eqref{eq:ac0_powers_x_as_sum} ставится ввиду того, что возможен случай
	$y_i \cdot m^j = y_{i'} \cdot m^{j'}$ для $j\neq j'$.)
	Понятно, что $p(z_j)=0$.
	Таким образом,
	\begin{equation}
		p(x) \leq p(z_1) + p(z_2) + \dots + p(z_K) + p(z'_{K+1}) = p(z'_{K+1})
		.
	\end{equation}
	Заметим, что в силу определения~\eqref{eq:ac0_powers_residue} $z'_j$ каждый отрезок из $m^{j+1}$ элементов
	содержит не более одной единицы, и потому $p(z'_j) \leq m^{-j-1} < m^{-j}$.
	Таким образом, для любого $K\in\mathbb{N}$ выполнена оценка $p(x) < m^{-K}$,
	откуда $p(x) = 0$.
\end{proof}

\begin{corollary}
	\label{cor:ac0_powers_finite_set_of_numbers}
	Пусть $\{p_1, ..., p_k\} \subset \mathbb{N}$,
	\begin{equation}
		x_k = \begin{cases}
			1, &\mbox{~если~} k = p_1^{j_1}\cdot p_2^{j_2}\cdot ... \cdot p_k^{j_k} \mbox{~для некоторых~} j_1,...,j_k\in\mathbb{N},
			\\
			0  &\mbox{~иначе}.
		\end{cases}
	\end{equation}
\end{corollary}


\begin{hypothesis}
	Пусть $y_n$ и $z_n$ --- строго возрастающие последовательности,
	$\chi_y,\chi_z\in\{0;1\}^\mathbb{N} \cap ac_0$.
	Тогда почти сходится к нулю последовательность $x_k$, определённая соотношением
	\begin{equation}
		x_k = \begin{cases}
			1, &\mbox{~если~} k = y_i \cdot z_j \mbox{~для некоторых~} i,j\in\mathbb{N},
			\\
			0  &\mbox{~иначе}
			.
		\end{cases}
	\end{equation}
\end{hypothesis}
