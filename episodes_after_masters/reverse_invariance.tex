Ранее мы, как правило, ставили задачу, которую можно назвать прямой задачей об инвариантности:
дан некоторый достаточно хороший оператор $H$, и требуется выяснить, непусто ли множество $\B(H)$
банаховых пределов, инвариантных относительно этого оператора.
(И~если это множество непусто, то исследовать его.)

Возникает закономерный вопрос: для любого ли банахова предела существует нетривиальный оператор,
относительно которого этот банахов предел инвариантен?
Или же такие операторы существуют только для <<достаточно хороших>>
банаховых пределов "--- например, для банаховых пределов, инвариантных относительно какого-нибудь из операторов растяжения?
Существуют ли нетривиальные операторы, инвариантные относительно хотя бы одного $B\in\ext B$?
(Здесь под тривиальным оператором понимается такой оператор $H:\ell_\infty \to \ell_\infty$, что $H-I:\ell_\infty \to ac_0$.)

Итак, в этом параграфе мы обсудим обратную задачу об инвариантности.
Она имеет неожиданно простое решение.

\begin{theorem}
	Для каждого $B\in \B$ существует такой оператор $G_B:\ell_\infty \to \ell_\infty$,
	что $\B(G_B) = \{B\}$.
\end{theorem}

\begin{proof}
	Определим оператор $G_B$ равенством
	\begin{equation}
		G_B x = (Bx, Bx, Bx, ...) = (Bx)\cdot\mathbbm 1
		.
	\end{equation}
	Легко видеть, что
	\begin{equation}
		B G_B x = B(Bx, Bx, Bx, ...) = B((Bx)\cdot\mathbbm 1) = (Bx)\cdot B(\mathbbm 1)= (Bx)\cdot 1 = Bx
	\end{equation}
	для любого $x\in\ell_\infty$.
	Значит, $B\in \B (G_B)$ и $\B (G_B)$ непусто.

	Рассмотрим теперь $B_1 \in \B \setminus\{B\}$.
	Последнее означает, что на некоторой последовательности $y\in\ell_\infty$ выполнено $B_1y \ne By$.
	Тогда
	\begin{equation}
		B_1 G_B y = B_1(By, By, By, ...) = B_1((By)\cdot\mathbbm 1) = (By)\cdot B_1(\mathbbm 1)= (By)\cdot 1 = By \ne B_1y
		.
	\end{equation}
	Таким образом, $B_1 G_B \ne B_1$, и $B_1 \notin \B (G_B)$.
	Это и означает, что $\B(G_B) = \{B\}$.
\end{proof}

\begin{remark}
	Легко заметить, что оператор $G_B$ удовлетворяет достаточным условиям инвариантности.
	%TODO: ссылка на теорему
\end{remark}

Если же известно, что банахов предел $B$ заведомо обладает дополнительными свойствами инвариантности,
то можно сконструировать и другие примеры операторов, относительно которых $B$ инвариантен.

\begin{example}
	Пусть $B\in\mathfrak B(\sigma_2)$.
	Напомним, что $B(\sigma_2) = B(\sigma_{1/2})$, где
	\begin{equation}
		\sigma_{1/2} x = \left(\dfrac{x_1+x_2}2, \dfrac{x_3+x_4}2, \dfrac{x_5+x_6}2, ...\right)
		.
	\end{equation}
	Рассмотрим оператор $H_B$, $H_Bx = (x_1, Bx, x_2, Bx, x_3, Bx, ...)$.
	Тогда
	\begin{multline}
		B H_B x = B \sigma_{1/2} H_B x = B\left(\dfrac{x_1+Bx}2, \dfrac{x_2+Bx}2, \dfrac{x_3+Bx}2, ...\right) =
		\\=
		B\left(\dfrac{x_1}2, \dfrac{x_2}2, \dfrac{x_3}2, ...\right) + B\left(\dfrac{Bx}2, \dfrac{Bx}2, \dfrac{Bx}2, ...\right)=
		\\=
		\dfrac12 Bx + \dfrac12 B\left(Bx, Bx, Bx, ...\right) = Bx
		.
	\end{multline}

	С другой стороны, $(H_B - \sigma_2) \ell_\infty \not\subseteq ac_0$.
	Действительно, пусть $x=(1,0,1,0,1,0,...)$.
	Тогда $\sigma_2 x = (1,1,0,0,1,1,0,0,...)$,
	$H_B x = (1,\frac12,0,\frac12,1,\frac12,0,\frac12,...)$,
	$(H_B - \sigma_2)x = (0, -\frac12, 0, \frac12,...)$.
\end{example}
