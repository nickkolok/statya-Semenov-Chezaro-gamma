Ранее мы, как правило, ставили задачу, которую можно назвать прямой задачей об инвариантности:
дан некоторый достаточно хороший оператор $H$, и требуется выяснить, непусто ли множество $\B(H)$
банаховых пределов, инвариантных относительно этого оператора.
(И~если это множество непусто, то исследовать его.)

Возникает закономерный вопрос: для любого ли банахова предела существует нетривиальный оператор,
относительно которого этот банахов предел инвариантен?
Или же такие операторы существуют только для <<достаточно хороших>>
банаховых пределов "--- например, для банаховых пределов, инвариантных относительно какого-нибудь из операторов растяжения?
Существуют ли нетривиальные операторы, инвариантные относительно хотя бы одного $B\in\ext \B$?
(Здесь под тривиальным оператором понимается такой оператор $H:\ell_\infty \to \ell_\infty$, что $H-I:\ell_\infty \to ac_0$.)

Итак, в этом параграфе мы обсудим обратную задачу об инвариантности.
Она имеет неожиданно простое решение.

\begin{theorem}
	Для каждого $B\in \B$ существует такой оператор $G_B:\ell_\infty \to \ell_\infty$,
	что $\B(G_B) = \{B\}$.
\end{theorem}

\begin{proof}
	Определим оператор $G_B$ равенством
	\begin{equation}
		G_B x = (Bx, Bx, Bx, ...) = (Bx)\cdot\mathbbm 1
		.
	\end{equation}
	Легко видеть, что
	\begin{equation}
		B G_B x = B(Bx, Bx, Bx, ...) = B((Bx)\cdot\mathbbm 1) = (Bx)\cdot B(\mathbbm 1)= (Bx)\cdot 1 = Bx
	\end{equation}
	для любого $x\in\ell_\infty$.
	Значит, $B\in \B (G_B)$ и $\B (G_B)$ непусто.

	Рассмотрим теперь $B_1 \in \B \setminus\{B\}$.
	Последнее означает, что на некоторой последовательности $y\in\ell_\infty$ выполнено $B_1y \ne By$.
	Тогда
	\begin{equation}
		B_1 G_B y = B_1(By, By, By, ...) = B_1((By)\cdot\mathbbm 1) = (By)\cdot B_1(\mathbbm 1)= (By)\cdot 1 = By \ne B_1y
		.
	\end{equation}
	Таким образом, $B_1 G_B \ne B_1$, и $B_1 \notin \B (G_B)$.
	Это и означает, что $\B(G_B) = \{B\}$.
\end{proof}

\begin{definition}
	Оператор $G_B$, построенный таким образом, будем называть оператором,
	\emph{порождённым} банаховым пределом $B$.
\end{definition}

\begin{remark}
	Легко заметить, что любой порождённый оператор удовлетворяет достаточным условиям инвариантности.
	%TODO: ссылка на теорему
\end{remark}

\begin{remark}
	Более того, порождённый оператор $G_B$ является в некотором смысле крайним примером В-регулярного оператора,
	поскольку $G_B^* \B = \{B\}$.
\end{remark}

\begin{lemma}
	Пусть $0 \leq \lambda \leq 1$, $B_1, B_2 \in \B$.
	Тогда
	\begin{equation}
		G_{\lambda B_1+(1-\lambda) B_2} =\lambda G_{B_1} + (1-\lambda)G_{B_2}
		.
	\end{equation}
\end{lemma}


\begin{proof}
	В силу выпуклости (на единичной сфере в $\ell_\infty^*$) множества $\B$ оператор $G_{\lambda B_1+(1-\lambda) B_2}$
	действительно определён корректно.
	\begin{multline}
		G_{\lambda B_1+(1-\lambda) B_2} =
		\left((\lambda B_1+(1-\lambda) B_2) x\right)\mathbbm 1 =
		(\lambda B_1 x) \mathbbm 1 +((1-\lambda) B_2 x)\mathbbm 1 =
		\\=
		\lambda (B_1 x) \mathbbm 1 +(1-\lambda) (B_2 x)\mathbbm 1 =
		\lambda G_{B_1} x +(1-\lambda) G_{B_2 x}
		.
	\end{multline}
\end{proof}


\begin{lemma}
	Пусть $B_1, B_2 \in \B$.
	Тогда
	\begin{equation}
		G_{B_1} G_{B_2} = G_{B_2}
		.
	\end{equation}
\end{lemma}
\begin{proof}
	Пусть $x\in\ell_\infty$, тогда
	\begin{equation}
		G_{B_1}G_{B_2}x =
		G_{B_1} ( (B_2 x) \cdot \mathbbm 1 ) =
		(B_2 x) \cdot G_{B_1} ( \mathbbm 1 ) =
		(B_2 x) \cdot  \mathbbm 1 =
		G_{B_2}x
		.
	\end{equation}
\end{proof}


Если же известно, что банахов предел $B$ заведомо обладает дополнительными свойствами инвариантности,
то можно сконструировать и другие примеры операторов, относительно которых $B$ инвариантен.

\begin{example}
	Пусть $B\in\mathfrak B(\sigma_2)$.
	Напомним, что $B(\sigma_2) = B(\sigma_{1/2})$, где
	\begin{equation}
		\sigma_{1/2} x = \left(\dfrac{x_1+x_2}2, \dfrac{x_3+x_4}2, \dfrac{x_5+x_6}2, ...\right)
		.
	\end{equation}
	Рассмотрим оператор $H_B$, $H_Bx = (x_1, Bx, x_2, Bx, x_3, Bx, ...)$.
	Тогда
	\begin{multline}
		B H_B x = B \sigma_{1/2} H_B x = B\left(\dfrac{x_1+Bx}2, \dfrac{x_2+Bx}2, \dfrac{x_3+Bx}2, ...\right) =
		\\=
		B\left(\dfrac{x_1}2, \dfrac{x_2}2, \dfrac{x_3}2, ...\right) + B\left(\dfrac{Bx}2, \dfrac{Bx}2, \dfrac{Bx}2, ...\right)=
		\\=
		\dfrac12 Bx + \dfrac12 B\left(Bx, Bx, Bx, ...\right) = Bx
		.
	\end{multline}

	При этом операторы $H_B$, получаемые таким образом, достаточно <<разнообразны>>.
	Пусть $B_1 x \ne B_2 x$ на некотором $x\in \ell_\infty$.
	Тогда $y = H_{B_1} x - H_{B_2} x = (0, B_1 x - B_2 x, 0, B_1 x - B_2 x, 0, B_1 x - B_2 x, ...)$ и $B_1y = B_2y = \dfrac{B_1 x - B_2 x}2\ne 0$, откуда $(H_{B_1} - H_{B_2})\ell_\infty \not\subseteq ac_0$.


	Покажем  теперь, что $B_2 \notin \mathfrak B (H_{B_1})$ при $B_1 \ne B_2$. Предположим противное. Значит, $B_1 x \ne B_2 x$ на некотором $x\in \ell_\infty$. Пусть $y$ такой же, как выше. Тогда $B_2 y = B_2(H_{B_1} x - H_{B_2} x) = B_2 H_{B_1} x - B_2 H_{B_2} x = B_2 x - B_2 x = 0$, но $B_2 y = \dfrac{B_1x - B_2 x}{2} \ne 0$. Получили противоречие.

	На операторы $\sigma_k$ полученная конструкция обобщается тривиально.
	%TODO: обобщить!
\end{example}
