На множестве $\Omega$ стандартным образом определяется размерность Хаусдорфа (см. например \cite[Секция 6]{Edgar}).
Для непустого подмножества $F\subset \mathbb R^n$ и $s > 0$ определим $s$-мерную меру Хаусдорфа множества $F$ следующим образом:
$$\mathcal H^s(F) := \lim_{\delta\to0} \inf \left\{\sum_{i=1}^\infty \left({\rm diam} \ U_i \right)^s \ : \ F\subset \bigcup_{i=1}^\infty  U_i, \  0\leqslant {\rm diam} \ U_i \leqslant \delta \right\}.$$

Размерность Хаусдорфа множества  $F\subset \mathbb R^n$ определяется по формуле:
$${\rm dim}_H F := \inf\{ s > 0 \ : \ \mathcal H^s(F)=0\}.$$



Мы приведём определение самоподобных подмножеств множества $\Omega$ (см., например, \cite{falconer1997techniques}).

\begin{definition}
Множество $E\subset\Omega$ называется самоподобным, если существуют $m\in\mathbb{N}$,
$m\geqslant2$, $0< r_1, \dots, r_m<1$ и функции $f_j : \Omega \to \Omega$, $j=1,\dots, m$ такие, что
$$\rho(f_j(x), f_j(y)) = r_j \rho(x,y), \ \forall \ x,y \in \Omega, \ j=1,\dots, m$$
и $E=\bigcup_{j=1}^m f_j(E).$
\end{definition}



\begin{lemma}
	\label{lem:Hausdorf_measure}
	Пусть $E\subset\Omega$ и $TE = E$.
	Тогда размерность Хаусдорфа $E$ равна $1$.
\end{lemma}

\begin{proof}
	Для $j=1,2$ определим функции $f_j : \Omega \to \Omega$ следующим образом:
	$$f_1(x_1, x_2, \dots)=(0, x_1, x_2, \dots), \quad f_2(x_1, x_2, \dots)=(1, x_1, x_2, \dots).$$

	Очевидно, что $E=f_1(E)\cup f_2(E).$

	Теперь мы покажем, что размерность Хаусдорфа множества $E$ равна $1$.

	Т.к. для $j=1,2$ верно
	 $$\rho(f_j(x),f_j(y))=\frac12\rho(x,y), \ \forall \ x, y \in E,$$
	 то функции $f_j$ являются преобразованиями подобия с коэффициентами $r_j=1/2$ для $j=1,2$.


	По~\cite[Теорема 9.3]{Edgar} размерность Хаусдорфа $d$ множества $E$ является решением уравнения:
	$$ r_1^d+r_2^d=1.$$
	Т.к. $r_j=1/2$, то
	$d=1.$
\end{proof}

Лемма~\ref{lem:Hausdorf_measure} позволяет несколько сократить доказательство
[ТОDO: ссылка на ASU\_a\_a].
Действительно, $x\in\Omega\setminus c$ принадлежит $W$ тогда и только тогда, когда
\begin{equation}
	\label{eq:dim_ext_B_W}
	(\ext \mathfrak{B})x = \{0;1\}
\end{equation}
Очевидно, что соотношение~\eqref{eq:dim_ext_B_W} выполнено для $x$ тогда и только тогда, когда оно выполнено для $Tx$.





