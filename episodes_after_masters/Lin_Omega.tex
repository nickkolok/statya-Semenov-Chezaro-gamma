\begin{theorem}
	\label{thm:Lin_Omega_Sucheston}
	Пусть $\Omega = \{0;1\}^\mathbb{N}$,
	$\Omega^a_b = \{x\in\Omega : 1 \geq p(x) = a > q(x) = b \geq 0\}$,
	где $p(x)$ и $q(x)$~--- верхний и нижний функционалы Сачестона~\cite{sucheston1967banach} соответственно.
	Тогда $\Omega \subset \operatorname{Lin} \Omega^a_b$.
\end{theorem}

\begin{proof}
	%Пусть $x\in\ell_\infty$
	%Учитывая представление $x = ky - mz$,
	%где $0\leq y \leq 1$, $0\leq z \leq 1$, $k>0$, $m>0$,
	%будем, не теряя общности, считать, что $0\leq x \leq 1$.

	Выберем $n\in\mathbb{N}$ таким образом, что
	\begin{equation}
		\label{eq:Omega_a_b_gap}
		a - b > \frac{3}{2^n}
		.
	\end{equation}
	Определим теперь двоичные приближения к числам $a$ и $b$:
	\begin{equation}
		\label{eq:binary_approximations_for_number}
		\frac{a_k}{2^k} \leq a < \frac{a_k+1}{2^k}
		,
		~~~a_k>0
	\end{equation}
	(и аналогично для $b$).

\longcomment{
	Кроме того, для компактности изложения дальнейшей конструкции
	нам потребуется несколько вспомогательных операторов.
	Определим оператор циклического периодического сдвига:

	TODO: а бывают ли инвариантные банаховы пределы для этого оператора?
	Очень вероятно, т.к. похоже, что этот оператор сохраняет $p(x)$ и $q(x)$ не только для векторов специального вида.

	\begin{equation}
		(R_m x)_k = \begin{cases}
			x_{k-1}, \mbox{~если~}  rm+2 \leq k \leq (r+1)m
			,\\
			x_{k+m-1}, \mbox{~если~}  k = rm+1
			.
		\end{cases}
	\end{equation}
	Определим оператор периодического <<выбора>> координат для $j\leq m$:
	\begin{equation}
		(S_{m,j} x)_k = \begin{cases}
			x_{k}, \mbox{~если~}  k = rm + j
			,\\
			0, \mbox{~иначе}
			.
		\end{cases}
	\end{equation}
}

	Теперь определим <<блоки>> из нулей и единиц, соответствующие числу $c\in[0;1]$ для $k \geq n$.
	Пусть определены $c_k$ аналогично~\eqref{eq:binary_approximations_for_number}.
	Тогда $\operatorname{Br}(k,c) \in \ell_\infty$.
	Дальнейшее определение построим рекурсивно.
	\begin{equation}
		(\operatorname{Br}(n,c))_j = \begin{cases}
			0, & \mbox{~если~} j \leq 2^n - c_n,
			\\
			1  & \mbox{~иначе}
			.
		\end{cases}
	\end{equation}
	Для каждого $k > n$ положим
	\begin{equation}
		(\operatorname{Br}(k+1,c))_j = \begin{cases}
			(\operatorname{Br}(k,c))_j, &  \mbox{~если~} j \leq 2^k,
			\\
			(\operatorname{Br}(k,c))_{j-2^k}, &  \mbox{~если~} 2^k < j \leq 2^{k+1} \mbox{~и~} j \neq 2^k + 2^n - c_n,
			\\
			1, & \mbox{~если~} j = 2^k + 2^n - c_n \mbox{~и~} c_{k+1} = 2 c_k + 1,
			\\
			0, & \mbox{~если~} j = 2^k + 2^n - c_n \mbox{~и~} c_{k+1} = 2 c_k,
			\\
			0  & \mbox{~иначе (т.е. для $j > 2^{k+1}$)}
			.
		\end{cases}
	\end{equation}
	(В силу построения перечисленные условия покрывают все возможные варианты.)

	Иначе говоря, $(k+1)$-й блок получается из $k$-го добавлением сдвинутой на $2^k$ <<копии>> $k$-го блока,
	в котором первый отрезок единиц при необходимости предварён единицей.

	Нетрудно заметить, что для любого таких $m$, что $n \leq m \leq k$, и $ i \leq 2^k - 2^m + 1$
	выполнено
	\begin{equation}
		c_m \leq \sum_{j=i}^{i+2^m-1} (\operatorname{Br}(k,c))_j \leq c_{m}+1
		.
	\end{equation}


	Так, например, для $n=2$ и $c=1/3$ имеем:
	\begin{equation*}
		\begin{array}{ll}
			\operatorname{Br}(2,1/3) = (&0,0,0,1,0,0,...)
			\\
			\operatorname{Br}(3,1/3) = (&0,0,0,1,0,0,0,1,0,0,...)
			\\
			\operatorname{Br}(4,1/3) = (&0,0,0,1, 0,0,0,1,   0,0,1,1, 0,0,0,1,  0,0,...)
			\\
			\operatorname{Br}(5,1/3) = (&
			                             0,0,0,1, 0,0,0,1,   0,0,1,1, 0,0,0,1,\\
			                           & 0,0,0,1, 0,0,0,1,   0,0,1,1, 0,0,0,1,
			0,0,...)
			\\
			\operatorname{Br}(6,1/3) = (&
			                             0,0,0,1, 0,0,0,1,   0,0,1,1, 0,0,0,1,\\
			                           & 0,0,0,1, 0,0,0,1,   0,0,1,1, 0,0,0,1,\\
			                           & 0,0,1,1, 0,0,0,1,   0,0,1,1, 0,0,0,1,\\
			                           & 0,0,0,1, 0,0,0,1,   0,0,1,1, 0,0,0,1,
			0,0,...)
		\end{array}
	\end{equation*}

	Однако напомним, что нам нужно представление для произвольного $x\in\Omega$ в виде линейной комбинации.
	Очевидно, что существует разложение
	\begin{equation}
		x = \sum_{i=0}^{k-1} T^i x_i
		,
	\end{equation}
	где $k\in\mathbb{N}$~--- любое, $T$~--- оператор сдвига, а все элементы последовательностей $x_i$,
	кроме имеющих индексы $km+1$, $m\in\mathbb{N}_0$, являются нулевыми.
	Пусть $k=2^n$; зафиксируем $i$ и в дальнейшем для удобства записи положим $w=x_i$.
	Наша задача~--- построить конечную линейную комбинацию элементов из $\Omega^a_b$, равную $w$.

	Положим
	\begin{equation}
		u_j = \begin{cases}
			w_j,  & \mbox{~если~} j \leq 2^n,
			\\
			(\operatorname{Br}(2^{2k  },a))_{j-2^{2k}},  & \mbox{~если~} 2^{2k} < j \leq 2^{2k+1}, 2k \geq n,
			\\
			(\operatorname{Br}(2^{4k+1},b))_{j-2^{4k+1}},  & \mbox{~если~} 2^{4k+1} < j \leq 2^{4k+2}, 4k + 1 \geq n,
			\\
			(\operatorname{Br}(2^{4k+3},b))_{j-2^{4k+3}} + w_j,  & \mbox{~если~} 2^{4k+3} < j \leq 2^{4k+4}, 4k + 3 \geq n
			;
		\end{cases}
	\end{equation}
	\begin{equation}
		v_j = \begin{cases}
			0,  & \mbox{~если~} j \leq 2^n,
			\\
			(\operatorname{Br}(2^{2k  },a))_{j-2^{2k  }},  & \mbox{~если~} 2^{2k  } < j \leq 2^{2k+1}, 2k   \geq n,
			\\
			(\operatorname{Br}(2^{2k+1},b))_{j-2^{2k+1}},  & \mbox{~если~} 2^{2k+1} < j \leq 2^{2k+2}, 2k+1 \geq n
			.
		\end{cases}
	\end{equation}
	Заметим, что все элементы, к которым прибавляются ненулевые элементы $w_j$, равны нулю.
	Кроме того, $p(u)=p(w)=a$ и $q(u)=q(w)=b$.
	(На <<подпорченном>> блоке $u$ сумма, соответствующая функционалу $q$,
	увеличивается не настолько cильно, чтобы <<испортить>> $p$,
	благодаря условию~\eqref{eq:Omega_a_b_gap}.)
	Следовательно, $u,v\in\Omega^a_b$.
	Заметим теперь, что
	\begin{equation}
		(u-v)_j = \begin{cases}
			w_j,  & \mbox{~если~} j \leq 2^n,
			\\
			%(\operatorname{Br}(2^{2k  },a))_{j-2^{2k  }} - (\operatorname{Br}(2^{2k  },a))_{j-2^{2k  }} = 0,  & \mbox{~если~} 2^{2k  } < j \leq 2^{2k+1}, 2k    \geq n,
			0,  & \mbox{~если~} 2^{2k  } < j \leq 2^{2k+1}, 2k    \geq n,
			\\
			%(\operatorname{Br}(2^{4k+1},b))_{j-2^{4k+1}} - (\operatorname{Br}(2^{4k+1},b))_{j-2^{4k+1}} = 0,  & \mbox{~если~} 2^{4k+1} < j \leq 2^{4k+2}, 4k + 1 \geq n,
			0,  & \mbox{~если~} 2^{4k+1} < j \leq 2^{4k+2}, 4k + 1 \geq n,
			\\
			%(\operatorname{Br}(2^{4k+3},b))_{j-2^{4k+3}} + w_j - (\operatorname{Br}(2^{4k+3},b))_{j-2^{4k+3}} = w_j,  & \mbox{~если~} 2^{4k+3} < j \leq 2^{4k+4}, 4k + 3 \geq n
			w_j,  & \mbox{~если~} 2^{4k+3} < j \leq 2^{4k+4}, 4k + 3 \geq n
			.
		\end{cases}
	\end{equation}

	Аналогично строятся пары элементов, разность которых равна $w_j$ на $2^{4k+i} < j \leq 2^{4k+i+1}, 4k + i \geq n$ для $i=0,1,2$
	(требуется только обнулить первые $2^n$ элементов).
	Складывая полученные таким образом $4\cdot 2^n$ разностей элементов из $\Omega^a_b$, получаем требуемый элемент $x$.

\end{proof}

\begin{corollary}
	Множество $\Omega^a_b$ является разделяющим.
	Т.к. при $a\neq 1$ или $b\neq 0$ множество $\Omega^a_b$ имеет меру нуль,
	то оно является разделяющим множеством нулевой меры.
\end{corollary}


\begin{remark}
	Подмножество $\Omega$, соответствующее иррациональным числам, является разделяющим,
	поскольку последовательности, соответствующие рациональным числам,
	становятся периодическими с некоторого элемента и потому почти сходятся.
\end{remark}


\begin{remark}
	Понятно, что неизмеримое разделяющее множество легко сконструировать,
	объединив разделяющее множество нулевой меры и произвольное неизмеримое множество.
	Для полноты изложения, однако, приведём другую конструкцию, являющуюся близким аналогом классического множества Лузина.
	%[TODO: ссылка?]
	% Множество не Лузина, а Витали?

	Две точки отрезка $x,y \in [0;1]$ назовём эквивалентными, если
	$x - y = a/2^k$, где $k\in\mathbb{N}$, $a\in\mathbb{Z}$.
	Множество $L$ составим, взяв из каждого класса эквивалентности по одному представителю.
	Тогда $L$, очевидно, неизмеримо (доказательство дословно повторяет доказательство неизмеримости множества Лузина),
	однако является разделяющим, т.к. для каждой последовательности из $\omega\in\Omega$
	найдётся элемент $\hat\omega\in L$, такой, что $\omega$ и $\hat\omega$ отличаются лишь конечным числом членов.
\end{remark}

