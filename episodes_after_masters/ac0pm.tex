\begin{definition}
	Пусть $x\in ac_0$ и пусть $x = x^+ +x^-$, $x^+\geq 0$, $x^- \leq 0$.
	Если $x^+ \in ac_0$ (или, эквивалентно, $x^- \in ac_0$),
	то будем говорить, что $x\in ac_0^\pm$.
\end{definition}

\begin{hypothesis}
	\label{hyp:ac0pm_space}
	Множество $ac_0^\pm$ замкнуто, т.е. является подпространством $ac_0$.
	(Замкнутость относительно сложения очевидна.)
\end{hypothesis}

\begin{hypothesis}
	Множество $x\in ac_0^\pm$ замкнуто по умножению, т.е. для любых $x, y\in ac_0^\pm$
	покоординатное произведение $x\cdot y \in ac_0^\pm$.
\end{hypothesis}


\begin{hypothesis}
	Верно включение $ac_0^\pm \subset \operatorname{St} ac_0$, т.е. для любых $x \in ac_0$, $y\in ac_0^\pm$
	покоординатное произведение $x\cdot y \in ac_0$.
\end{hypothesis}

\begin{hypothesis}
	Оба вложения в цепочке $c_0 \subset ac_0^\pm \subset ac_0$ недополняемы.
	(Это, конечно, если верна гипотеза~\ref{hyp:ac0pm_space} и $ac_0^\pm$ "--- действительно пространство.)
	(Первое включение доказывается аналогично моему препринту на арксиве "--- на него и желательно ссылаться.)
\end{hypothesis}


\begin{hypothesis}
	Верно равенство $ac_0^\pm = \operatorname{St} ac_0$.
	Ну или неверно, это тоже интересно.
\end{hypothesis}

%TODO: срезочный критерий для $ac_0^\pm$
