\newcommand{\Np}{\ensuremath{\mathbb{N}_p}}
\begin{designation}
	Через $\Np$ будем обозначать множество всех простых чисел:
	$\Np = \{2,3,5,7,11...\}$.
\end{designation}

\begin{definition}
	Через $\psi(A)$, где $A\subset\mathbb{N}$,
	будем обозначать последовательность из нулей и единиц,
	определяемую соотношением
	\begin{equation}
		(\psi(A))_k = \begin{cases}
			0, & \mbox{~если~} k = m\cdot a, m \in \mathbb{N}, a\in A,
			\\
			1  & \mbox{~иначе.}
		\end{cases}
	\end{equation}
\end{definition}



%\begin{theorem}
%	$\psi(A) \in ac$.
%	
%	? Только для подмножеств простых чисел ?
%\end{theorem}
%
%\begin{proof}
%\end{proof}


\begin{lemma}
	Для любого непустого $A\subset \mathbb{N} $ выполнено $\psi(A) \notin ac_1$.
\end{lemma}
\begin{proof}
	Пусть $a_1\in A$.
	Тогда из каждых идущих подряд $a_1$ элементов последовательности $\psi(A)$
	хотя бы один нулевой,
	следовательно,
	\begin{equation}
		p(\psi(A)) \leq \frac{a_1-1}{a_1} < 1
		.
	\end{equation}
\end{proof}

Оказывается, в более узком случае верно и более сильное утверждение.

\begin{lemma}
	Для любого бесконечного $A \subset \Np $ выполнено $q(\psi(A))=0$.
\end{lemma}
\begin{proof}
	Пусть $A = \{ a_1, a_2, ..., a_j, ... \}$.
	Для краткости будем обозначать $\psi(A) = x$ и
	\begin{equation}
		\label{eq:ac0_primes_A_j_prod_des}
		A_j = \prod_{i=1}^j a_i
		.
	\end{equation}
	
	Предъявим такие номера $n_k$, что для любого $k$
	\begin{equation}
		x_{n_k+1} = x_{n_k+2} = \dots = x_{n_k+k} = 0
		.
	\end{equation}
	Тем самым мы докажем, что отрезок из любого наперёд заданного количества нулей подряд
	встречается в последовательности $x$ и, следовательно, $q(x) = 0$.
	
	Действительно,
	пусть $n_1 = a_1 - 1$.
	Рассмотрим множество  $F_1 = \{ n_1 + A_1, n_1 + 2A_1, n_1 + 3A_1, \dots, n_1 + a_2A_1 \}$
	и отметим два следующих факта.
	
	Во-первых, пусть $f \in F_1$,
	тогда
	\begin{equation}
		f \equiv n_1 \mod a_1
		.
	\end{equation}
	Во-вторых, числа $a_2$ и $A_1$ взаимно просты (поскольку $a_2$ "--- простое).
	Следовательно, все $a_2$ чисел из множества $F_1$ дают разные остатки при делении на $a_2$.
	
	В качестве $n_2$ возьмём такое $f\in F_1$, что
	\begin{equation}
		f \equiv a_2 - 2 \mod a_2
		.
	\end{equation}
	Заметим, что тогда
	\begin{equation}
		n_2 + 1 \equiv n_1 + 1 \equiv 0 \mod a_1
	\end{equation}
	и
	\begin{equation}
		n_2 + 2 \equiv 0 \mod a_2
		,
	\end{equation}
	следовательно, 
	$x_{n_2 + 1} = x_{n_2 + 2} = 0$.
	
	Полученные рассуждения несложно продолжить по индукции.

	Рассмотрим множество  $F_{j+1} = \{ n_j + A_j, n_j + 2A_j, n_j + 3A_j, \dots, n_j + a_{j+1}A_j \}$
	и отметим два следующих факта.
	
	Во-первых, пусть $f \in F_{j+1}$,
	тогда
	\begin{equation}
		f \equiv n_j \mod A_j
	\end{equation}
	и, в силу~\eqref{eq:ac0_primes_A_j_prod_des},
	\begin{equation}
		\begin{array}{rl}
		f &\equiv n_j \mod a_1
		\\
		f &\equiv n_j \mod a_2
		\\
		&\dots
		\\
		f &\equiv n_j \mod a_j
		.
		\end{array}
	\end{equation}
	Во-вторых, числа $a_{j+1}$ и $A_j$ взаимно просты (поскольку $a_{j+1}$ "--- простое).
	Следовательно, все $a_{j+1}$ чисел из множества $F_j$ дают разные остатки при делении на $a_{j+1}$.
	
	В качестве $n_{j+1}$ возьмём такое $f\in F_j$, что
	\begin{equation}
		f \equiv a_{j+1} - (j+1) \mod a_{j+1}
		.
	\end{equation}
	Заметим, что тогда
	\begin{equation}
		\begin{array}{l}
			n_{j+1} + 1 \equiv n_j + 1 \equiv n_{j-1} + 1 \equiv \dots \equiv n_2 + 1 \equiv n_1 + 1 \equiv 0 \mod a_1
			\\
			n_{j+1} + 2 \equiv n_j + 2 \equiv n_{j-1} + 2 \equiv \dots \equiv n_2 + 2 \equiv 0 \mod a_2
			\\
			\dots
			\\
			n_{j+1} + j \equiv n_j + j  \equiv 0 \mod a_j
		\end{array}
	\end{equation}
	и
	\begin{equation}
		n_{j+1} + (j+1) \equiv 0 \mod a_{j+1}
		,
	\end{equation}
	следовательно, 
	$x_{n_{j+1} + j+1} = x_{n_{j} + j} = \dots = x_{n_2 + 1} = x_{n_2 + 2} = 0$.
	
	
\end{proof}




В~\cite{euler1737variae} доказана следующая
\begin{theorem}
	\label{thm:Euler_reverse_primes_diverge}
	Ряд обратных простых чисел
	\begin{equation}
		\frac{1}{2} + \frac{1}{3} + \frac{1}{5} + \frac{1}{7} + ...
		=
		\sum_j \frac{1}{j},
	\end{equation}
	где $j$ пробегает все простые числа, расходится.
\end{theorem}

Из теоремы~\ref{thm:Euler_reverse_primes_diverge} выводится следующее утверждение.

\begin{corollary}
	Ряд
	\begin{equation}
		\sum_j \ln \frac{j+1}{j}
		,
	\end{equation}
	где $j$ пробегает все простые числа, расходится.
\end{corollary}
\begin{proof}
	Из второго замечательного предела:
	\begin{equation}
		\lim_{n\to\infty} \left( 1 + \frac{1}{n} \right)^n = e
	\end{equation}
	получаем, что для достаточно больших $n$
	\begin{equation}
		\sqrt{e} < \left( 1 + \frac{1}{n} \right)^n < e^2
		.
	\end{equation}
	Прологарифмировав~\eqref{eq:ac0_primes_ineq_2nd_nice_lim}, получим
	\begin{equation}
		\frac{1}{2} < n \ln  1 + \frac{1}{n} < 2
		,
	\end{equation}
	откуда
	\begin{equation}
		\ln \frac{n+1}{n} > \frac{1}{2n}
		.
	\end{equation}
	Пусть теперь $j$ пробегает все простые числа, тогда
	\begin{equation}
		\sum_j \ln \frac{j+1}{j}  > \sum_j \frac{1}{2j} = \frac{1}{2} \sum_j \frac{1}{j}
		.
	\end{equation}
	Ряд в правой части есть ряд обратных простых чисел, который расходится в силу теоремы~\ref{thm:Euler_reverse_primes_diverge}.
	Про признаку сравнения отсюда следует, что и ряд в левой части расходится.
\end{proof}





\begin{lemma}
	Сумма логарифмов хитро подобранных чисел может сходиться куда угодно
\end{lemma}
\begin{proof}
\end{proof}

%\begin{theorem}
%	Для любого $s\in {[0;1)} $ существует такое $A\subset \Np$, что $p\psi(A)\in ac_s$.
%\end{theorem}
%\begin{proof}
%\end{proof}

