%\newcommand{\Np}{\ensuremath{\mathbb{N}_p}}
%\begin{designation}
%	Через $\Np$ будем обозначать множество всех простых чисел:
%	$\Np = \{2,3,5,7,11,...\}$.
%\end{designation}

\begin{definition}
	Через $\psi(A)$, где $A\subset\mathbb{N}\setminus\{1\}$,
	будем обозначать последовательность из нулей и единиц,
	определяемую соотношением
	\begin{equation}
		(\psi(A))_k = \begin{cases}
			0, & \mbox{~если~} k = m\cdot a, m \in \mathbb{N}, a\in A,
			\\
			1  & \mbox{~иначе.}
		\end{cases}
	\end{equation}
\end{definition}



%\begin{theorem}
%	$\psi(A) \in ac$.
%	
%	? Только для подмножеств простых чисел ?
%\end{theorem}
%
%\begin{proof}
%\end{proof}


\begin{lemma}
	Для любого непустого $A\subset \mathbb{N} $ выполнено $\psi(A) \notin ac_1$.
\end{lemma}
\begin{proof}
	Пусть $a_1\in A$.
	Тогда из каждых идущих подряд $a_1$ элементов последовательности $\psi(A)$
	хотя бы один нулевой,
	следовательно,
	\begin{equation}
		p(\psi(A)) \leq \frac{a_1-1}{a_1} < 1
		.
	\end{equation}
\end{proof}

Оказывается, в более узком случае верно и более сильное утверждение.

\begin{lemma}
	\label{lem:ac0_primes_infinity_mutually_prime_subset}
	Пусть $A'$ "--- бесконечное подмножество попарно взаимно простых чисел
	(т.е. для любых двух чисел $a_1, a_2 \in A'$ их наибольший общий делитель равен единице).
	Тогда для любого $A \supset A' $ выполнено $q(\psi(A))=0$.
\end{lemma}
\begin{proof}
	Пусть $A' = \{ a_1, a_2, ..., a_j, ... \}$.
	Для краткости будем обозначать $\psi(A) = x$ и
	\begin{equation}
		\label{eq:ac0_primes_A_j_prod_des}
		A_j = \prod_{i=1}^j a_i
		.
	\end{equation}
	
	Предъявим такие номера $n_k$, что для любого $k$
	\begin{equation}
		x_{n_k+1} = x_{n_k+2} = \dots = x_{n_k+k} = 0
		.
	\end{equation}
	Тем самым мы докажем, что отрезок из любого наперёд заданного количества нулей подряд
	встречается в последовательности $x$ и, следовательно, $q(x) = 0$.
	
	Действительно,
	пусть $n_1 = a_1 - 1$.
	Рассмотрим множество  $F_1 = \{ n_1 + A_1, n_1 + 2A_1, n_1 + 3A_1, \dots, n_1 + a_2A_1 \}$
	и отметим два следующих факта.
	
	Во-первых, пусть $f \in F_1$,
	тогда
	\begin{equation}
		f \equiv n_1 \mod a_1
		.
	\end{equation}
	Во-вторых, числа $a_2$ и $A_1$ взаимно просты.
	Следовательно, все $a_2$ чисел из множества $F_1$ дают разные остатки при делении на $a_2$.
	
	В качестве $n_2$ возьмём такое $f\in F_1$, что
	\begin{equation}
		f \equiv a_2 - 2 \mod a_2
		.
	\end{equation}
	Заметим, что тогда
	\begin{equation}
		n_2 + 1 \equiv n_1 + 1 \equiv 0 \mod a_1
	\end{equation}
	и
	\begin{equation}
		n_2 + 2 \equiv 0 \mod a_2
		,
	\end{equation}
	следовательно, 
	$x_{n_2 + 1} = x_{n_2 + 2} = 0$.
	
	Полученные рассуждения несложно продолжить по индукции.

	Рассмотрим множество  $F_{j+1} = \{ n_j + A_j, n_j + 2A_j, n_j + 3A_j, \dots, n_j + a_{j+1}A_j \}$
	и отметим два следующих факта.
	
	Во-первых, пусть $f \in F_{j+1}$,
	тогда
	\begin{equation}
		f \equiv n_j \mod A_j
	\end{equation}
	и, в силу~\eqref{eq:ac0_primes_A_j_prod_des},
	\begin{equation}
		\begin{array}{rl}
		f &\equiv n_j \mod a_1
		\\
		f &\equiv n_j \mod a_2
		\\
		&\dots
		\\
		f &\equiv n_j \mod a_j
		.
		\end{array}
	\end{equation}
	Во-вторых, числа $a_{j+1}$ и $A_j$ взаимно просты, поскольку $a_{j+1}$ взаимно просто с каждым из чисел $a_1,...,a_j$.
	Следовательно, все $a_{j+1}$ чисел из множества $F_j$ дают разные остатки при делении на $a_{j+1}$.
	
	В качестве $n_{j+1}$ возьмём такое $f\in F_j$, что
	\begin{equation}
		f \equiv a_{j+1} - (j+1) \mod a_{j+1}
		.
	\end{equation}
	Заметим, что тогда
	\begin{equation}
		\begin{array}{l}
			n_{j+1} + 1 \equiv n_j + 1 \equiv n_{j-1} + 1 \equiv \dots \equiv n_2 + 1 \equiv n_1 + 1 \equiv 0 \mod a_1
			\\
			n_{j+1} + 2 \equiv n_j + 2 \equiv n_{j-1} + 2 \equiv \dots \equiv n_2 + 2 \equiv 0 \mod a_2
			\\
			\dots
			\\
			n_{j+1} + j \equiv n_j + j  \equiv 0 \mod a_j
		\end{array}
	\end{equation}
	и
	\begin{equation}
		n_{j+1} + (j+1) \equiv 0 \mod a_{j+1}
		,
	\end{equation}
	следовательно, 
	$x_{n_{j+1} + j+1} = x_{n_{j} + j} = \dots = x_{n_2 + 1} = x_{n_2 + 2} = 0$.
	
	
\end{proof}

\begin{remark}
	Понятно, что в качестве примера бесконечного множества
	попарно взаимно простых чисел можно взять любое бесконечное множество простых чисел.
	Однако бывают бесконечные множества попарно взаимно простых чисел,
	не содержащие простых чисел вовсе, например множество
	\begin{equation}
		A = \{ 2\cdot 3,~5 \cdot 7,~11 \cdot 13,~17\cdot 19,~23\cdot29,~31\cdot 37,~...\},
	\end{equation}
	элементами которого яляются произведения пар соседних простых чисел.
\end{remark}

\begin{definition}
	Будем говорить, что множество $A\subset\mathbb{N}$ обладает $prp$-свойством (англ. ``pairwise relatively prime''),
	если для любого $n\in\mathbb{N}$ найдётся набор попарно взаимно простых чисел
	\begin{equation}
		\{a_{n,1}, a_{n,2}, ..., a_{n,n}  \} \subset A
		.
	\end{equation}
\end{definition}

Из доказательства леммы~\ref{lem:ac0_primes_infinity_mutually_prime_subset} понятно,
что для множества $A$ в условии леммы достаточно потребовать $prp$-свойства.
Интересно, что на самом деле $prp$-свойство эквивалентно условиям,
наложенным на множество $A$ в лемме~\ref{lem:ac0_primes_infinity_mutually_prime_subset}.

\begin{lemma}
	\label{lem:ac0_primes_infinity_mutually_prime_subset_equiv_to_prp_property}
	Пусть множество $A$ обладает $prp$-свойством.
	Тогда существует бесконечное подмножество $A'\subset A$ попарно взаимно простых чисел.
\end{lemma}
\begin{proof}
	Зафиксируем $f_0\in A$, $f_0 \neq 1$ и представим $A$ в виде объединения трёх попарно непересекающихся множеств:
	\begin{equation}
		A = \{f_0\} \cup E \cup F
		,
	\end{equation}
	где
	\begin{equation}
		E = \{ a \in A \mid a \mbox{~не взаимно просто с~} f_0 \}
		,
	\end{equation}
	\begin{equation}
		F = \{ a \in A \mid a \mbox{ ~~~~взаимно просто с~} f_0 \}
		.
		%TODO: нормальное выравнивание!!
	\end{equation}
	
	Пусть разложение $f_0$ на простые множители имеет вид
	\begin{equation}
		f_0 = p_1^{j_1} \cdot p_2^{j_2} \cdot ... \cdot p_k^{j_k}
		.
	\end{equation}

	Тогда множество $E$ можно представить в виде объединения (возможно пересекающихся) множеств:
	\begin{equation}
		E = \bigcup_{i=1}^{k} E_i,\quad E_i = \{a\in E \mid a \mbox{~кратно~} p_i\}
		.
	\end{equation}
	
	Покажем, что множество $F$ обладает $prp$-свойством.
	Действительно, зафиксируем $n\in\mathbb{N}$.
	Так как множество $A$ обладает $prp$-свойством,
	то в нём найдётся подмножество попарно взаимно простых чисел
	$$G=\{a_1, a_2, ..., a_{n+k-1}, a_{n+k}\}\subset A.$$
	
	Если $f_0\in G$, то $G\setminus f_0 \subset F$ в силу построения множеств $G$ и $F$, и требуемый набор попарно взаимно простых чисел предъявлен.
	
	Пусть теперь $f_0\notin G$.
	Заметим, что в каждое из множеств $E_i$ может входит не более одного элемента множества $G$
	(в силу того, что при фиксированном $i$ все элементы множества $E_i$ имеют нетривиальный общий делитель).
	Следовательно, как минимум $n$ элементов из $G$ принадлежат множеству $F$,
	и требуемый набор попарно взаимно простых чисел снова предъявлен.
	
	Поскольку множество $F$ обладает $prp$-свойством, то оно, очевидно, счётно.
	
	Итак, нам удалось получить число $f_0\in A$ и счётное множество $F$, обладающее $prp$-свойством
	и состоящее из чисел, взаимно простых с $f_0$.
	Продолжая по индукции, получим требуемое счётное множество попарно взаимно простых чисел. 
\end{proof}

%Условие леммы~\ref{lem:ac0_primes_infinity_mutually_prime_subset}
%является не только достаточным, но и необходимым.

\begin{lemma}
	\label{lem:ac0_primes_q_psi_A_0_causes_prp}
	Пусть для множества $A\subset\mathbb{N}\setminus\{1\}$ выполнено $q(\psi(A))=0$.
	Тогда $A$ обладает $prp$-свойством.
\end{lemma}

\begin{proof}
	Предположим противное: пусть $A$ не обладает $prp$-свойством.
	Тогда существует $n\in\mathbb{N}$ такое, что из множества $A$ можно выбрать $n$
	попарно взаимно простых чисел, но нельзя выбрать $n+1$.
	
	Пусть $\{a_1, a_2, ..., a_n\}\subset A$ "--- набор попарно взаимно простых чисел.
	Так как $q(\psi(A))=0$, то в последовательности $\psi(A)$ найдётся отрезок, состоящий сплошь из нулей,
	любой наперёд заданной длины.
	Выберем $k$ таким, что
	\begin{equation}
		(\psi(A))_{k+1} = (\psi(A))_{k+2} = ... = (\psi(A))_{k+a_1a_2\cdots a_n} = 0
		.
	\end{equation}
	Тогда существует такое число $k_1$, $k+1 \leq k_0 \leq k+a_1a_2\cdots a_n$,
	что $k_0$ даёт в остатке $1$ при делении на $a_1a_2\cdots a_n$.
	Так как $(\psi(A))_{k_0} = 0$, что $k_0 = m\cdot a_0$, где $m\in\mathbb{N}$, $a_0\in A$.
	С другой стороны, $k_0$ взаимно просто с каждым из чисел $a_1, a_2, ..., a_n$.
	Следовательно, $a_0$ также взаимно просто с каждым из чисел $a_1, a_2, ..., a_n$,
	и $\{a_0, a_1, a_2, ..., a_n\}\subset A$ "--- набор из $n+1$ попарно взаимно простых чисел.
	Полученное противоречие завершает доказательство.
\end{proof}

Таким образом,
из лемм~\ref{lem:ac0_primes_infinity_mutually_prime_subset},~\ref{lem:ac0_primes_infinity_mutually_prime_subset_equiv_to_prp_property}~и~\ref{lem:ac0_primes_q_psi_A_0_causes_prp}
незамедлительно следует
\begin{theorem}
	Пусть $A\subset \mathbb{N}\setminus\{1\}$.
	Тогда следующие три условия эквивалентны:
	\begin{enumerate}
		\item
			$A$ обладает $prp$-свойством
		\item
			В $A$ существует счётное подмножество попарно взаимно простых чисел
		\item
			$q(\psi(A))=0$.
	\end{enumerate}
\end{theorem}

\longcomment{

В~\cite{euler1737variae} доказана следующая
\begin{theorem}
	\label{thm:Euler_reverse_primes_diverge}
	Ряд обратных простых чисел
	\begin{equation}
		\frac{1}{2} + \frac{1}{3} + \frac{1}{5} + \frac{1}{7} + ...
		=
		\sum_j \frac{1}{j},
	\end{equation}
	где $j$ пробегает все простые числа, расходится.
\end{theorem}

Из теоремы~\ref{thm:Euler_reverse_primes_diverge} выводится следующее утверждение.

\begin{corollary}
	Ряд
	\begin{equation}
		\sum_j \ln \frac{j+1}{j}
		,
	\end{equation}
	где $j$ пробегает все простые числа, расходится.
\end{corollary}
\begin{proof}
	Из второго замечательного предела:
	\begin{equation}
		\lim_{n\to\infty} \left( 1 + \frac{1}{n} \right)^n = e
	\end{equation}
	получаем, что для достаточно больших $n$
	\begin{equation}
		\sqrt{e} < \left( 1 + \frac{1}{n} \right)^n < e^2
		.
	\end{equation}
	Прологарифмировав~\eqref{eq:ac0_primes_ineq_2nd_nice_lim}, получим
	\begin{equation}
		\frac{1}{2} < n \ln  1 + \frac{1}{n} < 2
		,
	\end{equation}
	откуда
	\begin{equation}
		\ln \frac{n+1}{n} > \frac{1}{2n}
		.
	\end{equation}
	Пусть теперь $j$ пробегает все простые числа, тогда
	\begin{equation}
		\sum_j \ln \frac{j+1}{j}  > \sum_j \frac{1}{2j} = \frac{1}{2} \sum_j \frac{1}{j}
		.
	\end{equation}
	Ряд в правой части есть ряд обратных простых чисел, который расходится в силу теоремы~\ref{thm:Euler_reverse_primes_diverge}.
	Про признаку сравнения отсюда следует, что и ряд в левой части расходится.
\end{proof}


}


%\begin{lemma}
%	Сумма логарифмов хитро подобранных чисел может сходиться куда угодно
%\end{lemma}
%\begin{proof}
%\end{proof}

%\begin{theorem}
%	Для любого $s\in {[0;1)} $ существует такое $A\subset \Np$, что $p\psi(A)\in ac_s$.
%\end{theorem}
%\begin{proof}
%\end{proof}

