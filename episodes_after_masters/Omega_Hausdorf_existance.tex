Хорошо известны некоторые примеры разделяющих множеств
[TODO: много ссылок, в т.ч. на новую статью в МЗ].
Применяя лемму~\ref{lem:Hausdorf_measure}, можно показать,
что хаусдорфову размерность 1 имеют множества
TODO: список!

В данном пункте строится пример разделяющего множества,
имеющего малую хаусдорфову размерность.


\begin{theorem}
	\label{lem:Hausdorf_measure}
	Пусть $n\in\mathbb{N}$.
	Тогда существует разделяющее множество $E\subset\Omega$ такое,
	что $\dim_H E = 1/n$.
\end{theorem}

\begin{proof}
	Пусть
	\begin{equation}
		Е= \{ x \in \Omega : k \neq mn \Rightarrow x_k = 0\}, m\in \mathbb{N}
		,
	\end{equation}
	т.е. у последовательности $x \in E$ равны нулю все элементы, кроме, быть может, $x_n$, $x_{2n}$, $x_{3n}$ и т.д.

	Для $j=1,2$ определим функции $f_j : \Omega \to \Omega$ следующим образом:
	\begin{equation}
		f_1(x_1, x_2, \dots)=(\underbrace{0, ..., 0,}_{\mbox{$n-1$ раз}} 0, x_1, x_2, \dots)
		,
		\quad
		f_2(x_1, x_2, \dots)=(\underbrace{0, ..., 0,}_{\mbox{$n-1$ раз}} 1, x_1, x_2, \dots)
		.
	\end{equation}

	Очевидно, что $E=f_1(E)\cup f_2(E).$

	Теперь мы покажем, что размерность Хаусдорфа множества $E$ равна $2^{-n}$.

	Т.к. для $j=1,2$ верно
	 $$\rho(f_j(x),f_j(y))=2^{-n}\rho(x,y), \ \forall \ x, y \in E,$$
	 то функции $f_j$ являются преобразованиями подобия с коэффициентами $r_j=2^{-n}$ для $j=1,2$.


	По~\cite[Теорема 9.3]{Edgar} размерность Хаусдорфа $d$ множества $E$ является решением уравнения:
	$$ r_1^d+r_2^d=1.$$
	Т.к. $r_j=2^{-n}$, то
	$d=1/n.$
\end{proof}





