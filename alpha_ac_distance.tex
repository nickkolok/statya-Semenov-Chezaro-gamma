Результаты этого пункта опубликованы в~\cite{our-mz2019ac0}.

Пусть $\rho(x,c)$ и $\rho(x,c_0)$~--- расстояния от $x$ до пространства сходящихся последовательностей $c$
и пространства сходящихся к нулю последовательностей $c_0$ соответственно.

Из условия Липшица на $\alpha$--функцию \eqref{alpha_Lipshitz}
и того, что на пространстве $c$
$\alpha$--функция обращается в нуль следует

\begin{lemma}
\label{thm:alpha_x_leq_2_rho_x_c}
	Для любого $x\in\ell_\infty$
	\begin{equation}
		\alpha(x) \leq 2\rho(x, c)
		.
	\end{equation}
\end{lemma}

%TODO: доказывать или очевидно?

Эта оценка точна.
\begin{example}
\label{ex:alpha_ac_rho_x_c}
	\begin{equation}
	\label{eq:alpha_ac_distance_example_y}
		x_k = \begin{cases}
			(-1)^n, &\mbox{~если~} k = 2^n,
			\\
			0 &\mbox{~иначе.}
		\end{cases}
	\end{equation}
\end{example}
Здесь $\alpha(x) = 2$, $\alpha(T^s x) = 1$ для любого $s\in\mathbb{N}$, $\rho(x,c) = \rho(x, c_0) = 1$.

Как выясняется, верна и оценка с другой стороны.

\begin{lemma}
\label{thm:rho_x_c_leq_alpha_t_s_x}
	Для любого $x\in ac$ и для любого натурального $s$
	\begin{equation}
		\rho(x,c)\leq \alpha(T^s x)
		.
	\end{equation}
\end{lemma}
\paragraph{Доказательство.}
Зафиксируем $s$.
Пусть $x\in ac$, $\alpha(T^s x)=\varepsilon$.
Так как $x\in ac$, то по критерию Лоренца существует такое число $t$,
что
\begin{equation}
	\lim_{n\to\infty} \frac{1}{n} \sum_{k=m+1}^{m+n} x_k = t
\end{equation}
равномерно по $m$.
Иначе говоря,
\begin{equation}
	\forall(p\in\mathbb{N})
	\exists(n_p \in\mathbb{N})\forall(n>n_p)\forall(m\in\mathbb{N})
	\left[
		\left|
			\frac{1}{n}\sum_{k=m+1}^{m+n} x_k
			-t
		\right|
		<\frac{\varepsilon}{p}
	\right]
	.
\end{equation}

Так как
\begin{equation}
	\alpha(T^s x) = \varlimsup_{k\to\infty} \max_{k<j\leq 2k-s} |x_k-x_j| = \varepsilon
	,
\end{equation}
то
\begin{equation}
	\forall(p\in\mathbb{N})
	\exists(k_p \in\mathbb{N})\forall(k>k_p)
	\forall(j: k< j \leq 2k-s)
	\left[
		|x_k - x_j|<\varepsilon + \frac{\varepsilon}{p}
	\right]
	.
\end{equation}
Положив $q_p = \max\{n_p, k_p\}$, имеем
\begin{equation*}
	\forall(p\in\mathbb{N})
	\exists(q_p \in\mathbb{N})
	\forall(q>q_p)
	\left[
		\left|
			\frac{1}{n}\sum_{k=m+1}^{m+n} x_k
			-t
		\right|
		<\frac{\varepsilon}{p}
		%\right.\\ \left. \phantom{\sum_0^0}
		\mbox{~~и~~}
		\forall(k:q<k \leq 2q-s)
		\left[
			|x_q - x_k|<\varepsilon + \frac{\varepsilon}{p}
		\right]
	\right]
	.
\end{equation*}
Т.е. среднее арифметическое чисел $x_q$, $x_{q+1}$, ... , $x_{2q-s}$ отличается от $t$
не более, чем на $\varepsilon/p$, причём разница любых двух из этих чисел меньше $\varepsilon + \varepsilon/p$.
Следовательно, любое из чисел $x_q$, $x_{q+1}$, ... , $x_{2q-s}$
отличается от $t$ менее, чем на $\varepsilon + 2\varepsilon/p$.
В частности,
\begin{equation}
	|x_q - t| < \varepsilon + \frac{2\varepsilon}{p}
	.
\end{equation}
Таким образом,
\begin{equation}
	\forall(p\in\mathbb{N})
	\exists(q_p \in\mathbb{N})
	\forall(q>q_p)
	\left[
		|x_q - t| < \varepsilon + \frac{2\varepsilon}{p}
	\right]
	,
\end{equation}
откуда немедленно следует, что $\rho(x,c) \leq \varepsilon$,
что и требовалось доказать.

Из лемм \ref{thm:alpha_x_leq_2_rho_x_c} и \ref{thm:rho_x_c_leq_alpha_t_s_x}
незамедлительно следует
\begin{theorem}
\label{thm:rho_x_c_leq_alpha_t_s_x_united}
	Для любого $x\in ac$
	\begin{equation}
		\frac{1}{2} \alpha(x) \leq \rho(x,c)\leq \lim_{s\to\infty} \alpha(T^s x)
		.
	\end{equation}
\end{theorem}

\begin{corollary}
	Для любого $x\in ac_0$
	\begin{equation}
		\frac{1}{2} \alpha(x) \leq \rho(x,c_0)\leq \lim_{s\to\infty} \alpha(T^s x)
		.
	\end{equation}
\end{corollary}

Точность оценок показывают пример \ref{ex:alpha_ac_rho_x_c} и следующие примеры.
\begin{example}
	\begin{equation}
		x_k = \begin{cases}
			1, &\mbox{~если~} k = 2^n,
			\\
			0 &\mbox{~иначе.}
		\end{cases}
	\end{equation}
\end{example}
Здесь $\alpha(T^s x) = 1$ для любого $s\in\mathbb{N}$, $\rho(x,c) = 1/2$, $\rho(x, c_0) = 1$.

\begin{example}
	\begin{equation}
		x_k = \begin{cases}
			1, &\mbox{~если~} k = 2^n,
			\\
			-1, &\mbox{~если~} k = 2^n + 1 > 2
			\\
			0 &\mbox{~иначе.}
		\end{cases}
	\end{equation}
\end{example}
Здесь $\alpha(T^s x) = 2$ для любого $s\in\mathbb{N}$, $\rho(x,c) = \rho(x, c_0) = 1$.


\begin{hypothesis}
	Для любого $x\in ac$
	\begin{equation}
		\frac{1}{2} \lim_{s\to\infty} \alpha(U^s x) \leq \rho(x,c)
		,
	\end{equation}
	где $U$~--- оператор сдвига вправо:
	\begin{equation}
		U(x_1, x_2, x_3, ...) = (0, x_1, x_2, x_3, ...)
		.
	\end{equation}
\end{hypothesis}

