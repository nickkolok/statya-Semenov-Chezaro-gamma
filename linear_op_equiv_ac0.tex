\paragraph{Лемма.}

Пусть $P,Q:\ell_\infty \to \ell_\infty$~--- линейные операторы и
\begin{equation}\label{P-Q:ell_infty_to_ac0}
	P-Q : \ell_\infty \to ac_0
	.
\end{equation}
Тогда
\begin{equation}
	\mathfrak{B}(P)=\mathfrak{B}(Q)
	.
\end{equation}

\paragraph{Доказательство.}
Пусть $B\in \mathfrak{B}(P)$.
Тогда
\begin{equation}
	B(Qx) = B((Q-P+P)x) =
	B((Q-P)x)+B(Px) =
	0 + B(Px) =
	Bx
	.
\end{equation}
Значит, $\mathfrak{B}(P) \subset \mathfrak{B}(Q)$
В силу симметричности утверждения леммы получаем $\mathfrak{B}(Q) \subset \mathfrak{B}(P)$,
откуда и следует требуемое утверждение.

Только что доказанная лемма означает, что множество всех линейных операторов,
действующих из $\ell_\infty$ в $\ell_\infty$, можно разбить на классы по отношению эквивалентности,
задаваемому условием \eqref{P-Q:ell_infty_to_ac0},
и тогда для операторов из одного класса множества инвариантных банаховых пределов будут совпадать.

TODO:
Неизвестно, верно ли обратное утверждение.
Скорее всего, существует контпример для $\mathfrak{B}(Q) = \mathfrak{B}(P) = \varnothing$.
Существует ли более содержательный контрпример --- пока неизвестно.

