\begin{lemma}
	Пусть $P,Q:\ell_\infty \to \ell_\infty$~--- линейные операторы и
	\begin{equation}\label{P-Q:ell_infty_to_ac0}
		P-Q : \ell_\infty \to ac_0
		.
	\end{equation}
	Тогда
	\begin{equation}
		\mathfrak{B}(P)=\mathfrak{B}(Q)
		.
	\end{equation}
\end{lemma}

\paragraph{Доказательство.}
Пусть $B\in \mathfrak{B}(P)$.
Тогда
\begin{equation}
	B(Qx) = B((Q-P+P)x) =
	B((Q-P)x)+B(Px) =
	0 + B(Px) =
	Bx
	.
\end{equation}
Значит, $\mathfrak{B}(P) \subset \mathfrak{B}(Q)$.
В силу симметричности утверждения леммы получаем $\mathfrak{B}(Q) \subset \mathfrak{B}(P)$,
откуда и следует требуемое утверждение.

Только что доказанная лемма означает, что множество всех линейных операторов,
действующих из $\ell_\infty$ в $\ell_\infty$, можно разбить на классы по отношению эквивалентности,
задаваемому условием \eqref{P-Q:ell_infty_to_ac0},
и тогда для операторов из одного класса множества инвариантных банаховых пределов будут совпадать.

Обратное, однако, неверно.

\begin{example}
	Пусть
	\begin{equation}
		P(x_1,x_2,x_3) = (x_1, 0, x_3, 0, x_5, 0, ...)
	\end{equation}
	и
	\begin{equation}
		Q(x_1,x_2,x_3) = (0, -x_2, 0, -x_4, 0, -x_6, 0, ...)
		.
	\end{equation}
	Легко видеть, что для любого банахова предела $B\in\mathfrak{B}$ выполнено
	\begin{equation}
		B(P\mathbb{I}) = \frac{1}{2}
	\end{equation}
	и
	\begin{equation}
		B(Q\mathbb{I}) = -\frac{1}{2}
		,
	\end{equation}
	откуда
	\begin{equation}
		\mathfrak{B}(Q) = \mathfrak{B}(P) = \varnothing
		.
	\end{equation}
	Однако разность $P-Q$ есть не что иное, как тождественный оператор $I$,
	который переводит пространство $\ell_\infty$ само в себя,
	а не в пространсво $ac_0$.
\end{example}


\begin{hypothesis}
	Существуют два таких линейных оператора $P, Q : \ell_\infty \to \ell_\infty$,
	что $\mathfrak{B}(P) = \mathfrak{B}(Q) \neq \varnothing$,
	но $(P-Q)(\ell_\infty) \setminus ac \neq \varnothing$.
\end{hypothesis}

