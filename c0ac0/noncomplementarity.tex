\documentclass[a4paper,14pt]{article} %размер бумаги устанавливаем А4, шрифт 12пунктов
\usepackage[T2A]{fontenc}
\usepackage[utf8]{inputenc}
\usepackage[russian,english]{babel} %используем русский и английский языки с переносами
\usepackage{amssymb,amsfonts,amsmath,mathtext,enumerate,float,amsthm} %подключаем нужные пакеты расширений
\usepackage[unicode,colorlinks=true,citecolor=black,linkcolor=black]{hyperref}
%\usepackage[pdftex,unicode,colorlinks=true,linkcolor=blue]{hyperref}
\usepackage{indentfirst} % включить отступ у первого абзаца
\usepackage[dvips]{graphicx} %хотим вставлять рисунки?
\graphicspath{{illustr/}}%путь к рисункам
\usepackage{enumitem}



\makeatletter
\renewcommand{\@biblabel}[1]{#1.} % Заменяем библиографию с квадратных скобок на точку:
\makeatother %Смысл этих трёх строчек мне непонятен, но поверим "Запискам дебианщика"

\usepackage{geometry} % Меняем поля страницы.
\geometry{left=2cm}% левое поле
\geometry{right=1cm}% правое поле
\geometry{top=2cm}% верхнее поле
\geometry{bottom=2cm}% нижнее поле

\renewcommand{\theenumi}{\arabic{enumi}}% Меняем везде перечисления на цифра.цифра
\renewcommand{\labelenumi}{\arabic{enumi}}% Меняем везде перечисления на цифра.цифра
\renewcommand{\theenumii}{.\arabic{enumii}}% Меняем везде перечисления на цифра.цифра
\renewcommand{\labelenumii}{\arabic{enumi}.\arabic{enumii}.}% Меняем везде перечисления на цифра.цифра
\renewcommand{\theenumiii}{.\arabic{enumiii}}% Меняем везде перечисления на цифра.цифра
\renewcommand{\labelenumiii}{\arabic{enumi}.\arabic{enumii}.\arabic{enumiii}.}% Меняем везде перечисления на цифра.цифра

\DeclareMathOperator{\supp}{supp}

\sloppy




\renewcommand\normalsize{\fontsize{14}{25.2pt}\selectfont}

\usepackage[backend=biber,style=gost-numeric,sorting=none]{biblatex}
\addbibresource{../bib/general_monographies.bib}
\addbibresource{../bib/ext.bib}
\addbibresource{../bib/my.bib}
\addbibresource{../bib/Semenov.bib}
\addbibresource{../bib/Bibliography_from_Usachev.bib}
\addbibresource{../bib/classic.bib}

\input{../bib/ext.hyphens.bib}


\theoremstyle{plain}
\newtheorem{theorem}{Theorem}[section]
\newtheorem{conjecture}[theorem]{Conjecture}
\newtheorem{lemma}[theorem]{Lemma}
\newtheorem{corollary}[theorem]{Corollary}
\newtheorem{proposition}[theorem]{Proposition}

\theoremstyle{definition}
\newtheorem{definition}[theorem]{Definition}
\newtheorem{construction}[theorem]{Construction}
\newtheorem{remark}[theorem]{Remark}
\newtheorem{problem}[theorem]{Problem}

\begin{document}

%\renewcommand{\bibname}{Список цитированной литературы}
%\renewcommand\refname{\bibname}
% !!!
% The text starts here

\title{
	The subspace $c_0$ is not coplemented in $ac_0$
}

\author{
	N.N. Avdeev
	\footnote{
		This work was carried out at Voronezh State University and supported by the Russian Science
		Foundation grant 19-11-00197.
	}
	\footnote{nickkolok@mail.ru, avdeev@math.vsu.ru}
}

\maketitle

\paragraph{Abstract.}
We proove that the subspace $c_0$ of sequences that converge to zero
is not complemented in the space $ac_0$ of sequences that almost converge to zero.





\section{Introduction}


The famous and unexpected result of Phillips~\cite{phillips1940linear}
states that the subspace $c_0$ of sequences that converge to zero
is not complemented in the space of all bounded sequences $\ell_\infty$
endowed with the usual norm
\begin{equation}
	\|x\|_{\ell_\infty} = \sup_{n\in\mathbb{N}} x_n
	,
\end{equation}
where $\mathbb{N}$ stands for the set of all positive integers:

\begin{theorem}[\cite{phillips1940linear}]
	\label{thm:phillips}
	A bounded linear opertor $P: \ell_\infty \to c_0$ such that for every
	$x \in c_0$ the equality $Px =x$ holds,
	does not exist.
\end{theorem}

Later, some shorter proofs of Theorem~\ref{thm:phillips} appeared, e.g.~\cite{whitley1968projecting}.


\begin{definition}
	A linear functional $B\in\ell_\infty^*$ is a \emph{Banach limit} iff
	\begin{enumerate}[label=(\roman*)]
		\item
			$B\geq0$, that is $Bx \geq 0$ for $x \geq 0$,
		\item
			$B\mathbb{I}=1$, where $\mathbb{I} =(1,1,\ldots)$,
		\item
			$B(Tx)=B(x)$ for $x\in \ell_\infty$,
			where $T$ is the shift operator $T(x_1,x_2,\ldots)=(x_2,x_3,\ldots)$.
	\end{enumerate}
\end{definition}
The set of Banach limits is denoted by $\mathfrak{B}$.
The existance of Banach limits was announced by Mazur~\cite{Mazur} and proved by Banach~\cite{banach1993theorie}.

\begin{definition}
	A sequence $x\in \ell_\infty$ is said to be \emph{almost convergent} to $t\in \mathbb{R}$
	if for every $B\in\mathfrak{B}$ the equality
	\begin{equation}
		Bx = t
	\end{equation}
	holds.
\end{definition}
The set of all sequences that are almost convergent to $t$ is denoted by $ac_t$;
the space of all almost convergent sequences is denoted by $ac$.

\begin{theorem}[Lorentz,~\cite{lorentz1948contribution}]
	A sequence $x=(x_1,x_2,...)$ is almost convergent to $t\in\mathbb{R}$ iff
	\begin{equation}
		\label{eq:crit_Lorentz}
		\lim_{n\to\infty} \frac{1}{n} \sum_{k=m+1}^{m+n} x_k = t
	\end{equation}
	uniformly by $m\in\mathbb{N}$.
\end{theorem}

Alekhno gave an elegant proof~\cite[Theorem 8]{alekhno2006propertiesII}
that $ac_0$ is not complemented in $\ell_\infty$.
The proof is based on the original Phillips's proof of Theorem~\ref{thm:phillips}
and uses somelemmas from there.

In Section~\ref{sec:c0_in_ac0} of the present article, we prove that $c_0$ is not complemented in $ac_0$.
So, all the three inclusions
\begin{equation}
	c_0 \subset \ell_\infty,
	\quad
	ac_0 \subset \ell_\infty,
	\quad\mbox{and}\quad
	c_0 \subset ac_0,
\end{equation}
are not complemented.
Our proof is inspired by Whitley's approach~\cite{whitley1968projecting}
and the discussion~\cite{mathSE_Phillips}.

In Section~\ref{sec:A0} we proceed with applying the same technique to the inclusion chain $c_0 \subset A_0 \subset \ell_\infty$,
where
\begin{equation}
	A_0 = \{x\in\ell_\infty: \alpha(x) = 0\}
	,
\end{equation}
\begin{equation}
	\alpha(x) = \varlimsup_{i\to\infty} \max_{i<j\leqslant 2i} |x_i - x_j|
	.
\end{equation}

The function $\alpha(x)$, first introduced in~\cite{our-vzms-2018},
may be considered as a way to tell \emph{how much non-Cauchy a given bounded sequence is}.
Notice that the condition $\alpha(x) =0$ is weaker than the Cauchy condition
as well as the Lorentz's criterion~\eqref{eq:crit_Lorentz} is obviously weaker than
than the classic definition of the limit.

It turned out that the functional $\alpha(x)$ is closely related to the theory of Banach limits.
In~\cite{avdeev2019space} the equality $A_0\cap ac_0 = c_0$ is proved and further investigation of $\alpha(x)$ is hold.
In~\cite{our-ped-2018-alpha-Tx} it is shown that $TA_0 = A_0$ despite the fact that for some $x\in\ell_\infty$
one has $\alpha(Tx)<\alpha(x)$.
For further investigation of $\alpha(x)$ we also refer the reader to~\cite{avdeev2021subsets}.

\section{$c_0$ is not complemented in $ac_0$}
\label{sec:c0_in_ac0}

The following lemma is a classic result, and we prove it here just for the sake of completeness.
\begin{lemma}
	\label{lem:uncountable_subsets_of_N_with_finite_intersections}
	There exists an uncountable family of subsets
	$\{S_i\}_{i\in I}$, $S_i \subset \mathbb{N}$,
	such that each $S_i$ is countable and for every $i\neq j$ the intersection $S_i \cap S_j$ is finite.
\end{lemma}

\begin{proof}
	Consider a bijection $\mathbb{N} \leftrightarrow \mathbb{Q}$
	and let $I = \mathbb{R}$.
	For $i\in I$ we set $S_i = \{q_n\}$,
	where $\{q_n\} \subset \mathbb{Q}$ is a sequence which converges monotonically to $i$.
\end{proof}









\begin{lemma}
	\label{lem:c_0_not_complemented_in_ac_0}
	For each linear operator $Q: ac_0 \to ac_0$ such that $c_0\subseteq \ker Q$,
	there exists infinite subset $S \subset \mathbb{N}$ such that
	\begin{equation}
		\forall(x \in ac_0 : \supp x \subset S)[Qx = 0]
		.
	\end{equation}
\end{lemma}


\begin{proof}
	Let $\{U_i\}_{i \in I}$ be a family of subsets of $\mathbb{N}$
	that satisfy the conditions of Lemma~\ref{lem:uncountable_subsets_of_N_with_finite_intersections}.
	Define sets $S_i$ by their characteristic functions:
	\begin{equation}
		\chi_{S_i} (n) = \begin{cases}
			1, & ~\mbox{if}~ n = k^2, k\in U_i,
			\\
			0  & ~\mbox{for other}~n.
		\end{cases}
	\end{equation}
	Obiously, the family of sets $\{S_i\}_{i \in I}$ also
	satisfies the conditions of Lemma~\ref{lem:uncountable_subsets_of_N_with_finite_intersections}.

	Suppose the contrary to the claim of the present lemma:
	\begin{equation}
		\forall(i\in I)\exists(x_i \in ac_0 : \supp x_i \subset S_i)[Q(x_i) \neq 0]
		.
	\end{equation}

	Note that $x_i \notin c_0$ because $c_0\subseteq \ker Q$.
	Without loss of generality we can assume that $\|x_i\|=1$ for all $i \in I$.

	Consider $I_n = \{i \in I\,:\,(Qx_i)_n \neq 0\}$,
	then $I = \bigcup\limits_{n\in\mathbb{N}} I_n$.
	Thus, we can find $n$ such that $I_n$ is also uncountable
	(otherwise $I$ would be countable as a countable union of countable sets,
	which contradicts to conditions of Lemma~\ref{lem:uncountable_subsets_of_N_with_finite_intersections}).

	Now set $I_{n,k} = \{i \in I_n\,:\,|(Qx_i)_n| \geq 1/k\}$,
	then $I_n = \bigcup\limits_{k\in\mathbb{N}} I_{n,k}$.
	Applying the same argument as above, one can easily see that the set $I_{n,k}$ is uncountable for some $k$.
	Let us pick such $I_{n,k}$ and proceed with it.

	So, we have an uncountable set $I_{n,k}$ and
	\begin{equation}
		\forall(i\in I_{n,k})\exists(x_i \in ac_0 : \supp x_i \subset S_i)\Bigl[\|x_i\|=1 \mbox{~~and~~} |(Qx_i)_n| \geq 1/k\Bigr]
		.
	\end{equation}

	Consider a finite set $J \subset I_{n,k}$ with $\#J>1$
	(here $\#J$ stands for the cardinality of the set $J$).
	Take
	\begin{equation}
		y = \sum_{j \in J} \operatorname{sign}{(Qx_j)_n} \cdot x_j
		.
	\end{equation}
	Since the intersection $S_i \cap S_j$ is finite for any $i \neq j$ and
	$\supp x_j \subset S_j$,
	the intersection $\bigcap\limits_{j\in J} \supp x_j$ is also finite.
	Hence, $y = f + z$,
	where $\supp f$ is finite and $\|z\| \leq 1$.

	On the other hand,
	\begin{equation}
		\label{eq:non_complemented_sum_cardinality}
		(Qy)_n = \sum_{j \in J}
		(\operatorname{sign}(Qx_j)_n)
		\cdot (Qx_j)_n \geq \frac{\# J}{k}
		.
	\end{equation}
	Note, that $f\in c_0$ and we have $Qf = 0$, because $c_0 \subseteq \ker Q$.
	Thus, $Qy = Q(f+z) = Qf + Qz = Qz$ and
	\begin{equation}
		\label{eq:norm_Q_estimate}
		\frac{\# J}{k} \leq (Qy)_n \leq \|Qy\| = \|Qz\| \leq \|Q\| \cdot \|z\| \leq \|Q\|
		.
	\end{equation}
	Due to~\eqref{eq:norm_Q_estimate}, we obtain $\# J \leq \|Q\| k$ for every $J\subset I_{n,k}$.
	This contradicts that $I_{n,k}$ is uncountable,
	and we are done.
\end{proof}

\begin{theorem}
	The subspace $c_0$ is not complemented in $ac_0$.
\end{theorem}

\begin{proof}
	Suppose the contrary:
	there exists a continuous projection $P: ac_0 \to c_0$.
	Applying Lemma~\ref{lem:c_0_not_complemented_in_ac_0} to $I-P$
	we can find infinite subset $S\subset\mathbb{N}$
	such that $\forall(x\in ac_0 : \supp x \subset S)[(I-P)x = 0]$
	(such $x \in ac_0 \setminus c_0$ exists even if $\chi_S \notin ac_0$).
	But then $Px\notin c_0$,
	which contradicts that $P$ is a projection onto $c_0$.
\end{proof}





\section{The space $A_0$}
\label{sec:A0}



\begin{lemma}
	\label{lem:c_0_not_complemented}
	For each linear operator $Q: \ell_\infty \to \ell_{\infty}$ such that $c_0\subseteq \ker Q$,
	there exists infinite subset $S \subset \mathbb{N}$ such that
	\begin{equation}
		\forall(x \in \ell_\infty : \supp x \subset S)[Qx = 0]
		.
	\end{equation}
\end{lemma}

\begin{proof}
	Let $\{S_i\}_{i \in I}$ be a family of subsets of $\mathbb{N}$
	such that conditions of Lemma~\ref{lem:uncountable_subsets_of_N_with_finite_intersections} are hold.
	Suppose the contrary:
	\begin{equation}
		\forall(i\in I)\exists(x_i \in \ell_\infty : \supp x_i \subset S_i)[Q(x_i) \neq 0]
		.
	\end{equation}
	The proof is completed with the same argument as Lemma~\ref{lem:c_0_not_complemented}.

\end{proof}





\begin{theorem}
	The subspace $A_0$ is not complemented in $\ell_\infty$.
\end{theorem}

\begin{proof}
	Suppose the contrary:
	there exists a continuous projection $P: \ell_\infty \to A_0$.
	Applying Lemma~\ref{lem:c_0_not_complemented} to $I-P$
	we can find infinite subset $S\subset\mathbb{N}$
	such that $\forall(x\in\ell_\infty : \supp x \subset S)[(I-P)x = 0]$.
	Let $S' \subset S$ be such that $\chi_S$ contains
	infinite quantities of both ones and zeros.
	But then $\alpha(P\chi_{S'}) = \alpha(\chi_{S'}) = 1$ and $P\chi_{S'} \notin A_0$,
	which contradicts that $P$ is a projection onto $A_0$.
\end{proof}



\section{Acknowledgements}
Author thanks Dr. Prof. E.M. Semenov and Dr. A.S. Usachev for encouragement and discussions.


\printbibliography

\end{document}
