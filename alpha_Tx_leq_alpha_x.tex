\documentclass[a4paper,12pt,openbib]{report}
\usepackage{amsmath}
\usepackage[utf8]{inputenc}
\usepackage[english,russian]{babel}
\usepackage{amsfonts}
\usepackage{amsfonts,amssymb}
\usepackage{amssymb}
\usepackage{latexsym}
\usepackage{euscript}
\usepackage{enumerate}
\usepackage{graphics}
\usepackage[dvips]{graphicx}
\usepackage{geometry}
\usepackage{wrapfig}

\geometry{verbose,a4paper,tmargin=1.75cm,bmargin=2.1cm,lmargin=1.75cm,rmargin=1.75cm}

\righthyphenmin=2


\begin{document}

\clubpenalty=10000
\widowpenalty=10000

Напомним, что на пространстве ограниченных последовательностей $\ell_\infty$
определяется $\alpha$--функция равенством
\begin{equation}
	\alpha(x) = \varlimsup_{i\to\infty}\max_{i < j \leqslant 2i} |x_i - x_j|.
\end{equation}

Однако, как выясняется, $\alpha$--функция не инвариантна относительно оператора сдвига
\begin{equation}
	T(x_1,x_2,x_3,...) = (x_2, x_3, ...).
\end{equation}

\paragraph{Пример.}
Пусть
\begin{equation}
	x_k = \begin{cases}
		(-1)^n, & \mbox{~если~} k = 2^n
		\\
		0 & \mbox{~иначе~}
	\end{cases}
\end{equation}

Вычислим $\alpha(x)$.
Заметим сначала, что из принадлежности $x_k\in\{-1,0,1\}$
немедленно следует, что $\alpha(x) \leq 2$.
Оценим теперь $\alpha(x)$ снизу:
\begin{multline}
	\alpha(x)
	=
	\varlimsup_{i\to\infty}\max_{i < j \leqslant 2i} |x_i - x_j|
	\geq
	\\\geq
	\mbox{(переход к частичному верхнему пределу по индексам специального вида $i=2^n$)}
	\geq
	\\\geq
	\varlimsup_{n\to\infty}\max_{2^n < j \leqslant 2^{n+1}} |x_{2^n} - x_j|
	=
	\varlimsup_{n\to\infty}\max_{2^n < j \leqslant 2^{n+1}} |(-1)^n - x_j|
	\geq
	\varlimsup_{n\to\infty} |(-1)^n - x_{2^{n+1}}|
	=
	\\=
	\varlimsup_{n\to\infty} |(-1)^n - (-1)^{n+1}|
	=
	2
\end{multline}

Итак, $\alpha(x) = 2$.
Вычислим теперь $\alpha(Tx)$:
\begin{multline}
	\alpha(Tx)
	=
	\varlimsup_{i\to\infty}~\max_{i < j \leqslant 2i} |(Tx)_i - (Tx)_j|
	=
	\varlimsup_{i\to\infty}~\max_{i < j \leqslant 2i} |x_{i+1} - x_{j+1}|
	=
	\\=
	(\mbox{замена}~k:=i+1, m:=j+1)
	=
	\varlimsup_{k\to\infty}~~\max_{k-1 < m-1 \leqslant 2k-2} |x_k - x_m|
	=
	\varlimsup_{k\to\infty}~~\max_{k < m \leqslant 2k-1} |x_k - x_m|
	=
	\\=
	\max\left\{
		\varlimsup_{k\to\infty, k  =   2^n}~~\max_{k < m \leqslant 2k-1} |x_k - x_m|
		,~~
		\varlimsup_{k\to\infty, k \neq 2^n}~~\max_{k < m \leqslant 2k-1} |x_k - x_m|
	\right\}
	=
	\\=
	\max\left\{
		\varlimsup_{n\to\infty}~~\max_{2^n < m \leqslant 2^{n+1}-1} |x_{2^n} - x_m|
		,~~
		\varlimsup_{k\to\infty, k \neq 2^n}~~\max_{k < m \leqslant 2k-1} |x_k - x_m|
	\right\}
	=
	\\=
	\max\left\{
		\varlimsup_{n\to\infty}~~\max_{2^n < m \leqslant 2^{n+1}-1} |(-1)^n - x_m|
		,~~
		\varlimsup_{k\to\infty, k \neq 2^n}~~\max_{k < m \leqslant 2k-1} |x_k - x_m|
	\right\}
	=
	\\=
	\max\left\{
		\varlimsup_{n\to\infty}~~\max_{2^n < m \leqslant 2^{n+1}-1} |(-1)^n - 0|
		,~~
		\varlimsup_{k\to\infty, k \neq 2^n}~~\max_{k < m \leqslant 2k-1} |x_k - x_m|
	\right\}
	=
	\\=
	\max\left\{
		1
		,~
		\varlimsup_{k\to\infty, k \neq 2^n}~~\max_{k < m \leqslant 2k-1} |x_k - x_m|
	\right\}
	=
	\\=
	\mbox{(если $k \neq 2^n$, то $x_k = 0$)}
	=
	\\=
	\max\left\{
		1
		,~
		\varlimsup_{k\to\infty, k \neq 2^n}~~\max_{k < m \leqslant 2k-1} |0 - x_m|
	\right\}
	=
	1
	.
\end{multline}
Таким образом, $\alpha(Tx) = 1 \neq 2 = \alpha(x)$,
что и требовалось показать.

Верна следующая
\paragraph{Теорема.}
Для любого $x \in \ell_\infty$ выполнено неравенство $\alpha(Tx)\leq \alpha(x)$.

\paragraph{Доказательство.}
\begin{multline}
	\alpha(Tx)
	=
	\varlimsup_{i\to\infty}~\max_{i < j \leqslant 2i} |(Tx)_i - (Tx)_j|
	=
	\varlimsup_{i\to\infty}~\max_{i < j \leqslant 2i} |x_{i+1} - x_{j+1}|
	=
	\\=
	(\mbox{замена}~k:=i+1, m:=j+1)
	=
	\\=
	\varlimsup_{k\to\infty}~~\max_{k-1 < m-1 \leqslant 2k-2} |x_k - x_m|
	=
	\varlimsup_{k\to\infty}~~\max_{k < m \leqslant 2k-1} |x_k - x_m|
	\leq
	\\ \leq
	\mbox{(переход к максимуму по большему множеству)}
	\leq
	\\ \leq
	\varlimsup_{k\to\infty}~~\max_{k < m \leqslant 2k} |x_k - x_m|
	=
	\alpha(x)
	.
\end{multline}

Более интересна, однако, следующая оценка.
\paragraph{Теорема.}
Для любого $n\in\mathbb{N}$
\begin{equation}
	\alpha(T^n x) \geq \frac{1}{2} \alpha(x)
	.
\end{equation}

\paragraph{Доказательство.}
Зафиксируем $n$.
Заметим, что
\begin{equation}
	\alpha(x) = \varlimsup_{i\to\infty} \alpha_i(x),
\end{equation}
где
\begin{equation}
	\alpha_i(x) = \max_{i < j \leqslant 2i} |x_i - x_j|.
\end{equation}

Рассмотрим $\alpha_i(x)$ при некотором фиксированном $i$, $i>2n$ (меньшие $i$ не влияют на верхний предел).
Если
\begin{equation}
	\max_{i < j \leqslant 2i} |x_i - x_j|
	=
	\max_{i < j \leqslant 2i-n} |x_i - x_j|
	,
\end{equation}
то
\begin{multline}\label{alpha_i(x)_leq_alpha_{i-n}(T_n x)}
	\alpha_i(x)
	=
	\max_{i < j \leqslant 2i} |x_i - x_j|
	=
	\max_{i < j \leqslant 2i-n} |x_i - x_j|
	=
	\\=
	\mbox{(замена $k=i-n$, $m=j-n$)}
	=
	\\=
	\max_{k+n < m+n \leqslant 2k+n} |x_{k+n} - x_{m+n}|
	=
	\max_{k < m \leqslant 2k} |(T^n x)_k - (T^n x)_m|
	=
	\alpha_{i-n}(T^n x)
	.
\end{multline}
Иначе
\begin{equation}
	\max_{i < j \leqslant 2i} |x_i - x_j|
	=
	\max_{2i-n < j \leqslant 2i} |x_i - x_j|
\end{equation}
и можно записать, что
\begin{multline}\label{alpha_i(x)_leq_alpha_{i-n}(T_n x) + alpha_{2i-2n}(T_n x)}
	\alpha_i(x)
	=
	\max_{2i-n < j \leqslant 2i} |x_i - x_j|
	=
	\max_{2i-n < j \leqslant 2i} |x_i - x_{2i-n} + x_{2i-n} - x_j|
	=
	\\=
	\max_{2i-n < j \leqslant 2i} \left( |x_i - x_{2i-n}| + |x_{2i-n} - x_j| \right)
	=
	|x_i - x_{2i-n}| + \max_{2i-n < j \leqslant 2i} |x_{2i-n} - x_j|
	=
	\\=
	|x_{i-n+n} - x_{2(i-n)+n}| + \max_{2i-n < j \leqslant 2i} |x_{2i-n} - x_j|
	=
	\\=
	|(T^n x)_{i-n} - (T^n x)_{2(i-n)}| + \max_{2i-n < j \leqslant 2i} |x_{2i-n} - x_j|
	\leq
	\\ \leq
	\alpha_{i-n}(T^n x) + \max_{2i-n < j \leqslant 2i} |x_{2i-n} - x_j|
	\leq
	\alpha_{i-n}(T^n x) + \max_{2i-n < j \leqslant 2i} |(T^n x)_{2i-2n} - (T^n x)_{j-n}|
	=
	\\=
	(\mbox{замена}~ m:=j-n ~)
	=
	\\=
	\alpha_{i-n}(T^n x) + \max_{2i-2n < m \leqslant 2i-n} |(T^n x)_{2i-2n} - (T^n x)_m|
	\leq
	\\ \leq
	\mbox{(т.к. $i>2n$, то $4i-4n > 2i+4n-4n = 2i > 2i-n$)}
	\leq
	\\ \leq
	\alpha_{i-n}(T^n x) + \max_{2i-2n < m \leqslant 4i-4n} |(T^n x)_{2i-2n} - (T^n x)_m|
	=
	\alpha_{i-n}(T^n x) + \alpha_{2i-2n}(T^n x)
	.
\end{multline}

Сравнивая \eqref{alpha_i(x)_leq_alpha_{i-n}(T_n x)} и \eqref{alpha_i(x)_leq_alpha_{i-n}(T_n x) + alpha_{2i-2n}(T_n x)},
делаем вывод, что
\begin{equation}
	\alpha_i(x) \leq \alpha_{i-n}(T^n x) + \alpha_{2i-2n}(T^n x)
	.
\end{equation}
Переходя к верхнему пределу, имеем
\begin{multline}
	\alpha(x)
	=
	\varlimsup_{i\to\infty} \alpha_i(x)
	\leq
	\varlimsup_{i\to\infty} \alpha_{i-n}(T^n x) + \alpha_{2i-2n}(T^n x)
	\leq
	\varlimsup_{i\to\infty} \alpha_{i-n}(T^n x) + \varlimsup_{i\to\infty} \alpha_{2i-2n}(T^n x)
	=
	\\=
	\varlimsup_{j=i-n, j\to\infty} \alpha_{j}(T^n x) + \varlimsup_{i\to\infty} \alpha_{2i-2n}(T^n x)
	=
	\alpha(T^n x) + \varlimsup_{i\to\infty} \alpha_{2i-2n}(T^n x)
	\leq
	\\ \leq
	\mbox{(верхний предел по индексам специального вида
	} \\ \mbox{
	заменим на верхний предел по всем индексам)}
	\leq
	\\ \leq
	\alpha(T^n x) + \varlimsup_{k\to\infty} \alpha_{k}(T^n x)
	=
	\alpha(T^n x) + \alpha(T^n x)
	=
	2 \alpha(T^n x)
	.
\end{multline}
Таким образом, $\alpha(T^n x) \geq \frac{1}{2} \alpha(x)$,
что и требовалось доказать.


\end{document}
