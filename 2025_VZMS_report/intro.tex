
\begin{frame}\frametitle{Пространство $\ell_\infty$}
	$\ell_\infty$~--- пространство всех ограниченных последовательностей
	$x=(x_1, x_2, ..., x_n, ...)$
	с~нормой
	$$
		\|x\|_{\ell_\infty} = \sup_{k\in\mathbb{N}} |x_k|
	$$
	{Свойства:}

	\begin{itemize}
		\item
			$\ell_\infty$~--- линейное пространство над полем $\mathbb{R}$
		\item
			$\ell_\infty$  несепарабельно
		\item
			$\ell_1 \subset \ell_2 \subset \dots \subset \ell_\infty$
		\item
			$\ell_1^* = \ell_\infty$
		\item
			$\ell_\infty^* \neq \ell_1$
	\end{itemize}

	\vfill
	~\\~\\
	Заголовки слайдов с результатами, полученными лично докладчиком, \underline{подчёркнуты}.
\end{frame}


\begin{frame}\frametitle{Обобщения понятия сходимости в $\ell_\infty$}
	%\hspace{1.468em}
	\begin{varwidth}[t]{\linewidth}
		\centering
		Верхний и нижний
		\\
		пределы:
		\\
		$\displaystyle \limsup_{n\to\infty} x_n$
		,~
		$\displaystyle \liminf_{n\to\infty} x_n$
		\\~\\
		\emph{
			не характеризуют
			\\
			распределение элементов
		}
	\end{varwidth}
	\hfill
	\begin{varwidth}[t]{\linewidth}
		\centering
		Сходимость в среднем
		\\
		(по Чезаро):
		$\displaystyle \lim_{n\to\infty} (Cx)_n$
		\\
		$\displaystyle (Cx)_n = \frac1n\sum_{i=1}^n x_i$
		\\~\\
		\emph{слишком универсальна}
	\end{varwidth}
	\hfill
	\begin{varwidth}[t]{\linewidth}
		\centering
		Банаховы пределы $B\in \mathfrak{B}$:
		\\
		т. Хана--Банаха к $\lim: c\to\mathbb R$
		\\
			$B \geqslant 0$
		\\
			$B\mathbbm{1} = 1$
		\\
			$B=BT$
		\\
		\emph{почти сходимость}
	\end{varwidth}
	\\~\\~\\
	\begin{varwidth}[t]{\linewidth}
		\centering
		\emph{Критерий Лоренца:}
		$\displaystyle
			x\in ac_r
			\Leftrightarrow
			\forall(B\in\mathfrak{B})[Bx = r]
			\Leftrightarrow
			\!
			\lim_{n\to\infty}  \sum_{k=m+1}^{m+n} \frac{x_k}n = r
		$
		равном.\,по $m\in\mathbb{N}$
	\end{varwidth}
	\\~\\
	\vspace{0.8em}
	\begin{varwidth}[t]{\linewidth}
		$\displaystyle ac = \mathop{\cup}\limits_{r\in\mathbb R} ac_r$ "--- пространство почти сходящихся последовательностей
	\end{varwidth}
\end{frame}


\begin{frame}\frametitle{Банаховы пределы}
	Банаховым пределом называется функционал $B\in \ell_\infty^*$ такой, что:
	\begin{enumerate}
		\item
			$B \geqslant 0$
		\item
			$B\mathbbm{1} = 1$
		\item
			$B=BT$
	\end{enumerate}
	Здесь $\mathbbm{1}=(1,1,1,1,1,...)$,
	$T$~--- оператор сдвига: $T(x_1, x_2, x_3, ...) = (x_2, x_3, ...)$.

	Простейшие свойства:
	\begin{itemize}
		\item
			$\|B\|_{\ell_\infty^*} = 1$
		\item
			$Bx = \lim_{n\to\infty} x_n$ для любого $x=(x_1, x_2, ...) \in c$,
			\\
			где $c$~--- множество сходящихся последовательностей.

			Таким образом,
			банахов предел~--- естественное обобщение понятия предела сходящейся последовательности
			на все ограниченные последовательности.
	\end{itemize}

	$\mathfrak{B}$~--- множество банаховых пределов.

	~\\~

	Анонсированы С. Мазуром~\cite{Mazur} в 1929 г.,
	доказательство приведено С. Банахом~\cite{banach2001theory_rus}.

\end{frame}

\longcomment{
\begin{frame}\frametitle{Свойства множества банаховых пределов $\mathfrak{B}$}
	\begin{enumerate}
		\item
			$\mathfrak{B}$~--- замкнутое выпуклое подмножество $ S_{\ell_\infty^*}$~---
			единичной сферы пространства $\ell_\infty^*$
		\item
			$d(\mathfrak{B}) = 2$
		\item
			$\mathfrak{B}$ слабо *-компактно
	\end{enumerate}
\end{frame}
}

\begin{frame}\frametitle{Теорема Лоренца~\cite{L}}
	Для заданного $r\in\mathbb{R}$ равенство $Bx=r$ выполнено для всех $B\in\mathfrak{B}$
	тогда и только тогда, когда
	\begin{equation*}
		\lim_{n\to\infty} \frac{1}{n} \sum_{k=m+1}^{m+n} x_k = r
	\end{equation*}
	сходится равномерно по всем $m\in\mathbb{N}$.

	Множество всех таких $x \in \ell_\infty$ обозначается $ac$.
\end{frame}

\begin{frame}\frametitle{Теорема Сачестона~\cite{S}}
	является уточнением теоремы Лоренца.
	Пусть
	\begin{equation*}
		q(x) = \lim_{n\to\infty} \inf_{m\in\mathbb{N}}  \frac{1}{n} \sum_{k=m+1}^{m+n} x_k,
		~~~~~~~~
		p(x) = \lim_{n\to\infty} \sup_{m\in\mathbb{N}}  \frac{1}{n} \sum_{k=m+1}^{m+n} x_k.
	\end{equation*}
	Тогда для любых $x\in \ell_\infty$ и $B\in\mathfrak{B}$
	\begin{equation}\label{Sucheston}
		q(x) \leqslant Bx \leqslant p(x)
	\end{equation}
	Неравенства (\ref{Sucheston}) точны:
	для данного $x$ для любого $r\in[q(x); p(x)]$ найдётся банахов предел
	$B\in\mathfrak{B}$ такой, что $Bx = r$.
\end{frame}

\begin{frame}\frametitle{Что такое почти сходимость?}
	\begin{varwidth}[t]{\linewidth}
		\hspace{3em}
		Пример $x\in ac_0$, но $\displaystyle0=\liminf_{n\to\infty}x_n < \limsup_{n\to\infty}x_n=1$:~~
		$$\displaystyle
			x_k=\begin{cases}
				1, & k=2^j,~j\in\mathbb N
				\\
				0 & \mbox{для прочих~} k
			\end{cases}
		$$
		$x=(0,1,0,1,\;0,0,0,1,\;\;0,0,0,0,\;0,0,0,1,\;\;0,0,0,0,\;0,0,0,0,\;\;0,0,0,0,\;0,0,0,1,\;\;0,0,0,0,\;...)$
	\end{varwidth}
	\\
	\vspace{2em}
	\begin{varwidth}[t]{\linewidth}
		Пример сходящейся по Чезаро к нулю последовательности $x\notin ac$:~~
		$$\displaystyle
			x_k=\begin{cases}
				1, & 2^j-j < k \leq 2^j,~j\in\mathbb N
				\\
				0 & \mbox{для прочих~} k
			\end{cases}
		$$
		$x=(0,1,1,1,\;0,1,1,1,\;\;0,0,0,0,\;1,1,1,1,$\\
		$\phantom{x=(}0,0,0,0,\;0,0,0,0,\;\;0,0,0,1,\;1,1,1,1,$\\
		$\phantom{x=(}0,0,0,0,\;0,0,0,0,\;\;0,0,0,0,\;0,0,0,0,$\\
		$\phantom{x=(}0,0,0,0,\;0,0,0,0,\;\;0,0,1,1,\;1,1,1,1,...)$
	\end{varwidth}
\end{frame}


