\begin{frame}\frametitle{\underline{Диадическая последовательность}}
	\begin{ddefinition}
		Последовательность $x\in\ell_\infty$
		называется \emph{диадической}, если существует такое $X=\{a;b\}$,
		что $x_k \in X$ для всех $k\in \mathbb N$.
	\end{ddefinition}

	Например, диадическими являются все последовательности из нулей и единиц.


	\begin{llemma}
		Пусть $x\in\ell_\infty$, $\|x\|\leq 1$.
		Тогда найдётся такая $h\in ac_0$, что $(x+h)_n = \pm 1$ для всех $n\in\mathbb N$.
	\end{llemma}

	\begin{ccorollary}
		Пусть $x\in\ell_\infty$.
		Тогда найдётся такая $h\in ac_0$, что для всех $n\in\mathbb N$
		\begin{equation*}
			(x+h)_n \in \{\inf_{n\in\mathbb N} x_n,\sup_{n\in\mathbb N} x_n\}
			.
		\end{equation*}
	\end{ccorollary}
	\vspace{-2em}
	\begin{ccorollary}
		Пусть $x\in\ell_\infty$.
		Тогда найдётся такая $h\in ac_0$, что $(x+h)_n =\pm \|x\|$ для всех $n\in\mathbb N$.
	\end{ccorollary}

	\vfill
\end{frame}




\begin{frame}\frametitle{Разделяющие множества}
	\label{page:p_q_linhulls}
	Множество $M\subset \ell_\infty$ называется разделяющим,
	если для любых двух различных банаховых пределов $B_1$ и $B_2$
	найдётся такой $x\in M$, что $B_1 x \ne B_2 x$.
	\\~\\
	Интересны <<маленькие>> разделяющие множества.
	\\~\\
	Например, $\Omega$ разделяющее~\cite{semenov2010characteristic}.
	\\~\\
	Этот результат можно получить непосредственно из теоремы о разложении.
\end{frame}





\begin{frame}\frametitle{\underline{Интервальный критерий}}
	%\begin{theorem}
		\label{thm:M_j_ac0_inf_lim}
		Пусть $n_i$~--- строго возрастающая последовательность натуральных чисел,
		\begin{equation}
			\label{eq:definition_M_j}
			M(j) = \liminf_{i\to\infty} \left(n_{i+j} - n_i\right),
		\end{equation}
		\begin{equation}
			x_k = \left\{\begin{array}{ll}
				1, & \mbox{~если~} k = n_i
				\\
				0  & \mbox{~иначе~}
			\end{array}\right.
		\end{equation}
		Тогда следующие условия эквивалентны:
		\\~\\
		(i)   $x \in ac_0$;
		\\~\\
		(ii)  $\lim\limits_{j \to \infty} \dfrac{M(j)}{j} = +\infty$;
		\\~\\
		(iii) $\inf\limits_{j \in \N}     \dfrac{M(j)}{j} = +\infty$.
	%\end{theorem}
\end{frame}



\begin{frame}\frametitle{\underline{Усиленная теорема Коннора}}
	Пусть на множестве $\Omega=\{0,1\}^\N$ задана вероятностная мера <<честной монетки>> $\mu$.
	Коннор доказал~\cite{connor1990almost}, что $\mu(\Omega\cap ac)=0$.

	Обобщим этот результат.

	\begin{ttheorem}
	%\label{thm:Connor_generalized}
	Мера множества $F=\{x\in\Omega : q(x) = 0 \wedge p(x)= 1\}$,
	где $p(x)$ и $q(x)$~--- верхний и нижний функционалы Сачестона соответственно,
	равна 1.
	\end{ttheorem}
\end{frame}

