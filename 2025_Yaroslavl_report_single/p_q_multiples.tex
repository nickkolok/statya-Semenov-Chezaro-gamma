\begin{frame}\frametitle{Сходимость по Чезаро}
	\label{page:p_q_multiples}
	Дальнейшее ослабление понятия сходимости~---
	\\
	сходимость по Чезаро (сходимость в среднем).
	\vfill
	Последовательность $\{x_n\}\in\ell_\infty$ сходится по Чезаро к $t$, если
	\begin{equation}
		\lim_{n\to\infty}\frac1{n}\sum_{i=1}^n x_i = t
	\end{equation}

	Напомним: последовательность $\{x_n\}\in\ell_\infty$ почти сходится к $t$, если

	\begin{equation}
			\lim_{n\to\infty}  \frac1{n}\sum_{i=m+1}^{m+n} x_i = t
			\qquad
			\mbox{равномерно по~~}
			m\in\mathbb{N}
	\end{equation}

	\vfill

	\begin{equation}
		\label{eq:generalization_of_limits}
		\liminf_{n\to\infty} x_n \leq q(x) \leq \liminf_{n\to\infty}\frac1{n}\sum_{i=1}^n x_i
		\leq
		%\\ \leq
		\limsup_{n\to\infty}\frac1{n}\sum_{i=1}^n x_i
		\leq p(x)
		\leq \limsup_{n\to\infty} x_n
	\end{equation}

\end{frame}

\begin{frame}\frametitle{Sets of multiples}

	Каждый $x\in \Omega$ отождествим с подмножеством множества натуральных чисел
	$\supp x \subset \N$.

	Вслед за~\cite{hall1992behrend} обозначим
	$
		\mathscr{M}A = \{ka: k\in\N, a\in A\}
		,
	$
	через $\chi F$ "--- характеристическую функцию множества $F$.

	Например,
	\begin{gather}
		\chi \mathscr{M}\{2\} = \chi \mathscr{M}\{2, 4\} = \chi \mathscr{M}\{2,4,8,16,...\}
		= (0,1,0,1,0,1,0,1,0,...),
	\\
		\chi \mathscr{M}\{3\} = \chi \mathscr{M}\{3,9,27,...\} = (0,0,1,\;0,0,1,\;0,0,1,\;0,0,1,\;0,0,1,\;...),
	\\
		\chi \mathscr{M}\{2,3\} = \chi \mathscr{M}\{2,3,6\} = (0,1,1,1,0,1,\;0,1,1,1,0,1,\;0,1,1,1,0,1,...).
	\end{gather}

	Как связана структура множества $A$
	и  обобщения верхнего и нижнего пределов на последовательности $\chi \mathscr{M}\!A$ ?
	В~\cite{davenport1936sequences,davenport1951sequences} доказано, что для любого
	$A=\{a_1,a_2,...\}\subset\N$
	\begin{equation}
		\liminf_{n\to\infty}\frac1{n}\sum_{i=1}^n (\chi\mathscr{M}A)_i =
		\lim_{j\to\infty}\lim_{n\to\infty}\frac1{n}\sum_{i=1}^n (\chi\mathscr{M}\{a_1,a_2,...,a_j\})_i
		.
	\end{equation}
	В работе~\cite[{\S 7}]{besicovitch1935density} построено такое множество $A\subset\N$, что
	\begin{equation}
		\liminf_{n\to\infty}\frac1{n}\sum_{i=1}^n (\chi\mathscr{M}A)_i \neq
		\limsup_{n\to\infty}\frac1{n}\sum_{i=1}^n (\chi\mathscr{M}A)_i
		.
	\end{equation}

\end{frame}


\begin{frame}\frametitle{\underline{Верхний функционал Сачестона}}


	\begin{llemma}
		Пусть $\{p_1, ..., p_k\} \subset \N$,
		\begin{equation}
			x_k = \begin{cases}
				1, &\mbox{~если~} k = p_1^{j_1}\cdot p_2^{j_2}\cdot ... \cdot p_k^{j_k} \mbox{~для некоторых~} j_1,...,j_k\in\N,
				\\
				0  &\mbox{~иначе}.
			\end{cases}
		\end{equation}
		Тогда $x\in ac_0$.
	\end{llemma}
\end{frame}


\begin{frame}\frametitle{\underline{Верхний функционал Сачестона}}

	\begin{ddefinition}
		Будем говорить, что множество $A\subset\N$ обладает $P$-свойством,
		если для любого $n\in\N$ найдётся набор попарно взаимно простых чисел
		\begin{equation}
			\{a_{n,1}, a_{n,2}, ..., a_{n,n}  \} \subset A
			.
		\end{equation}
	\end{ddefinition}

	\begin{ttheorem}
		Пусть $A\subset \N\setminus\{1\}$.
		Тогда следующие условия эквивалентны:
		\begin{enumerate}%[label=(\roman*)]
			\item
				$A$ обладает $P$-свойством
			\item
				В $A$ существует бесконечное подмножество попарно взаимно простых чисел
			\item
				$p(\chi\mathscr{M}A)=1$.
		\end{enumerate}
	\end{ttheorem}

\end{frame}


\begin{frame}\frametitle{\underline{Нижний функционал Сачестона}}


	\begin{ttheorem}
		Пусть $A = \{a_1, a_2, ..., a_n,...\}$ "--- бесконечное множество попарно взаимно простых чисел
		и $a_{n+1}>a_1\cdot...\cdot a_n$.
		Тогда
		\begin{equation}
			q(\chi\mathscr{M}A) = 1-\prod_{j=1}^\infty \left(1-\frac{1}{a_j}\right)
			.
		\end{equation}
	\end{ttheorem}

	\begin{llemma}
		Пусть $\varepsilon \in  (0; 1{]}$.
		Существует бесконечное множество попарно непересекающихся подмножеств простых чисел
		$A_i$ такое, что $q(\chi\mathscr{M}A_i)\geq\varepsilon$ для любого $i\in\N$.
	\end{llemma}
\end{frame}
