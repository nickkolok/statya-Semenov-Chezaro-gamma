\begin{frame}\frametitle{Система Хаара}
	\label{page:Haar}


Обозначим через $F$ множество индексов $(n,k)$ таких, что
$k=0,1$ для $n=0$ и $1\leq k \leq 2^n$ для $n\geq 1$, а для любых
$(n,k)\in F$ через $\Delta_n^k$ обозначим интервал $
(\frac{k-1}{2^n},\frac{k}{2^n})$.

\emph{Система Хаара} ~--- последовательность функций $\varphi_{0,0}=1$,
$$\varphi_{n,k} (t)=\begin{cases} 2^\frac{n}2, &t\in \Delta_{n+1}^{2k-1}\\
-2^\frac{n}2, &t\in \Delta_{n+1}^{2k}\\0,  &\text{для} \,\, t \,\,
\text{из остальных} \, \Delta_i^j,
\end{cases}$$
Значения в концах отрезка $[0,1]$ и в
точках разрыва выбираются из условий
\begin{equation}
	\varphi_{n,k} (0)= \!\lim_{\delta \to 0+}\!\varphi_{n,k}
	(\delta);
	\hfill
	\varphi_{n,k} (1)= \!\lim_{\delta \to
	0+}\!\varphi_{n,k} (1-\delta);
	\hfill
	\varphi_{n,k} (t)= \lim_{\delta \to 0}\frac{\varphi_{n,k}
	(t+\delta)+\varphi_{n,k} (t-\delta)}2
\end{equation}


$f_{n,k}(x)$ ~--- коэффициенты Фурье по
системе Хаара функции $x\in~L_1 [0,1]$.

Биекция $F \leftrightarrow \N$ задаётся формулой
$m=2^n+k$, где
$(n,k)\in F$.

\vfill

Система Хаара не является равномерно ограниченной.

\end{frame}

\begin{frame}\frametitle{Пространства $L_{p,\infty}$}

	Пусть $h(t)$ "--- некоторая измеримая на $[0;1]$ функция.
	Тогда её функция распределения:
	\begin{equation}
		\eta_h(\tau) = \mes \{t : h(t) > \tau \}
		.
	\end{equation}

	Невозрастающая перестановка $h^* : [0;1] \to \R$:
	\begin{equation}
		h^*(t) = \inf\{\tau: \eta_{|h|}(\tau) \leq t \}.
	\end{equation}


	$L_{p,\infty}\quad(1<p<\infty)$ "--- пространство
	функций с нормой
	\begin{equation}
		\| x \|_{p,\infty}=\sup\limits_{0\leq \tau \leq
		1} \tau^{\frac{1}{p}-1}\int_0^\tau  x^*(t)dt =
		%\\=
		%\| x \|_{p,\infty}=
		\sup_{e\subset[0,1], \ \mes\, e > 0}
		\{(\mes\, e)^{\frac{1}{p}-1}\int_e \mid x(t)\mid dt\}<\infty
		.
	\end{equation}


\end{frame}

\begin{frame}\frametitle{Коэффициенты Фурье-Хаара (Усачев, 2009)}


	\begin{ttheorem}[Мерсера]
		Коэффициенты Фурье любой суммируемой функции
		по равномерно ограниченной ортонормированной системе
		стремятся к нулю.
	\end{ttheorem}

	\vfill

	Система Хаара не является равномерно ограниченной.

	Существует функция $h\in L_p$, $1<p<2$, коэффициенты Фурье-Хаара
	которой не стремятся к нулю.


	\begin{ttheorem}
		Если $1<p<\infty$, $h\in
		L_{p,\infty}$, то $\left\{m^{\frac{1}2-\frac{1}p}\,
		f_m(h)\right\}_{m=1}^\infty \in ac_0$.
	\end{ttheorem}

	\begin{ccorollary}
		Если $h\in L_{2,\infty}$, то
		$\left\{f_m(h)\right\}_{m=1}^\infty\in ac_0$.
	\end{ccorollary}

	\vfill

	Это аналог теоремы Мерсера для
	пространства $L_{2,\infty}$ и системы Хаара.

\end{frame}
