\documentclass[12pt]{article}
\usepackage[T1,T2A]{fontenc}
\usepackage[utf8]{inputenc}
\usepackage[croatian,german,french,english,russian]{babel}
\usepackage{color}

\usepackage{amssymb,latexsym,amsmath, amsfonts}
\usepackage{amsthm,indentfirst}

\renewcommand{\baselinestretch}{1.3}
 \vbadness10000\hbadness10000
 \textwidth=167mm \textheight=230mm
\hoffset=-19mm \voffset=-20mm

\newcommand{\emm}{\mathrm{1\hspace{-0.3em} I}}

\newtheorem{thm}{Теорема}
\newtheorem{lem}[thm]{Лемма}
\newtheorem{prop}[thm]{Утверждение}
\newtheorem{cor}[thm]{Следствие}
\newtheorem{dfn}[thm]{Определение}

\def\cC{{\mathfrak{C}}}
\def\N{{\mathbb{N}}}
\def\Z{{\mathbb{Z}}}
\def\Rl{{\mathbb{R}}}
\def\Cx{{\mathbb{C}}}
\def\sH{{\mathcal{H}}}
\def\tr{{\mathrm{Tr}\,}}
\def\aM{{\mathcal{M}}}
\def\id{{\mathbb{1}}}
\def\sgn{{\mathrm{sgn}\,}}
\def\Re{{\mathrm{Re}\,}}
\def\mL{{\mathcal{L}}}

\def\A{{\mathfrak{A}}}
%\def\sH{{\mathbb{H}}}
\def\W{{{W}}}
\def\Pl{{{P}}}

\def\B{{\mathfrak{B}}}
\def\ext{{\mathrm{ext}\,}}

\usepackage[backend=biber,language=autobib,autolang=other,clearlang=true,style=gost-numeric,sorting=none]{biblatex}
\addbibresource{../bib/general_monographies.bib}
\addbibresource{../bib/ext.bib}
\addbibresource{../bib/my.bib}
\addbibresource{../bib/Semenov.bib}
\addbibresource{../bib/Bibliography_from_Usachev.bib}
\addbibresource{../bib/classic.bib}

\input{../bib/ext.hyphens.bib}




%Only referenced equations are numbered
\let\etoolboxforlistloop\forlistloop % save the good meaning of \forlistloop
\usepackage{mathtools}
\mathtoolsset{showonlyrefs}
\let\forlistloop\etoolboxforlistloop % restore the good meaning of \forlistloop

% https://tex.stackexchange.com/questions/220268/biblatex-and-autonum-dont-work-together

%\mathtoolsset{showonlyrefs=false}
% (write an equation/multline to be force-numbered here)
%\mathtoolsset{showonlyrefs=true}


\DeclareMathOperator{\mes}{mes}

\begin{document}

\begin{center}
{
	\Large Банаховы пределы: экстремальные свойства, инвариантность и теорема Фубини
	\footnote{
		Исследование было поддержано грантом Российского научного фонда (проект № 19-11-00197).

	}
}

Н. Н. Авдеев, Е. М. Семенов, А. С. Усачев
\end{center}

\begin{abstract}
Банаховым пределом на пространстве всех ограниченных вещественнозначных последовательностей называется положительный нормированный линейный функционал, инвариантный относительно сдвига. В работе изучаются такие свойства банаховых пределов, как мультипликативность и справедливость теоремы Фубини. Также изучается подмножество банаховых пределов, инвариантных относительно операторов растяжения.
\end{abstract}


\section{Введение}


Через $l_\infty$ обозначается множество ограниченных
последовательностей $x=(x_1,x_2,\ldots)$ с нормой
\[
 \|x\|_{l_\infty}=\sup_{n\in \mathbb N}|x_n|,
\]
и обычной полуупорядоченностью, где ${\mathbb N}$~--- множество натуральных чисел;
через $c$ и $c_0$ "--- множества сходящихся и сходящихся к нулю последовательностей соответственно.

\begin{dfn}
Линейный функционал $\gamma\in l_\infty^*$ называется обобщённым пределом,
если
\begin{enumerate}
    \item $\gamma\geqslant0$, т.~е. $\gamma x \geqslant 0$ для $x \geqslant 0$,
    \item $\gamma\emm=1$, где $\emm =(1,1,\ldots)$,
\end{enumerate}
\end{dfn}

\begin{dfn}
	Обобщённый предел $B$ называется банаховым пределом,
	если $B(Tx)=B(x)$ для всех $x\in l_\infty$, где $T$~---
    оператор сдвига, т.~е. $T(x_1,x_2,\ldots)=(x_2,x_3,\ldots)$.
\end{dfn}

Существование банаховых пределов было анонсированно С. Мазуром \cite{Mazur} и позднее доказано в книге С.~Банаха~\cite{banach2001theory_rus}.
Банаховы пределы зарекомендовали себя как удобный и полезный инструмент в различных областях математики, таких как некоммутативная геометрия и анализ~\cite{CS}, теория операторов \cite{Sem_Sht} и гармонический анализ \cite{Ast_Sem, SU, U2}. Применение банаховых пределов в этих и других областях подробно описано в недавнем обзоре \cite{SSU_survey}.

Из определения сразу вытекает, что
$$\liminf_{n\to\infty}x_n \leqslant Bx \leqslant \limsup_{n\to\infty}x_n,$$
для всех $x\in l_\infty$ и, в частности,
 $\displaystyle
Bx=\lim_{n\to\infty}x_n$ для любой сходящейся
последовательности $x\in l_\infty$ и $\|B\|_{l_\infty^*}=1$. Множество всех банаховых пределов обозначим
 через $\mathfrak{B}$. Ясно, что $\mathfrak{B}$ есть замкнутое выпуклое
множество на единичной сфере пространства $l_\infty^*$. Следовательно, по теореме Крейна--Мильмана $\B$ совпадает с замыканием выпуклой оболочки множества $\ext \B$ своих крайних точек в слабой$^*$ топологии, т.е. $\B={\overline{\rm conv}}^{w^*} \ext \B$.

В статье~\cite{L} Г.~Г.~Лоренц ввёл понятие почти сходящихся (к числу $a\in \mathbb R$) последовательностей, т.е. таких последовательностей $x\in l_\infty$, что $Bx=a$ для любого банахова предела $B$.

Множество всех почти сходящихся последовательностей обозначим через $ac$,
а множество всех последовательностей, почти сходящихся к нулю, "--- через $ac_0$.

Следующий критерий почти сходимости был доказан в работе~\cite{L}.
\begin{thm}\label{lorentz}
Последовательность $x\in l_\infty$ почти сходится к числу $a\in \mathbb R$ тогда и только тогда, когда
\begin{equation}
	\label{eq:thm_Lorentz}
	\lim_{n\to\infty}\frac1n\sum_{k=m+1}^{m+n}x_k=a
\end{equation}
равномерно по $m\in{\mathbb N}$.
\end{thm}

Результат Лоренца был усилен Л.~Сачестоном~\cite{S} следующим образом.
\begin{thm}\label{sucheston}
Для любого $x\in l_\infty$
$$\mathfrak B (x):=\{Bx: B\in \mathfrak B\}=[q(x),p(x)],$$
где
$$
 q(x)=\lim_{n\to\infty}\inf_{m\in\mathbb N}
  \frac1n \sum_{k=m+1}^{m+n} x_k \ ,
\qquad
 p(x)=\lim_{n\to\infty}\sup_{m\in\mathbb N}
  \frac1n \sum_{k=m+1}^{m+n} x_k.
$$
\end{thm}

Хорошо известно, что банаховы пределы не являются мультипликативными функционалами.
К примеру, пусть $x=(1,0,1,0,1,0,...)$.
Тогда по~\eqref{eq:thm_Lorentz} для любого банахова предела $B\in\mathfrak{B}$ выполнено
$Bx = BTx = 1/2$.
С другой стороны, $B(x\cdot Tx) = B(0,0,0,0...) = 0 \neq 1/4 = Bx \cdot BTx$
(в первом случае точка обозначает покоординатное умножение).

В работе \cite{Luxemburg} У. Люксембург ввёл в рассмотрение стабилизатор пространства $ac_0$, обозначаемый ${\rm St}(ac_0)$,
на котором банаховы пределы мультипликативны.
Это множество и его обобщения в дальнейшем изучались в работах \cite{Alekhno, SSU2, ASSU4}.
Дальнейшему изучению стабилизатора пространства $ac_0$ посвящён  \S\ref{sec:W} данной работы.
В частности, мы доказываем, что множество $W$,
которое является пересечением стабилизатора с множеством $\{0,1\}^\N \setminus c$
всех несходящихся последовательностей из нулей и единиц, "--- максимальное (по включению) подмножество $\{0,1\}^\N \setminus c$,
на котором банаховы пределы мультипликативны.

Особый интерес (мотивированный теорией сингулярных следов) представляет изучение банаховых пределов с дополнительными свойствами инвариантности.
В недавних работах изучалась связь между банаховыми пределами, инвариантными относительно операторов Чезаро и операторов растяжения (см. \S\ref{sec:inv}).
Так, в работе \cite{SSUZ2} было показано, что множество $\mathfrak{B}(C)$ банаховых пределов, инвариантных относительно оператора Чезаро $C$, вложено в пересечение множеств $\mathfrak{B}(\sigma_m)$ банаховых пределов, инвариантных относительно операторов растяжения $\sigma_m$.
В работе \cite{SSUZ3} показано, что это вложение строгое.
Дальнейшему изучению банаховых пределов, инвариантных относительно операторов растяжения, посвящён \S\ref{sec:inv} данной работы.

В \S\ref{sec:Fubini} мы рассматриваем аналоги теоремы Фубини для банаховых пределов.


Некоторые результаты \S \ref{sec:W} данной работы были анонсированы в~\cite{AvSU}.




%\section{Предварительные результаты}



\section{Подмножество $\ell_\infty$ с экстремальными свойствами}\label{sec:W}

Пусть $\{n_k\}_{k=1}^\infty$ "--- строго возрастающая  последовательность натуральных чисел и
\begin{equation}\label{e}
e =\bigcup\limits_{k=1}^\infty \lbrack n_{2k-1},n_{2k}).
\end{equation}

Обозначим через $W$ множество всех последовательностей $\chi_e$, где множество  $e \subset \N$ указанного вида
и выполнено условие
\begin{equation}\label{growth_cond}
    \lim\limits_{j\to\infty}\frac{n_{k+j}-n_k}j=\infty
\end{equation}
равномерно по $k\in \mathbb N$.

Ради простоты обозначений мы будем отождествлять $e \subset \N$ и характеристическую функцию $\chi_e$. Множество $W$ впервые появилось в~\cite[\S 5]{SSU2} естественным образом. Для $B\in \B$ и $e\subset \N$ таких, что $B\chi_e\neq 0$, определим на $\ell_\infty$ функционал
$$B_e(x)=\frac{B(x\chi_e)}{B\chi_e}.$$
В \cite[Т. 25]{SSU2} доказано, что функционал $B_e$ принадлежит $\B$ для любого $B\in \B$, удовлетворяющему условию $B\chi_e\neq 0$, тогда и только тогда, когда $e\in W$. Оказалось, что множество $W$ играет важную роль и в других задачах.

%Results in this section are motivated by the following question:
%Let $F$ be the set of all $x\in \{0,1\}^\N$ such that $Bx=0$ or $1$ for all $B\in {\rm ext}\mathfrak B$. Does there exist such $F$ with $mes F>0$?
%The results below answer this question in negative.

%The next two simple results show that the set $W$, introduced in~\cite{SSU2}, is maximal by inclusion from the sets $F$ above.

\begin{thm}\label{thm1}
Пусть $e\subset \N$ имеет вид \eqref{e}. Следующие условия эквивалентны

(i) $B\chi_e=0$ или $1$ для любого $B\in \ext\B$;

(ii) $B(\chi_e x)= B\chi_e \cdot Bx$ для любых $x\in \ell_\infty$ и $B\in \ext\B$;

(iii) $\chi_e\in W$.
\end{thm}

\begin{proof}
(i) $\Longrightarrow$ (ii). Пусть $B\in \ext\B$. Рассмотрим 2 возможных случая.

1. Пусть $B\chi_e=0$. Так как
$$-\|x\|_{\ell_\infty}\chi_e \leqslant x \chi_e \leqslant \|x\|_{\ell_\infty}\chi_e,$$
то
$$0=B(-\|x\|_{\ell_\infty}\chi_e) \leqslant B(x \chi_e) \leqslant B(\|x\|_{\ell_\infty}\chi_e)=0$$
и, следовательно, $B(\chi_e x)= B\chi_e \cdot Bx$.

2. Пусть $B\chi_e=1$. Тогда $B\chi_{ce}=0$, где $ce=\N\setminus e$. Применяя только что доказанное утверждение, получаем $B(x\chi_{ce})=0$ и
$$B(\chi_e x)= Bx - B(x\chi_{ce})= Bx,$$
что и доказывает равенство $B(\chi_e x)= B\chi_e \cdot Bx$.


(ii) $\Longrightarrow$ (iii). Е. А. Алехно \cite[Утверждение 2.3]{Alekhno} доказал, что (ii) влечёт $\chi_e \in {\rm St}(ac_0)$. Отсюда, как показано в  \cite[Лемма 35]{SSU2}, вытекает $\chi_e\in W$.

(iii) $\Longrightarrow$ (i). Эта импликация доказана в \cite[Следствие 29]{SSU2}.
\end{proof}

Таким образом, $W$ (объединённое с множеством последовательностей, стабилизирующихся на нуле или на единице) есть максимальное по включению подмножество $\{0,1\}^\N$, на котором каждый $B\in {\rm ext}\mathfrak B$ принимает только крайние значения: $0$ или $1$.

Каждой последовательности $(x_1, x_2, \dots)\in \{0,1\}^\N$ можно поставить в соответствие число
\begin{equation}\label{eq:bijection_omega_0_1}
	\sum_{k=1}^\infty 2^{-k} x_k \in [0,1]
	.
\end{equation}
С точностью до счётного множества это соответствие взаимно-однозначно и определяет на множестве $\{0,1\}^\N$ меру, которую мы, как и меру Лебега на $[0,1]$, обозначаем символом $\mes$. Классический результат Э. Бореля утверждает, что почти все последовательности из $\{0,1\}^\N$ сходятся по Чезаро к $1/2$. В \cite{SS} доказано следующее утверждение:
$$ \mes \, \{x\in \{0,1\}^\N: q(x)=0, \ p(x)=1\}=1.$$ Отсюда сразу вытекает утверждение \cite{Connor} о том, что
\begin{equation}\label{ac_meas}
\mes \, \{x\in \{0,1\}^\N: x\in ac\}=0.
\end{equation}

Нашей ближайшей целью является нахождение меры множества $W$. С этой целью рассмотрим следующее преобразование множества $\{0,1\}^\N$ в себя:
$$(Qx)_1=x_1, \ (Qx)_k=|x_k-x_{k-1}|, \ k\geqslant 2.$$
Ясно, что $Q$ "--- биекция.
Заметим, что $Q$ не совпадает с преобразованием взятия границы, введённым в \cite{keller1992invariant}.

\begin{dfn}
	Пусть $k\in\mathbb{N}$.
	Двоичным отрезком будем называть множество последовательностей из $\{0,1\}^\N$,
	в котором первые $k$ координат зафиксированы, а остальные выбираются произвольно,
	и соответствующий этому множеству по формуле~\eqref{eq:bijection_omega_0_1} полуинтервал длины $2^{-k}$.
\end{dfn}



\begin{lem}\label{lem:Q_Borel}
	Отображение $Q$ сохраняет меру любого борелевского множества $A\subset \{0,1\}^\N$.
\end{lem}

\begin{proof}
	Борелевская $\sigma$"=алгебра на отрезке порождается полуинтервалами;
	в свою очередь, любой полуинтервал может быть представлен в виде не более чем счётного объединения двоичных отрезков.
	Очевидно, что биекция $Q$ сохраняет меру любого двоичного отрезка.
	Следовательно, по теореме Каратеодори~\cite[теорема 1.53]{klenke2013probability}, для любого борелевского множества $A\subset [0,1]$ верно
	$\mes \, QA = \mes \, A$.
\end{proof}

\begin{lem}\label{lem:Q_Lebesgue}
	Отображение $Q$ сохраняет меру любого измеримого по Лебегу множества $A\subset \{0,1\}^\N$.
\end{lem}

\begin{proof}
	Лебеговская $\sigma$"=алгебра на отрезке порождается борелевской $\sigma$"=алгеброй,
	пополненной всевозможными подмножествами борелевских множеств меры нуль
	(таким подмножествам также приписывается мера нуль)~\cite[пример 1.71]{klenke2013probability}.
	В силу леммы~\ref{lem:Q_Borel} и теоремы Каратеодори нам достаточно показать,
	что для любого $A\subset[0,1]$, $\mes A = 0$ выполнено $\mes QA = 0$.

	Найдём такое борелевское множество $D$, что $\mes D = 0$ и $D\supset A$.
	Тогда в силу биективности $Q$ имеем $QA \subset QD$, а в силу леммы~\ref{lem:Q_Borel} выполнено $\mes QD = \mes D = 0$.
	Таким образом, действительно $\mes QA = 0$.
\end{proof}


\begin{thm}\label{thm9}
$\mes \, W=0$.
\end{thm}

\begin{proof}
В силу леммы~\ref{lem:Q_Lebesgue} достаточно показать, что $\mes \, QW=0$. Если $e\subset W$ имеет вид \eqref{e}, то
$$(Qe)_m=\begin{cases}
1, m=n_j, j\in\N, j>1,\\
0, n_j<m<n_{j+1}.
\end{cases}$$

Это означает, что $Qe$ есть характеристическая функция множества $\{n_j: j\in \N, j>1\}$. Согласно \cite[Лемма 1]{Avdeev2019}, характеристическая функция множества $\{n_j: j\in \N, j>1\}$ принадлежит пространству $ac_0$ тогда и только тогда, когда
\begin{equation}\label{2.4}
    \lim\limits_{j\to\infty}\liminf_{k\to\infty}\frac{n_{k+j}-n_k}j=\infty.
\end{equation}

Т.к. условие~\eqref{growth_cond} влечёт~\eqref{2.4}, то $QW \subset \{0,1\}^\N \cap ac_0$. Используя \eqref{ac_meas}, получаем
$$0\leqslant \mes \, W= \mes \, QW\leqslant \mes \, \{x\in \{0,1\}^\N: x\in ac_0\}\leqslant \mes \, \{x\in \{0,1\}^\N: x\in ac\}=0.$$
Отсюда, $\mes \, W=0$.
\end{proof}

Доказанное в теореме \ref{thm9} включение $QW \subset \{0,1\}^\N \cap ac_0$ является собственным. Действительно, для $x=(1, 0, 0, \dots)$ получаем $Qx=(1,1, 0, 0, \dots)\in ac_0$. Однако, $x\notin W$. Отметим и другой интересный факт. Пусть $x=(1,0,1,0,\dots)$. Тогда $Qx=\emm$ и $Q^2x=(1,0,0, \dots)\in ac_0$. Таким образом, $Q^{-2}ac_0$ не вложено в $ac_0$.


Заметим, что предел в левой части формулы \eqref{2.4} существует для любой строго возрастающей последовательности натуральных чисел $n_i$. Действительно,  рассмотрим последовательность
\begin{equation*}
\label{eq:definition_M_j}
M(j) = \liminf_{i\to\infty} (n_{i+j} - n_i),
\end{equation*}
введённую в работе \cite{Avdeev2019}. Легко видеть, что последовательность $\{M(j)\}_{j=1}^\infty$ вогнута. Однако, используя результат из работы \cite{Fekete} (см. также \cite[I, Задача 98]{polia1978zadachi}) получаем, что в таком случае предел выражения $M(j)/j$ существует и справедливо равенство
$$\lim_{j\to\infty}\frac{M(j)}{j} =\sup_{j\in\N}\frac{M(j)}{j}.$$


%С множеством $W$ тесно связано множество
%$$R:= \left\{t\in (0,1) : t=\sum_{k=1}^\infty x_k 2^{-k}, \ (x_1, x_2,...)\in W\right\}.$$
%By $t_x$ we denote an element of $R$ corresponding to a sequence $x\in W$.

На множестве всех последовательностей из нулей и единиц определим метрику
$$
	\rho(x,y)=
	\begin{cases}
		0, & \mbox{~если~} x=y, \\
		2^{-k}, k = \min\{n:x_n \neq y_n\} - 1, & \mbox{~если~} x\neq y. %x,y\in\{0,1\}^\N, \\
	\end{cases}
$$
Как показано в \cite[Утверждение 2.1.8]{Edgar}, пара $(\{0,1\}^\N, \rho)$ действительно является метрическим пространством.
Нетрудно показать, что пространство $(\{0,1\}^\N, \rho)$ полно.


\begin{thm}
1. Множество $W$ не содержит нетривиальных шаров из $\{0,1\}^\N$.

2. Множество $W$ и его дополнение плотны в $\{0,1\}^\N$.
\end{thm}

\begin{proof}
1. Рассмотрим произвольный элемент
$$x=\sum_{k=1}^\infty \chi_{[n_{2k-1},n_{2k})}$$
множества $W$.

Для любого $m\in \N$ найдём такое $k_0\in\N$, что $n_{2k_0}\geqslant m$ и положим
$$y=\sum_{k=k_0}^\infty \chi_{[n_{2k},n_{2k+1})}.$$

Имеем $(x+y)_i=1$ для любого $i\geqslant m$. Таким образом, $x+y\notin W$.

С другой стороны, по крайней мере, первые $m$ координат у последовательностей $x$ и $x+y$ совпадают. Следовательно, $\rho(x,x+y)\leqslant2^{-m}$, т.е. сколь угодно малый шар с центром в $x\in W$ содержит элементы, не принадлежащие $W$, что и доказывает утверждение.

2. Произвольная последовательность $x\in\{0,1\}^\N$ может быть записана в виде
$$x=\sum_{k=1}^\infty \chi_{[n_{2k-1},n_{2k})},$$
где $\{n_k\}_{k=1}^\infty$ "--- некоторая строго возрастающая последовательность.

Зафиксируем возрастающую последовательность $\{m_k\}_{k=1}^\infty$, удовлетворяющую условию \eqref{growth_cond}. Для любого $j\in \N$ рассмотрим последовательность
$$(m_j)_k:=
\begin{cases}
n_k, \ k=1,\dots 2j,\\
m_{k-2j}, \ k\geqslant 2j+1
\end{cases}$$
и определим
$$y_j=\sum_{n=1}^\infty \chi_{[(m_j)_{2k-1},(m_j)_{2k})}.$$

Поскольку последовательности $m_j$ отличаются от последовательности $m$ лишь конечным числом членов, добавленных в начале, то для любого $j\in \N$ последовательность $m_j$ также удовлетворяет условию \eqref{growth_cond}. Следовательно, $y_j\in W$.

Т.к. у последовательностей $x$ и $y_j$ по крайней мере первые $n_{2j}-1$ координат совпадают, то
$$\rho(x,y_j)\leqslant2^{-(n_{2j}-1)}\leqslant2^{1-2j}.$$
для любого $j\in \N$. Полученное неравенство доказывает, что множество $W$ плотно в $\{0,1\}^\N$.


Поскольку множество $W$ не содержит нетривиальных шаров из $\{0,1\}^\N$, то его дополнение плотно в $\{0,1\}^\N$.
\end{proof}

На множестве $\{0,1\}^\N$ стандартным образом определяется размерность Хаусдорфа (см. например \cite[Секция 6]{Edgar}). Мы напомним это определение для удобства читателя. Для непустого подмножества $F\subset \mathbb R^n$ и $s > 0$ определим $s$-мерную меру Хаусдорфа множества $F$ следующим образом:
$$\mathcal H^s(F) := \lim_{\delta\to0} \inf \left\{\sum_{i=1}^\infty \left({\rm diam} \ U_i \right)^s \ : \ F\subset \bigcup_{i=1}^\infty  U_i, \  0\leqslant {\rm diam} \ U_i \leqslant \delta \right\}.$$

Размерность Хаусдорфа множества  $F\subset \mathbb R^n$ определяется по формуле:
$${\rm dim}_H F := \inf\{ s > 0 \ : \ \mathcal H^s(F)=0\}.$$



 Мы приведём определение самоподобных подмножеств множества $\{0,1\}^\N$ (см., например, \cite{falconer1997techniques}).

\begin{dfn}
Множество $E\subset\{0,1\}^\N$ называется самоподобным, если существуют $m\in\N$, $m\geqslant2$, $0< r_1, \dots, r_m<1$ и функции $f_j : \{0,1\}^\N \to \{0,1\}^\N$, $j=1,\dots, m$ такие, что
$$\rho(f_j(x), f_j(y)) = r_j \rho(x,y), \ \forall \ x,y \in \{0,1\}^\N, \ j=1,\dots, m$$
и $E=\bigcup_{j=1}^m f_j(E).$
\end{dfn}


\begin{prop}
Множество $W$ самоподобно, и его размерность Хаусдорфа равна $1$.
\end{prop}

\begin{proof}
Для $j=1,2$ определим функции $f_j : \{0,1\}^\N \to \{0,1\}^\N$ следующим образом:
$$f_1(x_1, x_2, \dots)=(0, x_1, x_2, \dots), \quad f_2(x_1, x_2, \dots)=(1, x_1, x_2, \dots).$$

Сначала мы докажем, что $W=f_1(W)\cup f_2(W).$

Вложение $f_1(W)\cup f_2(W) \subset W$ следует из определения множества $W$.

Пусть
$$x=\sum_{k=1}^\infty \chi_{[n_{2k-1},n_{2k})}\in W.$$
 Если $x_1=0$ (т.е. $n_1>1$), то положим
 $$y=\sum_{k=1}^\infty \chi_{[n_{2k-1}-1,n_{2k}-1)}.$$
 Ясно, что $y\in W$ и $x=f_1(y)$.

 Если $x_1=1$ (т.е. $n_1=1$), рассмотрим два случая.

 (i) Если $x_2=1$, положим $m_1=1$, $m_k=n_k-1$ для $k\geqslant2$ и
 $$y=\sum_{k=1}^\infty \chi_{[m_{2k-1},m_{2k})}.$$
 Тогда $y\in W$ и $x=f_2(y)$.

 (ii) Если $x_2=0$, положим $m_k=n_k-1$ для $k\geqslant1$ и
 $$y=\sum_{k=1}^\infty \chi_{[m_{2k-1},m_{2k})}.$$
 Тогда $y\in W$ and $x=f_2(y)$.

 Таким образом, $W=f_1(W)\cup f_2(W).$

 Теперь мы покажем, что размерность Хаусдорфа множества $W$ равна $1$.

Т.к. для $j=1,2$ верно
 $$\rho(f_j(x),f_j(y))=\frac12\rho(x,y), \ \forall \ x, y \in W,$$
 то функции $f_j$ являются преобразованиями подобия с коэффициентами $r_j=1/2$ для $j=1,2$.

% Clearly, $f_j(0,1)$, $j=1,2$ are pairwise disjoint and
%$$f_1(0,1) \cup f_2(0,1) \subset (0,1).$$ This means, that the iterated function system  $\{f_1, f_2\}$ satisfies the open set condition.

По~\cite[Теорема 9.3]{Edgar} размерность Хаусдорфа $d$ множества $W$ является решением уравнения:
$$ r_1^d+r_2^d=1.$$
Т.к. $r_j=1/2$, то
$d=1.$
\end{proof}

\section{Инвариантные банаховы пределы}\label{sec:inv}

Существование банаховых пределов, инвариантных относительно регулярных преобразований Хаусдорфа, было доказано У. Эберлейном \cite{Eberlein}. Более общий результат в этом направлении был получен в \cite{SS_JFA}.
Обозначим через $\Gamma$ множество линейных операторов $H$ в $\ell_\infty$, удовлетворяющих следующим условиям:
\begin{enumerate}
%\renewcommand{\labelenumi}{(\roman{enumi})}
\item[(i)]
$H\geqslant0$ и $H\emm=\emm$;

\item[(ii)]
$Hc_0\subset c_0$;

\item[(iii)]
$\limsup\limits_{j\to\infty} \ (A(I-T)x)_j\geqslant0$ для всех $x\in
l_\infty$, $A\in
R$, $R=R(H)={\rm conv} \{H^n,n\in\mathbb N\cup\{0\}\}$.
\end{enumerate}

В \cite{SS_JFA} доказано, что для любого $H\in \Gamma$ существует такой $B\in\B$, что  $Bx=BHx$ для всех $x\in \ell_\infty$. Множество таких банаховых пределов обозначим через $\B(H)$. Ясно, что $\B(H)$ есть замкнутое выпуклое подмножество $\B$. Нетрудно показать, что $\B(H)$ слабо$^*$ замкнуто. Условиям (i), (ii), (iii) удовлетворяют оператор Чезаро
\begin{equation}\label{cesaro}
(Cx)_n=\frac1n\sum_{k=1}^n x_k,
\end{equation}
где $x=(x_1, x_2, \dots)$,
и оператор растяжения
$$\sigma_n(x_1, x_2, \dots)=(\underbrace{x_1, x_1, \dots, x_1}_{n}, \underbrace{x_2, x_2, \dots, x_2}_{n}, \dots), \ n\in\N, n\geqslant 2.$$

Множества $\B(C)$ и $\mathfrak B(\sigma_m)$, $m\in \N$, $m\geqslant 2$ изучались в работах \cite{SS_JFA}, \cite{ASSU2} и других. Все множества $\mathfrak B(\sigma_m)$ попарно различны. Включение $\mathfrak B(\sigma_n) \subset \mathfrak B(\sigma_m)$ справедливо тогда и только тогда, когда $m=n^k$ для некоторого $k\in \N$.


%We start with simple properties of the set of dilation invariant Banach limits.
%
% \begin{prop}
%  For every $m\in \N$ the set $\mathfrak B(\sigma_m)$ is weak$^*$ closed.
% \end{prop}
%
% \begin{proof}
% Let $\{B_i\}_{i\in I}$ be a net in $\mathfrak B(\sigma_m)$, which is weak$^*$ convergent to some element $B \in \ell_\infty^*$. Since the set $\mathfrak B$ is weak$^*$ closed and, in particular, $\{B_i\}_{i\in I} \in \mathfrak B$, it follows that $B \in \mathfrak B$.
%
% Recall that  the weak$^*$ convergence is pointwise, that is $B_i(x)=B(x)$ for every $x\in \ell_\infty$. Since $B_i(\sigma_mx-x)=0$ for every $i\in I$ and every   $x\in \ell_\infty$, it follows that $B(\sigma_mx-x)=0$ for every   $x\in \ell_\infty$. Hence, $B \in \mathfrak B(\sigma_m)$ and $\mathfrak B(\sigma_m)$ is weak$^*$ closed.
% \end{proof}

Легко показать, что для любого $m\in \N$ множество $\mathfrak B(\sigma_m)$ выпукло. Однако, это неверно для объединения всех этих множеств.

 \begin{thm}
  Множество $\bigcup_{m=2}^\infty\mathfrak B(\sigma_m)$ невыпукло.
 \end{thm}

\begin{proof}
В силу \cite[Теорема 10]{ASSU2} существуют такие $B_2\in \mathfrak B(\sigma_2)$ и $B_3\in \mathfrak B(\sigma_3)$,
что для любого $B\in \mathfrak B(\sigma_\alpha)$ и любого $B\in \mathfrak B(\sigma_\beta)$
\begin{equation}\label{dl1}
\|B_2-B\|_{\ell_\infty^*}=2,
\end{equation}
\begin{equation}\label{dl11}
\|B_3-B\|_{\ell_\infty^*}=2,
\end{equation}
где $\alpha\in \N \setminus \{1,2,4,8,\dots\}$, $\beta\in \N \setminus \{1,3,9,27,\dots\}$.

В частности,
\begin{equation}\label{dl2}
B_2 \notin \mathfrak B(\sigma_{3^k}) \ \text{и} \ B_3 \notin \mathfrak B(\sigma_{2^k}) \ \text{для любых} \ k\in \N.
\end{equation}

Покажем, что
$$B_1 = \frac12(B_2+B_3)$$
не принадлежит $\mathfrak B(\sigma_m)$ для любого $m\in \N$, $m\geqslant 2$. Предположим противное, т.е. $B_1 \in\mathfrak B(\sigma_m)$ для некоторого $m\in \N$, $m\geqslant2$ и рассмотрим 3 возможных варианта:

1. $m=2^k$ для некоторого $k\in \N$,

2. $m=3^k$ для некоторого $k\in \N$,

3. $m\neq 2^k$ и $m\neq 3^k$ для всех $k\in \N$.

Рассмотрим первый случай, т.е. $B_1 \in\mathfrak B(\sigma_{2^k})$. Так как $B_2 \in\mathfrak B(\sigma_{2^k})$, то  $B_3=2B_1-B_2$ инвариантен относительно $\sigma_{2^k}$. Так как $B_3 \in\mathfrak B$, то $B_3 \in\mathfrak B(\sigma_{2^k})$, что противоречит \eqref{dl2}.

Случай 2 рассматривается аналогично и мы снова приходим к противоречию.

Случай 3. Так как
$$\|B_2-B_1\|_{\ell_\infty^*}=\|B_3-B_1\|_{\ell_\infty^*}=2$$
и $B_1-B_2=B_3-B_1$, то
$$\|B_3-B_2\|_{\ell_\infty^*}=\|B_3-B_1+B_1-B_2\|_{\ell_\infty^*}=2\|B_3-B_1\|_{\ell_\infty^*}=4.$$
Расстояние между любыми банаховыми пределами не превосходит 2. Поэтому полученное выше равенство невозможно. Полученные во всех трёх случаях противоречия доказывают, что $\frac12(B_2+B_3)\notin \bigcup_{m=2}^\infty\mathfrak B(\sigma_m)$.
\end{proof}

\begin{thm}
Пусть $m\in \N$, $m\geqslant2$. Существует такой $B\in \B(\sigma_m)$, что для любого $n\in \N\setminus\{1, m, m^2,\dots\}$ и для любого $B_1\in \B(\sigma_n)$ справедливо равенство
\begin{equation*}\label{dl3}
\|B-B_1\|_{\ell_\infty^*}=2,
\end{equation*}
\end{thm}

\begin{proof}
В \cite[Tеорема 9]{ASSU2} доказано существование таких $B\in B(\sigma_m)$ и $e\subset \N$, что $B\chi_e =1$ и $B\chi_{\sigma_ne} =0$ для любого $n\in\N\setminus \{1, m, m^2, \dots\}$. Если $B_1\in B(\sigma_n)$, то $B_1\chi_e =B\chi_{\sigma_ne} =0$. Тогда $\|2\chi_e-\emm\|_{\ell_\infty^*}=1$ и
$$(B-B_1)(2\chi_e-\emm)=B(2\chi_e)-B_1\chi_e+(B-B_1)\emm=2-0+1-1=2.$$
Отсюда $\|B-B_1\|_{\ell_\infty^*}\geqslant2$. Противоположное неравенство $\|B-B_1\|_{\ell_\infty^*}\leqslant2$ очевидно.
\end{proof}

\begin{cor}\label{dl_cor}
Для любого $m\in \N$, $m\geqslant2$ существует такой $B\in \B(\sigma_{m^2})$, что
$$\|B-B_1\|_{\ell_\infty^*}=2 $$
 для любого $B_1\in \B(\sigma_m)$.
\end{cor}

\begin{thm}
Пусть $m\in \N$, $m\geqslant2$. Множество
\begin{equation}\label{dl4}
\bigcup_{m=1}^\infty\mathfrak B(\sigma_{2^{2^m}})
\end{equation}
выпукло и не замкнуто в $\B$.
\end{thm}

\begin{proof}
Выпуклость множества \eqref{dl4} вытекает непосредственно из выпуклости каждого множества $\mathfrak B(\sigma_{2^{2^m}})$ и включения
\begin{equation}\label{dl5}
\mathfrak B(\sigma_{2^{2^m}}) \subset \mathfrak B(\sigma_{2^{2^{m+1}}})
,
\end{equation}
справедливого для любого $m\in \N$.

В силу следствия \ref{dl_cor} существует такая последовательность $\{B_m \in \mathfrak B(\sigma_{2^{2^m}})\}_{m=2}^\infty$, что
\begin{equation}\label{dl6}
\|B_m-B\|_{\ell_\infty^*}=2
\end{equation}
для любого $B \in \mathfrak B(\sigma_{2^{2^{m-1}}})$ и любого $m\in \N$, $m\geqslant2$. В частности для любых $k,n\in\N$ и $D\in \mathfrak B(\sigma_{2^{2^k}})$
$$\left\|D-\sum_{m=k+1}^{k+n}B_m\right\|_{\ell_\infty^*}=n+1.$$
Отсюда и из выпуклости функции
$$(\alpha_{k+1}, \alpha_{k+2}, \dots, \alpha_{k+n}) \mapsto \left\|D-\sum_{m=k+1}^{k+n}\alpha_mB_m\right\|_{\ell_\infty^*}, \alpha_i \in[0,1]$$
по каждой переменной следует, что
$$\left\|D-\sum_{m=k+1}^{k+n}\alpha_mB_m\right\|_{\ell_\infty^*} = 1+ \sum_{m=k+1}^{k+n}|\alpha_m|. $$

Переходя в этом равенстве к пределу при $n\to\infty$, получаем
\begin{equation}\label{dl7}
\left\|D-\sum_{m=k+1}^\infty\alpha_mB_m\right\|_{\ell_\infty^*} = 1+ \sum_{m=k+1}^\infty|\alpha_m|
\end{equation}
для любой последовательности $\{\alpha_m\}_{m=k+1}^\infty$ из $\ell_1$.

Так как $\B$ есть замкнутое подмножество $\ell_\infty^*$, то функционал
$$B_0=\sum_{m=1}^\infty 2^{-m}B_m$$
есть банахов предел, принадлежащий замыканию множества \eqref{dl4} в $\ell_\infty^*$.

Для любого $D\in \mathfrak B(\sigma_{2^{2^k}})$ имеем
$$\left\|\sum_{m=1}^\infty 2^{-m}B_m-D\right\|_{\ell_\infty^*}\geqslant \left\|\sum_{m=k+1}^\infty 2^{-m}B_m-D\right\|_{\ell_\infty^*}-\left\|\sum_{m=1}^k 2^{-m}B_m\right\|_{\ell_\infty^*}.$$

В силу \eqref{dl7}
$$\left\|\sum_{m=k+1}^\infty 2^{-m}B_m-D\right\|_{\ell_\infty^*} = 1+ \sum_{m=k+1}^\infty 2^{-m}=1+2^{-k}.$$
Отсюда
$$\left\|\sum_{m=1}^\infty 2^{-m}B_m-D\right\|_{\ell_\infty^*}\geqslant 1+2^{-k} - \sum_{m=1}^k 2^{-m}= 2^{1-k}>0.$$
Это доказывает, что $B_0\notin \mathfrak B(\sigma_{2^{2^k}})$ для любого $k\in\N$. Следовательно,
$$B_0\notin \bigcup_{m=1}^\infty\mathfrak B(\sigma_{2^{2^m}}).$$
\end{proof}


\section{Банаховы пределы и последовательности ограниченных функций}\label{sec:Fubini}

Пусть $B\in \B$, $f_n$ "--- последовательность измеримых функций на $[0,1]$ такая, что
\begin{equation}\label{f1}
\sup_{n\in\N, \ 0\leqslant t\leqslant 1} |f_n(t)|<\infty
\end{equation}
и функция
\begin{equation}\label{f2}
F(t)=F_B(t)=B(f_n(t)) \mbox{\quad измерима на $[0,1]$.}
 \end{equation}
  Нашей ближайшей целью является доказательство равенства
\begin{equation}\label{Fubini}
\int_0^1 B(f_n(t)) dt = B\left(\int_0^1 f_n(t) dt\right).
 \end{equation}

 Оно является аналогом теоремы Фубини для банаховых пределов и последовательностей функций, удовлетворяющих предположению \eqref{f2}. Функция $F_B$ может быть как измеримой, так и неизмеримой \cite{FT,SS}. Поэтому предположение \eqref{f2} необходимо для корректной постановки задачи.

 \begin{lem}\label{l1}
 Если $x=(x_1,x_2,...)\in\ell_\infty$, то
 \begin{equation}\label{f3}
 \left[ q(x), p(x) \right] \subset \bigcap_{m=1}^\infty \overline{{\rm conv} \{x_k\}_{k\geqslant m}}.
 \end{equation}
 \end{lem}

 \begin{proof}
 Так как
 $$\left[ q(x), p(x) \right] = \left[\inf_{m\in\N} \frac1n \sum_{k=m+1}^{m+n} x_k, \sup_{m\in\N} \frac1n \sum_{k=m+1}^{m+n} x_k \right] \subset \overline{{\rm conv} \{x_k\}_{k\geqslant m}} $$
 для любых $m,n\in\N$, то отсюда вытекает \eqref{f3}.
 \end{proof}

  \begin{lem}\label{l2}
 Если $B\in \B$ и последовательность $f_n$  измеримых функций на $[0,1]$ удовлетворяет условию \eqref{f1}, то для любого $t\in[0,1]$ имеем
 \begin{equation}
B(f_n(t)) \in \bigcap_{m=1}^\infty \overline{{\rm conv} \{f_k(s)\}_{k\geqslant m}}
 \end{equation}
 где замыкание берётся в топологии поточечной сходимости на $[0,1]$.
 \end{lem}

\begin{proof}
Для любого $t\in[0,1]$, применяя теорему Сачестона \cite{S} к числовой последовательности $f_n(t)$, получаем
\begin{equation}\label{f4}
B(f_n(t)) \in \left[ q(f_n(t)), p(f_n(t)) \right].
 \end{equation}
 В силу леммы \ref{l1},
 \begin{equation}\label{f5}
\left[ q(f_n(t)), p(f_n(t)) \right]\subset \overline{{\rm conv} \{f_k(t)\}_{k\geqslant m}}
 \end{equation}
 для каждого $t\in[0,1]$, где замыкание выпуклой оболочки берётся поточечно, т. е. для каждого $t\in[0,1]$.
\end{proof}


\begin{lem}\label{l3}
Пусть $B\in \B$, последовательность $f_n$  измеримых функций на $[0,1]$  удовлетворяет условиям \eqref{f1}, \eqref{f2} и
\begin{equation}\label{f6}
\int_0^1 f_n(t) dt=0, \ \forall n\in \N.
\end{equation}
Тогда
\begin{equation}\label{f7}
\int_0^1 B(f_n(t)) dt=0.
\end{equation}
\end{lem}

 \begin{proof}
 Из предположения \eqref{f6} вытекает, что
 \begin{equation}\label{f8}
\int_0^1 x(t) dt=0
\end{equation}
для любого $x\in {\rm conv} \{f_k\}_{k\geqslant m}$ и для любого $m\in \N.$ В силу теоремы Лебега \eqref{f8} справедливо для любого $x\in \overline{{\rm conv} \{f_k\}_{k\geqslant m}}$, где замыкание берётся в топологии поточеченой сходимости на $[0,1]$. Следовательно, \eqref{f8} справедливо для любого $x\in \bigcap_{m=1}^\infty\overline{{\rm conv} \{f_k\}_{k\geqslant m}}$. Отсюда и из леммы \ref{l2} вытекает \eqref{f7}.
 \end{proof}

Лемма \ref{l3} сохраняет силу при замене условия \eqref{f6} более слабым предположением
$$ \lim_{n\to\infty} \int_0^1 f_n(t) dt=0.$$
Действительно, достаточно применить лемму \ref{l3} к последовательности функций
$$\left\{ t\mapsto f_n(t)-\int_0^1 f_n(s) ds\right\}_{n=1}^\infty$$
и воспользоваться тем фактом, что банаховы пределы совпадают с обычным пределом на сходящихся последовательностях.

 \begin{thm}\label{t1}
Пусть $B\in \B$, последовательность $f_n$  измеримых функций на $[0,1]$  удовлетворяет условиям \eqref{f1} и  \eqref{f2}.
Тогда
\begin{equation*}
\int_0^1 B(f_n(t)) dt = B\left(\int_0^1 f_n(t) dt\right).
 \end{equation*}
\end{thm}

\begin{proof}
В силу леммы \ref{l3} имеем
\begin{align*}
\int_0^1 B(f_n(t)) dt = \int_0^1 B\left(f_n(t)-\int_0^1 f_n(s) ds\right) dt +B\left(\int_0^1 f_n(t) dt\right)= B\left(\int_0^1 f_n(t) dt\right).
\end{align*}
\end{proof}

Из теоремы \ref{t1} вытекает

\begin{cor}
Пусть $B\in \B$, последовательность $f_n$  измеримых функций на $[0,1]$  удовлетворяет условиям \eqref{f1} и  \eqref{f2}.
Тогда
$$\|Bf_n \|_{L_1[0,1]}\leqslant \limsup_{n\to\infty} \|f_n \|_{L_1[0,1]}.$$
В частности, если $\lim_{n\to\infty} \|f_n \|_{L_1[0,1]} = 0$, то $B(f_n(t))=0$ для почти всех $t\in[0,1]$.
\end{cor}

В \cite{FT} было доказано существование такого $B\in \B$, что функция $t\mapsto Br_n(t)$ неизмерима, где $r_n(t)={\rm sign} \sin 2^n\pi t$ "--- функции Радемахера. Поэтому вопрос об измеримости функции $F_B$, определённой по формуле \eqref{f2}, представляется естественным.

\begin{lem}\label{l5}
Если $B\in \B$ и последовательность $f_n$  измеримых функций на $[0,1]$ содержит лишь конечное число различных функций $g_1, ..., g_r$, то функция $F_B\in {\rm conv} \{g_k : 1\leqslant k \leqslant r\}$ и, следовательно, измерима.

Более того,
$$\{F_B: B\in \B\}= \left\{f : f=\sum_{k=1}^r g_k\lambda_k, \sum_{k=1}^r \lambda_k=1, \lambda_k \in [q(\chi_{Q_k}), p(\chi_{Q_k})],  1\leqslant k \leqslant r\right\},$$
где $q$ и $p$  "--- функционалы Сачестона, а $Q_k =\{n\in\N : f_n=g_k\}.$
\end{lem}

\begin{proof}
Для любого $t\in[0,1]$ имеем
\begin{align*}
B(f_n(t))= B\left(\sum_{k=1}^r g_k(t) \chi_{Q_k}\right)= \sum_{k=1}^r g_k(t) B\chi_{Q_k}= \sum_{k=1}^r g_k(t) \lambda_k,
\end{align*}
где $\lambda_k=B\chi_{Q_k}.$ Так как множества $Q_k$ попарно не пересекаются, то
\begin{align*}
\sum_{k=1}^r \lambda_k= \sum_{k=1}^r B\chi_{Q_k}= B\left(\sum_{k=1}^r \chi_{Q_k}\right)=B\emm=1.
\end{align*}

Следовательно, $F_B=Bf_n\in {\rm conv} \{g_k : 1\leqslant k \leqslant r\}$ и $F_B$ измерима.

Вторая часть утверждения вытекает из теоремы \ref{sucheston}. Заметим, если $p(\chi_{Q_n})=0$ для некоторого $n\in \{1, \dots, r\}$, то $\lambda_n=0$, т.е. индекс $n$ отсутствует в представлении $f=\sum_{k=1}^r g_k\lambda_k$.

\end{proof}


\begin{thm}\label{l6}
Если $B\in \B$ и последовательность $\{f_n\}_{n=1}^\infty \subset L_\infty$  относительно компактна, то функция $F_B$ измерима.
\end{thm}

\begin{proof}
Из относительной компактности $\{f_n\}_{n=1}^\infty$ следует,
что $\sup_{n,t} |f_n(t)| < \infty$.

Для любого $k\in\N$ положим $\varepsilon_k=1/k$
и рассмотрим конечную $\varepsilon_k$-сеть $A_k\subset L_\infty$ для $\{f_n\}_{n=1}^\infty$.
Таким образом, для любых $k, n\in\N$ существует элемент $g_{n,k}\in A_k$ такой,
что $\|f_n-g_{n,k}\|_{L_\infty}<\varepsilon_k$.
Рассмотрим последовательности $g_k:=\{g_{n,k}\}_{n=1}^\infty$ состоящие лишь из конечного числа различных  элементов. По лемме \ref{l5} функция $Bg_k:= B(n\mapsto \{g_{n,k}\}_{n=1}^\infty)$ измерима для любого $k\in \N.$ Используя положительность банаховых пределов, получаем
$$\|F_B-Bg_{k}\|_{L_\infty}\leqslant\|f_n-g_{n,k}\|_{L_\infty}<\varepsilon_k.$$
Т.е. последовательность измеримых функций $Bg_k$ равномерно сходится к функции $F_B$. В силу~\cite[Теорема 4.2.2]{NatansonTF} $F_B$ также измерима.
\end{proof}

Естественный подкласс банаховых пределов при рассмотрении аналогов теоремы Фубини был введён в \cite{Mokobodzki}.
Зафиксируем некоторое вероятностное пространство $(\Omega, \mathcal A, P)$. Линейный функционал $B$ на $\ell_\infty$ принадлежит множеству $\mathfrak M$, если он является банаховым пределом и для любой равномерно ограниченной последовательности $f_n$  $\mathcal A$-измеримых функций на $\Omega$ функция $\omega \mapsto B(f_n(\omega))$ $\mathcal A$-измерима и выполнено
\begin{equation}\label{Fubini1}
\int_\Omega B(f_n(\omega)) dP(\omega) = B\left(\int_\Omega f_n(\omega) dP(\omega)\right).
 \end{equation}

 В работе \cite{Bjorklund} было показано, что пересечение множества $\mathfrak M$ и множества факторизованных банаховых пределов, т.е. функционалов вида
$$\{ B\in \B : B=\gamma \circ C \ \text{для некоторого обобщённого предела} \ \gamma\},$$
непусто.

%Следующий результат является следствием теоремы \ref{t1}.
%
%\begin{thm}
%Пусть $\Omega=[0,1]$, $\mathcal A$ - семейство всех измеримых по Лебегу подмножеств отрезка $[0,1]$ и $P$ - мера Лебега. Банахов предел является пределом Мокобоцкого если и только если для любой последовательности $f_n$  измеримых функций на $[0,1]$,  удовлетворяющей условию \eqref{f1}, выполнено условие \eqref{f2}.
%\end{thm}


\printbibliography


\end{document}
