В данном пункте обсуждаются некоторые свойства множеств
\begin{equation}
	\label{eq:alpha_T^n_x_equiv_alpha_x}
	\{x \in \ell_\infty : \alpha(T^n x) = \alpha(x) \}, ~n\in\mathbb{N},
\end{equation}
\begin{equation}
	\label{eq:cap_alpha_T^n_x_equiv_alpha_x}
	\bigcap\limits_{n\in\mathbb{N}}\{x \in \ell_\infty : \alpha(T^n x) = \alpha(x) \}
	,
\end{equation}
\begin{equation}
	\label{eq:cup_alpha_T^n_x_equiv_alpha_x}
	\bigcup_{n\in\mathbb{N}}\{x \in \ell_\infty : \alpha(T^n x) = \alpha(x) \}
	.
\end{equation}

\subsection{Незамкнутость относительно сложения}

\begin{theorem}
	Ни одно из множеств
	\eqref{eq:alpha_T^n_x_equiv_alpha_x}, \eqref{eq:cap_alpha_T^n_x_equiv_alpha_x}, \eqref{eq:cup_alpha_T^n_x_equiv_alpha_x}
	не замкнуто по сложению и, следовательно, не является пространством.
\end{theorem}

\paragraph{Доказательство.}
Построим два таких элемента, принадлежащих множеству \eqref{eq:alpha_T^n_x_equiv_alpha_x} при любых $n\in\mathbb{N}$,
сумма которых не принадлежит множеству \eqref{eq:alpha_T^n_x_equiv_alpha_x} ни при каких $n\in\mathbb{N}$.
Пусть $m\in\mathbb{N}$, $m \geq 3$.
Положим

\begin{equation}
	x_k = \begin{cases}
		\dfrac{1}{2}(-1)^m,  & \mbox{если } k = 2^m     \\
		1,                   & \mbox{если } k = 2^m + 1 \\
		-1,                  & \mbox{если } k = 2^m + 2 \\
		0                    & \mbox{иначе }
	\end{cases}
\end{equation}

и

\begin{equation}
	y_k = \begin{cases}
		\dfrac{1}{2}(-1)^m,  & \mbox{если } k = 2^m     \\
		-1,                  & \mbox{если } k = 2^m + 1 \\
		1,                   & \mbox{если } k = 2^m + 2 \\
		0                    & \mbox{иначе }
	\end{cases}
\end{equation}

Так как
\begin{equation}
	(T^n x)_{2^m-n+1} - (Т^n x)_{2^m-n+2} = 2
\end{equation}
и
\begin{equation}
	(T^n y)_{2^m-n+1} - (Т^n y)_{2^m-n+2} = 2
	,
\end{equation}
то
\begin{equation}
	\alpha(x) = \alpha(T^n x) = \alpha(y) = \alpha(T^n y) = 2
	.
\end{equation}
С другой стороны,
\begin{equation}
	(x+y)_k = \begin{cases}
		(-1)^m,  & \mbox{если } k = 2^m     \\
		0        & \mbox{иначе }
	\end{cases}
\end{equation}
и
\begin{equation}
	\alpha(x+y) = 2
	,
\end{equation}
но в то же время
\begin{equation}
	\alpha(T^n(x+y)) = 2
\end{equation}
(см. пример \ref{ex:alpha_x_neq_alpha_Tx}),
следовательно, $x+y$ не принадлежит ни одному из множеств
\eqref{eq:alpha_T^n_x_equiv_alpha_x}, \eqref{eq:cap_alpha_T^n_x_equiv_alpha_x}, \eqref{eq:cup_alpha_T^n_x_equiv_alpha_x}.

Теорема доказана.



