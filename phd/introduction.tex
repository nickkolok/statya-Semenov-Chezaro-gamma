Сходящиеся последовательности, т.е. последовательности, имеющие предел в смысле классического математического анализа,
изучены достаточно хорошо.
В частности, любая сходящаяся последовательность является ограниченной.
Пространство ограниченных последовательностей будем, вслед за классиками \cite{wojtaszczyk1996banach,lindenstrauss1973classical},
обозначать через $\ell_\infty$ и снабжать его нормой
\begin{equation*}
	\|x\| = \sup_{n\in\mathbb{N}}|x_n|
	.
\end{equation*}

Однако в приложениях часто возникают ограниченные последовательности,
которые не являются сходящимися.
В таком случае возникает закономерный вопрос:
как измерить <<недостаток сходимости>>?
<<насколько не сходится>> последовательность?

Наиболее очевидным кажется вычисление расстояния $\rho(x,c)$ от заданного элемента $x\in\ell_\infty$
до пространства сходящихся последовательностей $c$
(которое равно половине разности верхнего и нижнего пределов последовательности).
Однако выясняется, что имеют место быть и другие подходы.

Нетрудно заметить, что операция взятия классического предела на пространстве сходящихся последовательностей
является непрерывным (в норме $\ell_\infty$) линейным функционалом.
В 1929 г. С. Мазур анонсировал~\cite{Mazur}, а позже
С. Банах доказал \cite{banach2001theory_rus}, что этот функционал может быть непрерывно продолжен на всё пространство $\ell_\infty$.
На основе этой идеи были определены банаховы пределы
(иногда также называемые пределами Банаха--Мазура \cite{alekhno2012superposition,alekhno2015banach})
следующим образом.

Банаховым пределом называется функционал $B\in \ell_\infty^*$ такой, что:
\begin{enumerate}
	\item
		$B \geqslant 0$
	\item
		$B\one = 1$
	\item
		$B=BT$
\end{enumerate}

Простейшие свойства:
\begin{itemize}
	\item
		$\|B\|_{\ell_\infty^*} = 1$
	\item
		$Bx = \lim\limits_{n\to\infty} x_n$ для любого $x=(x_1, x_2, ...) \in c$.

		Таким образом,
		банахов предел~--- действительно естественное обобщение понятия предела сходящейся последовательности
		на все ограниченные последовательности.
\end{itemize}

Множество банаховых пределов обычно обозначают через $\mathfrak{B}$
(реже через $BM$~--- см., например, \cite{alekhno2012superposition,alekhno2015banach}).

Лоренц \cite{lorentz1948contribution} установил, что существует подпространство $\ell_\infty$,
на котором все банаховы пределы принимают одинаковое значение.
Это пространство названо пространством почти сходящихся последовательностей и обычно обозначается $ac$
(от англ. <<almost convergent>>).
Включение $c \subset ac$ собственное, т.е. $ac \setminus c \neq \varnothing$.


Обобщая критерий Лоренца, Сачестон \cite{sucheston1967banach} доказал, что для любого $x\in\ell_\infty$
и любого $B\in\mathfrak{B}$
\begin{equation*}
	q(x) =
	\lim_{n\to\infty} \sup_{m\in\mathbb{N}} \frac{1}{n} \sum_{k=m+1}^{k=m+n} x_k
	\leq
	Bx
	\leq
	\lim_{n\to\infty} \inf_{m\in\mathbb{N}} \frac{1}{n} \sum_{k=m+1}^{k=m+n} x_k
	= p(x)
\end{equation*}
и, более того,
\begin{equation*}
	\mathfrak{B}x = [q(x), p(x)]
	.
\end{equation*}


За более подробным обзором ранних исследований банаховых пределов отсылаем читателя к~\cite{greenleaf1969invariant,day1973normed,kangro1976theory}.
%Source: https://encyclopediaofmath.org/wiki/Banach_limit
Вскоре после работ Сачестона Дж. Куртц распространил понятие банаховых пределов
на векторные последовательности~\cite{kurtz1970almost},
а затем и на последовательности в произвольных банаховых пространствах~\cite{kurtz1972almost}.
За обсуждением банаховых пределов в векторных пространствах отсылаем читателя
к~\cite{deeds1968summability,hajdukovic1975almost,armario2013vectorvalued_rus,garcia2015extremal,garcia2016fundamental_rus}.
%Тут есть ещё ссылки: https://www.mathnet.ru/php/archive.phtml?wshow=paper&jrnid=faa&paperid=3146&option_lang=rus
%TODO2: В том числе и про то, где применяется.
В недавней работе~\cite{chen2007characterizations} Ч.~Чен и М.~Куо изучают обобщения банаховых пределов
на произвольные гильбертовы пространства и на пространства суммируемых функций $L_p$.
Другим обобщениям банаховых пределов посвящены работы
\cite{hajdukovic1975functionals,koga2016generalization}.
%tanaka2018banach - иероглифы, несовместимые с библиографией

Ещё одним достаточно плодотворным обобщением банаховых пределов оказались их аналоги на двойных последовательностях~\cite{robison1926divergent}, введённые Дж.~Д.~Хиллом в~\cite{hill1965almost}.
За дальнейшими результатами в этом направлении отсылаем читателя
к~\cite{moricz1988almost,bacsarir1995strong,mursaleen2003almost,edely2004almost,mursaleen2004almost}.
Из недавних работ стоит отдельно отметить статью М. Мурсалена и С.А. Мухиддина~\cite{mursaleen2012banach},
в которой с помощью понятия почти сходимости в пространстве ограниченных двойных последовательностей вводится ряд новых интересных подпространств.

Наконец, если исключить из определения банахова предела требование трансляционной инвариантности,
то мы получим объект, называемый обобщённым пределом,
подробно изучавшийся М.~Джерисоном в~\cite{jerison1957set} и многих других работах.


Таким образом, на вопрос: <<Насколько не сходится последовательность?>> %~---
можно дать ответ в терминах почти сходимости, т.е. принадлежности пространству $ac$,
а на вопрос: <<Насколько почти не сходится последовательность?>>~---
назвать длину отрезка $[q(x), p(x)]$.
В дальнейшем пространство почти сходящихся последовательностей неоднократно становилось предметом
различных исследований
\cite{semenov2006ac,usachev2008transforms}.
В частности, в работе~\cite{connor1990almost} доказано,
что последовательность из нулей и единиц почти наверное не принадлежит пространству $ac$.
Этот факт демонстрирует, что почти сходящиеся последовательности <<достаточно редки>>.

%TODO: ссылки! Хватит или ещё?

Банаховы пределы также нашли своё применение в приложениях
\cite{semenov2015banachtraces,SU,strukova2015spectres}.
%Известны обобщения банаховых пределов на двойные последовательности
%\cite{edely2004almost}.

%TODO: ссылки! Хватит или ещё?


В настоящей работе рассматриваются некоторые вопросы асимптотических характеристик ограниченных последовательностей,
в том числе банаховых пределов.
Нумерация приводимых ниже теорем, лемм, определений и следствий совпадает с их нумерацией в диссертации.


В главе 1 обсуждается пространство $ac$ и его подпространство $ac_0$,
даётся критерий почти сходимости к нулю (т.е. принадлежности пространству $ac_0$)
знакопоcтоянной последовательности.

\reflecttheorem{thm:M_j_ac0_inf_lim}
	Пусть $n_i$~--- строго возрастающая последовательность натуральных чисел,
	\begin{equation}
		\label{eq:definition_M_j}
		M(j) = \liminf_{i\to\infty} n_{i+j} - n_i,
	\end{equation}
	\begin{equation}
		x_k = \left\{\begin{array}{ll}
			1, & \mbox{~если~} k = n_i
			\\
			0  & \mbox{~иначе~}
		\end{array}\right.
	\end{equation}
	Тогда следующие условия эквивалентны:
	\\
	(i)   $x \in ac_0$;
	\\\\
	(ii)  $\lim\limits_{j \to \infty} \dfrac{M(j)}{j} = +\infty$;
	\\\\
	(iii) $\inf\limits_{j \in \N}     \dfrac{M(j)}{j} = +\infty$.


\reflecttheorem{thm:crit_ac0_Mj_lambda}
	Пусть $x\in\ell_\infty$, $x \geq 0$, $\lambda>0$.
	Обозначим через $n^{(\lambda)}_i$ возрастающую последовательность
	индексов таких элементов $x$, что $x_k \geq \lambda$ тогда и только тогда,
	когда $k=n^{(\lambda)}_i$ для некоторого $i$.
	Обозначим
	\begin{equation}
		M^{(\lambda)}(j) = \liminf_{i\to\infty} n^{(\lambda)}_{i+j} - n^{(\lambda)}_i
		.
	\end{equation}


	Тогда для того, чтобы $x\in ac_0$, необходимо и достаточно, чтобы
	для любого $\lambda>0$ было выполнено
	\begin{equation}
		\lim_{j \to \infty} \frac{M^{(\lambda)}(j)}{j} = +\infty
		.
	\end{equation}

\reflecttheorem{thm:rho_x_c_leq_alpha_t_s_x_united}
	Для любого $x\in ac$
	\begin{equation}
		\frac{1}{2} \alpha(x) \leq \rho(x,c)\leq \lim_{s\to\infty} \alpha(T^s x)
		.
	\end{equation}

\reflectcorollary{cor:rho_x_c0_leq_alpha_t_s_x_united}
	Для любого $x\in ac_0$
	\begin{equation}
		\frac{1}{2} \alpha(x) \leq \rho(x,c_0)\leq \lim_{s\to\infty} \alpha(T^s x)
		.
	\end{equation}

\reflecttheorem{thm:Connor_generalized}
	Мера множества $F=\{x\in\Omega : q(x) = 0 \wedge p(x)= 1\}$,
	где $p(x)$ и $q(x)$~--- верхний и нижний функционалы Сачестона соответственно,
	равна 1.


В главе 2
%TODO: \ref ???
изучается $\alpha$--функция, введённая в~\cite{our-vzms-2018}:
\begin{equation}
	\alpha(x) = \varlimsup_{i\to\infty} \max_{i<j\leqslant 2i} |x_i - x_j|
	.
\end{equation}
%
%TODO: ссылка на статью Семенова!
%
Поскольку $\alpha(c)=0$,
то $\alpha$--функцию также можно считать <<мерой несходимости>> последовательности;
равенство $\alpha(x) = 0$, однако, вовсе не гарантирует сходимость.

Устанавливается, что $\alpha$--функция не инвариантна относительно оператора сдвига $T$,
и даётся оценка на $\alpha(T^n x)$.
С другой стороны, $\alpha$--функция, в отличие от некоторых банаховых пределов
\cite{Semenov2010invariant,Semenov2011dan},
инвариантна относительно операторов растяжения $\sigma_n$.
Затем выявляется связь между $\alpha$--функцией, расстоянием от заданнной последовательности до пространства $c$
и почти сходимостью.
Рассмотрены и другие свойства $\alpha$--функции.
Приведём основные результаты.

\reflectcorollary{thm:est_alpha_Tn_x_full}
	Для любых $x\in\ell_\infty$ и $n \in \N$
	\begin{equation}\label{est_alpha_Tn_x}
		\frac{1}{2}\alpha(x) \leq \alpha(T^n x) \leq \alpha(x)
		.
	\end{equation}

\reflecttheorem{thm:alpha_beta_T_seq}
	Пусть $\beta_k$~--- монотонная невозрастающая последовательность,
	$\beta_k \to \beta$, $\beta\in\left[\frac{1}{2}; 1\right]$, $\beta_1 \leq 1$.
	Тогда существует такой $x\in\ell_\infty$, что для любого натурального $n$
	\begin{equation}
		\frac{\alpha(T^n x)}{\alpha(x)} = \beta_n.
	\end{equation}

\reflecttheorem{thm:alpha_sigma_n}
	Для любого $x\in\ell_\infty$ и для любого натурального $n$ верно равенство
	\begin{equation}
		\alpha(\sigma_n x) = \alpha(x)
		.
	\end{equation}

\reflecttheorem{thm:alpha_sigma_1_n}
	Для любого $n\in\N$ и любого $x\in\ell_\infty$ выполнено
	\begin{equation}
		\alpha(\sigma_{1/n} x) \leq \left( 2- \frac{1}{n} \right) \alpha(x)
		.
	\end{equation}

\reflecttheorem{thm:alpha_Cx_no_gamma}
	Имеет место равенство
	\begin{equation}
		\sup_{x\in\ell_\infty, \alpha(x)\neq 0} \frac{\alpha(Cx)}{\alpha(x)}=1
		.
	\end{equation}

\reflecttheorem{thm:alpha_xy}
	Пусть $(x\cdot y)_k = x_k\cdot y_k$.
	Тогда
	$\alpha(x\cdot y)\leq \alpha(x)\cdot \|y\|_* + \alpha(y)\cdot \|x\|_*$,
	где
	\begin{equation}
		\|x\|_* = \limsup_{k\to\infty} |x_k|
	\end{equation}
	есть  фактор-норма по $c_0$ на пространстве $\ell_\infty$.

В параграфе~\ref{sec:space_A0} исследуется пространство $A_0 = \{x: \alpha(x) = 0\}$.
Это пространство несепарабельно, замкнуто относительно покоординатного умножения,
операторов левого и правого сдвигов, оператора Чезаро,
операторов растяжения $\sigma_n$ и усредняющего сжатия $\sigma_{1/n}$.

В параграфе~\ref{sec:noncomplementarity} устанавливается, что оба вложения в цепочке
\begin{equation}
	c_0 \subset A_0 \subset \ell_\infty
\end{equation}
недополняемы.

Как сказано выше, банаховы пределы по определению (как и обычный предел на пространстве $c$) инвариантны относительно оператора сдвига.
Возникает закономерный вопрос: можно ли потребовать от банахова предела сохранять своё значение
при суперпозиции с некоторыми другими операторами на $\ell_\infty$?
Эту проблему исследовал У. Эберлейн в 1950 г. \cite{Eberlein},
т.е. через два года после классической работы Г. Г. Лоренца~\cite{lorentz1948contribution}.
Эберлейн установил, что существуют такие линейные операторы  $A : \ell_\infty\to \ell_\infty$,
для которых $BAx = Bx$ независимо от выбора $x$ и для банаховых пределов специального вида.

Будем говорить, что $B\in\mathfrak B(A)$, $A : \ell_\infty\to \ell_\infty$, если для любого $x\in \ell_\infty$
выполнено равенство $BAx = Bx$.
Такой банахов предел $B$ называют инвариантным относительно оператора $A$.

Можно ли выделить какие-то особые свойства оператора сдвига,
которые необходимы или достаточны оператору, чтобы относительно него были инвариантны все или некоторые банаховы пределы?
Понятно, что если оператор $A$ таков, что для любого $x\in\ell_\infty$ между $Ax$ и $x$
существует (конечное) расстояние Дамерау--Левенштейна \cite{damerau1964technique} (т.е. минимальное количество операций вставки, удаления, замены и перестановки двух соседних элементов последовательности, необходимых для перевода $x$ в $Ax$, причём для разных $x\in\ell_\infty$ эти операции, вообще говоря, не обязаны быть одинаковыми), то относительно данного оператора инвариантен любой банахов предел. Аналогичное утверждение справедливо и в случае, если $Ax -x \in c_0$ для любого $x\in \ell_\infty$.

Следующим по естественности (после сдвига и замены конечного числа элементов) действием, сохраняющем сходимость последовательности, является повторение элементов последовательности, например, оператор
\begin{equation}
	\sigma_2(x_1,x_2,x_3,...) = (x_1,x_1, \; x_2, x_2, \; x_3, x_3, \; ...)
	.
\end{equation}
Однако относительно такого оператора инвариантны не все, а только некоторые банаховы пределы.
Так, в~\cite[теорема 14]{ASSU2} показано, что
\begin{equation}
	\B(\sigma_n) \cap \ext \B = \varnothing  \mbox{~~для любого~~} n\in\N_2
	.
\end{equation}
Заметим, что если мы рассмотрим оператор неравномерного растяжения
\begin{equation}
	\sigma_{1,2}(x_1,x_2,x_3,x_4,x_5,...) = (x_1, \; x_2, x_2, \;  x_3, \; x_4, x_4, \; x_5, ...)
	,
\end{equation}
то увидим, что периодическую последовательность $y_n = (-1)^n$, $y\in ac_0$ оператор $\sigma_{1,2}$
переводит в периодическую последовательность
\begin{equation}
	(-1, 1, 1, \; -1, 1, 1, \; ...) \in ac_{1/3}
	,
\end{equation}
поскольку на периодической последовательности любой банахов предел принимает значение, равное среднему по периоду.
Таким образом, не существует банаховых пределов, инвариантных относительно оператора $\sigma_{1,2}$.

В главе 3 изложены некоторые примеры операторов и найдены множества банаховых пределов,
инвариантных относительно этих операторов.
Затем рассматриваются следующие классы линейных операторов $H:\ell_\infty \to \ell_\infty$:

-- полуэберлейновы: такие, что $B_1 H \in\B$ для некоторого $B_1\in \B$;

-- эберлейновы: такие, что $B_1 H = B_1$ для некоторого $B_1\in \B$;

-- В-регулярные: такие, что $B_1 H \in \B$ для любого   $B_1\in \B$;

-- существенно эберлейновы: такие, что $B_1 H \in \B$ для любого $B_1\in \B$ и $B_2 H \ne B_2$ для некоторого $B_2\in \B$.

Устанавливается (см. теоремы~\ref{thm:amiable_but_not_Eberlein_exists} и~\ref{thm:Eberlein_but_not_B-regular_exists}),
что каждый следующий из этих классов вложен в предыдущий и не совпадает с ним.
Кроме того, доказывается ещё ряд смежных результатов, в частности, решается обратная задача об инвариантности.

\reflecttheorem{thm:generated_operator_G_B}
	Для каждого $B\in \B$ существует такой оператор $G_B:\ell_\infty \to \ell_\infty$,
	что $\B(G_B) = \{B\}$.


Главы 4 и 5 посвящены верхнему и нижнему функционалам Сачестона $p(x)$ и $q(x)$"--- аналогам верхнего и нижнего пределов последовательности.
В главе 4 изучаются разделяющие множества.


\reflecttheorem{thm:Lin_Omega_Sucheston}
	Пусть
	$1 \geq a > b \geq 0$ и
	$\Omega^a_b = \{x\in\Omega : p(x) = a, q(x) = b\}$,
	где $p(x)$ и $q(x)$~--- верхний и нижний функционалы Сачестона~\cite{sucheston1967banach} соответственно.
	Тогда $\Omega \subset \operatorname{Lin} \Omega^a_b$.


\reflectcorollary{crl:Lin_Omega_Sucheston}
	Множество $\Omega^a_b$ является разделяющим.
	Т.к. при $a\neq 1$ или $b\neq 0$ множество $\Omega^a_b$ имеет меру нуль~\cite{semenov2010characteristic,connor1990almost},
	то оно является разделяющим множеством нулевой меры.


Пусть $X^a_b = \{x\in\ell_\infty : p(x) = a,~ q(x) = b\}$, $Y^a_b = \{x\in A_0 : p(x) = a, q(x) = b\}$, где $a>b$.
\reflecttheorem{thm:A_0_c_infty_lin}
	Пусть $a\neq -b$.
	Тогда справедливо равенство $\operatorname{Lin} Y^a_b = A_0$.

\reflecttheorem{thm:Lin_ell_infty}
	Справедливо равенство $\operatorname{Lin} X^a_b = \ell_\infty$.

Через $\dim_H E$ будем обозначать хаусдорфову размерность множества $E$.

\reflecttheorem{thm:Hausdorf_measure_1_n}
	Пусть $n\in\N$.
	Тогда существует разделяющее множество $E\subset\Omega$ такое,
	что $\dim_H E = 1/n$.

%Конец главы 4

Глава 5 посвящена связи мультипликативных свойств носителя последовательности из нулей и единиц
и значений, которые могут принимать функционалы Сачестона на такой последовательности.


\reflectcorollary{cor:ac0_powers_finite_set_of_numbers}
	Пусть $\{p_1, ..., p_k\} \subset \N$,
	\begin{equation}
		x_k = \begin{cases}
			1, &\mbox{~если~} k = p_1^{j_1}\cdot p_2^{j_2}\cdot ... \cdot p_k^{j_k} \mbox{~для некоторых~} j_1,...,j_k\in\N,
			\\
			0  &\mbox{~иначе}.
		\end{cases}
	\end{equation}
	Тогда $x\in ac_0$.

\reflectdefinition{def:P-property}
	Будем говорить, что множество $A\subset\N$ обладает $P$-свойством,
	если для любого $n\in\N$ найдётся набор попарно взаимно простых чисел
	\begin{equation}
		\{a_{n,1}, a_{n,2}, ..., a_{n,n}  \} \subset A
		.
	\end{equation}

\reflecttheorem{thm:p_x_infinite_multiples}
	Пусть $A\subset \N\setminus\{1\}$.
	Тогда следующие условия эквивалентны:
	\begin{enumerate}[label=(\roman*)]
		\item
			$A$ обладает $P$-свойством
		\item
			В $A$ существует бесконечное подмножество попарно взаимно простых чисел
		\item
			$p(\chi\mathscr{M}A)=1$.
	\end{enumerate}

\reflectcorollary{cor:ac0_primes_p_psi_A_prod}
	Пусть $A = \{a_1, a_2, ..., a_n,...\}$ "--- бесконечное множество попарно взаимно простых чисел
	и $a_{n+1}>a_1\cdot...\cdot a_n$.
	Тогда
	\begin{equation}
		q(\chi\mathscr{M}A) = 1-\prod_{j=1}^\infty \left(1-\frac{1}{a_j}\right)
		.
	\end{equation}

\reflectlemma{lem:q_x_infinite_Euler}
	Пусть $\varepsilon \in  (0; 1{]}$.
	Существует бесконечное множество попарно непересекающихся подмножеств простых чисел
	$A_i$ такое, что $q(\chi\mathscr{M}A_i)\geq\varepsilon$ для любого $i\in\N$.
