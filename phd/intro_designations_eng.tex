Below we list the main mathematical objects and notions,
as well as notations that may be used in the thesis text without additional explanation.

\paragraph{Sets of Numbers.}
We denote the set of natural numbers (positive integers) by \(\N\): \(\N = \{1;2;3;4;...\}\).
We denote by \(\N_k\) the set of integers not less than \(k\) (to avoid cumbersome notations like: let \(n\in \N\), \(n \geq 2\), …).
Thus,
\begin{equation}
	\N_1 = \N
	,
	\quad
	\N_0 = \N \cup\{0\} = \{0;1;2;3;4;...\}
	,
	\quad
	\N_3 = \{3;4;5;6;7;...\}
	.
\end{equation}

We denote the sets of rational and real numbers by \(\Q\) and \(\R\) respectively;
the sets of positive rationals, negative rationals,
positive reals, and negative reals by \(\Q^+\), \(\Q^-\), \(\R^+\), and \(\R^-\) respectively.

\paragraph{Spaces of Sequences and Their Subsets.}

The main space used is the space \(\ell_\infty\) of bounded sequences with the standard norm
\begin{equation}
	\|x\| = \sup_{n\in\N} |x_n|
	.
\end{equation}
This norm is also used in the other spaces and sets, such as:
\begin{itemize}
	\item
		\(c\) — the space of convergent sequences;
	\item
		\(c_\lambda\) — the set of sequences converging to \(\lambda \in \R\), in particular,
	\item
		\(c_0\) — the space of null sequences;
	\item
		\(c_{00}\) — the space of sequences with finite support~\cite[Theorem 4]{ASSU2};
	\item
		\(ac\) — the space of almost convergent sequences;
	\item
		\(ac_\lambda\) — the set of sequences almost convergent to \(\lambda \in \R\), in particular,
	\item
		\(ac_0\) — the space of sequences almost convergent to zero;
	\item
		\(\Iac\) — the maximal (by inclusion) multiplication ideal in the space \(ac_0\) (see Theorem~\ref{thm:Iac_criterion_pos_neg});
	\item
		\(A_0 = \{ x\in\ell_\infty : \alpha(x) = 0\}\) (see below; for properties of this space, see, e.g., Theorem~\ref{thm:A0_is_space} and further);
	\item
		\(\Omega = \{0;1\}^\N\) — the set of 0-1-sequences.
\end{itemize}

\paragraph{Norms.}
The notation \(\|\cdot\|\) by default denotes the norm in \(\ell_\infty\), \(\ell_\infty^*\),
or in the space of bounded linear operators from \(\ell_\infty\) to \(\ell_\infty\)
(denoted by \(\mathcal L (\ell_\infty, \ell_\infty)\)) depending on the norm argument's nature.
If a different norm needs to be introduced (for example, the quotient norm \(\ell_\infty / c_0\),
as in Theorem~\ref{thm:alpha_xy}), this is stated explicitly.

\paragraph{Sequences.}
For any \(x\in\ell_\infty\), we by default assume that
\begin{equation}
	x=(x_1, x_2, x_3, x_4, ...)
	.
\end{equation}
Also, we often use the constant unit \(\one = (1, 1, 1, 1, ...)\).

We write \(x\geq 0\) if \(x_n \geq 0\) for any \(n\in \N\), and \(x\leq 0\) if \(x_n \leq 0\) for any \(n\in \N\).
%TODO: periodic sequences and concatenation

Following~\cite{hall1992behrend}, we denote by \(\mathscr{M}A\) the set of multiples of \(A\subset\N\), i.e.
\begin{equation}
	\mathscr{M}A = \{ka: k\in\N, a\in A\}
	,
\end{equation}
by \(\chi F\) — the characteristic function of the set \(F\).

For example,
\begin{gather}
	\chi \mathscr{M}\!A(\{2\}) = \chi \mathscr{M}\!A(\{2, 4\}) = \chi \mathscr{M}\!A(\{2,4,8,16,...\})
	= (0,1,0,1,0,1,0,1,0,...),
\\
	\chi \mathscr{M}\!A(\{3\}) = \chi \mathscr{M}\!A(\{3,9,27,...\}) = (0,0,1,\;0,0,1,\;0,0,1,\;0,0,1,\;0,0,1,\;...),
\\
	\chi \mathscr{M}\!A(\{2,3\}) = \chi \mathscr{M}\!A(\{2,3,6\}) = (0,1,1,1,0,1,\;0,1,1,1,0,1,\;0,1,1,1,0,1,...).
\end{gather}

\paragraph{Operators.}
The following operator notations are widely used in the thesis.
\begin{itemize}
	\item
		Identity operator:
		\begin{equation}
			I(x_1, x_2, x_3, x_4, ...) = (x_1, x_2, x_3, x_4, ...)
			.
		\end{equation}
	\item
		Left shift operator:
		\begin{equation}
			T(x_1, x_2, x_3, x_4, ...) = (x_2, x_3, x_4, ...)
			.
		\end{equation}
	\item
		Right shift operator:
		\begin{equation}
			U(x_1, x_2, x_3, x_4, ...) = (0, x_1, x_2, x_3, x_4, ...)
			.
		\end{equation}
	\item
		Dilation operator (\(n\in\N\)):
		\begin{equation}
			\sigma_n (x_1, x_2, x_3, ...) = (
				\underbrace{x_1,...,x_1}_{n~\text{times}},
				\underbrace{x_2,...,x_2}_{n~\text{times}},
				\underbrace{x_3,...,x_3}_{n~\text{times}},
				...)
			.
		\end{equation}
	\item
		Averaging compression operator (\(n\in\N\)):
		\begin{equation}
			\sigma_{1/n} x = n^{-1}
			\left(
				\sum_{i=1}^{n} x_i,
				\sum_{i=n+1}^{2n} x_i,
				\sum_{i=2n+1}^{3n} x_i,
				...
			\right).
		\end{equation}
	\item
		Cesàro operator (sometimes called the Hardy operator):
		%TODO: references where it is Hardy?
		\begin{equation}
			(Cx)_n = n^{-1} \cdot \sum_{k=1}^n x_k
			,
		\end{equation}
		i.e.
		\begin{equation}
			C (x_1, x_2, x_3, ...) = \left(
			x_1,
			\dfrac{x_1+x_2}2,
			\dfrac{x_1+x_2 + x_3}3,
			\dfrac{x_1+x_2+x_3+x_4}4,
			...,
			\dfrac{x_1+...+x_n}n,
			...\right)
			.
		\end{equation}

	\item
		Superposition operator (coordinate-wise multiplication):
		\begin{equation}
			x \cdot y = (x_1\cdot y_1; ~~x_2\cdot y_2; ~~x_3\cdot y_3; ~...)
		\end{equation}
\end{itemize}

\paragraph{Special Functions and Sets.}

By \(\B\), we denote the set of all Banach limits;
by \(\B(H)\) — the set of all Banach limits invariant with respect to an operator \(H\).

By \(\ext A\), we denote the set of extreme points of the set \(A\).

The lower and upper Sucheston functionals respectively~\cite{sucheston1967banach}:
\begin{equation*}
	q(x) = \lim_{n\to\infty} \inf_{m\in\N}  \frac{1}{n} \sum_{k=m+1}^{m+n} x_k
	~~~~\mbox{and}~~~~
	p(x) = \lim_{n\to\infty} \sup_{m\in\N}  \frac{1}{n} \sum_{k=m+1}^{m+n} x_k
	.
\end{equation*}

Additionally, we often use the function~\selfcite{our-vzms-2018}:
\begin{equation}
	\alpha(x) = \varlimsup_{i\to\infty} \max_{i<j\leqslant 2i} |x_i - x_j|
	.
\end{equation}
