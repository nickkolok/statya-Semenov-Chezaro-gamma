The work investigates such objects and concepts defined on the space of bounded sequences, such as almost convergence (Chapter 1),
$\alpha$-seminorm, briefly referred to as the $\alpha$-function (Chapter 2),
invariant Banach limits (Chapter 3),
separating sets and linear hulls (Chapter 4),
Sucheston functionals and multiplicative properties of a sequence support (Chapter 5).

Chapter 1 derives special criteria of almost convergence and convergence to zero for bounded sequences,
as well as a criterion for convergence of an almost converging sequence.
These criteria are then applied to investigate sequences and sets that naturally arise in other problems concerning Banach limits.
For elements of the space of almost converging sequences $ac$,
a two-sided estimate is established for the distance to the space of converging sequences $c$,
using the $\alpha$-function.
Notably (though expected), along with the shift-noninvariant $\alpha$-function,
this estimate also uses the functional $\lim_{n\to\infty} \alpha(T^n x)$, which is obviously shift-invariant.

Chapter 2 is devoted to the $\alpha$-function
and contains a sufficiently detailed description of the properties of the superposition of this function with classical linear operators:
shift $T$, dilation $\sigma_n$, Cesàro $C$, etc.
Despite the fact that the $\alpha$-function turned out to be shift-noninvariant,
this non-invariance is in a sense homogeneous (see Corollary \ref{thm:est_alpha_Tn_x_full}).
This fact is rather expected, since the distance to the space $c$ and almost convergence themselves are shift-invariant characteristics.
Chapter 2 concludes with an investigation of the subspace of such bounded sequences on which the $\alpha$-function vanishes.
The main properties of this space are proved; in particular, it is not complemented in $\ell_\infty$.

Chapter 3 is directly devoted to invariant Banach limits
and contains a significant number of examples of finding the set of invariant Banach limits $\B(H)$
for various linear operators $H$ acting on the space of bounded sequences.
In particular, the substantial redundancy of previously known Eberleinness conditions of the operator
(i.e., the existence of at least one Banach limit that is invariant with respect to the given operator) is shown.
New and existing classes of linear operators on the space of bounded sequences
are introduced and studied according to the properties of their superposition with Banach limits:
semi-Eberlein, Eberlein, B-regular, essentially Eberlein.
It is shown that each subsequent class is contained in the previous one and does not coincide with it.
A simple solution to the inverse problem of invariance is found,
that is, the problem of constructing an operator by a Banach limit relative to which it is invariant.

Chapter 4 is devoted to separating sets and linear hulls, with the latter, having been studied in detail,
acting as an auxiliary element for constructing the former.
Chapter 4 concludes with the construction of a separating subset of sequences of zeros and ones that has (Lebesgue) measure zero
and arbitrarily small Hausdorff dimension.

Chapter 5 establishes a connection between the multiplicative properties of the support of a sequence of zeros and ones
and the values that the upper and lower Sucheston functionals (and therefore Banach limits) can take on such a sequence.
Constructions from number theory are used to establish this connection.

Based on the conducted research, a number of conjectures have been put forward,
the work on proving or disproving which is planned to be continued in the future.
Also, the conjectures can be given as tasks for young researchers during their first steps into the space of bounded sequences.
