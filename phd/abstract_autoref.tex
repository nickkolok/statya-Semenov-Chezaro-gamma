\begin{center}
	\textbf{
		Николай Николаевич Авдеев
		\\
		Инвариантные банаховы пределы
	}
\end{center}



Диссертация посвящена исследованию инвариантных банаховых пределов и связанных с ними объектов
(функционалов и линейных операторов).
Получены новые критерии почти сходимости ограниченной последовательности.
Изучена полунорма на пространстве ограниченных последовательностей,
которая в работах других учёных возникала естественным образом
(при изучении банаховых пределов, инвариантных относительно оператора Чезаро),
но лишь во вспомогательной роли, и самостоятельным объектом исследования ранее не становилась.
Выявлены новые свойства непрерывных линейных операторов на пространстве ограниченных последовательностей,
определяемые с помощью банаховых пределов.
Построены новые примеры множеств, разделяющих банаховы пределы и обладающих специальными свойствами.
Выявлена связь функционалов Сачестона с множествами кратных.

\medskip

\begin{center}
	\textbf{
		Nikolai Avdeev
		\\
		Invariant Banach limits
	}
\end{center}




The thesis is devoted to the investigation of invariant Banach limits and related objects (functionals and linear operators).
New criteria for the almost convergence of a bounded sequence have been obtained.
A special seminorm on the space of bounded sequences has been studied;
this seminorm arose naturally in the works of other scientists (while studying Cesàro-invariant Banach limits),
but only in an auxiliary role, and has not previously become an independent object of research.
New properties of continuous linear operators on the space of bounded sequences, defined via Banach limits, have been revealed.
New examples of separating sets for Banach limits with special properties have been constructed.
A connection between Sucheston functionals and sets of multiples has been established.
