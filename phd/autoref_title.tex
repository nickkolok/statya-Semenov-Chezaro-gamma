\documentclass[12pt]{article}
%\documentclass[12pt,oneside]{book}
%\usepackage[cp1251]{inputenc}

\usepackage [cp866]{inputenc}

\usepackage[russian]{babel}
\usepackage{amssymb,latexsym,amsmath}
\usepackage{amsfonts}
\pagestyle{plain} \textwidth=165mm \textheight=240mm
\voffset=-17mm \hoffset=-12mm
\renewcommand{\baselinestretch}{1.3}

\newtheorem{defin}{Определение}
\newtheorem{theorem}{Теорема}
\newtheorem{corollary}[theorem]{Следствие}
\newtheorem{lemma}[theorem]{Лемма}
\newtheorem{prop}[theorem]{Утверждение}

\begin{document}
\large

\thispagestyle{empty}

\begin{flushright}
на правах рукописи
\end{flushright}

\vspace{25mm}

\begin{center}
УСАЧЕВ АЛЕКСАНДР СЕРГЕЕВИЧ
\end{center}

\vspace{25mm}

\begin{center}
\textbf{Пространство почти сходящихся последовательностей и
банаховы пределы}
\end{center}

\vspace{10mm}

\begin{center}
01.01.01 --- математический анализ
\end{center}

\vspace{5mm}

\begin{center}
Автореферат

диссертации на соискание ученой степени кандидата

физико-математических наук
\end{center}

\vspace{75mm}

\begin{center}
Воронеж - 2009
\end{center}


\newpage
\thispagestyle{empty}

Работа выполнена в Воронежском государственном университете

\vspace{10mm}

Научный руководитель:

\hspace{25mm}доктор физико-математических наук,

\hspace{25mm}профессор Семенов Евгений Михайлович

\vspace{10mm}

Официальные оппоненты:

\hspace{25mm}доктор физико-математических наук,

\hspace{25mm}профессор Асташкин Сергей Владимирович;

\vspace{10mm}

\hspace{25mm}доктор физико-математических наук,

\hspace{25mm}профессор Баскаков Анатолий Григорьевич.


\bigskip

Ведущая организация - Московский физико-технический институт

\vspace{10mm}

Защита состоится <8> сентября 2009 г. в 15 часов 10 минут на
заседании диссертационного совета Д 212.038.22 при Воронежском
государственном университете по адресу: 394006, г. Воронеж,
Университетская пл., 1, ВГУ, математический факультет.

\vspace{5mm}

С диссертацией можно ознакомиться в библиотеке Воронежского
государственного университета.

\vspace{10mm}

Автореферат разослан \_ \_ \_ июня 2009 г.

\vspace{10mm}

Ученый секретарь

\noindent диссертационного совета Д 212.038.22,

\noindent доктор физико-математических наук,

\noindent профессор \hspace{110mm}Ю.Е.Гликлих


\end{document}
