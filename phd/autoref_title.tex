\documentclass[12pt]{article}
%\documentclass[12pt,oneside]{book}
%\usepackage[cp1251]{inputenc}

\usepackage [utf8]{inputenc}

\usepackage[russian]{babel}
\usepackage{amssymb,latexsym,amsmath}
\usepackage{amsfonts}
\pagestyle{plain} \textwidth=165mm \textheight=240mm
\voffset=-17mm \hoffset=-12mm
\renewcommand{\baselinestretch}{1.3}

\newtheorem{defin}{Определение}
\newtheorem{theorem}{Теорема}
\newtheorem{corollary}[theorem]{Следствие}
\newtheorem{lemma}[theorem]{Лемма}
\newtheorem{prop}[theorem]{Утверждение}

\begin{document}
\large

\thispagestyle{empty}

\begin{flushright}
на правах рукописи
\end{flushright}

\vspace{25mm}

\begin{center}
АВДЕЕВ НИКОЛАЙ НИКОЛАЕВИЧ
\end{center}

\vspace{25mm}

\begin{center}
\textbf{Инвариантные банаховы пределы}
\end{center}

\vspace{10mm}

\begin{center}
01.01.01 --- вещественный, комплексный и функциональный анализ
\end{center}

\vspace{5mm}

\begin{center}
Автореферат

диссертации на соискание ученой степени кандидата

физико-математических наук
\end{center}

\vspace{75mm}

\begin{center}
Воронеж 2024
\end{center}


\newpage
\thispagestyle{empty}

Работа выполнена в Воронежском государственном университете

\vspace{10mm}

Научный руководитель:

\hspace{25mm}доктор физико-математических наук,

\hspace{25mm}профессор Семенов Евгений Михайлович

\vspace{10mm}

Официальные оппоненты:

\hspace{25mm}доктор физико-математических наук,

\hspace{25mm}((TODO)).

\vspace{10mm}

\hspace{25mm}((TODO)) физико-математических наук,

\hspace{25mm}((TODO)).


\bigskip

Ведущая организация - ((TODO))

\vspace{10mm}

Защита состоится TODO 2024 г. в TODO часов TODO минут на
заседании диссертационного совета TODO при TODO по адресу:
TODO.

\vspace{5mm}

С диссертацией можно ознакомиться в библиотеке TODO.

\vspace{10mm}

Автореферат разослан \_ \_ \_ TODO 2024 г.

\vspace{10mm}

Ученый секретарь

\noindent диссертационного совета TODO,

\noindent TODO,

\noindent TODO \hspace{110mm}TODO


\end{document}
