\documentclass[12pt,a4paper,openbib]{report}
\usepackage{amsmath}
\usepackage[utf8]{inputenc}
\usepackage[english,russian]{babel}
\usepackage{amsfonts,amssymb}
\usepackage{latexsym}
\usepackage{euscript}
\usepackage{enumerate}
\usepackage{graphics}
\usepackage[dvips]{graphicx}
\usepackage{geometry}
\usepackage{wrapfig}
\usepackage[colorlinks=true,allcolors=black]{hyperref}
\usepackage{bbm}
\usepackage{enumitem}
\usepackage{mathrsfs}


% https://tex.stackexchange.com/questions/634509/show-hide-thumbnail-sidebar-by-default-in-pdf
\hypersetup{pdfpagemode=UseNone}


\righthyphenmin=2

%\usepackage[14pt]{extsizes}

\geometry{left=2.5cm}% левое поле
\geometry{right=1cm}% правое поле
\geometry{top=2cm}% верхнее поле
\geometry{bottom=2cm}% нижнее поле

\renewcommand{\baselinestretch}{1.3}

\renewcommand{\le}{\leqslant}
\renewcommand{\ge}{\geqslant}
\renewcommand{\leq}{\leqslant}
\renewcommand{\geq}{\geqslant} % И делись оно всё нулём!

\renewcommand{\varlimsup}{\limsup}
\renewcommand{\varliminf}{\liminf} % ... по самую асимптоту!



\DeclareMathOperator{\ext}{ext}
\DeclareMathOperator{\mes}{mes}
\DeclareMathOperator{\supp}{supp}
\DeclareMathOperator{\conv}{conv}
\DeclareMathOperator{\diam}{diam}

\newcommand{\N}{\ensuremath{\mathbb{N}}}
\newcommand{\Q}{\ensuremath{\mathbb{Q}}}
\newcommand{\R}{\ensuremath{\mathbb{R}}}
\newcommand{\B}{\ensuremath{\mathfrak{B}}}
\newcommand{\Iac}{\mathcal{I}(ac_0)}
\newcommand{\Dac}{\mathcal{D}(ac_0)}
\newcommand{\one}{\ensuremath{\mathbbm 1}}


\newcommand{\longcomment}[1]{}

\usepackage[backend=biber,style=gost-numeric,sorting=none]{biblatex}
\addbibresource{../bib/Semenov.bib}
\addbibresource{../bib/my.bib}
\addbibresource{../bib/ext.bib}
\addbibresource{../bib/classic.bib}
\addbibresource{../bib/Damerau-Levenstein.bib}
\addbibresource{../bib/general_monographies.bib}
\addbibresource{../bib/Bibliography_from_Usachev.bib}

\input{../bib/ext.hyphens.bib}

\usepackage{amsthm}
\theoremstyle{definition}
\newtheorem{lemma}{Лемма}[section]
\newtheorem{theorem}[lemma]{Теорема}
\newtheorem{example}[lemma]{Пример}
\newtheorem{property}[lemma]{Свойство}
\newtheorem{remark}[lemma]{Замечание}
\newtheorem{definition}[lemma]{Определение}
\newtheorem{proposition}[lemma]{Утверждение}
\newtheorem{corollary}{Следствие}[lemma]

\newtheorem{hhypothesis}[lemma]{Гипотеза}


\newcommand\hypotlist{ }
\newcounter{hypcount}

\makeatletter
\usepackage{environ}
\NewEnviron{hypothesis}{%

	\edef\curlabel{hhypothesis\thehypcount}
    \begin{hhypothesis}
		\label{\curlabel}
		\BODY%
    \end{hhypothesis}
	\edef\curref{\noexpand\ref{\curlabel}}

	\expandafter\g@addto@macro\expandafter\hypotlist\expandafter
	{\paragraph{Гипотеза\!\!\!}}


	\expandafter\g@addto@macro\expandafter\hypotlist\expandafter
	{\expandafter\textbf\expandafter{\curref}}

	\expandafter\g@addto@macro\expandafter\hypotlist\expandafter
	{\textbf{.}~~}

	\expandafter\g@addto@macro\expandafter\hypotlist\expandafter
	{\BODY}

	\addtocounter{hypcount}{1}
}
\makeatother

%Only referenced equations are numbered
\usepackage{mathtools}
\mathtoolsset{showonlyrefs}

%\mathtoolsset{showonlyrefs=false}
% (an equation/multline to be force-numbered)
%\mathtoolsset{showonlyrefs=true}

% https://superuser.com/questions/517025/how-can-i-append-two-pdfs-that-have-links
\usepackage{pdfpages}% http://ctan.org/pkg/pdfpages

\begin{document}
\clubpenalty=10000
\widowpenalty=10000
\includepdf{title.pdf}
\setcounter{page}{2}
\tableofcontents

\chapter*{Введение}
\addcontentsline{toc}{chapter}{Введение}

\section*{Общая характеристика работы}
\addcontentsline{toc}{section}{Общая характеристика работы}
\paragraph{Актуальность темы исследования и степень её разработанности.}

Один из основателей функционального анализа как области современной математики С. Банах
в 1932 году ввёл в рассмотрение множество непрерывных линейных функционалов на пространстве ограниченных последовательностей,
которые совпадают с обычным пределом на всех сходящихся последовательностях.
Эти функционалы в дальнейшем и были названы банаховыми пределами;
их изучением занимались Г.Г. Лоренц, Л. Сачестон, Г. Дас, У.Ф. Эберлейн и другие математики.

В 1948 году Г.Г. Лоренц, используя банаховы пределы, ввёл понятие почти сходящейся последовательности "---
последовательности, на которой значение банахова предела не зависит от выбора этого банахова предела.
В 1967 году Л. Сачестон построил аналоги верхнего и нижнего пределов для банаховых пределов "---
(нелинейные) функционалы Сачестона.
Изучению и обобщению понятия почти сходимости посвящены работы
Р.А. Раими, М. Мурсалена, Г. Беннета, Н.Дж. Калтона, Д. Хаджуковича, Е.А. Алехно, Д. Занина и др.
Отдельный интерес представляет вопрос о банаховых пределах, инвариантных относительно некоторых линейных операторов.

Банаховы пределы тесно связаны с другими областями математики.
Так, в исследованиях С. Лорда, Дж. Филлипса, А.Л. Кери, П.Г. Доддса, Е.М. Семёнова, Б. де Пагтера,
А.А.~Седаева, А.С. Усачёва, Ф.А. Сукочева банаховы пределы применяются к изучению следов Диксмье;
следы же Диксмье, в свою очередь, широко применяются в некоммутативной геометрии А. Конна.
Связи банаховых пределов с эргодической теорией посвящены работы Л. Сачестона и др.

В настоящей диссертации исследуются пространство почти сходящихся последовательностей
(и его подпространство последовательностей, почти сходящихся к нулю),
функционалы Сачестона и инвариантные банаховы пределы.

\paragraph{Цели и задачи работы.}
Целью работы является изучение банаховых пределов и их свойств инвариантности.

Задачи работы:
\begin{itemize}
	\item
		исследование структуры множества банаховых пределов;
	\item
		исследование подпространств пространства ограниченных последовательностей,
		определяемых с помощью банаховых пределов;
	\item
		исследование свойств композиции линейных операторов и банаховых пределов.
\end{itemize}


\paragraph{Научная новизна.}
Все основные результаты диссертации являются новыми.


\paragraph{Теоретическая и практическая значимость работы.}
Работа носит теоретический характер.
Результаты диссертации могут быть использованы в учебном процессе, спецкурсах и научных исследованиях,
проводимых в Воронежском, Ростовском, Самарском, Ярославском государственных университетах и др.
В работе сформулированы гипотезы,
которые могут быть использованы в составе заданий для выполнения
студентами и выпускниками бакалавриата, специалитета или магистратуры
курсовых или выпускных квалификационных работ.





\paragraph{Методология и методы исследования.}
Для исследования банаховых пределов и связанных с ними математических объектов применяются
понятия, методы и подходы современного функционального анализа,
а также отдельные элементы и факты топологии и теории чисел.

\paragraph{Положения, выносимые на защиту.}
На защиту выносятся следующие основные результаты:
\begin{enumerate}
	\item
		сформулированы и доказаны специальные критерии почти сходимости
		ограниченной последовательности;
	\item
		исследована новая асимптотическая характеристика ограниченной последовательности,
		позволяющая выявлять дополнительные (по сравнению с банаховыми пределами)
		свойства таких последовательностей;
	\item
		построена иерархическая система классов ограниченных линейных операторов
		на пространстве $\ell_\infty$ в зависимости от свойств их суперпозиции с банаховыми пределами;
	\item
		построен пример множества последовательностей из нулей и единиц, разделяющего банаховы пределы
		и имеющего нулевую меру, индуцированную мерой Лебега на отрезке $[0;1]$;
	\item
		исследована связь мультипликативной структуры носителя последовательности из нулей и единиц
		и значений, которые на такой последовательности могут принимать верхний и нижний функционалы Сачестона.
\end{enumerate}



\paragraph{Степень достоверности и апробация результатов.}

Все включенные в диссертацию результаты доказаны
в соответствии с современными стандартами достоверности математических доказательств.

Результаты диссертации докладывались и обсуждались:
\begin{itemize}
	\item
		на Международной конференции <<Воронежская Зимняя Математическая школа С.Г.~Крейна>> в 2018, 2022, 2025~гг.;
	\item
		на Международной конференции «Понтрягинские чтения — XXIX», посвященной 90-летию Владимира Александровича Ильина (Воронеж, 2018~г.);
	\item
		на Международной молодёжной научной школе «Актуальные направления математического анализа и смежные вопросы» (Воронеж, 2018~г.);
	\item
		на конкурсе научно-исследовательских работ студентов и аспирантов российских вузов
		<<Наука будущего --- наука молодых>> в секции <<Информационные технологии и математика>>
		(присуждено II место среди аспирантов) в ноябре 2021~г.;
	\item
		на Международной (53-й Всероссийской) молодёжной школе-конференции
		<<Современные проблемы математики и её приложений>>
		(Екатеринбург, 2022~г.);
	\item
		на научной сессии ВГУ в 2020, 2021, 2022, 2024, 2025 гг.; %(TODO: проверить недостающие годы - письмо отправлено Г.И.; в 2023 выступления нет - академ)
	\item
		16.10.2024 г. на семинаре в МИАН под рук. чл.-корр. РАН О.В. Бесова;
	\item
		15.10.2024 г. на семинаре в МГУ под рук. проф. РАН П.А. Бородина;
	\item
		13.11.2024 г. на семинаре под рук. проф. С.В. Асташкина (Самара);
	\item
		19.11.2024 г. на семинаре под рук. проф. А.Л. Скубачевского (Москва);
	\item
		29.01.2025 г. на семинаре под рук. проф. А.Г. Кусраева и М.А. Плиева (Владикавказ);
	\item
		на международной (56-й Всероссийской) молодёжной школе-конференции
		<<Современные проблемы математики и её приложений>>
		(Екатеринбург, февраль 2025~г.);
	\item
		на Всероссийской молодёжной научной конференции «Путь в науку. Математика»
		(Ярославль, май 2025~г.; доклад признан одним из лучших и награждён дипломом).
\end{itemize}

Научно-исследовательская работа соискателя по теме диссертации была поддержана грантами:
\begin{itemize}
	\item
		Российского научного фонда (грант №~19-11-00197);
	\item
		Российского научного фонда (грант №~24-21-00220);
	\item
		Фонда развития теоретической физики и математики ``БАЗИС'' (проект №~22-7-2-27-3).
\end{itemize}

\paragraph{Публикации.}
Основные результаты диссертации опубликованы в работах~\selfcite{
%Солидные статьи
our-mz2019ac0,
our-mz2019measure,
our-mz2021linearhulls,
avdeed2021AandA,
avdeev2021vestnik,
avdeev2021vmzprimes,
avdeev2024decomposition,
avdeev2024set_DAN_rus,
%Тезисы
our-vvmsh-2018,
our-vzms-2018,
our-ped-2018-inf-dim-ker,
our-ped-2018-alpha-Tx,
vzms2022setsofmultiples,
avdeev2022measure,
vzms2025linear,
%TODO: если будут ещё работы
}.
Из совместных работ~\selfcite{our-mz2019measure,avdeed2021AandA,avdeev2024decomposition,avdeev2024set_DAN_rus,our-vzms-2018}
в диссертацию вошли только результаты, принадлежащие лично диссертанту.

\paragraph{Структура и объём диссертации.}
Диссертация состоит из введения, пяти глав, разбитых на TODO параграфов,
и списка литературы, включающего TODO источника.
Общий объём диссертации TODO страницы.
%TODO: а подпараграфы?


\section*{Используемые обозначения}
\addcontentsline{toc}{section}{Используемые обозначения}
Приведём список основных математических объектов и понятий,
а также обозначений, которые в тексте диссертации могут быть использованы без дополнительного пояснения.


\paragraph{Числовые множества.}
Через $\N$ будем обозначать множество натуральных чисел: $\N = \{1;2;3;4;...\}$.
Через $\N_k$ ~--- множество целых чисел, не меньших $k$ (чтобы избежать громоздкой записи вида: пусть $n\in \N$, $n \geq 2$, ...).
Так,
\begin{equation}
	\N_1 = \N
	,
	\quad
	\N_0 = \N \cup\{0\} = \{0;1;2;3;4;...\}
	,
	\quad
	\N_3 = \{3;4;5;6;7;...\}
	.
\end{equation}

Через $\Q$ и $\R$ будем обозначать множества рациональных и вещественных чисел соответственно;
через $\Q^+$, $\Q^-$, $\R^+$ и $\R^-$ ~--- множества положительных рациональных, отрицательных рациональных,
положительных вещественных и отрицательных вещественных чисел соответственно.

\paragraph{Пространства последовательностей и их подмножества.}

Основным используемым пространством будет пространство $\ell_\infty$ ограниченных последовательностей со стандартной нормой
\begin{equation}
	\|x\| = \sup_{n\in\N} |x_n|
	.
\end{equation}
Эта же норма будет использоваться и в остальных пространствах и множествах, среди которых:
\begin{itemize}
	\item
		$c$ ~--- пространство сходящихся последовательностей;
	\item
		$c_\lambda$ ~--- множество последовательностей, сходящихся к $\lambda \in \R$, в частности,
	\item
		$c_0$ ~--- пространство последовательностей, сходящихся к нулю;
	\item
		$c_{00}$ ~--- пространство последовательностей с конечным носителем~\cite[теорема 4]{ASSU2};
	\item
		$ac$ ~--- пространство почти сходящихся последовательностей;
	\item
		$ac_\lambda$ ~--- множество последовательностей, почти сходящихся к $\lambda \in \R$, в частности,
	\item
		$ac_0$ ~--- пространство последовательностей, почти сходящихся к нулю;
	\item
		$\Iac$ ~--- максимальный (по включению) идеал по умножению в пространстве $ac_0$ (см. теорему~\ref{thm:Iac_criterion_pos_neg});
	\item
		$A_0 = \{ x\in\ell_\infty : \alpha(x) = 0\}$ (см. ниже; о свойствах этого пространства см., напр., теорему~\ref{thm:A0_is_space} и далее);
	\item
		$\Omega = \{0;1\}^\N$ ~--- множество последовательностей, состоящих из нулей и единиц.
\end{itemize}



\paragraph{Нормы.}
Запись $\|\cdot\|$ без уточнений будет обозначать норму в пространстве $\ell_\infty$, $\ell_\infty^*$
или в пространстве ограниченных линейных операторов, действующих из $\ell_\infty$ в $\ell_\infty$ (обозначаемом $\mathcal L (\ell_\infty, \ell_\infty)$~)
в зависимости от природы аргумента.
В случае, если нужно ввести другую норму (например, фактор-норму $\ell_\infty / c_0$, как в теореме~\ref{thm:alpha_xy}),
это будет оговорено явно.


\paragraph{Последовательности.}
Для любого $x\in\ell_\infty$ будем по умолчанию полагать, что
\begin{equation}
	x=(x_1, x_2, x_3, x_4, ...)
	.
\end{equation}
Кроме того, нам будет часто требоваться константная единица $\one = (1, 1, 1, 1, ...)$.

Будем писать, что $x\geq 0$, если $x_n \geq 0$ для любого $n\in \N$, и $x\leq 0$, если $x_n \leq 0$ для любого $n\in \N$.
%TODO: периодические последовательности и конкатенация

Вслед за~\cite{hall1992behrend} будем обозначать через $\mathscr{M}A$ множество всех чисел,
кратных элементам множества $A\subset\N$, т.е.
\begin{equation}
	\mathscr{M}A = \{ka: k\in\N, a\in A\}
	,
\end{equation}
через $\chi F$ "--- характеристическую функцию множества $F$.

Так, например,
\begin{gather}
	\chi \mathscr{M}\!A(\{2\}) = \chi \mathscr{M}\!A(\{2, 4\}) = \chi \mathscr{M}\!A(\{2,4,8,16,...\})
	= (0,1,0,1,0,1,0,1,0,...),
\\
	\chi \mathscr{M}\!A(\{3\}) = \chi \mathscr{M}\!A(\{3,9,27,...\}) = (0,0,1,\;0,0,1,\;0,0,1,\;0,0,1,\;0,0,1,\;...),
\\
	\chi \mathscr{M}\!A(\{2,3\}) = \chi \mathscr{M}\!A(\{2,3,6\}) = (0,1,1,1,0,1,\;0,1,1,1,0,1,\;0,1,1,1,0,1,...).
\end{gather}


\paragraph{Операторы.}
Следующие обозначения операторов будут широко использоваться в тексте диссертации.
\begin{itemize}
	\item
		Тождественный оператор:
		\begin{equation}
			I(x_1, x_2, x_3, x_4, ...) = (x_1, x_2, x_3, x_4, ...)
			.
		\end{equation}
	\item
		Оператор сдвига влево:
		\begin{equation}
			T(x_1, x_2, x_3, x_4, ...) = (x_2, x_3, x_4, ...)
			.
		\end{equation}
	\item
		Оператор сдвига вправо:
		\begin{equation}
			U(x_1, x_2, x_3, x_4, ...) = (0, x_1, x_2, x_3, x_4, ...)
			.
		\end{equation}
	\item
		Оператор растяжения ($n\in\N$):
		\begin{equation}
			\sigma_n (x_1, x_2, x_3, ...) = (
				\underbrace{x_1,...,x_1}_{n~\text{раз}},
				\underbrace{x_2,...,x_2}_{n~\text{раз}},
				\underbrace{x_3,...,x_3}_{n~\text{раз}},
				...)
			.
		\end{equation}
	\item
		Оператор усредняющего сжатия ($n\in\N$):
		\begin{equation}
			\sigma_{1/n} x = n^{-1}
			\left(
				\sum_{i=1}^{n} x_i,
				\sum_{i=n+1}^{2n} x_i,
				\sum_{i=2n+1}^{3n} x_i,
				...
			\right).
		\end{equation}
	\item
		Оператор Чезаро (иногда называемый оператором Харди):
		%TODO: ссылки, где это Харди?
		\begin{equation}
			(Cx)_n = n^{-1} \cdot \sum_{k=1}^n x_k
			,
		\end{equation}
		т.е.
		\begin{equation}
			C (x_1, x_2, x_3, ...) = \left(
			x_1,
			\dfrac{x_1+x_2}2,
			\dfrac{x_1+x_2 + x_3}3,
			\dfrac{x_1+x_2+x_3+x_4}4,
			...,
			\dfrac{x_1+...+x_n}n,
			...\right)
			.
		\end{equation}

	\item
		Оператор суперпозиции (покоординатного умножения):
		\begin{equation}
			x \cdot y = (x_1\cdot y_1; ~~x_2\cdot y_2; ~~x_3\cdot y_3; ~...)
		\end{equation}
\end{itemize}


\paragraph{Специальные функции и множества.}

Через $\B$ будем обозначать множество всех банаховых пределов;
через $\B(H)$ ~--- множество всех банаховых пределов, инвариантных относительно оператора $H$.
%TODO: ссылку на теорему/определние?

Через $\ext A$ будем обозначать множество крайних точек множества $A$.

Нижний и верхний функционалы Сачестона соответственно~\cite{sucheston1967banach}:
\begin{equation*}
	q(x) = \lim_{n\to\infty} \inf_{m\in\N}  \frac{1}{n} \sum_{k=m+1}^{m+n} x_k
	~~~~\mbox{и}~~~~
	p(x) = \lim_{n\to\infty} \sup_{m\in\N}  \frac{1}{n} \sum_{k=m+1}^{m+n} x_k
	.
\end{equation*}

Кроме того, мы будем часто пользоваться функцией~\cite{our-vzms-2018}:
\begin{equation}
	\alpha(x) = \varlimsup_{i\to\infty} \max_{i<j\leqslant 2i} |x_i - x_j|
	.
\end{equation}


\section*{Краткое содержание работы}
\addcontentsline{toc}{section}{Краткое содержание работы}
Сходящиеся последовательности, т.е. последовательности, имеющие предел в смысле классического математического анализа,
изучены достаточно хорошо.
В частности, любая сходящаяся последовательность является ограниченной.
Пространство ограниченных последовательностей будем, вслед за классиками \cite{wojtaszczyk1996banach,lindenstrauss1973classical},
обозначать через $\ell_\infty$ и снабжать его нормой
\begin{equation*}
	\|x\| = \sup_{n\in\mathbb{N}}|x_n|
	.
\end{equation*}

Однако в приложениях часто возникают ограниченные последовательности,
которые не являются сходящимися.
В таком случае возникает закономерный вопрос:
как измерить <<недостаток сходимости>>?
<<насколько не сходится>> последовательность?

Наиболее очевидным кажется вычисление расстояния $\rho(x,c)$ от заданного элемента $x\in\ell_\infty$
до пространства сходящихся последовательностей $c$
(которое равно половине разности верхнего и нижнего пределов последовательности).
Однако выясняется, что имеют место быть и другие подходы.

Нетрудно заметить, что операция взятия классического предела на пространстве сходящихся последовательностей
является непрерывным (в норме $\ell_\infty$) линейным функционалом.
В 1929 г. С. Мазур анонсировал~\cite{Mazur}, а позже
С. Банах доказал \cite{banach2001theory_rus}, что этот функционал может быть непрерывно продолжен на всё пространство $\ell_\infty$.
На основе этой идеи были определены банаховы пределы
(иногда также называемые пределами Банаха--Мазура \cite{alekhno2012superposition,alekhno2015banach})
следующим образом.

Банаховым пределом называется функционал $B\in \ell_\infty^*$ такой, что:
\begin{enumerate}
	\item
		$B \geqslant 0$
	\item
		$B\one = 1$
	\item
		$B=BT$
\end{enumerate}

Простейшие свойства:
\begin{itemize}
	\item
		$\|B\|_{\ell_\infty^*} = 1$
	\item
		$Bx = \lim\limits_{n\to\infty} x_n$ для любого $x=(x_1, x_2, ...) \in c$.

		Таким образом,
		банахов предел~--- действительно естественное обобщение понятия предела сходящейся последовательности
		на все ограниченные последовательности.
\end{itemize}

Множество банаховых пределов обычно обозначают через $\mathfrak{B}$
(реже через $BM$~--- см., например, \cite{alekhno2012superposition,alekhno2015banach}).

Лоренц \cite{lorentz1948contribution} установил, что существует подпространство $\ell_\infty$,
на котором все банаховы пределы принимают одинаковое значение.
Это пространство названо пространством почти сходящихся последовательностей и обычно обозначается $ac$
(от англ. <<almost convergent>>).
Включение $c \subset ac$ собственное, т.е. $ac \setminus c \neq \varnothing$.


Обобщая критерий Лоренца, Сачестон \cite{sucheston1967banach} доказал, что для любого $x\in\ell_\infty$
и любого $B\in\mathfrak{B}$
\begin{equation*}
	q(x) =
	\lim_{n\to\infty} \sup_{m\in\mathbb{N}} \frac{1}{n} \sum_{k=m+1}^{k=m+n} x_k
	\leq
	Bx
	\leq
	\lim_{n\to\infty} \inf_{m\in\mathbb{N}} \frac{1}{n} \sum_{k=m+1}^{k=m+n} x_k
	= p(x)
\end{equation*}
и, более того,
\begin{equation*}
	\mathfrak{B}x = [q(x), p(x)]
	.
\end{equation*}


За более подробным обзором ранних исследований банаховых пределов отсылаем читателя к~\cite{greenleaf1969invariant,day1973normed,kangro1976theory}.
%Source: https://encyclopediaofmath.org/wiki/Banach_limit
Вскоре после работ Сачестона Дж. Куртц распространил понятие банаховых пределов
на векторные последовательности~\cite{kurtz1970almost},
а затем и на последовательности в произвольных банаховых пространствах~\cite{kurtz1972almost}.
За обсуждением банаховых пределов в векторных пространствах отсылаем читателя
к~\cite{deeds1968summability,hajdukovic1975almost,armario2013vectorvalued_rus,garcia2015extremal,garcia2016fundamental_rus}.
%Тут есть ещё ссылки: https://www.mathnet.ru/php/archive.phtml?wshow=paper&jrnid=faa&paperid=3146&option_lang=rus
%TODO2: В том числе и про то, где применяется.
В недавней работе~\cite{chen2007characterizations} Ч.~Чен и М.~Куо изучают обобщения банаховых пределов
на произвольные гильбертовы пространства и на пространства суммируемых функций $L_p$.
Другим обобщениям банаховых пределов посвящены работы
\cite{hajdukovic1975functionals,koga2016generalization}.
%tanaka2018banach - иероглифы, несовместимые с библиографией

Ещё одним достаточно плодотворным обобщением банаховых пределов оказались их аналоги на двойных последовательностях~\cite{robison1926divergent}, введённые Дж.~Д.~Хиллом в~\cite{hill1965almost}.
За дальнейшими результатами в этом направлении отсылаем читателя
к~\cite{moricz1988almost,bacsarir1995strong,mursaleen2003almost,edely2004almost,mursaleen2004almost}.
Из недавних работ стоит отдельно отметить статью М. Мурсалена и С.А. Мухиддина~\cite{mursaleen2012banach},
в которой с помощью понятия почти сходимости в пространстве ограниченных двойных последовательностей вводится ряд новых интересных подпространств.

Наконец, если исключить из определения банахова предела требование трансляционной инвариантности,
то мы получим объект, называемый обобщённым пределом,
подробно изучавшийся М.~Джерисоном в~\cite{jerison1957set} и многих других работах.


Таким образом, на вопрос: <<Насколько не сходится последовательность?>> %~---
можно дать ответ в терминах почти сходимости, т.е. принадлежности пространству $ac$,
а на вопрос: <<Насколько почти не сходится последовательность?>>~---
назвать длину отрезка $[q(x), p(x)]$.
В дальнейшем пространство почти сходящихся последовательностей неоднократно становилось предметом
различных исследований
\cite{semenov2006ac,usachev2008transforms}.
В частности, в работе~\cite{connor1990almost} доказано,
что последовательность из нулей и единиц почти наверное не принадлежит пространству $ac$.
Этот факт демонстрирует, что почти сходящиеся последовательности <<достаточно редки>>.

%TODO: ссылки! Хватит или ещё?

Банаховы пределы также нашли своё применение в приложениях
\cite{semenov2015banachtraces,SU,strukova2015spectres}.
%Известны обобщения банаховых пределов на двойные последовательности
%\cite{edely2004almost}.

%TODO: ссылки! Хватит или ещё?


В настоящей работе рассматриваются некоторые вопросы асимптотических характеристик ограниченных последовательностей,
в том числе банаховых пределов.
Нумерация приводимых ниже теорем, лемм, определений и следствий совпадает с их нумерацией в диссертации.


В главе 1 обсуждается пространство $ac$ и его подпространство $ac_0$,
даётся критерий почти сходимости к нулю (т.е. принадлежности пространству $ac_0$)
знакопоcтоянной последовательности.

\reflecttheorem{thm:M_j_ac0_inf_lim}
	Пусть $n_i$~--- строго возрастающая последовательность натуральных чисел,
	\begin{equation}
		\label{eq:definition_M_j}
		M(j) = \liminf_{i\to\infty} n_{i+j} - n_i,
	\end{equation}
	\begin{equation}
		x_k = \left\{\begin{array}{ll}
			1, & \mbox{~если~} k = n_i
			\\
			0  & \mbox{~иначе~}
		\end{array}\right.
	\end{equation}
	Тогда следующие условия эквивалентны:
	\\
	(i)   $x \in ac_0$;
	\\\\
	(ii)  $\lim\limits_{j \to \infty} \dfrac{M(j)}{j} = +\infty$;
	\\\\
	(iii) $\inf\limits_{j \in \N}     \dfrac{M(j)}{j} = +\infty$.


\reflecttheorem{thm:crit_ac0_Mj_lambda}
	Пусть $x\in\ell_\infty$, $x \geq 0$, $\lambda>0$.
	Обозначим через $n^{(\lambda)}_i$ возрастающую последовательность
	индексов таких элементов $x$, что $x_k \geq \lambda$ тогда и только тогда,
	когда $k=n^{(\lambda)}_i$ для некоторого $i$.
	Обозначим
	\begin{equation}
		M^{(\lambda)}(j) = \liminf_{i\to\infty} n^{(\lambda)}_{i+j} - n^{(\lambda)}_i
		.
	\end{equation}


	Тогда для того, чтобы $x\in ac_0$, необходимо и достаточно, чтобы
	для любого $\lambda>0$ было выполнено
	\begin{equation}
		\lim_{j \to \infty} \frac{M^{(\lambda)}(j)}{j} = +\infty
		.
	\end{equation}

\reflecttheorem{thm:rho_x_c_leq_alpha_t_s_x_united}
	Для любого $x\in ac$
	\begin{equation}
		\frac{1}{2} \alpha(x) \leq \rho(x,c)\leq \lim_{s\to\infty} \alpha(T^s x)
		.
	\end{equation}

\reflectcorollary{cor:rho_x_c0_leq_alpha_t_s_x_united}
	Для любого $x\in ac_0$
	\begin{equation}
		\frac{1}{2} \alpha(x) \leq \rho(x,c_0)\leq \lim_{s\to\infty} \alpha(T^s x)
		.
	\end{equation}

\reflecttheorem{thm:Connor_generalized}
	Мера множества $F=\{x\in\Omega : q(x) = 0 \wedge p(x)= 1\}$,
	где $p(x)$ и $q(x)$~--- верхний и нижний функционалы Сачестона соответственно,
	равна 1.


В главе 2
%TODO: \ref ???
изучается $\alpha$--функция, введённая в~\cite{our-vzms-2018}:
\begin{equation}
	\alpha(x) = \varlimsup_{i\to\infty} \max_{i<j\leqslant 2i} |x_i - x_j|
	.
\end{equation}
%
%TODO: ссылка на статью Семенова!
%
Поскольку $\alpha(c)=0$,
то $\alpha$--функцию также можно считать <<мерой несходимости>> последовательности;
равенство $\alpha(x) = 0$, однако, вовсе не гарантирует сходимость.

Устанавливается, что $\alpha$--функция не инвариантна относительно оператора сдвига $T$,
и даётся оценка на $\alpha(T^n x)$.
С другой стороны, $\alpha$--функция, в отличие от некоторых банаховых пределов
\cite{Semenov2010invariant,Semenov2011dan},
инвариантна относительно операторов растяжения $\sigma_n$.
Затем выявляется связь между $\alpha$--функцией, расстоянием от заданнной последовательности до пространства $c$
и почти сходимостью.
Рассмотрены и другие свойства $\alpha$--функции.
Приведём основные результаты.

\reflectcorollary{thm:est_alpha_Tn_x_full}
	Для любых $x\in\ell_\infty$ и $n \in \N$
	\begin{equation}\label{est_alpha_Tn_x}
		\frac{1}{2}\alpha(x) \leq \alpha(T^n x) \leq \alpha(x)
		.
	\end{equation}

\reflecttheorem{thm:alpha_beta_T_seq}
	Пусть $\beta_k$~--- монотонная невозрастающая последовательность,
	$\beta_k \to \beta$, $\beta\in\left[\frac{1}{2}; 1\right]$, $\beta_1 \leq 1$.
	Тогда существует такой $x\in\ell_\infty$, что для любого натурального $n$
	\begin{equation}
		\frac{\alpha(T^n x)}{\alpha(x)} = \beta_n.
	\end{equation}

\reflecttheorem{thm:alpha_sigma_n}
	Для любого $x\in\ell_\infty$ и для любого натурального $n$ верно равенство
	\begin{equation}
		\alpha(\sigma_n x) = \alpha(x)
		.
	\end{equation}

\reflecttheorem{thm:alpha_sigma_1_n}
	Для любого $n\in\N$ и любого $x\in\ell_\infty$ выполнено
	\begin{equation}
		\alpha(\sigma_{1/n} x) \leq \left( 2- \frac{1}{n} \right) \alpha(x)
		.
	\end{equation}

\reflecttheorem{thm:alpha_Cx_no_gamma}
	Имеет место равенство
	\begin{equation}
		\sup_{x\in\ell_\infty, \alpha(x)\neq 0} \frac{\alpha(Cx)}{\alpha(x)}=1
		.
	\end{equation}

\reflecttheorem{thm:alpha_xy}
	Пусть $(x\cdot y)_k = x_k\cdot y_k$.
	Тогда
	$\alpha(x\cdot y)\leq \alpha(x)\cdot \|y\|_* + \alpha(y)\cdot \|x\|_*$,
	где
	\begin{equation}
		\|x\|_* = \limsup_{k\to\infty} |x_k|
	\end{equation}
	есть  фактор-норма по $c_0$ на пространстве $\ell_\infty$.

В параграфе~\ref{sec:space_A0} исследуется пространство $A_0 = \{x: \alpha(x) = 0\}$.
Это пространство несепарабельно, замкнуто относительно покоординатного умножения,
операторов левого и правого сдвигов, оператора Чезаро,
операторов растяжения $\sigma_n$ и усредняющего сжатия $\sigma_{1/n}$.

В параграфе~\ref{sec:noncomplementarity} устанавливается, что в цепочке вложений
\begin{equation}
	c_0 \subset A_0 \subset \ell_\infty
\end{equation}
оба подпространства недополняемы.

Как сказано выше, банаховы пределы по определению (как и обычный предел на пространстве $c$) инвариантны относительно оператора сдвига.
Возникает закономерный вопрос: можно ли потребовать от банахова предела сохранять своё значение
при суперпозиции с некоторыми другими операторами на $\ell_\infty$?
Эту проблему исследовал У. Эберлейн в 1950 г. \cite{Eberlein},
т.е. через два года после классической работы Г. Г. Лоренца~\cite{lorentz1948contribution}.
Эберлейн установил, что существуют такие линейные операторы  $A : \ell_\infty\to \ell_\infty$,
для которых $BAx = Bx$ независимо от выбора $x$ и для банаховых пределов специального вида.

Будем говорить, что $B\in\mathfrak B(A)$, $A : \ell_\infty\to \ell_\infty$, если для любого $x\in \ell_\infty$
выполнено равенство $BAx = Bx$.
Такой банахов предел $B$ называют инвариантным относительно оператора $A$.

Можно ли выделить какие-то особые свойства оператора сдвига,
которые необходимы или достаточны оператору, чтобы относительно него были инвариантны все или некоторые банаховы пределы?
Понятно, что если оператор $A$ таков, что для любого $x\in\ell_\infty$ между $Ax$ и $x$
существует (конечное) расстояние Дамерау--Левенштейна \cite{damerau1964technique} (т.е. минимальное количество операций вставки, удаления, замены и перестановки двух соседних элементов последовательности, необходимых для перевода $x$ в $Ax$, причём для разных $x\in\ell_\infty$ эти операции, вообще говоря, не обязаны быть одинаковыми), то относительно данного оператора инвариантен любой банахов предел. Аналогичное утверждение справедливо и в случае, если $Ax -x \in c_0$ для любого $x\in \ell_\infty$.

Следующим по естественности (после сдвига и замены конечного числа элементов) действием, сохраняющем сходимость последовательности, является повторение элементов последовательности, например, оператор
\begin{equation}
	\sigma_2(x_1,x_2,x_3,...) = (x_1,x_1, \; x_2, x_2, \; x_3, x_3, \; ...)
	.
\end{equation}
Однако относительно такого оператора инвариантны не все, а только некоторые банаховы пределы.
Так, в~\cite[теорема 14]{ASSU2} показано, что
\begin{equation}
	\B(\sigma_n) \cap \ext \B = \varnothing  \mbox{~~для любого~~} n\in\N_2
	.
\end{equation}
Заметим, что если мы рассмотрим оператор неравномерного растяжения
\begin{equation}
	\sigma_{1,2}(x_1,x_2,x_3,x_4,x_5,...) = (x_1, \; x_2, x_2, \;  x_3, \; x_4, x_4, \; x_5, ...)
	,
\end{equation}
то увидим, что периодическую последовательность $y_n = (-1)^n$, $y\in ac_0$ оператор $\sigma_{1,2}$
переводит в периодическую последовательность
\begin{equation}
	(-1, 1, 1, \; -1, 1, 1, \; ...) \in ac_{1/3}
	,
\end{equation}
поскольку на периодической последовательности любой банахов предел принимает значение, равное среднему по периоду.
Таким образом, не существует банаховых пределов, инвариантных относительно оператора $\sigma_{1,2}$.

В главе 3 изложены некоторые примеры операторов и найдены множества банаховых пределов,
инвариантных относительно этих операторов.
Затем рассматриваются следующие классы линейных операторов $H:\ell_\infty \to \ell_\infty$:

-- полуэберлейновы: такие, что $B_1 H \in\B$ для некоторого $B_1\in \B$;

-- эберлейновы: такие, что $B_1 H = B_1$ для некоторого $B_1\in \B$;

-- В-регулярные: такие, что $B_1 H \in \B$ для любого   $B_1\in \B$;

-- существенно эберлейновы: такие, что $B_1 H \in \B$ для любого $B_1\in \B$ и $B_2 H \ne B_2$ для некоторого $B_2\in \B$.

Устанавливается (см. теоремы~\ref{thm:amiable_but_not_Eberlein_exists} и~\ref{thm:Eberlein_but_not_B-regular_exists}),
что каждый следующий из этих классов вложен в предыдущий и не совпадает с ним.
Кроме того, доказывается ещё ряд смежных результатов, в частности, решается обратная задача об инвариантности.

\reflecttheorem{thm:generated_operator_G_B}
	Для каждого $B\in \B$ существует такой оператор $G_B:\ell_\infty \to \ell_\infty$,
	что $\B(G_B) = \{B\}$.


Главы 4 и 5 посвящены верхнему и нижнему функционалам Сачестона $p(x)$ и $q(x)$"--- аналогам верхнего и нижнего пределов последовательности.
В главе 4 изучаются разделяющие множества.

Обозначим через $\Omega$ множество всех последовательностей, состоящих из нулей и единиц.

\reflecttheorem{thm:Lin_Omega_Sucheston}
	Пусть
	$1 \geq a > b \geq 0$ и
	$\Omega^a_b = \{x\in\Omega : p(x) = a, q(x) = b\}$,
	где $p(x)$ и $q(x)$~--- верхний и нижний функционалы Сачестона~\cite{sucheston1967banach} соответственно.
	Тогда $\Omega \subset \operatorname{Lin} \Omega^a_b$.


\reflectcorollary{crl:Lin_Omega_Sucheston}
	Множество $\Omega^a_b$ является разделяющим.
	Т.к. при $a\neq 1$ или $b\neq 0$ множество $\Omega^a_b$ имеет меру нуль~\cite{semenov2010characteristic,connor1990almost},
	то оно является разделяющим множеством нулевой меры.


Пусть $X^a_b = \{x\in\ell_\infty : p(x) = a,~ q(x) = b\}$, $Y^a_b = \{x\in A_0 : p(x) = a, q(x) = b\}$, где $a>b$.
\reflecttheorem{thm:A_0_c_infty_lin}
	Пусть $a\neq -b$.
	Тогда справедливо равенство $\operatorname{Lin} Y^a_b = A_0$.

\reflecttheorem{thm:Lin_ell_infty}
	Справедливо равенство $\operatorname{Lin} X^a_b = \ell_\infty$.

Через $\dim_H E$ будем обозначать хаусдорфову размерность множества $E$.

\reflecttheorem{thm:Hausdorf_measure_1_n}
	Пусть $n\in\N$.
	Тогда существует разделяющее множество $E\subset\Omega$ такое,
	что $\dim_H E = 1/n$.

%Конец главы 4

Глава 5 посвящена связи мультипликативных свойств носителя последовательности из нулей и единиц
и значений, которые могут принимать функционалы Сачестона на такой последовательности.


\reflectcorollary{cor:ac0_powers_finite_set_of_numbers}
	Пусть $\{p_1, ..., p_k\} \subset \N$,
	\begin{equation}
		x_k = \begin{cases}
			1, &\mbox{~если~} k = p_1^{j_1}\cdot p_2^{j_2}\cdot ... \cdot p_k^{j_k} \mbox{~для некоторых~} j_1,...,j_k\in\N,
			\\
			0  &\mbox{~иначе}.
		\end{cases}
	\end{equation}
	Тогда $x\in ac_0$.

\reflectdefinition{def:P-property}
	Будем говорить, что множество $A\subset\N$ обладает $P$-свойством,
	если для любого $n\in\N$ найдётся набор попарно взаимно простых чисел
	\begin{equation}
		\{a_{n,1}, a_{n,2}, ..., a_{n,n}  \} \subset A
		.
	\end{equation}

\reflecttheorem{thm:p_x_infinite_multiples}
	Пусть $A\subset \N\setminus\{1\}$.
	Тогда следующие условия эквивалентны:
	\begin{enumerate}[label=(\roman*)]
		\item
			$A$ обладает $P$-свойством
		\item
			В $A$ существует бесконечное подмножество попарно взаимно простых чисел
		\item
			$p(\chi\mathscr{M}A)=1$.
	\end{enumerate}

\reflectcorollary{cor:ac0_primes_p_psi_A_prod}
	Пусть $A = \{a_1, a_2, ..., a_n,...\}$ "--- бесконечное множество попарно взаимно простых чисел
	и $a_{n+1}>a_1\cdot...\cdot a_n$.
	Тогда
	\begin{equation}
		q(\chi\mathscr{M}A) = 1-\prod_{j=1}^\infty \left(1-\frac{1}{a_j}\right)
		.
	\end{equation}

\reflectlemma{lem:q_x_infinite_Euler}
	Пусть $\varepsilon \in  (0; 1{]}$.
	Существует бесконечное множество попарно непересекающихся подмножеств простых чисел
	$A_i$ такое, что $q(\chi\mathscr{M}A_i)\geq\varepsilon$ для любого $i\in\N$.


\chapter{$\alpha$--функция как асимптотическая характеристика ограниченной последовательности}

	\section{Определение и элементарные свойства $\alpha$--функции}
	На пространстве $\ell_\infty$ определяется $\alpha$--функция следующим равенством:
\begin{equation}
	\alpha(x) = \varlimsup_{i\to\infty} \max_{i<j\leqslant 2i} |x_i - x_j|
	.
\end{equation}
Иногда удобнее использовать одно из нижеследующих равносильных определений:
\begin{equation}
	\alpha(x) = \varlimsup_{i\to\infty} \max_{i \leqslant j\leqslant 2i} |x_i - x_j|
	,
\end{equation}
\begin{equation}
	\alpha(x) = \varlimsup_{i\to\infty} \sup_{i<j\leqslant 2i} |x_i - x_j|
	,
\end{equation}
\begin{equation}
	\alpha(x) = \varlimsup_{i\to\infty} \sup_{i \leqslant j\leqslant 2i} |x_i - x_j|
	.
\end{equation}

Легко видеть, что $\alpha$--функция неотрицательна.
\begin{property}
	\label{thm:alpha_x_triangle_ineq}
	Более того, $\alpha$--функция удовлетворяет неравенству треугольника:
	\begin{equation}
		\alpha(x+y) \leq \alpha(x) + \alpha(y)
		.
	\end{equation}
\end{property}
%TODO2: доказывать нужно или очевидно?

\begin{property}
	На пространстве $\ell_\infty$ $\alpha$--функция удовлетворяет условию Липшица:
	\begin{equation}\label{alpha_Lipshitz}
		|\alpha(x) - \alpha(y)| \leq 2 \|x-y\|
		.
	\end{equation}
	(и эта оценка точна).
%	TODO2: доказывать нужно или очевидно?
\end{property}

\begin{property}
	Если $y\in c$, то $\alpha(y) = 0$ и $\alpha(x+y) = \alpha(x)$ для любого $x \in \ell_\infty$.
\end{property}

Кроме того, иногда полезно помнить про следующее очевидное
\begin{property}
	\label{thm:alpha_x_leq_limsup_minus_liminf}
	\begin{equation}
		\alpha(x) \leq \varlimsup_{k\to\infty} x_k - \varliminf_{k\to\infty} x_k
		.
	\end{equation}
\end{property}

Отдельный интерес представляет множество
\begin{equation}
	A_0 = \{x\in\ell_\infty : \alpha(x) = 0\}
	.
\end{equation}

Ниже в этой главе мы покажем, что оно является подпространством $\ell_\infty$ и обладает рядом интересных свойств.
Одно из этих свойств доказывается уже не в настоящей главе, в а теореме~\ref{thm:A_0_c_infty_lin}.


	\section{$\alpha$--функция и оператор сдвига $T$}
	Результаты этого пункта опубликованы в~\cite{our-ped-2018-alpha-Tx} и используются в~\cite{our-mz2019ac0}.

Как выясняется, $\alpha$--функция не инвариантна относительно оператора сдвига
\begin{equation}
	T(x_1,x_2,x_3,...) = (x_2, x_3, ...).
\end{equation}

\begin{example}
\label{ex:alpha_x_neq_alpha_Tx}
	Пусть
	\begin{equation}
		x_k = \begin{cases}
			(-1)^n, & \mbox{~если~} k = 2^n
			\\
			0 & \mbox{~иначе~}
		\end{cases}
	\end{equation}
\end{example}

Вычислим $\alpha(x)$.
Заметим сначала, что из принадлежности $x_k\in\{-1,0,1\}$
немедленно следует, что $\alpha(x) \leq 2$.
Оценим теперь $\alpha(x)$ снизу:
\begin{multline}
	\alpha(x)
	=
	\varlimsup_{i\to\infty}\max_{i < j \leqslant 2i} |x_i - x_j|
	\geq
	\\\geq
	\mbox{(переход к частичному верхнему пределу
	}\\ \mbox{
	по индексам специального вида $i=2^n$)}
	\geq
	\\\geq
	\varlimsup_{n\to\infty}\max_{2^n < j \leqslant 2^{n+1}} |x_{2^n} - x_j|
	=
	\varlimsup_{n\to\infty}\max_{2^n < j \leqslant 2^{n+1}} |(-1)^n - x_j|
	\geq
	\\ \geq
	\varlimsup_{n\to\infty} |(-1)^n - x_{2^{n+1}}|
	=
	\varlimsup_{n\to\infty} |(-1)^n - (-1)^{n+1}|
	=
	2
	.
\end{multline}

Итак, $\alpha(x) = 2$.
Вычислим теперь $\alpha(Tx)$:
\begin{multline}
	\alpha(Tx)
	=
	\varlimsup_{i\to\infty}~\max_{i < j \leqslant 2i} |(Tx)_i - (Tx)_j|
	=
	\varlimsup_{i\to\infty}~\max_{i < j \leqslant 2i} |x_{i+1} - x_{j+1}|
	=
	\\=
	(\mbox{замена}~k:=i+1, m:=j+1)
	=
	\\=
	\varlimsup_{k\to\infty}~~\max_{k-1 < m-1 \leqslant 2k-2} |x_k - x_m|
	=
	\varlimsup_{k\to\infty}~~\max_{k < m \leqslant 2k-1} |x_k - x_m|
	=
	\\=
	\max\left\{
		\varlimsup_{k\to\infty, k  =   2^n}~~\max_{k < m \leqslant 2k-1} |x_k - x_m|
		,~~
		\varlimsup_{k\to\infty, k \neq 2^n}~~\max_{k < m \leqslant 2k-1} |x_k - x_m|
	\right\}
	=
	\\=
	\max\left\{
		\varlimsup_{n\to\infty}~~\max_{2^n < m \leqslant 2^{n+1}-1} |x_{2^n} - x_m|
		,~~
		\varlimsup_{k\to\infty, k \neq 2^n}~~\max_{k < m \leqslant 2k-1} |x_k - x_m|
	\right\}
	=
	\\=
	\max\left\{
		\varlimsup_{n\to\infty}~~\max_{2^n < m \leqslant 2^{n+1}-1} |(-1)^n - x_m|
		,~~
		\varlimsup_{k\to\infty, k \neq 2^n}~~\max_{k < m \leqslant 2k-1} |x_k - x_m|
	\right\}
	=
	\\=
	\max\left\{
		\varlimsup_{n\to\infty}~~\max_{2^n < m \leqslant 2^{n+1}-1} |(-1)^n - 0|
		,~~
		\varlimsup_{k\to\infty, k \neq 2^n}~~\max_{k < m \leqslant 2k-1} |x_k - x_m|
	\right\}
	=
	\\=
	\max\left\{
		1
		,~
		\varlimsup_{k\to\infty, k \neq 2^n}~~\max_{k < m \leqslant 2k-1} |x_k - x_m|
	\right\}
	=
	\\=
	\mbox{(если $k \neq 2^n$, то $x_k = 0$)}
	=
	\\=
	\max\left\{
		1
		,~
		\varlimsup_{k\to\infty, k \neq 2^n}~~\max_{k < m \leqslant 2k-1} |0 - x_m|
	\right\}
	=
	1
	.
\end{multline}
Таким образом, $\alpha(Tx) = 1 \neq 2 = \alpha(x)$,
что и требовалось показать.

Верна следующая
\begin{theorem}
	Для любого $x \in \ell_\infty$ выполнено неравенство $\alpha(Tx)\leq \alpha(x)$.
\end{theorem}

\begin{proof}
	\begin{multline}
		\alpha(Tx)
		=
		\varlimsup_{i\to\infty}~\max_{i < j \leqslant 2i} |(Tx)_i - (Tx)_j|
		=
		\varlimsup_{i\to\infty}~\max_{i < j \leqslant 2i} |x_{i+1} - x_{j+1}|
		=
		\\=
		(\mbox{замена}~k:=i+1, m:=j+1)
		=
		\\=
		\varlimsup_{k\to\infty}~~\max_{k-1 < m-1 \leqslant 2k-2} |x_k - x_m|
		=
		\varlimsup_{k\to\infty}~~\max_{k < m \leqslant 2k-1} |x_k - x_m|
		\leq
		\\ \leq
		\mbox{(переход к максимуму по большему множеству)}
		\leq
		\\ \leq
		\varlimsup_{k\to\infty}~~\max_{k < m \leqslant 2k} |x_k - x_m|
		=
		\alpha(x)
		.
	\end{multline}
\end{proof}

Более интересна, однако, следующая оценка.
\begin{theorem}
	Для любого $n\in\N$
	\begin{equation}
		\alpha(T^n x) \geq \frac{1}{2} \alpha(x)
		.
	\end{equation}
\end{theorem}

\begin{proof}
	Зафиксируем $n$.
	Заметим, что
	\begin{equation}
		\alpha(x) = \varlimsup_{i\to\infty} \alpha_i(x),
	\end{equation}
	где
	\begin{equation}
		\alpha_i(x) = \max_{i < j \leqslant 2i} |x_i - x_j|.
	\end{equation}

	Рассмотрим $\alpha_i(x)$ при некотором фиксированном $i$, $i>2n$ (меньшие $i$ не влияют на верхний предел).
	Если
	\begin{equation}
		\max_{i < j \leqslant 2i} |x_i - x_j|
		=
		\max_{i < j \leqslant 2i-n} |x_i - x_j|
		,
	\end{equation}
	то
	\begin{multline}\label{alpha_i(x)_leq_alpha_{i-n}(T_n x)}
		\alpha_i(x)
		=
		\max_{i < j \leqslant 2i} |x_i - x_j|
		=
		\max_{i < j \leqslant 2i-n} |x_i - x_j|
		=
		\\=
		\mbox{(замена $k=i-n$, $m=j-n$)}
		=
		\\=
		\max_{k+n < m+n \leqslant 2k+n} |x_{k+n} - x_{m+n}|
		=
		\max_{k < m \leqslant 2k} |(T^n x)_k - (T^n x)_m|
		=
		\alpha_{i-n}(T^n x)
		.
	\end{multline}
	Иначе
	\begin{equation}
		\max_{i < j \leqslant 2i} |x_i - x_j|
		=
		\max_{2i-n < j \leqslant 2i} |x_i - x_j|
	\end{equation}
	и можно записать, что
	\begin{multline}\label{alpha_i(x)_leq_alpha_{i-n}(T_n x) + alpha_{2i-2n}(T_n x)}
		\alpha_i(x)
		=
		\max_{2i-n < j \leqslant 2i} |x_i - x_j|
		=
		\max_{2i-n < j \leqslant 2i} |x_i - x_{2i-n} + x_{2i-n} - x_j|
		\leq
		\\\leq
		\max_{2i-n < j \leqslant 2i} \left( |x_i - x_{2i-n}| + |x_{2i-n} - x_j| \right)
		=
		\\=
		|x_i - x_{2i-n}| + \max_{2i-n < j \leqslant 2i} |x_{2i-n} - x_j|
		=
		\\=
		|x_{i-n+n} - x_{2(i-n)+n}| + \max_{2i-n < j \leqslant 2i} |x_{2i-n} - x_j|
		=
		\\=
		|(T^n x)_{i-n} - (T^n x)_{2(i-n)}| + \max_{2i-n < j \leqslant 2i} |x_{2i-n} - x_j|
		\leq
		\\ \leq
		\alpha_{i-n}(T^n x) + \max_{2i-n < j \leqslant 2i} |x_{2i-n} - x_j|
		\leq
		\\ \leq
		\alpha_{i-n}(T^n x) + \max_{2i-n < j \leqslant 2i} |(T^n x)_{2i-2n} - (T^n x)_{j-n}|
		=
		\\=
		(\mbox{замена}~ m:=j-n ~)
		=
		\\=
		\alpha_{i-n}(T^n x) + \max_{2i-2n < m \leqslant 2i-n} |(T^n x)_{2i-2n} - (T^n x)_m|
		\leq
		\\ \leq
		\mbox{(т.к. $i>2n$, то $4i-4n > 2i+4n-4n = 2i > 2i-n$)}
		\leq
		\\ \leq
		\alpha_{i-n}(T^n x) + \max_{2i-2n < m \leqslant 4i-4n} |(T^n x)_{2i-2n} - (T^n x)_m|
		=
		\alpha_{i-n}(T^n x) + \alpha_{2i-2n}(T^n x)
		.
	\end{multline}

	Сравнивая \eqref{alpha_i(x)_leq_alpha_{i-n}(T_n x)} и \eqref{alpha_i(x)_leq_alpha_{i-n}(T_n x) + alpha_{2i-2n}(T_n x)},
	делаем вывод, что
	\begin{equation}
		\alpha_i(x) \leq \alpha_{i-n}(T^n x) + \alpha_{2i-2n}(T^n x)
		.
	\end{equation}
	Переходя к верхнему пределу, имеем
	\begin{multline}
		\alpha(x)
		=
		\varlimsup_{i\to\infty} \alpha_i(x)
		\leq
		\varlimsup_{i\to\infty} (\alpha_{i-n}(T^n x) + \alpha_{2i-2n}(T^n x))
		\leq
		\\ \leq
		\varlimsup_{i\to\infty} \alpha_{i-n}(T^n x) + \varlimsup_{i\to\infty} \alpha_{2i-2n}(T^n x)
		=
		\\=
		\varlimsup_{j=i-n, j\to\infty} \alpha_{j}(T^n x) + \varlimsup_{i\to\infty} \alpha_{2i-2n}(T^n x)
		=
		\\=
		\alpha(T^n x) + \varlimsup_{i\to\infty} \alpha_{2i-2n}(T^n x)
		\leq
		\\ \leq
		\mbox{(верхний предел по индексам специального вида
		} \\ \mbox{
		заменим на верхний предел по всем индексам)}
		\leq
		\\ \leq
		\alpha(T^n x) + \varlimsup_{k\to\infty} \alpha_{k}(T^n x)
		=
		\alpha(T^n x) + \alpha(T^n x)
		=
		2 \alpha(T^n x)
		.
	\end{multline}
	Таким образом, $\alpha(T^n x) \geq \frac{1}{2} \alpha(x)$,
	что и требовалось доказать.
\end{proof}

Две предыдущие теоремы немедленно влекут

\begin{corollary}
	\label{thm:est_alpha_Tn_x_full}
	Для любых $x\in\ell_\infty$ и $n \in \N$
	\begin{equation}\label{est_alpha_Tn_x}
		\frac{1}{2}\alpha(x) \leq \alpha(T^n x) \leq \alpha(x)
		.
	\end{equation}
\end{corollary}

\begin{remark}
	Оценки \eqref{est_alpha_Tn_x} точны: нижняя достигается, например,
	в примере выше, верхняя же~--- на любой периодической последовательности.
\end{remark}


	\section{О характере сходимости последовательности $\alpha(T^n x)/ \alpha(x)$}
	В зависимости от выбора $x$ последовательность $\left\{\frac{\alpha(T^n x)}{\alpha(x)}\right\}$
может монотонно сходиться к любому числу из отрезка $\left[\frac{1}{2}; 1\right]$ с любой скоростью
(в том числе и сколь угодно медленно).
Говоря строже, верна следующая

\begin{theorem}
	\label{thm:alpha_beta_T_seq}
	Пусть $\beta_k$~--- монотонная невозрастающая последовательность,
	$\beta_k \to \beta$, $\beta\in\left[\frac{1}{2}; 1\right]$, $\beta_1 \leq 1$.
	Тогда существует такой $x\in\ell_\infty$, что для любого натурального $n$
	\begin{equation}
		\frac{\alpha(T^n x)}{\alpha(x)} = \beta_n.
	\end{equation}
\end{theorem}

\begin{proof}
	Для удобства обозначим $\beta_0 = 1$.
	Это логично, так как
	\begin{equation}
		\frac{\alpha(T^0 x)}{\alpha(x)} = \frac{\alpha(x)}{\alpha(x)} = 1
		.
	\end{equation}

	Пусть $m\geq 3$, $m\in\N$.
	Положим
	\begin{equation}
		x_k = \begin{cases}
			0,  & \mbox{если } k = 2^{2m}     \\
			\beta_l,  & \mbox{если } 2^{2m} < k = 2^{2m+1}-l, l\in\N\cup\{0\}     \\
			\dfrac{1}{2}                    & \mbox{иначе.}
		\end{cases}
	\end{equation}

	Тогда $\alpha(x) = 1$.
	Действительно, $\left| x_{2^{2m}} - x_{2^{2m+1}} \right| =1$,
	а по свойству \ref{thm:alpha_x_leq_limsup_minus_liminf} $\alpha(x) \leq 1$.

	Пусть
	\begin{equation}
		\alpha_{i,n}(x)= \max_{i< j \leq 2i - n} |x_i - x_j|
		,
		\quad
		i>n
		,
	\end{equation}
	тогда
	\begin{equation}
		\alpha(T^n x) = \varlimsup_{i \to \infty} \alpha_{i,n}(x)
		.
	\end{equation}

	Вычислим теперь все $\alpha_{i,n}(x)$.
	Заметим, что
	\begin{multline}
		\alpha_{2^{2m}, n} (x)
		=
		\max_{2^{2m}< j \leq 2^{2m+1} - n} |0 - x_j|
		=
		\max_{2^{2m}< j \leq 2^{2m+1} - n} x_j
		=
		\\=
		\mbox{(замена: $q = 2^{2m+1} - j$, тогда $j = 2^{2m+1} - q$)}
		=
		\\=
		\max_{2^{2m}< 2^{2m+1} - q \leq 2^{2m+1} - n} x_{2^{2m+1} - q}
		=
		\max_{0< 2^{2m} - q \leq 2^{2m} - n} x_{2^{2m+1} - q}
		=
		\max_{n \leq q < 2^{2m}} x_{2^{2m+1} - q}
		=
		\\=
		\max_{n \leq q < 2^{2m}} \beta_q
		=
		%\\=
		\mbox{(в силу невозрастания $\beta_q$)}
		=
		\beta_n
		\geq
		\frac{1}{2}
		.
	\end{multline}

	Пусть теперь $i$ таково, что $2^{2m+1}<i<2^{2m+2}$,
	тогда $x_i = \frac{1}{2}$.
	Так как
	$\forall(j)\left[x_j\in[0;1]\right]$,
	то
	$|x_i - x_j| \leq \frac{1}{2}$
	и, следовательно,
	$\alpha_{i,n}(x)  \leq \frac{1}{2}$.

	Пусть, наконец, $i$ таково, что $2^{2m}<i \leq 2^{2m+2}$,
	тогда
	$x_i = \beta_k \in [1/2;1]$.
	Для таких $j$, что $i<j\leq 2i-n$ и, более того,
	для любых таких $j$, что $2^{2m}<j<2^{2m+2}$
	выполнено $x_j\in[1/2; 1]$
	и, значит, снова $|x_i - x_j| \leq \frac{1}{2}$.

	Таким образом, получаем, что
	\begin{equation}
		\alpha(T^n x) = \varlimsup_{i \to \infty} \alpha_{i,n}(x) = \beta_n
		.
	\end{equation}
\end{proof}


	\section{О множествах $\{x: \alpha(T^n x) = \alpha(x)\}$}
	В данном параграфе обсуждаются некоторые свойства множеств
\begin{equation}
	\label{eq:alpha_T^n_x_equiv_alpha_x}
	\{x \in \ell_\infty : \alpha(T^n x) = \alpha(x) \}, ~n\in\N,
\end{equation}
\begin{equation}
	\label{eq:cap_alpha_T^n_x_equiv_alpha_x}
	\bigcap\limits_{n\in\N}\{x \in \ell_\infty : \alpha(T^n x) = \alpha(x) \}
	,
\end{equation}
\begin{equation}
	\label{eq:cup_alpha_T^n_x_equiv_alpha_x}
	\bigcup_{n\in\N}\{x \in \ell_\infty : \alpha(T^n x) = \alpha(x) \}
	.
\end{equation}

\subsection{Аддитивные свойства}

\begin{theorem}
	Ни одно из множеств
	\eqref{eq:alpha_T^n_x_equiv_alpha_x}, \eqref{eq:cap_alpha_T^n_x_equiv_alpha_x}, \eqref{eq:cup_alpha_T^n_x_equiv_alpha_x}
	не замкнуто по сложению и, следовательно, не является пространством.
\end{theorem}

\begin{proof}
	Построим два таких элемента, принадлежащих множеству \eqref{eq:alpha_T^n_x_equiv_alpha_x} при любых $n\in\N$,
	сумма которых не принадлежит множеству \eqref{eq:alpha_T^n_x_equiv_alpha_x} ни при каких $n\in\N$.
	Пусть $m\in\N_3$.
	Положим

	\begin{equation}
		x_k = \begin{cases}
			\dfrac{1}{2}(-1)^m,  & \mbox{если } k = 2^m     \\
			1,                   & \mbox{если } k = 2^m + 1 \\
			-1,                  & \mbox{если } k = 2^m + 2 \\
			0                    & \mbox{иначе }
		\end{cases}
	\end{equation}

	и

	\begin{equation}
		y_k = \begin{cases}
			\dfrac{1}{2}(-1)^m,  & \mbox{если } k = 2^m     \\
			-1,                  & \mbox{если } k = 2^m + 1 \\
			1,                   & \mbox{если } k = 2^m + 2 \\
			0                    & \mbox{иначе }
		\end{cases}
	\end{equation}

	Так как
	\begin{equation}
		(T^n x)_{2^m-n+1} - (T^n x)_{2^m-n+2} = 2
	\end{equation}
	и
	\begin{equation}
		(T^n y)_{2^m-n+1} - (T^n y)_{2^m-n+2} = -2
		,
	\end{equation}
	то
	\begin{equation}
		\alpha(x) = \alpha(T^n x) = \alpha(y) = \alpha(T^n y) = 2
		.
	\end{equation}
	С другой стороны,
	\begin{equation}
		(x+y)_k = \begin{cases}
			(-1)^m,  & \mbox{если } k = 2^m     \\
			0        & \mbox{иначе }
		\end{cases}
	\end{equation}
	и
	\begin{equation}
		\alpha(x+y) = 2
		,
	\end{equation}
	но в то же время
	\begin{equation}
		\alpha(T^n(x+y)) = 1
	\end{equation}
	(см. пример \ref{ex:alpha_x_neq_alpha_Tx}),
	следовательно, $x+y$ не принадлежит ни одному из множеств
	\eqref{eq:alpha_T^n_x_equiv_alpha_x}, \eqref{eq:cap_alpha_T^n_x_equiv_alpha_x}, \eqref{eq:cup_alpha_T^n_x_equiv_alpha_x}.
\end{proof}




\subsection{Мультипликативные свойства}

\begin{theorem}
	Ни одно из множеств
	\eqref{eq:alpha_T^n_x_equiv_alpha_x}, \eqref{eq:cap_alpha_T^n_x_equiv_alpha_x}, \eqref{eq:cup_alpha_T^n_x_equiv_alpha_x}
	не замкнуто по умножению.
\end{theorem}

\begin{proof}
	Снова построим два таких элемента, принадлежащих множеству \eqref{eq:alpha_T^n_x_equiv_alpha_x} при любых $n\in\N$,
	произведение которых не принадлежит множеству \eqref{eq:alpha_T^n_x_equiv_alpha_x} ни при каких $n\in\N$.
	Пусть $m\in\N_3$.
	Положим

	\begin{equation}
		x_k = \begin{cases}
			(-1)^m,  & \mbox{если } k = 2^m     \\
			1,                   & \mbox{если } k = 2^m + 1 \\
			0                    & \mbox{иначе }
		\end{cases}
	\end{equation}

	и

	\begin{equation}
		y_k = \begin{cases}
			(-1)^{m+1},  & \mbox{если } k = 2^m     \\
			1,                   & \mbox{если } k = 2^m + 2 \\
			0                    & \mbox{иначе }
		\end{cases}
	\end{equation}

	Так как
	\begin{equation}
		(T^n x)_{2^{2m+1}-n} - (T^n x)_{2^{2m+1}-n+1} = -2
	\end{equation}
	и
	\begin{equation}
		(T^n y)_{2^{2m}-n} - (T^n y)_{2^{2m}-n+2} = -2
		,
	\end{equation}
	то
	\begin{equation}
		\alpha(x) = \alpha(T^n x) = \alpha(y) = \alpha(T^n y) = 2
		.
	\end{equation}
	С другой стороны,
	\begin{equation}
		(x\cdot y)_k = \begin{cases}
			(-1)^m,  & \mbox{если } k = 2^m     \\
			0        & \mbox{иначе }
		\end{cases}
	\end{equation}
	и
	\begin{equation}
		\alpha(x+y) = 2
		,
	\end{equation}
	но в то же время
	\begin{equation}
		\alpha(T^n(x \cdot y)) = 1
	\end{equation}
	(см. пример \ref{ex:alpha_x_neq_alpha_Tx}),
	следовательно, $x \cdot y$ не принадлежит ни одному из множеств
	\eqref{eq:alpha_T^n_x_equiv_alpha_x}, \eqref{eq:cap_alpha_T^n_x_equiv_alpha_x}, \eqref{eq:cup_alpha_T^n_x_equiv_alpha_x}.

\end{proof}


	\section{$\alpha$--функция и семейство операторов $\sigma_n$}
	\begin{theorem}
	\label{thm:alpha_sigma_n}
	Для любого $x\in\ell_\infty$ и для любого натурального $n$ верно равенство
	\begin{equation}
		\alpha(\sigma_n x) = \alpha(x)
		.
	\end{equation}
\end{theorem}

\begin{proof}
	По определению
	\begin{equation}
		\alpha(x) = \varlimsup_{i\to\infty} \max_{i<j\leqslant 2i} |x_i - x_j|
	\end{equation}

	Положим
	\begin{equation}
		\alpha_i(x) =
		\max_{i<j\leqslant 2i} |x_i - x_j| =
		\max_{i\leqslant j\leqslant 2i} |x_i - x_j|
	\end{equation}

	Тогда
	\begin{equation}
		\alpha(x) = \varlimsup_{i\to\infty} \alpha_i(x)
	\end{equation}

	Пусть $y = \sigma_n x$.
	Тогда для $k=1, ..., n-1$, $a\in\N$ имеем
	\begin{multline}
		\alpha_{an-k}(y) =
		\max_{an-k \leqslant j \leqslant 2an-2k} |y_{an-k} - y_j| =
		\\=
		(\mbox{т.к.}~y_{an-(n-1)}=y_{an-(n-2)}=...=y_{an-k}=...=y_{an-1}=y_{an})=
		\\=
		\max_{an \leqslant j \leqslant 2an-2k} |y_{an} - y_j| \leqslant
		\\ \leqslant
		(\mbox{переходим к максимуму по большему множеству}) \leqslant
		\\ \leqslant
		\max_{an \leqslant j \leqslant 2an} |y_{an} - y_j| =
		\alpha_{an}(y)
	\end{multline}

	С другой стороны,
	\begin{multline}
		\alpha_{an}(y) =
		\max_{an \leqslant j \leqslant 2an} |y_{an} - y_j| =
		\\ =
		(\mbox{т.к.}~y=\sigma_n x,~\mbox{можем рассматривать только}~j=kn)=
		\\ =
		\max_{an \leqslant kn \leqslant 2an} |y_{an} - y_{kn}| =
		\max_{a \leqslant k \leqslant 2a} |y_{an} - y_{kn}| =
		\max_{a \leqslant k \leqslant 2a} |x_a - x_k| =
		\alpha_a(x)
	\end{multline}

	Таким образом, для $k=1, ..., n-1$, $a\in\N$ имеем соотношения:
	\begin{gather}
		\alpha_{an}(y) = \alpha_a(x),
	\\
		\alpha_{an-k}(y) \leqslant \alpha_a(x),
	\end{gather}
	откуда немедленно следует, что
	\begin{equation}
		\varlimsup_{i\to\infty} \alpha_i(y) =
		\varlimsup_{i\to\infty} \alpha_i(x),
	\end{equation}
	т.е.
	\begin{equation}
		\alpha(\sigma_n x) = \alpha(x),
	\end{equation}
	что и требовалось доказать.
\end{proof}

В статье \cite[lemma 16]{Semenov2010invariant} доказывается, что
\begin{equation}
	\sigma_2 C - C \sigma_2 : \ell_\infty \to c_0
	.
\end{equation}

\begin{corollary}
	$$
		\alpha(C\sigma_2 x) =
		\alpha(\sigma_2 Cx) =
		\alpha(Cx)
	$$
\end{corollary}

\begin{hypothesis}
	Для любого $n\in\N$
	$$
		\alpha(C\sigma_n x) =
		\alpha(\sigma_n Cx) =
		\alpha(Cx)
	$$
\end{hypothesis}


	\section{$\alpha$--функция и семейство операторов $\sigma_{1/n}$}
	Введём, следуя~\cite[p. 131, prop. 2.b.2]{lindenstrauss1979classical},
на $\ell_\infty$ оператор
\begin{equation}
	\sigma_{1/n} x = n^{-1}
	\left(
		\sum_{i=1}^{n} x_i,
		\sum_{i=n+1}^{2n} x_i,
		\sum_{i=2n+1}^{3n} x_i,
		...
	\right).
\end{equation}

Понятно, что если последовательность $x$~--- периодическая с периодом $n$,
то $\alpha(\sigma_{1/n}x)=0$.
Значит, оценить $\alpha(\sigma_{1/n}x)$ снизу через $\alpha(x)$ не удастся.

Для построения верхней оценки нам потребуется следующая

\begin{lemma}
	\label{thm:distance_from_average}
	Пусть $a=\frac{a_1+...+a_n}{n}$, $a_1 \leq ... \leq a_n$.
	Тогда $a_n - a \leq \frac{n-1}{n} (a_n - a_1)$.
\end{lemma}

\begin{proof}
	\begin{multline}
		a_n - a = a_n - \frac{a_1+...+a_n}{n}
		=
		\frac{n-1}{n}a_n - \frac{a_1+...+a_{n-1}}{n}
		\leq
		\\\leq
		\frac{n-1}{n}a_n - \frac{(n-1)a_1}{n}
		=
		\frac{n-1}{n}(a_n - a_1)
		.
	\end{multline}
\end{proof}

\begin{theorem}
	\label{thm:alpha_sigma_1_n}
	Для любого $n\in\N$ и любого $x\in\ell_\infty$ выполнено
	\begin{equation}
		\alpha(\sigma_{1/n} x) \leq \left( 2- \frac{1}{n} \right) \alpha(x)
		.
	\end{equation}
\end{theorem}

\begin{proof}
	Положим
	\begin{equation}
		\alpha_i(x) =
		\max_{i<j\leqslant 2i} |x_i - x_j| =
		\max_{i\leqslant j\leqslant 2i} |x_i - x_j|
		.
	\end{equation}
	Тогда
	\begin{equation}
		\alpha(x) = \varlimsup_{i\to\infty} \alpha_i(x)
		.
	\end{equation}
	Пусть $y=\sigma_n \sigma_{1/n} x$.
	Из теоремы~\ref{thm:alpha_sigma_n} следует, что $\alpha(\sigma_{1/n} x)=\alpha(y)$.
	Сосредоточим наши усилия на оценке $\alpha(y)$.

	Пусть $1\leq j \leq n$.
	Заметим, что
	\begin{multline}
		\alpha_{kn+j}(y)
		=
		\max_{kn+j \leq i \leq 2kn+2j } |y_{kn+j} - y_i|
		=
		\\=
		\mbox{(т.к. $y_{kn+j}=y_{kn+1}=y_{kn+n} = (\sigma_{1/n}x)_k$)}
		=
		\\=
		\max_{kn+n \leq i \leq 2kn+2j } |y_{kn+n} - y_i|
		\leq
		\\\leq
		\mbox{(переходим к максимуму по не меньшему множеству)}
		\leq
		\\\leq
		\max_{kn+n \leq i \leq 2kn+2n } |y_{kn+n} - y_i|
		=
		\alpha_{kn+n}(y)
		.
	\end{multline}

	Итак, $\alpha_{kn+j}(y) \leq \alpha_{kn+n}(y)$,
	значит,
	\begin{equation}
		\label{eq:alpha_sigma_1_n_subseq_limsup}
		\alpha(\sigma_{1/n}x) = \alpha(y) = \varlimsup_{i\to\infty} \alpha_i(y)
		=
		\varlimsup_{k\to\infty} \alpha_{kn+n}(y)
		.
	\end{equation}

	По лемме~\ref{thm:distance_from_average} имеем
	\begin{multline}
		\label{eq:alpha_sigma_1_n_distance}
		|x_{kn+n}-y_{kn+n}|
		\leq
		\frac{n-1}{n}\max_{1\leq i<j \leq n}|x_{kn+i}-x_{kn+j}|
		\leq
		\\\leq
		\frac{n-1}{n} \max_{1\leq i \leq n} \alpha_{kn+i}(x)
		=
		\frac{n-1}{n}\alpha_{kn+i_k}(x)
		.
	\end{multline}

	Из того, что $y_j = \frac{1}{n}(x_{kn+1}+...+x_{kn+n})$,
	следует, что
	\begin{equation}
		\label{eq:alpha_sigma_1_n_alpha_x}
		\max_{kn+n \leq i \leq 2kn+2n } |x_{kn+n} - y_i|
		\leq
		\max_{kn+n \leq i \leq 2kn+2n } |x_{kn+n} - x_i|
		=
		\alpha_{kn+n}(x)
		.
	\end{equation}

	Оценим:
	\begin{multline}
		\alpha_{kn+n}(y)
		=
		\max_{kn+n \leq i \leq 2kn+2n } |y_{kn+n} - y_i|
		=
		\\=
		\max_{kn+n \leq i \leq 2kn+2n } |y_{kn+n} - x_{kn+n} + x_{kn+n} - y_i|
		\leq
		\\\leq
		|y_{kn+n} - x_{kn+n}| + \max_{kn+n \leq i \leq 2kn+2n } |x_{kn+n} - y_i|
		\mathop{\leq}^{\eqref{eq:alpha_sigma_1_n_distance}}
		\\\leq
		\frac{n-1}{n} \alpha_{kn+i_k}(x) + \max_{kn+n \leq i \leq 2kn+2n } |x_{kn+n} - y_i|
		\mathop{\leq}^{\eqref{eq:alpha_sigma_1_n_alpha_x}}
		\frac{n-1}{n} \alpha_{kn+i_k}(x)+\alpha_{kn+n}(x)
		.
	\end{multline}

	С учётом~\eqref{eq:alpha_sigma_1_n_subseq_limsup} имеем
	\begin{multline}
		\alpha(\sigma_{1/n}x)
		=
		\varlimsup_{k\to\infty} \alpha_{kn+n}(y)
		\leq
		\\\leq
		\varlimsup_{k\to\infty} \left( \frac{n-1}{n} \alpha_{kn+i_k}(x)+\alpha_{kn+n}(x) \right)
		=
		\left(2-\frac{1}{n}\right)\alpha(x)
		.
	\end{multline}
\end{proof}

Точность теоремы~\ref{thm:alpha_sigma_1_n} для $n=1$ очевидна.
Для $n=2$ её показывает
\begin{example}
	Положим для всех $p\in\N$:
	\begin{equation}
		x_k=\begin{cases}
			0, & k \leq 2^3, \\
			0, & k = 2^{3p}+1, \\
			1, & k = 2^{3p}+2, \\
			1, & 2^{3p}+3 \leq k \leq 2^{3p+1}+2, \\
			2, & 2^{3p+1}+3 \leq k \leq 2^{3p+1}+4, \\
			1, & 2^{3p+1}+5 \leq k \leq 2^{3(p+1)},
		\end{cases}
	\end{equation}
	тогда
	\begin{equation}
		(\sigma_{1/2}x)_k=\begin{cases}
			0, & k \leq 2^3, \\
			1/2, & k = 2^{3p}+1, \\
			1/2, & k = 2^{3p}+2, \\
			1, & 2^{3p}+3 \leq k \leq 2^{3p+1}+2, \\
			2, & 2^{3p+1}+3 \leq k \leq 2^{3p+1}+4, \\
			1, & 2^{3p+1}+5 \leq k \leq 2^{3(p+1)}.
		\end{cases}
	\end{equation}
	%\begin{table}
	%	\begin{tabular}{c||c|c|c|c|c|c|}
	%		\hline
	%		$k$     & $0..2^3$ & $2^{3p}+1$ & $2^{3p}+2$ & $2^{3p}+3 .. 2^{3p+1}+2$ & $2^{3p+1}+3 .. 2^{3p+1}+4 $ & $2^{3p+1}+5 .. 2^{3(p+1)}$ \\
	%	\end{tabular}
	%\end{table}
	Очевидно, что $\alpha(x)=1$, но $\alpha(\sigma_{1/2}x)=3/2$
	(достигается на $i=2^{3p}+2$, $j=2^{3p+1}+4$).
\end{example}

\begin{hypothesis}
	Оценка теоремы~\ref{thm:alpha_sigma_1_n} точна для любого $n\in\N$.
\end{hypothesis}


	\section{$\alpha$--функция и оператор Чезаро $C$}
	Ниже приводится расширенная версия материала, опубликованного в
\cite{our-vzms-2018}.

На пространстве ограниченных последовательностей $\ell_\infty$ определяется оператор Чезаро $C$
равенством
\begin{equation}
	(Cx)_n = {1}/{n} \cdot \sum_{k=1}^n x_k
	.
\end{equation}

Можно доказать, что верна
\begin{theorem}
	\label{thm:alpha_Cx_leq_alpha_x}
	%TODO: ссылка?
	$\alpha(Cx) \leqslant \alpha(x)$.
\end{theorem}
Выясняется, что эта оценка достаточно точна.

\subsection{Вспомогательная сумма специального вида}
\begin{lemma}
	Если $p\geq 2$, то
	\begin{equation}\label{summa_drobey}
		\sum_{i=0}^{p-1} \frac{i \cdot 2^i}{p} = \frac{2^p(p-2) + 2}{p}
	\end{equation}
\end{lemma}

% В Демидовиче этого не нашёл

\paragraph{Доказательство.}
Равенство \eqref{summa_drobey} равносильно равенству
\begin{equation}\label{summa_drobey_multiplied}
	\sum_{i=0}^{p-1} i \cdot 2^i = 2^p(p-2) + 2
	.
\end{equation}
Докажем это равенство методом математической индукции.

\paragraph{База индукции.}
Для $p=2$ имеем
\begin{equation}
	\sum_{i=0}^{2-1} i \cdot 2^i = 0 \cdot 2^0 + 1 \cdot 2^1 = 2
\end{equation}
и
\begin{equation}
	2^2(2-2) + 2 = 2
	.
\end{equation}
Видим, что для $p=2$ соотношение \eqref{summa_drobey_multiplied} выполняется.

\paragraph{Шаг индукции.}
Пусть соотношение \eqref{summa_drobey_multiplied} выполняется для $p=m$, $m\geq 2$, т.е.
\begin{equation}\label{summa_drobey_multiplied_m}
	\sum_{i=0}^{m-1} i \cdot 2^i = 2^m(m-2) + 2
	.
\end{equation}

Покажем, что тогда соотношение \eqref{summa_drobey_multiplied} выполняется и для $p=m+1$.
Действительно,
\begin{multline}
	\sum_{i=0}^{(m+1)-1} i \cdot 2^i
	=
	\sum_{i=0}^{m} i \cdot 2^i
	=
	\sum_{i=0}^{m - 1} i \cdot 2^i + m\cdot 2 ^m
	\mathop{=}^{\eqref{summa_drobey_multiplied_m}}
	2^m(m-2) + 2 + m\cdot 2 ^m
	=
	\\=
	m\cdot2^m-2\cdot2^m  + 2 + m\cdot 2 ^m
	=
	m\cdot2^{m+1}-2^{m+1}  + 2
	=
	\\=
	2^{m+1}(m-1)  + 2
	=
	2^{m+1}((m+1)-1)  + 2
	,
\end{multline}
т.е. соотношение \eqref{summa_drobey_multiplied} выполняется и для $p=m+1$,
что и требовалось доказать.



\subsection{Вспомогательный оператор $S$}
Пусть $y\in \ell_\infty$.
Определим оператор $S:\ell_\infty \to \ell_\infty$ следующим образом:
\begin{equation}\label{operator_S}
	(Sy)_k = y_{i+2}, \mbox{ где } 2^i < k \leq 2^i+1
\end{equation}
Этот оператор вводится исключительно для упрощения изложения конструкции.

\begin{example}
	$$
		S(\{1,2,3,4,5,6, ...\}) = \{1,2,3,3,4,4,4,4,5,5,5,5,5,5,5,5,6...\}
	$$
\end{example}

Теперь нам потребуются некоторые свойства оператора $S$.

\begin{lemma}
	\label{thm:alpha_S}
	\begin{equation}\label{alpha_S}
		\alpha(Sx) = \varlimsup_{k\to\infty} |x_{k+1} - x_{k}|
	\end{equation}
\end{lemma}

\paragraph{Доказательство.}

\begin{equation*}
	\alpha(Sx) =
	\varlimsup_{i\to\infty} \sup_{i < j \leq 2i} | (Sx)_i - (Sx)_j | = ...
\end{equation*}
Положим для каждого $i$ число $m_i$ так,
что $m_i = 2^{k_i}$, $i \leq m_i < 2i$
(очевидно, это всегда можно сделать).
\begin{equation*}
	... =
	\varlimsup_{i\to\infty} \max \left\{
		\max_{i   < j \leq m_i} | (Sx)_i - (Sx)_j |,
		\max_{m_i < j \leq 2i } | (Sx)_i - (Sx)_j |
	\right\} =
	...
\end{equation*}
Но при $2^{k_i - 1} < i < j \leq m_i = 2^{k_i}$
имеем $(Sx)_i = (Sx)_j$, и первый модуль обращается в нуль.

\begin{equation*}
	... =
	\varlimsup_{i\to\infty}
		\max_{m_i < j \leq 2i } | (Sx)_i - (Sx)_j |
	=
	...
\end{equation*}
Но при $2^{k_i - 1} < i \leq m_i = 2^{k_i} < j \leq 2^{k_i+1}$
имеем $(Sx)_i = x_{k_i+1}$, $(Sx)_j = x_{k_i+2}$, откуда
\begin{equation}\label{alpha_S_sosedi}
	... =
	\varlimsup_{k\to\infty}
		| x_{k+1} - x_k |
\end{equation}

Лемма доказана.

\begin{lemma}
	\begin{equation}\label{summa_S_less}
		\sum_{k=2}^{2^p} (Sy)_k =
		\sum_{i=0}^{p-1} 2^i y_{i+2}
	\end{equation}
\end{lemma}

\paragraph{Доказательство.}

\begin{equation*}
	\sum_{k=2}^{2^p} (Sy)_k =
	\sum_{i=0}^{p-1} \sum_{k=2^i+1}^{2^{i+1}} (Sy)_k =
	\sum_{i=0}^{p-1} \sum_{k=2^i+1}^{2^{i+1}} y_{i+2} =
	\sum_{i=0}^{p-1} 2^i y_{i+2}
\end{equation*}

Лемма доказана.


\begin{lemma}
	\begin{equation}\label{summa_S}
		\sum_{k=2^i+1}^{2^{i+j+1}} (Sx)_k =
		2^i\sum_{k=2}^{2^{j+1}} (ST^ix)_k
	\end{equation}

	Здесь и далее $(Tx)_n = x_{n+1}$.
\end{lemma}

\paragraph{Доказательство.}

\begin{multline*}
	\sum_{k=2^i+1}^{2^{i+j+1}} (Sx)_k =
	\sum_{m = i}^{i+j}\sum_{k=2^m+1}^{2^{m+1}} (Sx)_k =
	\sum_{m = i}^{i+j}2^m \cdot x_{m+2} =
	\\=
	2^i \cdot \sum_{n = 0}^{j}2^n \cdot x_{n+2+i} =
	2^i \cdot \sum_{n = 0}^{j}2^n (T^i x)_{n+2} =
	2^i \cdot \sum_{k=2}^{2^{j+1}} (ST^i x)_k
\end{multline*}

Лемма доказана.

\subsection{Вспомогательная функция $k_b$}

Введём функцию
\begin{equation}\label{def_k_b}
	k_b(x) = \frac{1}{2b}\left|
		\sum_{k=1}^{b}x_k - \sum_{k=b+1}^{2b}x_k
	\right|
\end{equation}

\begin{lemma}
	\begin{equation}\label{alpha_greater_k_b}
		\alpha (Cx) \geq \varlimsup_{i\to \infty} k_i(x)
	\end{equation}
\end{lemma}

\paragraph{Доказательство.}

\begin{multline*}
	\alpha (Cx) \mathop{=}\limits^{def}
	\varlimsup_{i\to \infty} \sup_{i<j\leq 2i} |(Cx)_i - (Cx)_j| \geq
	\varlimsup_{i\to \infty} |(Cx)_i - (Cx)_{2i}| =
	\\ =
	\varlimsup_{i\to \infty} \left|\frac{1}{i}\sum_{k=1}^i  - \frac{1}{2i}\sum_{k=1}^{2i} \right| =
	\varlimsup_{i\to \infty} \left|\frac{1}{i}\sum_{k=1}^i  - \frac{1}{2i}\sum_{k=1}^{i}- \frac{1}{2i}\sum_{k=i+1}^{2i}\right| =
	\\=
	\varlimsup_{i\to \infty} \left|\frac{1}{2i}\sum_{k=1}^i - \frac{1}{2i}\sum_{k=i+1}^{2i}\right| =
	\varlimsup_{i\to \infty} k_i(x)
\end{multline*}

\paragraph{Примечание.}
Введение функции $k_b(x)$ позволит нам в дальнейшем перейти от работы с оператором Чезаро
к несложным преобразованиям сумм.



\subsection{Основные построения}

Построим вектор $y\in \ell_\infty$ следующим образом:

\begin{equation}\label{y_construction}
	y = \left\{
		0, 0, \frac{1}{p}, \frac{2}{p}, \frac{3}{p},
		...,
		\frac{p-1}{p}, 1, \frac{p-1}{p},
		...,
		\frac{1}{p},
		~~
		\underbrace{
		\phantom{\frac{1}{1}\!\!\!}
			0, 0, 0, ..., 0,
		}_\text{$\phantom{\frac{1}{1}\!\!\!}$\!\!\!$3p+1$ раз}
		~~
		\frac{1}{p}, ...
	\right\}
\end{equation}
так, что
\begin{equation}\label{T_y}
	T^{5p}y = y
\end{equation}
%(0 повторяется $3p+1$ раз или около того, надо будет ещё очень аккуратно пересчитать).


Положим $x = Sy$.
Тогда с учётом (\ref{alpha_S})
\begin{equation}\label{alpha_x}
	\alpha (x) = \alpha (Sy) = \frac{1}{p}
\end{equation}


Оценим $\alpha(Cx)$:

\begin{multline*}
	\alpha (Cx) \mathop{\geq}^{(\ref{alpha_greater_k_b})}
	\varlimsup_{b\to \infty} k_b(x) =
	\varlimsup_{b\to \infty}\frac{1}{2b}\left|
		\sum_{k=1}^{b}x_k - \sum_{k=b+1}^{2b}x_k
	\right| \geq
	\\ \geq
	\varlimsup_{
		i\to \infty,~
		b=2^i~
	}\frac{1}{2^{i+1}}\left|
		\sum_{k=1}^{2^i}(Sy)_k - \sum_{k=2^i+1}^{2^{i+1}}(Sy)_k
	\right| =
	\\=
	\varlimsup_{i\to \infty}\frac{1}{2^{i+1}}\left|
		\sum_{k=1}^{2^i}(Sy)_k - 2^i y_{i+2}
	\right| =
	\varlimsup_{i\to \infty}\left|
		\frac{1}{2^{i+1}}\sum_{k=1}^{2^i}(Sy)_k - \frac{y_{i+2}}{2}
	\right| \geq
\end{multline*}
\begin{multline*}
	\\ \geq
	\varlimsup_{
		m\to \infty,~
		i=5pm+p~
	}\left|
		\frac{1}{2^{5pm+p+1}}\sum_{k=1}^{2^{5pm+p}}(Sy)_k - \frac{y_{5pm+p+2}}{2}
	\right| =
	\\=
	\varlimsup_{m\to \infty}\left|
		\frac{1}{2^{5pm+p+1}}\sum_{k=1}^{2^{5pm+p}}(Sy)_k - \frac{1}{2}
	\right| =
	\\=
	\varlimsup_{m\to \infty}\left|
		\frac{1}{2^{5pm+p+1}}\sum_{k=1}^{2^{5pm}}(Sy)_k
		+
		\frac{1}{2^{5pm+p+1}}\sum_{k=2^{5pm}+1}^{2^{5pm+p}}(Sy)_k
		- \frac{1}{2}
	\right|
	\mathop{=}^{(\ref{summa_S})}
	\\=
	\varlimsup_{m\to \infty}\left|
		\frac{1}{2^{5pm+p+1}}\sum_{k=1}^{2^{5pm}}(Sy)_k
		+
		\frac{1}{2^{5pm+p+1}} \cdot 2^{5pm} \cdot \sum_{k=2}^{2^p}(ST^{5pm}y)_k
		- \frac{1}{2}
	\right|
	\mathop{=}^{(\ref{T_y})}
	\\=
	\varlimsup_{m\to \infty}\left|
		\frac{1}{2^{5pm+p+1}}\sum_{k=1}^{2^{5pm}}(Sy)_k
		+
		\frac{1}{2^{5pm+p+1}} \cdot 2^{5pm} \cdot \sum_{k=2}^{2^p}(Sy)_k
		- \frac{1}{2}
	\right| =
	\\=
	\varlimsup_{m\to \infty}\left|
		\frac{1}{2^{5pm+p+1}}\sum_{k=1}^{2^{5pm}}(Sy)_k
		+
		\frac{1}{2^{p+1}} \sum_{k=2}^{2^p}(Sy)_k
		- \frac{1}{2}
	\right|
	\mathop{=}^{(\ref{summa_S_less})}
	\\=
	\varlimsup_{m\to \infty}\left|
		\frac{1}{2^{5pm+p+1}}\sum_{k=1}^{2^{5pm}}(Sy)_k
		+
		\frac{1}{2^{p+1}} \sum_{i=0}^{p-1}2^i y_{i+2}
		- \frac{1}{2}
	\right| =
	\\=
	\varlimsup_{m\to \infty}\left|
		\frac{1}{2^{5pm+p+1}}\sum_{k=1}^{2^{5pm}}(Sy)_k
		+
		\frac{1}{2^{p+1}} \sum_{i=0}^{p-1}2^i \cdot \frac{i}{p}
		- \frac{1}{2}
	\right|
	\mathop{=}^{(\ref{summa_drobey})}
	\\=
	\varlimsup_{m\to \infty}\left|
		\frac{1}{2^{5pm+p+1}}\sum_{k=1}^{2^{5pm}}(Sy)_k
		+
		\frac{1}{2^{p+1}} \cdot \frac{2^p(p-2)+2}{p}
		- \frac{1}{2}
	\right| =
	\\=
	\varlimsup_{m\to \infty}\left|
		\frac{1}{2^{5pm+p+1}}\sum_{k=1}^{2^{5pm}}(Sy)_k
		+
		\frac{1}{2} \cdot \frac{p-2}{p} + \frac{1}{p 2^p}
		- \frac{1}{2}
	\right| =
	\\=
	\varlimsup_{m\to \infty}\left|
		\frac{1}{2^{5pm+p+1}}\sum_{k=1}^{2^{5pm}}(Sy)_k
		-
		\frac{1}{p} + \frac{1}{p 2^p}
	\right| =
	\\=
	\varlimsup_{m\to \infty}\left|
		\frac{1}{2^{5pm+p+1}}\sum_{k=1}^{2^{5pm-2p}}(Sy)_k
		+
		\frac{1}{2^{5pm+p+1}}\sum_{k=2^{5pm-2p}+1}^{2^{5pm}}(Sy)_k
		-\frac{1}{p} + \frac{1}{p 2^p}
	\right| =
	\\\mbox{но во второй сумме все $(Sy)_k$ --- нули по построению}
\end{multline*}
\begin{multline*}
	\\=
	\varlimsup_{m\to \infty}\left|
		\frac{1}{2^{5pm+p+1}}\sum_{k=1}^{2^{5pm-2p}}(Sy)_k
		-\frac{1}{p} + \frac{1}{p 2^p}
	\right| = h
\end{multline*}

Но $0 \leq (Sy)_k \leq 1$,
значит,
$$
	\frac{1}{2^{5pm+p+1}}\sum_{k=1}^{2^{5pm-2p}}(Sy)_k
	\leq
	\frac{1}{2^{5pm+p+1}} \cdot 2^{5pm-2p}
	=
	\frac{1}{2^{3p+1}}
$$
Модуль раскрываем со знаком ``-''

\begin{multline*}
	h=
	\varlimsup_{m\to \infty} \left(
		\frac{1}{p} (1-2^{-p})
		- \frac{1}{2^{3p+1}}
	\right) =
	\frac{1}{p} (1-2^{-p})
	- \frac{1}{2^{3p+1}}
	= \\ =
	\frac{1}{p} (1-2^{-p})
	- \frac{1}{2^{2p+1}} \cdot 2^{-p}
	>
	\frac{1}{p} (1-2^{-p})
	- \frac{1}{p} \cdot 2^{-p}
	=
	\frac{1}{p} (1-2^{-p+1})
\end{multline*}


Таким образом,
$$
	\frac{\alpha(Cx)}{\alpha(x)} \geq
	\frac{	\frac{1}{p} (1-2^{-p+1}) }{\frac{1}{p}} =
	1-2^{-p+1}
$$

Рассматривая $x$ как функцию от $p$, имеем:
$$
	\sup_{p\in\mathbb{N}} \frac{\alpha(Cx(p))}{\alpha(x(p))} \geq
	\sup_{p\in\mathbb{N}} (1-2^{-p+1}) =
	1
	.
$$
Таким образом, может быть сформулирована следующая

\begin{theorem}
	\label{thm:alpha_Cx_no_gamma}
	\begin{equation}
		\sup_{x\in\ell_\infty, \alpha(x)\neq 0} \frac{\alpha(Cx)}{\alpha(x)}=1
		.
	\end{equation}
\end{theorem}

\subsection{Некоторые гипотезы}

\begin{hypothesis}
	Пусть $x\in\ell_\infty$ и $0 \leq x_k \leq 1$.
	Тогда для любого $n\in\mathbb{N}$
	\begin{equation}
		\alpha(C^n x) \leq \frac{1}{2^n}
		.
	\end{equation}
\end{hypothesis}

\begin{hypothesis}
	Для любых $x\in\ell_\infty$ и $n\in\mathbb{N}$
	\begin{equation}
		\alpha(C^n x) - \alpha(C^{n+1} x) \leq \alpha(C^{n-1} x) - \alpha(C^{n} x)
		.
	\end{equation}
\end{hypothesis}

\begin{hypothesis}
	Для любого $0<a<1$ существует такой $x\in\ell_\infty$, что
	\begin{equation}
		\lim_{m\to\infty} \frac{\alpha(C^m x)}{a^m} = \infty
		.
	\end{equation}
\end{hypothesis}


	\section{$\alpha$--функция и оператор суперпозиции}
	Введём на пространстве $\ell_\infty$ фактор-норму по $c_0$:

\begin{equation}
	\|x\|_* = \limsup_{k\to\infty} |x_k|
	.
\end{equation}

\begin{theorem}
	\label{thm:alpha_xy}
	Пусть $(x\cdot y)_k = x_k\cdot y_k$.
	Тогда
	$\alpha(x\cdot y)\leq \alpha(x)\cdot \|y\|_* + \alpha(y)\cdot \|x\|_*$.
\end{theorem}

\begin{proof}
	\begin{multline}
		\alpha(x\cdot y)
		=
		\limsup_{i\to\infty} \max_{i\leq j \leq 2i} |x_i y_i - x_j y_j|
		=
		\limsup_{i\to\infty} \max_{i\leq j \leq 2i} |x_i y_i - x_j y_i + x_j y_i - x_j y_j|
		\leq
		\\ \leq
		\limsup_{i\to\infty} \max_{i\leq j \leq 2i} \left(|x_i y_i - x_j y_i| + |x_j y_i - x_j y_j| \right)
		=
		\\=
		\limsup_{i\to\infty} \max_{i\leq j \leq 2i} \left(|y_i|\cdot|x_i - x_j| + |x_j|\cdot|y_i - y_j| \right)
		\leq
		\\ \leq
		\limsup_{i\to\infty} \max_{i\leq j \leq 2i} |y_i|\cdot|x_i - x_j| + \limsup_{i\to\infty} \max_{i\leq j \leq 2i}|x_j|\cdot|y_i - y_j|
		=
		\\ =
		\limsup_{i\to\infty} |y_i| \max_{i\leq j \leq 2i} \cdot|x_i - x_j| + \limsup_{i\to\infty} \max_{i\leq j \leq 2i}|x_j|\cdot|y_i - y_j|
		\leq
		\\ \leq
		\limsup_{i\to\infty} |y_i| \cdot \limsup_{i\to\infty} \max_{i\leq j \leq 2i} \cdot|x_i - x_j| + \limsup_{i\to\infty} \max_{i\leq j \leq 2i}|x_j|\cdot|y_i - y_j|
		=
		\\ =
		\|y\|_* \cdot \limsup_{i\to\infty} \max_{i\leq j \leq 2i} \cdot|x_i - x_j| + \limsup_{i\to\infty} \max_{i\leq j \leq 2i}|x_j|\cdot|y_i - y_j|
		=
		\\ =
		\|y\|_* \cdot \alpha(x) + \limsup_{i\to\infty} \max_{i\leq j \leq 2i}|x_j|\cdot|y_i - y_j|
		\leq
		\\ \leq
		\|y\|_* \cdot \alpha(x) + \limsup_{i\to\infty} \max_{i\leq j \leq 2i}|x_j|\cdot\limsup_{i\to\infty} \max_{i\leq j \leq 2i}|y_i - y_j|
		=
		\\ =
		\|y\|_* \cdot \alpha(x) + \limsup_{i\to\infty} \max_{i\leq j \leq 2i}|x_j|\cdot \alpha(y)
		\leq
		\\ \leq
		\|y\|_* \cdot \alpha(x) + \limsup_{j\to\infty} |x_j|\cdot \alpha(y)
		=
		%\\ =
		\|y\|_* \cdot \alpha(x) + \|x\|_* \cdot \alpha(y)
		.
	\end{multline}
\end{proof}

\begin{example}
	Покажем, что нижнюю оценку на $\alpha(x\cdot y)$ дать нельзя.
	В самом деле,
	пусть $x_k = (-1)^k$.
	Тогда $\alpha(x) = 2$, $\|x\|_* = 1$,
	но $\alpha(x\cdot x) = 0$.
\end{example}

\begin{example}
	\label{ex:alpha-c_not_ideal}
	Покажем, что оценка теоремы~\ref{thm:alpha_xy} точна в том смысле,
	что равенство достижимо.
	В самом деле,
	пусть $x_k = (-1)^k$, $y_k = 1$.
	Тогда $\alpha(x) = 2$, $\alpha(y) = 0$,
	но $\alpha(x\cdot y) = 2$.
\end{example}

\begin{example}
	Пусть
	\begin{equation}
		y = \left(
			0, \frac{1}{p}, \frac{2}{p}, \frac{3}{p},
			...,
			\frac{p-1}{p}, 1, \frac{p-1}{p},
			...,
			\frac{1}{p},
			0,
			\frac{1}{p}, \frac{2}{p}, ...
		\right)
		.
	\end{equation}
	Тогда $\alpha(Sy) = \frac{1}{p}$, $\|Sy\|_* = 1$,
	\begin{equation}
		\alpha((Sy)\cdot(Sy)) =
		\left| \frac{(p-1)^2}{p^2} - 1 \right| =
		\frac{2}{p}-\frac{1}{p^2}
		.
	\end{equation}
\end{example}

\begin{example}
	Для того же $y$ и любого $k>0$ имеем
	$\alpha(kSy) = \dfrac{k}{p}$, $\|kSy\|_* = k$,
	\begin{equation}
		\alpha((kSy)\cdot(kSy)) =
		\left| \frac{k^2(p-1)^2}{p^2} - k^2 \right| =
		\frac{2k^2}{p}-\frac{k^2}{p^2}
		.
	\end{equation}
	Оценка теоремы~\ref{thm:alpha_xy} даёт
	\begin{equation}
		\alpha((kSy)\cdot(kSy)) \leq
		\frac{2k^2}{p}
		.
	\end{equation}
\end{example}


% Увы, неверна - контрпример выше, k=2
%\begin{hypothesis}
%	$\alpha(x\cdot y)\leq \alpha(x)\cdot \|y\|_* + \alpha(y)\cdot \|x\|_* - \alpha(x)\alpha(y) \cdot \|y\|_* \cdot \|x\|_*$.
%\end{hypothesis}
% Убрать фактор-нормы тоже нельзя:

\begin{example}
	Пусть $x_k = (-1)^k$.
	Тогда $\alpha(x) = \alpha(\sigma_2 x) = 2$, $\|x\|_* = \|\sigma_2 x\|_* = 1$,
	но $\alpha(x\cdot \sigma_2 x) = 2 < 4$.
\end{example}

Исходя из построенных примеров может быть выдвинута
\begin{hypothesis}
	Если $\alpha(x\cdot y)= \alpha(x)\cdot \|y\|_* + \alpha(y)\cdot \|x\|_*$,
	то $\alpha(x) = 0$ или $\alpha(y) = 0$.
\end{hypothesis}


	\section{Функционалы $\alpha^*$ и $\alpha_*$}
	Как мы выяснили, $\alpha$--функция не инвариантна относительно сдвига.
Чтобы избавиться от этого недостатка, введём функционалы
\begin{equation}
	\alpha^* = \lim_{n\to\infty} U^n x
\end{equation}
и
\begin{equation}
	\alpha_* = \lim_{n\to\infty} T^n x
\end{equation}
(оба пределы существуют как пределы монотонных ограниченных последовательностей).
Тогда верны следующие соотношения.
\begin{lemma}
	\begin{equation}
		\alpha^*(x) = \alpha^*(\sigma_n x) = \alpha^*(Tx) = \alpha^*(Ux)
		,
		%TODO: \sigma_{1/n}
	\end{equation}
	\begin{equation}
		\alpha_*(x) = \alpha_*(\sigma_n x) = \alpha_*(Tx) = \alpha_*(Ux)
		.
		%TODO: \sigma_{1/n}
	\end{equation}
\end{lemma}

\begin{lemma}
	\begin{equation}
		\frac{1}{2} \alpha(x) \leq \alpha_*(x) \leq \alpha(x) \leq \alpha^*(x) \leq 2 \alpha(x)
		.
	\end{equation}
\end{lemma}

Кроме того, для учёта обоих функционалов и сохранения однородности предлагается ввести функционал
\begin{equation}
	\alpha_*^*(x) = \sqrt{\alpha_*(x)\alpha^*(x)}
	.
\end{equation}

\begin{hypothesis}
	Функционал $\alpha_*^*(x)$ удовлетворяет неравенству треугольника.
\end{hypothesis}

\begin{hypothesis}
	Аналог теоремы~\ref{thm:alpha_Cx_no_gamma} верен для функционалов
	$\alpha^*(x)$, $\alpha_*(x)$ и $\alpha_*^*(x)$
\end{hypothesis}


	\section{Пространство $\{x: \alpha(x) = 0\}$}
	\label{sec:space_A0}

Из свойства~\ref{thm:alpha_x_triangle_ineq} и однородности $\alpha$--функции немедленно вытекает
\begin{theorem}
	\label{thm:A0_is_space}
	Множество $A_0 = \{x: \alpha(x) = 0\}$
	является пространством.
\end{theorem}
Изучим свойства этого пространства.
\begin{property}
	Пространство $A_0$ замкнуто.
\end{property}
\begin{proof}
	Прообраз замкнутого множества $\{0\}$
	при непрерывном отображении $\alpha : \ell_\infty \to \R$
	замкнут.
\end{proof}
\begin{theorem}
	Включение $c \subset A_0$ собственное.
\end{theorem}
\begin{proof}
	Рассмотрим
	\begin{equation}
		x=\left(
			0,1,
			0,\frac{1}{2},1,\frac{1}{2},
			0,\frac{1}{3},\frac{2}{3},1,\frac{2}{3},\frac{1}{3},
			0,
			...
		\right)
		.
	\end{equation}
	Тогда по лемме~\ref{thm:alpha_S} (при определённом тем же образом операторе $S$) имеем
	\begin{equation}
		\alpha(Sx) = \varlimsup_{k\to\infty} |x_{k+1} - x_{k}| = 0
		,
	\end{equation}
	однако, очевидно, $Sx\notin c$.
\end{proof}
Очевидно следующее
\begin{property}
	Если периодическая последовательность принадлежит $A_0$,
	то эта последовательность~--- константа.
\end{property}

Из результатов предыдущих параграфов вытекает
\begin{theorem}
	Пространство $A_0$ замкнуто относительно операторов $T$, $U$, $C$, $\sigma_n$, $\sigma_{1/n}$, $S$.
\end{theorem}

Из теоремы~\ref{thm:alpha_xy} следует
\begin{theorem}
	Пространство $A_0$ замкнуто относительно умножения,
	т.е. если $x,y\in A_0$, то $x\cdot y \in A_0$.
\end{theorem}

Пример~\ref{ex:alpha-c_not_ideal} показывает, что $A_0$
не является идеалом по умножению.

Из теоремы~\ref{thm:rho_x_c_leq_alpha_t_s_x_united} следует
\begin{theorem}
	$ac \cap A_0 = c$.
\end{theorem}

Некоторый интерес представляет следующая теорема,
основанная на идеях~\cite{usachev2009_phd_vsu}.

\begin{theorem}
	Пространство $A_0$ несепарабельно.
\end{theorem}

\begin{proof}
	Напомним, что через $\Omega = \{0;1\}^\N$
	обозначается множество всех последовательностей, состоящих из нулей и единиц.
	Для каждого $\omega\in\Omega$ положим
	\begin{multline}
		\label{eq:x_omega_alpha_c}
		x(\omega)=\left(
			0, 1\omega_1,
			0, \frac{1}{2}\omega_2, 1\omega_2, \frac{1}{2}\omega_2,
			0, \frac{1}{3}\omega_3, \frac{2}{3}\omega_3, 1\omega_3, \frac{2}{3}\omega_3, \frac{1}{3}\omega_3,
			0, ...,
		\right. \\ \left.
			0, \frac{1}{p}\omega_p, \frac{2}{p}\omega_p, ..., \frac{p-1}{p}\omega_p, 1\omega_p,
				\frac{p-1}{p}\omega_p, ..., \frac{2}{p}\omega_p, \frac{1}{p}\omega_p,
			0, \frac{1}{p+1}\omega_{p+1}, ...
		\right).
	\end{multline}
	Тогда по лемме~\ref{thm:alpha_S} (при определённом тем же образом операторе $S$) имеем
	$\alpha(Sx(\omega)) = 0$.
	Заметим, что при $\omega,\omega^* \in \Omega$ и $\omega\neq\omega^*$ выполнено
	$\|Sx(\omega)-Sx(\omega^*)\|=1$ и $|Sx(\Omega)|=\mathfrak{c}$.
	Следовательно, $A_0$ несепарабельно.
\end{proof}

%\begin{theorem}
%	$|A_0| = |\ell_\infty|$.
%\end{theorem}

%\begin{proof}
%	Достаточно заметить, что
%\end{proof}


	\section{О недополняемости некоторых вложений}
	\label{sec:noncomplementarity}

В работе~\cite{phillips1940linear} Филлипс доказал весьма неожиданный (для своего времени) результат:
простанство $c_0$ недополняемо в $\ell_\infty$.
Говоря формально, справедлива следующая


\begin{theorem}[Филлипса]
	\label{thm:phillips}
	Не существует непрерывного линейного оператора $P: \ell_\infty \to c_0$ такого, что для любого
	$x \in c_0$ выполнено равенство $Px =x$.
\end{theorem}
Подобные операторы называются \emph{проекторами}.

Теорема~\ref{thm:phillips} была первым примером недополняемого вложения пространств.
Позже были найдены и другие примеры;
%TODO: краткий обзор - передрать из Conway 2nd ed., p. 94
отсылаем читателя, например, к~\cite{lindenstrauss1979classical}.

Е.А. Алехно привёл~\cite[Theorem 8]{alekhno2006propertiesII} элегантное доказательство
того, что $ac_0$ недополняемо в $\ell_\infty$.
Это доказательство основано на изначальном доказательстве Филлипса теоремы~\ref{thm:phillips}
и использует некоторые леммы из~\cite{phillips1940linear}.

Вложение $c_0 \subset ac_0$ также недополняемо.
Непосредственное явное упоминание этого факта найти не удалось, однако он достаточно легко следует из~\cite[теорема 4]{ASSU2},
доказательство которой опирается на идеи~\cite{whitley1968projecting}~и~\cite[Theorem 6.9]{Carothers}.
Приведём эту теорему (в несколько ослабленной форме, для которой достаточно уже введённой терминологии)
и соотвествующее следствие, предварив одной вспомогательной леммой, доказательство коей является столь же классическим, сколь и кратким,
и приводится здесь исключительно ради полноты изложения.

\begin{definition}
	Семейство множеств $\{A_\lambda\}_{\lambda \in \Lambda}$ называется почти дизъюнктным,
	если для $\lambda, \mu \in \Lambda$, $\lambda \ne \mu$,
	пересечение $A_\lambda \cap A_\mu$ конечно.
\end{definition}

\begin{lemma}
	\label{lem:uncountable_subsets_of_N_with_finite_intersections}
	Существует почти дизъюнктное несчётное семейство подмножеств $\{S_i\}_{i\in I}$, $S_i \subset \mathbb{N}$.
\end{lemma}

\begin{proof}
	Рассмотрим биекцию $\N \leftrightarrow \Q$.
	Пусть $I = \R$.
	Для каждого $i\in I$ положим $S_i = \{q_n\}$,
	где $\{q_n\} \subset \mathbb{Q}$ "--- некоторая последовательность рациональных чисел,
	сходящаяся к $i$.
	%(Proof: switch from $\mathbb{N}$ to $\mathbb{Q}$, pick convergent sequences to irrationals.
	%More details in [this question](https://math.stackexchange.com/q/162387).)
\end{proof}

\begin{theorem}
	\label{thm:Alekhno_noncomplementarity_general}
	Пусть $X$ и $Y$ "--- такие линейные подпространства в $\ell_\infty$,
	что $c_{00} \subseteq Y \subsetneq X \subseteq \ell_\infty$.
	Пусть существует такое несчётное почти дизъюнктное семейство множеств натуральных чисел $\{A_\lambda\}_{\lambda \in \Lambda}$,
	$A_\lambda \subseteq \N$, что для любого $\lambda \in \Lambda$ имеет место включение $\chi_{A\lambda} \in X \setminus Y$.
	Тогда вложение $Y \subset X$ недополняемо.
\end{theorem}

\begin{corollary}
	Вложение $c_0\subsetneq ac_0$ недополняемо.
	Вложение $c_0\subsetneq \Iac$ недополняемо.
\end{corollary}

\begin{proof}
	Возьмём почти дизъюнктное семейство $\{S_i\}_{i\in I}$, $S_i \subset \mathbb{N}$
	из леммы~\ref{lem:uncountable_subsets_of_N_with_finite_intersections}
	и построим новое семейство $\{A_i\}_{i\in I}$, $A_i \subset \mathbb{N}$
	по правилу
	\begin{equation}
		\label{eq:c0_noncomplemented_in_ac0_set_family}
		A_i = \{ 2^n : n\in S_i \}
		.
	\end{equation}
	Легко видеть, что семейство $\{A_i\}_{i\in I}$ снова является почти дизъюнктным.
	Более того, $\chi_{A_i} \in ac_0 \setminus c_0$ и $\chi_{A_i} \in \Iac \setminus c_0$ для любого $i\in I$.

	Таким образом, выполнены все условия теоремы~\ref{thm:Alekhno_noncomplementarity_general}
	для цепочки вложений $c_{00} \subseteq c_0 \subsetneq ac_0 \subseteq \ell_\infty$ и
	для цепочки вложений $c_{00} \subseteq c_0 \subsetneq \Iac \subseteq \ell_\infty$.
\end{proof}

%TODO: а что там с \Iac \subset ac_0 ?

Итак, все три вложения в цепочке
\begin{equation}
	c_0 \subset \ell_\infty,
	\quad
	ac_0 \subset \ell_\infty,
	\quad\mbox{и}\quad
	c_0 \subset ac_0,
\end{equation}
недополняемы.

Перейдём теперь к изучению пространства $A_0$.

\begin{lemma}
	Вложение $A_0 \subset \ell_\infty$ недополняемо.
\end{lemma}

\begin{proof}
	В теореме~\ref{thm:Alekhno_noncomplementarity_general}
	положим $Y=A_0$, $X=\ell_\infty$.
	Тогда для сесмества множеств~\eqref{eq:c0_noncomplemented_in_ac0_set_family}
	выполнены все условия теоремы~\ref{thm:Alekhno_noncomplementarity_general}.
\end{proof}

Однако для вложения $c_0 \subset A_0$ применить теорему~\ref{thm:Alekhno_noncomplementarity_general}
непосредственно не удастся.
Осноная проблема заключается в том, что для любого бесконечного множества $F \subset \N$ такого, что
дополнение $\N \setminus F$ также бесконечно, последовательность $\chi_F \notin A_0$,
поскольку $\alpha(F) = 1$.

Поэтому мы проведём доказательство недополняемости полностью,
во многом опираясь на идеи~\cite{whitley1968projecting} и дискуссию~\cite{mathSE_Phillips}.
Для этого нам потребуется напомнить некоторые вспомогательные конструкции,
с которыми читатель уже встречался в этой главе.


Определим линейный оператор  $F:\ell_\infty \to \ell_\infty$ соотношением
\begin{equation}
	\label{operator_F}
	(Fy)_k = y_{i+2}, \mbox{ for } 2^i < k \leq 2^{i+1}
	,
\end{equation}
т.е.
\begin{equation}
	F(\{x_1,x_2,x_3,x_4,x_5,x_6, ...\}) = \{x_1,x_2,\,x_3,x_3,\,x_4,x_4,x_4,x_4,\,x_5,x_5,x_5,x_5,x_5,x_5,x_5,x_5,\,x_6...\}
\end{equation}

%TODO:ссылки?..
Напомним, что выполнено равенство
\begin{equation}
	\label{eq:alpha_F}
	\alpha(Fx) = \varlimsup_{k\to\infty} |x_{k+1} - x_{k}|
	.
\end{equation}


Определим линейный оператор $M:\ell_\infty \to \ell_\infty$ следующим образом:
\begin{multline*}
	M(\omega_1,\omega_2,...)=\left(
		0, 1\omega_1,
		0, \frac{1}{2}\omega_2, 1\omega_2, \frac{1}{2}\omega_2,
		0, \frac{1}{3}\omega_3, \frac{2}{3}\omega_3, 1\omega_3, \frac{2}{3}\omega_3, \frac{1}{3}\omega_3,
		0, ...,
	\right. \\ \left.
		0, \frac{1}{p}\omega_p, \frac{2}{p}\omega_p, ..., \frac{p-1}{p}\omega_p, 1\omega_p,
			\frac{p-1}{p}\omega_p, ..., \frac{2}{p}\omega_p, \frac{1}{p}\omega_p,
		0, \frac{1}{p+1}\omega_{p+1}, ...
	\right)
	.
\end{multline*}
Заметим, что в силу~\eqref{eq:alpha_F} мы имеем $FM: \ell_\infty \to A_0$.


\begin{lemma}
	\label{lem:c_0_not_complemented_in_A_0}
	Пусть линейный оператор  $Q: A_0 \to A_0$ таков, что $c_0\subseteq \ker Q$.
	Тогда существует счётное подмножество $S \subset \N$ такое, что
	\begin{equation}
		\forall(x \in A_0 : \supp x \subset S)[Qx = 0]
	\end{equation}
	и $x\in A_0\setminus c_0$ такой, что $\supp x \subseteq S$.
\end{lemma}

\begin{proof}
	Пусть $\{U_i\}_{i \in I}$ есть семейство подмножеств $\N$,
	удовлетворяющее условиям леммы~\ref{lem:uncountable_subsets_of_N_with_finite_intersections}.
	Пусть $\{S_i\}_{i \in I}$ есть семейство подмножеств $\N$
	определённое соотношением $S_i = \supp FM\chi_{U_i}$.
	Очевидно, что семество множеств $\{S_i\}_{i \in I}$ также
	удовлетворяет условиям леммы~\ref{lem:uncountable_subsets_of_N_with_finite_intersections}.
	%satisfies the conditions of Lemma~\ref{lem:uncountable_subsets_of_N_with_finite_intersections}.
	Более того, для любого $i\in I$ мы имеем $x = FM\chi_{U_i} \in A_0\setminus c_0$.

	Предположим противное:
	\begin{equation}
		\forall(\mbox{infinite }S\subset\N)\exists(x \in A_0 : \supp x \subset S)[Qx \neq 0]
		.
	\end{equation}
	В частности,
	%In particular,
	\begin{equation}
		\forall(i\in I)\exists(x_i \in A_0 : \supp x_i \subset S_i)[Q(x_i) \neq 0]
		.
	\end{equation}

	Заметим, что $x_i \notin c_0$, поскольку $c_0\subseteq \ker Q$.
	Не теряя общности, будем полагать $\|x_i\|=1$ для всех $i \in I$.
	%Without loss of generality we can assume that $\|x_i\|=1$ for all $i \in I$.

	Положим $I_n = \{i \in I\,:\,(Qx_i)_n \neq 0\}$,
	%Consider $I_n = \{i \in I\,:\,(Qx_i)_n \neq 0\}$,
	тогда $I = \bigcup\limits_{n\in\N} I_n$.
	%then $I = \bigcup\limits_{n\in\N} I_n$.
	Таким образом, найдётся $n$ такой, что $I_n$ также несчётно
	%Thus, we can find $n$ such that $I_n$ is also uncountable
	(иначе $I$ было бы счётным как объединение счётного количества счётных множеств,
	%(otherwise $I$ would be countable as a countable union of countable sets,
	что противоречит условиям леммы~\ref{lem:uncountable_subsets_of_N_with_finite_intersections}).
	%which contradicts to conditions of Lemma~\ref{lem:uncountable_subsets_of_N_with_finite_intersections}).

	Рассмотрим теперь $I_{n,k} = \{i \in I_n\,:\,|(Qx_i)_n| \geq 1/k\}$,
	%Сonsider now $I_{n,k} = \{i \in I_n\,:\,|(Qx_i)_n| \geq 1/k\}$,
	тогда $I_n = \bigcup\limits_{k\in\N} I_{n,k}$.
	%then $I_n = \bigcup\limits_{k\in\N} I_{n,k}$.
	Аналогичные рассуждения показывают, что для некоторого $k$ множество $I_{n,k}$ несчётно.
	%Applying the same argument as above, one can easily see that the set $I_{n,k}$ is uncountable for some $k$.
	Зафиксируем такое $I_{n,k}$ и в дальнейшем будем работать с ним.
	%Let us choose such $I_{n,k}$ and proceed with it.

	Итак, у нас есть несчётное множество $I_{n,k}$ и
	%So, we have an uncountable set $I_{n,k}$ and
	\begin{equation}
		\forall(i\in I_{n,k})\exists(x_i \in ac_0 : \supp x_i \subset S_i)\Bigl[\|x_i\|=1 \mbox{~~и~~} |(Qx_i)_n| \geq 1/k\Bigr]
		.
	\end{equation}

	Рассмотрим конечное множество $J \subset I_{n,k}$, $\#J>1$
	%Consider a finite set $J \subset I_{n,k}$ with $\#J>1$
	(здесь $\#J$ означает мощность множества $J$).
	%(here $\#J$ stands for the cardinality of the set $J$).
	Возьмём
	\begin{equation}
		y = \sum_{j \in J} \operatorname{sign}{(Qx_j)_n} \cdot x_j
		.
	\end{equation}
	Поскольку пересечение $S_i \cap S_j$ конечно для любых $i \neq j$ и
	$\supp x_j \subset S_j$,
	пересечение $\bigcap\limits_{j\in J} \supp x_j$ также конечно.
	Таким образом, $y = f + z$,
	причём $\supp f$ конечен и $\|z\| \leq 1$.

	С другой стороны,
	\begin{equation}
		\label{eq:non_complemented_sum_cardinality}
		(Qy)_n = \sum_{j \in J}
		(\operatorname{sign}(Qx_j)_n)
		\cdot (Qx_j)_n \geq \frac{\# J}{k}
		.
	\end{equation}
	Заметив, что $f\in c_0$, мы получаем $Qf = 0$, поскольку $c_0 \subseteq \ker Q$.
	Значит, $Qy = Q(f+z) = Qf + Qz = Qz$ и
	\begin{equation}
		\label{eq:norm_Q_estimate}
		\frac{\# J}{k} \leq (Qy)_n \leq \|Qy\| = \|Qz\| \leq \|Q\| \cdot \|z\| \leq \|Q\|
		.
	\end{equation}
	С  учётом~\eqref{eq:norm_Q_estimate} получаем $\# J \leq \|Q\| k$ для любого $J\subset I_{n,k}$.
	%Due to~\eqref{eq:norm_Q_estimate}, we obtain $\# J \leq \|Q\| k$ for every  $J\subset I_{n,k}$.
	Таким образом мы получаем противоречие с тем фактом, что $I_{n,k}$ несчётно.
	%This contradicts the fact that $I_{n,k}$ is uncountable,
	%and we are done.
\end{proof}

\begin{theorem}
	Пространство $c_0$ недополняемо в $A_0$.
\end{theorem}

\begin{proof}
	Предположим противное.
	Тогда существует непрерывный проектор $P: A_0 \to c_0$.
	Применим лемму~\ref{lem:c_0_not_complemented_in_A_0} к оператору $I-P$
	и найдём бесконечное подмножество $S\subset\N$ такое,
	что $\forall(x\in A_0 : \supp x \subset S)[(I-P)x = 0]$,
	и $x\in A_0 \setminus c_0$ такой, что $\supp x \subseteq S$.
	Но тогда  $Px = x\notin c_0$,
	что противоречит предположению, что $P$ есть проектор на $c_0$.
\end{proof}




\chapter{Последовательности, почти сходящиеся к нулю (пространство $ac_0$)}

	Почти сходимость является естественным обобщением понятия сходимости.
История исследования почти сходимости начинается с работы Г.Г. Лоренца~\cite{lorentz1948contribution}.
Напомним определение банахова предела.

\begin{definition}
	\label{def:Banach_limit}
	Линейный функционал $B\in \ell_\infty^*$ называется банаховым пределом,
	если
	\begin{enumerate}
		\item
			$B\geq0$, т.~е. $Bx \geq 0$ для $x \geq 0$,
		\item
			$B\one=1$, где $\one =(1,1,\ldots)$,
		\item
			$B(Tx)=B(x)$ для всех $x\in \ell_\infty$, где $T$~---
		оператор сдвига, т.~е. $T(x_1,x_2,\ldots)=(x_2,x_3,\ldots)$.
	\end{enumerate}
\end{definition}
Множество всех банаховых пределов обозначим через $\mathfrak{B}$.
Существование банаховых пределов было анонсировано С. Мазуром \cite{Mazur} и позднее доказано в книге С.~Банаха~\cite{banach2001theory_rus}.


Лоренц установил, что существуют такие последовательности $x\in\ell_\infty$,
что значение выражения $Bx$ не зависит от выбора $B\in\mathfrak{B}$.
Такие последовательности называются почти сходящимися (англ. \textit{almost convergent}).
Пишут: $x\in ac$.

Лоренц доказал следующий критерий почти сходимости,
который оказывается исключительно удобен при проверке последовательности на принадлежность пространству $ac$.

\begin{theorem}[Критерий Лоренца]
	Для заданных $t\in\R$ и $x\in\ell_\infty$ равенство $Bx=t$ выполнено для всех $B\in\mathfrak{B}$
	тогда и только тогда, когда
	\begin{equation}
		\label{eq:crit_Lorentz}
		\lim_{n\to\infty} \frac{1}{n} \sum_{k=m+1}^{m+n} x_k = t
	\end{equation}
	равномерно по $m\in\N$.
\end{theorem}

Если некоторый $x\in\ell_\infty$ удовлетворяет~\eqref{eq:crit_Lorentz},
то мы будем говорить, что $x$ почти сходится к $t$,
и писать: $x\in ac_t$.
Таким образом, очевидно, $ac = \bigcup\limits_{t\in\R} ac_t$.

Равномерный предел в критерии Лоренца можно заменить на двойной~\cite[Теорема 1]{zvol2022ac}:
\begin{theorem}
	Для заданного $t\in\R$ равенство $Bx=t$ выполнено для всех $B\in\B$
	тогда и только тогда, когда
	\begin{equation}
		\label{eq:crit_Lorentz}
		\lim_{n,m\to\infty} \frac{1}{n} \sum_{k=m+1}^{m+n} x_k = t
		.
	\end{equation}
\end{theorem}

(Заметим, что в общем случае равномерный предел и двойной предел "--- это разные объекты;
за подробными комментариями отсылаем к классическим трудам по математическому анализу,
например,~\cite[с. 154]{kudryavcev2004mathanalys}.)
В настоящей главе в целях удобства доказывается модифицированный критерий Лоренца "--- теорема~\ref{thm:Lorentz_mod}.

Приведём важнейшее следствие из критерия Лоренца, позволяющее в ряде случаев без особых усилий показывать почти сходимость последовательности, которое также содержится в~\cite{lorentz1948contribution}.

\begin{corollary}
	\label{thm:period_ac_avg}
	Всякая периодическая последовательность почти сходится к среднему по периоду.
	Иначе говоря, для любого $B\in\mathfrak B$
	\begin{equation}
		B(x_1,x_2, ..., x_n, \; x_1,x_2, ..., x_n, \; x_1,x_2, ..., x_n, \; x_1, ...) = \frac{x_1+x_2+...+x_n}{n}
		.
	\end{equation}
\end{corollary}

Сачестон~\cite{sucheston1967banach} установил, что
для любых $x\in \ell_\infty$ и $B\in\mathfrak{B}$
\begin{equation}\label{Sucheston}
	q(x) \leqslant Bx \leqslant p(x)
	,
\end{equation}
где
\begin{equation*}
	q(x) = \lim_{n\to\infty} \inf_{m\in\N}  \frac{1}{n} \sum_{k=m+1}^{m+n} x_k
	~~~~\mbox{и}~~~~
	p(x) = \lim_{n\to\infty} \sup_{m\in\N}  \frac{1}{n} \sum_{k=m+1}^{m+n} x_k
	.
\end{equation*}
называют нижним и верхним функционалом Сачестона соотвественно.
Заметим, что $p(x) = -q(-x)$.
Неравенства \eqref{Sucheston} точны:
для данного $x$ для любого $r\in[q(x); p(x)]$ найдётся банахов предел
$B\in\mathfrak{B}$ такой, что $Bx = r$.

Множество таких $x\in\ell_\infty$, что $p(x)=q(x)$, и
образует подпространство почти сходящихся последовательностей $ac$.
Таким образом, функционалы Сачестона являются удобным инструментом для доказательства того,
что некоторая последовательность $x$ не является почти сходящейся:
для этого достаточно показать, что $p(x)\ne q(x)$.

В~\cite[теорема 5]{Jerison} показано, что нижний и верхний функционалы Сачестона могут быть переписаны в эквивалентном виде:
\begin{equation*}
	q(x) = \lim_{n\to\infty} \liminf_{m\to\infty}  \frac{1}{n} \sum_{k=m+1}^{m+n} x_k
	~~~~\mbox{и}~~~~
	p(x) = \lim_{n\to\infty} \limsup_{m\to\infty}  \frac{1}{n} \sum_{k=m+1}^{m+n} x_k
	.
\end{equation*}

Пространство $ac$ имеет интересную структуру.
За глобальным обзором его свойств отсылаем читателя к~\cite{semenov2006ac};
ряд интересных фактов можно почерпнуть в~\cite{usachev2008transforms},
а также в диссертации А.С. Усачёва~\cite{usachev2009_phd_vsu}.
%TODO2: включить англоязычную версию!
Стоит также отметить недавнюю работу~\cite{zvolinsky2021subspace},
в которой исследуется почти сходимость последовательностей,
определённых с помощью тригонометрических функций.

Часто мы будем иметь дело не с пространством всех почти сходящихся последовательностей $ac$,
а с его подпространством $ac_0$ последовательностей, почти сходящихся к нулю.
%TODO2: $ac$ не замкнуто относительно умножения.
%Рассмотрим  1/2, 1/2, 1, 0, 1, 0, 1/2, 1/2, 1/2, 1/2, 1, 0, 1, 0, 1, 0, 1, 0, 1/2, ...
% и          1/2, 1/2, 0, 1, 0, 1, 1/2, 1/2, 1/2, 1/2, 0, 1, 0, 1, 0, 1, 0, 1, 1/2, ...
Пространство $ac_0$ имеет ряд особенностей,
существенно отличающих его от пространства $c_0$ последовательностей, сходящихся к нулю.
То, что $ac_0\ne c_0$, показывает следующий классический
\begin{example}
	Пусть
	\begin{equation}
		x_k = \begin{cases}
			1, & ~\mbox{если}~ k = 2^n, ~ n\in\N,
			\\
			0  & ~\mbox{иначе}.
		\end{cases}
	\end{equation}
	Тогда $x\in ac_0 \setminus c_0$.
\end{example}
Более того, в отличие от пространства $c_0$, пространство $ac_0$ не замкнуто относительно
оператора взятия подпоследовательности, относительно покоординатного умножения и относительно возведения в степень.
Три этих свойства показывает
\begin{example}
	Пусть $x_n = (-1)^{n+1}$,
	т.е. $x = (1, -1,1,-1,1, -1,1,-1,...)$.
	Тогда $x\in ac_0$ в силу периодичности (см. следствие~\ref{thm:period_ac_avg}),
	но, очевидно, $x\cdot x = x^2\in ac_1$.
	Если же мы рассмотрим оператор перехода к подпоследовательности с чётными индексами
	\begin{equation}
		E(y_1, y_2, ...)  = (y_2, y_4, y_6, ...)
		,
	\end{equation}
	то обнаружим, что $Ex = (-1,-1,-1,-1,...)\in ac_{-1}$.
\end{example}

Можно привести и пример, когда взятие подпоследовательности выводит из всего пространства $ac$.
\begin{example}
	Пусть $x_n = (-1)^{n+1}$,
	т.е. $x = (1, -1,1,-1,1, -1,1,-1,...)$.
	Рассмотрим подпоследовательность
	\begin{equation}
		y = (1,-1, \; 1,1, -1,-1, \; 1,1,1, -1,-1,-1, \; 1,1,1,1,...)
		.
	\end{equation}
	Легко видеть, что верхний и нижний функционалы Сачестона принимают на последовательности $y$
	различные значения:
	$p(y) =1$, $q(y) = -1$.
	Следовательно, $y\notin ac$.
\end{example}

Е.А. Алехно доказал~\cite{alekhno2012superposition},
что в $ac_0$ существует максимальный идеал по умножению, обозначаемый $\Iac$ и,
более того, этот идеал может быть ёмко описан следующим критерием:
\begin{theorem}
	\label{thm:Iac_criterion_pos_neg}
	Пусть $x\in ac_0$.
	Последовательность $x\in\Iac$ тогда и только тогда,
	$x = x^+ +x^-$, $x^+\geq 0$, $x^- \leq 0$ и $x^+ \in ac_0$
	(последнее включение эквивалентно условию $x^- \in ac_0$).
\end{theorem}

%TODO2: а что там с идеалом по умножению в ac ?

Более того, $\Iac$ является подпространством в $ac_0$.
%TODO2: что там с недополняемостью?

Е.А. Алехно также исследовал~\cite{alekhno2012superposition,alekhno2015banach,ASSU2}
\emph{стабилизатор} пространства $ac_0$:
\begin{equation}
	\Dac = \{x\in\ell_\infty : x\cdot y \in ac_0 \mbox{~для любого~} y \in ac_0\}
\end{equation}
(встречается~\cite{Luxemburg} также обозначение $\operatorname{St} (ac_0)$).

$\Dac$ также является подпространством (уже в $\ell_\infty$),
однако в настоящей работе в дальнейшем не используется,
и потому мы не будем останавливаться на его свойствах;
отсылаем читателя к~\cite{Luxemburg}.
%TODO2: что там с недополняемостью?

Результаты, излагаемые в данной главе, опубликованы в%
~\cite{
our-mz2019ac0,
avdeed2021AandA,
our-mz2019measure,
}.



	\section{Эквивалентность Дамерау--Левенштейна}
	Этот параграф носит вспомогательный характер.
Основная его цель "--- ввести полезное отношение эквивалентности,
в некоторых случаях упрощающее запись.

В математической лингвистике широко известно расстояние Дамерау--Левенштейна
"---
мера разницы двух строк символов, определяемая как минимальное количество операций вставки,
удаления, замены и транспозиции (перестановки двух соседних символов),
необходимых для перевода одной строки в другую~\cite{damerau1964technique,wagner1974string,gasfield2003strings}.
Расстояние Дамерау--Левенштейна является модификацией расстояния Левенштейна (в сторону невозрастания):
к операциям вставки, удаления и замены символов, определённых в расстоянии Левенштейна~\cite{levenstein1965binary},
добавлена операция транспозиции (перестановки) двух соседних символов.

Эти расстояния действительно являются метриками на множестве всех конечных слов из букв заданного алфавита;
основное их современное приложение "--- это, например, определение опечаток при наборе текста
и выявление помех при передаче алфавитных сигналов~\cite{oommen1997pattern,brill2000improved,bard2006spelling,li2006exploring}.
%TODO2: если нужно, больше ссылок из
%https://en.wikipedia.org/wiki/Damerau%E2%80%93Levenshtein_distance#References


Нам, однако, расстояние Дамерау--Левенштейна интересно в другом контексте.
Совершенно очевидно, что его определение можно обобщить на бесконечные последовательности чисел,
при этом метрикой полученный объект быть перестаёт,
так как может принимать значение, равное бесконечности.

\begin{definition}
	\label{def:Damerau_Levenshein_distance}
	Для $x, y \in \ell_\infty$ будем говорить,
	что расстояние Дамерау--Левенштейна между $x$ и $y$ конечно,
	и писать $x\approx y$,
	если последовательность $x$ можно получить из последовательности $y$
	конечным числом вставок элементов и конечным числом удалений элементов.
\end{definition}

\begin{remark}
	Здесь мы опускаем упоминание операций замены элемента и транспозиции элементов,
	поскольку каждая из этих операций представима в виде комбинации операции вставки и операции удаления.
	Это обусловлено тем, что численное значение самого расстояния нам безразлично "---
	важен лишь факт его конечности.
\end{remark}

\begin{lemma}
	Пусть $B\in\B$, $x\approx y$.
	Тогда $Bx=By$.
\end{lemma}

\begin{proof}
	Предположим сначала, что последовательность $x$ получена из последовательности $y$ удалением одного элемента $y_k$:
	\begin{equation}
		(x_1, x_2, x_3,...) = (y_1, y_2, ..., y_{k-1}, y_{k+1}, ...)
		.
	\end{equation}
	Тогда, очевидно,
	\begin{equation}
		T^{k-1} x = T^{k}y = (y_{k+1}, y_{k+2}, y_{k+3}, ...)
		,
	\end{equation}
	поэтому
	\begin{equation}
		Bx = BT^{k-1}x = BT^{k}y = By
		.
	\end{equation}
	Если же последовательность $x$ получена из последовательности $y$ вставкой одного элемента $x_k$,
	то в вышеприведённых рассуждениях последовательности $x$ и $y$ меняются местами.

	Наконец, по индукции приведённые рассуждения можно повторить для любого числа удалений и вставок,
	то есть для любых последовательностей, расстояние Дамерау--Левенштейна между которыми конечно.
\end{proof}

Более того, очевидна следующая
\begin{lemma}
	Соотношение $x\approx y$ выполнено тогда и только тогда, когда для некоторых $m,n\in\N$ верно равенство $T^m x = T^n y$.
\end{lemma}

Непосредственной проверкой аксиом (рефлексивности, симметричности и транзитивности) легко убедиться,
что отношение $\approx$ является отношением эквивалентности (будем называть его \emph{эквивалентностью Дамерау--Левенштейна})
и разбивает $\ell_\infty$ на классы эквивалентности.

\begin{remark}
	\label{rem:Damerau_vs_c0}
	Эти классы оказываются достаточно узкими.
	Так, например, в разные классы попадут последовательности $x$ и $y$, связанные соотношением $x_n = y_n + \frac1n$,
	хотя $x-y \in c_0$ и для любого $B\in\B$ выполнено $Bx=By$.
\end{remark}

Мы будем пользоваться эквивалентностью Дамерау--Левенштейна, чтобы упростить запись равенств,
в ряде случаев избегая оператора сдвига и пренебрегая конечным количеством элементов последовательности.


	\section{Переформулировка критерия Лоренца почти сходимости последовательности к нулю}
	Результаты этого пункта использованы в~\cite{our-mz2019ac0}.

Дадим переформулировку критерия Лоренца
\cite{lorentz1948contribution,bennett1974consistency}
почти сходимости последовательности,
которая иногда позволяет упростить доказательство:
брать предел равномерно не по всем $m\in \N$,
а только по достаточно большим значениям.


\begin{theorem}
	[Модифицированный критерий Лоренца]
	\label{thm:Lorentz_mod}
	Пусть $x\in\ell_\infty$.

	$x\in ac_0$ тогда и только тогда, когда
	\begin{equation}\label{crit_pos_ac0}
		\forall(A_2\in\N)
		\exists(n_0\in\N)
		\exists(m_0\in\N)
		\forall(n\geq n_0)
		\forall(m\geq m_0)
		\\
		\left[
			\left|
			\frac{1}{n}
			\sum_{k=m+1}^{m+n} x_k
			\right|
			<
			\frac{1}{A_2}
		\right]
		.
	\end{equation}

\end{theorem}

\begin{proof}
	По теореме Лоренца $x\in ac_0$ тогда и только тогда, когда
	\begin{equation}\label{Lorencz_ac0}
		\lim_{n\to\infty} \frac{1}{n} \sum_{k=m+1}^{m+n} x_k = 0
	\end{equation}
	равномерно по $m$.

	Или, переводя на язык кванторов,
	\begin{equation}\label{crit_ac0}
		\forall(A_1\in\N)
		\exists(n_1\in\N)
		\forall(n\geq n_0)
		\forall(m \in \N)
		\\
		\left[
			\left|
			\frac{1}{n}
			\sum_{k=m+1}^{m+n} x_k
			\right|
			<
			\frac{1}{A_1}
		\right]
		.
	\end{equation}
	Очевидно, что из \eqref{crit_ac0} следует \eqref{crit_pos_ac0} (например, положив $m_0 = 1$),
	тем самым необходимость \eqref{crit_pos_ac0} доказана.

	\paragraph{Достаточность.}
	Пусть выполнено \eqref{crit_pos_ac0}.
	Покажем, что выполнено \eqref{crit_ac0}.
	Зафиксируем $A_1$.
	Положим $A_2 = 2A_1$ и отыщем $n_0$ и $m_0$ в соотвествии с \eqref{crit_pos_ac0}.
	Положим $n_1 = 2A_2(n_0+m_0)\|x\|$.
	Покажем, что \eqref{crit_ac0} верно для любых $n\geq n_1$, $m\in \N$.
	Зафиксируем $n$ и рассмотрим $m$.

	Пусть сначала $m\geq m_0$.
	Тогда в силу того, что $n\geq n_1 = 2A_2(n_0+m_0)\|x\| > n_0$ имеем $n>n_0$.
	Применим \eqref{crit_pos_ac0}:
	\begin{equation}
		\left|
		\frac{1}{n}
		\sum_{k=m+1}^{m+n} x_k
		\right|
		<
		\frac{1}{A_2}
		=
		\frac{1}{2A_1}
		<
		\frac{1}{A_1}
		,
	\end{equation}
	т.е. \eqref{crit_ac0} выполнено.

	Пусть теперь $m < m_0$.
	Заметим, что
	\begin{equation}
		\left|
			\sum_{k=m+1}^{m+n} x_k
			-
			\sum_{k=m_0+1}^{m_0+n} x_k
		\right|
		\leq 2(m_0 - m) \|x\|
		,
	\end{equation}
	откуда
	\begin{equation}
		\left| \sum_{k=m+1}^{m+n} x_k \right|
		\leq
		2(m_0 - m) \|x\| + \left| \sum_{k=m_0+1}^{m_0+n} x_k \right|
		\leq
		2 m_0 \|x\| + \left| \sum_{k=m_0+1}^{m_0+n} x_k \right|
		.
	\end{equation}


	Тогда
	\begin{multline}
		\left| \frac{1}{n} \sum_{k=m+1}^{m+n} x_k \right|
		\leq
		\frac{2 m_0 \|x\|}{n} + \left| \frac{1}{n} \sum_{k=m_0+1}^{m_0+n} x_k \right|
		\mathop{\leq}^{\mbox{~~в силу \eqref{crit_pos_ac0}~~}}
		\frac{2 m_0 \|x\|}{n} + \frac{1}{A_2}
		\leq
		\\ \leq
		\frac{2 m_0 \|x\|}{n_1} + \frac{1}{A_2}
		\leq
		\frac{2 m_0 \|x\|}{2A_2(n_0+m_0)\|x\|} + \frac{1}{A_2}
		=
		\\=
		\frac{m_0}{A_2(n_0+m_0)} + \frac{1}{A_2}
		<
		\frac{1}{A_2} + \frac{1}{A_2}
		=
		\frac{1}{A_1}
		,
	\end{multline}
	т.е. \eqref{crit_ac0} тоже выполнено.
\end{proof}

Удобство критерия \eqref{crit_pos_ac0} в том,
что можно выбирать $m_0$ в зависимости от $A_2$.


	\section{О почти сходимости к нулю последовательности из нулей и единиц}
	Результаты этого пункта опубликованы в~\cite{our-mz2019ac0}.

\paragraph{Задача о пасьянсе из нулей и единиц.}

Эта задача основана на идеях, изложенных в \cite[\S 5]{Semenov2014geomprops}.

Пусть $n_i$~--- строго возрастающая последовательность натуральных чисел,
\begin{equation}
	\label{eq:definition_M_j}
	M(j) = \liminf_{i\to\infty} n_{i+j} - n_i,
\end{equation}
\begin{equation}
	x_k = \left\{\begin{array}{ll}
		1, & \mbox{~если~} k = n_i
		\\
		0  & \mbox{~иначе~}
	\end{array}\right.
\end{equation}

\paragraph{Замечание.}
Так как $M(j)$ есть нижний предел последовательности натуральных чисел,
то он всегда достигается,
т.е. $M(j)\in\N$.

Более того, для любого $j$ существует лишь конечное количество отрезков длины $M(j)$,
содержащих более $j$ единиц,
и бесконечное количество отрезков длины $M(j)$,
содержащих ровно $j$ единиц.

Через $E_j$ будем обозначать конец последнего отрезка длины $M(j)$,
содержащего более $j$ единиц.

\begin{lemma}
	\label{thm:lim_M(j)/j_neobh}
	Если $x \in ac_0$, то
	\begin{equation}\label{lim_M(j)/j}
		\lim_{j \to \infty} \frac{M(j)}{j} = +\infty
		.
	\end{equation}
\end{lemma}

\paragraph{Доказательство.}
Очевидно, если
\begin{equation}
	\liminf_{j \to \infty} \frac{M(j)}{j} = +\infty
	,
\end{equation}
то выполнено \eqref{lim_M(j)/j}.
Предположим противное:
\begin{equation}
	\liminf_{j \to \infty} \frac{M(j)}{j} = с < +\infty
	.
\end{equation}
Очевидно, что в таком случае $c>0$.

По определению нижнего предела найдётся счётное множество
$J\subset\N$ такое, что
\begin{equation}
	\forall(j\in J)\left[c \leq \frac{M(j)}{j} \leq c+1 \right],
\end{equation}
т.е. для любого $j\in J$ существует бесконечно много отрезков длины $j\cdot(c+1)$,
на каждом из которых не менее $j$ единиц.

Т.к. $x\in ac_0$, то
\begin{equation}\label{Lorencz_ac0_epsilon}
	\forall(\varepsilon>0)
	\exists(n_0\in\N)
	\forall(n \geq n_0)
	\forall(m\in\N)
	\left[
		\frac{1}{n} \sum_{k=m+1}^{m+n}x_k < \varepsilon
	\right]
	.
\end{equation}

Положим $\varepsilon = 1/(c+2)$ и отыщем $n_0$.
Положим $n\in J$, $n\geq n_0$.
(Такое $n$ всегда найдётся, т.к. $J$ счётно и $J\subset\N$.)
Выберем $m$ так, чтобы отрезок длины $n\cdot(c+1)$,
содержащий не менее $n$ единиц,
начинался с $m+1$.
Тогда
\begin{equation}
	\frac{1}{n\cdot(c+1)}\sum_{k=m+1}^{m+n\cdot(c+1)}x_k
	\geq
	\frac{1}{n\cdot(c+1)} \cdot n
	=
	\frac{1}{c+1}
	>
	\frac{1}{c+2}
	=
	\varepsilon,
\end{equation}
что противоречит \eqref{Lorencz_ac0_epsilon}.

Полученное противоречие завершает доказательство.


\begin{lemma}
	\label{thm:lim_M(j)/j_dost}
	Если
	\begin{equation}\label{lim_M(j)/j_dost}
		\lim_{j \to \infty} \frac{M(j)}{j} = +\infty
		,
	\end{equation}
	то $x \in ac_0$.
\end{lemma}

\paragraph{Доказательство.}

По определению предела \eqref{lim_M(j)/j_dost} означает, что
\begin{equation}\label{lim_M(j)/j_ifty_def}
	\forall(C  \in\N)
	\exists(j_0\in\N)
	\forall(j \geq j_0)
	\left[
		\frac{M(j)}{j}>C
	\right]
	.
\end{equation}
Покажем, что выполнен модификированный критерий Лоренца почти сходимости последовательности к нулю
\eqref{crit_pos_ac0}, т.е.
\begin{equation}
	\forall(B  \in\N)
	\exists(n_0\in\N)
	\exists(m_0\in\N)
	\forall(n\geq n_0)
	\forall(m\geq m_0)
	\\
	\left[
		\frac{1}{n}
		\sum_{k=m+1}^{m+n} x_k
		<
		\frac{1}{B}
	\right]
	.
\end{equation} Действительно, зафиксируем $B$.
Используя \eqref{lim_M(j)/j_ifty_def} и положив $C=2B$,
отыщем $j_0$ такое, что для любого $j\geq j_0$ выполнено
$M(j)>2Bj$.
Положим $n_0 = 2Bj_0$.
Выберем
$$
	m_0 = 2+\max_{1\leq j \leq j_0} E_j
	.
$$

Тогда для любых $m\geq m_0$ и $n\geq n_0$ имеем
\begin{multline}
	\frac{1}{n} \sum_{k=m+1}^{m+n} x_k
	<
	\frac{1}{n} \cdot \left( \frac{n}{M(j_0)} + 1 \right) j_0
	=
	\frac{j_0}{M(j_0)} + \frac{j_0}{n}
	\leq
	\frac{1}{2B} + \frac{j_0}{n}
	\leq
	\\ \leq
	\frac{1}{2B} + \frac{j_0}{n_0}
	=
	\frac{1}{2B} + \frac{1}{2B}
	=
	\frac{1}{B}
	,
\end{multline}
т.е. условие критерия выполнено
и $x\in ac_0$,
что и требовалось доказать.

Последовательность $\{M(j)\}$, как легко выяснить, удовлетворяет некоторым условиям.


	\section{Замечание о свойствах последовательности $M(j)$}
	\begin{lemma}
	Пусть $n_i$~--- строго возрастающая последовательность натуральных чисел,
	\begin{equation}
		C_j = \liminf_{i\to\infty} n_{i+j} - n_i
	\end{equation}

	Тогда для любых $i$, $j$ имеет место быть
	\begin{equation}\label{C_j_addit}
		C_{i+j} \geq C_i + C_j
		.
	\end{equation}
\end{lemma}

\begin{proof}
	По определению нижнего предела существует лишь конечное число номеров $k$
	таких, что $n_{k+i} - n_k < C_i$ или $n_{k+j} - n_k < C_j$.
	Зафиксируем $p$, большее всех таких $k$.

	По определению нижнего предела и с учётом того, что выражение под знаком предела
	принимает лишь натуральные значения,
	существует бесконечное количество номеров $q$ таких, что $n_{q+i+j} - n_q = C_{i+j}$.

	Обозначим через $s$ некоторый такой номер, больший $p$.
	Тогда
	\begin{equation}
		C_{i+j} = n_{s+i+j} - n_s = n_{s+i+j} - n_{s+i} + n_{s+i} - n_s
		%\geq \\
		\geq C_j + n_{s+i} - n_s \geq C_j + C_i,
	\end{equation}
	так как из $s>p$ следует, что $n_{s+i+j} - n_{s+i} \geq C_j$ и $n_{s+i} - n_s \geq C_i$.
\end{proof}

\begin{example}
	Последовательность $C_j = j+1$ не удовлетворяет условию \eqref{C_j_addit}:
	действительно, $3=C_2 < C_1+C_1 = 2+2 = 4$.
\end{example}

\begin{remark}
	Условие \eqref{C_j_addit} является необходимым, но неизвестно, является ли оно достаточным.
\end{remark}

\begin{remark}
	Предел в формулировке лемм~\ref{thm:lim_M(j)/j_neobh}~и~\ref{thm:lim_M(j)/j_dost}
	существует для любой строго возрастающей последовательности натуральных чисел $n_i$.
	Легко видеть, что последовательность $\{M(j)\}_{j=1}^\infty$ вогнута.
	Однако, используя результат из работы \cite{Fekete} (см. также \cite[I, Задача 98]{polia1978zadachi}) получаем,
	что в таком случае предел выражения $M(j)/j$ существует и справедливо равенство
	\begin{equation}
		\lim_{j\to\infty}\frac{M(j)}{j} =\inf_{j\in\N}\frac{M(j)}{j}
		.
	\end{equation}
\end{remark}


	\section{Существование предела последовательности $M(j)$}
	\begin{hypothesis}
	Пусть $M(j)$~--- последовательность натуральных чисел, такая, что существует предел
	\begin{equation}
		\lim_{j\to\infty} \frac{M(j)}{j} \leq +\infty
	\end{equation}
	и для любых $i,j\in\N$ выполнено
	\begin{equation}
		M(i)+M(j) \leq M(i+j)
		.
	\end{equation}
	Тогда существует последовательность $\{x_k\}$,
	удовлетворяющая условию~\eqref{eq:definition_M_j}.
\end{hypothesis}

В качестве итогa параграфа может быть сформулирована следующая

\begin{theorem}
	Пусть $n_i$~--- строго возрастающая последовательность натуральных чисел,
	\begin{equation}
		\label{eq:definition_M_j}
		M(j) = \liminf_{i\to\infty} n_{i+j} - n_i,
	\end{equation}
	\begin{equation}
		x_k = \left\{\begin{array}{ll}
			1, & \mbox{~если~} k = n_i
			\\
			0  & \mbox{~иначе~}
		\end{array}\right.
	\end{equation}
	Тогда следующие условия эквивалентны:
	\\
	(i)   $x \in ac_0$;
	\\\\
	(ii)  $\lim\limits_{j \to \infty} \dfrac{M(j)}{j} = +\infty$;
	\\\\
	(iii) $\inf\limits_{j \in \N}     \dfrac{M(j)}{j} = +\infty$.
\end{theorem}


	\section{Срезочный критерий почти сходимости к нулю неотрицательной последовательности}
	Результаты этого пункта опубликованы в~\cite{our-mz2019ac0}.

Определим (нелинейный) оператор $\lambda$--порога $A_\lambda$ на пространстве $\ell_\infty$.
Для $x = (x_1, x_2, ...) \in \ell_\infty$ положим
\begin{equation}
	(A_\lambda x)_k = \begin{cases}
		1, & \mbox{~если~} x_k \geq \lambda
		\\
		0  & \mbox{~иначе.~}
	\end{cases}
\end{equation}

\begin{theorem}
	[Пороговый критерий почти сходимости к нулю неотрицательной последовательности]
	Пусть $x\in\ell_\infty$, $x\geq 0$.
	Тогда
	\begin{equation}
		x\in ac_0 \Leftrightarrow
		\forall(\lambda>0)[A_\lambda x \in ac_0]
		.
	\end{equation}
\end{theorem}

\paragraph{Необходимость.}
Пусть $x\in ac_0$.
Зафиксируем $\lambda > 0$.
Пусть $y=A_\lambda x$, тогда
\begin{equation}
	(\lambda y)_k = \begin{cases}
		\lambda, & \mbox{~если~} x_k \geq \lambda
		\\
		0  & \mbox{~иначе~}
	\end{cases}
\end{equation}
Таким образом, $0 \leq \lambda y \leq x$.
Следовательно, если $x \in ac_0$,
то $\lambda y \in ac_0$ и $y \in ac_0$,
что и требовалось доказать.

\paragraph{Достаточность.}
Очевидно, что
\begin{equation}
	x\in ac_0 \Leftrightarrow
	\frac{x}{\|x\|}\in ac_0
	.
\end{equation}
Поэтому, не теряя общности, будем полагать $\|x\|\leq 1$.
Более того,
\begin{equation}
	A_\lambda x \in ac_0 \Leftrightarrow
	(1-\lambda)A_\lambda x \in ac_0
	.
\end{equation}

Предположим противное, т.е. $x\notin ac_0$,
но $\forall(\lambda>0)[(1-\lambda)A_\lambda x \in ac_0]$.
Запишем покванторное отрицание критерия Лоренца:
\begin{equation}\label{ac0_lambda_Lorencz_neg}
	\exists(\varepsilon_0 > 0)
	\forall(n_0 \in \N)
	\exists(n > n_0)
	\exists(m \in \N)
	\left[
		\frac{1}{n}\sum_{k=m+1}^{m+n} x_k \geq \varepsilon_0
	\right]
	.
\end{equation}
Найдём такое $\varepsilon_0$ и положим
\begin{equation}
	\varepsilon = \min\left\{ \frac{\varepsilon_0}{2}, \frac{1}{2} \right\}
	.
\end{equation}
Легко видеть, что
\begin{equation}\label{ac0_lambda_Lorencz_neg_epsilon}
	\forall(n_0 \in \N)
	\exists(n > n_0)
	\exists(m \in \N)
	\left[
		\frac{1}{n}\sum_{k=m+1}^{m+n} x_k > \varepsilon
	\right]
	.
\end{equation}
(знак неравенства сменился на строгий, это будет играть ключевую роль в дальнейших выкладках).

Построим последовательность
\begin{equation}
	y = \left( 1 - \frac{\varepsilon}{2} \right) A_{\varepsilon/2} x
	.
\end{equation}
Заметим, что $y\in ac_0$, т.е. по критерию Лоренца
\begin{equation}\label{ac0_lambda_Lorencz}
	\forall(\varepsilon_1 > 0)
	\exists(n_1 \in \N)
	\forall(n' > n_1)
	\forall(m' \in \N)
	\left[
		\frac{1}{n}\sum_{k=m+1}^{m+n} y_k < \varepsilon_1
	\right]
	.
\end{equation}

Положим в \eqref{ac0_lambda_Lorencz} $\varepsilon_1 = \varepsilon/2$
и отыщем $n_1$.
Положим в \eqref{ac0_lambda_Lorencz_neg_epsilon}
$n_0 = n_1$ и отыщем $n$ и $m$.
Положим в \eqref{ac0_lambda_Lorencz} $n' = n$, $m' = m$.
Тогда получим, что по \eqref{ac0_lambda_Lorencz_neg_epsilon}
\begin{equation}
	\frac{1}{n}\sum_{k=m+1}^{m+n} x_k > \varepsilon
	.
\end{equation}
С другой стороны, по \eqref{ac0_lambda_Lorencz}
\begin{equation}
	\frac{1}{n}\sum_{k=m+1}^{m+n} y_k < \varepsilon/2
	.
\end{equation}
Вычитая, получим
\begin{equation}
	\frac{1}{n}\sum_{k=m+1}^{m+n} (x_k - y_k) > \varepsilon/2
	.
\end{equation}
Если среднее арифметическое чисел вида $x_k - y_k$ больше $\varepsilon/2$,
то существует хотя бы один индекс $k$ такой, что $x_k - y_k > \varepsilon/2$.

Предположим, что $k$ таково, что $x_k < \varepsilon/2$.
Тогда $y_k = 0$ и $x_k - y_k < \varepsilon/2$.
Значит, предположение неверно и $x_k \geq \varepsilon/2$.
Тогда $y_k = 1-\varepsilon/2$ и с учётом $\|x\|\leq 1$ имеем
\begin{equation}
	x_k - y_k \leq 1- y_k = 1 - (1-\varepsilon/2) = \varepsilon/2
	.
\end{equation}
Следовательно, требуемого индекса $k$ не существует,
и \eqref{ac0_lambda_Lorencz_neg} не выполнено.

Полученное противоречие завершает доказательство.

Сформулируем теперь этот критерий в терминах функций $M^{(\lambda)}(j)$.

\begin{theorem}
	\label{thm:crit_ac0_Mj_lambda}
	Пусть $x\in\ell_\infty$, $x \geq 0$, $\lambda>0$.
	Обозначим через $n^{(\lambda)}_i$ возрастающую последовательность
	индексов таких элементов $x$, что $x_k \geq \lambda$ тогда и только тогда,
	когда $k=n^{(\lambda)}_i$ для некоторого $i$.
	Обозначим
	\begin{equation}
		M^{(\lambda)}(j) = \liminf_{i\to\infty} n^{(\lambda)}_{i+j} - n^{(\lambda)}_i
		.
	\end{equation}


	Тогда для того, чтобы $x\in ac_0$, необходимо и достаточно, чтобы
	для любого $\lambda>0$ было выполнено
	\begin{equation}
		\lim_{j \to \infty} \frac{M^{(\lambda)}(j)}{j} = +\infty
		.
	\end{equation}
\end{theorem}


	\section{Пространство $ac_0$ и $\alpha$--функция}
	Докажем теперь ещё одну теорему,
раскрывающую связь между сходимостью, почти сходимостью и $\alpha$--функцией.

\begin{theorem}
	\label{thm:alpha_c_ac_c}
	Пусть $x\in ac_0$ и $\alpha(x)=0$.
	Тогда $x \in c_0$.
\end{theorem}

\begin{proof}
	Предположим противное, т.е. $x\notin c_0$.
	Тогда
	\begin{equation}
		\varlimsup_{k\to\infty} x_k \neq 0 ~~\mbox{или}~~ \varliminf_{k\to\infty} x_k \neq 0
		.
	\end{equation}

	Не теряя общности, положим
	\begin{equation}
		\varepsilon = \varlimsup_{k\to\infty} x_k > 0
	\end{equation}
	(иначе домножим всю последовательность на $-1$, что, очевидно, не повлияет на сходимость к нулю).

	Тогда существует бесконечно много таких $n$, что
	\begin{equation}\label{alpha_ac0_c0_limsup}
		x_n > \varlimsup_{k\to\infty} x_k - \frac{\varepsilon}{4} = \frac{3\varepsilon}{4}
		.
	\end{equation}

	Так как
	\begin{equation}
		\alpha(x) = \varlimsup_{i\to\infty} \max_{i \leq j \leq 2i} |x_i-x_j| = 0
		,
	\end{equation}
	то
	\begin{equation}
		\exists(N_1\in\N)\forall(n > N_1)\left[\max_{n \leq j \leq 2n} |x_n-x_j| < \frac{\varepsilon}{4}\right]
		,
	\end{equation}
	или, что то же самое,
	\begin{equation}\label{alpha_ac0_c0_alpha}
		\exists(N_1\in\N)\forall(n > N_1)\forall(j: n \leq j \leq 2n)\left[ |x_n-x_j| < \frac{\varepsilon}{4}\right]
		.
	\end{equation}

	Поскольку $x \in ac_0$, то  по критерию Лоренца
	\begin{equation}
		\exists(N_2 \in\N)\forall(n > N_2)\forall(m\in\N)
		\left[ \left| \frac{1}{n}\sum_{k=m+1}^{m+n}x_k\right| < \frac{\varepsilon}{4} \right]
		,
	\end{equation}
	в частности,
	\begin{equation}\label{alpha_ac0_c0_Lorencz}
		\exists(N_2 \in\N)\forall(n > N_2)
		\left[ \left| \frac{1}{n}\sum_{k=n+1}^{2n}x_k\right| < \frac{\varepsilon}{4} \right]
		,
	\end{equation}

	Выберем $n$ так, чтобы оно удовлетворяло \eqref{alpha_ac0_c0_limsup} и \eqref{alpha_ac0_c0_alpha}.
	Тогда
	\begin{multline}
		\left| \frac{1}{n}\sum_{k=n+1}^{2n}x_k\right|
		=
		\\=
		\mbox{(по \eqref{alpha_ac0_c0_alpha} имеем $x_k \geq 3\varepsilon/4 > 0$)}
		=
		\\=
		\frac{1}{n}\sum_{k=n+1}^{2n}x_k
		\geq
		\frac{3\varepsilon}{4}
		>
		\frac{\varepsilon}{4}
		,
	\end{multline}
	что противоречит \eqref{alpha_ac0_c0_Lorencz}.

	Полученное противоречие завершает доказательство.
\end{proof}


\begin{corollary}
	Пусть $x\in ac$ и $\alpha(x)=0$.
	Тогда $x \in c$.
\end{corollary}

%TODO2: доказывать нужно или очевидно?

Однако, как выясняется, справедлив и более общий результат.
Мы опишем его в следующем параграфе.


	\section{$\alpha$--функция на пространстве $ac$ и расстояние до пространства $c$}
	Пусть $\rho(x,c)$ и $\rho(x,c_0)$~--- расстояния от $x$ до пространства сходящихся последовательностей $c$
и пространства сходящихся к нулю последовательностей $c_0$ соответственно.

Из условия Липшица на $\alpha$--функцию \eqref{alpha_Lipshitz}
и того, что на пространстве $c$
всех сходящихся последовательностей
$\alpha$--функция обращается в нуль следует

\begin{lemma}
\label{thm:alpha_x_leq_2_rho_x_c}
	Для любого $x\in\ell_\infty$
	\begin{equation}
		\alpha(x) \leq 2\rho(x, c)
		.
	\end{equation}
\end{lemma}

%TODO2: доказывать или очевидно?

Эта оценка точна.
\begin{example}
\label{ex:alpha_ac_rho_x_c}
	\begin{equation}
	\label{eq:alpha_ac_distance_example_y}
		x_k = \begin{cases}
			(-1)^n, &\mbox{~если~} k = 2^n,
			\\
			0 &\mbox{~иначе.}
		\end{cases}
	\end{equation}
\end{example}
Здесь $\alpha(x) = 2$, $\alpha(T^s x) = 1$ для любого $s\in\N$, $\rho(x,c) = \rho(x, c_0) = 1$.

Как выясняется, верна и оценка с другой стороны.

\begin{lemma}
\label{thm:rho_x_c_leq_alpha_t_s_x}
	Для любого $x\in ac$ и для любого натурального $s$
	\begin{equation}
		\rho(x,c)\leq \alpha(T^s x)
		.
	\end{equation}
\end{lemma}

\begin{proof}
	Зафиксируем $s$.
	Пусть $x\in ac$, $\alpha(T^s x)=\varepsilon$.
	Так как $x\in ac$, то по критерию Лоренца существует такое число $t$,
	что
	\begin{equation}
		\lim_{n\to\infty} \frac{1}{n} \sum_{k=m+1}^{m+n} x_k = t
	\end{equation}
	равномерно по $m$.
	Иначе говоря,
	\begin{equation}
		\forall(p\in\N)
		\exists(n_p \in\N)\forall(n>n_p)\forall(m\in\N)
		\left[
			\left|
				\frac{1}{n}\sum_{k=m+1}^{m+n} x_k
				-t
			\right|
			<\frac{\varepsilon}{p}
		\right]
		.
	\end{equation}

	Так как
	\begin{equation}
		\alpha(T^s x) = \varlimsup_{k\to\infty} \max_{k<j\leq 2k-s} |x_k-x_j| = \varepsilon
		,
	\end{equation}
	то
	\begin{equation}
		\forall(p\in\N)
		\exists(k_p \in\N)\forall(k>k_p)
		\forall(j: k< j \leq 2k-s)
		\left[
			|x_k - x_j|<\varepsilon + \frac{\varepsilon}{p}
		\right]
		.
	\end{equation}
	Положив $q_p = \max\{n_p, k_p\}$, имеем
	\begin{multline}
		\forall(p\in\N)
		\exists(q_p \in\N)
		\forall(q>q_p)
		\left[
			\left|
				\frac{1}{n}\sum_{k=m+1}^{m+n} x_k
				-t
			\right|
			<\frac{\varepsilon}{p}
			\right.\\ \left. \phantom{\sum_0^0}
			\mbox{~~и~~}
			\forall(k:q<k \leq 2q-s)
			\left[
				|x_q - x_k|<\varepsilon + \frac{\varepsilon}{p}
			\right]
		\right]
		.
	\end{multline}
	Т.е. среднее арифметическое чисел $x_q$, $x_{q+1}$, ... , $x_{2q-s}$ отличается от $t$
	не более, чем на $\varepsilon/p$, причём разница любых двух из этих чисел меньше $\varepsilon + \varepsilon/p$.
	Следовательно, любое из чисел $x_q$, $x_{q+1}$, ... , $x_{2q-s}$
	отличается от $t$ менее, чем на $\varepsilon + 2\varepsilon/p$.
	В частности,
	\begin{equation}
		|x_q - t| < \varepsilon + \frac{2\varepsilon}{p}
		.
	\end{equation}
	Таким образом,
	\begin{equation}
		\forall(p\in\N)
		\exists(q_p \in\N)
		\forall(q>q_p)
		\left[
			|x_q - t| < \varepsilon + \frac{2\varepsilon}{p}
		\right]
		,
	\end{equation}
	откуда немедленно следует, что $\rho(x,c) \leq \varepsilon$,
	что и требовалось доказать.
\end{proof}


Из лемм \ref{thm:alpha_x_leq_2_rho_x_c} и \ref{thm:rho_x_c_leq_alpha_t_s_x}
незамедлительно следует
\begin{theorem}
	\label{thm:rho_x_c_leq_alpha_t_s_x_united}
	Для любого $x\in ac$
	\begin{equation}
		\frac{1}{2} \alpha(x) \leq \rho(x,c)\leq \lim_{s\to\infty} \alpha(T^s x)
		.
	\end{equation}
\end{theorem}

\begin{corollary}
	\label{cor:rho_x_c0_leq_alpha_t_s_x_united}
	Для любого $x\in ac_0$
	\begin{equation}
		\frac{1}{2} \alpha(x) \leq \rho(x,c_0)\leq \lim_{s\to\infty} \alpha(T^s x)
		.
	\end{equation}
\end{corollary}

Точность оценок показывают пример \ref{ex:alpha_ac_rho_x_c} и следующие примеры.
\begin{example}
	\begin{equation}
		x_k = \begin{cases}
			1, &\mbox{~если~} k = 2^n,
			\\
			0 &\mbox{~иначе.}
		\end{cases}
	\end{equation}
\end{example}
Здесь $\alpha(T^s x) = 1$ для любого $s\in\N$, $\rho(x,c) = 1/2$, $\rho(x, c_0) = 1$.

\begin{example}
	\begin{equation}
		x_k = \begin{cases}
			1, &\mbox{~если~} k = 2^n,
			\\
			-1, &\mbox{~если~} k = 2^n + 1 > 2
			\\
			0 &\mbox{~иначе.}
		\end{cases}
	\end{equation}
\end{example}
Здесь $\alpha(T^s x) = 2$ для любого $s\in\N$, $\rho(x,c) = \rho(x, c_0) = 1$.


\begin{hypothesis}
	Для любого $x\in ac$
	\begin{equation}
		\frac{1}{2} \lim_{s\to\infty} \alpha(U^s x) \leq \rho(x,c)
		,
	\end{equation}
	где $U$~--- оператор сдвига вправо:
	\begin{equation}
		U(x_1, x_2, x_3, ...) = (0, x_1, x_2, x_3, ...)
		.
	\end{equation}
\end{hypothesis}


	\section{Усиленная теорема Коннора}
	Пусть на множестве $\Omega=\{0,1\}^\N$ задана вероятностная мера <<честной монетки>> $\mu$.
Тогда, согласно~\cite{connor1990almost}, $\mu(\Omega\cap ac)=0$.

Обобщим этот результат.
\begin{theorem}
	Мера множества $F=\{x\in\Omega : q(x) = 0 \wedge p(x)= 1\}$,
	где $p(x)$ и $q(x)$~--- верхний и нижний функционалы Сачестона соответственно,
	равна 1.
\end{theorem}

\begin{proof}
	Пусть $F_1=\{x\in\Omega : p(x) \neq 1\}$, $F_0=\{x\in\Omega : q(x) \neq 0\}$.
	Заметим, что
	\begin{equation}
		\label{eq:Connor_gen}
		\mu F = \mu (\Omega\setminus(F_1 \cup F_0)) = 1 - \mu (F_1 \cup F_0)
		.
	\end{equation}
	Докажем, что $\mu F_1 = 0 $	(для $\mu F_0$  доказательство полностью аналогично).

	Согласно критерию Сачестона,
	\begin{equation}
		p(x) = \lim_{n\to\infty} \sup_{m\in\N} \frac{1}{n} \sum_{k=m+1}^{k=m+n} x_k
		,
	\end{equation}
	откуда следует, что
	для $x\in\Omega$ равенство $p(x) = 1$ выполнено тогда и только тогда,
	когда для любого $n$ последовательность $x$ содержит отрезок из $n$ единиц подряд.
%
	Следовательно, если $x\in F_1$,
	то существует такое $n$,
	что $x$ не содержит $n$ единиц подряд.
%
%
	Обозначим
	\begin{equation}
		A_n^k = \{x\in\Omega : x_{kn+1} = ... = x_{kn+n} = 1\}
		,
		~~~
		B_n^k = \Omega \setminus A_n^k
		%,
		%B_n = \bigcap_{k\in\N} B_n^k
		.
	\end{equation}
	Тогда
	\begin{equation}
		\forall(x\in F_1)\exists(n_x\in\N)\forall(k\in\N)[x\in B_{n_x}^k]
		,
	\end{equation}
	т.е.
	\begin{equation}
		F_1 \subset \bigcup_{n\in\N} \bigcap_{k\in\N} B_{n}^k
		.
	\end{equation}
	Учитывая, что при $i\neq j$ события $x\in B_n^i$ и $x\in B_n^j$ независимы и $\mu B_n^j = 1-\frac{1}{2^n}$,
	получаем
	\begin{multline}
		\mu F_1 \leq \mu \bigcup_{n\in\N} \bigcap_{k\in\N} B_{n}^k
		=
		\sum_{n\in\N} \mu \bigcap_{k\in\N} B_{n}^k
		=
		\sum_{n\in\N}  \prod_{k\in\N} \mu B_{n}^k
		=
		\sum_{n\in\N}  \prod_{k\in\N} \left( 1-\frac{1}{2^n} \right)
		=0
		.
	\end{multline}
	Тем самым получаем из \eqref{eq:Connor_gen}, что $\mu F = 1$.
\end{proof}

\begin{remark}
	Примечательно, что в случае тройных последовательностей дело обстоит иначе.
	Так, в работе~\cite{esi2014almost} доказывается, что почти все тройные последовательности из нулей и единиц почти сходятся.
\end{remark}


	\section{О мере одного множества}
	Пусть на множестве $\Omega=\{0,1\}^\N$ задана вероятностная мера <<честной монетки>> $\mu$.
Тогда, согласно~\cite{connor1990almost}, $\mu(\Omega\cap ac)=0$.
Применим этот результат к нахождению меры множества $W$,
введённого в~\cite[\S 5]{Semenov2014geomprops}.

Пусть $W$~--- множество всех последовательностей $\chi_e$, где $e =\bigcup_{k=1}^{\infty} [n_{2k-1}, n_{2k} )$
и $\{n_k \}_{k=1}^{\infty}$
удовлетворяет условию
\begin{equation}
	\label{eq:lim_j_n_kj_measure}
	\lim_{j\to\infty}\frac{n_{k+j} - n_k}{j} = \infty
\end{equation}
равномерно по $k \in \N$.

Вероятностная мера <<честной монетки>> означает,
что каждая последовательность из нулей и единиц соответствует бесконечной серии
бросков честной монетки, причём выпадение орла означает нуль, а выпадение решки~--- единицу.
%(В практических целях рекомендуем читателю использовать рубль, поскольку он падает быстрее и охотнее.)
Тогда мера подмножества $\omega\subset\Omega$
равна вероятности события <<выпала одна из серий монетки, закодированных в $\omega$>>.
Например, $\mu(\Omega)=1$, $\mu(\{x\in\Omega:x_1=1, x_2=0\})=1/4$.

Введём теперь нелинейную биекцию $Q:\Omega\leftrightarrow\Omega$ по следующему правилу:
\begin{equation}
	(Qx)_k = \begin{cases}
		x_k, &\mbox{если~} k = 1,
		\\
		|x_k-x_{k-1}|&\mbox{иначе}.
	\end{cases}
\end{equation}

\begin{remark}
	Может показаться, что оператор $Q$ очень сходен с оператором границы последовательности $\operatorname{bd}$
	~\cite{keller1992invariant},
	однако это не так.
\end{remark}

\begin{example}
	\begin{equation}
		\begin{array}{rl}
			                   x &= (0,0,0,1,1,1,0,0,0,0,0,1,1,1,1,1,0,...)
			\\
			\operatorname{bd}(x) &= (0,0,0,1,0,1,0,0,0,0,0,1,0,0,0,1,0,...)
			\\
			                  Qx &= (0,0,0,1,0,0,1,0,0,0,0,1,0,0,0,0,1,...)
		\end{array}
	\end{equation}
\end{example}

\begin{example}
	Но всё меняется, когда в последовательности достаточно часто встречаются блоки $(...,0,1,0,...)$
	(а по закону больших чисел они будут встречаться с большой вероятностью).
	\begin{equation}
		\begin{array}{rl}
			                   x &= (0,0,1,1,0,1,0,1,0,0,0,...)
			\\
			\operatorname{bd}(x) &= (0,0,1,1,0,1,0,1,0,0,0,...)
			\\
			                  Qx &= (0,0,1,0,1,1,1,1,1,0,0,...)
		\end{array}
	\end{equation}
\end{example}


\begin{lemma}
	Биекция $Q$ сохраняет меру множества.
\end{lemma}

\paragraph{Идея.}
	Пусть последовательность $x$ соответствует серии бросков монетки так, как описано выше.
	Будем теперь интерпретировать ту же самую серию бросков иначе.
	Первый бросок интерпретируем так же,
	а начиная со второго сопоставим нулю событие <<выпала та же сторона монетки, что и в прошлый раз>>,
	а единице~--- противоположное событие.
	Cобытия <<выпала решка>> и <<выпала та же сторона монетки, что и в прошлый раз>> независимы
	и вероятность каждого из них равна $1/2$.
	Осталось заметить, что новая интерпретация той же серии бросков дала нам последовательность $Qx$.
\begin{proof}
	В рамках данного доказательства будем соотносить последовательности из нулей и единиц
	с точками полуинтервала $[0;1)$, представленными в виде двоичных дробей.
	Назовём двоичным отрезком множество последовательностей из $\Omega$,
	в котором первые $k$ координат зафиксированы, а остальные выбираются произвольно.
	Двоичный отрезок действительно соответствует отрезку длины $1/2^k$,
	состоящему из двоичных дробей, в которых первые $k$ цифр после запятой зафиксированы,
	а остальные выбираются произвольно.

	Заметим, что $Q$ отображает двоичный отрезок длины $1/2^k$ в двоичный отрезок длины $1/2^k$.
	Так как любой отрезок может быть представлен в виде объединения не более чем счётного числа
	двоичных отрезков, то $Q$ сохраняет меру любого отрезка.
	В силу биективности $Q$ отсюда следует, что $Q$ сохраняет меру любого борелевского множества
	(так как сигма-алгебра борелевских множеств порождается отрезками).
\end{proof}

\begin{remark}
	На самом деле соответствие точек отрезка и последовательностей из нулей и единиц не однозначно,
	а почти однозначно: например, точка $1/2$ может быть с равным успехом
	представлена как $x_1=0.01111...$ и как $x_2=0.10000...$.
	Однако $Qx_1 = 0.010000...$, а $Qx_2=0.100000$.

	%TODO2: аккуратно объяснить этот момент.

	%Усачёв: неоднозначность разложения двоично-рациональных чисел нужно обсудить до введения оператора Q.
	%Иначе он неоднозначно определен.

	Представляется целесообразным отказаться не от записи $x_1$ (как это обычно принимается),
	а от записи $x_2$, поскольку все последовательности, стабилизирующиеся на нуле,
	биекция $Q$ переводит в последовательности, стабилизирующиеся на нуле.
	Более того, мера множества всех последовательностей, стабилизирующихся на нуле, равна нулю,
	поэтому их исключение из рассмотрения не мешает доказательству.
\end{remark}

Несмотря на это, дадим здесь ещё одно
\begin{proof}
	Доказательство из~\cite{connor1990almost} использует наименьшую $\sigma$--алгебру $\mathcal{S}$,
	содержащую множества
	\begin{equation}
		\Omega_i = \{x\in\Omega: x_i = 1\}
		.
	\end{equation}
	Очевидно, что $\mathcal{S}$~--- это также и наименьшая $\sigma$--алгебра, содержащая множества
	$\Omega^*_{i,j}$ последовательностей, в которых зафиксированы первые $i$ координат.
	Как мы уже знаем, $\mu(\Omega^*_{i,j}) = \mu(Q\Omega^*_{i,j})$.
	Так как $Q$~--- биекция, то $\mu^*(A) = \mu(QA)$~--- мера.
	Действтельно, неотрицательность $\mu^*(A)$ очевидна.
	Покажем выполнение аксиомы счётной аддитивности.
	Пусть $A_i \cap A_j = \varnothing$ при $i\neq j$,
	тогда в силу биективности $Q$ имеем $QA_i \cap QA_j = \varnothing$ и
	\begin{equation}
		\mu^*\left( \bigcup_i A_i \right)
		=
		\mu\left( Q \bigcup_i A_i \right)
		=
		\mu\left( \bigcup_i QA_i \right)
		=
		\sum_i 	\mu( QA_i )
		=
		\sum_i 	\mu^*( A_i )
		.
	\end{equation}
	Согласно теореме Каратеодори,
	если две меры совпадают на некоторой системе подмножеств $\mathcal{S}^*$,
	то они совпадают и на минимальной $\sigma$--алгебре, содержащей $\mathcal{S}^*$.
	Следовательно, биекция $Q$ действительно сохраняет меру.
\end{proof}

%NB: это скользкий момент.
%Не всякая биекция сохраняет меру,
%но тут вроде как спасает то, что мы интерпретируем ту же самую последовательность бросков двумя разными способами.

Равномерное стремление~\eqref{eq:lim_j_n_kj_measure}
буквально означает, что
\begin{equation}
	\forall(A>0)\exists(j_A\in\N)\forall(j > j_A)\forall(k\in\N)
	\left[
		\frac{n_{k+j}-n_k}{j}>A
	\right]
	.
\end{equation}
Так как последнее неравенство выполнено для любого $k$,
то оно верно и для нижнего предела по $k$:
\begin{equation}
	\forall(A>0)\exists(j_A\in\N)\forall(j > j_A)
	\left[
		\liminf_{k\to\infty}\frac{n_{k+j}-n_k}{j}>A
	\right]
	.
\end{equation}
<<Свернув>> определение предела по $j$ из кванторной записи, получаем:
\begin{equation}
	\lim_{j\to\infty}\liminf_{k\to\infty}\frac{n_{k+j}-n_k}{j} = \infty
	,
\end{equation}
откуда по лемме~\ref{thm:lim_M(j)/j_dost} немедленно имеем $QW \subset \Omega\cap ac_0$.
Тогда
\begin{equation}
	0 \leq \mu(W) = \mu(QW) \leq \mu(\Omega\cap ac_0) \leq \mu(\Omega\cap ac) = 0
	,
\end{equation}
откуда $\mu(W)=0$.

\begin{lemma}
	Включение $QW\subset \Omega\cap ac_0$ собственное.
\end{lemma}

\begin{proof}
	Пусть
	\begin{equation}
		x = (1,0,0,...,0,0,...),
	\end{equation}
	тогда
	\begin{equation}
		Qx = (1,1,0,0,...,0,0,...) \notin W
		.
	\end{equation}
\end{proof}

\begin{hypothesis}
	Включение $QW\subset (\Omega\cap ac_0) \setminus c_{00}$, где через $c_{00}$ обозначается множество последовательностей,
	стабилизирующихся на нуле, собственное.
\end{hypothesis}

\begin{lemma}
	$Q^{-2} ac_0 \not \subset ac_0$.
\end{lemma}

\begin{proof}
	Пусть $x=(1,0,1,0,...)$.
	Тогда $Qx = \one$ и $Q^2 x = (1,0,0,0,...) \in c_{00} \subset ac_0$,
	хотя $x\notin ac_0$.
\end{proof}

\begin{hypothesis}
	$Q^{-2} ac_0 \subset ac$.
\end{hypothesis}


	\section{Пространство $ac_0$ и возведение в степень}
	Вопрос о покоординатном возведении в степень почти сходящейся к нулю последовательности сколь-либо системно впервые поднят в~\cite{zvol2022ac}.
Возведение в отрицательную степень может выводить из пространства ограниченных последовательностей $\ell_\infty$ вовсе;
например, очевидна следующая
\begin{lemma}
	Пусть $x\in c_0$, $\alpha< 0$ и $x_k\ne 0 $ для любого $k$.
	Тогда
	\begin{equation*}
		x^\alpha = (x_1^\alpha,x_2^\alpha,x_3^\alpha,...) \notin \ell_\infty
		.
	\end{equation*}
\end{lemma}
(Здесь и далее мы полагаем, что возведение в соответствующую степень определено однозначно и в действительных числах;
то есть, если мы пишем $x_k^\alpha$, то мы неявно предполагаем, что значение этого выражения корректно определено.)


Для $x\in ac_0\setminus c_0$ ситуация, вообще говоря, не столь однозначна.

\begin{example}
	\label{example:ac0_pow_signum_classic}
	Рассмотрим классическую почти сходящуюся к нулю последовательность
	\begin{equation}
		x = (1;-1;1;-1;1;-1;...) \in ac_0
		.
	\end{equation}
	Очевидно, что для любой целой нечётной отрицательной степени $\alpha$ имеем $x^\alpha = x \in \ell_\infty$,
	однако для любой целой чётной отрицательной степени $\alpha$ мы получаем $x^\alpha = (1;1;1;1;1;1;...) \in \ell_\infty$.
\end{example}

\begin{example}
	Рассмотрим почти сходящуюся к нулю последовательность
	\begin{equation}
		y = \left(1;-1;\frac12;-\frac12;1;-1;\frac13;-\frac13;1;-1;\frac14;-\frac14;1;-1;...\right) \in ac_0
		.
	\end{equation}
	Очевидно, что
	\begin{equation}
		y^{-1} = \left(1;-1;2;-2;1;-1;3;-3;1;-1;4;-4;1;-1;...\right)  \notin \ell_\infty
		.
	\end{equation}
\end{example}

Возведение в отрицательную степень мы более обсуждать не будем.

Итак, в недавней статье Р.Е. Зволинского~\cite{zvol2022ac} доказаны два следующих факта (теорема 3 и следствие 2 соответственно):
\begin{theorem}
	\label{thm:Zvol_pow_pos}
	Пусть $x \geqslant 0, x \in a c_0$ и $\alpha>0$, тогда $x^\alpha \in a c_0$.
\end{theorem}

\begin{theorem}
	\label{thm:Zvol_pow_composed}
	Пусть $x\in ac_0$, $x = x^+ +x^-$, $x^+\geq 0$, $x^- \leq 0$ и $x^+ \in ac_0$
	(последнее включение эквивалентно условию $x^- \in ac_0$).
	Пусть $n\in\mathbb{N}$ или $n = \frac1{2k+1}$, $k\in\mathbb{N}$.
	Тогда $x^n \in ac_0$.
\end{theorem}

Возникает закономерный вопрос о том, что происходит при возведении в степень почти сходящейся к нулю последовательности,
которую нельзя разложить в сумму знакопостоянных почти сходящихся к нулю последовательностей.
Следующая теорема показывает, что условия теоремы~\ref{thm:Zvol_pow_composed} существенны.

\begin{theorem}
	\label{thm:ac0_pow_even}
	Пусть $x\in ac_0$, $x = x^+ +x^-$, $x^+\geq 0$, $x^- \leq 0$ и $x^+ \notin ac_0$
	(последнее условие эквивалентно условию $x^- \notin ac_0$).
	Пусть $n = 2k$, $k\in\mathbb{N}$.
	Тогда $x^n \notin ac_0$.
\end{theorem}

\begin{proof}
	Рассмотрим последовательность $y = (x^+)^n \geq 0$.

	Предположим, что $y \in ac_0$.
	Тогда $x^+ = y^{1/n}$ и по теореме~\ref{thm:Zvol_pow_pos} выполнено $x^+\in ac_0$,
	что противоречит условию доказываемой теоремы.
	Значит, $y = (x^+)^n \notin ac_0$ и выполнено неравенство $p\left((x^+)^n\right) > 0$.

	Заметим теперь, что  в силу чётности $n$ выполнено $(x^-)^n \geq 0$, откуда $p\left((x^-)^n\right) \geq 0$.
	(Строго говоря, можно по аналогии с $(x^+)^n$ показать, что $(x^-)^n\notin ac_0$ и, следовательно, $p\left((x^-)^n\right) > 0$.)
	В силу построения $x^+$ и $x^-$ мы имеем $\supp x^+ \cap \supp x^- = \varnothing$,
	откуда
	\begin{equation}
		x^n = (x^+ + x^-)^n = (x^+)^n + (x^-)^n
		.
	\end{equation}
	Очевидно, что для любых ограниченных последовательностей $a\geq0$, $b\geq 0$ выполнено неравенство для верхнего функционала Сачестона $p(a+b) \geq p(a)$.
	Отсюда получаем
	\begin{equation}
		p(x^n) = p\left((x^+)^n + (x^-)^n\right) \geq p\left((x^+)^n\right) > 0
		,
	\end{equation}
	что по теореме Сачестона означает, что $x^n \notin ac_0$.
\end{proof}

Для нечётной степени условие разложение в сумму двух знакопостоянных почти сходящихся последовательностей в теореме~\ref{thm:Zvol_pow_composed} тоже существенно.

\begin{example}
	Напомним,
	%TODO: ссылка!
	что любая периодическая последовательность почти сходится к своему среднему по периоду
	(см. следствие~\ref{thm:period_ac_avg}).
	Пусть
	\begin{equation}
		x = (1;1;-2;\ 1;1;-2;\ 1;1;-2;\ ...) \in ac_0
	\end{equation}
	и пусть $\alpha = 3$.
	Тогда
	\begin{equation}
		x^+ = (1;1;0;\ 1;1;0;\ 1;1;0;\ ...) \notin ac_0, \quad x^+ \in ac_{2/3}
		,
	\end{equation}
	\begin{equation}
		x^- = (0;0;-2;\ 0;0;-2;\ 0;0;-2;\ ...) \notin ac_0, \quad x^- \in ac_{-2/3}
		,
	\end{equation}
	\begin{equation}
		x^\alpha = (1;1;-8;\ 1;1;-8;\ 1;1;-8;\ ...) \notin ac_0, \quad x^\alpha \in ac_{-2}
		.
	\end{equation}
\end{example}

Однако для нечётной степени доказать аналог теоремы~\ref{thm:ac0_pow_even} не удастся "--- это показывает пример~\ref{example:ac0_pow_signum_classic}, в котором $x^+\in ac_1$, $x^-\in ac_{-1}$.

В заключение приведём пример, в котором возведение в нечётную степень выводит не только из пространства $ac_0$,
но и из более широкого пространства $ac$.

\begin{example}
	\label{example:cube_out_of_ac0}
	%Напомним, что $\mathbb{N}_k = \{k, k+1, k+2, k+3,...\}$.
	Пусть
	\begin{equation}
		x_n = \begin{cases}
			 0, & \mbox{~если~} n < 2^{10},
			\\
			 1, & \mbox{~если~} n \ge 2^{10}, 2^k\le n < 2^k+3k \mbox{~и~}  n\neq 2^k + 3m, m\in\mathbb{N},
			\\
			-2, & \mbox{~если~} n \ge 2^{10}, 2^k\le n < 2^k+3k \mbox{~и~}  n  =  2^k + 3m, m\in\mathbb{N}.
		\end{cases}
	\end{equation}
	Таким образом,
	\begin{multline}
		x = (0,0,...,0,0, \; -2, 1, 1, \; -2, 1, 1, \; -2, 1, 1, ..., -2, 1, 1, \; 0, 0, 0, \\ ..., 0, 0, 0, ..., -2, 1, 1, \; -2, 1, 1, ... )
		.
	\end{multline}
	Легко заметить, что в силу критерия Лоренца $x\in ac_0$.
	С другой стороны,
		\begin{equation}
		(x^3)_n = \begin{cases}
			 0, & \mbox{~если~} n < 2^{10},
			\\
			 1, & \mbox{~если~} n \ge 2^{10}, 2^k\le n < 2^k+3k \mbox{~и~}  n\neq 2^k + 3m, m\in\mathbb{N},
			\\
			-8, & \mbox{~если~} n \ge 2^{10}, 2^k\le n < 2^k+3k \mbox{~и~}  n  =  2^k + 3m, m\in\mathbb{N}.
		\end{cases}
	\end{equation}
	Получаем $p(x^3) = 0$, $q(x^3) = -2$, откуда $x^3 \notin ac$.
\end{example}

Аналогично строится и пример, когда возведение в нечётную степень, наоборот, вводит в пространство $ac_0$.

\begin{example}
	%Напомним, что $\mathbb{N}_k = \{k, k+1, k+2, k+3,...\}$.
	Пусть
	\begin{equation}
		y_n = \begin{cases}
			 0, & \mbox{~если~} n < 2^{10},
			\\
			 1, & \mbox{~если~} n \ge 2^{10}, 2^k\le n < 2^k+3k \mbox{~и~}  n\neq 2^k + 3m, m\in\mathbb{N},
			\\
			-\sqrt[3]{2}, & \mbox{~если~} n \ge 2^{10}, 2^k\le n < 2^k+3k \mbox{~и~}  n  =  2^k + 3m, m\in\mathbb{N}.
		\end{cases}
	\end{equation}
	Тогда $p(y) = \frac{2-\sqrt[3]2}{3}$, $q(y) = 0$, и потому $y \notin ac_0$.
	Однако легко заметить, что $y^3 = x$ из примера~\ref{example:cube_out_of_ac0} и потому $y^3\in ac_0$.
\end{example}




Рассуждения данного параграфа подталкивают нас к изучению следующего объекта.

Пусть
\begin{equation}
	ac_0^{(2n+1)} = \{ x\in ac_0 : x^{2n+1} \in ac_0\}, \quad n\in\mathbb N
	.
\end{equation}
(Множества $ac_0^{(2n)}$ мы в рассмотрение не вводим, поскольку каждое из них совпадает с $\Iac$ в силу теоремы~\ref{thm:ac0_pow_even}.)

Очевидна следующая
\begin{lemma}
	Пусть $x \approx y$.
	В этом случае $x \in ac_0^{(2n+1)}$ тогда и только тогда, когда $y \in ac_0^{(2n+1)}$.
\end{lemma}

\begin{lemma}
	Пусть $x - y \in \Iac$.
	В этом случае $x \in ac_0^{(2n+1)}$ тогда и только тогда, когда $y \in ac_0^{(2n+1)}$.
\end{lemma}

%TODO! Через бином Ньютона.
%\begin{proof}
%	\begin{equation}
%		x
%	\end{equation}
%\end{proof}

\begin{hypothesis}
	$ac_0^{(2n+1)}$ замкнуто относительно сложения.
\end{hypothesis}

\begin{hypothesis}
	$ac_0^{(2n+1)}$ замкнуто относительно умножения.
\end{hypothesis}

\begin{hypothesis}
	$ac_0^{(2n+1)}$ замкнуто (топологически).
\end{hypothesis}

\begin{hypothesis}
	$ac_0^{(2n+1)} = ac_0^{(2n+3)}$ для любого $n$ (или же, наоборот, там есть собственное включение).
\end{hypothesis}

\begin{hypothesis}
	Если предыдущая гипотеза неверна, то встаёт вопрос об исследовании множеств
	$\bigcup_{n\in\N}ac_0^{(2n+1)}$ и 	$\bigcap_{n\in\N}ac_0^{(2n+1)}$.
\end{hypothesis}

\begin{hypothesis}
	Пусть $x\in ac_0^{(2n+1)}$.
	Тогда $x^{2n+1} \in ac_0^{(2n+1)}$.
\end{hypothesis}

\begin{hypothesis}
	Пусть $x\in ac_0^{(2n+1)}$.
	Тогда $\sigma_k x \in ac_0^{(2n+1)}$ для любого $k\in \N$
\end{hypothesis}


\chapter{Банаховы пределы, инвариантные относительно операторов}

	По определению, каждый банахов предел инвариантен относительно сдвига: $B=BT$.
Возникает закономерный вопрос: относительно каких ещё операторов инвариантны банаховы пределы?

Мы не будем рассматривать здесь операторы обобщённого сдвига~\cite{marchenko2006generalized,lewitan1945normed};
%TODO2: при критической необходимости ссылок можно набрать тут:
%https://encyclopediaofmath.org/index.php?title=Generalized_displacement_operators#References
достаточно естественным обобщением обычного сдвига являются <<движения>> "--- инъективные отображения $\N$ в себя без периодических точек; за их обсуждением в свете банаховых пределов отсылаем читателя к работе Р.\,Нильсена~\cite{Nillsen}.

Уже для совершенно естественного "--- если бы мы говорили об обычной сходимости последовательностей "---
оператора растяжения $\sigma_k$
(напомним, что этот оператор просто повторяет каждый элемент последовательности $k$ раз)
выясняется, что относительно этого оператора инварианты не все банаховы пределы!
Более того, инвариантность конкретного банахова предела зависит от выбора $k\in\N_2$.

Для оператора Чезаро
\begin{equation}
	C (x_1, x_2, x_3, ...) = \left(
	x_1,
	\dfrac{x_1+x_2}2,
	\dfrac{x_1+x_2 + x_3}3,
	\dfrac{x_1+x_2+x_3+x_4}4,
	...,
	\dfrac{x_1+...+x_n}n,
	...\right)
	.
\end{equation}
ситуация с инвариантностью банаховых пределов обстоит ещё <<хуже>>.
Напомним, что через $\B(H)$ мы обозначаем множество банаховых пределов,
инвариантных относительно оператора $H$,
т.е. таких $B\in \B$, что $B=BH$.
Так, в~\cite[\S2, Theorem 4]{semenov2020dilation} показано, что
\begin{equation}
	\B(C) \subsetneq \bigcap_{n\in \N}\B(\sigma_n)
	.
\end{equation}
(это утверждение обобщает~\cite[Theorem 3]{semenov2020invariant_noncommutative}
и~\cite[Theorem 4.8]{ASSU4}).

Важной вехой в изучении инвариантных банаховых пределов явилась следующая теорема,
доказанная в~\cite[\S2]{Semenov2010invariant}.
Мы ещё не раз будем возвращаться к её обсуждению.

\begin{theorem}
	\label{thm:Semenov_Sukochev_conditions}
	Пусть линейный оператор $H:\ell_\infty\to\ell_\infty$ таков, что:
	\\(i)   $H \geqslant 0, H \one=\one$;
	\\(ii)  $H c_0 \subset c_0$;
	\\(iii) $\limsup _{j \rightarrow \infty}(A(I-T) x)_j \geqslant 0$ для всех $x \in \ell_{\infty}, A \in R$,
	\\где
	\begin{equation}
		R=R(H):=\operatorname{conv}\left\{H^n, n=1,2, \ldots\right\}
		.
	\end{equation}
	Тогда $\B(H) \ne\varnothing$.
\end{theorem}

В настоящей главе в теореме~\ref{thm:Eberlein_but_not_B-regular_exists}
строится пример такого оператора $E$, что $\B(E)\ne\varnothing$,
но $E\one \ne \one$, что показывает существенную избыточность условий теоремы~\ref{thm:Semenov_Sukochev_conditions}.


Достаточно <<плохими>> в смысле инвариантности оказываются крайние точки множества банаховых пределов $\ext \B$.
Это множество не является слабо$^*$ замкнутым~\cite{Nillsen,Talagrand},
и его мощность совпадает~\cite{Chou} с мощностью всего пространства $\ell_\infty^*$
и составляет $2^{\mathfrak c}$.

%TODO2: про расстояние от $\ext \B$ до $\B(C)$ и прочих там всяких $\B(\sigma_n)$ (а также их объединения?).

Тем не менее, в настоящей главе удалось использовать элементы $\ext\B$ для построения инвариантных банаховых пределов.
%TODO1: ссылки на теоремы


Инвариантные банаховы пределы находят приложения в некоммутативной геометрии
и теории сингулярных следов, где используются в качестве элемента конструкции
различных подклассов следов Диксмье
~\cite{carey2003spectral,lord2012singular,sukochev2015characterization,sukochev2016dixmier}.
В недавних работах~\cite{astashkin2015constants_rus_DAN,astashkin2016constants_rus_SMJ} инвариантные банаховы пределы были применены для исследования
асимптотики констант Лебега для системы Уолша.

%Invariant Banach limits have also
%found important applications in noncommutative geometry and the theory
%of singular traces, where they were employed in the construction of various
%subclasses of Dixmier traces [12, 19, 28, 29]. Recently, invariant Banach
%limits have been used to study the asymptotics of the Lebesgue constants
%of the Walsh system [9].


	\section{Оператор с конечномерным ядром, для которого не существует инвариантного банахова предела}
	В работе \cite{Semenov2010invariant} изучаются условия,
при которых для оператора $H:\ell_\infty\to \ell_\infty$ существует банаховы пределы,
инвариантные относительно данного оператора, то есть такие $B\in\mathfrak{B}$,
что $B(Hx) = Bx$ для любого $x\in\ell_\infty$,
а также приводятся примеры операторов, для которых инвариантные банаховы пределы существуют:
операторы $\sigma_n$ и оператор Чезаро $C$.

Заметим, что операторы $\sigma_n$ и $C$ имеют вырожденное ядро,
однако не для любого оператора с вырожденным ядром существует инвариантный банахов предел.

\begin{example}
	Пусть для $x = (x_1, x_2, ..., x_n, ...)\in \ell_\infty$
	\begin{equation*}
		Ax = (x_1, 0, x_2, 0, x_3, 0, x_4, 0, ...).
	\end{equation*}
	Очевидно, что $\ker A = \{0\}$.
\end{example}

Пусть $B\in\mathfrak{B}$, $BA = B$.
Очевидно, что
\begin{equation*}
	\frac{n-1}{2}\leqslant \sum_{k=m+1}^{m+n} (A\one)_k \leqslant \frac{n+1}{2},
\end{equation*}
где $\one = (1, 1, 1, 1, 1, 1, ...)$.
Тогда по теореме Лоренца
\begin{equation*}
	BA\one =
	\lim_{n\to\infty} \frac{1}{n}\sum_{k=m+1}^{m+n} (A\one)_k = \frac{1}{2}
\end{equation*}
Однако по определению банахова предела
\begin{equation*}
	B\one = 1 \neq \frac{1}{2} = BA\one.
\end{equation*}
Пришли к противоречию, следовательно, банаховых пределов, инвариантных относительно $A$, не существует.


	\section{Оператор с бесконечномерным ядром, относительно которого инвариантен любой банахов предел}
	Результаты этого пункта опубликованы в~\cite{our-ped-2018-inf-dim-ker}.


Конечномерность ядра оператора не является необходимым условием существования
банахова предела, инвариантного относительно данного оператора.

Приведём сначала несложную лемму, а потом соответствующий пример.

\begin{lemma}
	%TODO2: а не баян ли?
	%\textit{Скорее всего, это настолько очевидно, что это никто не пишет и не доказывает.}

	Пусть $Q:\ell_\infty \to \ell_\infty$.
	\begin{equation}
		\forall(B\in\mathfrak{B})[QB = B]
	\end{equation}
	тогда и только тогда, когда
	\begin{equation}\label{I-Q_to_ac_0}
		I-Q : \ell_\infty \to ac_0,
	\end{equation}
	где $I$~--- тождественный оператор на $\ell_\infty$.

\end{lemma}

\paragraph{Необходимость.}
\begin{equation}
	B((I-Q)x) =
	B(Ix) - B(Qx) =
	Bx - B(Qx)=
	Bx-Bx
	=
	0
	,
\end{equation}
откуда в силу произвольности выбора $B$ и $x$ вкупе с тем, что $B((I-Q)x)=0$,
имеем $(I-Q)x \in ac_0$ и, следовательно, $I-Q : \ell_\infty \to ac_0$.

\paragraph{Достаточность.}
\begin{equation}
	B(Qx) = B((I-(I-Q))x) =
	B(Ix)-B((I-Q)x) =
	B(x) - 0 = B(x).
\end{equation}


\begin{example}
	\begin{equation}
		(Qx)_k =
		\begin{cases}
			0,~\mbox{если}~ k = 2^n, n \in\N,
			\\
			x_k~\mbox{иначе.}
		\end{cases}
	\end{equation}
\end{example}
Очевидно, что $\dim \ker Q = \infty$.

Покажем, что $Q$ удовлетворяет условию (\ref{I-Q_to_ac_0}).
\begin{equation}
	\sum_{k=m+1}^{m+n} x_k - \sum_{k=m+1}^{m+n} (Qx)_k \leqslant (2 + \log_2 n) \|x\|,
\end{equation}
%(TODO2:это очевидно??)\\
откуда немедленно
\begin{equation}
	\frac{1}{n}\sum_{k=m+1}^{m+n} x_k - \frac{1}{n}\sum_{k=m+1}^{m+n} (Qx)_k \leqslant \frac{(2 + \log_2 n) \|x\|}{n} \to 0.
\end{equation}

По предыдущему утверждению отсюда следует, что относительно $Q$ инвариантен любой банахов предел.


	\section{О классах линейных операторов, для которых множества инвариантных банаховых пределов совпадают}
	\begin{lemma}
	\label{thm:linear_op_equiv_ac0}
	Пусть $P,Q:\ell_\infty \to \ell_\infty$~--- линейные операторы и
	\begin{equation}\label{P-Q:ell_infty_to_ac0}
		P-Q : \ell_\infty \to ac_0
		.
	\end{equation}
	Тогда
	\begin{equation}
		\mathfrak{B}(P)=\mathfrak{B}(Q)
		.
	\end{equation}
\end{lemma}

\begin{proof}
	Пусть $B\in \mathfrak{B}(P)$.
	Тогда
	\begin{equation}
		B(Qx) = B((Q-P+P)x) =
		B((Q-P)x)+B(Px) =
		0 + B(Px) =
		Bx
		.
	\end{equation}
	Значит, $\mathfrak{B}(P) \subset \mathfrak{B}(Q)$.
	В силу симметричности утверждения леммы получаем $\mathfrak{B}(Q) \subset \mathfrak{B}(P)$,
	откуда и следует требуемое утверждение.
\end{proof}

Только что доказанная лемма означает, что множество всех линейных операторов,
действующих из $\ell_\infty$ в $\ell_\infty$, можно разбить на классы по отношению эквивалентности,
задаваемому условием \eqref{P-Q:ell_infty_to_ac0},
и тогда для операторов из одного класса множества инвариантных банаховых пределов будут совпадать.

Обратное, однако, неверно.

\begin{example}
	Пусть
	\begin{equation}
		P(x_1,x_2,x_3) = (x_1, 0, x_3, 0, x_5, 0, ...)
	\end{equation}
	и
	\begin{equation}
		Q(x_1,x_2,x_3) = (0, -x_2, 0, -x_4, 0, -x_6, 0, ...)
		.
	\end{equation}
	Легко видеть, что для любого банахова предела $B\in\B$ выполнено
	\begin{equation}
		B(P\one) = \frac{1}{2}
	\end{equation}
	и
	\begin{equation}
		B(Q\one) = -\frac{1}{2}
		,
	\end{equation}
	откуда
	\begin{equation}
		\B(Q) = \B(P) = \varnothing
		.
	\end{equation}
	Однако разность $P-Q$ есть не что иное, как тождественный оператор $I$,
	который переводит пространство $\ell_\infty$ само в себя,
	а не в пространсво $ac_0$.
\end{example}


\begin{hypothesis}
	Существуют два таких линейных оператора $P, Q : \ell_\infty \to \ell_\infty$,
	что $\B(P) = \B(Q) \neq \varnothing$,
	но $(P-Q)(\ell_\infty) \setminus ac \neq \varnothing$.
\end{hypothesis}



	\section{Мощность множества линейных операторов, относительно которых инвариантен банахов предел}
	Очевидна следующая
\begin{lemma}
	Пусть $A:\ell_\infty\to\ell_\infty$,
	\begin{equation}
		(Ax)_k=\begin{cases}
			x_n, & \mbox{если~} k=2^n,
			\\
			0    & \mbox{иначе.}
		\end{cases}
	\end{equation}
	Тогда $A:\ell_\infty\to ac_0$.
\end{lemma}

\begin{corollary}
	$|\mathcal{L}(\ell_\infty,\ell_\infty)|=|\mathcal{L}(\ell_\infty,ac_0)|$.
\end{corollary}

\begin{proof}
	Пусть $Q\in \mathcal{L}(\ell_\infty,\ell_\infty)$.
	Тогда $AQ\in \mathcal{L}(\ell_\infty,ac_0)$.
	Из того, что отображение $Q\mapsto AQ$~--- инъекция,
	следует, что $|\mathcal{L}(\ell_\infty,ac_0)|\geq      |\mathcal{L}(\ell_\infty,\ell_\infty)|$.
	Из того, что $ \mathcal{L}(\ell_\infty,ac_0) \subsetneq \mathcal{L}(\ell_\infty,\ell_\infty) $,
	следует, что $|\mathcal{L}(\ell_\infty,ac_0)|\leq      |\mathcal{L}(\ell_\infty,\ell_\infty)|$.
	Значит,      $|\mathcal{L}(\ell_\infty,ac_0)|=         |\mathcal{L}(\ell_\infty,\ell_\infty)|$,
	что и требовалось доказать.
\end{proof}

\begin{corollary}
	Пусть $P:\ell_\infty \to \ell_\infty$ и $\mathfrak{B}(P)\neq\varnothing$.
	Пусть $G = \{Q: \mathfrak{B}(P)= \mathfrak{B}(Q)\}$.
	Тогда $|G| = |\mathcal{L}(\ell_\infty,\ell_\infty)|$.
\end{corollary}

\begin{proof}
	Действительно,
	$P+\mathcal{L}(\ell_\infty,ac_0) \subset G$ по лемме~\ref{thm:linear_op_equiv_ac0}.
	Значит,
	\begin{equation}
		|G| \geq |\mathcal{L}(\ell_\infty,ac_0)|=|\mathcal{L}(\ell_\infty,\ell_\infty)|
		.
	\end{equation}
	Из того, что $G\subsetneq|\mathcal{L}(\ell_\infty,\ell_\infty)|$,
	следует, что $|G|\leq|\mathcal{L}(\ell_\infty,\ell_\infty)|$.
	Значит, $|G|=|\mathcal{L}(\ell_\infty,\ell_\infty)|$,
	что и требовалось доказать.
\end{proof}





	\section{Банаховы пределы, инвариантные относительно операторов $\sigma_{1/n}$}
	Введём, вслед за~\cite[с. 131, утверждение 2.b.2]{lindenstrauss1979classical},
на $\ell_\infty$ оператор
\begin{equation}
	\sigma_{1/n} x = n^{-1}
	\left(
		\sum_{i=1}^{n} x_i,
		\sum_{i=n+1}^{2n} x_i,
		\sum_{i=2n+1}^{3n} x_i,
		...
	\right).
\end{equation}

Из критерия Лоренца немедленно следует
\begin{lemma}
	$\sigma_n \sigma_{1/n} - I : \ell_\infty \to ac_0$.
\end{lemma}

\begin{theorem}
	\label{thm:B_sigma_n_eq_B_sigma_1_n}
	Для любого $n\in\N$ выполнено $\mathfrak{B}(\sigma_n) = \mathfrak{B}(\sigma_{1/n})$.
\end{theorem}

\begin{proof}
	Пусть сначала $B \in \mathfrak{B}(\sigma_{1/n})$.
	Тогда
	\begin{equation}
		B(x) = B(Ix) = B(\sigma_{1/n} \sigma_n x) = B(\sigma_n x)
		.
	\end{equation}
	Пусть теперь $B \in \mathfrak{B}(\sigma_n)$.
	Тогда
	\begin{equation}
		B(x) = B(Ix) = B((I+(\sigma_n \sigma_{1/n} - I)) x) = B(\sigma_n \sigma_{1/n} x) = B(\sigma_{1/n} x)
		.
	\end{equation}
\end{proof}

\begin{remark}
	Мы видим, что множество банаховых пределов, инвариантных относительно суперпозиции операторов,
	может быть шире, чем объединение множеств банаховых пределов,
	инвариантных относительно каждого из операторов:
	\begin{equation}
		\mathfrak{B}(\sigma_n) \cup \mathfrak{B}(\sigma_{1/n}) = \mathfrak{B}(\sigma_n) \subsetneq \mathfrak{B}(\sigma_{1/n}\sigma_n) = \mathfrak{B}(I)
		.
	\end{equation}
\end{remark}


	\section{Операторы $\tilde\sigma_k$}
	Введём в рассмотрение семейство операторов
$\tilde\sigma_k : \ell_\infty \to \ell_\infty$, $k>0$,
определяемых следующим образом:
\begin{equation}
	(\tilde\sigma_k x)_n = x_{\left\lceil \dfrac{n}{k}\right\rceil}
	.
\end{equation}

\begin{example}
	\label{example:sigma_3_2}
	\begin{equation}
		\tilde\sigma_{3/2} x =
		(x_1, x_2, x_2, \; x_3, x_4, x_4, \; x_5, ...)
		.
	\end{equation}
\end{example}

\begin{example}
	\label{example:sigma_2_3}
	\begin{equation}
		\tilde\sigma_{2/3} x =
		(x_2, x_3, \; x_5, x_6, \; x_8, ...)
		.
	\end{equation}
\end{example}


Заметим, что для $k\in \N$ выполнено равенство $\sigma_k = \tilde\sigma_k$
(однако использовать обозначение без тильды мы не можем, чтобы избежать путаницы с операторами усреднения $\sigma_{1/k}$).
Однако соотношение $\tilde\sigma_k \tilde\sigma_m = \tilde\sigma_{km}$ для нецелых $k$, вообще говоря, не выполняется.
Чтобы это увидеть, достаточно рассмотреть суперпозиции $\tilde\sigma_{3/2} \tilde\sigma_{2/3}$ и
$\tilde\sigma_{2/3} \tilde\sigma_{3/2}$ (см. примеры~\ref{example:sigma_3_2} и~\ref{example:sigma_2_3} выше).

Таким образом, операторы $\tilde\sigma_k$ обобщают операторы $\sigma_k$,
и возникает закономерный вопрос о соответствующих инвариантных банаховых пределах.

\begin{theorem}
	Пусть $k>0$, $k\in \mathbb{Q} \setminus \N$.
	Тогда $\B(\tilde\sigma_k)=\varnothing$.
\end{theorem}

\begin{proof}
	Пусть $k$ представимо в виде несократимой дроби $k=p/q$, $p\in \N$, $q\in\N_2$.
	Рассмотрим последовательность $x\in \ell_\infty$, заданную соотношением
	\begin{equation}
		x_k = \begin{cases}
			1, \mbox{~если~} m=qn+1, n\in\N,
			\\
			0  \mbox{иначе.}
		\end{cases}
	\end{equation}
	Последовательность $x$ периодична, и её период равен $q$.

	Пусть $B\in \B$, тогда $Bx=\dfrac1q$ , поскольку любой банахов предел на периодической последовательности принимает значение, равное среднему арифметическому по периоду.
	%TODO: ссылка!
	Заметим, что $\tilde\sigma_{p/q}x \in \Omega$.
	Более того, последовательность $\tilde\sigma_{p/q}x \in \Omega$ также периодична и имеет период, равный $p$.

	Действительно,
	\begin{multline}
		(\tilde\sigma_{p/q}x )_{m+p} =
		x_{\left\lceil \dfrac{m+p}{p/q}\right\rceil} =
		x_{\left\lceil \dfrac{qm+qp}{p}\right\rceil} =
		x_{\left\lceil \dfrac{qm}{p}+q\right\rceil} =
		\\=
		x_{q+\left\lceil \dfrac{qm}{p}\right\rceil} =
		x_{\left\lceil \dfrac{qm}{p}\right\rceil} =
		x_{\left\lceil \dfrac{m}{p/q}\right\rceil} =
		(\tilde\sigma_{p/q}x \in)_{m}
		.
	\end{multline}

\end{proof}

\begin{theorem}
	Пусть $k>0$, $k\in \mathbb{Q} \setminus \N$.
	Тогда $\tilde\sigma_k ac_0 \not \subset ac_0$.
\end{theorem}




\begin{theorem}
	Пространство $ac_0$ не замкнуто относительно любого оператора $\tilde\sigma_k$, $k\in\Q^+\setminus \N$.
\end{theorem}

\begin{proof}
	Пусть $k=p/q$ есть несократимая дробь.
	Определим последовательность $x\in\ell_\infty$ соотношением:
	\begin{equation}
		x_m = \begin{cases}
		\end{cases}
	\end{equation}
	TODO: набрать доказательство.
\end{proof}

\begin{hypothesis}
	\label{hyp:tilde_sigma_k_x_notin_ac0}
	Для любого $k \in \R^+\setminus \N$ ($k \in \Q^+\setminus \N$) существует такой $x\in ac_0$, что $\tilde\sigma_{k} x \notin ac$.
\end{hypothesis}

В пользу гипотезы~\ref{hyp:tilde_sigma_k_x_notin_ac0} говорит следующий
\begin{example}
	Определим последовательность $x\in\ell_\infty$ следующим соотношением:
	\begin{equation}
		x_k = \begin{cases}
			(-1)^k, &~\mbox{если}~ 2^{2n} < k \leq 2^{2n+1}, ~ n \in \N_0
			\\
			(-1)^{k+1} &~\mbox{иначе}
			.
		\end{cases}
	\end{equation}
	Тогда $x\in ac_0$, но
	\begin{equation}
		(\tilde\sigma_{1/2} x)_k = \begin{cases}
			-1, &~\mbox{если}~ 2^{2n} < k \leq 2^{2n+1}, ~ n \in \N_0
			\\
			1 &~\mbox{иначе}
			.
		\end{cases}
	\end{equation}
	Имеем $p(\tilde\sigma_{1/2} x) = 1$, $q(\tilde\sigma_{1/2} x) = -1$,
	откуда $\tilde\sigma_{1/2} x \notin ac$.
\end{example}



\begin{hypothesis}
	Для любых натуральных $k > m$ существуют такие $r, s\in\N$, что  $\tilde\sigma_{m/k} T^r \tilde\sigma_{k/m} = T^s$.
\end{hypothesis}


\begin{hypothesis}
	Для любого $k \in \N$ выполнено $(\tilde\sigma_{\sqrt{k}})^2 - \sigma_k : \ell_\infty \to ac_0$.
\end{hypothesis}



\begin{hypothesis}
	Пусть $x\in ac_0$.
	Для того, чтобы $x\in \mathcal{I}(ac_0)$, необходимо и достаточно,
	чтобы $\tilde\sigma_{k} x \in ac_0$ для любого $k\in \Q^+$.
\end{hypothesis}


\begin{hypothesis}
	Пусть $x\in ac_0$.
	Для того, чтобы $x\in \mathcal{I}(ac_0)$, необходимо и достаточно,
	чтобы $\tilde\sigma_{k} x \in ac_0$ для любого $k\in \R^+$.
\end{hypothesis}


	\section{Мера прообраза числа при инвариантности банахова предела относительно оператора Чезаро}
	При изучении банаховых пределов и меры на множестве $\Omega$
возникает закономерный вопрос о мере множества
\begin{equation}
	\{ x \in \Omega : Bx = \beta \}
\end{equation}
для заданного банахова предела $B$ и числа $\beta\in[0;1]$.
(Хаусдорфова размерность этого множества равна 1 в силу леммы~\ref{lem:Hausdorf_measure}.)

\begin{theorem}
	Пусть $B \in \mathfrak{B}(C)$.
	Тогда
	\begin{equation}
		\mes \{ x \in \Omega : Bx = 1/2 \} = 1
		.
	\end{equation}
\end{theorem}

\begin{proof}
	Так как $B \in \mathfrak{B}(C)$, то
	\begin{equation}
		\{ x \in \Omega : Bx = 1/2 \}
		=
		\{ x \in \Omega : BCx = 1/2 \}
		\supset
		\{ x \in \Omega : \lim_{n\to\infty} (Cx)_n = 1/2 \}
		.
	\end{equation}
	Вместе с тем,
	\begin{equation}
		\mes \left\{ x \in \Omega : \lim_{n\to\infty} (Cx)_n  = 1/2 \right\} = 1
	\end{equation}
	(это следует из закона больших чисел~\cite{connor1990almost}).
\end{proof}

\begin{corollary}
	Пусть $B \in \mathfrak{B}(C)$ и $\lambda \ne \frac 1 2$. Тогда
	\begin{equation}
		\mes \{ x \in \Omega : Bx = \lambda \} = 0
		.
	\end{equation}
\end{corollary}



	\section{Классификация линейных операторов в свете банаховых пределов}
	При изучении инвариантности банаховых пределов относительно различных непрерывных линейных операторов
возникает закономерный вопрос о выделении некоторых классов этих операторов.

\begin{definition}
	Будем говорить, что оператор $H : \ell_\infty \to \ell_\infty$ \emph{эберлейнов},
	если $\B (H) \ne \varnothing$.
\end{definition}

Выбор именования этого класса операторов обусловлен тем, что именно Эберлейн в работе~\cite{Eberlein}
впервые сколь-либо системно изучил инвариантные банаховы пределы
(хотя отдельные шаги в этом направлении были сделаны ещё Эгнью и Морсом~\cite{agnew1938linear,agnew1938extensions}).

В работе~\cite{alekhno2018invariant} вводится следующее

\begin{definition}
	\label{def:B-regular_operator}
	Оператор $H : \ell_\infty \to \ell_\infty$ называется \emph{B-регулярным},
	если $H^*\B \subseteq \B$.
	(Или, что то же самое, $BH\in\B$ для любого $B\in\B$.)
\end{definition}

В той же работе с помощью теоремы Брауэра--Шаудера--Тихонова о неподвижной точке~\cite[Corollary  17.56]{aliprantis2006infinite}
доказывается следующая
\begin{theorem}
	\label{thm:B-regular_is_Eberlein}
	Любой B-регулярный оператор "--- эберлейнов.
\end{theorem}
Там же приводится следующее необходимое и достаточное условие В-регулярности.

\begin{theorem}
	\label{thm:crit_B_regularity}
	Оператор $H:\ell_\infty \to \ell_\infty$ является B-регулярным тогда и только тогда, когда:
	\\(i) $H\one \in ac_1$;
	\\(ii) $q(Hx)\geq 0$ для любого $x\geq 0$;
	\\(iii) $H ac_0 \subseteq ac_0$.
\end{theorem}
Легко заметить, что эти условия являются в целом более слабыми, чем достаточные условия эберлейновости
(теорема~\ref{thm:Semenov_Sukochev_conditions}).

\begin{corollary}
	Пусть $x\in ac_\lambda$ и оператор $H$ является В-регулярным.
	Тогда $Hx \in ac_\lambda$.
\end{corollary}

Кроме того, в той же работе~\cite{alekhno2018invariant}
с помощью теоремы Крейна--Мильмана~\cite[Theorem  9.14]{aliprantis2006positive}
доказывается следующая
\begin{lemma}
	Пусть оператор $H:\ell_\infty\to\ell_\infty$ является B-регулярным.
	Тогда множество $\ext\B(H)$ непусто.
\end{lemma}

Возникает закономерный вопрос: существует ли эберлейнов оператор, не являющийся B-регулярным?
Интуитивно кажется, что существует, однако в вопросах, связанных с банаховыми пределами,
многие факты оказываются контринтуитивными.
Пример эберлейнова не В-регулярного оператора строится ниже
в теореме~\ref{thm:Eberlein_but_not_B-regular_exists}.

Подобные рассуждения можно продолжить в обе стороны, а именно "--- следующими двумя определениями:

\begin{definition}
	Будем называть оператор $A:\ell_\infty \to \ell_\infty$ \emph{полуэберлейновым}, если $BA\in\mathfrak B$ для некоторого $B\in\mathfrak B$.
\end{definition}

\begin{definition}
	Будем называть оператор $A:\ell_\infty \to \ell_\infty$ \emph{существенно эберлейновым}, если $BA\in\mathfrak B$ для любого $B\in\mathfrak B$ и $BA\ne B$ для некоторого $B\in\mathfrak B$.
\end{definition}

Эти четыре класса операторов (существенно эберлейновы, В-регулярные, эберлейновы, полуэберлейновы "--- в порядке включения)
получены последовательным ослаблением условий, естественных для <<достаточно хороших>> операторов:
$\sigma_k$, $C$
и образуют иерархию по включению.
Возникает закономерный вопрос о совпадении классов.
Ясно, что оператор сдвига $T$ является В-регулярным, но не является существенно эберлейновым, поскольку относительно него любой банахов предел инвариантен по определению.
Более того, ясно, что операторы растяжения $\sigma_k$, $k\in\N_2$, и оператор Чезаро $C$ являются существенно эберлейновыми,
поскольку для каждого из них существуют как инвариантные, так и неинвариантные банаховы пределы.

Существует ли полуэберлейнов оператор, не являющийся эберлейновым?
Положительный (и конструктивный) ответ на этот вопрос даёт теорема~\ref{thm:amiable_but_not_Eberlein_exists} ниже.



Далее возникает вопрос о свойствах классов операторов.
Из определения В-регулярности незамедлительно следует

\begin{lemma}
	\label{lem:B-regular_superposition_and_addition}
	Множество В-регулярных операторов замкнуто относительно суперпозиции и выпуклой комбинации.
\end{lemma}


%Несколько менее очевидна выпуклость множества полуэберлейновых операторов на единичной сфере соответствующего пространства.
%
%\begin{lemma}
%	Пусть $A, H : \ell_\infty \to \ell_\infty$ ~--- полуэберлейновы операторы и $0\leq \lambda \leq 1$.
%	Тогда $W = \lambda A + (1-\lambda) H$ ~--- также полуэберлейнов оператор.
%\end{lemma}
%\begin{proof}
%	Пусть $B_A\in\B(A)$, $B_H\in\B(H)$.
%	Положим $B_W = \lambda B_A + (1-\lambda) B_H$ и рассмотрим произвольный $x\in\ell_\infty$.
%	\begin{equation}
%
%	\end{equation}
%\end{proof}

%%%% Судя по всему, доказательство неверно!


	\section{Пример дружелюбного не эберлейнового оператора}
	\begin{theorem}
	\label{thm:amiable_but_not_Eberlein_exists}
	Существует дружелюбный оператор, не являющийся эберлейновым.
\end{theorem}

\begin{proof}
	Пусть $B_1, B_2 \in \ext \B$.
	Положим
	\begin{equation}
		\label{eq:am_not_eber_def}
		B_3 = B_1 + 2(B_2-B_1) = 2B_2-B_1,
	\end{equation}
	тогда $B_3 \notin \B$.
	Действительно, если $B_3 \in \B$, то из~\eqref{eq:am_not_eber_def} следует, что
	\begin{equation}
		B_2 = \frac{B_1 + B_3}2 \in \B \setminus \ext \B
		.
	\end{equation}
	Введём в рассмотрение оператор $H:\ell_\infty\to\ell_\infty$, определённый равенством
	\begin{equation}
		2Hx = (x_1 + B_3x, x_2 + B_3x, ...) = x + (B_3 x) \cdot \mathbbm 1
		.
	\end{equation}
	Убедимся, что оператор $H$ дружелюбен.
	Действительно, для произвольного $x\in\ell_\infty$ имеем
	\begin{equation}
		2 B_1 H x = B_1 x + B_1 ((B_3 x) \cdot\mathbbm 1) = B_1 x + B_3 x =
		B_1 x + 2 B_2 x - B_1 x = 2B_2 x
		,
	\end{equation}
	откуда $B_1 H = B_2$.

	Убедимся теперь, что оператор $H$ не эберлейнов.
	Пусть $B = BH$ для некоторого $B\in\B$.
	Тогда для всех $x\in\ell_\infty$ имеем
	\begin{equation}
		2Bx = B (x + (B_3 x) \cdot \mathbbm 1)
		,
	\end{equation}
	т.е.
	\begin{equation}
		Bx =  B((B_3 x) \cdot \mathbbm 1)
		,
	\end{equation}
	откуда незамедлительно следует, что $Bx = B_3x$ и в силу произвольности выбора $x$ имеем $B=B_3$.
	Но ранее мы уже показали, что $B_3\notin \B$.
	Полученное противоречие завершает доказательство.
\end{proof}

\begin{hypothesis}
	$B_2H\notin \B$.
\end{hypothesis}

\begin{hypothesis}
	Существуют такие дружелюбный оператор $H:\ell_\infty\to\ell_\infty$ и банахов предел $B\in \B$,
	что $BH \in \B$, но $B_1 H \notin \B$ для любого $B\in \B\setminus\{B\}$.
	Также интересно наложение дополнительных свойств на $B$ и $BH$, например, принадлежности к $\ext\B$, $\B(C)$ и т.д.
\end{hypothesis}


	\section{Об одном В-регулярном операторе}
	Изучим сначала оператор $A:\ell_\infty\to\ell_\infty$,
определённый равенством:
\begin{multline}
	\label{eq:oper_A_throws_out_2power_blocks}
	Ax = (x_1, x_2, \not x_3, \not x_4, x_5, x_6, x_7, x_8, \not x_9, ..., \not x_{16}, x_{17}, x_{18}, ..., x_{31}, x_{32}, \not x_{33}, \not x_{34}, ..., \not x_{64},
	\\
	x_{65}, x_{66}, ..., x_{128}, \not x_{129}, ...)=
	\\=
	(x_1, x_2, \ x_5, x_6, x_7, x_8, \ x_{17}, x_{18}, ..., x_{31}, x_{32}, \ x_{65}, x_{66}, ..., x_{128}, \ x_{257},
	\\
	..., \ x_{2^{2n} +1}, x_{2^{2n} +2},  x_{2^{2n+1}}, \ \ x_{2^{2(n+1)} +1},  x_{2^{2(n+1)} +2},  x_{2^{2(n+1)+1}}, ...)
\end{multline}

В дальнейшем этот оператор окажется очень полезен при построении других операторов и банаховых пределов.

Очевидна следующая

\begin{lemma}
	\label{lem:supp_I_ac0}
	Пусть $x\in \mathcal I(ac_0)$, $Y \subset \supp x$ и $y = x \cdot \chi_Y$.
	Тогда $y \in \mathcal I(ac_0)$.
\end{lemma}

%TODO2: если совсем прижмёт - аккуратно, со вкусом доказать!

\begin{lemma}
	\label{lem:AT-TA}
	$AT - TA: \ell_\infty \to \mathcal{I}(ac_0)$.
\end{lemma}

\begin{proof}
	Пусть $\varphi(n) = 1 + 2^0 + 2^2 + 2^4 + ... + 2^{2n}$
	и пусть $y = \chi_{\{k\in\N : k = \varphi(n), n\in \N\}}$.
	Тогда ясно, что $y \in ac_0$ (в силу быстрого роста функции $\varphi(n)$)
	%TODO2: аккуратное доказательство с помощью теоремы из матзаметок?
	и, более того, $y\in \mathcal{I}(ac_0)$.
	Заметим, что
	\begin{multline}
		ATx =
		(x_2, x_3, \ x_6, x_7, x_8, x_9, \ x_{18}, x_{19}, ..., x_{32}, x_{33}, \ x_{66}, x_{67}, ..., x_{129}, \ x_{258},
		\\
		..., \ x_{2^{2n} +2}, x_{2^{2n} +3},  x_{ 2^{2n+1} +1}, \ \ x_{2^{2(n+1)} +2},  x_{2^{2(n+1)} +3},  x_{ 2^{2(n+1)+1} +1}, ...)
		.
	\end{multline}
	С другой стороны,
	\begin{multline}
		TAx =
		(x_2, \ x_5, x_6, x_7, x_8, \ x_{17}, x_{18}, ..., x_{31}, x_{32}, \ x_{65}, x_{66}, ..., x_{128}, \ x_{257},
		\\
		..., \ x_{2^{2n} +1}, x_{2^{2n} +2},  x_{2^{2n+1}}, \ \ x_{2^{2(n+1)} +1},  x_{2^{2(n+1)} +2},  x_{2^{2(n+1)+1}}, ...)
		.
	\end{multline}
	Таким образом, для любого $x\in\ell_\infty$
	\begin{equation}
		(ATx-TAx)_k=\begin{cases}
			x_{2^{2n-1}} -x_{2^{2n} + 1}, ~~&\mbox{если}~ k = \varphi(n), ~n\in \N_0,
			\\
			0 ~~&\mbox{иначе}.
		\end{cases}
	\end{equation}
	Ввиду включения $\supp (AT-TA)x \subset \supp y$ имеем $(AT-TA)x \in \mathcal I (ac_0)$ по лемме~\ref{lem:supp_I_ac0}.
\end{proof}

Докажем следующую вспомогательную лемму.

\begin{lemma}
	\label{lem:suff_B_reg}
	Пусть оператор $H$ удовлетворяет следующим условиям:
	\\i)   $H \geq 0$;
	\\ii)  $H\one\in ac_1$;
	\\iii) $H c_0 \subset ac_0$;
	\\iv)  $HT-TH : \ell_\infty \to ac_0$.% для некоторых $k,m \in \N_0$.

	Тогда оператор $H$ является В-регулярным.
\end{lemma}

\begin{proof}
	Доказательство проведём непосредственной проверкой определения банахова предела
	(определение~\ref{def:Banach_limit})
	для суперпозиции $B = B_1 H$,
	где $B_1 \in \B$ "--- произвольный банахов предел.

	Действительно, если $H\geq 0$, то $B = B_1 H \geq 0$,
	поскольку $B_1 \geq 0$ по определению банахова предела $B_1$.
	Далее, $B\one = B_1 H \one = 1$, поскольку $H\one \in ac_1$,
	а $B_1(ac_1) = 1$ по определению банахова предела $B_1$.
	Заметим теперь, что
	\begin{equation}
		Bc_0 = B_1 H c_0 = B_1 ac_0 = 0
	\end{equation}
	снова в силу того, что $B_1\in\B$.
	Наконец, для любого $x\in\ell_\infty$ выполнено
	\begin{equation}
		(BT-B)x = BTx - Bx = B_1 HTx - B_1 H x = B_1 HTx - B_1 T H x = B_1 (HT -  T H) x = 0
		.
	\end{equation}
	В силу произвольности выбора $x\in \ell_\infty$ последнее равенство и означает, что $BT=B$.

	Значит, для любого банахова предела $B_1$ функционал $B=B_1 H$ снова есть банахов предел,
	что по определению~\ref{def:B-regular_operator}
	и означает В-регулярность оператора $H$.
\end{proof}

\begin{remark}
	Условия леммы~\ref{lem:suff_B_reg} не являются необходимыми;
	так, контрпример к условию (i) очевиден:
	\begin{equation}
		H_1(x_1,x_2,x_3,x_4...) = (-x_1, x_2, x_3, x_4, ...)
		.
	\end{equation}
	Оператор $H_1$ является В-регулярным и, более того, $\B(H_1) = \B$.
\end{remark}


\begin{theorem}
	\label{thm:A_block_thrower_is_B-regular}
	Оператор $A$ является В-регулярным.
\end{theorem}

\begin{proof}
	Непосредственно проверим условия леммы~\ref{lem:suff_B_reg}.

	i) Всякий оператор взятия подпоследовательности является неотрицательным.

	ii) $A\one = \one \in ac_1$.

	iii)  В силу того, что оператор $A$ является оператором взятия подпоследовательности,
	выполнено включение $Ac_0 \subset c_0 \subset ac_0$.

	iv) В силу леммы~\ref{lem:AT-TA} для любого $x\in \ell_\infty$ выполнено $(AT-TA)x \in \mathcal{I}(ac_0) \subset ac_0$.

	Таким образом, оператор $A$ действительно является В-регулярным.
\end{proof}

\begin{lemma}
	Для любого банахова предела $B_1$ выполено $B_1 A \in \mathfrak B \setminus \B (\sigma_{2^{2n-1}})$, $n\in \N$.
\end{lemma}

\begin{proof}
	Напомним (см. определение~\ref{def:Damerau_Levenshein_distance}), что знак приближенного равенства $ u \approx w$ означает,
	что конечно расстояние Дамерау--Левенштейна между последовательностями
	$u$ и $w$,
	т.е. что $w$ можно получить из $u$ конечным числом вставок, удалений и замен элементов.
	Тогда $Bu = Bw$ для любого $B\in\B$ и $u \approx w$.

	Положим $x = \chi_{\cup_{m=0}^{\infty}\left[2^{2 m}+1, 2^{2 m+1}\right]}$.
	Тогда $\sigma_{2^{2n}} x \approx x$ и $\sigma_{2^{2n-1}} x \approx \one - x$.
	Заметим, что $Ax \approx \one$.

	Предположим противное утверждению теоремы.
	Пусть $B \in \B(A) \cap \B (\sigma_{2^{2n-1}})$, $n\in \N$.
	Тогда
	\begin{multline}
		1 = B\one = BAx = (BA)x = Bx =
		(B\sigma_{2^{2n-1}}) x = B (\sigma_{2^{2n-1}} x) =
		\\=
		B(\one -x) = 1 - Bx = 1 - (BA)x = 1-1 = 0
		.
	\end{multline}
	Полученное противоречие завершает доказательство.
\end{proof}

\begin{remark}
	Например, для $B_1 \in \mathfrak B (\sigma_2)$ имеем $B_1 \ne B_1 A$, что показывает,
	что оператор $A$ является существенно эберлейновым.
\end{remark}

\begin{remark}
	Оператор $A$ переводит банаховы пределы <<достаточно далеко>>. Действительно, пусть
	\begin{equation}
		y \approx 2\chi_{\cup_{m=0}^{\infty}\left[2^{2 m}+1, 2^{2 m+1}\right]} - \one
		,
	\end{equation}
	тогда
	\begin{equation}
		\sigma_2 y \approx 2\chi_{\cup_{m=0}^{\infty}\left[2^{2 m+1}+1, 2^{2 m+2}\right]} - \one
		.
	\end{equation}
	Пусть $B\in\B$, тогда $BA\in\B$ и $BA\sigma_2\in\B$ в силу теоремы~\ref{thm:A_block_thrower_is_B-regular}.
	Но
	\begin{multline}
		\|BA-BA\sigma_2\| \geq \dfrac{\| BAy-BA\sigma_2y \|}{\|y\|} = \| BAy-BA\sigma_2y \| =
		\\=
		\left\| BA\left(2\chi_{\cup_{m=0}^{\infty}\left[2^{2 m}+1, 2^{2 m+1}\right]} - \one\right)-
		BA\left(2\chi_{\cup_{m=0}^{\infty}\left[2^{2 m+1}+1, 2^{2 m+2}\right]} - \one\right) \right\|=
		\\=
		\left\| 2BA\chi_{\cup_{m=0}^{\infty}\left[2^{2 m}+1, 2^{2 m+1}\right]}-
		2BA\chi_{\cup_{m=0}^{\infty}\left[2^{2 m+1}+1, 2^{2 m+2}\right]}  \right\|=
		\\=
		2\left\| BA\chi_{\cup_{m=0}^{\infty}\left[2^{2 m}+1, 2^{2 m+1}\right]}-
		BA\chi_{\cup_{m=0}^{\infty}\left[2^{2 m+1}+1, 2^{2 m+2}\right]}  \right\|=
		\\=
		2\left\| B \one- B (0\cdot \one)  \right\|=
		2\cdot| 1- 0  |= 2 = \diam \B
		.
		\end{multline}
\end{remark}

\begin{remark}
	Оператор $A$ переводит банаховы пределы <<достаточно далеко>> от $\B(\sigma_{2^{2n-1}})$.
	%TODO2: \sigma_{2^{2n+1}}
	Пусть снова
	\begin{equation}
		y \approx 2\chi_{\cup_{m=0}^{\infty}\left[2^{2 m}+1, 2^{2 m+1}\right]} - \one
		.
	\end{equation}
	Рассмотрим $B_1 \in \B(\sigma_{2^{2n-1}})$, $n\in\N$.
	Тогда  $y \approx - \sigma_{2^{2n-1}} y$,
	откуда $B_1 y = B_1 \sigma_{2^{2n-1}} y = 0$.
	Однако для любого $B_2 \in \B$ имеем $B_2 A y = 1$,
	т.е.
	\begin{equation}
		\|B_1 - B_2 A\| \geq \dfrac{\|B_1y - B_2 Ay\|}{\|y\|} = \|B_1y - B_2 Ay\| = 1
		.
	\end{equation}
\end{remark}

\begin{hypothesis}
	Пусть $B_1 \in \B(\sigma_{2^{2n-1}})$, $n\in\N$, и $B_2\in \B$.
	Тогда $\|B_1 - B_2 A\| = 2 = \diam \B$.
\end{hypothesis}

\begin{hypothesis}
	Множество $\B(A) \cap \B (\sigma_{2^{2n}})$ непусто для любого $n\in \N$.
\end{hypothesis}

\begin{hypothesis}
	Если $B\in\ext\B$, то $BA \in \ext \B$.
\end{hypothesis}

Более того, реалистичной видится даже существенно более сильная

\begin{hypothesis}
	Для любого $B\in\B$ выполнено $BA \in \ext \B$.
\end{hypothesis}


	\section{Пример эберлейнового не В-регулярного оператора}
	\begin{theorem}
	\label{thm:Eberlein_but_not_B-regular_exists}
	Существует эберлейнов оператор, не являющийся В-регулярным.
\end{theorem}

\begin{proof}
	Пусть оператор $A$ определён формулой~\eqref{eq:oper_A_throws_out_2power_blocks}.
	Пусть оператор $E:\ell_\infty\to\ell_\infty$ определён формулой
	\begin{equation}
		Ex = x \cdot \chi_{\cup_{m=0}^{\infty}\left[2^{2 m}+1, 2^{2 m+1}\right] \cup\{1\}}
		.
	\end{equation}
	Тогда, очевидно, $AE=A$.
	Поскольку $A$ есть В-регулярный оператор, то он эберлейнов,
	т.е. множество $\B(A)$ непусто.

	Пусть $B\in\B(A)$. Тогда
	\begin{equation}
		B = BA = B(AE) = (BA)E = BE
		,
	\end{equation}
	то есть $B\in\B(E)$.

	Покажем теперь, что оператор $E$ не является В-регулярным.
	Действительно, $q(E\one) =0$, т.е. $ E\one \notin ac_1$,
	и не выполнено условие (i) критерия В-регулярности (теорема~\ref{thm:crit_B_regularity}).
\end{proof}

\begin{hypothesis}
	Выполнено равенство
	\begin{equation}
		\B(E) = \B(A)
		.
	\end{equation}
\end{hypothesis}

\begin{remark}
	В конструкции оператора $E$ вместо блоков из нулей можно вставлять любые другие блоки "---
	они всё равно будут поглощены оператором $A$.
	Эти блоки могут иметь различный знак, например, содержать значительное количество
	элементов вида $-kx_1$, $kx_2$ и т.д.
	Таким образом, можно сконструировать эберлейновы операторы, очень и очень далёкие
	от достаточных условий эберлейновости (теорема~\ref{thm:Semenov_Sukochev_conditions}),
	что показывает значительную избыточность этих условий.
\end{remark}


	\section{Обратная задача об инвариантности и порождённый оператор}
	Некоторые результаты этого параграфа опубликованы в~\cite[\S 3]{avdeev2024decomposition}.

Ранее мы, как правило, ставили задачу, которую можно назвать прямой задачей об инвариантности:
дан некоторый достаточно хороший оператор $H$, и требуется выяснить, непусто ли множество $\B(H)$
банаховых пределов, инвариантных относительно этого оператора.
(И~если это множество непусто, то исследовать его.)

Возникает закономерный вопрос: для любого ли банахова предела существует нетривиальный оператор,
относительно которого этот банахов предел инвариантен?
Или же такие операторы существуют только для <<достаточно хороших>>
банаховых пределов "--- например, для банаховых пределов, инвариантных относительно какого-нибудь из операторов растяжения?
Существуют ли нетривиальные операторы, инвариантные относительно хотя бы одного $B\in\ext \B$?
(Здесь под тривиальным оператором понимается такой оператор
$H:\ell_\infty \to \ell_\infty$, что $H-I:\ell_\infty \to ac_0$ и,
как следствие, $\B(H) = \B$.)

Итак, в этом параграфе мы обсудим обратную задачу об инвариантности.
Она имеет неожиданно простое решение.

\begin{theorem}
	\label{thm:generated_operator_G_B}
	Для каждого $B\in \B$ существует такой оператор $G_B:\ell_\infty \to \ell_\infty$,
	что $\B(G_B) = \{B\}$.
\end{theorem}

\begin{proof}
	Определим оператор $G_B$ равенством
	\begin{equation}
		G_B x = (Bx, Bx, Bx, ...) = (Bx)\cdot\one
		.
	\end{equation}
	Легко видеть, что
	\begin{equation}
		B G_B x = B(Bx, Bx, Bx, ...) = B((Bx)\cdot\one) = (Bx)\cdot B(\one)= (Bx)\cdot 1 = Bx
	\end{equation}
	для любого $x\in\ell_\infty$.
	Значит, $B\in \B (G_B)$ и $\B (G_B)$ непусто.

	Рассмотрим теперь $B_1 \in \B \setminus\{B\}$.
	Последнее означает, что на некоторой последовательности $y\in\ell_\infty$ выполнено $B_1y \ne By$.
	Тогда
	\begin{equation}
		B_1 G_B y = B_1(By, By, By, ...) = B_1((By)\cdot\one) = (By)\cdot B_1(\one)= (By)\cdot 1 = By \ne B_1y
		.
	\end{equation}
	Таким образом, $B_1 G_B \ne B_1$, и $B_1 \notin \B (G_B)$.
	Это и означает, что $\B(G_B) = \{B\}$.
\end{proof}

\begin{definition}
	Оператор $G_B$, построенный таким образом, будем называть оператором,
	\emph{порождённым} банаховым пределом $B$.
\end{definition}

\begin{remark}
	Легко заметить, что любой порождённый оператор удовлетворяет достаточным условиям эберлейновости
	(теорема~\ref{thm:Semenov_Sukochev_conditions}).
\end{remark}

\begin{remark}
	Более того, порождённый оператор $G_B$ является в некотором смысле крайним примером В-регулярного оператора,
	поскольку $G_B^* \B = \{B\}$.
\end{remark}

\begin{lemma}
	Пусть $0 \leq \lambda \leq 1$, $B_1, B_2 \in \B$.
	Тогда
	\begin{equation}
		G_{\lambda B_1+(1-\lambda) B_2} =\lambda G_{B_1} + (1-\lambda)G_{B_2}
		.
	\end{equation}
\end{lemma}


\begin{proof}
	В силу выпуклости множества $\B$ оператор $G_{\lambda B_1+(1-\lambda) B_2}$
	действительно определён корректно.
	\begin{multline}
		G_{\lambda B_1+(1-\lambda) B_2} =
		\left((\lambda B_1+(1-\lambda) B_2) x\right)\one =
		(\lambda B_1 x) \one +((1-\lambda) B_2 x)\one =
		\\=
		\lambda (B_1 x) \one +(1-\lambda) (B_2 x)\one =
		\lambda G_{B_1} x +(1-\lambda) G_{B_2} x
		.
	\end{multline}
\end{proof}


\begin{lemma}
	Пусть $B_1, B_2 \in \B$.
	Тогда
	\begin{equation}
		G_{B_1} G_{B_2} = G_{B_2}
		.
	\end{equation}
\end{lemma}
\begin{proof}
	Пусть $x\in\ell_\infty$, тогда
	\begin{equation}
		G_{B_1}G_{B_2}x =
		G_{B_1} ( (B_2 x) \cdot \one ) =
		(B_2 x) \cdot G_{B_1} ( \one ) =
		(B_2 x) \cdot  \one =
		G_{B_2}x
		.
	\end{equation}
\end{proof}


Если же известно, что банахов предел $B$ заведомо обладает дополнительными свойствами инвариантности,
то можно сконструировать и другие примеры операторов, относительно которых $B$ инвариантен.

\begin{example}
	Пусть $B\in\mathfrak B(\sigma_2)$.
	Напомним, что $\B(\sigma_2) = \B(\sigma_{1/2})$ в силу теоремы~\ref{thm:B_sigma_n_eq_B_sigma_1_n}, где
	\begin{equation}
		\sigma_{1/2} x = \left(\dfrac{x_1+x_2}2, \dfrac{x_3+x_4}2, \dfrac{x_5+x_6}2, ...\right)
		.
	\end{equation}
	Рассмотрим оператор $H_B$, $H_Bx = (x_1, Bx, x_2, Bx, x_3, Bx, ...)$.
	Тогда
	\begin{multline}
		B H_B x = B \sigma_{1/2} H_B x = B\left(\dfrac{x_1+Bx}2, \dfrac{x_2+Bx}2, \dfrac{x_3+Bx}2, ...\right) =
		\\=
		B\left(\dfrac{x_1}2, \dfrac{x_2}2, \dfrac{x_3}2, ...\right) + B\left(\dfrac{Bx}2, \dfrac{Bx}2, \dfrac{Bx}2, ...\right)=
		\\=
		\dfrac12 Bx + \dfrac12 B\left(Bx, Bx, Bx, ...\right) = Bx
		.
	\end{multline}

	При этом операторы $H_B$, получаемые таким образом, достаточно <<разнообразны>>.
	Пусть $B_1 x \ne B_2 x$ на некотором $x\in \ell_\infty$.
	Тогда
	\begin{equation}
		y = H_{B_1} x - H_{B_2} x = (0, B_1 x - B_2 x, 0, B_1 x - B_2 x, 0, B_1 x - B_2 x, ...)
	\end{equation}
	 и
	\begin{equation}
		B_1y = B_2y = \dfrac{B_1 x - B_2 x}2\ne 0
		,
	\end{equation}
	откуда
	\begin{equation}
		(H_{B_1} - H_{B_2})\ell_\infty \not\subseteq ac_0
		.
	\end{equation}


	Покажем  теперь, что $B_2 \notin \mathfrak B (H_{B_1})$ при $B_1 \ne B_2$. Предположим противное. Значит, $B_1 x \ne B_2 x$ на некотором $x\in \ell_\infty$. Пусть $y$ такой же, как выше. Тогда
	\begin{equation}
		B_2 y = B_2(H_{B_1} x - H_{B_2} x) = B_2 H_{B_1} x - B_2 H_{B_2} x = B_2 x - B_2 x = 0
		,
	\end{equation}
	но
	\begin{equation}
		B_2 y = \dfrac{B_1x - B_2 x}{2} \ne 0
		.
	\end{equation}
	Получили противоречие.
\end{example}

Полученная конструкция может быть обобщена на операторы $\sigma_k$, $k\in\N_3$.

\begin{example}
	Пусть $k\in \N_2$, $1 \leq m \leq k$, $m\in\N$, и пусть $B \in \B(\sigma_k)$.
	Определим оператор $H_{B,k,m}:\ell_\infty \to \ell_\infty$ равенством
	\begin{equation}
		H_{B,k,m} x = (
			\underbrace{x_1, ..., x_1}_{k-m~\mbox{раз}}, \underbrace{Bx, ..., Bx}_{m~\mbox{раз}},
			\underbrace{x_2, ..., x_2}_{k-m~\mbox{раз}}, \underbrace{Bx, ..., Bx}_{m~\mbox{раз}},
			...)
		.
	\end{equation}
	Тогда для любого $x\in\ell_\infty$
	\begin{multline}
		B H_{B,k,m} x =
		B \sigma_{1/k} H_{B,k,m} x =
		\\=
		B\left(\dfrac{mx_1 + (k-m) Bx}{k}, \dfrac{mx_2 + (k-m) Bx}{k}, \dfrac{mx_3 + (k-m) Bx}{k}, ...\right)=
		\\=
		\dfrac{m}{k} \cdot Bx + \dfrac{k-m}{k} B ( (Bx)\cdot \one) =
		(Bx) \cdot \left(\dfrac m k + \dfrac{k-m}k \right) = Bx
		,
	\end{multline}
	что и означает принадлежность $B\in\B(H_{B,k,m})$.

	И снова покажем, если $B_1, B_2 \in \B(\sigma_k)$ и $B_1 \ne B_2$, то $B_2 \ne \B (H_{B_1,k,m})$.
	Действительно, если $B_1 \ne B_2$, то $B_1y \ne B_2x$ для некоторого $x\in\ell_\infty$.
	Тогда
	\begin{multline}
		B_2 H_{B_1,k,m} x =
		B_2 \sigma_{1/k} H_{B_1,k,m} x =
		\\=
		B_2\left(\dfrac{mx_1 + (k-m) B_1x}{k}, \dfrac{mx_2 + (k-m) B_1x}{k}, \dfrac{mx_3 + (k-m) B_1x}{k}, ...\right)=
		\\=
		\dfrac{m}{k} \cdot B_2x + \dfrac{k-m}{k} B_2 ( (B_1x)\cdot \one) =
		(B_2x) \cdot \dfrac m k + (B_1x)\cdot \dfrac{k-m}k \ne B_2x
		,
	\end{multline}
	откуда $B_2\notin \B(H_{B_1,k,m})$.
\end{example}

\begin{remark}
	Внутри каждого отрезка последовательности из определения оператора $H_{B,k,m}$,
	состоящего из $k$ элементов и имеющего вид $(x_j, ..., x_j, Bx, ..., Bx)$,
	порядок элементов можно произвольно менять (независимо от порядка в других отрезках).
\end{remark}

\begin{hypothesis}
	Для любого $B\in\B(\sigma_k)$ выполнено равенство $\B(H_{B,k,m}) = \{B\}$ или, что то же самое,
	включение $\B(H_{B,k,m}) \subset \B(\sigma_k)$.
\end{hypothesis}

При изучении введённых выше классов операторов возникает вопрос о том,
как обсуждаемые свойства (полуэберлейновость, эберлейновость и т.д.)
соотносятся с классическими свойствами линейных операторов.

Например, существует ли компактный эберлейнов оператор?
Положительный ответ на этот вопрос даёт

\begin{theorem}
	Пусть $B\in\B$.
	Тогда порождённый оператор $G_B$ компактен.
\end{theorem}

\begin{proof}
	Пусть $X$ "--- единичный шар в $\ell_\infty$,
	тогда $BX = [-1;1]$
	и множество $G_B X = [-1;1] \cdot \one$
	компактно,
	поскольку сходимость любой его подпоследовательности $\{y_j = G_B x_j = (Bx_j) \cdot \one\}_{j\in\N}$
	в норме пространства $\ell_\infty$ эквивалентна сходимости последовательности
	$\{Bx_j\}_{j\in\N}$ в норме пространства $\R$,
	в котором любой отрезок "--- компакт.
\end{proof}

Таким образом, мы предъявили компактный существенно эберлейнов оператор.
Заметим, что многие часто встречающиеся операторы некомпактны
(что и привело к постановке обсуждаемого вопроса).
Так, очевидно, некомпактны операторы растяжения $\sigma_k$, $\sigma_{1/k}$ и $\tilde\sigma_k$.
Некомпактность оператора Чезаро $C$ установлена, например, в~\cite[теорема 4.4]{ALALAM20181038}.


	\section{Мультипорождённые операторы}
	\begin{definition}
	Линейный непрерывный оператор $G_{\{B_k\}}$, определённый равенством
	\begin{equation}
		(G_{\{B_k\}}x)_n = B_n x
		,
	\end{equation}
	будем называть мультипорождённым последовательностью банаховых пределов $\{B_k\}\subset \B$.
\end{definition}

Понятно, что порождённый оператор является частным случаем мультипорождённого.

\begin{lemma}
	\label{lem:multigen_is_B-regular}
	Всякий мультипорождённый оператор $G_{\{B_k\}}$ является В-регулярным оператором.
\end{lemma}

\begin{proof}
	Проверим условия критерия В-регулярности (теорема~\ref{thm:crit_B_regularity}).
	Действительно,
	\begin{equation}
		G_{\{B_k\}} \one = \one \in ac_1
		,
	\end{equation}
	\begin{equation}
		G_{\{B_k\}}\geq 0, \ \mbox{т.к.}\ (G_{\{B_k\}}x)_n \geq q(x) \ \mbox{для любого}~n\in\N
		,
	\end{equation}
	\begin{equation}
		G_{\{B_k\}} ac_0 = \{0\cdot \one \} \subset ac_0
		.
	\end{equation}
\end{proof}

\begin{corollary}
	Для всякого мультипорождённого оператора $G_{\{B_k\}}$ непусто множество $\B(G_{\{B_k\}})$
\end{corollary}

\begin{hypothesis}
	Всякий мультипорождённый оператор является существенно дружелюбным.
\end{hypothesis}
В пользу этой гипотезы говорит тот факт~\cite{Chou},
что $|\ext B| = 2^{\mathfrak c}$.

\begin{lemma}
	Пусть множество ${\{B_k\}}$ конечно, тогда оператор $G_{\{B_k\}}$ компактен.
\end{lemma}

\begin{proof}
	Достаточно заметить, что в таком случае ранг подпространства $G_{\{B_k\}} \ell_\infty$ конечен.
\end{proof}

\begin{hypothesis}
	Оператор $G_{\{B_k\}}$ компактен тогда и только тогда, когда множество ${\{B_k\}}$ конечно.
\end{hypothesis}

\begin{hypothesis}
	Для мультипорождённого оператора $G_{\{B_k\}}$ имеет место включение
	\begin{equation}
		\B (G_{\{B_k\}}) \subset \conv {\{B_k\}}
		.
	\end{equation}
	(Та же гипотеза "--- для равенства или для обратного включения)
\end{hypothesis}

\begin{remark}
	Мультипорождённые операторы по своему определению в некотором смысле противоположны матричным операторам
	(к последним относятся, например, оператор Чезаро $C$, операторы растяжения $\sigma_k$, усредняющего сжатия $\sigma_{1/k}$,
	операторы неравномерного растяжения $\tilde\sigma_k$, оператор прореживания $A$ (равенство~\eqref{eq:oper_A_throws_out_2power_blocks}) и оператор покоординатного умножения $E$ из теоремы~\ref{thm:Eberlein_but_not_B-regular_exists}).
\end{remark}

\begin{theorem}
	Мультипорождённый оператор не может иметь матричного представления.
\end{theorem}

\begin{proof}
	Справедливость теоремы непосредствено вытекает из нижеследующего результата А.А. Седаева~\cite[\S 6.3]{sedaev2009_doc_vgasu}:
	\begin{equation}
		\ell_\infty^* = \ell_1 \oplus \{\varphi \in\ell_\infty^* : \varphi c_0 = \{0\}\}
		.
	\end{equation}
\end{proof}

Перейдём теперь к примеру мультипорождённого оператора, важному для понимания инвариантности банаховых пределов.

\begin{example}
	\label{ex:multigen_invariant_interval}
	Пусть оператор $A:\ell_\infty\to\ell_\infty$
	опеределён равенством~\eqref{eq:oper_A_throws_out_2power_blocks}:
	\begin{multline}
		Ax = (x_1, x_2, \not x_3, \not x_4, x_5, x_6, x_7, x_8, \not x_9, ..., \not x_{16}, x_{17}, x_{18}, ..., x_{31}, x_{32}, \not x_{33}, \not x_{34}, ..., \not x_{64},
		\\
		x_{65}, x_{66}, ..., x_{128}, \not x_{129}, ...)=
		\\=
		(x_1, x_2, \ x_5, x_6, x_7, x_8, \ x_{17}, x_{18}, ..., x_{31}, x_{32}, \ x_{65}, x_{66}, ..., x_{128}, \ x_{257},
		\\
		..., \ x_{2^{2n} +1}, x_{2^{2n} +2},  x_{2^{2n+1}}, \ \ x_{2^{2(n+1)} +1},  x_{2^{2(n+1)} +2},  x_{2^{2(n+1)+1}}, ...)
	\end{multline}
	Тогда оператор $A$ В-регулярен по теореме~\ref{thm:A_block_thrower_is_B-regular}.
	Более того, в силу В-регулярности оператора $\sigma_2$ и того факта,
	что суперпозиция двух В-регулярных операторов снова В-регулярна (лемма~\ref{lem:B-regular_superposition_and_addition}),
	оператор $A\sigma_2$ также В-регулярен.

	В силу теоремы~\ref{thm:B-regular_is_Eberlein} мы можем выбрать $B_1 \in \B (A)$ и $B_2 \in \B(A\sigma_2)$.
	Убедимся, что $B_1 \ne B_2$, для этого достаточно предъявить одну последовательность, на которой
	значения этих банаховых пределов не совпадают.
	Пусть
	\begin{equation}
		y = \chi_{\cup_{n\in\N}(2^{2n}+1, 2^{2n+1}]}
		,
	\end{equation}
	тогда
	\begin{equation}
		\sigma_2 y \approx \chi_{\cup_{n\in\N}(2^{2n+1}+1, 2^{2n+2}]}
		.
	\end{equation}
	Следовательно, $B_1y = B_1Ay = B_1\one = 1$, но $B_2y = B_2A\sigma_2 y = B_2(0\cdot \one) = 0$.

	Определим оператор $H$ соотношением
	\begin{equation}
		Hx = (B_1 x, B_1 x, \ B_2 x, B_2 x, \ B_1 x, B_1 x, B_1 x, B_1 x, \ \underbrace{B_2 x, ..., B_2 x}_{8~\mbox{раз}}, ...)
		.
	\end{equation}

	Оператор $H$ является мультипорождённым и потому В-регулярным в силу леммы~\ref{lem:multigen_is_B-regular}.
	Значит, множество $\B(H)$ непусто.
	Однако в нашем случае мы можем дополнительно охарактеризовать его.

	Покажем, что $[B_1; B_2]\subset \B(H)$.
	Действительно,
	\begin{equation}
		B_1 Hx  = (B_1 A) Hx = B_1 (AHx)  = B_1 ((B_1 x) \cdot \one) = B_1 x
	\end{equation}
	и
	\begin{equation}
		B_2 Hx  = (B_2 A\sigma_2) Hx = B_2 (A\sigma_2 Hx)  = B_2 ((B_2 x) \cdot \one) = B_2 x
		,
	\end{equation}
	откуда и следует включение $[B_1; B_2]\subset \B(H)$.
\end{example}

\begin{hypothesis}
	Верно равенство $[B_1; B_2] = \B(H)$.
\end{hypothesis}

\begin{hypothesis}
	Верно равенство $[B_1; B_2] \cap \B(\sigma_2) = \varnothing$.
\end{hypothesis}

\begin{hypothesis}
	Верно равенство $B(H) \cap \B(\sigma_2) = \varnothing$.
\end{hypothesis}

\begin{remark}
	Оператор $(\sigma_2 H)^*$ переводит отрезок $[B_1; B_2]$ в $[B_2; B_1]$,
	т.е. <<меняет местами>> два банаховых предела.
	Действительно,
	\begin{equation}
		B_1 (\sigma_2 H)x  = (B_1 A) (\sigma_2 H)x = B_1 (A(\sigma_2 H)x)  = B_1 ((B_2 x) \cdot \one) = B_2 x
	\end{equation}
	и
	\begin{equation}
		B_2 (\sigma_2 H)x  = (B_2 A\sigma_2) (\sigma_2 H)x = B_2 (A\sigma_4 Hx)  = B_2 ((B_1 x) \cdot \one) = B_1 x
		.
	\end{equation}
	Тогда, очевидно, $\dfrac{B_1 + B_2}2 \in \B(\sigma_2 H)$.
\end{remark}

\begin{hypothesis}
	Пример~\ref{ex:multigen_invariant_interval} может быть обобщён на симплекс произвольной конечной размерности
	(треугольник, тетраэдр и т.д.).
\end{hypothesis}



\chapter{Функционалы Сачестона и линейные оболочки}

	При исследовании банаховых пределов особый интерес представляют разделяющие множества~\cite[\S 3]{Semenov2014geomprops}.
Множество $Q\in\ell_\infty$ называют разделяющим, если
для любых неравных $B_1, B_2\in\mathfrak{B}$ существует такая последовательность $x\in Q$,
что $B_1 x \neq B_2 x$.
В частности, разделяющим является~\cite{semenov2010characteristic} множество всех последовательностей из 0 и 1,
которое в дальнейшем мы будем обозначать через $\Omega$
(иногда в литературе встречается также обозначение $\{0;1\}^\mathbb{N}$).

Каждой последовательности $(x_1, x_2, \dots)\in \Omega$ можно поставить в соответствие число
\begin{equation}\label{eq:bijection_omega_0_1}
	\sum_{k=1}^\infty 2^{-k} x_k \in [0,1]
	.
\end{equation}
С точностью до счётного множества это соответствие взаимно однозначно и определяет на множестве $\Omega$ меру,
которую мы будем отождествлять с мерой Лебега на $[0,1]$.

Оказывается, что из $\Omega$ можно выделить некоторые подмножества, которые также будут разделяющими,
например \cite[\S 3, Теорема 11]{Semenov2014geomprops},
\begin{equation}
	U = \{ x\in\Omega: q(x) = 0, p(x) = 1 \}
	.
\end{equation}

Однако множество $U$ имеет меру 1~\cite{semenov2010characteristic}.

В настоящей главе строится пример разделяющего множества,
являющегося подмножеством $\Omega$ и имеющего меру нуль.
Для построения такого множества используется следующий факт.

\begin{lemma}[{\cite[\S 3, замечание 6]{Semenov2014geomprops}}]
	Пусть $X$~--- разделяющее множество и $X \subset \operatorname{Lin} Y$,
	где $\operatorname{Lin} Y$ обозначает линейную оболочку $Y$.
	Тогда $Y$ также является разделяющим множеством.
\end{lemma}

Затем обсуждаются свойства линейных оболочек множеств, определённых с помощью функционалов Сачестона.
В частности, доказывается,
что наряду с использованным при построении разделяющего множества меры нуль включением
\begin{equation}
	\Omega \subset \operatorname{Lin}\{x\in\Omega : p(x) = a,~ q(x) = b\}
\end{equation}
для любых $0\leq b < a \leq 1$,
имеет место равенство
\begin{equation}
	\ell_\infty = \operatorname{Lin}\{x\in\ell_\infty : p(x) = a,~ q(x) = b\}
\end{equation}
для любых $a>b$.

Возникает закономерный вопрос: для каких ещё подмножеств пространства $\ell_\infty$
верны аналогичные соотношения?

Оказывается, что аналогичным свойством обладает и ещё одно подмножество пространства $\ell_\infty$: подпространство
$A_0 = \{ x \in \ell_\infty : \alpha(x) =0 \}$,
где, напомним,
\begin{equation*}
	\alpha(x) = \varlimsup_{i\to\infty} \max_{i<j\leqslant 2i} |x_i - x_j|
	.
\end{equation*}

Пространство $A_0$ обладает рядом интересных свойств.


\begin{theorem}[{\cite[следствие 2]{our-mz2019ac0}}]
	\label{thm:alpha_c_ac_c}
	Пусть $x\in ac$, т.е. $p(x) = q(x)$.
	Тогда $x\in c$ если и только если $\alpha(x) = 0$.
\end{theorem}
Таким образом, $c = ac \cap A_0$.
Включение $c\subset A_0$ собственное.

Некоторые результаты данной главы были анонсированы в~\cite{our-mz2021linearhulls}
и опубликованы в~\cite{avdeev2021vestnik}.


	\section{Вспомогательные построения}
	В данном параграфе вводятся некоторые вспомогательные объекты,
которые потребуются далее при доказательстве теоремы~\ref{thm:Lin_Omega_Sucheston}.

\subsection{Двоичные приближения}

\begin{definition}
	$k$-м двоичным приближением к произвольному числу $d\in[0;1]$
	называется такое число $d_{(k)}\in\mathbb{N}\cup\{0\}$,
	что
	\begin{equation}
		\label{eq:binary_approximations_for_number}
		\frac{d_{(k)}}{2^k} < d \leq \frac{d_{(k)}+1}{2^k}
		.
	\end{equation}
\end{definition}

\begin{remark}
	Очевидно, что $d_{(k+1)}\in\{2d_{(k)},2d_{(k)}+1\}$.
\end{remark}

Ввиду того, что мы вводим двоичные приближения к дроби из отрезка $[0;1]$ как натуральные числа,
дадим примеры построения таких приближений.

\begin{example}
	Пусть $d=\frac12$.
	Тогда
	$$
		d_{(1)} = 0, ~~~\mbox{поскольку}~~ \frac{0}{2^1} < \frac12 \leq \frac{0+1}{2^1};
	$$
	$$
		d_{(2)} = 1, ~~~\mbox{поскольку}~~ \frac{1}{2^2} < \frac12 \leq \frac{1+1}{2^2};
	$$
	$$
		d_{(3)} = 3, ~~~\mbox{поскольку}~~ \frac{3}{2^3} < \frac12 \leq \frac{3+1}{2^3};
	$$
	$$
		d_{(4)} = 7, ~~~\mbox{поскольку}~~ \frac{7}{2^4} < \frac12 \leq \frac{7+1}{2^4};
	$$
	$$
		\dots\dots\dots
	$$
	$$
		d_{(k)} = 2^{k-1}-1, ~\mbox{поскольку}~~ \frac{2^{k-1}-1}{2^k} < \frac12 \leq \frac{2^{k-1}}{2^k}.
	$$
\end{example}

\begin{example}
	Пусть $d=\frac13$.
	Тогда
	$$
		d_{(1)} = 0, ~~~\mbox{поскольку}~~ \frac{0}{2^1} < \frac13 \leq \frac{1}{2^1};
	$$
	$$
		d_{(2)} = 1, ~~~\mbox{поскольку}~~ \frac{1}{2^2} < \frac13 \leq \frac{2}{2^2};
	$$
	$$
		d_{(3)} = 2, ~~~\mbox{поскольку}~~ \frac{2}{2^3} < \frac13 \leq \frac{3}{2^3};
	$$
	$$
		d_{(4)} = 5, ~~~\mbox{поскольку}~~ \frac{5}{2^4} < \frac13 \leq \frac{6}{2^4};
	$$
	$$
		\dots\dots\dots
	$$
\end{example}


\subsection{Последовательности-<<блоки>>}

Введём последовательности-<<блоки>> "---
стабилизирующиеся на нуле последовательности,
которые затем будут использованы для формирования последовательностей,
обладающих некоторыми интересными свойствами.

Пусть $n$ зафиксировано.
Пусть
\begin{equation}
	K = \{k\in\mathbb{N} : k \geq n\} = \{n, n+1, n+2, ...\}
	.
\end{equation}


Определим функцию $\operatorname{Br}:K\times [0;1] \to \ell_\infty$,
генерирующую <<блоки>> из нулей и единиц,
соответствующие приближению $d_{(k)}$ к числу $d\in[0;1]$ для $k \geq n$.

Определение $\operatorname{Br}$ построим рекурсивно.
Сначала определим $\operatorname{Br}(k,d)$ для $k=n$ по следующему правилу:
\begin{equation}
	(\operatorname{Br}(n,d))_j = \begin{cases}
		1, & \mbox{~если~} 2^n - d_{(n)} < j \leq 2^n,
		\\
		0  & \mbox{~для остальных~} j
		.
	\end{cases}
\end{equation}
Заметим, что все элементы $\operatorname{Br}(n,d)$, начиная с $(2^n+1)$-го, равны нулю;
кроме того, в $\operatorname{Br}(n,d)$ ровно $d_{(n)}$ единиц.

Для каждого $k \geq n$ положим
\begin{equation}
	\label{eq:Br(k+1,d)}
	\operatorname{Br}(k+1,d) = \operatorname{Br}(k,d) + T^{2^k}\operatorname{Br}(k,d) + (d_{(k+1)}-2d_{(k)})e_{2^k+2^n-d_{(n)}}
	,
\end{equation}
где через $e_j$ обозначен $j$-й орт.


\begin{proposition}
	\label{prop:Br_k_c_0_1}
	Последовательность $\operatorname{Br}(k,d)$ состоит из нулей и единиц.
\end{proposition}
\begin{proof}
	Легко доказать по индукции, что все элементы $\operatorname{Br}(k,d)$, начиная с $(2^k+1)$-го, равны нулю.
	Следовательно, носители первых двух слагаемых в~\eqref{eq:Br(k+1,d)} не пересекаются.
	Далее заметим, что третье слагаемое отлично от нуля тогда и только тогда,
	когда переход между приближениями $d_{(k)} / 2^k$ и $d_{(k+1)}/2^{k+1}$
	приводит к улучшению приближения.
	Более того,
	\begin{multline}
		\left(\operatorname{Br}(k,d) + T^{2^k}\operatorname{Br}(k,d)\right)_{2^k+2^n-d_{(n)}}
		=
		(\operatorname{Br}(k,d))_{2^n-d_{(n)}}
		=
		(\operatorname{Br}(k-1,d))_{2^n-d_{(n)}}
		=
		\\=
		...
		=
		(\operatorname{Br}(n,d))_{2^n-d_{(n)}}
		=
		0
		,
	\end{multline}
	т.е. выражение~\eqref{eq:Br(k+1,d)} действительно задаёт последовательность из нулей и единиц.
\end{proof}

\begin{remark}
	Из доказательства утверждения~\ref{prop:Br_k_c_0_1} следует, что в $k$-м блоке ровно $d_{(k)}$ единиц.
\end{remark}


\begin{remark}
	Выполнено включение $\operatorname{supp}\operatorname{Br}(k,d) \subset \operatorname{supp}\operatorname{Br}(k+1,d)$
	и, более того, справедливо соотношение
	\begin{equation}
		(\operatorname{Br}(k,d))_j = \begin{cases}
			(\operatorname{Br}(k+1,d))_j, & \mbox{~если~}  j \leq 2^k,
			\\
			0  & \mbox{~для остальных~} j
			.
		\end{cases}
	\end{equation}
\end{remark}

\begin{lemma}
	\label{lem:sum_Br_k_c}
	Для любых таких $m$, $k$ и $i$, что $n \leq m \leq k$ и  $ i + 2^m - 1 \leq 2^k$,
	выполнено
	\begin{equation}
		d_{(m)} \leq \sum_{j=i}^{i+2^m-1} (\operatorname{Br}(k,d))_j \leq d_{(m)}+1
		.
	\end{equation}
\end{lemma}
\begin{proof}
	Представление~\eqref{eq:Br(k+1,d)} может быть переписано в виде:
	\begin{equation}
		\operatorname{Br}(m+1,d) = \sum_{j=0}^{1} T^{j2^m} \operatorname{Br}(m,d) + \sum_{j=0}^{1} \gamma_{j} e_{j 2^m+2^n-d_{(n)}}
		,
	\end{equation}
	где $\gamma_{j} \in \{0,1\}$.
	Продолжая по индукции, получаем
	\begin{equation}
		\operatorname{Br}(k,d) = \sum_{j=0}^{2^{k-m}-1} T^{j2^m} \operatorname{Br}(m,d) +
		\sum_{j=0}^{2^{k-m}-1} T^{j2^m} \gamma_{j} e_{j2^m+2^n-d_{(n)}}
		.
	\end{equation}
	Тогда
	\begin{equation}
		\sum_{j=i}^{i+2^m-1} (\operatorname{Br}(k,d))_j
		=
		\sum_{j=1}^{2^m-1} (\operatorname{Br}(m,d))_j
		+ \gamma_h
		=
		d_{(m)} + \gamma_h
		\in \{d_{(m)}, d_{(m)}+1\}
		.
	\end{equation}
\end{proof}

Из~\eqref{eq:binary_approximations_for_number} непосредственно вытекает следующий факт.
\begin{proposition}
	\label{prop:Br_has_nulls}
	Пусть $d<1-3/2^n$.
	Тогда
	\begin{equation}
		(\operatorname{Br}(d,k))_j = 0 ~~\mbox{для}~~j = m\cdot 2^n + 1, m\in\mathbb{N} \cup\{0\}
		.
	\end{equation}
\end{proposition}


\begin{example}
	Для $n=2$ и $d=1/3$ имеем:
	\begin{equation*}
		\begin{array}{lll}
			d_{(2)} = 1/4, &
			\operatorname{Br}(2,1/3) = (&0,0,0,1, \ 0,0,...)
			\\
			d_{(3)} = 2/8, &
			\operatorname{Br}(3,1/3) = (&0,0,0,1, \ 0,0,0,1, \ \   0,0,...)
			\\
			d_{(4)} = 5/16, &
			\operatorname{Br}(4,1/3) = (&0,0,0,1, \ 0,0,0,1, \ \   0,0,1,1, \ 0,0,0,1, \ \ \  0,0,...)
			\\
			d_{(5)} = 10/32, &
			\operatorname{Br}(5,1/3) = (&
			                             0,0,0,1, \ 0,0,0,1, \ \   0,0,1,1, \ 0,0,0,1,\\&
			                           & 0,0,0,1, \ 0,0,0,1, \ \   0,0,1,1, \ 0,0,0,1,
			\ \ \ 0,0,...)
			\\
			d_{(6)} = 21/64, &
			\operatorname{Br}(6,1/3) = (&
			                             0,0,0,1, \ 0,0,0,1, \ \   0,0,1,1, \ 0,0,0,1,\\&
			                           & 0,0,0,1, \ 0,0,0,1, \ \   0,0,1,1, \ 0,0,0,1,\\&
			                           & 0,0,1,1, \ 0,0,0,1, \ \   0,0,1,1, \ 0,0,0,1,\\&
			                           & 0,0,0,1, \ 0,0,0,1, \ \   0,0,1,1, \ 0,0,0,1,
			\ \ \ 0,0,...)
		\end{array}
	\end{equation*}
\end{example}

\subsection{Частичный предел в функционале Сачестона}

\begin{proposition}
	\label{prop:Sucheston_partial_limit}
	Предел в функционале Сачестона можно заменить частичным пределом, а именно
	\begin{equation*}
		p(x) = \lim_{n\to\infty} \sup_{m\in\mathbb{N}}  \frac{1}{n} \sum_{k=m+1}^{m+n} x_k
		= \lim_{n\to\infty} \sup_{m\in\mathbb{N}}  \frac{1}{2^n} \sum_{k=m+1}^{m+2^n} x_k
		.
	\end{equation*}
\end{proposition}
Аналогичное соотношение выполнено и для функционала $q(x)$.



	\section{Разделяющее множество нулевой меры}
	\begin{theorem}
	\label{thm:Lin_Omega_Sucheston}
	Пусть
	$1 \geq a > b \geq 0$ и
	$\Omega^a_b = \{x\in\Omega : p(x) = a, q(x) = b\}$,
	где $p(x)$ и $q(x)$~--- верхний и нижний функционалы Сачестона~\cite{sucheston1967banach} соответственно.
	Тогда $\Omega \subset \operatorname{Lin} \Omega^a_b$.
\end{theorem}

\begin{proof}
	Выберем $n\in\N$ таким образом, что
	\begin{equation}
		\label{eq:Omega_a_b_gap}
		a - b > \frac{3}{2^n}
	\end{equation}
	и $n$ чётно.

	Очевидно, что существует разложение
	\begin{equation}
		x = \sum_{i=0}^{k-1} T^i x_i, \quad x_i \in \Omega
		,
	\end{equation}
	где $k\in\N$ и все элементы последовательностей $x_i$,
	кроме имеющих индексы $km+1$, $m\in\N_0$, являются нулевыми.
	Пусть $k=2^n$; зафиксируем $i$ и в дальнейшем для удобства записи положим $w=x_i$.
	Наша задача~--- построить конечную линейную комбинацию элементов из $\Omega^a_b$, равную $w$.
	%
	Положим
	%\begin{equation}
	%	v = T^{2^n} \operatorname{Br}(2^ n   ,a) + T^{2^{n+1}} \operatorname{Br}(2^{n+1},b) +
	%	T^{2^{n+2}} \operatorname{Br}(2^{n+2},a) + T^{2^{n+3}} \operatorname{Br}(2^{n+3},b) + ...
	%\end{equation}
	%Эта сумма является формальной, т.е. не сходится в смысле ряда по норме,
	%однако носители слагаемых попарно не пересекаются.
	%Иначе говоря,
	\begin{equation}
		v_j = \begin{cases}
			0,  & \mbox{~если~} j \leq 2^n,
			\\
			(\operatorname{Br}(2^{2k  },a))_{j-2^{2k  }},  & \mbox{~если~} 2^{2k  } < j \leq 2^{2k+1}, 2k   \geq n,
			\\
			(\operatorname{Br}(2^{2k+1},b))_{j-2^{2k+1}},  & \mbox{~если~} 2^{2k+1} < j \leq 2^{2k+2}, 2k+1 \geq n
			.
		\end{cases}
	\end{equation}
	Иначе говоря, сначала <<резервируется>> $2^n$ нулевых элементов
	(большей частью для удобства записи, поскольку конечное количество членов в начале последовательности
	не влияет на функционалы Сачестона),
	а затем по очереди приписываются блоки "--- от первого элемента (нулевого) до последнего ненулевого элемента
	(конца носителя).
	Положим далее
	\begin{equation}
		u_j = \begin{cases}
			v_j + w_j,  & \mbox{~если~} j \leq 2^n
			\\
			            & \mbox{~или~} 2^{4k+3} < j \leq 2^{4k+4} \mbox{~и~} 4k + 3 \geq n,
			\\
			v_j         & \mbox{~для остальных~} j
			.
		\end{cases}
	\end{equation}

	В силу утверждения~\ref{prop:Br_has_nulls} все элементы, к которым прибавляются ненулевые элементы $w_j$, равны нулю.
	Кроме того, с учётом леммы~\ref{lem:sum_Br_k_c} и утверждения~\ref{prop:Sucheston_partial_limit}
	имеем $p(u)=p(w)=a$ и $q(u)=q(w)=b$.
	(На <<возмущённом>> блоке $u$ среднее, соответствующее функционалу $q$,
	увеличивается не более чем на $2^{-n}$ и не влияет на значение функционала $p$,
	в силу условия~\eqref{eq:Omega_a_b_gap}.)
	Следовательно, $u,v\in\Omega^a_b$.
	Заметим теперь, что
	\begin{equation}
		(u-v)_j = \begin{cases}
			w_j,  & \mbox{~если~} j \leq 2^n,
			\\
			0,  & \mbox{~если~} 2^{2k  } < j \leq 2^{2k+1}, 2k    \geq n,
			\\
			0,  & \mbox{~если~} 2^{4k+1} < j \leq 2^{4k+2}, 4k + 1 \geq n,
			\\
			w_j,  & \mbox{~если~} 2^{4k+3} < j \leq 2^{4k+4}, 4k + 3 \geq n
			.
		\end{cases}
	\end{equation}

	Аналогично строятся пары элементов, разность которых равна $w_j$ на $2^{4k+i} < j \leq 2^{4k+i+1}, 4k + i \geq n$ для $i=0,1,2$
	(требуется только обнулить первые $2^n$ элементов).
	Складывая полученные таким образом $4\cdot 2^n$ разностей элементов из $\Omega^a_b$, получаем требуемый элемент $x$.

\end{proof}

\begin{corollary}
	\label{crl:Lin_Omega_Sucheston}
	Множество $\Omega^a_b$ является разделяющим.
	Т.к. при $a\neq 1$ или $b\neq 0$ множество $\Omega^a_b$ имеет меру нуль~\cite{semenov2010characteristic,connor1990almost},
	то оно является разделяющим множеством нулевой меры.
\end{corollary}


	\section{Линейные оболочки множеств, определяемых функционалами Сачестона}
	Итак, $\Omega \subset \operatorname{Lin}\{x\in\Omega : p(x) = a,~ q(x) = b\}$, где $1\geq a>b\geq 0$.


Пусть $Y^a_b = \{x\in A_0 : p(x) = a, q(x) = b\}$, где $a>b$.
Подготовим сначала вспомогательные леммы о константе.

\begin{lemma}
	\label{lem:const_Lin_alpha_0}
	Пусть $a\neq -b$.
	Тогда справедливо включение
	$\mathbbm{1}\in \operatorname{Lin} Y^a_b$.
\end{lemma}

\begin{proof}
	Не теряя общности, будем полагать, что $a>0$.

	Определим оператор $S:\ell_\infty \to \ell_\infty$ следующим образом:
	\begin{equation}\label{operator_S}
		(Sy)_k = y_{i+2}, \mbox{ где } 2^i < k \leq 2^i+1
		.
	\end{equation}
	\begin{lemma}[{\cite{our-vzms-2018}}]
		Для любого $x\in \ell_\infty$ выполнено равенство
		\begin{equation}\label{alpha_S}
			\alpha(Sx) = \varlimsup_{k\to\infty} |x_{k+1} - x_{k}|
			.
		\end{equation}
	\end{lemma}
	Положим
	\begin{equation}
		\label{eq:y_for_s_alpha}
		y = \left(0,1,0,\frac{1}{2},1,\frac{1}{2},0,\frac{1}{3},\frac{2}{3},1,\frac{2}{3},\frac{1}{3},0,...\right)
		,
	\end{equation}
	тогда $Sy\in A_0$ и $\mathbbm{1}-Sy = S(\mathbbm{1}-y)\in A_0$.

	Пусть $x=(a-b)Sy+b\mathbbm{1}$, $z=(a-b)(\mathbbm{1}-Sy)+b\mathbbm{1}$.
	Тогда $p(x)=p(z)=a$, $q(x)=q(z)=b$ и, следовательно, $x,z\in Y^a_b$.
	Кроме того, заметим, что
	\begin{equation}
		x+z = (a-b)Sy+b\mathbbm{1} + (a-b)(\mathbbm{1}-Sy)+b\mathbbm{1}
		=
		(a-b)(Sy-Sy+\mathbbm{1}) + 2 b\mathbbm{1} = (a+b)\mathbbm{1}
		,
	\end{equation}
	откуда и следует, что $\mathbbm{1}\in Y^a_b$.
\end{proof}


\begin{lemma}
	\label{lem:const_Lin_alpha_0_a_eq_-b}
	Справедливо включение
	$\mathbbm{1}\in \operatorname{Lin} Y^a_{-a}$.
\end{lemma}

\begin{proof}
	Определим линейный оператор $M:\ell_\infty \to \ell_\infty$ следующим образом:
	\begin{multline}
		M\omega=\left(
			0, 1\omega_1,
			0, \frac{1}{2}\omega_2, 1\omega_2, \frac{1}{2}\omega_2,
			0, \frac{1}{3}\omega_3, \frac{2}{3}\omega_3, 1\omega_3, \frac{2}{3}\omega_3, \frac{1}{3}\omega_3,
			0, ...,
		\right. \\ \left.
			0, \frac{1}{p}\omega_p, \frac{2}{p}\omega_p, ..., \frac{p-1}{p}\omega_p, 1\omega_p,
				\frac{p-1}{p}\omega_p, ..., \frac{2}{p}\omega_p, \frac{1}{p}\omega_p,
			0, \frac{1}{p+1}\omega_{p+1}, ...
		\right)
		.
	\end{multline}
	Тогда $SM: \ell_\infty \to A_0$.
	Положим
	\begin{gather}
		x=aS(2M(\mathbbm{1})-\mathbbm{1}),
		\\
		y=-aS(2M(1,0,1,0,1,0,1,0,...)-\mathbbm{1}),
		\\
		z=-aS(2M(0,1,0,1,0,1,0,1,...)-\mathbbm{1}).
	\end{gather}

	Тогда, очевидно, каждая из последовательностей $x,y,z$ содержит отрезки сколь угодно большой длины,
	состоящие из $a$ (равно как и из $-a$), при этом $-a \leq x,y,z \leq a$.
	Следовательно, $p(x)=p(y)=p(z) = a$ и $q(x)=q(y)=q(z) = -a$,
	откуда $x,y,z \in Y^a_{-a}$.

	Заметим теперь, что
	\begin{multline}
		x + y + z
		=
		\\=
		aS(2M(\mathbbm{1})-\mathbbm{1}) - aS(2M(1,0,1,0,1,0,...)-\mathbbm{1}) - aS(2M(0,1,0,1,0,1,...)-\mathbbm{1})
		=
		\\=
		aS(2M(\mathbbm{1})-\mathbbm{1}  -    2M(1,0,1,0,1,0,...)+\mathbbm{1}  -    2M(0,1,0,1,0,1,...)+\mathbbm{1})
		=
		\\=
		aS(2M(\mathbbm{1}) - 2M(1,0,1,0,1,0,...) - 2M(0,1,0,1,0,1,...)+\mathbbm{1}+\mathbbm{1}-\mathbbm{1})
		=
		\\=
		aS(2M(\mathbbm{1}) - 2M(1,0,1,0,1,0,...) - 2M(0,1,0,1,0,1,...)+\mathbbm{1})
		=
		aS\mathbbm{1}
		=
		a\mathbbm{1}
		,
	\end{multline}
	откуда $\mathbbm{1} \in \operatorname{Lin} Y^a_{-a}$.
\end{proof}


\begin{lemma}
	\label{lem:c_0_Lin_alpha_0}
	Справедливо включение $c_0 \subset Y^a_b$.
\end{lemma}

\begin{proof}
	Зафиксируем $z\in c_0$.
	Выберем произвольный $x \in Y^a_b$.
	Тогда $x+z\in Y^a_b$ и, очевидно, $z=(x+z)-x$.
\end{proof}

\begin{theorem}
	\label{thm:A_0_c_infty_lin}
	Пусть $a\neq -b$.
	Тогда справедливо равенство $\operatorname{Lin} Y^a_b = A_0$.
\end{theorem}

\begin{proof}
	Зафиксируем $x \in A_0$.

	Пусть сначала $p(x) = q(x)$.
	Тогда, согласно теореме~\ref{thm:alpha_c_ac_c}, $x\in c$
	и $x$ может быть представлен в виде суммы константы и последовательности из $c_0$.
	Утверждение теоремы следует из лемм~\ref{lem:const_Lin_alpha_0}, ~\ref{lem:const_Lin_alpha_0_a_eq_-b} и~\ref{lem:c_0_Lin_alpha_0}.

	Пусть теперь $p(x) > q(x)$.
	Положим $y=k\cdot x + C\cdot\mathbbm{1}$,
	где $k=\frac{a-b}{p(x)-q(x)}$, $C=\frac{bp(x)-aq(x)}{p(x)-q(x)}$.
	Тогда, очевидно,
	\begin{equation}
		\label{eq:x_representation}
		x=(y-C\cdot\mathbbm{1})/k
		.
	\end{equation}
	Представление~\eqref{eq:x_representation} искомое.
	Действительно, в силу лемм~\ref{lem:const_Lin_alpha_0}~и~\ref{lem:const_Lin_alpha_0_a_eq_-b} выполнено
	$C\cdot\mathbbm{1}\in Y^a_b$; кроме того,
	\begin{equation}
		p(y) = k\cdot p(x) + C
		=
		%\\=
		\frac{ap(x)-bp(x)+bp(x)-aq(x)}{p(x)-q(x)}
		=
		a
		,
	\end{equation}
	\begin{equation}
		q(y) = k\cdot q(x) + C
		=
		%\\=
		\frac{aq(x)-bq(x)+bp(x)-aq(x)}{p(x)-q(x)}
		=
		b
		.
	\end{equation}


\end{proof}

Факт, аналогичный теоремам~\ref{thm:Lin_Omega_Sucheston} и~\ref{thm:A_0_c_infty_lin}, верен и для
всего пространство $\ell_\infty$:
$\ell_\infty\subset \operatorname{Lin} X^a_b$, где
$X^a_b = \{x\in\ell_\infty : p(x) = a,~ q(x) = b\}$, $a>b$.


\begin{lemma}
	\label{lem:const_Lin_ell_infty}
	Справедливо включение
	$\mathbbm{1}\in \operatorname{Lin} X^a_b$.
\end{lemma}

\begin{proof}
	В самом деле,
	$Y^a_b \subset X^a_b$
	и, следовательно,
	\begin{equation}
		\mathbbm{1} \in \operatorname{Lin} Y^a_b \subset \operatorname{Lin} X^a_b
		.
	\end{equation}
\end{proof}

\begin{theorem}
	\label{thm:Lin_ell_infty}
	Справедливо равенство $\operatorname{Lin} X^a_b = \ell_\infty$.
\end{theorem}

\begin{proof}
	Зафиксируем $x \in \ell_\infty$ и представим его в виде линейной комбинации последовательностей из $X^a_b$.

	Не теряя общности, положим $x\geq 0$
	(иначе представим сначала $x$ в виде $x = y - z$, где $y \geq 0$, $z \geq 0$.
	и найдём представления для $y$ и $z$).

	Если $p(x) = q(x)$, то возьмём некоторый $y\in\ell_\infty$,
	такой, что $p(y) > p(x) = q(x)  \geq q(y) \geq 0$.
	Тогда в силу выпуклости функционала $p$ имеем
	\begin{equation}
		p(x+y) \geq p(y) > p(x) = q(x)
		,
	\end{equation}


	\begin{equation}
		q(x+y) = -p(-x-y) \leq -p(-x) -p(-y) = q(x) + q(y) \leq q(x) < p(x+y)
		,
	\end{equation}
	и задача сведена к отысканию представлений для $y$ и $x+y$.
	Таким образом, случай $p(x) = q(x)$ можно исключить,
	и, не теряя общности, рассматривать только такие $x$, что $p(x) > q(x)$.

	Снова, как и в доказательстве теоремы~\ref{thm:A_0_c_infty_lin},
	положим $y=k\cdot x + C\cdot\mathbbm{1}$,
	где $k=\frac{a-b}{p(x)-q(x)}$, $C=\frac{bp(x)-aq(x)}{p(x)-q(x)}$.
	Дальнейшее доказательство переносится дословно.
\end{proof}


	\section{О хаусдорфовой размерности одного класса множеств}
	На множестве $\Omega$ стандартным образом определяется размерность Хаусдорфа (см. например \cite[Секция 6]{Edgar}).
Для непустого подмножества $F\subset \mathbb R^n$ и $s > 0$ определим $s$-мерную меру Хаусдорфа множества $F$ следующим образом:
$$\mathcal H^s(F) := \lim_{\delta\to0} \inf \left\{\sum_{i=1}^\infty \left({\rm diam} \ U_i \right)^s \ : \ F\subset \bigcup_{i=1}^\infty  U_i, \  0\leqslant {\rm diam} \ U_i \leqslant \delta \right\}.$$

Размерность Хаусдорфа множества  $F\subset \mathbb R^n$ определяется по формуле:
$${\rm dim}_H F := \inf\{ s > 0 \ : \ \mathcal H^s(F)=0\}.$$



Мы приведём определение самоподобных подмножеств множества $\Omega$ (см., например, \cite{falconer1997techniques}).

\begin{definition}
Множество $E\subset\Omega$ называется самоподобным, если существуют $m\in\mathbb{N}$,
$m\geqslant2$, $0< r_1, \dots, r_m<1$ и функции $f_j : \Omega \to \Omega$, $j=1,\dots, m$ такие, что
$$\rho(f_j(x), f_j(y)) = r_j \rho(x,y), \ \forall \ x,y \in \Omega, \ j=1,\dots, m$$
и $E=\bigcup_{j=1}^m f_j(E).$
\end{definition}



\begin{lemma}
	\label{lem:Hausdorf_measure}
	Пусть $E\subset\Omega$ и $TE = E$.
	Тогда размерность Хаусдорфа $E$ равна $1$.
\end{lemma}

\begin{proof}
	Для $j=1,2$ определим функции $f_j : \Omega \to \Omega$ следующим образом:
	$$f_1(x_1, x_2, \dots)=(0, x_1, x_2, \dots), \quad f_2(x_1, x_2, \dots)=(1, x_1, x_2, \dots).$$

	Очевидно, что $E=f_1(E)\cup f_2(E).$

	Теперь мы покажем, что размерность Хаусдорфа множества $E$ равна $1$.

	Т.к. для $j=1,2$ верно
	 $$\rho(f_j(x),f_j(y))=\frac12\rho(x,y), \ \forall \ x, y \in E,$$
	 то функции $f_j$ являются преобразованиями подобия с коэффициентами $r_j=1/2$ для $j=1,2$.


	По~\cite[Теорема 9.3]{Edgar} размерность Хаусдорфа $d$ множества $E$ является решением уравнения:
	$$ r_1^d+r_2^d=1.$$
	Т.к. $r_j=1/2$, то
	$d=1.$
\end{proof}

Лемма~\ref{lem:Hausdorf_measure} позволяет несколько сократить доказательство
[ТОDO: ссылка на ASU\_a\_a].
Действительно, $x\in\Omega\setminus c$ принадлежит $W$ тогда и только тогда, когда
\begin{equation}
	\label{eq:dim_ext_B_W}
	(\ext \mathfrak{B})x = \{0;1\}
\end{equation}
Очевидно, что соотношение~\eqref{eq:dim_ext_B_W} выполнено для $x$ тогда и только тогда, когда оно выполнено для $Tx$.
Следовательно, $W=TW$ и $\dim_H W = 1$.

Аналогично получаем $\dim_H (\Omega \cap ac) = \dim_H (\Omega \cap ac_0) = \dim_H (\Omega \cap c) = \dim_H (\Omega \cap c_0) = 1$.







	\section{О существовании разделяющих множеств малой хаусдорфовой размерности}
	Хорошо известны некоторые примеры разделяющих множеств
[TODO1: много ссылок, в т.ч. на новую статью в МЗ].
Применяя лемму~\ref{lem:Hausdorf_measure}, можно показать,
что хаусдорфову размерность 1 имеют множества
TODO1: список!

В данном пункте строится пример разделяющего множества,
имеющего малую хаусдорфову размерность.


\begin{theorem}
	\label{thm:Hausdorf_measure_1_n}
	Пусть $n\in\N$.
	Тогда существует разделяющее множество $E\subset\Omega$ такое,
	что $\dim_H E = 1/n$.
\end{theorem}

\begin{proof}
	Пусть
	\begin{equation}
		Е= \{ x \in \Omega : k \neq mn \Rightarrow x_k = 0\}, m\in \N
		,
	\end{equation}
	т.е. у последовательности $x \in E$ равны нулю все элементы, кроме, быть может, $x_n$, $x_{2n}$, $x_{3n}$ и т.д.

	Для $j=1,2$ определим функции $f_j : \Omega \to \Omega$ следующим образом:
	\begin{equation}
		f_1(x_1, x_2, \dots)=(\underbrace{0, ..., 0,}_{\mbox{$n-1$ раз}} 0, x_1, x_2, \dots)
		,
		\quad
		f_2(x_1, x_2, \dots)=(\underbrace{0, ..., 0,}_{\mbox{$n-1$ раз}} 1, x_1, x_2, \dots)
		.
	\end{equation}

	Очевидно, что $E=f_1(E)\cup f_2(E).$

	Теперь мы покажем, что размерность Хаусдорфа множества $E$ равна $2^{-n}$.

	Т.к. для $j=1,2$ верно
	 $$\rho(f_j(x),f_j(y))=2^{-n}\rho(x,y), \ \forall \ x, y \in E,$$
	 то функции $f_j$ являются преобразованиями подобия с коэффициентами $r_j=2^{-n}$ для $j=1,2$.


	По~\cite[Теорема 9.3]{Edgar} размерность Хаусдорфа $d$ множества $E$ является решением уравнения:
	$$ r_1^d+r_2^d=1.$$
	Т.к. $r_j=2^{-n}$, то
	$d=1/n.$
\end{proof}

\begin{remark}
	У читателя может возникнуть закономерный вопрос о свойствах пересечения
	\begin{equation}
		\bigcap_{n\in \N} E_{2^n}
		,
	\end{equation}
	где $E_n$ "--- множество, построенное по теореме~\ref{thm:Hausdorf_measure_1_n}, т.е.
	\begin{equation}
		Е_n = \{ x \in \Omega : k \neq mn \Rightarrow x_k = 0\}, m\in \N
		,
	\end{equation}
	в частности о том, является ли оно разделяющим множеством нулевой хаусдорфовой размерности.
	К сожалению, ответ на этот вопрос положителен только во второй части, а именно
	\begin{equation}
		\bigcap_{n\in \N} E_n = {(0,0,0,...)}
		.
	\end{equation}
	Таким образом, пересечение вложенной последовательности разделяющих множеств может не быть разделяющим множеством.
	Впрочем, существует и более простой пример:
	в качестве вложенной последовательности разделяющих множеств следует взять $\{M_n\} = [0, 2^{-n}]$.
	Очевидно, что $\bigcup\limits M_n = \{0\}$.
\end{remark}


\begin{remark}
	Пусть $n>1$ и множество $E$ построено по теореме~\ref{thm:Hausdorf_measure_1_n}.
	Тогда $p(E)< 1$ и, [TODO1: ref], мера $E$ равна нулю.
\end{remark}



\chapter{Функционалы Сачестона и мультипликативные свойства носителя последовательности}

	Дальнейшим ослаблением понятия сходимости является сходимость по Чезаро (сходимость в среднем).
Говорят, что последовательность $\{x_n\}\in\ell_\infty$ сходится по Чезаро к $t$, если
\begin{equation}
	\lim_{n\to\infty}\frac1{n}\sum_{i=1}^n x_i = t
	.
\end{equation}
Легко заметить, что обсуждаемые обобщения верхнего и нижнего пределов удовлетворяют соотношению
\begin{equation}
	\label{eq:generalization_of_limits}
	\liminf_{n\to\infty} x_n \leq q(x) \leq \liminf_{n\to\infty}\frac1{n}\sum_{i=1}^n x_i
	\leq
	%\\ \leq
	\limsup_{n\to\infty}\frac1{n}\sum_{i=1}^n x_i
	\leq p(x)
	\leq \limsup_{n\to\infty} x_n
	.
\end{equation}

Отдельный интерес представляет множество всех последовательностей из 0 и 1,
которое, как и выше, мы будем обозначать через $\Omega$.
%(иногда в литературе~\cite{semenov2020geomBL,Semenov2014geomprops} встречается также обозначение $\{0;1\}^\N$).
Понятно, что каждый $x\in \Omega$ можно отождествить с подмножеством множества натуральных чисел
$\supp x \subset \N$.

Вслед за~\cite{hall1992behrend} будем обозначать через $\mathscr{M}A$ множество всех чисел,
кратных элементам множества $A\subset\N$, т.е.
\begin{equation}
	\mathscr{M}A = \{ka: k\in\N, a\in A\}
	,
\end{equation}
через $\chi F$ "--- характеристическую функцию множества $F$.

Так, например,
\begin{gather}
	\chi \mathscr{M}\{2\} = \chi \mathscr{M}\{2, 4\} = \chi \mathscr{M}\{2,4,8,16,...\}
	= (0,1,0,1,0,1,0,1,0,...),
\\
	\chi \mathscr{M}\{3\} = \chi \mathscr{M}\{3,9,27,...\} = (0,0,1,\;0,0,1,\;0,0,1,\;0,0,1,\;0,0,1,\;...),
\\
	\chi \mathscr{M}\{2,3\} = \chi \mathscr{M}\{2,3,6\} = (0,1,1,1,0,1,\;0,1,1,1,0,1,\;0,1,1,1,0,1,...).
\end{gather}

Возникает закономерный вопрос о взаимосвязи структуры множества $A$
и значений, которые принимают обобщения верхнего и нижнего пределов~\eqref{eq:generalization_of_limits}
на последовательности $\chi \mathscr{M}\!A$.
Так, в работах~\cite{davenport1936sequences,davenport1951sequences} доказано, что для любого
$A=\{a_1,a_2,...\}\subset\N$
выполнено
\begin{equation}
	\liminf_{n\to\infty}\frac1{n}\sum_{i=1}^n (\chi\mathscr{M}A)_i =
	\lim_{j\to\infty}\lim_{n\to\infty}\frac1{n}\sum_{i=1}^n (\chi\mathscr{M}\{a_1,a_2,...,a_j\})_i
	.
\end{equation}
В работе~\cite[\S 7]{besicovitch1935density} построено такое множество $A\subset\N$, что
\begin{equation}
	\liminf_{n\to\infty}\frac1{n}\sum_{i=1}^n (\chi\mathscr{M}A)_i \neq
	\limsup_{n\to\infty}\frac1{n}\sum_{i=1}^n (\chi\mathscr{M}A)_i
	.
\end{equation}
За более подробной информацией о множествах типа $\mathscr{M}A$ отсылаем читателя к монографии~\cite{hall1996multiples}.

В этой главе изучается зависимость значений, которые могут принимать функционалы Сачестона
на последовательностях $\chi\mathscr{M}A$, от свойств множества $A$.


	\section{Конечное число множителей}
	Пользуясь критерием Лоренца~\eqref{eq:crit_Lorentz},%TODO: fix ref if broken
нетрудно доказать, что для $x_n = m^n$, $m\in\mathbb{N}$, $m\geq 2$ выполнено $\chi x \in ac_0$.

\begin{lemma}
	Пусть $y = \{y_n\}$ --- строго возрастающая последовательность,
	$\chi y\in\Omega \cap ac_0$.
	Пусть $m \in \mathbb{N}$
	и последовательность $x=\{x_k\}$ определена соотношением
	\begin{equation}
		x_k = \begin{cases}
			1, &\mbox{~если~} k = y_i \cdot m^j \mbox{~для некоторых~} i,j\in\mathbb{N},
			\\
			0  &\mbox{~иначе}
			.
		\end{cases}
	\end{equation}
	Тогда $x\in ac_0$.
\end{lemma}

\begin{proof}
	Зафиксируем некоторое $K \in \mathbb{N}$ и покажем, что $p(x) < m^{-K}$.
	Действительно, представим $x$ в виде суммы
	\begin{equation}
		\label{eq:ac0_powers_x_as_sum}
		x \leq z_1 + z_2 + \dots + z_K + z'_{K+1}
		,
	\end{equation}
	где каждое слагаемое $z_j$ соответствует умножению индексов на $m^j$:
	\begin{equation}
		(z_j)_k = \begin{cases}
			1, &\mbox{~если~} k = y_i \cdot m^j \mbox{~для некоторого~} i\in\mathbb{N},
			\\
			0  &\mbox{~иначе}
			,
		\end{cases}
	\end{equation}
	а слагаемое $z'_j$ соответствует умножению индексов на $m^{j+1}$, $m^{j+2},...$:
	\begin{equation}
		\label{eq:ac0_powers_residue}
		(z'_j)_k = \begin{cases}
			1, &\mbox{~если~} k = y_i \cdot m^{j'} \mbox{~для некоторых~} i,j'\in\mathbb{N},~~ j' > j
			,%\\ &\mbox{~и ни для какого~} j' \leq j \mbox{~не выполнено~} k = y_i \cdot m^{j'},
			\\
			0  &\mbox{~иначе}
			.
		\end{cases}
	\end{equation}
	(Знак неравенства в~\eqref{eq:ac0_powers_x_as_sum} возникает ввиду того, что возможен случай
	$y_i \cdot m^j = y_{i'} \cdot m^{j'}$ для $j\neq j'$.
	Например, в случае $y_1 = 3$, $y_2 = 6$ и $m=2$ имеем $y_1 \cdot m^2 = y_2 \cdot m^1$.)
	Понятно, что $p(z_j)=0$.
	Таким образом,
	\begin{equation}
		p(x) \leq p(z_1) + p(z_2) + \dots + p(z_K) + p(z'_{K+1}) = p(z'_{K+1})
		.
	\end{equation}
	Заметим, что в силу определения~\eqref{eq:ac0_powers_residue} каждый отрезок $z'_j$ из $m^{j+1}$ элементов
	содержит не более одной единицы, и потому $p(z'_j) \leq m^{-j-1} < m^{-j}$.
	Таким образом, для любого $K\in\mathbb{N}$ выполнена оценка $p(x) < m^{-K}$,
	откуда $p(x) = 0$.
\end{proof}

\begin{corollary}
	\label{cor:ac0_powers_finite_set_of_numbers}
	Пусть $\{p_1, ..., p_k\} \subset \mathbb{N}$,
	\begin{equation}
		x_k = \begin{cases}
			1, &\mbox{~если~} k = p_1^{j_1}\cdot p_2^{j_2}\cdot ... \cdot p_k^{j_k} \mbox{~для некоторых~} j_1,...,j_k\in\mathbb{N},
			\\
			0  &\mbox{~иначе}.
		\end{cases}
	\end{equation}
	Тогда $x\in ac_0$.
\end{corollary}


%\begin{hypothesis}
%	Пусть $y=\{y_n\}$ и $z=\{z_n\}$ --- строго возрастающие последовательности,
%	$\chi_y,\chi_z\in\{0;1\}^\mathbb{N} \cap ac_0$.
%	Тогда почти сходится к нулю последовательность $x=\{x_k\}$, определённая соотношением
%	\begin{equation}
%		x_k = \begin{cases}
%			1, &\mbox{~если~} k = y_i \cdot z_j \mbox{~для некоторых~} i,j\in\mathbb{N},
%			\\
%			0  &\mbox{~иначе}
%			.
%		\end{cases}
%	\end{equation}
%\end{hypothesis}


	\section{Бесконечное число множителей и \\  верхний функционал Сачестона}
	\begin{lemma}
	Для любого непустого $A\subset \N $ выполнено $\chi\mathscr{M}A \notin ac_0$.
\end{lemma}
\begin{proof}
	Пусть $a_1\in A$.
	Тогда из каждых идущих подряд $a_1$ элементов последовательности $\chi\mathscr{M}A$
	хотя бы один равен единице,
	следовательно,
	\begin{equation}
		q(\chi\mathscr{M}A) \geq \frac{1}{a_1} > 0
		.
	\end{equation}
\end{proof}

При дополнительных ограничениях верно и более сильное утверждение.

\begin{theorem}
	\label{lem:ac0_primes_infinity_mutually_prime_subset}
	Пусть $A'$ "--- бесконечное подмножество попарно взаимно простых чисел
	(т.е. для любых двух чисел $a_1, a_2 \in A'$ их наибольший общий делитель равен единице).
	Тогда для любого $A \supset A' $ выполнено $p(\chi\mathscr{M}A)=1$.
\end{theorem}
\begin{proof}
	Пусть $A' = \{ a_1, a_2, ..., a_j, ... \}$ и
	\begin{equation}
		\label{eq:ac0_primes_A_j_prod_des}
		A_j = \prod_{i=1}^j a_i
		.
	\end{equation}

	Для каждого $k$ найдём такие номера $n_k$, что
	\begin{equation}
		(\chi\mathscr{M}A)_{n_k+1} = (\chi\mathscr{M}A)_{n_k+2} = \dots = (\chi\mathscr{M}A)_{n_k+k} = 1
		.
	\end{equation}
	Тем самым мы докажем, что отрезок из любого наперёд заданного количества единиц подряд
	встречается в последовательности $\chi\mathscr{M}A$ и, следовательно, $p(\chi\mathscr{M}A) = 1$.

	Действительно,
	пусть $n_1 = a_1 - 1$.
	Рассмотрим множество  $F_1 = \{ n_1 + A_1, n_1 + 2A_1, n_1 + 3A_1, \dots, n_1 + a_2A_1 \}$
	и отметим два следующих факта.

	Во-первых, пусть $f \in F_1$,
	тогда
	\begin{equation}
		f \equiv n_1 \mod a_1
		.
	\end{equation}
	Во-вторых, числа $a_2$ и $A_1$ взаимно просты.
	Следовательно, все $a_2$ чисел из множества $F_1$ дают разные остатки при делении на $a_2$.

	В качестве $n_2$ возьмём такое $f\in F_1$, что
	\begin{equation}
		f \equiv a_2 - 2 \mod a_2
		.
	\end{equation}
	Заметим, что тогда
	\begin{equation}
		n_2 + 1 \equiv n_1 + 1 \equiv 0 \mod a_1
	\end{equation}
	и
	\begin{equation}
		n_2 + 2 \equiv 0 \mod a_2
		,
	\end{equation}
	следовательно,
	$(\chi\mathscr{M}A)_{n_2 + 1} = (\chi\mathscr{M}A)_{n_2 + 2} = 1$.

	Полученные рассуждения несложно продолжить по индукции.

	Рассмотрим множество  $F_{j} = \{ n_j + A_j, n_j + 2A_j, n_j + 3A_j, \dots, n_j + a_{j+1}A_j \}$
	и отметим два следующих факта.

	Во-первых, пусть $f \in F_{j}$,
	тогда
	\begin{equation}
		f \equiv n_j \mod A_j
	\end{equation}
	и, в силу~\eqref{eq:ac0_primes_A_j_prod_des},
	\begin{equation}
		\begin{array}{rl}
		f &\equiv n_j \mod a_1
		\\
		f &\equiv n_j \mod a_2
		\\
		&\dots
		\\
		f &\equiv n_j \mod a_j
		.
		\end{array}
	\end{equation}
	Во-вторых, числа $a_{j+1}$ и $A_j$ взаимно просты, поскольку $a_{j+1}$ взаимно просто с каждым из чисел $a_1,...,a_j$.
	Следовательно, все $a_{j+1}$ чисел из множества $F_j$ дают разные остатки при делении на $a_{j+1}$.

	В качестве $n_{j+1}$ возьмём такое $f\in F_j$, что
	\begin{equation}
		f \equiv a_{j+1} - (j+1) \mod a_{j+1}
		.
	\end{equation}
	Заметим, что тогда
	\begin{equation}
		\begin{array}{l}
			n_{j+1} + 1 \equiv n_j + 1 \equiv n_{j-1} + 1 \equiv \dots \equiv n_2 + 1 \equiv n_1 + 1 \equiv 0 \mod a_1
			\\
			n_{j+1} + 2 \equiv n_j + 2 \equiv n_{j-1} + 2 \equiv \dots \equiv n_2 + 2 \equiv 0 \mod a_2
			\\
			\dots
			\\
			n_{j+1} + j \equiv n_j + j  \equiv 0 \mod a_j
		\end{array}
	\end{equation}
	и
	\begin{equation}
		n_{j+1} + (j+1) \equiv 0 \mod a_{j+1}
		,
	\end{equation}
	следовательно,
	\begin{equation}
		(\chi\mathscr{M}A)_{n_{j+1} + j+1} = (\chi\mathscr{M}A)_{n_{j} + j} =
		\dots = (\chi\mathscr{M}A)_{n_{j} + 2} = (\chi\mathscr{M}A)_{n_{j} + 1} = 1
		.
	\end{equation}


\end{proof}

\begin{remark}
	Понятно, что в качестве примера бесконечного множества
	попарно взаимно простых чисел можно взять любое бесконечное множество простых чисел.
	Однако бывают бесконечные множества попарно взаимно простых чисел,
	не содержащие простых чисел вовсе, например множество
	\begin{equation}
		A = \{ 2\cdot 3,~5 \cdot 7,~11 \cdot 13,~17\cdot 19,~23\cdot29,~31\cdot 37,~...\},
	\end{equation}
	элементами которого являются произведения пар соседних простых чисел.
\end{remark}

\begin{definition}
	\label{def:P-property}
	Будем говорить, что множество $A\subset\N$ обладает $P$-свойством,
	если для любого $n\in\N$ найдётся набор попарно взаимно простых чисел
	\begin{equation}
		\{a_{n,1}, a_{n,2}, ..., a_{n,n}  \} \subset A
		.
	\end{equation}
\end{definition}

Из доказательства теоремы~\ref{lem:ac0_primes_infinity_mutually_prime_subset} понятно,
что для множества $A$ в условии теоремы достаточно потребовать $P$-свойства.
Интересно, что на самом деле $P$-свойство эквивалентно условиям,
наложенным на множество $A$ в теореме~\ref{lem:ac0_primes_infinity_mutually_prime_subset}.

\begin{lemma}
	\label{lem:ac0_primes_infinity_mutually_prime_subset_equiv_to_P_property}
	Пусть множество $A$ обладает $P$-свойством.
	Тогда существует бесконечное подмножество $A'\subset A$ попарно взаимно простых чисел.
\end{lemma}
\begin{proof}
	Зафиксируем $f_0\in A$, $f_0 \neq 1$ и представим $A$ в виде объединения трёх попарно непересекающихся множеств:
	\begin{equation}
		A = \{f_0\} \cup E \cup F
		,
	\end{equation}
	где
	\begin{alignat*}{4}
		& E &&= \{ a \in A \mid a \mbox{~не~}&&\mbox{взаимно просто с~} f_0 \mbox{~и~} a\neq f_0\}
		,
		\\
		& F &&= \{ a \in A \mid a            &&\mbox{взаимно просто с~} f_0 \}
		.
	\end{alignat*}

	Пусть разложение $f_0$ на простые множители имеет вид
	\begin{equation}
		f_0 = p_1^{j_1} \cdot p_2^{j_2} \cdot ... \cdot p_k^{j_k}
		.
	\end{equation}

	Тогда множество $E$ можно представить в виде объединения (возможно пересекающихся) множеств:
	\begin{equation}
		E = \bigcup_{i=1}^{k} E_i,\quad E_i = \{a\in E \mid a \mbox{~кратно~} p_i\}
		.
	\end{equation}

	Покажем, что множество $F$ обладает $P$-свойством.
	Действительно, зафиксируем $n\in\N$.
	Так как множество $A$ обладает $P$-свойством,
	то в нём найдётся подмножество попарно взаимно простых чисел
	$$G=\{a_1, a_2, ..., a_{n+k-1}, a_{n+k}\}\subset A.$$

	Если $f_0\in G$, то $G\setminus f_0 \subset F$ в силу построения множеств $G$ и $F$, и требуемый набор попарно взаимно простых чисел предъявлен.

	Пусть теперь $f_0\notin G$.
	Заметим, что в каждое из множеств $E_i$ может входить не более одного элемента множества $G$
	в силу того, что при фиксированном $i$ все элементы множества $E_i$ имеют нетривиальный общий делитель.
	Следовательно, как минимум $n$ элементов из $G$ принадлежат множеству $F$,
	и требуемый набор попарно взаимно простых чисел снова предъявлен.

	%Поскольку множество $F$ обладает $P$-свойством, то оно, очевидно, счётно.

	Итак, нам удалось получить число $f_0\in A$ и бесконечное множество $F$, обладающее $P$-свойством
	и состоящее из чисел, взаимно простых с $f_0$.
	Продолжая по индукции, получим требуемое бесконечное множество попарно взаимно простых чисел.
\end{proof}

%Условие теоремы~\ref{lem:ac0_primes_infinity_mutually_prime_subset}
%является не только достаточным, но и необходимым.

\begin{lemma}
	\label{lem:ac0_primes_q_psi_A_0_causes_P}
	Пусть для множества $A\subset\N\setminus\{1\}$ выполнено $p(\chi\mathscr{M}A)=1$.
	Тогда $A$ обладает $P$-свойством.
\end{lemma}

\begin{proof}
	Предположим противное: пусть $A$ не обладает $P$-свойством.
	Тогда существует $n\in\N$ такое, что из множества $A$ можно выбрать $n$
	попарно взаимно простых чисел, но нельзя выбрать $n+1$.

	Пусть $\{a_1, a_2, ..., a_n\}\subset A$ "--- набор попарно взаимно простых чисел.
	Так как $p(\chi\mathscr{M}A)=1$, то в последовательности $\chi\mathscr{M}A$ найдётся отрезок, состоящий сплошь из единиц,
	любой наперёд заданной длины.
	Выберем $k$ таким, что
	\begin{equation}
		(\chi\mathscr{M}A)_{k+1} = (\chi\mathscr{M}A)_{k+2} = ... = (\chi\mathscr{M}A)_{k+a_1a_2\cdots a_n} = 1
		.
	\end{equation}
	Тогда существует такое число $k_0$, $k+1 \leq k_0 \leq k+a_1a_2\cdots a_n$,
	что $k_0$ даёт в остатке $1$ при делении на $a_1a_2\cdots a_n$.
	Так как $(\chi\mathscr{M}A)_{k_0} = 1$, то $k_0 = m\cdot a_0$ для некоторых $m\in\N$ и $a_0\in A$.
	С другой стороны, $k_0$ взаимно просто с каждым из чисел $a_1, a_2, ..., a_n$.
	Следовательно, $a_0$ также взаимно просто с каждым из чисел $a_1, a_2, ..., a_n$,
	и $\{a_0, a_1, a_2, ..., a_n\}\subset A$ "--- набор из $n+1$ попарно взаимно простых чисел.
	Полученное противоречие завершает доказательство.
\end{proof}

Таким образом,
из теоремы~\ref{lem:ac0_primes_infinity_mutually_prime_subset}
и лемм~\ref{lem:ac0_primes_infinity_mutually_prime_subset_equiv_to_P_property},~\ref{lem:ac0_primes_q_psi_A_0_causes_P}
незамедлительно следует
\begin{theorem}
	\label{thm:p_x_infinite_multiples}
	Пусть $A\subset \N\setminus\{1\}$.
	Тогда следующие условия эквивалентны:
	\begin{enumerate}[label=(\roman*)]
		\item
			$A$ обладает $P$-свойством
		\item
			В $A$ существует бесконечное подмножество попарно взаимно простых чисел
		\item
			$p(\chi\mathscr{M}A)=1$.
	\end{enumerate}
\end{theorem}


	\section{Бесконечное количество множителей и \\ нижний функционал Сачестона}
	Перейдём теперь к изучению нижнего функционала Сачестона $q(\chi\mathscr{M}A)$.

\begin{theorem}
	\label{thm:ac0_primes_p_psi_A_prod}
	Пусть $A = \{a_1, a_2, ..., a_n,...\}$ "--- бесконечное множество попарно взаимно простых чисел.
	Тогда
	\begin{equation}
		q(\chi\mathscr{M}A) \geq 1-\prod_{j=1}^\infty \left(1-\frac{1}{a_j}\right)
		.
	\end{equation}
\end{theorem}

\begin{proof}
	%Зафиксируем $k\in\N$.

	Заметим, что нижний функционал Сачестона можно представить в виде
	\begin{equation}
		\label{eq:ac0_primes_px_lim_pnx}
		q(x) = \lim_{n\to\infty} q_n(x)
		,
	\end{equation}
	где
	\begin{equation}
		q_n(x) = \inf_{m\in\N}  \frac{1}{n} \sum_{j=m}^{m+n-1} x_j
		.
	\end{equation}

	Поскольку предел~\eqref{eq:ac0_primes_px_lim_pnx} существует, то для его оценки можно использовать предел подпоследовательности
	$q_{n_k}(x)$, где $n_k = a_1\cdot a_2 \cdot ... \cdot a_k$.

	Заметим теперь, что в любом отрезке последовательности $\chi\mathscr{M}A$ длины $n_k$
	содержится не более $\prod_{j=1}^k (a_j-1)$ нулей
	(могут попадаться <<дополнительные>> единицы "--- элементы с индексами, кратными $a_j$ для $j>k$).
	Значит,
	\begin{equation}
		q_{n_k}(\chi\mathscr{M}A) =
		\sup_{m\in\N}  \frac{1}{n_k} \sum_{j=m}^{m+n_k-1} (\chi\mathscr{M}A)_j
		\geq
		1-\prod_{i=1}^k \frac{a_i-1}{a_i}
		.
	\end{equation}
	Снова перейдя к пределу по $k$, получим
	\begin{equation}
		\label{eq:ac0_primes_p_psi_A_upper_bound}
		q(\chi\mathscr{M}A) \geq \lim_{k\to \infty} \prod_{i=1}^k \frac{a_i-1}{a_i}
		=
		1-\prod_{i=1}^\infty \frac{a_i-1}{a_i}
		.
	\end{equation}

\end{proof}




\begin{corollary}
	\label{cor:ac0_primes_p_psi_A_prod}
	Пусть $A = \{a_1, a_2, ..., a_n,...\}$ "--- бесконечное множество попарно взаимно простых чисел
	и $a_{n+1}>a_1\cdot...\cdot a_n$.
	Тогда
	\begin{equation}
		q(\chi\mathscr{M}A) = 1-\prod_{j=1}^\infty \left(1-\frac{1}{a_j}\right)
		.
	\end{equation}
\end{corollary}

\begin{proof}
	%Зафиксируем $k\in\N$.

	Заметим, что нижний функционал Сачестона можно представить в виде
	\begin{equation}
		\label{eq:ac0_primes_px_lim_pnx}
		q(x) = \lim_{n\to\infty} q_n(x)
		,
	\end{equation}
	где
	\begin{equation}
		q_n(x) = \inf_{m\in\N}  \frac{1}{n} \sum_{j=m}^{m+n-1} x_j
		.
	\end{equation}

	Поскольку предел~\eqref{eq:ac0_primes_px_lim_pnx} существует, то для его оценки можно использовать предел подпоследовательности
	$q_{n_k}(x)$, где $n_k = a_1\cdot a_2 \cdot ... \cdot a_k$.


	Cреди первых $n_k$ элементов последовательности $\chi\mathscr{M}A$
	ровно $\prod_{j=1}^k (a_j-1)$ нулей, поскольку комбинации остатков от деления
	на взаимно простые числа $a_1, a_2, ..., a_k$ <<не успевают>> повторяться.
	Следовательно,
	\begin{equation}
		q_{n_k}(\chi\mathscr{M}A) =
		\inf_{m\in\N}  \frac{1}{n_k} \sum_{j=m}^{m+n_k-1} (\chi\mathscr{M}A)_j
		\leq
		\frac{1}{n_k} \sum_{j=1}^{n_k} (\chi\mathscr{M}A)_j
		=
		1-\prod_{i=1}^k \frac{a_i-1}{a_i}
		.
	\end{equation}
	Переходя к пределу по $k$, имеем
	\begin{equation}
		\label{eq:ac0_primes_p_psi_A_lower_bound}
		q(\chi\mathscr{M}A) \leq 1-\lim_{k\to \infty} \prod_{i=1}^k \frac{a_i-1}{a_i}
		=
		1-\prod_{i=1}^\infty \frac{a_i-1}{a_i}
		.
	\end{equation}

	Сопоставив~\eqref{eq:ac0_primes_p_psi_A_lower_bound} и~\eqref{eq:ac0_primes_p_psi_A_upper_bound}, получим утверждение следствия.
\end{proof}

\begin{corollary}
	В случае, когда $\N\setminus A$ конечно,
	из следствия~\ref{cor:ac0_powers_finite_set_of_numbers} непосредственно вытекает, что $\chi\mathscr{M}A\in ac_1$
	и, соответственно, $q(\chi\mathscr{M}A)=1$.
\end{corollary}



Классическая теорема Эйлера~\cite{euler1737variae} говорит о том, что
ряд обратных простых чисел
\begin{equation}
	\frac{1}{2} + \frac{1}{3} + \frac{1}{5} + \frac{1}{7} + ...
	=
	\sum_j \frac{1}{j},
\end{equation}
где $j$ пробегает все простые числа, расходится.

С учётом этого факта
из теоремы~\ref{thm:ac0_primes_p_psi_A_prod} вытекает
\begin{lemma}
	Пусть $\varepsilon \in  (0; 1{]}$.
	Существует бесконечное множество попарно непересекающихся подмножеств простых чисел
	$A_i$ такое, что $q(\chi\mathscr{M}A)\geq\varepsilon$ для любого $i\in\N$.
\end{lemma}



\begin{theorem}
	Пусть $A=\{a_1, a_2, ..., a_n, ...\}\subset\N$ есть бесконечное множество попарно взаимно простых чисел
	и
	\begin{equation}
		\label{eq:prod_causes_q}
		\prod_{j=1}^\infty \left(1-\frac{1}{a_j}\right) = 0
	\end{equation}
	Тогда $\chi\mathscr{M}A\in ac$ и, более того, $\chi\mathscr{M}A\in ac_1$.
\end{theorem}

\begin{proof}
	По теореме~\ref{thm:ac0_primes_p_psi_A_prod} условие~\eqref{eq:prod_causes_q} влечёт равенство $q(\chi\mathscr{M}A)=1$.
	По теореме~\ref{lem:ac0_primes_infinity_mutually_prime_subset} $p(\chi\mathscr{M}A)=1$,
	откуда и следует требуемое.
\end{proof}





\chapter*{Заключение}
\addcontentsline{toc}{chapter}{Заключение}
В работе исследованы такие асимптотические характеристики ограниченных последовательностей,
как $\alpha$--функция (глава 1) и  почти сходимость (глава 2).% и банаховы пределы (глава 3).

Несмотря на то, что $\alpha$--функция оказалась трансляционно неинвариантной,
эта неинвариантность в некотором смысле однородна (см. следствие \ref{thm:est_alpha_Tn_x_full}).
Для элементов пространства почти сходящихся последовательностей $ac $
установлена двусторонняя оценка на расстояние до пространства сходящихся последовательностей $c$,
использующая $\alpha$--функцию.
Примечательно (хотя и ожидаемо), что наряду с трансляционно неинвариантной $\alpha$--функцией
в этой оценке используется и функционал $\lim_{n\to\infty}\alpha(T^n x)$,
который, очевидно, трансляционно инвариантен.
Это вполне логично, поскольку расстояние до пространства $c$ и почти сходимость
сами суть трансляционно инвариантные характеристики.




По итогам исследований, результаты которых составили данную работу,
опубликованы тезисы \cite{our-vvmsh-2018,our-vzms-2018,our-ped-2018-inf-dim-ker,our-ped-2018-alpha-Tx},
а также краткое сообщение~\cite{our-mz2019ac0};
планируется публикация ещё нескольких печатных работ.

Выдвинут ряд гипотез,
работу над доказательством или опровержением которых планируется продолжать в дальнейшем.



\chapter*{Список сформулированных гипотез}
\addcontentsline{toc}{chapter}{Список сформулированных гипотез}

\renewcommand\label[1]{}
\hypotlist


\addcontentsline{toc}{chapter}{Список литературы}

\makeatletter
\ltx@iffilelater{biblatex-gost.def}{2017/02/01}%
{\toggletrue{bbx:gostbibliography}%
\renewcommand*{\revsdnamepunct}{\addcomma}}{}
\makeatother

\printbibliography{}

\end{document}
