\documentclass[12pt,a4paper,openbib]{report}
\usepackage{amsmath}
\usepackage[utf8]{inputenc}
\usepackage[english,russian]{babel}
\usepackage{amsfonts,amssymb}
\usepackage{latexsym}
\usepackage{euscript}
\usepackage{enumerate}
\usepackage{graphics}
\usepackage[dvips]{graphicx}
\usepackage{geometry}
\usepackage{wrapfig}
\usepackage[colorlinks=true,allcolors=black]{hyperref}
\usepackage{bbm}
\usepackage{enumitem}
\usepackage{mathrsfs}


\usepackage{xr}
\externaldocument{phd_Avdeev}


% https://tex.stackexchange.com/questions/634509/show-hide-thumbnail-sidebar-by-default-in-pdf
\hypersetup{pdfpagemode=UseNone}


\righthyphenmin=2

%\usepackage[14pt]{extsizes}

\geometry{left=2.5cm}% левое поле
\geometry{right=1cm}% правое поле
\geometry{top=2cm}% верхнее поле
\geometry{bottom=2cm}% нижнее поле

\renewcommand{\baselinestretch}{1.3}

\renewcommand{\le}{\leqslant}
\renewcommand{\ge}{\geqslant}
\renewcommand{\leq}{\leqslant}
\renewcommand{\geq}{\geqslant} % И делись оно всё нулём!

\renewcommand{\varlimsup}{\limsup}
\renewcommand{\varliminf}{\liminf} % ... по самую асимптоту!



\DeclareMathOperator{\ext}{ext}
\DeclareMathOperator{\mes}{mes}
\DeclareMathOperator{\supp}{supp}
\DeclareMathOperator{\conv}{conv}
\DeclareMathOperator{\diam}{diam}

\newcommand{\N}{\ensuremath{\mathbb{N}}}
\newcommand{\Q}{\ensuremath{\mathbb{Q}}}
\newcommand{\R}{\ensuremath{\mathbb{R}}}
\newcommand{\B}{\ensuremath{\mathfrak{B}}}
\newcommand{\Iac}{\mathcal{I}(ac_0)}
\newcommand{\Dac}{\mathcal{D}(ac_0)}
\newcommand{\one}{\ensuremath{\mathbbm 1}}


\newcommand{\longcomment}[1]{}

\usepackage[backend=biber,style=gost-numeric,sorting=none]{biblatex}
%\usepackage[backend=biber,style=gost-footnote,sorting=none,bibstyle=gost-numeric]{biblatex}


\DeclareSourcemap{
  \maps[datatype=bibtex]{
    \map{
      \step[fieldsource=author,
            match=\detokenize{Avdeev},
            final]
      \step[fieldset=keywords, fieldvalue=avdeev]
    }
    \map{
      \step[fieldsource=author,
            match=\detokenize{Авдеев},
            final]
      \step[fieldset=keywords, fieldvalue=avdeev]
    }
  }
}




\addbibresource{../bib/Semenov.bib}
\addbibresource{../bib/my.bib}
\addbibresource{../bib/ext.bib}
\addbibresource{../bib/classic.bib}
\addbibresource{../bib/Damerau-Levenstein.bib}
\addbibresource{../bib/general_monographies.bib}
\addbibresource{../bib/Bibliography_from_Usachev.bib}

\input{../bib/ext.hyphens.bib}


\def\selfcite{\parencite}
%\renewcommand\cite[2][]{\footfullcite{#2}}
\renewcommand\cite[2][]{\splitfootfullcite{#2}}
%\renewcommand\cite[2][]{\splitfootfullcite{\detokenize{#2}}}


\DeclareBibliographyCategory{vakpapers}
\def\vakpaperskeys{
%Солидные статьи
our-mz2019ac0,
our-mz2019measure,
our-mz2021linearhulls,
avdeed2021AandA,
avdeev2021vestnik,
avdeev2021vmzprimes,
avdeev2024decomposition,
avdeev2024set_DAN_rus}
\addtocategory{vakpapers}{\vakpaperskeys}


%Splitting the footnoterefs
\usepackage{etoolbox}

\newcounter{citefootnote}
\makeatletter
\newcommand\splitfootfullcite[1]{%
    \begingroup
    \setcounter{citefootnote}{0} % reset footnote counter
    \forcsvlist{\makeenumerate@item}{#1}%
    \endgroup
}
\newcommand\makeenumerate@item[1]{%
    \ifnum\c@citefootnote>0{\textsuperscript{,}}\fi % Adjust this line to change the separator
    \footfullcite{#1}%
    \addtocounter{citefootnote}{1}%
}
\makeatother





\usepackage{amsthm}
\theoremstyle{definition}
\newtheorem{lemma}{Лемма}[section]
\newtheorem{theorem}[lemma]{Теорема}
\newtheorem{example}[lemma]{Пример}
\newtheorem{property}[lemma]{Свойство}
\newtheorem{remark}[lemma]{Замечание}
\newtheorem{definition}[lemma]{Определение}
\newtheorem{proposition}[lemma]{Утверждение}
\newtheorem{corollary}[lemma]{Следствие}




%For the introduction
\newcommand\reflecttheorem[1]{\paragraph{Теорема \ref{#1}}}
\newcommand\reflectlemma[1]{\paragraph{Лемма \ref{#1}}}
\newcommand\reflectcorollary[1]{\paragraph{Следствие \ref{#1}}}
\newcommand\reflectdefinition[1]{\paragraph{Определение \ref{#1}}}

%Only referenced equations are numbered
\usepackage{mathtools}
\mathtoolsset{showonlyrefs}

%\mathtoolsset{showonlyrefs=false}
% (an equation/multline to be force-numbered)
%\mathtoolsset{showonlyrefs=true}

% https://superuser.com/questions/517025/how-can-i-append-two-pdfs-that-have-links
\usepackage{pdfpages}% http://ctan.org/pkg/pdfpages



% Центрируем заголовки, как того требует ГОСТ
\makeatletter
\patchcmd {\@makechapterhead}{\Huge \bfseries}{\Large \bfseries \centering }{}{\ddy}
\patchcmd {\@makechapterhead}{\huge \bfseries}{\large \bfseries \centering }{}{\ddy}
\patchcmd{\@makeschapterhead}{\Huge \bfseries}{\Large \bfseries \centering }{}{\ddy}

\patchcmd{\section}         {\Large \bfseries}{\large \bfseries \centering }{}{\ddy}
\patchcmd{\subsection}      {\large \bfseries}{       \bfseries \centering }{}{\ddy}
\makeatother



\begin{document}

\clubpenalty=10000
\widowpenalty=10000
\includepdf[scale=1,pages={1-}]{autoref_title.pdf}
\setcounter{page}{3}


\chapter*{Введение}
\addcontentsline{toc}{chapter}{Введение}

\section*{Общая характеристика работы}
\addcontentsline{toc}{section}{Общая характеристика работы}
\paragraph{Актуальность темы исследования и степень её разработанности.}

Один из основателей функционального анализа как области современной математики С. Банах
в 1932 году ввёл в рассмотрение множество непрерывных линейных функционалов на пространстве ограниченных последовательностей,
которые совпадают с обычным пределом на всех сходящихся последовательностях.
Эти функционалы в дальнейшем и были названы банаховыми пределами;
их изучением занимались Г.Г. Лоренц, Л. Сачестон, Г. Дас, У.Ф. Эберлейн и другие математики.

В 1948 году Г.Г. Лоренц, используя банаховы пределы, ввёл понятие почти сходящейся последовательности "---
последовательности, на которой значение банахова предела не зависит от выбора этого банахова предела.
В 1967 году Л. Сачестон построил аналоги верхнего и нижнего пределов для банаховых пределов "---
(нелинейные) функционалы Сачестона.
Изучению и обобщению понятия почти сходимости посвящены работы
Р.А. Раими, М. Мурсалена, Г. Беннета, Н.Дж. Калтона, Д. Хаджуковича, Е.А. Алехно, Д. Занина и др.
Отдельный интерес представляет вопрос о банаховых пределах, инвариантных относительно некоторых линейных операторов.

Банаховы пределы тесно связаны с другими областями математики.
Так, в исследованиях С. Лорда, Дж. Филлипса, А.Л. Кери, П.Г. Доддса, Е.М. Семёнова, Б. де Пагтера,
А.А.~Седаева, А.С. Усачёва, Ф.А. Сукочева банаховы пределы применяются к изучению следов Диксмье;
следы же Диксмье, в свою очередь, широко применяются в некоммутативной геометрии А. Конна.
Связи банаховых пределов с эргодической теорией посвящены работы Л. Сачестона и др.

В настоящей диссертации исследуются пространство почти сходящихся последовательностей
(и его подпространство последовательностей, почти сходящихся к нулю),
функционалы Сачестона и инвариантные банаховы пределы.

\paragraph{Цели и задачи работы.}
Целью работы является изучение банаховых пределов и их свойств инвариантности.

Задачи работы:
\begin{itemize}
	\item
		исследование структуры множества банаховых пределов;
	\item
		исследование подпространств пространства ограниченных последовательностей,
		определяемых с помощью банаховых пределов;
	\item
		исследование свойств композиции линейных операторов и банаховых пределов.
\end{itemize}


\paragraph{Научная новизна.}
Все основные результаты диссертации являются новыми.


\paragraph{Теоретическая и практическая значимость работы.}
Работа носит теоретический характер.
Результаты диссертации могут быть использованы в учебном процессе, спецкурсах и научных исследованиях,
проводимых в Воронежском, Ростовском, Самарском, Ярославском государственных университетах и др.
В работе сформулированы гипотезы,
которые могут быть использованы в составе заданий для выполнения
студентами и выпускниками бакалавриата, специалитета или магистратуры
курсовых или выпускных квалификационных работ.





\paragraph{Методология и методы исследования.}
Для исследования банаховых пределов и связанных с ними математических объектов применяются
понятия, методы и подходы современного функционального анализа,
а также отдельные элементы и факты топологии и теории чисел.

\paragraph{Положения, выносимые на защиту.}
На защиту выносятся следующие основные результаты:
\begin{enumerate}
	\item
		сформулированы и доказаны специальные критерии почти сходимости
		ограниченной последовательности;
	\item
		исследована новая асимптотическая характеристика ограниченной последовательности,
		позволяющая выявлять дополнительные (по сравнению с банаховыми пределами)
		свойства таких последовательностей;
	\item
		построена иерархическая система классов ограниченных линейных операторов
		на пространстве $\ell_\infty$ в зависимости от свойств их суперпозиции с банаховыми пределами;
	\item
		построен пример множества последовательностей из нулей и единиц, разделяющего банаховы пределы
		и имеющего нулевую меру, индуцированную мерой Лебега на отрезке $[0;1]$;
	\item
		исследована связь мультипликативной структуры носителя последовательности из нулей и единиц
		и значений, которые на такой последовательности могут принимать верхний и нижний функционалы Сачестона.
\end{enumerate}



\paragraph{Степень достоверности и апробация результатов.}

Все включенные в диссертацию результаты доказаны
в соответствии с современными стандартами достоверности математических доказательств.

Результаты диссертации докладывались и обсуждались:
\begin{itemize}
	\item
		на Международной конференции <<Воронежская Зимняя Математическая школа С.Г.~Крейна>> в 2018, 2022, 2025~гг.;
	\item
		на Международной конференции «Понтрягинские чтения — XXIX», посвященной 90-летию Владимира Александровича Ильина (Воронеж, 2018~г.);
	\item
		на Международной молодёжной научной школе «Актуальные направления математического анализа и смежные вопросы» (Воронеж, 2018~г.);
	\item
		на конкурсе научно-исследовательских работ студентов и аспирантов российских вузов
		<<Наука будущего --- наука молодых>> в секции <<Информационные технологии и математика>>
		(присуждено II место среди аспирантов) в ноябре 2021~г.;
	\item
		на Международной (53-й Всероссийской) молодёжной школе-конференции
		<<Современные проблемы математики и её приложений>>
		(Екатеринбург, 2022~г.);
	\item
		на научной сессии ВГУ в 2020, 2021, 2022, 2024, 2025 гг.; %(TODO: проверить недостающие годы - письмо отправлено Г.И.; в 2023 выступления нет - академ)
	\item
		16.10.2024 г. на семинаре в МИАН под рук. чл.-корр. РАН О.В. Бесова;
	\item
		15.10.2024 г. на семинаре в МГУ под рук. проф. РАН П.А. Бородина;
	\item
		13.11.2024 г. на семинаре под рук. проф. С.В. Асташкина (Самара);
	\item
		19.11.2024 г. на семинаре под рук. проф. А.Л. Скубачевского (Москва);
	\item
		29.01.2025 г. на семинаре под рук. проф. А.Г. Кусраева и М.А. Плиева (Владикавказ);
	\item
		на международной (56-й Всероссийской) молодёжной школе-конференции
		<<Современные проблемы математики и её приложений>>
		(Екатеринбург, февраль 2025~г.);
	\item
		на Всероссийской молодёжной научной конференции «Путь в науку. Математика»
		(Ярославль, май 2025~г.; доклад признан одним из лучших и награждён дипломом).
\end{itemize}

Научно-исследовательская работа соискателя по теме диссертации была поддержана грантами:
\begin{itemize}
	\item
		Российского научного фонда (грант №~19-11-00197);
	\item
		Российского научного фонда (грант №~24-21-00220);
	\item
		Фонда развития теоретической физики и математики ``БАЗИС'' (проект №~22-7-2-27-3).
\end{itemize}

\paragraph{Публикации.}
Основные результаты диссертации опубликованы в работах~\selfcite{
%Солидные статьи
our-mz2019ac0,
our-mz2019measure,
our-mz2021linearhulls,
avdeed2021AandA,
avdeev2021vestnik,
avdeev2021vmzprimes,
avdeev2024decomposition,
avdeev2024set_DAN_rus,
%Тезисы
our-vvmsh-2018,
our-vzms-2018,
our-ped-2018-inf-dim-ker,
our-ped-2018-alpha-Tx,
vzms2022setsofmultiples,
avdeev2022measure,
vzms2025linear,
%TODO: если будут ещё работы
}.
Из совместных работ~\selfcite{our-mz2019measure,avdeed2021AandA,avdeev2024decomposition,avdeev2024set_DAN_rus,our-vzms-2018}
в диссертацию вошли только результаты, принадлежащие лично диссертанту.

\paragraph{Структура и объём диссертации.}
Диссертация состоит из введения, пяти глав, разбитых на TODO параграфов,
и списка литературы, включающего TODO источника.
Общий объём диссертации TODO страницы.
%TODO: а подпараграфы?


\section*{Используемые обозначения}
\addcontentsline{toc}{section}{Используемые обозначения}
Приведём список основных математических объектов и понятий,
а также обозначений, которые в тексте диссертации могут быть использованы без дополнительного пояснения.


\paragraph{Числовые множества.}
Через $\N$ будем обозначать множество натуральных чисел: $\N = \{1;2;3;4;...\}$.
Через $\N_k$ ~--- множество целых чисел, не меньших $k$ (чтобы избежать громоздкой записи вида: пусть $n\in \N$, $n \geq 2$, ...).
Так,
\begin{equation}
	\N_1 = \N
	,
	\quad
	\N_0 = \N \cup\{0\} = \{0;1;2;3;4;...\}
	,
	\quad
	\N_3 = \{3;4;5;6;7;...\}
	.
\end{equation}

Через $\Q$ и $\R$ будем обозначать множества рациональных и вещественных чисел соответственно;
через $\Q^+$, $\Q^-$, $\R^+$ и $\R^-$ ~--- множества положительных рациональных, отрицательных рациональных,
положительных вещественных и отрицательных вещественных чисел соответственно.

\paragraph{Пространства последовательностей и их подмножества.}

Основным используемым пространством будет пространство $\ell_\infty$ ограниченных последовательностей со стандартной нормой
\begin{equation}
	\|x\| = \sup_{n\in\N} |x_n|
	.
\end{equation}
Эта же норма будет использоваться и в остальных пространствах и множествах, среди которых:
\begin{itemize}
	\item
		$c$ ~--- пространство сходящихся последовательностей;
	\item
		$c_\lambda$ ~--- множество последовательностей, сходящихся к $\lambda \in \R$, в частности,
	\item
		$c_0$ ~--- пространство последовательностей, сходящихся к нулю;
	\item
		$c_{00}$ ~--- пространство последовательностей с конечным носителем~\cite[теорема 4]{ASSU2};
	\item
		$ac$ ~--- пространство почти сходящихся последовательностей;
	\item
		$ac_\lambda$ ~--- множество последовательностей, почти сходящихся к $\lambda \in \R$, в частности,
	\item
		$ac_0$ ~--- пространство последовательностей, почти сходящихся к нулю;
	\item
		$\Iac$ ~--- максимальный (по включению) идеал по умножению в пространстве $ac_0$ (см. теорему~\ref{thm:Iac_criterion_pos_neg});
	\item
		$A_0 = \{ x\in\ell_\infty : \alpha(x) = 0\}$ (см. ниже; о свойствах этого пространства см., напр., теорему~\ref{thm:A0_is_space} и далее);
	\item
		$\Omega = \{0;1\}^\N$ ~--- множество последовательностей, состоящих из нулей и единиц.
\end{itemize}



\paragraph{Нормы.}
Запись $\|\cdot\|$ без уточнений будет обозначать норму в пространстве $\ell_\infty$, $\ell_\infty^*$
или в пространстве ограниченных линейных операторов, действующих из $\ell_\infty$ в $\ell_\infty$ (обозначаемом $\mathcal L (\ell_\infty, \ell_\infty)$~)
в зависимости от природы аргумента.
В случае, если нужно ввести другую норму (например, фактор-норму $\ell_\infty / c_0$, как в теореме~\ref{thm:alpha_xy}),
это будет оговорено явно.


\paragraph{Последовательности.}
Для любого $x\in\ell_\infty$ будем по умолчанию полагать, что
\begin{equation}
	x=(x_1, x_2, x_3, x_4, ...)
	.
\end{equation}
Кроме того, нам будет часто требоваться константная единица $\one = (1, 1, 1, 1, ...)$.

Будем писать, что $x\geq 0$, если $x_n \geq 0$ для любого $n\in \N$, и $x\leq 0$, если $x_n \leq 0$ для любого $n\in \N$.
%TODO: периодические последовательности и конкатенация

Вслед за~\cite{hall1992behrend} будем обозначать через $\mathscr{M}A$ множество всех чисел,
кратных элементам множества $A\subset\N$, т.е.
\begin{equation}
	\mathscr{M}A = \{ka: k\in\N, a\in A\}
	,
\end{equation}
через $\chi F$ "--- характеристическую функцию множества $F$.

Так, например,
\begin{gather}
	\chi \mathscr{M}\!A(\{2\}) = \chi \mathscr{M}\!A(\{2, 4\}) = \chi \mathscr{M}\!A(\{2,4,8,16,...\})
	= (0,1,0,1,0,1,0,1,0,...),
\\
	\chi \mathscr{M}\!A(\{3\}) = \chi \mathscr{M}\!A(\{3,9,27,...\}) = (0,0,1,\;0,0,1,\;0,0,1,\;0,0,1,\;0,0,1,\;...),
\\
	\chi \mathscr{M}\!A(\{2,3\}) = \chi \mathscr{M}\!A(\{2,3,6\}) = (0,1,1,1,0,1,\;0,1,1,1,0,1,\;0,1,1,1,0,1,...).
\end{gather}


\paragraph{Операторы.}
Следующие обозначения операторов будут широко использоваться в тексте диссертации.
\begin{itemize}
	\item
		Тождественный оператор:
		\begin{equation}
			I(x_1, x_2, x_3, x_4, ...) = (x_1, x_2, x_3, x_4, ...)
			.
		\end{equation}
	\item
		Оператор сдвига влево:
		\begin{equation}
			T(x_1, x_2, x_3, x_4, ...) = (x_2, x_3, x_4, ...)
			.
		\end{equation}
	\item
		Оператор сдвига вправо:
		\begin{equation}
			U(x_1, x_2, x_3, x_4, ...) = (0, x_1, x_2, x_3, x_4, ...)
			.
		\end{equation}
	\item
		Оператор растяжения ($n\in\N$):
		\begin{equation}
			\sigma_n (x_1, x_2, x_3, ...) = (
				\underbrace{x_1,...,x_1}_{n~\text{раз}},
				\underbrace{x_2,...,x_2}_{n~\text{раз}},
				\underbrace{x_3,...,x_3}_{n~\text{раз}},
				...)
			.
		\end{equation}
	\item
		Оператор усредняющего сжатия ($n\in\N$):
		\begin{equation}
			\sigma_{1/n} x = n^{-1}
			\left(
				\sum_{i=1}^{n} x_i,
				\sum_{i=n+1}^{2n} x_i,
				\sum_{i=2n+1}^{3n} x_i,
				...
			\right).
		\end{equation}
	\item
		Оператор Чезаро (иногда называемый оператором Харди):
		%TODO: ссылки, где это Харди?
		\begin{equation}
			(Cx)_n = n^{-1} \cdot \sum_{k=1}^n x_k
			,
		\end{equation}
		т.е.
		\begin{equation}
			C (x_1, x_2, x_3, ...) = \left(
			x_1,
			\dfrac{x_1+x_2}2,
			\dfrac{x_1+x_2 + x_3}3,
			\dfrac{x_1+x_2+x_3+x_4}4,
			...,
			\dfrac{x_1+...+x_n}n,
			...\right)
			.
		\end{equation}

	\item
		Оператор суперпозиции (покоординатного умножения):
		\begin{equation}
			x \cdot y = (x_1\cdot y_1; ~~x_2\cdot y_2; ~~x_3\cdot y_3; ~...)
		\end{equation}
\end{itemize}


\paragraph{Специальные функции и множества.}

Через $\B$ будем обозначать множество всех банаховых пределов;
через $\B(H)$ ~--- множество всех банаховых пределов, инвариантных относительно оператора $H$.
%TODO: ссылку на теорему/определние?

Через $\ext A$ будем обозначать множество крайних точек множества $A$.

Нижний и верхний функционалы Сачестона соответственно~\cite{sucheston1967banach}:
\begin{equation*}
	q(x) = \lim_{n\to\infty} \inf_{m\in\N}  \frac{1}{n} \sum_{k=m+1}^{m+n} x_k
	~~~~\mbox{и}~~~~
	p(x) = \lim_{n\to\infty} \sup_{m\in\N}  \frac{1}{n} \sum_{k=m+1}^{m+n} x_k
	.
\end{equation*}

Кроме того, мы будем часто пользоваться функцией~\cite{our-vzms-2018}:
\begin{equation}
	\alpha(x) = \varlimsup_{i\to\infty} \max_{i<j\leqslant 2i} |x_i - x_j|
	.
\end{equation}


\section*{Краткое содержание работы}
\addcontentsline{toc}{section}{Краткое содержание работы}
Сходящиеся последовательности, т.е. последовательности, имеющие предел в смысле классического математического анализа,
изучены достаточно хорошо.
В частности, любая сходящаяся последовательность является ограниченной.
Пространство ограниченных последовательностей будем, вслед за классиками \cite{wojtaszczyk1996banach,lindenstrauss1973classical},
обозначать через $\ell_\infty$ и снабжать его нормой
\begin{equation*}
	\|x\| = \sup_{n\in\mathbb{N}}|x_n|
	.
\end{equation*}

Однако в приложениях часто возникают ограниченные последовательности,
которые не являются сходящимися.
В таком случае возникает закономерный вопрос:
как измерить <<недостаток сходимости>>?
<<насколько не сходится>> последовательность?

Наиболее очевидным кажется вычисление расстояния $\rho(x,c)$ от заданного элемента $x\in\ell_\infty$
до пространства сходящихся последовательностей $c$
(которое равно половине разности верхнего и нижнего пределов последовательности).
Однако выясняется, что имеют место быть и другие подходы.

Нетрудно заметить, что операция взятия классического предела на пространстве сходящихся последовательностей
является непрерывным (в норме $\ell_\infty$) линейным функционалом.
В 1929 г. С. Мазур анонсировал~\cite{Mazur}, а позже
С. Банах доказал \cite{banach2001theory_rus}, что этот функционал может быть непрерывно продолжен на всё пространство $\ell_\infty$.
На основе этой идеи были определены банаховы пределы
(иногда также называемые пределами Банаха--Мазура \cite{alekhno2012superposition,alekhno2015banach})
следующим образом.

Банаховым пределом называется функционал $B\in \ell_\infty^*$ такой, что:
\begin{enumerate}
	\item
		$B \geqslant 0$
	\item
		$B\one = 1$
	\item
		$B=BT$
\end{enumerate}

Простейшие свойства:
\begin{itemize}
	\item
		$\|B\|_{\ell_\infty^*} = 1$
	\item
		$Bx = \lim\limits_{n\to\infty} x_n$ для любого $x=(x_1, x_2, ...) \in c$.

		Таким образом,
		банахов предел~--- действительно естественное обобщение понятия предела сходящейся последовательности
		на все ограниченные последовательности.
\end{itemize}

Множество банаховых пределов обычно обозначают через $\mathfrak{B}$
(реже через $BM$~--- см., например, \cite{alekhno2012superposition,alekhno2015banach}).

Лоренц \cite{lorentz1948contribution} установил, что существует подпространство $\ell_\infty$,
на котором все банаховы пределы принимают одинаковое значение.
Это пространство названо пространством почти сходящихся последовательностей и обычно обозначается $ac$
(от англ. <<almost convergent>>).
Включение $c \subset ac$ собственное, т.е. $ac \setminus c \neq \varnothing$.


Обобщая критерий Лоренца, Сачестон \cite{sucheston1967banach} доказал, что для любого $x\in\ell_\infty$
и любого $B\in\mathfrak{B}$
\begin{equation*}
	q(x) =
	\lim_{n\to\infty} \sup_{m\in\mathbb{N}} \frac{1}{n} \sum_{k=m+1}^{k=m+n} x_k
	\leq
	Bx
	\leq
	\lim_{n\to\infty} \inf_{m\in\mathbb{N}} \frac{1}{n} \sum_{k=m+1}^{k=m+n} x_k
	= p(x)
\end{equation*}
и, более того,
\begin{equation*}
	\mathfrak{B}x = [q(x), p(x)]
	.
\end{equation*}


За более подробным обзором ранних исследований банаховых пределов отсылаем читателя к~\cite{greenleaf1969invariant,day1973normed,kangro1976theory}.
%Source: https://encyclopediaofmath.org/wiki/Banach_limit
Вскоре после работ Сачестона Дж. Куртц распространил понятие банаховых пределов
на векторные последовательности~\cite{kurtz1970almost},
а затем и на последовательности в произвольных банаховых пространствах~\cite{kurtz1972almost}.
За обсуждением банаховых пределов в векторных пространствах отсылаем читателя
к~\cite{deeds1968summability,hajdukovic1975almost,armario2013vectorvalued_rus,garcia2015extremal,garcia2016fundamental_rus}.
%Тут есть ещё ссылки: https://www.mathnet.ru/php/archive.phtml?wshow=paper&jrnid=faa&paperid=3146&option_lang=rus
%TODO2: В том числе и про то, где применяется.
В недавней работе~\cite{chen2007characterizations} Ч.~Чен и М.~Куо изучают обобщения банаховых пределов
на произвольные гильбертовы пространства и на пространства суммируемых функций $L_p$.
Другим обобщениям банаховых пределов посвящены работы
\cite{hajdukovic1975functionals,koga2016generalization}.
%tanaka2018banach - иероглифы, несовместимые с библиографией

Ещё одним достаточно плодотворным обобщением банаховых пределов оказались их аналоги на двойных последовательностях~\cite{robison1926divergent}, введённые Дж.~Д.~Хиллом в~\cite{hill1965almost}.
За дальнейшими результатами в этом направлении отсылаем читателя
к~\cite{moricz1988almost,bacsarir1995strong,mursaleen2003almost,edely2004almost,mursaleen2004almost}.
Из недавних работ стоит отдельно отметить статью М. Мурсалена и С.А. Мухиддина~\cite{mursaleen2012banach},
в которой с помощью понятия почти сходимости в пространстве ограниченных двойных последовательностей вводится ряд новых интересных подпространств.

Наконец, если исключить из определения банахова предела требование трансляционной инвариантности,
то мы получим объект, называемый обобщённым пределом,
подробно изучавшийся М.~Джерисоном в~\cite{jerison1957set} и многих других работах.


Таким образом, на вопрос: <<Насколько не сходится последовательность?>> %~---
можно дать ответ в терминах почти сходимости, т.е. принадлежности пространству $ac$,
а на вопрос: <<Насколько почти не сходится последовательность?>>~---
назвать длину отрезка $[q(x), p(x)]$.
В дальнейшем пространство почти сходящихся последовательностей неоднократно становилось предметом
различных исследований
\cite{semenov2006ac,usachev2008transforms}.
В частности, в работе~\cite{connor1990almost} доказано,
что последовательность из нулей и единиц почти наверное не принадлежит пространству $ac$.
Этот факт демонстрирует, что почти сходящиеся последовательности <<достаточно редки>>.

%TODO: ссылки! Хватит или ещё?

Банаховы пределы также нашли своё применение в приложениях
\cite{semenov2015banachtraces,SU,strukova2015spectres}.
%Известны обобщения банаховых пределов на двойные последовательности
%\cite{edely2004almost}.

%TODO: ссылки! Хватит или ещё?


В настоящей работе рассматриваются некоторые вопросы асимптотических характеристик ограниченных последовательностей,
в том числе банаховых пределов.
Нумерация приводимых ниже теорем, лемм, определений и следствий совпадает с их нумерацией в диссертации.


В главе 1 обсуждается пространство $ac$ и его подпространство $ac_0$,
даётся критерий почти сходимости к нулю (т.е. принадлежности пространству $ac_0$)
знакопоcтоянной последовательности.

\reflecttheorem{thm:M_j_ac0_inf_lim}
	Пусть $n_i$~--- строго возрастающая последовательность натуральных чисел,
	\begin{equation}
		\label{eq:definition_M_j}
		M(j) = \liminf_{i\to\infty} n_{i+j} - n_i,
	\end{equation}
	\begin{equation}
		x_k = \left\{\begin{array}{ll}
			1, & \mbox{~если~} k = n_i
			\\
			0  & \mbox{~иначе~}
		\end{array}\right.
	\end{equation}
	Тогда следующие условия эквивалентны:
	\\
	(i)   $x \in ac_0$;
	\\\\
	(ii)  $\lim\limits_{j \to \infty} \dfrac{M(j)}{j} = +\infty$;
	\\\\
	(iii) $\inf\limits_{j \in \N}     \dfrac{M(j)}{j} = +\infty$.


\reflecttheorem{thm:crit_ac0_Mj_lambda}
	Пусть $x\in\ell_\infty$, $x \geq 0$, $\lambda>0$.
	Обозначим через $n^{(\lambda)}_i$ возрастающую последовательность
	индексов таких элементов $x$, что $x_k \geq \lambda$ тогда и только тогда,
	когда $k=n^{(\lambda)}_i$ для некоторого $i$.
	Обозначим
	\begin{equation}
		M^{(\lambda)}(j) = \liminf_{i\to\infty} n^{(\lambda)}_{i+j} - n^{(\lambda)}_i
		.
	\end{equation}


	Тогда для того, чтобы $x\in ac_0$, необходимо и достаточно, чтобы
	для любого $\lambda>0$ было выполнено
	\begin{equation}
		\lim_{j \to \infty} \frac{M^{(\lambda)}(j)}{j} = +\infty
		.
	\end{equation}

\reflecttheorem{thm:rho_x_c_leq_alpha_t_s_x_united}
	Для любого $x\in ac$
	\begin{equation}
		\frac{1}{2} \alpha(x) \leq \rho(x,c)\leq \lim_{s\to\infty} \alpha(T^s x)
		.
	\end{equation}

\reflectcorollary{cor:rho_x_c0_leq_alpha_t_s_x_united}
	Для любого $x\in ac_0$
	\begin{equation}
		\frac{1}{2} \alpha(x) \leq \rho(x,c_0)\leq \lim_{s\to\infty} \alpha(T^s x)
		.
	\end{equation}

\reflecttheorem{thm:Connor_generalized}
	Мера множества $F=\{x\in\Omega : q(x) = 0 \wedge p(x)= 1\}$,
	где $p(x)$ и $q(x)$~--- верхний и нижний функционалы Сачестона соответственно,
	равна 1.


В главе 2
%TODO: \ref ???
изучается $\alpha$--функция, введённая в~\cite{our-vzms-2018}:
\begin{equation}
	\alpha(x) = \varlimsup_{i\to\infty} \max_{i<j\leqslant 2i} |x_i - x_j|
	.
\end{equation}
%
%TODO: ссылка на статью Семенова!
%
Поскольку $\alpha(c)=0$,
то $\alpha$--функцию также можно считать <<мерой несходимости>> последовательности;
равенство $\alpha(x) = 0$, однако, вовсе не гарантирует сходимость.

Устанавливается, что $\alpha$--функция не инвариантна относительно оператора сдвига $T$,
и даётся оценка на $\alpha(T^n x)$.
С другой стороны, $\alpha$--функция, в отличие от некоторых банаховых пределов
\cite{Semenov2010invariant,Semenov2011dan},
инвариантна относительно операторов растяжения $\sigma_n$.
Затем выявляется связь между $\alpha$--функцией, расстоянием от заданнной последовательности до пространства $c$
и почти сходимостью.
Рассмотрены и другие свойства $\alpha$--функции.
Приведём основные результаты.

\reflectcorollary{thm:est_alpha_Tn_x_full}
	Для любых $x\in\ell_\infty$ и $n \in \N$
	\begin{equation}\label{est_alpha_Tn_x}
		\frac{1}{2}\alpha(x) \leq \alpha(T^n x) \leq \alpha(x)
		.
	\end{equation}

\reflecttheorem{thm:alpha_beta_T_seq}
	Пусть $\beta_k$~--- монотонная невозрастающая последовательность,
	$\beta_k \to \beta$, $\beta\in\left[\frac{1}{2}; 1\right]$, $\beta_1 \leq 1$.
	Тогда существует такой $x\in\ell_\infty$, что для любого натурального $n$
	\begin{equation}
		\frac{\alpha(T^n x)}{\alpha(x)} = \beta_n.
	\end{equation}

\reflecttheorem{thm:alpha_sigma_n}
	Для любого $x\in\ell_\infty$ и для любого натурального $n$ верно равенство
	\begin{equation}
		\alpha(\sigma_n x) = \alpha(x)
		.
	\end{equation}

\reflecttheorem{thm:alpha_sigma_1_n}
	Для любого $n\in\N$ и любого $x\in\ell_\infty$ выполнено
	\begin{equation}
		\alpha(\sigma_{1/n} x) \leq \left( 2- \frac{1}{n} \right) \alpha(x)
		.
	\end{equation}

\reflecttheorem{thm:alpha_Cx_no_gamma}
	Имеет место равенство
	\begin{equation}
		\sup_{x\in\ell_\infty, \alpha(x)\neq 0} \frac{\alpha(Cx)}{\alpha(x)}=1
		.
	\end{equation}

\reflecttheorem{thm:alpha_xy}
	Пусть $(x\cdot y)_k = x_k\cdot y_k$.
	Тогда
	$\alpha(x\cdot y)\leq \alpha(x)\cdot \|y\|_* + \alpha(y)\cdot \|x\|_*$,
	где
	\begin{equation}
		\|x\|_* = \limsup_{k\to\infty} |x_k|
	\end{equation}
	есть  фактор-норма по $c_0$ на пространстве $\ell_\infty$.

В параграфе~\ref{sec:space_A0} исследуется пространство $A_0 = \{x: \alpha(x) = 0\}$.
Это пространство несепарабельно, замкнуто относительно покоординатного умножения,
операторов левого и правого сдвигов, оператора Чезаро,
операторов растяжения $\sigma_n$ и усредняющего сжатия $\sigma_{1/n}$.

В параграфе~\ref{sec:noncomplementarity} устанавливается, что в цепочке вложений
\begin{equation}
	c_0 \subset A_0 \subset \ell_\infty
\end{equation}
оба подпространства недополняемы.

Как сказано выше, банаховы пределы по определению (как и обычный предел на пространстве $c$) инвариантны относительно оператора сдвига.
Возникает закономерный вопрос: можно ли потребовать от банахова предела сохранять своё значение
при суперпозиции с некоторыми другими операторами на $\ell_\infty$?
Эту проблему исследовал У. Эберлейн в 1950 г. \cite{Eberlein},
т.е. через два года после классической работы Г. Г. Лоренца~\cite{lorentz1948contribution}.
Эберлейн установил, что существуют такие линейные операторы  $A : \ell_\infty\to \ell_\infty$,
для которых $BAx = Bx$ независимо от выбора $x$ и для банаховых пределов специального вида.

Будем говорить, что $B\in\mathfrak B(A)$, $A : \ell_\infty\to \ell_\infty$, если для любого $x\in \ell_\infty$
выполнено равенство $BAx = Bx$.
Такой банахов предел $B$ называют инвариантным относительно оператора $A$.

Можно ли выделить какие-то особые свойства оператора сдвига,
которые необходимы или достаточны оператору, чтобы относительно него были инвариантны все или некоторые банаховы пределы?
Понятно, что если оператор $A$ таков, что для любого $x\in\ell_\infty$ между $Ax$ и $x$
существует (конечное) расстояние Дамерау--Левенштейна \cite{damerau1964technique} (т.е. минимальное количество операций вставки, удаления, замены и перестановки двух соседних элементов последовательности, необходимых для перевода $x$ в $Ax$, причём для разных $x\in\ell_\infty$ эти операции, вообще говоря, не обязаны быть одинаковыми), то относительно данного оператора инвариантен любой банахов предел. Аналогичное утверждение справедливо и в случае, если $Ax -x \in c_0$ для любого $x\in \ell_\infty$.

Следующим по естественности (после сдвига и замены конечного числа элементов) действием, сохраняющем сходимость последовательности, является повторение элементов последовательности, например, оператор
\begin{equation}
	\sigma_2(x_1,x_2,x_3,...) = (x_1,x_1, \; x_2, x_2, \; x_3, x_3, \; ...)
	.
\end{equation}
Однако относительно такого оператора инвариантны не все, а только некоторые банаховы пределы.
Так, в~\cite[теорема 14]{ASSU2} показано, что
\begin{equation}
	\B(\sigma_n) \cap \ext \B = \varnothing  \mbox{~~для любого~~} n\in\N_2
	.
\end{equation}
Заметим, что если мы рассмотрим оператор неравномерного растяжения
\begin{equation}
	\sigma_{1,2}(x_1,x_2,x_3,x_4,x_5,...) = (x_1, \; x_2, x_2, \;  x_3, \; x_4, x_4, \; x_5, ...)
	,
\end{equation}
то увидим, что периодическую последовательность $y_n = (-1)^n$, $y\in ac_0$ оператор $\sigma_{1,2}$
переводит в периодическую последовательность
\begin{equation}
	(-1, 1, 1, \; -1, 1, 1, \; ...) \in ac_{1/3}
	,
\end{equation}
поскольку на периодической последовательности любой банахов предел принимает значение, равное среднему по периоду.
Таким образом, не существует банаховых пределов, инвариантных относительно оператора $\sigma_{1,2}$.

В главе 3 изложены некоторые примеры операторов и найдены множества банаховых пределов,
инвариантных относительно этих операторов.
Затем рассматриваются следующие классы линейных операторов $H:\ell_\infty \to \ell_\infty$:

-- полуэберлейновы: такие, что $B_1 H \in\B$ для некоторого $B_1\in \B$;

-- эберлейновы: такие, что $B_1 H = B_1$ для некоторого $B_1\in \B$;

-- В-регулярные: такие, что $B_1 H \in \B$ для любого   $B_1\in \B$;

-- существенно эберлейновы: такие, что $B_1 H \in \B$ для любого $B_1\in \B$ и $B_2 H \ne B_2$ для некоторого $B_2\in \B$.

Устанавливается (см. теоремы~\ref{thm:amiable_but_not_Eberlein_exists} и~\ref{thm:Eberlein_but_not_B-regular_exists}),
что каждый следующий из этих классов вложен в предыдущий и не совпадает с ним.
Кроме того, доказывается ещё ряд смежных результатов, в частности, решается обратная задача об инвариантности.

\reflecttheorem{thm:generated_operator_G_B}
	Для каждого $B\in \B$ существует такой оператор $G_B:\ell_\infty \to \ell_\infty$,
	что $\B(G_B) = \{B\}$.


Главы 4 и 5 посвящены верхнему и нижнему функционалам Сачестона $p(x)$ и $q(x)$"--- аналогам верхнего и нижнего пределов последовательности.
В главе 4 изучаются разделяющие множества.

Обозначим через $\Omega$ множество всех последовательностей, состоящих из нулей и единиц.

\reflecttheorem{thm:Lin_Omega_Sucheston}
	Пусть
	$1 \geq a > b \geq 0$ и
	$\Omega^a_b = \{x\in\Omega : p(x) = a, q(x) = b\}$,
	где $p(x)$ и $q(x)$~--- верхний и нижний функционалы Сачестона~\cite{sucheston1967banach} соответственно.
	Тогда $\Omega \subset \operatorname{Lin} \Omega^a_b$.


\reflectcorollary{crl:Lin_Omega_Sucheston}
	Множество $\Omega^a_b$ является разделяющим.
	Т.к. при $a\neq 1$ или $b\neq 0$ множество $\Omega^a_b$ имеет меру нуль~\cite{semenov2010characteristic,connor1990almost},
	то оно является разделяющим множеством нулевой меры.


Пусть $X^a_b = \{x\in\ell_\infty : p(x) = a,~ q(x) = b\}$, $Y^a_b = \{x\in A_0 : p(x) = a, q(x) = b\}$, где $a>b$.
\reflecttheorem{thm:A_0_c_infty_lin}
	Пусть $a\neq -b$.
	Тогда справедливо равенство $\operatorname{Lin} Y^a_b = A_0$.

\reflecttheorem{thm:Lin_ell_infty}
	Справедливо равенство $\operatorname{Lin} X^a_b = \ell_\infty$.

Через $\dim_H E$ будем обозначать хаусдорфову размерность множества $E$.

\reflecttheorem{thm:Hausdorf_measure_1_n}
	Пусть $n\in\N$.
	Тогда существует разделяющее множество $E\subset\Omega$ такое,
	что $\dim_H E = 1/n$.

%Конец главы 4

Глава 5 посвящена связи мультипликативных свойств носителя последовательности из нулей и единиц
и значений, которые могут принимать функционалы Сачестона на такой последовательности.


\reflectcorollary{cor:ac0_powers_finite_set_of_numbers}
	Пусть $\{p_1, ..., p_k\} \subset \N$,
	\begin{equation}
		x_k = \begin{cases}
			1, &\mbox{~если~} k = p_1^{j_1}\cdot p_2^{j_2}\cdot ... \cdot p_k^{j_k} \mbox{~для некоторых~} j_1,...,j_k\in\N,
			\\
			0  &\mbox{~иначе}.
		\end{cases}
	\end{equation}
	Тогда $x\in ac_0$.

\reflectdefinition{def:P-property}
	Будем говорить, что множество $A\subset\N$ обладает $P$-свойством,
	если для любого $n\in\N$ найдётся набор попарно взаимно простых чисел
	\begin{equation}
		\{a_{n,1}, a_{n,2}, ..., a_{n,n}  \} \subset A
		.
	\end{equation}

\reflecttheorem{thm:p_x_infinite_multiples}
	Пусть $A\subset \N\setminus\{1\}$.
	Тогда следующие условия эквивалентны:
	\begin{enumerate}[label=(\roman*)]
		\item
			$A$ обладает $P$-свойством
		\item
			В $A$ существует бесконечное подмножество попарно взаимно простых чисел
		\item
			$p(\chi\mathscr{M}A)=1$.
	\end{enumerate}

\reflectcorollary{cor:ac0_primes_p_psi_A_prod}
	Пусть $A = \{a_1, a_2, ..., a_n,...\}$ "--- бесконечное множество попарно взаимно простых чисел
	и $a_{n+1}>a_1\cdot...\cdot a_n$.
	Тогда
	\begin{equation}
		q(\chi\mathscr{M}A) = 1-\prod_{j=1}^\infty \left(1-\frac{1}{a_j}\right)
		.
	\end{equation}

\reflectlemma{lem:q_x_infinite_Euler}
	Пусть $\varepsilon \in  (0; 1{]}$.
	Существует бесконечное множество попарно непересекающихся подмножеств простых чисел
	$A_i$ такое, что $q(\chi\mathscr{M}A_i)\geq\varepsilon$ для любого $i\in\N$.



\chapter*{Заключение}
\addcontentsline{toc}{chapter}{Заключение}
В работе исследованы такие асимптотические характеристики ограниченных последовательностей,
как $\alpha$--функция (глава 1) и  почти сходимость (глава 2).% и банаховы пределы (глава 3).

Несмотря на то, что $\alpha$--функция оказалась трансляционно неинвариантной,
эта неинвариантность в некотором смысле однородна (см. следствие \ref{thm:est_alpha_Tn_x_full}).
Для элементов пространства почти сходящихся последовательностей $ac $
установлена двусторонняя оценка на расстояние до пространства сходящихся последовательностей $c$,
использующая $\alpha$--функцию.
Примечательно (хотя и ожидаемо), что наряду с трансляционно неинвариантной $\alpha$--функцией
в этой оценке используется и функционал $\lim_{n\to\infty}\alpha(T^n x)$,
который, очевидно, трансляционно инвариантен.
Это вполне логично, поскольку расстояние до пространства $c$ и почти сходимость
сами суть трансляционно инвариантные характеристики.




По итогам исследований, результаты которых составили данную работу,
опубликованы тезисы \cite{our-vvmsh-2018,our-vzms-2018,our-ped-2018-inf-dim-ker,our-ped-2018-alpha-Tx},
а также краткое сообщение~\cite{our-mz2019ac0};
планируется публикация ещё нескольких печатных работ.

Выдвинут ряд гипотез,
работу над доказательством или опровержением которых планируется продолжать в дальнейшем.



~~


Соискатель выражает горячую благодарность:
\begin{itemize}
	\item
		научному руководителю, д.ф.-м.н., проф. Евгению Михайловичу Семенову,
		за
		постановку интересных задач и возможность из них выбирать,
		за
		содержательные дискуссии
		и умение сказать нужные слова в нужный момент;
	\item
		коллегам, к.ф.-м.н. Александру Сергеевичу Усачёву и асп. Роману Евгеньевичу Зволинскому,
		за
		плодотворное обсуждение результатов и текста диссертационного исследования,
		а также рабочую атмосферу в целом;
	\item
		маме, Елене Анатольевне Матвеевой,
		и бабушке, Ольге Евгеньевне Матвеевой,
		за необыкновенные терпение и заботу
		в период подготовки текста диссертации и сопроводительной документации;
	\item
		супруге, к.ф.-м.н. Анастасии Сергеевне Червинской,
		за
		передачу ценного личного опыта,
		а также моральную, организационную и логистическую поддержку;
	\item
		секретарю диссертационного совета, д.ф.-м.н. Антону Юрьевичу Савину,
		за подробные и терпеливые ответы на десятки моих вопросов, возникших по мере продвижения к процедуре защиты;
	\item
		секретарю Учёного совета математического факультета ВГУ, Галине Ивановне Моисеевой,
		за
		огромную помощь и содействие при подготовке сопроводительной документации;
	\item
		начальнику отдела аспирантуры ВГУ, Любови Николаевне Костиной, а также ведущему специалисту, Елене Валентиновне Сыромятниковой,
		за
		помощь и консультации при подготовке сопроводительной документации;
	\item
		специалисту ДАНК РУДН, Марине Юрьевне Анисенко,
		за
		помощь и консультации при подготовке сопроводительной документации.
\end{itemize}




\makeatletter
\ltx@iffilelater{biblatex-gost.def}{2017/02/01}%
{\toggletrue{bbx:gostbibliography}%
\renewcommand*{\revsdnamepunct}{\addcomma}}{}
\makeatother




\section*{Работы автора по теме диссертации}
\paragraph{Статьи в научных журналах}

\printbibliography[keyword=avdeev,category=vakpapers,heading=none,title={~}]{}

Работы~\selfcite{\vakpaperskeys}
из списка опубликованы в изданиях, входящих в перечень
ВАК и/или индексируемых базами данных Web of Science и/или Scopus.

\paragraph{Тезисы конференций}


\printbibliography[keyword=avdeev,notcategory=vakpapers,heading=none,title={~}]{}


%\printbibliography[notkeyword=avdeev,title={Список использованных источников}]{}

\end{document}
