Приведём список обозначений, которые в тексте диссертации могут быть использованы без дополнительного пояснения.


\paragraph{Числовые множества.}
Через $\N$ будем обозначать множество натуральных чисел: $\N = \{1;2;3;4;...\}$.
Через $\N_k$ ~--- множество целых чисел, не меньших $k$ (чтобы избежать громоздкой записи вида: пусть $n\in \N$, $n \geq 2$, ...).
Так,
\begin{equation}
	\N_1 = \N
	,
	\quad
	\N_0 = \N \cup\{0\} = \{0;1;2;3;4;...\}
	,
	\quad
	\N_3 = \{3;4;5;6;7;...\}
	.
\end{equation}

Через $\Q$ и $\R$ будем обозначать множества рациональных и вещественных чисел соответственно;
через $\Q^+$, $\Q^-$, $\R^+$ и $\R^-$ ~--- множества положительных рациональных, отрицательных рациональных,
положительных вещественных и отрицательных вещественных чисел соответственно.

\paragraph{Пространства последовательностей и их подмножества.}

Основным используемым пространством будет пространство $\ell_\infty$ ограниченных последовательностей со стандартной нормой
\begin{equation}
	\|x\| = \sup_{n\in\N} |x_n|
	.
\end{equation}
Эта же норма будет использоваться и в остальных пространствах и множествах, среди которых:
\begin{itemize}
	\item
		$c$ ~--- пространство сходящихся последовательностей;
	\item
		$c_\lambda$ ~--- множество последовательностей, сходящихся к $\lambda \in \R$, в частности,
	\item
		$c_0$ ~--- пространство последовательностей, сходящихся к нулю;
	\item
		$c_{00}$ ~--- пространство последовательностей с конечным носителем~\cite[теорема 4]{ASSU2};
	\item
		$ac$ ~--- пространство почти сходящихся последовательностей;
	\item
		$ac_\lambda$ ~--- множество последовательностей, почти сходящихся к $\lambda \in \R$, в частности,
	\item
		$ac_0$ ~--- пространство последовательностей, почти сходящихся к нулю;
	\item
		$\Iac$ ~--- максимальный (по включению) идеал по умножению в пространстве $ac_0$ (см. теорему~\ref{thm:Iac_criterion_pos_neg});
	\item
		$A_0 = \{ x\in\ell_\infty : \alpha(x) = 0\}$ (см. ниже; о свойствах этого пространства см., напр., теорему~\ref{thm:A0_is_space} и далее);
	\item
		$\Omega = \{0;1\}^\N$ ~--- множество последовательностей, состоящих из нулей и единиц.
\end{itemize}



\paragraph{Нормы.}
Запись $\|\cdot\|$ без уточнений будет обозначать норму в пространстве $\ell_\infty$, $\ell_\infty^*$
или в пространстве операторов, действующих из $\ell_\infty$ в $\ell_\infty$ (обозначаемом $\mathcal L (\ell_\infty, \ell_\infty)$~)
в зависимости от природы аргумента.
В случае, если нужно ввести другую норму (например, фактор-норму $\ell_\infty / c_0$, как в теореме~\ref{thm:alpha_xy}),
это будет оговорено явно.


\paragraph{Последовательности.}
Для любого $x\in\ell_\infty$ будем по умолчанию полагать, что
\begin{equation}
	x=(x_1, x_2, x_3, x_4, ...)
	.
\end{equation}
Кроме того, нам будет часто требоваться константная единица $\one = (1, 1, 1, 1, ...)$.

Будем писать, что $x\geq 0$, если $x_n \geq 0$ для любого $n\in \N$, и $x\leq 0$, если $x_n \leq 0$ для любого $n\in \N$.
%TODO: периодические последовательности и конкатенация

Вслед за~\cite{hall1992behrend} будем обозначать через $\mathscr{M}A$ множество всех чисел,
кратных элементам множества $A\subset\N$, т.е.
\begin{equation}
	\mathscr{M}A = \{ka: k\in\N, a\in A\}
	,
\end{equation}
через $\chi F$ "--- характеристическую функцию множества $F$.

Так, например,
\begin{gather}
	\chi \mathscr{M}\!A(\{2\}) = \chi \mathscr{M}\!A(\{2, 4\}) = \chi \mathscr{M}\!A(\{2,4,8,16,...\})
	= (0,1,0,1,0,1,0,1,0,...),
\\
	\chi \mathscr{M}\!A(\{3\}) = \chi \mathscr{M}\!A(\{3,9,27,...\}) = (0,0,1,\;0,0,1,\;0,0,1,\;0,0,1,\;0,0,1,\;...),
\\
	\chi \mathscr{M}\!A(\{2,3\}) = \chi \mathscr{M}\!A(\{2,3,6\}) = (0,1,1,1,0,1,\;0,1,1,1,0,1,\;0,1,1,1,0,1,...).
\end{gather}


\paragraph{Операторы.}
Следующие обозначения операторов будут широко использоваться в тексте диссертации.
\begin{itemize}
	\item
		Оператор сдвига влево:
		\begin{equation}
			T(x_1, x_2, x_3, x_4, ...) = (x_2, x_3, x_4, ...)
			.
		\end{equation}
	\item
		Оператор сдвига вправо:
		\begin{equation}
			U(x_1, x_2, x_3, x_4, ...) = (0, x_1, x_2, x_3, x_4, ...)
			.
		\end{equation}
	\item
		Оператор растяжения ($n\in\N$):
		\begin{equation}
			\sigma_n (x_1, x_2, x_3, ...) = (
				\underbrace{x_1,...,x_1}_{n~\text{раз}},
				\underbrace{x_2,...,x_2}_{n~\text{раз}},
				\underbrace{x_3,...,x_3}_{n~\text{раз}},
				...)
			.
		\end{equation}
	\item
		Оператор усредняющего сжатия ($n\in\N$):
		\begin{equation}
			\sigma_{1/n} x = n^{-1}
			\left(
				\sum_{i=1}^{n} x_i,
				\sum_{i=n+1}^{2n} x_i,
				\sum_{i=2n+1}^{3n} x_i,
				...
			\right).
		\end{equation}
	\item
		Оператор Чезаро:
		\begin{equation}
			(Cx)_n = n^{-1} \cdot \sum_{k=1}^n x_k
			,
		\end{equation}
		т.е.
		\begin{equation}
			C (x_1, x_2, x_3, ...) = \left(
			x_1,
			\dfrac{x_1+x_2}2,
			\dfrac{x_1+x_2 + x_3}3,
			\dfrac{x_1+x_2+x_3+x_4}4,
			...,
			\dfrac{x_1+...+x_n}n,
			...\right)
			.
		\end{equation}

	\item
		Оператор суперпозиции (покоординатного умножения):
		\begin{equation}
			x \cdot y = (x_1\cdot y_1; ~~x_2\cdot y_2; ~~x_3\cdot y_3; ~...)
		\end{equation}
\end{itemize}


\paragraph{Специальные функции и множества.}

Через $\B$ будем обозначать множество всех банаховых пределов;
через $\B(H)$ ~--- множество всех банаховых пределов, инвариантных относительно оператора $H$.
%TODO: ссылку на теорему/определние?

Через $\ext A$ будем обозначать множество крайних точек множества $A$.

Нижний и верхний функционалы Сачестона соответственно:
\begin{equation*}
	q(x) = \lim_{n\to\infty} \inf_{m\in\N}  \frac{1}{n} \sum_{k=m+1}^{m+n} x_k
	~~~~\mbox{и}~~~~
	p(x) = \lim_{n\to\infty} \sup_{m\in\N}  \frac{1}{n} \sum_{k=m+1}^{m+n} x_k
	.
\end{equation*}

Кроме того, мы будем часто пользоваться функцией
\begin{equation}
	\alpha(x) = \varlimsup_{i\to\infty} \max_{i<j\leqslant 2i} |x_i - x_j|
	.
\end{equation}
