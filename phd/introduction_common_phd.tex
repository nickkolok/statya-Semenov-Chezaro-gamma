\paragraph{Актуальность работы.}

Один из основателей функционального анализа как области современной математики С. Банах
в 1932 году ввёл в рассмотрение множество непрерывных линейных функционалов на пространстве ограниченных последовательностей,
которые совпадают с обычным пределом на всех сходящихся последовательностях.
Эти функционалы в дальнейшем и были названы банаховыми пределами;
их изучением занимались Г.Г. Лоренц, Л. Сачестон, Г. Дас, У.Ф. Эберлейн и другие математики.

В 1948 году Г.Г. Лоренц, используя банаховы пределы, ввёл понятие почти сходящейся последовательности "---
последовательности, на которой значение банахова предела не зависит от выбора этого банахова предела.
В 1967 году Л. Сачестон построил аналоги верхнего и нижнего пределов для банаховых пределов "---
(нелинейные) функционалы Сачестона.
Изучению и обобщению понятия почти сходимости посвящены работы
Р.А. Раими, М. Мурсалена, Г. Беннета, Н.Дж. Калтона, Д. Хаджуковича, Е.А. Алехно, Д. Занина и др.
Отдельный интерес представляет вопрос о банаховых пределах, инвариантных относительно некоторых линейных операторов.

Банаховы пределы тесно связаны с другими областями математики.
Так, в исследованиях С. Лорда, Дж. Филлипса, А.Л. Кери, П.Г. Доддса, Е.М. Семёнова, Б. де Пагтера,
А.А. Седаева, А.С. Усачёва, Ф.А. Сукочева банаховы пределы применяются к изучению следов Диксмье;
следы же Диксмье, в свою очередь, широко применяются в некоммутативной геометрии А. Конна.
Связи банаховых пределов с эргодической теорией посвящены работы Л. Сачестона и др.

В настоящей диссертации исследуются пространство почти сходящихся последовательностей
(и его подпространство последовательностей, почти сходящихся к нулю),
функционалы Сачестона и инвариантные банаховы пределы.

\paragraph{Целью работы}
является изучение банаховых пределов и их свойств инвариантности.

\paragraph{Методика исследований.}
Используются понятия, методы и подходы современного функционального анализа,
а также отдельные элементы и факты топологии и теории чисел.


\paragraph{Научная новизна.}
Все основные результаты диссертации являются новыми.
Среди них можно выделить следующие наиболее важные:
\begin{enumerate}
	\item
		сформулированы и доказаны специальные критерии почти сходимости
		ограниченной последовательности;
	\item
		исследована новая асимптотическая характеристика ограниченной последовательности,
		позволяющая выявлять дополнительные (по сравнению с банаховыми пределами)
		свойства таких последовательнсотей;
	\item
		построена иерархическая система классов ограниченных линейных операторов
		на пространстве $\ell_\infty$ в зависимости от свойств их суперпозиции с банаховыми пределами;
	\item
		построен пример множества последовательностей из нулей и единиц, разделяющего банаховы пределы
		и имеющего нулевую меру, индуцированную мерой Лебега на отрезке $[0;1]$;
	\item
		исследована связь мультпликативной структуры носителя последовательности из нулей и единиц
		и значений, которые на такой последовательности могут принимать верхний и нижний функционалы Сачестона.
\end{enumerate}


\paragraph{Теоретическая и практическая значимость.}
Работа носит теоретический характер.
Результаты диссертации могут быть использованы в учебном процессе, спецкурсах и научных исследованиях,
проводимых в Воронежском, Ростовском, Самарском, Ярославском государственных университетах и др.
В работе сформулированы гипотезы, изучение которых может быть выдано студентам и выпускникам бакалавриата, специалитета или магистратуры
в составе задания на выполнение курсовых или выпускных квалификационных работ.


\paragraph{Апробация работы.}
Результаты диссертации докладывались и обсуждались на:
\begin{itemize}
	\item
		Международной конференции <<Воронежская Зимняя Математическая школа С.Г. Крейна – 2018>>;
	\item
		Международной конференции «Понтрягинские чтения — XXIX», посвященной 90-летию Владимира Александровича Ильина (Воронеж, 2018~г.);
	\item
		Международной молодежной научной школе «Актуальные направления математического анализа и смежные вопросы» (Воронеж, 2018~г.);
	\item
		Международной (53-й Всероссийской) молодёжной школе-конференции
		<<Современные проблемы математики и её приложений>>
		(Екатеринбург, 2022~г.);
	\item
		конкурсе научно-исследовательских работ студентов и аспирантов российских вузов
		<<Наука будущего --- наука молодых>> в секции <<Информационные технологии и математика>>
		(присуждено II место среди аспирантов) в ноябре 2021~г.;
	\item научной сессии ВГУ в 2020, 2021, 2022, 2024 гг. %(TODO: проверить недостающие годы - письмо отправлено Г.И.; в 2023 выступления нет - академ)
\end{itemize}


\paragraph{Публикации.}
Основные результаты диссертации опубликованы в работах~\cite{
%Солидные статьи
our-mz2019ac0,
our-mz2019measure,
our-mz2021linearhulls,
avdeed2021AandA,
avdeev2021vestnik,
avdeev2024decomposition,
avdeev2021vmzprimes,
%Тезисы
our-vvmsh-2018,
our-vzms-2018,
our-ped-2018-inf-dim-ker,
our-ped-2018-alpha-Tx,
avdeev2022measure,
%TODO: если будут ещё работы
}.
Из совместных работ~\cite{our-mz2019measure,avdeed2021AandA,avdeev2024decomposition,our-vzms-2018} в диссертацию вошли только результаты, принадлежащие лично диссертанту.

\paragraph{Структура и объём диссертации.}
Диссертация состоит из введения, пяти глав, разбитых на TODO параграфов,
и списка литературы, включающего TODO источника.
Общий объём диссертации TODO страницы.
%TODO: а подпараграфы?
