В работе исследованы такие объекты и понятия, определяемые на пространстве ограниченных последовательностей,
как почти сходимость (глава 1), $\alpha$--полунорма, для краткости называемая $\alpha$--функцией (глава 2),
инвариантные банаховы пределы (глава 3), разделяющие множества и линейные оболочки (глава 4),
функционалы Сачестона и мультипликативные свойства носителя (глава 5).

В главе 1 выведены критерии принадлежности ограниченных последовательностей
специального вида к пространству почти сходящихся последовательностей
и пространству последовательностей, почти сходящихся к нулю,
а также критерий сходимости почти сходящейся последовательности.
Далее эти критерии применены для исследования последовательностей и множеств,
естественным образом возникающих в других задачах о банаховых пределах.
Для элементов пространства почти сходящихся последовательностей $ac $
установлена двусторонняя оценка на расстояние до пространства сходящихся последовательностей $c$,
использующая $\alpha$--функцию.
Примечательно (хотя и ожидаемо), что наряду с трансляционно неинвариантной $\alpha$--функцией
в этой оценке используется и функционал $\lim_{n\to\infty}\alpha(T^n x)$,
который, очевидно, трансляционно инвариантен.

Глава 2 посвящена $\alpha$--функции и содержит достаточно подробное описание
свойств суперпозиции этой функции с классическими линейными операторами:
сдвига $T$, повторения $\sigma_n$, Чезаро $C$ и т.д.
Несмотря на то, что $\alpha$--функция оказалась трансляционно неинвариантной,
эта неинвариантность в некотором смысле однородна (см. следствие \ref{thm:est_alpha_Tn_x_full}).
Это вполне логично, поскольку расстояние до пространства $c$ и почти сходимость
сами суть трансляционно инвариантные характеристики.
Завершается глава 2 исследованием подпространства таких ограниченных последовательностей,
на которых $\alpha$--функция обращается в нуль.
Доказаны основные свойства этого пространства;
в частности, оно недополняемо в $\ell_\infty$.

Глава 3 непосредственно посвящена инвариантным банаховым пределам и содержит
значительное количество примеров отыскания множества инвариантных банаховых пределов $\B(H)$
для различных линейных операторов $H$, действующих на пространстве ограниченных последовательностей.
В частности, показана существенная избыточность ранее известных условий эберлейновости оператора
(т.е. существования хотя бы одного банахова предела, инвариантного относительно данного оператора).
Введены новые и исследованы существующие классы линейных операторов на пространстве ограниченных последовательностей
в соответствии со свойствами их суперпозиции с банаховыми пределами:
полуэберлейновы, эберлейновы, В-регулярные, существенно эберлейновы.
Показано, что каждый следующий класс содержится в предыдущем и не совпадает с ним.
Найдено простое решение обратной задачи об инвариантности,
т.е. задачи о построении по банахову пределу оператора, относительно которого он инвариантен.

Глава 4 посвящена разделяющим множествам и линейным оболочкам, при этом вторые,
будучи детально исследованы, выступают как вспомогательный элемент для построения первых.
Завершается глава 4 построением разделяющего подмножества последовательностей из нулей и единиц,
имеющего (лебеговскую) меру нуль и сколь угодно малую хаусдорфову размерность.

Глава 5 устанавливает связь между мультипликативными свойствами носителя последовательности
из нулей и единиц и значениями, который могут принимать верхний и нижний функционалы Сачестона
(а значит, и банаховы пределы) на такой последовательности.
Для установления этой связи используются построения из теории чисел.


По итогам проведённых исследований выдвинут ряд гипотез,
работу над доказательством или опровержением которых планируется продолжать в дальнейшем,
или же использовать эти гипотезы как задачи для молодых исследователей,
знакомящихся с пространством ограниченных последовательностей.
