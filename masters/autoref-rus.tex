\documentclass[a4paper,openbib]{report}
\usepackage{amsmath}
\usepackage[utf8]{inputenc}
\usepackage[english,russian]{babel}
\usepackage{amsfonts,amssymb}
\usepackage{latexsym}
\usepackage{euscript}
\usepackage{enumerate}
\usepackage{graphics}
\usepackage[dvips]{graphicx}
\usepackage{geometry}
\usepackage{wrapfig}
\usepackage[colorlinks=true,allcolors=black]{hyperref}



\righthyphenmin=2

\usepackage[14pt]{extsizes}

\geometry{left=3cm}% левое поле
\geometry{right=1cm}% правое поле
\geometry{top=2cm}% верхнее поле
\geometry{bottom=2cm}% нижнее поле

\renewcommand{\baselinestretch}{1.3}

\renewcommand{\leq}{\leqslant}
\renewcommand{\geq}{\geqslant} % И делись оно всё нулём!

\newcommand{\longcomment}[1]{}

\usepackage[backend=biber,style=gost-numeric,sorting=none]{biblatex}
\addbibresource{../bib/Semenov.bib}
\addbibresource{../bib/my.bib}
\addbibresource{../bib/ext.bib}

\input{../bib/ext.hyphens.bib}

%Only referenced equations are numbered
\usepackage{mathtools}
\mathtoolsset{showonlyrefs}

%\mathtoolsset{showonlyrefs=false}
% (an equation/multline to be force-numbered)
%\mathtoolsset{showonlyrefs=true}



\begin{document}
\clubpenalty=10000
\widowpenalty=10000

%Автореферат Авдеева Н.Н.
%Тема магистерской диссертации: <<Асимптотические характеристики ограниченных последовательностей>>.


\paragraph{Введение.}
Сходящиеся последовательности, т.е. последовательности, имеющие предел в смысле классического математического анализа,
изучены достаточно хорошо.
В частности, любая сходящаяся последовательность является ограниченной.
Пространство ограниченных последовательностей будем, вслед за классиками \cite{wojtaszczyk1996banach,lindenstrauss1973classical},
обозначать через $\ell_\infty$ и снабжать его нормой
\begin{equation*}
	\|x\| = \sup_{n\in\mathbb{N}}|x_n|
	.
\end{equation*}

Именно пространство ограниченных последовательностей и стало объектом исследования данной работы.
Предметом же исследования явились их асимптотические свойства, т.е. такие свойства,
которые не изменяются при изменении конечного количества элементов последовательности.

В приложениях часто возникают ограниченные последовательности,
которые не являются сходящимися.
В таком случае возникает закономерный вопрос:
как измерить <<недостаток сходимости>>?
<<насколько не сходится>> последовательность?

Наиболее очевидным кажется вычисление расстояния $\rho(x,c)$ от заданного элемента $x\in\ell_\infty$
до пространства сходящихся последовательностей $c$
(которое равно половине разности верхнего и нижнего пределов последовательности).
Однако выясняется, что имеют место быть и другие подходы.

Нетрудно заметить, что операция взятия классического предела на пространстве сходящихся последовательностей
является непрерывным (в норме $\ell_\infty$) линейным функционалом.
Банах доказал \cite{banach1993theorie}, что этот функционал может быть непрерывно продолжен на всё пространство $\ell_\infty$.
На основе этой идеи были определены банаховы пределы
(иногда также называемые пределами Банаха--Мазура \cite{alekhno2012superposition,alekhno2015banach})
следующим образом.

Банаховым пределом называется функционал $B\in \ell_\infty^*$ такой, что:
\begin{enumerate}
	\item
		$B \geqslant 0$
	\item
		$B\mathbb{I} = 1$
	\item
		$B=BT$
\end{enumerate}
Здесь $\mathbb{I}=(1,1,1,1,1,...)$,
$T$~--- оператор сдвига: $T(x_1, x_2, x_3, ...) = (x_2, x_3, ...)$.

Простейшие свойства:
\begin{itemize}
	\item
		$\|B\|_{\ell_\infty^*} = 1$
	\item
		$Bx = \lim\limits_{n\to\infty} x_n$ для любого $x=(x_1, x_2, ...) \in c$.

		%Таким образом,
		%банахов предел~--- действительно естественное обобщение понятия предела сходящейся последовательности
		%на все ограниченные последовательности.
\end{itemize}

Множество банаховых пределов обычно обозначают через $\mathfrak{B}$
(реже через $BM$~--- см., например, \cite{alekhno2012superposition,alekhno2015banach}).
%
Лоренц \cite{lorentz1948contribution} установил, что существует подпространство $\ell_\infty$,
на котором все банаховы пределы принимают одинаковое значение.
Это пространство названо пространством почти сходящихся последовательностей и обычно обозначается $ac$
(от англ. <<almost convergent>>).
Включение $c \subset ac$ собственное.
%
%
Обобщая критерий Лоренца, Сачестон \cite{sucheston1967banach} доказал, что для любого $x\in\ell_\infty$
и любого $B\in\mathfrak{B}$
\begin{equation*}
	p(x) = \varliminf_{n\to\infty} \frac{1}{n} \sum_{k=m+1}^{k=m+n} x_k
	\leq
	Bx
	\leq
	\varlimsup_{n\to\infty} \frac{1}{n} \sum_{k=m+1}^{k=m+n} x_k
	= q(x)
\end{equation*}
и, более того,
$
	\mathfrak{B}x = [p(x), q(x)]
	.
$

Таким образом, на вопрос: <<Насколько не сходится последовательность?>>~---
можно давать ответ в терминах почти сходимости, т.е. принадлежности пространству $ac$,
а на вопрос: <<Насколько почти не сходится последовательность?>>~---
назвать длину отрезка $[p(x), q(x)]$.
В дальнейшем пространство почти сходящихся последовательностей неоднократно становилось предметом
различных исследований
\cite{semenov2006space,usachev2008transformations}.
В частности, в работе~\cite{connor1990almost} доказано,
что последовательность из нулей и единиц почти наверное не принадлежит пространству $ac$.
Этот факт демонстрирует, что почти сходящиеся последовательности <<достаточно редки>>.

%TODO: ссылки! Хватит или ещё?

Актуальность работы обусловлена тем, что
банаховы пределы нашли своё применение в приложениях
\cite{semenov2015banachtraces,semenov2009fourier,strukova2015spectres}.

%TODO: ссылки! Хватит или ещё?


\paragraph{Основная часть.}
В главе 1
%TODO: \ref ???
изучается $\alpha$--функция.
%
Поскольку $\alpha(c)=0$,
то $\alpha$--функцию также можно считать <<мерой несходимости>> последовательности;
равенство $\alpha(x) = 0$, однако, вовсе не гарантирует сходимость.

Устанавливается, что $\alpha$--функция не инвариантна относительно оператора сдвига $T$,
и даётся оценка на $\alpha(T^n x)$, показывающая,
что эта неинвариантность в некотором смысле однородна.
С другой стороны, $\alpha$--функция, в отличие от некоторых банаховых пределов
\cite{Semenov2010invariant,Semenov2011dan},
инвариантна относительно операторов растяжения $\sigma_n$.
Рассмотрены и другие свойства $\alpha$--функции, в частности,
её суперпозиция с оператором Чезаро и семейством операторов сжимающего усреднения $\sigma_{1/n}$.

Далее изучаются множества, определяемые с помощью $\alpha$--функции, в частности,
пространство $\alpha c = \{x\in\ell_\infty: \alpha(x)=0\}$.
Установлена несепарабельность данного пространства и его замкнутость относительно оператора суперпозиции
(покоординатного умножения последовательностей), операторов левого и правого сдвига,
растяжения и усредняющего сжатия, а также оператора Чезаро.

В главе 2 обсуждается пространство $ac$ и его подпространство $ac_0$,
даётся критерий почти сходимости к нулю (т.е. принадлежности пространству $ac_0$)
знакопоcтоянной последовательности.
Для элементов пространства почти сходящихся последовательностей $ac $
установлена двусторонняя оценка на расстояние до пространства сходящихся последовательностей $c$,
использующая $\alpha$--функцию.

Примечательно (хотя и ожидаемо), что наряду с трансляционно неинвариантной $\alpha$--функцией
в этой оценке используется и функционал $\lim_{n\to\infty}\alpha(T^n x)$,
который, очевидно, трансляционно инвариантен.
Это вполне логично, поскольку расстояние до пространства $c$ и почти сходимость
сами суть трансляционно инвариантные характеристики.

Вычисляется мера одного специального множества последовательностей,
состоящих из нулей и единиц, используемого в приложениях
(данный результат получен совместно с А.С. Усачёвым).




\paragraph{Заключение.}
Итак, в работе исследованы такие асимптотические характеристики ограниченных последовательностей,
как $\alpha$--функция (глава 1),  почти сходимость (глава 2) и банаховы пределы (глава 3).


По итогам исследований, результаты которых составили данную работу,
опубликованы тезисы \cite{our-vvmsh-2018,our-vzms-2018,our-ped-2018-inf-dim-ker,our-ped-2018-alpha-Tx},
планируется публикация ещё нескольких печатных работ.

Выдвинут ряд гипотез,
работу над доказательством или опровержением которых планируется продолжать в дальнейшем.

\printbibliography{}



\end{document}

