В работе исследованы такие асимптотические характеристики ограниченных последовательностей,
как $\alpha$--функция (глава 1) и  почти сходимость (глава 2).% и банаховы пределы (глава 3).

Несмотря на то, что $\alpha$--функция оказалась трансляционно неинвариантной,
эта неинвариантность в некотором смысле однородна (см. следствие \ref{thm:est_alpha_Tn_x_full}).
Для элементов пространства почти сходящихся последовательностей $ac $
установлена двусторонняя оценка на расстояние до пространства сходящихся последовательностей $c$,
использующая $\alpha$--функцию.
Примечательно (хотя и ожидаемо), что наряду с трансляционно неинвариантной $\alpha$--функцией
в этой оценке используется и функционал $\lim_{n\to\infty}\alpha(T^n x)$,
который, очевидно, трансляционно инвариантен.
Это вполне логично, поскольку расстояние до пространства $c$ и почти сходимость
сами суть трансляционно инвариантные характеристики.




По итогам исследований, результаты которых составили данную работу,
опубликованы тезисы \cite{our-vvmsh-2018,our-vzms-2018,our-ped-2018-inf-dim-ker,our-ped-2018-alpha-Tx},
а также краткое сообщение~\cite{our-mz2019ac0};
планируется публикация ещё нескольких печатных работ.

Выдвинут ряд гипотез,
работу над доказательством или опровержением которых планируется продолжать в дальнейшем.
