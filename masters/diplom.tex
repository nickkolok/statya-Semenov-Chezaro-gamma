\documentclass[12pt,a4paper,openbib]{report}
\usepackage{amsmath}
\usepackage[utf8]{inputenc}
\usepackage[english,russian]{babel}
\usepackage{amsfonts,amssymb}
\usepackage{latexsym}
\usepackage{euscript}
\usepackage{enumerate}
\usepackage{graphics}
\usepackage[dvips]{graphicx}
\usepackage{geometry}
\usepackage{wrapfig}
\usepackage[colorlinks=true,allcolors=black]{hyperref}
\usepackage{bbm}
\usepackage{enumitem}
\usepackage{mathrsfs}


% https://tex.stackexchange.com/questions/634509/show-hide-thumbnail-sidebar-by-default-in-pdf
\hypersetup{pdfpagemode=UseNone}


\righthyphenmin=2

%\usepackage[14pt]{extsizes}

\geometry{left=2.5cm}% левое поле
\geometry{right=1cm}% правое поле
\geometry{top=2cm}% верхнее поле
\geometry{bottom=2cm}% нижнее поле

\renewcommand{\baselinestretch}{1.3}

\renewcommand{\leq}{\leqslant}
\renewcommand{\geq}{\geqslant} % И делись оно всё нулём!

\DeclareMathOperator{\ext}{ext}
\DeclareMathOperator{\mes}{mes}
\DeclareMathOperator{\supp}{supp}

\newcommand{\N}{\ensuremath{\mathbb{N}}}
\newcommand{\Q}{\ensuremath{\mathbb{Q}}}
\newcommand{\B}{\ensuremath{\mathfrak{B}}}
\newcommand{\Iac}{\mathcal{I}(ac_0)}
\newcommand{\Dac}{\mathcal{D}(ac_0)}


\newcommand{\longcomment}[1]{}

\usepackage[backend=biber,style=gost-numeric,sorting=none]{biblatex}
\addbibresource{../bib/Semenov.bib}
\addbibresource{../bib/my.bib}
\addbibresource{../bib/ext.bib}
\addbibresource{../bib/classic.bib}
\addbibresource{../bib/general_monographies.bib}
\addbibresource{../bib/Bibliography_from_Usachev.bib}

\input{../bib/ext.hyphens.bib}

\usepackage{amsthm}
\theoremstyle{definition}
\newtheorem{lemma}{Лемма}[section]
\newtheorem{theorem}[lemma]{Теорема}
\newtheorem{example}[lemma]{Пример}
\newtheorem{property}[lemma]{Свойство}
\newtheorem{remark}[lemma]{Замечание}
\newtheorem{definition}[lemma]{Определение}
\newtheorem{proposition}[lemma]{Утверждение}
\newtheorem{corollary}{Следствие}[lemma]

\newtheorem{hhypothesis}[lemma]{Гипотеза}


\newcommand\hypotlist{ }
\newcounter{hypcount}

\makeatletter
\usepackage{environ}
\NewEnviron{hypothesis}{%

	\edef\curlabel{hhypothesis\thehypcount}
    \begin{hhypothesis}
		\label{\curlabel}
		\BODY%
    \end{hhypothesis}
	\edef\curref{\noexpand\ref{\curlabel}}

	\expandafter\g@addto@macro\expandafter\hypotlist\expandafter
	{\paragraph{Гипотеза\!\!\!}}


	\expandafter\g@addto@macro\expandafter\hypotlist\expandafter
	{\expandafter\textbf\expandafter{\curref}}

	\expandafter\g@addto@macro\expandafter\hypotlist\expandafter
	{\textbf{.}~~}

	\expandafter\g@addto@macro\expandafter\hypotlist\expandafter
	{\BODY}

	\addtocounter{hypcount}{1}
}
\makeatother

%Only referenced equations are numbered
\usepackage{mathtools}
\mathtoolsset{showonlyrefs}

%\mathtoolsset{showonlyrefs=false}
% (an equation/multline to be force-numbered)
%\mathtoolsset{showonlyrefs=true}

% https://superuser.com/questions/517025/how-can-i-append-two-pdfs-that-have-links
\usepackage{pdfpages}% http://ctan.org/pkg/pdfpages

\begin{document}
\clubpenalty=10000
\widowpenalty=10000
\includepdf{title.pdf}
\setcounter{page}{2}
\tableofcontents

\chapter*{Введение}
\addcontentsline{toc}{chapter}{Введение}
Сходящиеся последовательности, т.е. последовательности, имеющие предел в смысле классического математического анализа,
изучены достаточно хорошо.
В частности, любая сходящаяся последовательность является ограниченной.
Пространство ограниченных последовательностей будем, вслед за классиками \cite{wojtaszczyk1996banach,lindenstrauss1973classical},
обозначать через $\ell_\infty$ и снабжать его нормой
\begin{equation*}
	\|x\| = \sup_{n\in\mathbb{N}}|x_n|
	.
\end{equation*}

Однако в приложениях часто возникают ограниченные последовательности,
которые не являются сходящимися.
В таком случае возникает закономерный вопрос:
как измерить <<недостаток сходимости>>?
<<насколько не сходится>> последовательность?

Наиболее очевидным кажется вычисление расстояния $\rho(x,c)$ от заданного элемента $x\in\ell_\infty$
до пространства сходящихся последовательностей $c$
(которое равно половине разности верхнего и нижнего пределов последовательности).
Однако выясняется, что имеют место быть и другие подходы.

Нетрудно заметить, что операция взятия классического предела на пространстве сходящихся последовательностей
является непрерывным (в норме $\ell_\infty$) линейным функционалом.
Банах доказал \cite{banach1993theorie}, что этот функционал может быть непрерывно продолжен на всё пространство $\ell_\infty$.
На основе этой идеи были определены банаховы пределы
(иногда также называемые пределами Банаха--Мазура \cite{alekhno2012superposition,alekhno2015banach})
следующим образом.

Банаховым пределом называется функционал $B\in \ell_\infty^*$ такой, что:
\begin{enumerate}
	\item
		$B \geqslant 0$
	\item
		$B\one = 1$
	\item
		$B=BT$
\end{enumerate}

Простейшие свойства:
\begin{itemize}
	\item
		$\|B\|_{\ell_\infty^*} = 1$
	\item
		$Bx = \lim\limits_{n\to\infty} x_n$ для любого $x=(x_1, x_2, ...) \in c$.

		Таким образом,
		банахов предел~--- действительно естественное обобщение понятия предела сходящейся последовательности
		на все ограниченные последовательности.
\end{itemize}

Множество банаховых пределов обычно обозначают через $\mathfrak{B}$
(реже через $BM$~--- см., например, \cite{alekhno2012superposition,alekhno2015banach}).

Лоренц \cite{lorentz1948contribution} установил, что существует подпространство $\ell_\infty$,
на котором все банаховы пределы принимают одинаковое значение.
Это пространство названо пространством почти сходящихся последовательностей и обычно обозначается $ac$
(от англ. <<almost convergent>>).
Включение $c \subset ac$ собственное, т.е. $ac \setminus c \neq \varnothing$.


Обобщая критерий Лоренца, Сачестон \cite{sucheston1967banach} доказал, что для любого $x\in\ell_\infty$
и любого $B\in\mathfrak{B}$
\begin{equation*}
	q(x) =
	\lim_{n\to\infty} \sup_{m\in\mathbb{N}} \frac{1}{n} \sum_{k=m+1}^{k=m+n} x_k
	\leq
	Bx
	\leq
	\lim_{n\to\infty} \inf_{m\in\mathbb{N}} \frac{1}{n} \sum_{k=m+1}^{k=m+n} x_k
	= p(x)
\end{equation*}
и, более того,
\begin{equation*}
	\mathfrak{B}x = [q(x), p(x)]
	.
\end{equation*}

Таким образом, на вопрос: <<Насколько не сходится последовательность?>> %~---
можно давать ответ в терминах почти сходимости, т.е. принадлежности пространству $ac$,
а на вопрос: <<Насколько почти не сходится последовательность?>>~---
назвать длину отрезка $[q(x), p(x)]$.
В дальнейшем пространство почти сходящихся последовательностей неоднократно становилось предметом
различных исследований
\cite{semenov2006space,usachev2008transformations}.
В частности, в работе~\cite{connor1990almost} доказано,
что последовательность из нулей и единиц почти наверное не принадлежит пространству $ac$.
Этот факт демонстрирует, что почти сходящиеся последовательности <<достаточно редки>>.

%TODO: ссылки! Хватит или ещё?

Банаховы пределы также нашли своё применение в приложениях
\cite{semenov2015banachtraces,semenov2009fourier,strukova2015spectres}.

%TODO: ссылки! Хватит или ещё?


В настоящей работе рассматриваются некоторые вопросы асимптотических характеристик ограниченных последовательностей.

В главе 1
%TODO: \ref ???
изучается $\alpha$--функция, введённая в~\cite{our-vzms-2018}.
%
%TODO: ссылка на статью Семенова!
%
Поскольку $\alpha(c)=0$,
то $\alpha$--функцию также можно считать <<мерой несходимости>> последовательности;
равенство $\alpha(x) = 0$, однако, вовсе не гарантирует сходимость.

Устанавливается, что $\alpha$--функция не инвариантна относительно оператора сдвига $T$,
и даётся оценка на $\alpha(T^n x)$.
С другой стороны, $\alpha$--функция, в отличие от некоторых банаховых пределов
\cite{Semenov2010invariant,Semenov2011dan},
инвариантна относительно операторов растяжения $\sigma_n$.
Рассмотрены и другие свойства $\alpha$--функции.

В главе 2 обсуждается пространство $ac$ и его подпространство $ac_0$,
даётся критерий почти сходимости к нулю (т.е. принадлежности пространству $ac_0$)
знакопоcтоянной последовательности.
Затем выявляется связь между $\alpha$--функцией, расстоянием от элемента до пространства $c$
и почти сходимостью.


Как сказано выше, банаховы пределы по определению (как и обычный предел на пространстве $c$) инвариантны относительно оператора сдвига.
Возникает закономерный вопрос: можно ли потребовать от банахова предела сохранять своё значение
при суперпозиции с некоторыми другими операторами на $\ell_\infty$?
Первым эту проблему исследовал У. Эберлейн в 1950 г. \cite{Eberlein},
т.е. через два года после классической работы Г. Г. Лоренца~\cite{lorentz1948contribution}.
Эберлейн установил, что существуют такие линейные операторы  $A : \ell_\infty\to \ell_\infty$,
для которых $BAx = Bx$ независимо от выбора $x$ и для банаховых пределов специального вида.

Будем говорить, что $B\in\mathfrak B(A)$, $A : \ell_\infty\to \ell_\infty$, если для любого $x\in \ell_\infty$
выполнено равенство $BAx = Bx$.
Такой банахов предел $B$ называют инвариантным относительно оператора $A$.

Можно ли выделить какие-то особые свойства оператора сдвига,
которые необходимы или достаточны оператору, чтобы относительно него были инвариантны все или некоторые банаховы пределы?
Понятно, что если оператор $A$ таков, что для любого $x\in\ell_\infty$ между $Ax$ и $x$
существует (конечное) расстояние Дамерау--Левенштейна \cite{damerau1964technique} (т.е. минимальное количество операций вставки, удаления, замены и перестановки двух соседних элементов последовательности, необходимых для перевода $x$ в $Ax$, причём для разных $x\in\ell_\infty$ эти операции, вообще говоря, не обязаны быть одинаковыми), то относительно данного оператора инвариантен любой банахов предел. Аналогичное утверждение справедливо и в случае, если $Ax -x \in c_0$ для любого $x\in \ell_\infty$.

Следующим по естественности (после сдвига и замены конечного числа элементов) действием, сохраняющем сходимость последовательности, является повторение элементов последовательности, например, оператор
\begin{equation}
	\sigma_2(x_1,x_2,x_3,...) = (x_1,x_1, \; x_2, x_2, \; x_3, x_3, \; ...)
	.
\end{equation}
Недавно [TODO: ссылка] было выяснено, что относительно такого оператор инвариантны не все, а только некоторые банаховы пределы.
Заметим, что если мы рассмотрим оператор неравномерного растяжения
\begin{equation}
	\sigma_{1,2}(x_1,x_2,x_3,x_4,x_5,...) = (x_1, \; x_2, x_2, \;  x_3, \; x_4, x_4, \; x_5, ...)
	,
\end{equation}
то увидим, что периодическую последовательность $y_n = (-1)^n$, $y\in ac_0$ оператор $\sigma_{1,2}$
переводит в периодическую последовательность
\begin{equation}
	(-1, 1, 1, \; -1, 1, 1, \; ...) \in ac_{1/3}
	,
\end{equation}
поскольку на периодической последовательности любой банахов предел принимает значение, равное среднему по периоду.
Таким образом, не сущетвует банаховых пределов, инвариантных относительно оператора $\sigma_{1,2}$.

В главе 3 изложены некоторые примеры операторов и найдены множества банаховых пределов,
инвариантных относительно этих операторов.


\chapter{$\alpha$--функция как асимптотическая характеристика ограниченной последовательности}

	\section{Определение и элементарные свойства $\alpha$--функции}
	На пространстве $\ell_\infty$ определяется $\alpha$--функция следующим равенством:
\begin{equation}
	\alpha(x) = \varlimsup_{i\to\infty} \max_{i<j\leqslant 2i} |x_i - x_j|
	.
\end{equation}

Эта функция является полунормой на $\ell_\infty$
и естественным образом возникла в работе~\cite[\S 2]{semenov2020invariant_noncommutative}
при исследовании свойств суперпозиции оператора Чезаро $C$ и банаховых пределов.
Как выяснилось, эта полунорма обладает достаточно интересными свойствами сама по себе.

Иногда удобнее использовать одно из нижеследующих равносильных определений:
\begin{equation}
	\alpha(x) = \varlimsup_{i\to\infty} \max_{i \leqslant j\leqslant 2i} |x_i - x_j|
	,
\end{equation}
\begin{equation}
	\alpha(x) = \varlimsup_{i\to\infty} \sup_{i<j\leqslant 2i} |x_i - x_j|
	,
\end{equation}
\begin{equation}
	\alpha(x) = \varlimsup_{i\to\infty} \sup_{i \leqslant j\leqslant 2i} |x_i - x_j|
	.
\end{equation}

Легко видеть, что $\alpha$--функция неотрицательна.
\begin{property}
	\label{thm:alpha_x_triangle_ineq}
	Более того, $\alpha$--функция удовлетворяет неравенству треугольника:
	\begin{equation}
		\alpha(x+y) \leq \alpha(x) + \alpha(y)
		.
	\end{equation}
\end{property}
%TODO2: доказывать нужно или очевидно?

\begin{property}
	На пространстве $\ell_\infty$ $\alpha$--функция удовлетворяет условию Липшица:
	\begin{equation}\label{alpha_Lipshitz}
		|\alpha(x) - \alpha(y)| \leq 2 \|x-y\|
		.
	\end{equation}
	(и эта оценка точна).
%	TODO2: доказывать нужно или очевидно?
\end{property}

\begin{property}
	Если $y\in c$, то $\alpha(y) = 0$ и $\alpha(x+y) = \alpha(x)$ для любого $x \in \ell_\infty$.
\end{property}

Кроме того, иногда полезно помнить про следующее очевидное
\begin{property}
	\label{thm:alpha_x_leq_limsup_minus_liminf}
	\begin{equation}
		\alpha(x) \leq \varlimsup_{k\to\infty} x_k - \varliminf_{k\to\infty} x_k
		.
	\end{equation}
\end{property}

Отдельный интерес представляет множество
\begin{equation}
	A_0 = \{x\in\ell_\infty : \alpha(x) = 0\}
	.
\end{equation}

Ниже в этой главе мы покажем, что оно является подпространством $\ell_\infty$ и обладает рядом интересных свойств.
Одно из этих свойств доказывается уже не в настоящей главе, в а теореме~\ref{thm:A_0_c_infty_lin}.


	\section{$\alpha$--функция и оператор сдвига $T$}
	\input{../alpha_Tx_leq_alpha_x.tex}

	\section{О характере сходимости последовательности $\alpha(T^n x)/ \alpha(x)$}
	\input{../alpha_Tx_frac_alpha_x.tex}

	\section{О множествах $\{x: \alpha(T^n x) = \alpha(x)\}$}
	\input{../alpha_Tx_equiv_alpha_x.tex}

	\section{$\alpha$--функция и семейство операторов $\sigma_n$}
	\input{../alpha_sigma_n.tex}

	\section{$\alpha$--функция и семейство операторов $\sigma_{1/n}$}
	\input{../alpha_sigma_1_n.tex}

	\section{$\alpha$--функция и оператор Чезаро $C$}
	\input{../alpha_Cx.tex}

	\section{$\alpha$--функция и оператор суперпозиции}
	\input{../alpha_xy.tex}

	\section{Функционалы $\alpha^*$ и $\alpha_*$}
	\input{../alpha_star.tex}

	\section{Пространство $\{x: \alpha(x) = 0\}$}
	\label{sec:space_A0}

Из свойства~\ref{thm:alpha_x_triangle_ineq} и однородности $\alpha$--функции немедленно вытекает
\begin{theorem}
	\label{thm:A0_is_space}
	Множество $A_0 = \{x: \alpha(x) = 0\}$
	является пространством.
\end{theorem}
Изучим свойства этого пространства.
\begin{property}
	Пространство $A_0$ замкнуто.
\end{property}
\begin{proof}
	Прообраз замкнутого множества $\{0\}$
	при непрерывном отображении $\alpha : \ell_\infty \to \R$
	замкнут.
\end{proof}
\begin{theorem}
	Включение $c \subset A_0$ собственное.
\end{theorem}
\begin{proof}
	Рассмотрим
	\begin{equation}
		x=\left(
			0,1,
			0,\frac{1}{2},1,\frac{1}{2},
			0,\frac{1}{3},\frac{2}{3},1,\frac{2}{3},\frac{1}{3},
			0,
			...
		\right)
		.
	\end{equation}
	Тогда по лемме~\ref{thm:alpha_S} (при определённом тем же образом операторе $S$) имеем
	\begin{equation}
		\alpha(Sx) = \varlimsup_{k\to\infty} |x_{k+1} - x_{k}| = 0
		,
	\end{equation}
	однако, очевидно, $Sx\notin c$.
\end{proof}
Очевидно следующее
\begin{property}
	Если периодическая последовательность принадлежит $A_0$,
	то эта последовательность~--- константа.
\end{property}

Из результатов предыдущих пунктов вытекает
\begin{theorem}
	Пространство $A_0$ замкнуто относительно операторов $T$, $U$, $C$, $\sigma_n$, $\sigma_{1/n}$, $S$.
\end{theorem}

Из теоремы~\ref{thm:alpha_xy} следует
\begin{theorem}
	Пространство $A_0$ замкнуто относительно умножения,
	т.е. если $x,y\in A_0$, то $x\cdot y \in A_0$.
\end{theorem}

Пример~\ref{ex:alpha-c_not_ideal} показывает, что $A_0$
не является идеалом по умножению.

Из теоремы~\ref{thm:rho_x_c_leq_alpha_t_s_x_united} следует
\begin{theorem}
	$ac \cap A_0 = c$.
\end{theorem}

Некоторый интерес представляет следующая теорема,
основанная на идеях~\cite{usachev2009_phd_vsu}.

\begin{theorem}
	Пространство $A_0$ несепарабельно.
\end{theorem}

\begin{proof}
	Пусть $\Omega = \{0;1\}^\N$.
	Для каждого $\omega\in\Omega$ положим
	\begin{multline}
		\label{eq:x_omega_alpha_c}
		x(\omega)=\left(
			0, 1\omega_1,
			0, \frac{1}{2}\omega_2, 1\omega_2, \frac{1}{2}\omega_2,
			0, \frac{1}{3}\omega_3, \frac{2}{3}\omega_3, 1\omega_3, \frac{2}{3}\omega_3, \frac{1}{3}\omega_3,
			0, ...,
		\right. \\ \left.
			0, \frac{1}{p}\omega_p, \frac{2}{p}\omega_p, ..., \frac{p-1}{p}\omega_p, 1\omega_p,
				\frac{p-1}{p}\omega_p, ..., \frac{2}{p}\omega_p, \frac{1}{p}\omega_p,
			0, \frac{1}{p+1}\omega_{p+1}, ...
		\right).
	\end{multline}
	Тогда по лемме~\ref{thm:alpha_S} (при определённом тем же образом операторе $S$) имеем
	$\alpha(Sx(\omega)) = 0$.
	Заметим, что при $\omega,\omega^* \in \Omega$ и $\omega\neq\omega^*$ выполнено
	$\|Sx(\omega)-Sx(\omega^*)\|=1$ и $|Sx(\Omega)|=\mathfrak{c}$.
	Следовательно, $A_0$ несепарабельно.
\end{proof}

%\begin{theorem}
%	$|A_0| = |\ell_\infty|$.
%\end{theorem}

%\begin{proof}
%	Достаточно заметить, что
%\end{proof}


	\section{О недополняемости некоторых вложений}
	В работе~\cite{phillips1940linear} Филлипс доказал весьма неожиданный (для своего времени) результат:
простанство $c_0$ недополняемо в $\ell_\infty$.
Говоря формально, справедлива следующая


\begin{theorem}[Филлипса]
	\label{thm:phillips}
	Не существует непрерывного линейного оператора $P: \ell_\infty \to c_0$ такого, что для любого
	$x \in c_0$ выполнено равенство $Px =x$.
\end{theorem}
Подобные операторы называются \emph{проекторами}.

Теорема~\ref{thm:phillips} была первым примером недополняемого вложения пространств.
Позже были найдены и другие примеры;
%TODO: краткий обзор - передрать из Conway 2nd ed., p. 94
отсылаем читателя, например, к~\cite{lindenstrauss1979classical}.

Е.А. Алехно привёл~\cite[Theorem 8]{alekhno2006propertiesII} элегентное доказательство
того, что $ac_0$ недополняемо в $\ell_\infty$.
Это доказательство основано на изначальном доказательстве Филлипса теоремы~\ref{thm:phillips}
и использует некоторые леммы из~\cite{phillips1940linear}.

Вложение $c_0 \subset ac_0$ также недополняемо.
Непосредственное явное упоминание этого факта найти не удалось, однако он достаточно легко следует из~\cite[теорема 4]{ASSU2},
доказательство которой опирается на идеи~\cite{whitley1968projecting}~и~\cite[Theorem 6.9]{Carothers}.
Приведём эту теорему (в несколько ослабленной форме, для которой достаточно уже введённой терминологии)
и соотвествующее следствие, предварив одной вспомогательной леммой, доказательство коей является столь же классическим, сколь и кратким,
и приводится здесь исключительно ради полноты изложения.

\begin{definition}
	Семейство множеств $\{A_\lambda\}_{\lambda \in \Lambda}$ называется почти дизъюнктным,
	если для $\lambda, \mu \in \Lambda$, $\lambda \ne \mu$,
	пересечение $A_\lambda \cap A_\mu$ конечно.
\end{definition}

\begin{lemma}
	\label{lem:uncountable_subsets_of_N_with_finite_intersections}
	Существует почти дизъюнктное несчётное семейство подмножеств $\{S_i\}_{i\in I}$, $S_i \subset \mathbb{N}$.
\end{lemma}

\begin{proof}
	Рассмотрим биекцию $\N \leftrightarrow \Q$.
	Пусть $I = \R$.
	Для каждого $i\in I$ положим $S_i = \{q_n\}$,
	где $\{q_n\} \subset \mathbb{Q}$ "--- некоторая последовательность рациональных чисел,
	сходящаяся к $i$.
	%(Proof: switch from $\mathbb{N}$ to $\mathbb{Q}$, pick convergent sequences to irrationals.
	%More details in [this question](https://math.stackexchange.com/q/162387).)
\end{proof}

\begin{theorem}
	\label{thm:Alekhno_noncomplementarity_general}
	Пусть $X$ и $Y$ "--- такие линейные подпространства в $\ell_\infty$,
	что $c_{00} \subseteq Y \subsetneq X \subseteq \ell_\infty$.
	Пусть существует такое несчётное почти дизъюнктное семейство множеств натуральных чисел $\{A_\lambda\}_{\lambda \in \Lambda}$,
	$A_\lambda \subseteq \N$, что для любого $\lambda \in \Lambda$ имеет место включение $\chi_{A\lambda} \in X \setminus Y$.
	Тогда вложение $Y \subset X$ недополняемо.
\end{theorem}

\begin{corollary}
	Вложение $c_0\subsetneq ac_0$ недополняемо.
	Вложение $c_0\subsetneq \Iac$ недополняемо.
\end{corollary}

\begin{proof}
	Возьмём почти дизъюнктное семейство $\{S_i\}_{i\in I}$, $S_i \subset \mathbb{N}$
	из леммы~\ref{lem:uncountable_subsets_of_N_with_finite_intersections}
	и построим новое семейство $\{A_i\}_{i\in I}$, $A_i \subset \mathbb{N}$
	по правилу
	\begin{equation}
		\label{eq:c0_noncomplemented_in_ac0_set_family}
		A_i = \{ 2^n : n\in S_i \}
		.
	\end{equation}
	Легко видеть, что семейство $\{A_i\}_{i\in I}$ снова является почти дизъюнктным.
	Более того, $\chi_{A_i} \in ac_0 \setminus c_0$ и $\chi_{A_i} \in \Iac \setminus c_0$ для любого $i\in I$.

	Таким образом, выполнены все условия теоремы~\ref{thm:Alekhno_noncomplementarity_general}
	для цепочки вложений $c_{00} \subseteq c_0 \subsetneq ac_0 \subseteq \ell_\infty$ и
	для цепочки вложений $c_{00} \subseteq c_0 \subsetneq \Iac \subseteq \ell_\infty$.
\end{proof}

%TODO: а что там с \Iac \subset ac_0 ?

Итак, все три вложения в цепочке
\begin{equation}
	c_0 \subset \ell_\infty,
	\quad
	ac_0 \subset \ell_\infty,
	\quad\mbox{и}\quad
	c_0 \subset ac_0,
\end{equation}
недополняемы.

Перейдём теперь к изучению пространства $A_0$.

\begin{lemma}
	Вложение $A_0 \subset \ell_\infty$ недополняемо.
\end{lemma}

\begin{proof}
	В теореме~\ref{thm:Alekhno_noncomplementarity_general}
	положим $Y=A_0$, $X=\ell_\infty$.
	Тогда для сесмества множеств~\eqref{eq:c0_noncomplemented_in_ac0_set_family}
	выполнены все условия теоремы~\ref{thm:Alekhno_noncomplementarity_general}.
\end{proof}

Однако для вложения $c_0 \subset A_0$ применить теорему~\ref{thm:Alekhno_noncomplementarity_general}
непосредственно не удастся.
Осноная проблема заключается в том, что для любого бесконечного множества $F \subset \N$ такого, что
дополнение $\N \setminus F$ также бесконечно, последовательность $\chi_F \notin A_0$,
поскольку $\alpha(F) = 1$.

Поэтому мы проведём доказательство недополняемости полностью,
во многом опираясь на идеи~\cite{whitley1968projecting} и дискуссию~\cite{mathSE_Phillips}




In Sections~\ref{sec:A0_in_ell_infty} and~\ref{sec:c0_in_A0}
we proceed with applying the same approach to the inclusion chain $c_0 \subset A_0 \subset \ell_\infty$,
where
\begin{equation}
	A_0 = \{x\in\ell_\infty: \alpha(x) = 0\}
	,
\end{equation}
\begin{equation}
	\alpha(x) = \varlimsup_{i\to\infty} \max_{i<j\leqslant 2i} |x_i - x_j|
	.
\end{equation}



\section{The subspace $c_0$ is not complemented in $ac_0$}
\label{sec:c0_in_ac0}

The following lemma is a classical result, and we prove it for the sake of completeness.
\begin{lemma}
	\label{lem:uncountable_subsets_of_N_with_finite_intersections}
	There exists an uncountable family of subsets
	$\{S_i\}_{i\in I}$, $S_i \subset \N$,
	such that each $S_i$ is countable and for every $i\neq j$ the intersection $S_i \cap S_j$ is finite.
\end{lemma}

\begin{proof}
	Consider a bijection $\N \leftrightarrow \Q$
	and let $I = \R$.
	For $i\in I$ we set $S_i = \{q_n\}$,
	where $\{q_n\} \subset \Q$ is a sequence which converges monotonically to $i$.
\end{proof}








The following lemma is inspired by~\cite{mathSE_Phillips}.

\begin{lemma}
	\label{lem:c_0_not_complemented_in_ac_0}
	For each linear operator $Q: ac_0 \to ac_0$ such that $c_0\subseteq \ker Q$,
	there exists infinite subset $S \subset \N$ such that
	\begin{equation}
		\forall(x \in ac_0 : \supp x \subset S)[Qx = 0]
		.
	\end{equation}
\end{lemma}



\begin{proof}
	Notice first that for any infinite subset $S \subset \N$
	there exists $x\in ac_0\setminus c_0$ with $\supp x \subseteq S$.
	Indeed, we can find such $x$ with $0 < x \leq \chi_S$
	that contains infinitely many ones, and the distance between the ones
	is enough for Lorentz's criterion~\eqref{eq:crit_Lorentz} to be hold.



	Let $\{S_i\}_{i \in I}$ be a family of subsets of $\N$
	that satisfy the conditions of Lemma~\ref{lem:uncountable_subsets_of_N_with_finite_intersections}.
	Suppose to the contrary that
	\begin{equation}
		\forall(\mbox{infinite }S\subset\N)\exists(x \in ac_0 : \supp x \subset S)[Qx \neq 0]
		.
	\end{equation}
	In particular,
	\begin{equation}
		\forall(i\in I)\exists(x_i \in ac_0 : \supp x_i \subset S_i)[Q(x_i) \neq 0]
		.
	\end{equation}

	Note that $x_i \notin c_0$ because $c_0\subseteq \ker Q$.
	Without loss of generality we can assume that $\|x_i\|=1$ for all $i \in I$.

	Consider $I_n = \{i \in I\,:\,(Qx_i)_n \neq 0\}$,
	then $I = \bigcup\limits_{n\in\N} I_n$.
	Thus, we can find $n$ such that $I_n$ is also uncountable
	(otherwise $I$ would be countable as a countable union of countable sets,
	which contradicts to conditions of Lemma~\ref{lem:uncountable_subsets_of_N_with_finite_intersections}).

	Сonsider now $I_{n,k} = \{i \in I_n\,:\,|(Qx_i)_n| \geq 1/k\}$,
	then $I_n = \bigcup\limits_{k\in\N} I_{n,k}$.
	Applying the same argument as above, one can easily see that the set $I_{n,k}$ is uncountable for some $k$.
	Let us choose such $I_{n,k}$ and proceed with it.

	So, we have an uncountable set $I_{n,k}$ and
	\begin{equation}
		\forall(i\in I_{n,k})\exists(x_i \in ac_0 : \supp x_i \subset S_i)\Bigl[\|x_i\|=1 \mbox{~~and~~} |(Qx_i)_n| \geq 1/k\Bigr]
		.
	\end{equation}

	Consider a finite set $J \subset I_{n,k}$ with $\#J>1$
	(here $\#J$ stands for the cardinality of the set $J$).
	Take
	\begin{equation}
		y = \sum_{j \in J} \operatorname{sign}{(Qx_j)_n} \cdot x_j
		.
	\end{equation}
	Since the intersection $S_i \cap S_j$ is finite for any $i \neq j$ and
	$\supp x_j \subset S_j$,
	the intersection $\bigcap\limits_{j\in J} \supp x_j$ is also finite.
	Hence, $y = f + z$,
	where $\supp f$ is finite and $\|z\| \leq 1$.

	On the other hand,
	\begin{equation}
		\label{eq:non_complemented_sum_cardinality}
		(Qy)_n = \sum_{j \in J}
		(\operatorname{sign}(Qx_j)_n)
		\cdot (Qx_j)_n \geq \frac{\# J}{k}
		.
	\end{equation}
	Note, that $f\in c_0$ and we have $Qf = 0$, because $c_0 \subseteq \ker Q$.
	Thus, $Qy = Q(f+z) = Qf + Qz = Qz$ and
	\begin{equation}
		\label{eq:norm_Q_estimate}
		\frac{\# J}{k} \leq (Qy)_n \leq \|Qy\| = \|Qz\| \leq \|Q\| \cdot \|z\| \leq \|Q\|
		.
	\end{equation}
	Due to~\eqref{eq:norm_Q_estimate}, we obtain $\# J \leq \|Q\| k$ for every $J\subset I_{n,k}$.
	This contradicts the fact that $I_{n,k}$ is uncountable,
	and we are done.
\end{proof}

\begin{theorem}
	The subspace $c_0$ is not complemented in $ac_0$.
\end{theorem}

\begin{proof}
	Suppose to the contrary that
	there exists a continuous projection $P: ac_0 \to c_0$.
	Applying Lemma~\ref{lem:c_0_not_complemented_in_ac_0} to $I-P$
	we can find infinite subset $S\subset\N$
	such that $\forall(x\in ac_0 : \supp x \subset S)[(I-P)x = 0]$
	(such $x \in ac_0 \setminus c_0$ exists even if $\chi_S \notin ac_0$).
	But then $Px\notin c_0$,
	which contradicts the fact that $P$ is a projection onto $c_0$.
\end{proof}







\section{The subspace $c_0$ is not complemented in $A_0$}
\label{sec:c0_in_A0}

To prove the fact, we need some auxiliary constructions from~\cite{our-vzms-2018}.


Let us define linear operator $F:\ell_\infty \to \ell_\infty$ as following:
\begin{equation}
	\label{operator_F}
	(Fy)_k = y_{i+2}, \mbox{ for } 2^i < k \leq 2^i+1
\end{equation}

For example,
$$
	F(\{1,2,3,4,5,6, ...\}) = \{1,2,\,3,3,\,4,4,4,4,\,5,5,5,5,5,5,5,5,\,6...\}
$$


It is easy to see that the equality
\begin{equation}
	\label{eq:alpha_F}
	\alpha(Fx) = \varlimsup_{k\to\infty} |x_{k+1} - x_{k}|
\end{equation}
holds.

Let us also define linear operator $M:\ell_\infty \to \ell_\infty$ as following:
\begin{multline*}
	M(\omega_1,\omega_2,...)=\left(
		0, 1\omega_1,
		0, \frac{1}{2}\omega_2, 1\omega_2, \frac{1}{2}\omega_2,
		0, \frac{1}{3}\omega_3, \frac{2}{3}\omega_3, 1\omega_3, \frac{2}{3}\omega_3, \frac{1}{3}\omega_3,
		0, ...,
	\right. \\ \left.
		0, \frac{1}{p}\omega_p, \frac{2}{p}\omega_p, ..., \frac{p-1}{p}\omega_p, 1\omega_p,
			\frac{p-1}{p}\omega_p, ..., \frac{2}{p}\omega_p, \frac{1}{p}\omega_p,
		0, \frac{1}{p+1}\omega_{p+1}, ...
	\right)
	.
\end{multline*}
Note that due to~\eqref{eq:alpha_F} we have $FM: \ell_\infty \to A_0$.


\begin{lemma}
	\label{lem:c_0_not_complemented_in_A_0}
	For each linear operator $Q: A_0 \to A_0$ such that $c_0\subseteq \ker Q$,
	there exists infinite subset $S \subset \N$ such that
	\begin{equation}
		\forall(x \in A_0 : \supp x \subset S)[Qx = 0]
	\end{equation}
	and $x\in A_0\setminus c_0$ such that $\supp x \subseteq S$.
\end{lemma}

\begin{proof}
	Let $\{U_i\}_{i \in I}$ be a family of subsets of $\N$
	such that conditions of Lemma~\ref{lem:uncountable_subsets_of_N_with_finite_intersections} are hold.
	Let $\{S_i\}_{i \in I}$ be a family of subsets of $\N$
	defined by the equality $S_i = \supp FM\chi_{U_i}$.
	Obviously, the family of sets $\{S_i\}_{i \in I}$ also
	satisfies the conditions of Lemma~\ref{lem:uncountable_subsets_of_N_with_finite_intersections}.
	Moreover, for every $i\in I$ we have $x = FM\chi_{U_i} \in A_0\setminus c_0$.

	Suppose to the contrary that
	\begin{equation}
		\forall(\mbox{infinite }S\subset\N)\exists(x \in A_0 : \supp x \subset S)[Qx \neq 0]
		.
	\end{equation}
	In particular,
	\begin{equation}
		\forall(i\in I)\exists(x_i \in A_0 : \supp x_i \subset S_i)[Q(x_i) \neq 0]
		.
	\end{equation}
	The rest of the proof is similar to that of Lemma~\ref{lem:c_0_not_complemented_in_ac_0}.

\end{proof}

\begin{theorem}
	The subspace $c_0$ is not complemented in $A_0$.
\end{theorem}

\begin{proof}
	Suppose to the contrary that
	there exists a continuous projection $P: A_0 \to c_0$.
	Applying Lemma~\ref{lem:c_0_not_complemented_in_A_0} to $I-P$
	we can find infinite subset $S\subset\N$
	such that $\forall(x\in A_0 : \supp x \subset S)[(I-P)x = 0]$,
	and $x\in A_0 \setminus c_0$ such that $\supp x \subseteq S$.
	But then $Px = x\notin c_0$,
	which contradicts  the fact  that $P$ is a projection onto $c_0$.
\end{proof}




\chapter{Последовательности, почти сходящиеся к нулю (пространство $ac_0$)}

	Почти сходимость является естественным обобщением понятия сходимости.
История исследования почти сходимости начинается с работы Г.Г. Лоренца~\cite{lorentz1948contribution}.
Напомним определение банахова предела.

\begin{definition}
	Линейный функционал $B\in \ell_\infty^*$ называется банаховым пределом,
	если
	\begin{enumerate}
		\item
			$B\geq0$, т.~е. $Bx \geq 0$ для $x \geq 0$,
		\item
			$B\one=1$, где $\one =(1,1,\ldots)$,
		\item
			$B(Tx)=B(x)$ для всех $x\in \ell_\infty$, где $T$~---
		оператор сдвига, т.~е. $T(x_1,x_2,\ldots)=(x_2,x_3,\ldots)$.
	\end{enumerate}
\end{definition}
Множество всех банаховых пределов обозначим через $\mathfrak{B}$.
Существование банаховых пределов было анонсировано С. Мазуром \cite{Mazur} и позднее доказано в книге С.~Банаха~\cite{B}.


Лоренц установил, что существуют такие последовательности $x\in\ell_\infty$,
что значение выражения $Bx$ не зависит от выбора $B\in\mathfrak{B}$.
Такие последовательности называются почти сходящимися (англ. \textit{almost convergent}).
Пишут: $x\in ac$.

Лоренц доказал следующий критерий почти сходимости,
который оказывается исключительно удобен при проверке последовательности на принадлежность пространству $ac$.

\begin{theorem}[Критерий Лоренца]
	Для заданного $t\in\R$ равенство $Bx=t$ выполнено для всех $B\in\mathfrak{B}$
	тогда и только тогда, когда
	\begin{equation}
		\label{eq:crit_Lorentz}
		\lim_{n\to\infty} \frac{1}{n} \sum_{k=m+1}^{m+n} x_k = t
	\end{equation}
	равномерно по $m\in\N$.
\end{theorem}

Если некоторый $x\in\ell_\infty$ удовлетворяет~\eqref{eq:crit_Lorentz},
то мы будем говорить, что $x$ почти сходится к $t$,
и писать: $x\in ac_t$.
Таким образом, очевидно, $ac = \bigcup\limits_{t\in\R} ac_t$.

Равномерный предел в критерии Лоренца можно заменить на двойной~\cite[Теорема 1]{zvol2022ac}:
\begin{theorem}
	Для заданного $t\in\R$ равенство $Bx=t$ выполнено для всех $B\in\B$
	тогда и только тогда, когда
	\begin{equation}
		\label{eq:crit_Lorentz}
		\lim_{n,m\to\infty} \frac{1}{n} \sum_{k=m+1}^{m+n} x_k = t
		.
	\end{equation}
\end{theorem}

(Заметим вскользь, что в общем случае равномерный предел и двойной предел "--- это разные объекты;
за подробными комментариями отсылаем к классическим трудам по математическому анализу,
например,~\cite[с. 154]{kudryavcev2004mathanalys}.)
В настоящей главе в целях удобства доказывается модифицированный критерий Лоренца "--- теорема~\ref{thm:Lorentz_mod}.

Приведём важнейшее следствие из критерия Лоренца, позволяющее в ряде случаев без особых усилий показывать почти сходимость последовательности.

%TODO: ссылка? Есть у Усачёва
\begin{corollary}
	\label{thm:period_ac_avg}
	Всякая периодическая последовательность почти сходится к среднему по периоду.
	Иначе говоря, для любого $B\in\mathfrak B$
	\begin{equation}
		B(x_1,x_2, ..., x_n, \; x_1,x_2, ..., x_n, \; x_1,x_2, ..., x_n, \; x_1, ...) = \frac{x_1+x_2+...+x_n}{n}
		.
	\end{equation}
\end{corollary}

Сачестон~\cite{sucheston1967banach} установил, что
для любых $x\in \ell_\infty$ и $B\in\mathfrak{B}$
\begin{equation}\label{Sucheston}
	q(x) \leqslant Bx \leqslant p(x)
	,
\end{equation}
где
\begin{equation*}
	q(x) = \lim_{n\to\infty} \inf_{m\in\N}  \frac{1}{n} \sum_{k=m+1}^{m+n} x_k
	~~~~\mbox{и}~~~~
	p(x) = \lim_{n\to\infty} \sup_{m\in\N}  \frac{1}{n} \sum_{k=m+1}^{m+n} x_k
	.
\end{equation*}
называют нижним и верхним функционалом Сачестона соотвественно.
Заметим, что $p(x) = -q(-x)$.
Неравенства \eqref{Sucheston} точны:
для данного $x$ для любого $r\in[q(x); p(x)]$ найдётся банахов предел
$B\in\mathfrak{B}$ такой, что $Bx = r$.

Множество таких $x\in\ell_\infty$, что $p(x)=q(x)$ и
образует подпространство почти сходящихся последовательностей $ac$.
Таким образом, функционалы Сачестона являются удобным инструментом для доказательства того,
что некоторая последовательность $x$ не является почти сходящейся:
для этого достаточно показать, что $p(x)\ne q(x)$.

В~\cite[Theorem 5]{Jerison} показано, что нижний и верхний функционалы Сачестона могут быть переписаны в эквивалентном виде:
\begin{equation*}
	q(x) = \lim_{n\to\infty} \liminf_{m\to\infty}  \frac{1}{n} \sum_{k=m+1}^{m+n} x_k
	~~~~\mbox{и}~~~~
	p(x) = \lim_{n\to\infty} \limsup_{m\to\infty}  \frac{1}{n} \sum_{k=m+1}^{m+n} x_k
	.
\end{equation*}

Пространство $ac$ имеет интересную структуру.
За глобальным обзором его свойств отсылаем читателя к~\cite{semenov2006ac};
ряд интересных фактов можно почерпнуть в~\cite{usachev2008transforms},
а также в диссертации А.С. Усачёва~\cite{usachev2009_phd_vsu}.
%TODO2: включить англоязычную версию!
Стоит также отметить недавнюю работу~\cite{zvolinsky2021subspace},
в которой исследуется почти сходимость последовательностей,
определённых с помощью тригонометрических функций.

Часто мы будем иметь дело не с пространством всех почти сходящихся последовательностей $ac$,
а с его подпространством $ac_0$ последовательностей, почти сходящихся к нулю.
%TODO2: $ac$ не замкнуто относительно умножения.
%Рассмотрим  1/2, 1/2, 1, 0, 1, 0, 1/2, 1/2, 1/2, 1/2, 1, 0, 1, 0, 1, 0, 1, 0, 1/2, ...
% и          1/2, 1/2, 0, 1, 0, 1, 1/2, 1/2, 1/2, 1/2, 0, 1, 0, 1, 0, 1, 0, 1, 1/2, ...
Пространство $ac_0$ имеет ряд особенностей,
существенно отличающих его от пространства $c_0$ последовательностей, сходящихся к нулю.
То, что $ac_0\ne c_0$, показывает следующий классический
\begin{example}
	Пусть
	\begin{equation}
		x_k = \begin{cases}
			1, & ~\mbox{если}~ k = 2^n, ~ n\in\N,
			\\
			0  & ~\mbox{иначе}.
		\end{cases}
	\end{equation}
	Тогда $x\in ac_0 \setminus c_0$.
\end{example}
Более того, в отличие от пространства $c_0$, пространство $ac_0$ не замкнуто относительно
оператора взятия подпоследовательности, относительно покоординатного умножения и относительно возведения в степень.
Три этих свойства показывает
\begin{example}
	Пусть $x_n = (-1)^{n+1}$,
	т.е. $x = (1, -1,1,-1,1, -1,1,-1,...)$.
	Тогда $x\in ac_0$ в силу периодичности (см. следствие~\ref{thm:period_ac_avg}),
	но, очевидно, $x\cdot x = x^2\in ac_1$.
	Если же мы рассмотрим оператор перехода к подпоследовательности с чётными индексами
	\begin{equation}
		E(y_1, y_2, ...)  = (y_2, y_4, y_6, ...)
		,
	\end{equation}
	то обнаружим, что $Ex = (-1,-1,-1,-1,...)\in ac_{-1}$.
\end{example}

Можно привести и пример, когда взятие подпоследовательности выводит из всего пространства $ac$.
\begin{example}
	Пусть $x_n = (-1)^{n+1}$,
	т.е. $x = (1, -1,1,-1,1, -1,1,-1,...)$.
	Очевидно, что можно рассмотреть подпоследовательность
	\begin{equation}
		y = (1,-1, \; 1,1, -1,-1, \; 1,1,1, -1,-1,-1, \; 1,1,1,1,...)
		.
	\end{equation}
	Легко видеть, что верхний и нижний функционалы Сачестона принимают на последовательности $y$
	различные значения:
	$p(y) =1$, $q(y) = -1$.
	Следовательно, $x\notin ac$.
\end{example}

Е.А. Алехно доказал~\cite{alekhno2012superposition},
что в $ac_0$ существует максимальный идеал по умножению, обозначаемый $\Iac$ и,
более того, этот идеал может быть ёмко описан следующим критерием:
\begin{theorem}
	\label{thm:Iac_criterion_pos_neg}
	Последовательность $x\in\Iac$ тогда и только тогда,
	$x = x^+ +x^-$, $x^+\geq 0$, $x^- \leq 0$ и $x^+ \in ac_0$
	(последнее включение эквивалентно условию $x^- \in ac_0$).
\end{theorem}

%TODO2: а что там с идеалом по умножению в ac ?

Более того, $\Iac$ является подпространством в $ac_0$.
%TODO2: что там с недополняемостью?

Е.А. Алехно также исследовал
%TODO1: ссылка
\emph{стабилизатор} пространства $ac_0$:
\begin{equation}
	\Dac = \{x\in\ell_\infty : x\cdot y \in ac_0 \mbox{~для любого~} y \in ac_0\}
\end{equation}
(встречается также обозначение $\operatorname{St} ac_0$).
%TODO1: ссылка на то, где встречается
$\Dac$ также является подпространством (уже в $\ell_\infty$),
однако в настоящей работе в дальнейшем не используется,
и потому мы не будем останавливаться на его свойствах;
отсылаем читателя к~\cite{Luxemburg}.
%TODO2: что там с недополняемостью?




	\section{Переформулировка критерия Лоренца почти сходимости последовательности к нулю}
	\input{../criterion_ac0_quantors.tex}

	\section{О почти сходимости к нулю последовательности из нулей и единиц}
	\input{../episode_criterion_ac0.tex}

	\section{Замечание о свойствах последовательности $M(j)$}
	\input{../episode_C_j.tex}

	\section{Существование предела последовательности $M(j)$}
	\input{../episode_M_j_limit.tex}

	\section{Срезочный критерий почти сходимости к нулю неотрицательной последовательности}
	\input{../ac0_criterion_lambda.tex}

	\section{Пространство $ac_0$ и $\alpha$--функция}
	\input{../alpha_ac0_c0.tex}

	\section{$\alpha$--функция на пространстве $ac$ и расстояние до пространства $c$}
	\input{../alpha_ac_distance.tex}

	\section{Усиленная теорема Коннора}
	\input{../episode_Connor_generalized.tex}

	\section{О мере одного множества}
	\input{../ac_W_measure.tex}

	\section{Пространство $ac_0$ и возведение в степень}
	Вопрос о покоординатном возведении в степень почти сходящейся к нулю последовательности сколь-либо системно впервые поднят в~\cite{zvol2022ac}.
Возведение в отрицательную степень может выводить из пространства ограниченных последовательностей $\ell_\infty$ вовсе;
например, очевидна следующая
\begin{lemma}
	Пусть $x\in c_0$, $\alpha< 0$ и $x_k\ne 0 $ для любого $k$.
	Тогда
	\begin{equation*}
		x^\alpha = (x_1^\alpha,x_2^\alpha,x_3^\alpha,...) \notin \ell_\infty
		.
	\end{equation*}
\end{lemma}
(Здесь и далее мы полагаем, что возведение в соответствующую степень определено однозначно и в действительных числах;
то есть, если мы пишем $x_k^\alpha$, то мы неявно предполагаем, что значение этого выражения корректно определено.)


Для $x\in ac_0\setminus c_0$ ситуация, вообще говоря, не столь однозначна.

\begin{example}
	\label{example:ac0_pow_signum_classic}
	Рассмотрим классическую почти сходящуюся к нулю последовательность
	\begin{equation}
		x = (1;-1;1;-1;1;-1;...) \in ac_0
		.
	\end{equation}
	Очевидно, что для любой целой нечётной отрицательной степени $\alpha$ имеем $x^\alpha = x \in \ell_\infty$,
	однако для любой целой чётной отрицательной степени $\alpha$ мы получаем $x^\alpha = (1;1;1;1;1;1;...) \in \ell_\infty$.
\end{example}

\begin{example}
	Рассмотрим почти сходящуюся к нулю последовательность
	\begin{equation}
		y = \left(1;-1;\frac12;-\frac12;1;-1;\frac13;-\frac13;1;-1;\frac14;-\frac14;1;-1;...\right) \in ac_0
		.
	\end{equation}
	Очевидно, что
	\begin{equation}
		y^{-1} = \left(1;-1;2;-2;1;-1;3;-3;1;-1;4;-4;1;-1;...\right)  \notin \ell_\infty
		.
	\end{equation}
\end{example}

Возведение в отрицательную степень мы более обсуждать не будем.

Итак, в недавней статье Р.Е. Зволинского~\cite{zvol2022ac} доказаны два следующих факта (теорема 3 и следствие 2 соответственно):
\begin{theorem}
	\label{thm:Zvol_pow_pos}
	Пусть $x \geqslant 0, x \in a c_0$ и $\alpha>0$, тогда $x^\alpha \in a c_0$.
\end{theorem}

\begin{theorem}
	\label{thm:Zvol_pow_composed}
	Пусть $x\in ac_0$, $x = x^+ +x^-$, $x^+\geq 0$, $x^- \leq 0$ и $x^+ \in ac_0$
	(последнее включение эквивалентно условию $x^- \in ac_0$).
	Пусть $n\in\mathbb{N}$ или $n = \frac1{2k+1}$, $k\in\mathbb{N}$.
	Тогда $x^n \in ac_0$.
\end{theorem}

Возникает закономерный вопрос о том, что происходит при возведении в степень почти сходящейся к нулю последовательности,
которую нельзя разложить в сумму знакопостоянных почти сходящихся к нулю последовательностей.
Следующая теорема показывает, что условия теоремы~\ref{thm:Zvol_pow_composed} существенны.

\begin{theorem}
	\label{thm:ac0_pow_even}
	Пусть $x\in ac_0$, $x = x^+ +x^-$, $x^+\geq 0$, $x^- \leq 0$ и $x^+ \notin ac_0$
	(последнее условие эквивалентно условию $x^- \notin ac_0$).
	Пусть $n = 2k$, $k\in\mathbb{N}$.
	Тогда $x^n \notin ac_0$.
\end{theorem}

\begin{proof}
	Рассмотрим последовательность $y = (x^+)^n \geq 0$.

	Предположим, что $y \in ac_0$.
	Тогда $x^+ = y^{1/n}$ и по теореме~\ref{thm:Zvol_pow_pos} выполнено $x^+\in ac_0$,
	что противоречит условию доказываемой теоремы.
	Значит, $y = (x^+)^n \notin ac_0$ и выполнено неравенство $p\left((x^+)^n\right) > 0$.

	Заметим теперь, что  в силу чётности $n$ выполнено $(x^-)^n \geq 0$, откуда $p\left((x^-)^n\right) \geq 0$.
	(Строго говоря, можно по аналогии с $(x^+)^n$ показать, что $(x^-)^n\notin ac_0$ и, следовательно, $p\left((x^-)^n\right) > 0$.)
	В силу построения $x^+$ и $x^-$ мы имеем $\supp x^+ \cap \supp x^- = \varnothing$,
	откуда
	\begin{equation}
		x^n = (x^+ + x^-)^n = (x^+)^n + (x^-)^n
		.
	\end{equation}
	Очевидно, что для любых ограниченных последовательностей $a\geq0$, $b\geq 0$ выполнено неравенство для верхнего функционала Сачестона $p(a+b) \geq p(a)$.
	Отсюда получаем
	\begin{equation}
		p(x^n) = p\left((x^+)^n + (x^-)^n\right) \geq p\left((x^+)^n\right) > 0
		,
	\end{equation}
	что по теореме Сачестона означает, что $x^n \notin ac_0$.
\end{proof}

Для нечётной степени условие разложение в сумму двух знакопостоянных почти сходящихся последовательностей в теореме~\ref{thm:Zvol_pow_composed} тоже существенно.

\begin{example}
	Напомним,
	%TODO: ссылка!
	что любая периодическая последовательность почти сходится к своему среднему по периоду.
	Пусть
	\begin{equation}
		x = (1;1;-2;\ 1;1;-2;\ 1;1;-2;\ ...) \in ac_0
	\end{equation}
	и пусть $\alpha = 3$.
	Тогда
	\begin{equation}
		x^+ = (1;1;0;\ 1;1;0;\ 1;1;0;\ ...) \notin ac_0, \quad x^+ \in ac_{2/3}
		,
	\end{equation}
	\begin{equation}
		x^- = (0;0;-2;\ 0;0;-2;\ 0;0;-2;\ ...) \notin ac_0, \quad x^- \in ac_{-2/3}
		,
	\end{equation}
	\begin{equation}
		x^\alpha = (1;1;-8;\ 1;1;-8;\ 1;1;-8;\ ...) \notin ac_0, \quad x^\alpha \in ac_{-2}
		.
	\end{equation}
\end{example}

Однако для нечётной степени доказать аналог теоремы~\ref{thm:ac0_pow_even} не удастся "--- это показывает пример~\ref{example:ac0_pow_signum_classic}, в котором $x^+\in ac_1$, $x^-\in ac_{-1}$.


\chapter{Банаховы пределы, инвариантные относительно операторов}

	\section{Оператор с конечномерным ядром, для которого не существует инвариантного банахова предела}
	\input{../example_finite_dim_kernel.tex}

	\section{Оператор с бесконечномерным ядром, относительно которого инвариантен любой банахов предел}
	\input{../example_infinite_dim_kernel.tex}

	\section{О классах линейных операторов, для которых множества инвариантных банаховых пределов совпадают}
	\input{../linear_op_equiv_ac0.tex}

	\section{Мощность множества линейных операторов, относительно которых инвариантен банахов предел}
	\input{../power_operators_invariant.tex}

	\section{Банаховы пределы, инвариантные относительно операторов $\sigma_{1/n}$}
	Введём, вслед за~\cite[с. 131, утверждение 2.b.2]{lindenstrauss1979classical},
на $\ell_\infty$ оператор
\begin{equation}
	\sigma_{1/n} x = n^{-1}
	\left(
		\sum_{i=1}^{n} x_i,
		\sum_{i=n+1}^{2n} x_i,
		\sum_{i=2n+1}^{3n} x_i,
		...
	\right).
\end{equation}

Следующая лемма была впервые сформулирована в~\cite{ASSU2},
однако для полноты изложения мы приводим здесь более подробное доказательство.

\begin{lemma}[{\cite[{Лемма~4}]{ASSU2}}]
	Для любого $k\in\N_2$ выполнено
	\begin{equation}
		\sigma_k \sigma_{1/k} - I : \ell_\infty \to ac_0
		.
	\end{equation}
\end{lemma}

\begin{proof}
	Покажем, что для любого $x\in\ell_\infty$ выполнен критерий Лоренца~\eqref{eq:crit_Lorentz},
	т.е.
	\begin{equation}
		\label{eq:Lorentz_sigma_1_k_I}
		\lim_{n\to\infty} \frac{1}{n} \sum_{i=m+1}^{m+n} ((\sigma_k \sigma_{1/k} - I)x)_i = 0
	\end{equation}
	равномерно по $m\in\N$.

	Зафиксируем сначала $m$ и выберем $a,b\in\N$
	таким образом, что
	\begin{equation}
		k(a-1) <    m+1 \leq ka
		,
		\quad
		kb     \leq m+n <    k(b+1)
		.
	\end{equation}
	Тогда
	\begin{multline}
		0 \leq \left|
		\sum_{i=m     +1}^{m+n} \left( (\sigma_k \sigma_{1/k} - I)x \right)_i
		\right|
		\leq
		\\\leq
		\left|\sum_{i=m     +1}^{ka    } \left( (\sigma_k \sigma_{1/k} - I)x \right)_i\right| +
		\left|\sum_{i=ka    +1}^{kb    } \left( (\sigma_k \sigma_{1/k} - I)x \right)_i\right| +
		\left|\sum_{i=kb    +1}^{m+n   } \left( (\sigma_k \sigma_{1/k} - I)x \right)_i\right|
		\leq
		\\\leq
		\left|\sum_{i=m     +1}^{ka    } \left\| (\sigma_k \sigma_{1/k} - I)x \right\|\right| +
		\left|\sum_{i=ka    +1}^{kb    } \left( (\sigma_k \sigma_{1/k} - I)x \right)_i\right| +
		\left|\sum_{i=kb    +1}^{m+n   } \left\| (\sigma_k \sigma_{1/k} - I)x \right\|\right|
		\leq
		\\\leq
		      \sum_{i=k(a-1)+1}^{ka    } \left\| (\sigma_k \sigma_{1/k} - I)x \right\| +
		\left|\sum_{i=ka    +1}^{kb    } \left( (\sigma_k \sigma_{1/k} - I)x \right)_i\right| +
		      \sum_{i=kb    +1}^{k(b+1)} \left\| (\sigma_k \sigma_{1/k} - I)x \right\|
		=
		\\=
		2k \left\| (\sigma_k \sigma_{1/k} - I)x \right\| +
		\left|\sum_{i=ka    +1}^{kb    } \left( (\sigma_k \sigma_{1/k} - I)x \right)_i\right|
		\leq
		\\\leq
		2k \left(\left\| (\sigma_k \sigma_{1/k} - I)\| \cdot \|x \right\|\right) +
		\left|\sum_{i=ka    +1}^{kb    } \left( (\sigma_k \sigma_{1/k} - I)x \right)_i\right|
		\leq
		\\\leq
		2k \left( (\left\|\sigma_k\| \cdot \| \sigma_{1/k}\| + \|I\|) \cdot \|x \right\|\right) +
		\left|\sum_{i=ka    +1}^{kb    } \left( (\sigma_k \sigma_{1/k} - I)x \right)_i\right|
		\leq
		\\\leq
		4k  \|x \| +
		\left|\sum_{i=ka    +1}^{kb    } \left( (\sigma_k \sigma_{1/k} - I)x \right)_i\right|
		=
		4k  \|x \| +
		\left|\sum_{i=ka    +1}^{kb    } (\sigma_k \sigma_{1/k}x)_i - \sum_{i=ka    +1}^{kb    } x_i\right|
		=
		\\=
		4k  \|x \| +
		\left|
			\underbrace{\frac{x_{ka+1} + x_{ka+2} + ... + x_{ka+k}}{k} + ... + \frac{x_{ka+1} + x_{ka+2} + ... + x_{ka+k}}{k}}_{k~\text{~раз}}
			+
			...+
		\right.
		\\
		\left.
			+
			\underbrace{\frac{x_{k(b-1)+1}  + ... + x_{k(b-1)+k}}{k} + ... + \frac{x_{k(b-1)+1} + ... + x_{k(b-1)+k}}{k}}_{k~\text{~раз}}
			-\sum_{i=ka    +1}^{kb    } x_i
		\right|
		=
		4k\|x\|
		.
	\end{multline}

	Заметим, что полученное выражение не зависит от $m$.
	Следовательно,
	\begin{equation}
		\left|\lim_{n\to\infty} \frac{1}{n} \sum_{i=m+1}^{m+n} ((\sigma_k \sigma_{1/k} - I)x)_i\right|
		=
		%\\=
		\lim_{n\to\infty} \frac{1}{n} \left|\sum_{i=m+1}^{m+n} ((\sigma_k \sigma_{1/k} - I)x)_i\right|
		\leq
		\lim_{n\to\infty} \left( \frac{1}{n} \cdot 4k \|x\|\right)
		= 0
	\end{equation}
	равномерно по $m$,
	откуда и вытекает~\eqref{eq:Lorentz_sigma_1_k_I}.

	%Осталось заметить, что для знакопеременной последовательности $x = x^+ + x^-$, $x^+ \geq 0$, $x^- \leq 0$
	%мы имеем $(\sigma_k \sigma_{1/k} - I)x^+ \in ac_0$ и $(\sigma_k \sigma_{1/k} - I)(-x^-) \in ac_0$,
	%а потому $(\sigma_k \sigma_{1/k} - I)x \in ac_0$.
\end{proof}

\begin{theorem}
	\label{thm:B_sigma_n_eq_B_sigma_1_n}
	Для любого $n\in\N$ выполнено $\mathfrak{B}(\sigma_n) = \mathfrak{B}(\sigma_{1/n})$.
\end{theorem}

\begin{proof}
	Пусть сначала $B \in \mathfrak{B}(\sigma_{1/n})$.
	Тогда
	\begin{equation}
		B(x) = B(Ix) = B(\sigma_{1/n} \sigma_n x) = B(\sigma_n x)
		.
	\end{equation}
	Пусть теперь $B \in \mathfrak{B}(\sigma_n)$.
	Тогда
	\begin{equation}
		B(x) = B(Ix) = B((I+(\sigma_n \sigma_{1/n} - I)) x) = B(\sigma_n \sigma_{1/n} x) = B(\sigma_{1/n} x)
		.
	\end{equation}
\end{proof}

\begin{remark}
	Мы видим, что множество банаховых пределов, инвариантных относительно суперпозиции операторов,
	может быть шире, чем объединение множеств банаховых пределов,
	инвариантных относительно каждого из операторов:
	\begin{equation}
		\mathfrak{B}(\sigma_n) \cup \mathfrak{B}(\sigma_{1/n}) = \mathfrak{B}(\sigma_n) \subsetneq \mathfrak{B}(\sigma_{1/n}\sigma_n) = \mathfrak{B}(I)
		.
	\end{equation}
\end{remark}


	\section{Операторы $\tilde\sigma_k$}
	Введём в рассмотрение семейство операторов
$\tilde\sigma_k : \ell_\infty \to \ell_\infty$, $k>0$,
определяемых следующим образом:
\begin{equation}
	(\tilde\sigma_k x)_n = x_{\left\lceil \dfrac{n}{k}\right\rceil}
	.
\end{equation}

\begin{example}
	\label{example:sigma_3_2}
	\begin{equation}
		\tilde\sigma_{3/2} x =
		(x_1, x_2, x_2, \; x_3, x_4, x_4, \; x_5, ...)
		.
	\end{equation}
\end{example}

\begin{example}
	\label{example:sigma_2_3}
	\begin{equation}
		\tilde\sigma_{2/3} x =
		(x_2, x_3, \; x_5, x_6, \; x_8, ...)
		.
	\end{equation}
\end{example}


Заметим, что для $k\in \N$ выполнено равенство $\sigma_k = \tilde\sigma_k$
(однако использовать обозначение без тильды мы не можем, чтобы избежать путаницы с операторами усреднения $\sigma_{1/k}$).
Однако соотношение $\tilde\sigma_k \tilde\sigma_m = \tilde\sigma_{km}$ для нецелых $k$, вообще говоря, не выполняется.
Чтобы это увидеть, достаточно рассмотреть суперпозиции $\tilde\sigma_{3/2} \tilde\sigma_{2/3}$ и
$\tilde\sigma_{2/3} \tilde\sigma_{3/2}$ (см. примеры~\ref{example:sigma_3_2} и~\ref{example:sigma_2_3} выше).

Таким образом, операторы $\tilde\sigma_k$ обобщают операторы $\sigma_k$,
и возникает закономерный вопрос о соответствующих инвариантных банаховых пределах.

\begin{theorem}
	Пусть $k>0$, $k\in \mathbb{Q} \setminus \N$.
	Тогда $\B(\tilde\sigma_k)=\varnothing$.
\end{theorem}

\begin{proof}
	Пусть $k$ представимо в виде несократимой дроби $k=p/q$, $p\in \N$, $q\in\N_2$.
	Рассмотрим последовательность $x\in \ell_\infty$, заданную соотношением
	\begin{equation}
		x_k = \begin{cases}
			1, \mbox{~если~} m=qn+1, n\in\N,
			\\
			0  \mbox{иначе.}
		\end{cases}
	\end{equation}
	Последовательность $x$ периодична, и её период равен $q$.

	Пусть $B\in \B$, тогда $Bx=\dfrac1q$ , поскольку любой банахов предел на периодической последовательности принимает значение, равное среднему арифметическому по периоду.
	%TODO: ссылка!
	Заметим, что $\tilde\sigma_{p/q}x \in \Omega$.
	Более того, последовательность $\tilde\sigma_{p/q}x \in \Omega$ также периодична и имеет период, равный $p$.

	Действительно,
	\begin{multline}
		(\tilde\sigma_{p/q}x \in \Omega)_{m+p} =
		x_{\left\lceil \dfrac{m+p}{p/q}\right\rceil} =
		x_{\left\lceil \dfrac{qm+qp}{p}\right\rceil} =
		x_{\left\lceil \dfrac{qm}{p}+q\right\rceil} =
		\\=
		x_{q+\left\lceil \dfrac{qm}{p}\right\rceil} =
		x_{\left\lceil \dfrac{qm}{p}\right\rceil} =
		x_{\left\lceil \dfrac{m}{p/q}\right\rceil} =
		(\tilde\sigma_{p/q}x \in \Omega)_{m+p}
		.
	\end{multline}

\end{proof}

\begin{theorem}
	Пусть $k>0$, $k\in \mathbb{Q} \setminus \N$.
	Тогда $\tilde\sigma_k ac_0 \not \subset ac_0$.
\end{theorem}



\begin{hypothesis}
	Для любых натуральных $k > m$ существует такие $r, s\in\N$, что  $\tilde\sigma_{m/k} T^r \tilde\sigma_{k/m} = T^s$.
\end{hypothesis}


\begin{hypothesis}
	Для любого $k \in \N$ выполнено $(\tilde\sigma_{\sqrt{k}})^2 - \sigma_k : \ell_\infty \to ac_0$.
\end{hypothesis}


	\section{Мера прообраза числа при инвариантности банахова предела относительно оператора Чезаро}
	При изучении банаховых пределов и меры на множестве $\Omega$
возникает закономерный вопрос о мере множества
\begin{equation}
	\{ x \in \Omega : Bx = \beta \}
\end{equation}
для заданного банахова предела $B$ и числа $\beta\in[0;1]$.
(Хаусдорфова размерность этого множества равна 1 в силу леммы~\ref{lem:Hausdorf_measure}.)

\begin{theorem}
	Пусть $B \in \mathfrak{B}(C)$.
	Тогда
	\begin{equation}
		\mes \{ x \in \Omega : Bx = 1/2 \} = 1
		.
	\end{equation}
\end{theorem}

\begin{proof}
	Так как $B \in \mathfrak{B}(C)$, то
	\begin{equation}
		\{ x \in \Omega : Bx = 1/2 \}
		=
		\{ x \in \Omega : BCx = 1/2 \}
		\supset
		\{ x \in \Omega : \lim_{n\to\infty} (Cx)_n = 1/2 \}
		.
	\end{equation}
	Вместе с тем,
	\begin{equation}
		\mes \left\{ x \in \Omega : \lim_{n\to\infty} (Cx)_n  = 1/2 \right\} = 1
	\end{equation}
	(это следует из закона больших чисел~\cite{connor1990almost}).
\end{proof}



	\section{Классификация линейных операторов в свете банаховых пределов}
	При изучении инвариантности банаховых пределов относительно различных непрерывных линейных операторов
возникает закономерный вопрос о классификации этих операторов.

\begin{definition}
	Будем говорить, что оператор $H : \ell_\infty \to \ell_\infty$ \emph{эберлейнов},
	если $\B (H) \ne \varnothing$.
\end{definition}

Выбор именования этого класса операторов обусловлен тем, что именно Эберлейн в работе~\cite{Eberlein}
впервые сколь-либо системно изучил инвариантные банаховы пределы
(хотя отдельные шаги в этом направлении были сделаны ещё в работе~\cite{agnew1938extensions}).

В работе~\cite{alekhno2018invariant} вводится следующее

\begin{definition}
	Оператор $H : \ell_\infty \to \ell_\infty$ называется \emph{B-регулярным},
	если $H*\B \subseteq \B$.
	(Или, что то же самое, $HB\in\B$ для любого $B\in\B$.)
\end{definition}

В той же работе с помощью теоремы о неподвижной точке доказывается,
что любой B-регулярный оператор "--- эберлейнов, приводится следующее необходимое и достаточное условия В-регулярности.

\begin{theorem}
	Оператор $H:\ell_\infty \to \ell_\infty$ является B-регулярным тогда и только тогда, когда:
	\\(i) $H\mathbbm{1} \in ac_1$;
	\\(ii) $q(Hx)\geq 0$ для любого $x\geq 0$;
	\\(iii) $H ac_0 \subseteq ac_0$.
\end{theorem}
Легко заметить, что эти условия являются более слабыми, чем достаточные условия эберлейновости.
%TODO: ссылка


\begin{hypothesis}
	Существует эберлейнов оператор, не являющийся B-регулярным.
\end{hypothesis}

Эту классификацию можно надстроить в обе стороны (по включению) следующими двумя определениями:

\begin{definition}
	Будем называть оператор $A:\ell_\infty \to \ell_\infty$ \emph{дружелюбным (amiable)}, если $BA\in\mathfrak B$ для некоторого $B\in\mathfrak B$.
\end{definition}

\begin{definition}
	Будем называть оператор $A:\ell_\infty \to \ell_\infty$ \emph{существенно дружелюбным}, если $BA\in\mathfrak B$ для любого $B\in\mathfrak B$ и $BA\ne B$ для некоторого $B\in\mathfrak B$.
\end{definition}

Эти четыре класса операторов (существенно дружелюбные, В-регулярные, эберлейновы, дружелюбные "--- в порядке включения)
получены последовательным ослаблением условий, естественных для <<достаточно хороших>> операторов:
$\sigma_k$, $C$
и образуют иерархию по включению.
Возникает закономерный вопрос о совпадении классов.
Ясно, что оператор сдвига $T$ является В-регулярным, но не является существенно дружелюбным.

\begin{hypothesis}
	Существует дружелюбный оператор, не являющийся эберлейновым.
\end{hypothesis}


Далее возникает вопрос о свойствах классов операторов.

Из определения В-регулярности незамедлительно следует

\begin{lemma}
	Множество В-регулярных операторов замкнуто относительно суперпозиции.
\end{lemma}


	\section{Пример дружелюбного не эберлейнового оператора}
	\begin{theorem}
	Существует дружелюбный оператор, не являющийся эберлейновым.
\end{theorem}

\begin{proof}
	Пусть $B_1, B_2 \in \ext \B$.
	Положим
	\begin{equation}
		\label{eq:am_not_eber_def}
		B_3 = B_1 + 2(B_2-B_1) = 2B_2-B_1,
	\end{equation}
	тогда $B_3 \notin \B$.
	Действительно, если $B_3 \in \B$, то из~\eqref{eq:am_not_eber_def} следует, что
	\begin{equation}
		B_2 = \frac{B_1 + B_3}2 \in \B \setminus \ext \B
		.
	\end{equation}
	Введём в рассмотрение оператор $H:\ell_\infty\to\ell_\infty$, определённый равенством
	\begin{equation}
		2Hx = (x_1 + B_3x, x_2 + B_3x, ...) = x + (B_3 x) \cdot \mathbbm 1
		.
	\end{equation}
	Убедимся, что оператор $H$ дружелюбен.
	Действительно, для произвольного $x\in\ell_\infty$ имеем
	\begin{equation}
		2 B_1 H x = B_1 x + B_1 ((B_3 x) \cdot\mathbbm 1) = B_1 x + B_3 x =
		B_1 x + 2 B_2 x - B_1 x = 2B_2 x
		,
	\end{equation}
	откуда $B_1 H = B_2$.

	Убедимся теперь, что оператор $H$ не эберлейнов.
	Пусть $B = BH$ для некоторого $B\in\B$.
	Тогда для всех $x\in\ell_\infty$ имеем
	\begin{equation}
		2Bx = B (x + (B_3 x) \cdot \mathbbm 1)
		,
	\end{equation}
	т.е.
	\begin{equation}
		Bx =  B((B_3 x) \cdot \mathbbm 1)
		,
	\end{equation}
	откуда незамедлительно следует, что $Bx = B_3x$ и в силу произвольности выбора $x$ имеем $B=B_3$.
	Но ранее мы уже показали, что $B_3\notin \B$.
	Полученное противоречие завершает доказательство.
\end{proof}

\begin{hypothesis}
	$B_2H\notin \B$.
\end{hypothesis}

\begin{hypothesis}
	Существуют такие дружелюбный оператор $H:\ell_\infty\to\ell_\infty$ и банахов предел $B\in \B$,
	что $BH \in \B$, но $B_1 H \notin \B$ для любого $B\in \B\setminus\{B\}$.
	Также интересно наложение дополнительных свойств на $B$ и $BH$, например, принадлежности к $\ext\B$, $\B(C)$ и т.д.
\end{hypothesis}


	\section{Обратная задача об инвариантности}
	Ранее мы, как правило, ставили задачу, которую можно назвать прямой задачей об инвариантности:
дан некоторый достаточно хороший оператор $H$, и требуется выяснить, непусто ли множество $\B(H)$
банаховых пределов, инвариантных относительно этого оператора.
(И~если это множество непусто, то исследовать его.)

Возникает закономерный вопрос: для любого ли банахова предела существует нетривиальный оператор,
относительно которого этот банахов предел инвариантен?
Или же такие операторы существуют только для <<достаточно хороших>>
банаховых пределов "--- например, для банаховых пределов, инвариантных относительно какого-нибудь из операторов растяжения?
Существуют ли нетривиальные операторы, инвариантные относительно хотя бы одного $B\in\ext \B$?
(Здесь под тривиальным оператором понимается такой оператор $H:\ell_\infty \to \ell_\infty$, что $H-I:\ell_\infty \to ac_0$.)

Итак, в этом параграфе мы обсудим обратную задачу об инвариантности.
Она имеет неожиданно простое решение.

\begin{theorem}
	Для каждого $B\in \B$ существует такой оператор $G_B:\ell_\infty \to \ell_\infty$,
	что $\B(G_B) = \{B\}$.
\end{theorem}

\begin{proof}
	Определим оператор $G_B$ равенством
	\begin{equation}
		G_B x = (Bx, Bx, Bx, ...) = (Bx)\cdot\mathbbm 1
		.
	\end{equation}
	Легко видеть, что
	\begin{equation}
		B G_B x = B(Bx, Bx, Bx, ...) = B((Bx)\cdot\mathbbm 1) = (Bx)\cdot B(\mathbbm 1)= (Bx)\cdot 1 = Bx
	\end{equation}
	для любого $x\in\ell_\infty$.
	Значит, $B\in \B (G_B)$ и $\B (G_B)$ непусто.

	Рассмотрим теперь $B_1 \in \B \setminus\{B\}$.
	Последнее означает, что на некоторой последовательности $y\in\ell_\infty$ выполнено $B_1y \ne By$.
	Тогда
	\begin{equation}
		B_1 G_B y = B_1(By, By, By, ...) = B_1((By)\cdot\mathbbm 1) = (By)\cdot B_1(\mathbbm 1)= (By)\cdot 1 = By \ne B_1y
		.
	\end{equation}
	Таким образом, $B_1 G_B \ne B_1$, и $B_1 \notin \B (G_B)$.
	Это и означает, что $\B(G_B) = \{B\}$.
\end{proof}

\begin{definition}
	Оператор $G_B$, построенный таким образом, будем называть оператором,
	\emph{порождённым} банаховым пределом $B$.
\end{definition}

\begin{remark}
	Легко заметить, что любой порождённый оператор удовлетворяет достаточным условиям инвариантности.
	%TODO: ссылка на теорему
\end{remark}

\begin{remark}
	Более того, порождённый оператор $G_B$ является в некотором смысле крайним примером В-регулярного оператора,
	поскольку $G_B^* \B = \{B\}$.
\end{remark}

\begin{lemma}
	Пусть $0 \leq \lambda \leq 1$, $B_1, B_2 \in \B$.
	Тогда
	\begin{equation}
		G_{\lambda B_1+(1-\lambda) B_2} =\lambda G_{B_1} + (1-\lambda)G_{B_2}
		.
	\end{equation}
\end{lemma}


\begin{proof}
	В силу выпуклости (на единичной сфере в $\ell_\infty^*$) множества $\B$ оператор $G_{\lambda B_1+(1-\lambda) B_2}$
	действительно определён корректно.
	\begin{multline}
		G_{\lambda B_1+(1-\lambda) B_2} =
		\left((\lambda B_1+(1-\lambda) B_2) x\right)\mathbbm 1 =
		(\lambda B_1 x) \mathbbm 1 +((1-\lambda) B_2 x)\mathbbm 1 =
		\\=
		\lambda (B_1 x) \mathbbm 1 +(1-\lambda) (B_2 x)\mathbbm 1 =
		\lambda G_{B_1} x +(1-\lambda) G_{B_2 x}
		.
	\end{multline}
\end{proof}


\begin{lemma}
	Пусть $B_1, B_2 \in \B$.
	Тогда
	\begin{equation}
		G_{B_1} G_{B_2} = G_{B_2}
		.
	\end{equation}
\end{lemma}
\begin{proof}
	Пусть $x\in\ell_\infty$, тогда
	\begin{equation}
		G_{B_1}G_{B_2}x =
		G_{B_1} ( (B_2 x) \cdot \mathbbm 1 ) =
		(B_2 x) \cdot G_{B_1} ( \mathbbm 1 ) =
		(B_2 x) \cdot  \mathbbm 1 =
		G_{B_2}x
		.
	\end{equation}
\end{proof}


Если же известно, что банахов предел $B$ заведомо обладает дополнительными свойствами инвариантности,
то можно сконструировать и другие примеры операторов, относительно которых $B$ инвариантен.

\begin{example}
	Пусть $B\in\mathfrak B(\sigma_2)$.
	Напомним, что $B(\sigma_2) = B(\sigma_{1/2})$, где
	\begin{equation}
		\sigma_{1/2} x = \left(\dfrac{x_1+x_2}2, \dfrac{x_3+x_4}2, \dfrac{x_5+x_6}2, ...\right)
		.
	\end{equation}
	Рассмотрим оператор $H_B$, $H_Bx = (x_1, Bx, x_2, Bx, x_3, Bx, ...)$.
	Тогда
	\begin{multline}
		B H_B x = B \sigma_{1/2} H_B x = B\left(\dfrac{x_1+Bx}2, \dfrac{x_2+Bx}2, \dfrac{x_3+Bx}2, ...\right) =
		\\=
		B\left(\dfrac{x_1}2, \dfrac{x_2}2, \dfrac{x_3}2, ...\right) + B\left(\dfrac{Bx}2, \dfrac{Bx}2, \dfrac{Bx}2, ...\right)=
		\\=
		\dfrac12 Bx + \dfrac12 B\left(Bx, Bx, Bx, ...\right) = Bx
		.
	\end{multline}

	При этом операторы $H_B$, получаемые таким образом, достаточно <<разнообразны>>.
	Пусть $B_1 x \ne B_2 x$ на некотором $x\in \ell_\infty$.
	Тогда $y = H_{B_1} x - H_{B_2} x = (0, B_1 x - B_2 x, 0, B_1 x - B_2 x, 0, B_1 x - B_2 x, ...)$ и $B_1y = B_2y = \dfrac{B_1 x - B_2 x}2\ne 0$, откуда $(H_{B_1} - H_{B_2})\ell_\infty \not\subseteq ac_0$.


	Покажем  теперь, что $B_2 \notin \mathfrak B (H_{B_1})$ при $B_1 \ne B_2$. Предположим противное. Значит, $B_1 x \ne B_2 x$ на некотором $x\in \ell_\infty$. Пусть $y$ такой же, как выше. Тогда $B_2 y = B_2(H_{B_1} x - H_{B_2} x) = B_2 H_{B_1} x - B_2 H_{B_2} x = B_2 x - B_2 x = 0$, но $B_2 y = \dfrac{B_1x - B_2 x}{2} \ne 0$. Получили противоречие.

	На операторы $\sigma_k$ полученная конструкция обобщается тривиально.
	%TODO: обобщить!
\end{example}



\chapter{Функционалы Сачестона и линейные оболочки}
	При исследовании банаховых пределов особый интерес представляют разделяющие множества~\cite[\S 3]{Semenov2014geomprops}.
Множество $Q\subset\ell_\infty$ называют разделяющим, если
для любых неравных $B_1, B_2\in\mathfrak{B}$ существует такая последовательность $x\in Q$,
что $B_1 x \neq B_2 x$.
В частности, разделяющим является~\cite{semenov2010characteristic} множество всех последовательностей из 0 и 1,
которое, как и выше, мы будем обозначать через $\Omega$
(иногда в литературе встречается также обозначение $\{0;1\}^\N$).

Каждой последовательности $(x_1, x_2, \dots)\in \Omega$ можно поставить в соответствие число
\begin{equation}\label{eq:bijection_omega_0_1}
	\sum_{k=1}^\infty 2^{-k} x_k \in [0,1]
	.
\end{equation}
С точностью до счётного множества это соответствие взаимно однозначно и определяет на множестве $\Omega$ меру,
которую мы будем отождествлять с мерой Лебега на $[0,1]$.

Оказывается, что из $\Omega$ можно выделить некоторые подмножества, которые также будут разделяющими,
например \cite[\S 3, Теорема 11]{Semenov2014geomprops},
\begin{equation}
	U = \{ x\in\Omega: q(x) = 0, p(x) = 1 \}
	.
\end{equation}

Однако множество $U$ имеет меру 1~\cite{semenov2010characteristic}.
Счётных разделяющих множеств не существует~\cite[{следствие 22}]{Semenov2014geomprops}.

В настоящей главе строится пример разделяющего множества,
являющегося подмножеством $\Omega$ и имеющего меру нуль.
Для построения такого множества используется следующий факт.

\begin{lemma}[{\cite[\S 3, замечание 6]{Semenov2014geomprops}}]
	Пусть $X$~--- разделяющее множество и $X \subset \operatorname{Lin} Y$,
	где $\operatorname{Lin} Y$ обозначает линейную оболочку $Y$.
	Тогда $Y$ также является разделяющим множеством.
\end{lemma}

Затем обсуждаются свойства линейных оболочек множеств, определённых с помощью функционалов Сачестона.
В частности, доказывается,
что наряду с использованным при построении разделяющего множества меры нуль включением
\begin{equation}
	\Omega \subset \operatorname{Lin}\{x\in\Omega : p(x) = a,~ q(x) = b\}
\end{equation}
для любых $0\leq b < a \leq 1$,
имеет место равенство
\begin{equation}
	\ell_\infty = \operatorname{Lin}\{x\in\ell_\infty : p(x) = a,~ q(x) = b\}
\end{equation}
для любых $a>b$.

Возникает закономерный вопрос: для каких ещё подмножеств пространства $\ell_\infty$
верны аналогичные соотношения?

Оказывается, что аналогичным свойством обладает и ещё одно подмножество пространства $\ell_\infty$: подпространство
$A_0 = \{ x \in \ell_\infty : \alpha(x) =0 \}$,
где, напомним,
\begin{equation*}
	\alpha(x) = \varlimsup_{i\to\infty} \max_{i<j\leqslant 2i} |x_i - x_j|
	,
\end{equation*}
уже знакомое нам, например, по теореме~\ref{thm:A0_is_space}.

Некоторые результаты данной главы были анонсированы в~\cite{our-mz2021linearhulls}
и опубликованы в~\cite{avdeev2021vestnik}.


	\section{О хаусдорфовой размерности одного класса множеств}
	На множестве $\Omega$ стандартным образом определяется размерность Хаусдорфа (см. например \cite[Секция 6]{Edgar}).
Для непустого подмножества $F\subset \mathbb R^n$ и $s > 0$ определим $s$-мерную меру Хаусдорфа множества $F$ следующим образом:
$$\mathcal H^s(F) := \lim_{\delta\to0} \inf \left\{\sum_{i=1}^\infty \left({\rm diam} \ U_i \right)^s \ : \ F\subset \bigcup_{i=1}^\infty  U_i, \  0\leqslant {\rm diam} \ U_i \leqslant \delta \right\}.$$

Размерность Хаусдорфа множества  $F\subset \mathbb R^n$ определяется по формуле:
$${\rm dim}_H F := \inf\{ s > 0 \ : \ \mathcal H^s(F)=0\}.$$



Мы приведём определение самоподобных подмножеств множества $\Omega$ (см., например, \cite{falconer1997techniques}).

\begin{definition}
Множество $E\subset\Omega$ называется самоподобным, если существуют $m\in\mathbb{N}$,
$m\geqslant2$, $0< r_1, \dots, r_m<1$ и функции $f_j : \Omega \to \Omega$, $j=1,\dots, m$ такие, что
$$\rho(f_j(x), f_j(y)) = r_j \rho(x,y), \ \forall \ x,y \in \Omega, \ j=1,\dots, m$$
и $E=\bigcup_{j=1}^m f_j(E).$
\end{definition}



\begin{lemma}
	\label{lem:Hausdorf_measure}
	Пусть $E\subset\Omega$ и $TE = E$.
	Тогда размерность Хаусдорфа $E$ равна $1$.
\end{lemma}

\begin{proof}
	Для $j=1,2$ определим функции $f_j : \Omega \to \Omega$ следующим образом:
	$$f_1(x_1, x_2, \dots)=(0, x_1, x_2, \dots), \quad f_2(x_1, x_2, \dots)=(1, x_1, x_2, \dots).$$

	Очевидно, что $E=f_1(E)\cup f_2(E).$

	Теперь мы покажем, что размерность Хаусдорфа множества $E$ равна $1$.

	Т.к. для $j=1,2$ верно
	 $$\rho(f_j(x),f_j(y))=\frac12\rho(x,y), \ \forall \ x, y \in E,$$
	 то функции $f_j$ являются преобразованиями подобия с коэффициентами $r_j=1/2$ для $j=1,2$.


	По~\cite[Теорема 9.3]{Edgar} размерность Хаусдорфа $d$ множества $E$ является решением уравнения:
	$$ r_1^d+r_2^d=1.$$
	Т.к. $r_j=1/2$, то
	$d=1.$
\end{proof}

Лемма~\ref{lem:Hausdorf_measure} позволяет несколько сократить доказательство
[ТОDO: ссылка на ASU\_a\_a].
Действительно, $x\in\Omega\setminus c$ принадлежит $W$ тогда и только тогда, когда
\begin{equation}
	\label{eq:dim_ext_B_W}
	(\ext \mathfrak{B})x = \{0;1\}
\end{equation}
Очевидно, что соотношение~\eqref{eq:dim_ext_B_W} выполнено для $x$ тогда и только тогда, когда оно выполнено для $Tx$.







	\section{О существовании разделяющих множеств малой хаусдорфовой размерности}
	Хорошо известны некоторые примеры разделяющих множеств
[TODO: много ссылок, в т.ч. на новую статью в МЗ].
Применяя лемму~\ref{lem:Hausdorf_measure}, можно показать,
что хаусдорфову размерность 1 имеют множества
TODO: список!

В данном пункте строится пример разделяющего множества,
имеющего малую хаусдорфову размерность.


\begin{theorem}
	\label{lem:Hausdorf_measure}
	Пусть $n\in\mathbb{N}$.
	Тогда существует разделяющее множество $E\subset\Omega$ такое,
	что $\dim_H E = 1/n$.
\end{theorem}

\begin{proof}
	Пусть
	\begin{equation}
		Е= \{ x \in \Omega : k \neq mn \Rightarrow x_k = 0\}, m\in \mathbb{N}
		,
	\end{equation}
	т.е. у последовательности $x \in E$ равны нулю все элементы, кроме, быть может, $x_n$, $x_{2n}$, $x_{3n}$ и т.д.

	Для $j=1,2$ определим функции $f_j : \Omega \to \Omega$ следующим образом:
	\begin{equation}
		f_1(x_1, x_2, \dots)=(\underbrace{0, ..., 0,}_{\mbox{$n-1$ раз}} 0, x_1, x_2, \dots)
		,
		\quad
		f_2(x_1, x_2, \dots)=(\underbrace{0, ..., 0,}_{\mbox{$n-1$ раз}} 1, x_1, x_2, \dots)
		.
	\end{equation}

	Очевидно, что $E=f_1(E)\cup f_2(E).$

	Теперь мы покажем, что размерность Хаусдорфа множества $E$ равна $2^{-n}$.

	Т.к. для $j=1,2$ верно
	 $$\rho(f_j(x),f_j(y))=2^{-n}\rho(x,y), \ \forall \ x, y \in E,$$
	 то функции $f_j$ являются преобразованиями подобия с коэффициентами $r_j=2^{-n}$ для $j=1,2$.


	По~\cite[Теорема 9.3]{Edgar} размерность Хаусдорфа $d$ множества $E$ является решением уравнения:
	$$ r_1^d+r_2^d=1.$$
	Т.к. $r_j=2^{-n}$, то
	$d=1/n.$
\end{proof}








\chapter{Функционалы Сачестона и мультипликативные свойства носителя последовательности}
	Дальнейшим ослаблением понятия сходимости является сходимость по Чезаро (сходимость в среднем).
Говорят, что последовательность $\{x_n\}\in\ell_\infty$ сходится по Чезаро к $t$, если
\begin{equation}
	\lim_{n\to\infty}\frac1{n}\sum_{i=1}^n x_i = t
	.
\end{equation}
Легко заметить, что обсуждаемые обобщения верхнего и нижнего пределов удовлетворяют соотношению
\begin{equation}
	\label{eq:generalization_of_limits}
	\liminf_{n\to\infty} x_n \leq q(x) \leq \liminf_{n\to\infty}\frac1{n}\sum_{i=1}^n x_i
	\leq
	%\\ \leq
	\limsup_{n\to\infty}\frac1{n}\sum_{i=1}^n x_i
	\leq p(x)
	\leq \limsup_{n\to\infty} x_n
	.
\end{equation}

Отдельный интерес представляет множество всех последовательностей из 0 и 1,
которое, как и выше, мы будем обозначать через $\Omega$.
%(иногда в литературе~\cite{semenov2020geomBL,Semenov2014geomprops} встречается также обозначение $\{0;1\}^\N$).
Понятно, что каждый $x\in \Omega$ можно отождествить с подмножеством множества натуральных чисел
$\supp x \subset \N$.

Вслед за~\cite{hall1992behrend} будем обозначать через $\mathscr{M}A$ множество всех чисел,
кратных элементам множества $A\subset\N$, т.е.
\begin{equation}
	\mathscr{M}A = \{ka: k\in\N, a\in A\}
	,
\end{equation}
через $\chi F$ "--- характеристическую функцию множества $F$.

Так, например,
\begin{gather}
	\chi \mathscr{M}\{2\} = \chi \mathscr{M}\{2, 4\} = \chi \mathscr{M}\{2,4,8,16,...\}
	= (0,1,0,1,0,1,0,1,0,...),
\\
	\chi \mathscr{M}\{3\} = \chi \mathscr{M}\{3,9,27,...\} = (0,0,1,\;0,0,1,\;0,0,1,\;0,0,1,\;0,0,1,\;...),
\\
	\chi \mathscr{M}\{2,3\} = \chi \mathscr{M}\{2,3,6\} = (0,1,1,1,0,1,\;0,1,1,1,0,1,\;0,1,1,1,0,1,...).
\end{gather}

Возникает закономерный вопрос о взаимосвязи структуры множества $A$
и значений, которые принимают обобщения верхнего и нижнего пределов~\eqref{eq:generalization_of_limits}
на последовательности $\chi \mathscr{M}\!A$.
Так, в работах~\cite{davenport1936sequences,davenport1951sequences} доказано, что для любого
$A=\{a_1,a_2,...\}\subset\N$
выполнено
\begin{equation}
	\liminf_{n\to\infty}\frac1{n}\sum_{i=1}^n (\chi\mathscr{M}A)_i =
	\lim_{j\to\infty}\lim_{n\to\infty}\frac1{n}\sum_{i=1}^n (\chi\mathscr{M}\{a_1,a_2,...,a_j\})_i
	.
\end{equation}
В работе~\cite[\S 7]{besicovitch1935density} построено такое множество $A\subset\N$, что
\begin{equation}
	\liminf_{n\to\infty}\frac1{n}\sum_{i=1}^n (\chi\mathscr{M}A)_i \neq
	\limsup_{n\to\infty}\frac1{n}\sum_{i=1}^n (\chi\mathscr{M}A)_i
	.
\end{equation}
За более подробной информацией о множествах типа $\mathscr{M}A$ отсылаем читателя к монографии~\cite{hall1996multiples}.

В этой главе изучается зависимость значений, которые могут принимать функционалы Сачестона
на последовательностях $\chi\mathscr{M}A$, от свойств множества $A$.



\chapter*{Заключение}
\addcontentsline{toc}{chapter}{Заключение}
В работе исследованы такие объекты и понятия, определяемые на пространстве ограниченных последовательностей,
как почти сходимость (глава 1), $\alpha$--полунорма, для краткости называемая $\alpha$--функцией (глава 2),
инвариантные банаховы пределы (глава 3), разделяющие множества и линейные оболочки (глава 4),
функционалы Сачестона и мультипликативные свойства носителя (глава 5).

В главе 1 выведены критерии принадлежности ограниченных последовательностей
специального вида к пространству почти сходящихся последовательностей
и пространству последовательностей, почти сходящихся к нулю,
а также критерий сходимости почти сходящейся последовательности.
Далее эти критерии применены для исследования последовательностей и множеств,
естественным образом возникающих в других задачах о банаховых пределах.
Для элементов пространства почти сходящихся последовательностей $ac $
установлена двусторонняя оценка на расстояние до пространства сходящихся последовательностей $c$,
использующая $\alpha$--функцию.
Примечательно (хотя и ожидаемо), что наряду с трансляционно неинвариантной $\alpha$--функцией
в этой оценке используется и функционал $\lim_{n\to\infty}\alpha(T^n x)$,
который, очевидно, трансляционно инвариантен.

Глава 2 посвящена $\alpha$--функции и содержит достаточно подробное описание
свойств суперпозиции этой функции с классическими линейными операторами:
сдвига $T$, повторения $\sigma_n$, Чезаро $C$ и т.д.
Несмотря на то, что $\alpha$--функция оказалась трансляционно неинвариантной,
эта неинвариантность в некотором смысле однородна (см. следствие \ref{thm:est_alpha_Tn_x_full}).
Это вполне логично, поскольку расстояние до пространства $c$ и почти сходимость
сами суть трансляционно инвариантные характеристики.
Завершается глава 2 исследованием подпространства таких ограниченных последовательностей,
на которых $\alpha$--функция обращается в нуль.
Доказаны основные свойства этого пространства;
в частности, оно недополняемо в $\ell_\infty$.

Глава 3 непосредственно посвящена инвариантным банаховым пределам и содержит
значительное количество примеров отыскания множества инвариантных банаховых пределов $\B(H)$
для различных линейных операторов $H$, действующих на пространстве ограниченных последовательностей.
В частности, показана существенная избыточность ранее известных условий эберлейновости оператора
(т.е. существования хотя бы одного банахова предела, инвариантного относительно данного оператора).
Введены новые и исследованы существующие классы линейных операторов на пространстве ограниченных последовательностей
в соответствии со свойствами их суперпозиции с банаховыми пределами:
полуэберлейновы, эберлейновы, В-регулярные, существенно эберлейновы.
Показано, что каждый следующий класс содержится в предыдущем и не совпадает с ним.
Найдено простое решение обратной задачи об инвариантности,
т.е. задачи о построении по банахову пределу оператора, относительно которого он инвариантен.

Глава 4 посвящена разделяющим множествам и линейным оболочкам, при этом вторые,
будучи детально исследованы, выступают как вспомогательный элемент для построения первых.
Завершается глава 4 построением разделяющего подмножества последовательностей из нулей и единиц,
имеющего (лебеговскую) меру нуль и сколь угодно малую хаусдорфову размерность.

Глава 5 устанавливает связь между мультипликативными свойствами носителя последовательности
из нулей и единиц и значениями, который могут принимать верхний и нижний функционалы Сачестона
(а значит, и банаховы пределы) на такой последовательности.
Для установления этой связи используются построения из теории чисел.


По итогам проведённых исследований выдвинут ряд гипотез,
работу над доказательством или опровержением которых планируется продолжать в дальнейшем,
или же использовать эти гипотезы как задачи для молодых исследователей,
знакомящихся с пространством ограниченных последовательностей.



\chapter*{Список сформулированных гипотез}
\addcontentsline{toc}{chapter}{Список сформулированных гипотез}

\renewcommand\label[1]{}
\hypotlist


\addcontentsline{toc}{chapter}{Список литературы}

\makeatletter
\ltx@iffilelater{biblatex-gost.def}{2017/02/01}%
{\toggletrue{bbx:gostbibliography}%
\renewcommand*{\revsdnamepunct}{\addcomma}}{}
\makeatother

\printbibliography{}

\end{document}
