\documentclass[12pt,a4paper,openbib]{report}
\usepackage{amsmath}
\usepackage[utf8]{inputenc}
\usepackage[english,russian]{babel}
\usepackage{amsfonts,amssymb}
\usepackage{latexsym}
\usepackage{euscript}
\usepackage{enumerate}
\usepackage{graphics}
\usepackage[dvips]{graphicx}
\usepackage{geometry}
\usepackage{wrapfig}
\usepackage[colorlinks=true,allcolors=black]{hyperref}
\usepackage{bbm}
\usepackage{enumitem}
\usepackage{mathrsfs}


% https://tex.stackexchange.com/questions/634509/show-hide-thumbnail-sidebar-by-default-in-pdf
\hypersetup{pdfpagemode=UseNone}


\righthyphenmin=2

%\usepackage[14pt]{extsizes}

\geometry{left=2.5cm}% левое поле
\geometry{right=1cm}% правое поле
\geometry{top=2cm}% верхнее поле
\geometry{bottom=2cm}% нижнее поле

\renewcommand{\baselinestretch}{1.3}

\renewcommand{\leq}{\leqslant}
\renewcommand{\geq}{\geqslant} % И делись оно всё нулём!

\DeclareMathOperator{\ext}{ext}
\DeclareMathOperator{\mes}{mes}
\DeclareMathOperator{\supp}{supp}

\newcommand{\N}{\ensuremath{\mathbb{N}}}
\newcommand{\Q}{\ensuremath{\mathbb{Q}}}
\newcommand{\R}{\ensuremath{\mathbb{R}}}
\newcommand{\B}{\ensuremath{\mathfrak{B}}}
\newcommand{\Iac}{\mathcal{I}(ac_0)}
\newcommand{\Dac}{\mathcal{D}(ac_0)}


\newcommand{\longcomment}[1]{}

\usepackage[backend=biber,style=gost-numeric,sorting=none]{biblatex}
\addbibresource{../bib/Semenov.bib}
\addbibresource{../bib/my.bib}
\addbibresource{../bib/ext.bib}
\addbibresource{../bib/classic.bib}
\addbibresource{../bib/general_monographies.bib}
\addbibresource{../bib/Bibliography_from_Usachev.bib}

\input{../bib/ext.hyphens.bib}

\usepackage{amsthm}
\theoremstyle{definition}
\newtheorem{lemma}{Лемма}[section]
\newtheorem{theorem}[lemma]{Теорема}
\newtheorem{example}[lemma]{Пример}
\newtheorem{property}[lemma]{Свойство}
\newtheorem{remark}[lemma]{Замечание}
\newtheorem{definition}[lemma]{Определение}
\newtheorem{proposition}[lemma]{Утверждение}
\newtheorem{corollary}{Следствие}[lemma]

\newtheorem{hhypothesis}[lemma]{Гипотеза}


\newcommand\hypotlist{ }
\newcounter{hypcount}

\makeatletter
\usepackage{environ}
\NewEnviron{hypothesis}{%

	\edef\curlabel{hhypothesis\thehypcount}
    \begin{hhypothesis}
		\label{\curlabel}
		\BODY%
    \end{hhypothesis}
	\edef\curref{\noexpand\ref{\curlabel}}

	\expandafter\g@addto@macro\expandafter\hypotlist\expandafter
	{\paragraph{Гипотеза\!\!\!}}


	\expandafter\g@addto@macro\expandafter\hypotlist\expandafter
	{\expandafter\textbf\expandafter{\curref}}

	\expandafter\g@addto@macro\expandafter\hypotlist\expandafter
	{\textbf{.}~~}

	\expandafter\g@addto@macro\expandafter\hypotlist\expandafter
	{\BODY}

	\addtocounter{hypcount}{1}
}
\makeatother

%Only referenced equations are numbered
\usepackage{mathtools}
\mathtoolsset{showonlyrefs}

%\mathtoolsset{showonlyrefs=false}
% (an equation/multline to be force-numbered)
%\mathtoolsset{showonlyrefs=true}

% https://superuser.com/questions/517025/how-can-i-append-two-pdfs-that-have-links
\usepackage{pdfpages}% http://ctan.org/pkg/pdfpages

\begin{document}
\clubpenalty=10000
\widowpenalty=10000
\includepdf{title.pdf}
\setcounter{page}{2}
\tableofcontents

\chapter*{Введение}
\addcontentsline{toc}{chapter}{Введение}
Сходящиеся последовательности, т.е. последовательности, имеющие предел в смысле классического математического анализа,
изучены достаточно хорошо.
В частности, любая сходящаяся последовательность является ограниченной.
Пространство ограниченных последовательностей будем, вслед за классиками \cite{wojtaszczyk1996banach,lindenstrauss1973classical},
обозначать через $\ell_\infty$ и снабжать его нормой
\begin{equation*}
	\|x\| = \sup_{n\in\mathbb{N}}|x_n|
	.
\end{equation*}

Однако в приложениях часто возникают ограниченные последовательности,
которые не являются сходящимися.
В таком случае возникает закономерный вопрос:
как измерить <<недостаток сходимости>>?
<<насколько не сходится>> последовательность?

Наиболее очевидным кажется вычисление расстояния $\rho(x,c)$ от заданного элемента $x\in\ell_\infty$
до пространства сходящихся последовательностей $c$
(которое равно половине разности верхнего и нижнего пределов последовательности).
Однако выясняется, что имеют место быть и другие подходы.

Нетрудно заметить, что операция взятия классического предела на пространстве сходящихся последовательностей
является непрерывным (в норме $\ell_\infty$) линейным функционалом.
В 1929 г. С. Мазур анонсировал~\cite{Mazur}, а позже
С. Банах доказал \cite{banach2001theory_rus}, что этот функционал может быть непрерывно продолжен на всё пространство $\ell_\infty$.
На основе этой идеи были определены банаховы пределы
(иногда также называемые пределами Банаха--Мазура \cite{alekhno2012superposition,alekhno2015banach})
следующим образом.

Банаховым пределом называется функционал $B\in \ell_\infty^*$ такой, что:
\begin{enumerate}
	\item
		$B \geqslant 0$
	\item
		$B\one = 1$
	\item
		$B=BT$
\end{enumerate}

Простейшие свойства:
\begin{itemize}
	\item
		$\|B\|_{\ell_\infty^*} = 1$
	\item
		$Bx = \lim\limits_{n\to\infty} x_n$ для любого $x=(x_1, x_2, ...) \in c$.

		Таким образом,
		банахов предел~--- действительно естественное обобщение понятия предела сходящейся последовательности
		на все ограниченные последовательности.
\end{itemize}

Множество банаховых пределов обычно обозначают через $\mathfrak{B}$
(реже через $BM$~--- см., например, \cite{alekhno2012superposition,alekhno2015banach}).

Лоренц \cite{lorentz1948contribution} установил, что существует подпространство $\ell_\infty$,
на котором все банаховы пределы принимают одинаковое значение.
Это пространство названо пространством почти сходящихся последовательностей и обычно обозначается $ac$
(от англ. <<almost convergent>>).
Включение $c \subset ac$ собственное, т.е. $ac \setminus c \neq \varnothing$.


Обобщая критерий Лоренца, Сачестон \cite{sucheston1967banach} доказал, что для любого $x\in\ell_\infty$
и любого $B\in\mathfrak{B}$
\begin{equation*}
	q(x) =
	\lim_{n\to\infty} \sup_{m\in\mathbb{N}} \frac{1}{n} \sum_{k=m+1}^{k=m+n} x_k
	\leq
	Bx
	\leq
	\lim_{n\to\infty} \inf_{m\in\mathbb{N}} \frac{1}{n} \sum_{k=m+1}^{k=m+n} x_k
	= p(x)
\end{equation*}
и, более того,
\begin{equation*}
	\mathfrak{B}x = [q(x), p(x)]
	.
\end{equation*}


За более подробным обзором ранних исследований банаховых пределов отсылаем читателя к~\cite{greenleaf1969invariant,day1973normed,kangro1976theory}.
%Source: https://encyclopediaofmath.org/wiki/Banach_limit
Вскоре после работ Сачестона Дж. Куртц распространил понятие банаховых пределов
на векторные последовательности~\cite{kurtz1970almost},
а затем и на последовательности в произвольных банаховых пространствах~\cite{kurtz1972almost}.
За обсуждением банаховых пределов в векторных пространствах отсылаем читателя
к~\cite{deeds1968summability,hajdukovic1975almost,armario2013vectorvalued_rus,garcia2015extremal,garcia2016fundamental_rus}.
%Тут есть ещё ссылки: https://www.mathnet.ru/php/archive.phtml?wshow=paper&jrnid=faa&paperid=3146&option_lang=rus
%TODO2: В том числе и про то, где применяется.
В недавней работе~\cite{chen2007characterizations} Ч.~Чен и М.~Куо изучают обобщения банаховых пределов
на произвольные гильбертовы пространства и на пространства суммируемых функций $L_p$.
Другим обобщениям банаховых пределов посвящены работы
\cite{hajdukovic1975functionals,koga2016generalization}.
%tanaka2018banach - иероглифы, несовместимые с библиографией

Ещё одним достаточно плодотворным обобщением банаховых пределов оказались их аналоги на двойных последовательностях~\cite{robison1926divergent}, введённые Дж.~Д.~Хиллом в~\cite{hill1965almost}.
За дальнейшими результатами в этом направлении отсылаем читателя
к~\cite{moricz1988almost,bacsarir1995strong,mursaleen2003almost,edely2004almost,mursaleen2004almost}.
Из недавних работ стоит отдельно отметить статью М. Мурсалена и С.А. Мухиддина~\cite{mursaleen2012banach},
в которой с помощью понятия почти сходимости в пространстве ограниченных двойных последовательностей вводится ряд новых интересных подпространств.

Наконец, если исключить из определения банахова предела требование трансляционной инвариантности,
то мы получим объект, называемый обобщённым пределом,
подробно изучавшийся М.~Джерисоном в~\cite{jerison1957set} и многих других работах.


Таким образом, на вопрос: <<Насколько не сходится последовательность?>> %~---
можно дать ответ в терминах почти сходимости, т.е. принадлежности пространству $ac$,
а на вопрос: <<Насколько почти не сходится последовательность?>>~---
назвать длину отрезка $[q(x), p(x)]$.
В дальнейшем пространство почти сходящихся последовательностей неоднократно становилось предметом
различных исследований
\cite{semenov2006ac,usachev2008transforms}.
В частности, в работе~\cite{connor1990almost} доказано,
что последовательность из нулей и единиц почти наверное не принадлежит пространству $ac$.
Этот факт демонстрирует, что почти сходящиеся последовательности <<достаточно редки>>.

%TODO: ссылки! Хватит или ещё?

Банаховы пределы также нашли своё применение в приложениях
\cite{semenov2015banachtraces,SU,strukova2015spectres}.
%Известны обобщения банаховых пределов на двойные последовательности
%\cite{edely2004almost}.

%TODO: ссылки! Хватит или ещё?


В настоящей работе рассматриваются некоторые вопросы асимптотических характеристик ограниченных последовательностей,
в том числе банаховых пределов.
Нумерация приводимых ниже теорем, лемм, определений и следствий совпадает с их нумерацией в диссертации.


В главе 1 обсуждается пространство $ac$ и его подпространство $ac_0$,
даётся критерий почти сходимости к нулю (т.е. принадлежности пространству $ac_0$)
знакопоcтоянной последовательности.

\reflecttheorem{thm:M_j_ac0_inf_lim}
	Пусть $n_i$~--- строго возрастающая последовательность натуральных чисел,
	\begin{equation}
		\label{eq:definition_M_j}
		M(j) = \liminf_{i\to\infty} n_{i+j} - n_i,
	\end{equation}
	\begin{equation}
		x_k = \left\{\begin{array}{ll}
			1, & \mbox{~если~} k = n_i
			\\
			0  & \mbox{~иначе~}
		\end{array}\right.
	\end{equation}
	Тогда следующие условия эквивалентны:
	\\
	(i)   $x \in ac_0$;
	\\\\
	(ii)  $\lim\limits_{j \to \infty} \dfrac{M(j)}{j} = +\infty$;
	\\\\
	(iii) $\inf\limits_{j \in \N}     \dfrac{M(j)}{j} = +\infty$.


\reflecttheorem{thm:crit_ac0_Mj_lambda}
	Пусть $x\in\ell_\infty$, $x \geq 0$, $\lambda>0$.
	Обозначим через $n^{(\lambda)}_i$ возрастающую последовательность
	индексов таких элементов $x$, что $x_k \geq \lambda$ тогда и только тогда,
	когда $k=n^{(\lambda)}_i$ для некоторого $i$.
	Обозначим
	\begin{equation}
		M^{(\lambda)}(j) = \liminf_{i\to\infty} n^{(\lambda)}_{i+j} - n^{(\lambda)}_i
		.
	\end{equation}


	Тогда для того, чтобы $x\in ac_0$, необходимо и достаточно, чтобы
	для любого $\lambda>0$ было выполнено
	\begin{equation}
		\lim_{j \to \infty} \frac{M^{(\lambda)}(j)}{j} = +\infty
		.
	\end{equation}

\reflecttheorem{thm:rho_x_c_leq_alpha_t_s_x_united}
	Для любого $x\in ac$
	\begin{equation}
		\frac{1}{2} \alpha(x) \leq \rho(x,c)\leq \lim_{s\to\infty} \alpha(T^s x)
		.
	\end{equation}

\reflectcorollary{cor:rho_x_c0_leq_alpha_t_s_x_united}
	Для любого $x\in ac_0$
	\begin{equation}
		\frac{1}{2} \alpha(x) \leq \rho(x,c_0)\leq \lim_{s\to\infty} \alpha(T^s x)
		.
	\end{equation}

\reflecttheorem{thm:Connor_generalized}
	Мера множества $F=\{x\in\Omega : q(x) = 0 \wedge p(x)= 1\}$,
	где $p(x)$ и $q(x)$~--- верхний и нижний функционалы Сачестона соответственно,
	равна 1.


В главе 2
%TODO: \ref ???
изучается $\alpha$--функция, введённая в~\cite{our-vzms-2018}:
\begin{equation}
	\alpha(x) = \varlimsup_{i\to\infty} \max_{i<j\leqslant 2i} |x_i - x_j|
	.
\end{equation}
%
%TODO: ссылка на статью Семенова!
%
Поскольку $\alpha(c)=0$,
то $\alpha$--функцию также можно считать <<мерой несходимости>> последовательности;
равенство $\alpha(x) = 0$, однако, вовсе не гарантирует сходимость.

Устанавливается, что $\alpha$--функция не инвариантна относительно оператора сдвига $T$,
и даётся оценка на $\alpha(T^n x)$.
С другой стороны, $\alpha$--функция, в отличие от некоторых банаховых пределов
\cite{Semenov2010invariant,Semenov2011dan},
инвариантна относительно операторов растяжения $\sigma_n$.
Затем выявляется связь между $\alpha$--функцией, расстоянием от заданнной последовательности до пространства $c$
и почти сходимостью.
Рассмотрены и другие свойства $\alpha$--функции.
Приведём основные результаты.

\reflectcorollary{thm:est_alpha_Tn_x_full}
	Для любых $x\in\ell_\infty$ и $n \in \N$
	\begin{equation}\label{est_alpha_Tn_x}
		\frac{1}{2}\alpha(x) \leq \alpha(T^n x) \leq \alpha(x)
		.
	\end{equation}

\reflecttheorem{thm:alpha_beta_T_seq}
	Пусть $\beta_k$~--- монотонная невозрастающая последовательность,
	$\beta_k \to \beta$, $\beta\in\left[\frac{1}{2}; 1\right]$, $\beta_1 \leq 1$.
	Тогда существует такой $x\in\ell_\infty$, что для любого натурального $n$
	\begin{equation}
		\frac{\alpha(T^n x)}{\alpha(x)} = \beta_n.
	\end{equation}

\reflecttheorem{thm:alpha_sigma_n}
	Для любого $x\in\ell_\infty$ и для любого натурального $n$ верно равенство
	\begin{equation}
		\alpha(\sigma_n x) = \alpha(x)
		.
	\end{equation}

\reflecttheorem{thm:alpha_sigma_1_n}
	Для любого $n\in\N$ и любого $x\in\ell_\infty$ выполнено
	\begin{equation}
		\alpha(\sigma_{1/n} x) \leq \left( 2- \frac{1}{n} \right) \alpha(x)
		.
	\end{equation}

\reflecttheorem{thm:alpha_Cx_no_gamma}
	Имеет место равенство
	\begin{equation}
		\sup_{x\in\ell_\infty, \alpha(x)\neq 0} \frac{\alpha(Cx)}{\alpha(x)}=1
		.
	\end{equation}

\reflecttheorem{thm:alpha_xy}
	Пусть $(x\cdot y)_k = x_k\cdot y_k$.
	Тогда
	$\alpha(x\cdot y)\leq \alpha(x)\cdot \|y\|_* + \alpha(y)\cdot \|x\|_*$,
	где
	\begin{equation}
		\|x\|_* = \limsup_{k\to\infty} |x_k|
	\end{equation}
	есть  фактор-норма по $c_0$ на пространстве $\ell_\infty$.

В параграфе~\ref{sec:space_A0} исследуется пространство $A_0 = \{x: \alpha(x) = 0\}$.
Это пространство несепарабельно, замкнуто относительно покоординатного умножения,
операторов левого и правого сдвигов, оператора Чезаро,
операторов растяжения $\sigma_n$ и усредняющего сжатия $\sigma_{1/n}$.

В параграфе~\ref{sec:noncomplementarity} устанавливается, что в цепочке вложений
\begin{equation}
	c_0 \subset A_0 \subset \ell_\infty
\end{equation}
оба подпространства недополняемы.

Как сказано выше, банаховы пределы по определению (как и обычный предел на пространстве $c$) инвариантны относительно оператора сдвига.
Возникает закономерный вопрос: можно ли потребовать от банахова предела сохранять своё значение
при суперпозиции с некоторыми другими операторами на $\ell_\infty$?
Эту проблему исследовал У. Эберлейн в 1950 г. \cite{Eberlein},
т.е. через два года после классической работы Г. Г. Лоренца~\cite{lorentz1948contribution}.
Эберлейн установил, что существуют такие линейные операторы  $A : \ell_\infty\to \ell_\infty$,
для которых $BAx = Bx$ независимо от выбора $x$ и для банаховых пределов специального вида.

Будем говорить, что $B\in\mathfrak B(A)$, $A : \ell_\infty\to \ell_\infty$, если для любого $x\in \ell_\infty$
выполнено равенство $BAx = Bx$.
Такой банахов предел $B$ называют инвариантным относительно оператора $A$.

Можно ли выделить какие-то особые свойства оператора сдвига,
которые необходимы или достаточны оператору, чтобы относительно него были инвариантны все или некоторые банаховы пределы?
Понятно, что если оператор $A$ таков, что для любого $x\in\ell_\infty$ между $Ax$ и $x$
существует (конечное) расстояние Дамерау--Левенштейна \cite{damerau1964technique} (т.е. минимальное количество операций вставки, удаления, замены и перестановки двух соседних элементов последовательности, необходимых для перевода $x$ в $Ax$, причём для разных $x\in\ell_\infty$ эти операции, вообще говоря, не обязаны быть одинаковыми), то относительно данного оператора инвариантен любой банахов предел. Аналогичное утверждение справедливо и в случае, если $Ax -x \in c_0$ для любого $x\in \ell_\infty$.

Следующим по естественности (после сдвига и замены конечного числа элементов) действием, сохраняющем сходимость последовательности, является повторение элементов последовательности, например, оператор
\begin{equation}
	\sigma_2(x_1,x_2,x_3,...) = (x_1,x_1, \; x_2, x_2, \; x_3, x_3, \; ...)
	.
\end{equation}
Однако относительно такого оператора инвариантны не все, а только некоторые банаховы пределы.
Так, в~\cite[теорема 14]{ASSU2} показано, что
\begin{equation}
	\B(\sigma_n) \cap \ext \B = \varnothing  \mbox{~~для любого~~} n\in\N_2
	.
\end{equation}
Заметим, что если мы рассмотрим оператор неравномерного растяжения
\begin{equation}
	\sigma_{1,2}(x_1,x_2,x_3,x_4,x_5,...) = (x_1, \; x_2, x_2, \;  x_3, \; x_4, x_4, \; x_5, ...)
	,
\end{equation}
то увидим, что периодическую последовательность $y_n = (-1)^n$, $y\in ac_0$ оператор $\sigma_{1,2}$
переводит в периодическую последовательность
\begin{equation}
	(-1, 1, 1, \; -1, 1, 1, \; ...) \in ac_{1/3}
	,
\end{equation}
поскольку на периодической последовательности любой банахов предел принимает значение, равное среднему по периоду.
Таким образом, не существует банаховых пределов, инвариантных относительно оператора $\sigma_{1,2}$.

В главе 3 изложены некоторые примеры операторов и найдены множества банаховых пределов,
инвариантных относительно этих операторов.
Затем рассматриваются следующие классы линейных операторов $H:\ell_\infty \to \ell_\infty$:

-- полуэберлейновы: такие, что $B_1 H \in\B$ для некоторого $B_1\in \B$;

-- эберлейновы: такие, что $B_1 H = B_1$ для некоторого $B_1\in \B$;

-- В-регулярные: такие, что $B_1 H \in \B$ для любого   $B_1\in \B$;

-- существенно эберлейновы: такие, что $B_1 H \in \B$ для любого $B_1\in \B$ и $B_2 H \ne B_2$ для некоторого $B_2\in \B$.

Устанавливается (см. теоремы~\ref{thm:amiable_but_not_Eberlein_exists} и~\ref{thm:Eberlein_but_not_B-regular_exists}),
что каждый следующий из этих классов вложен в предыдущий и не совпадает с ним.
Кроме того, доказывается ещё ряд смежных результатов, в частности, решается обратная задача об инвариантности.

\reflecttheorem{thm:generated_operator_G_B}
	Для каждого $B\in \B$ существует такой оператор $G_B:\ell_\infty \to \ell_\infty$,
	что $\B(G_B) = \{B\}$.


Главы 4 и 5 посвящены верхнему и нижнему функционалам Сачестона $p(x)$ и $q(x)$"--- аналогам верхнего и нижнего пределов последовательности.
В главе 4 изучаются разделяющие множества.

Обозначим через $\Omega$ множество всех последовательностей, состоящих из нулей и единиц.

\reflecttheorem{thm:Lin_Omega_Sucheston}
	Пусть
	$1 \geq a > b \geq 0$ и
	$\Omega^a_b = \{x\in\Omega : p(x) = a, q(x) = b\}$,
	где $p(x)$ и $q(x)$~--- верхний и нижний функционалы Сачестона~\cite{sucheston1967banach} соответственно.
	Тогда $\Omega \subset \operatorname{Lin} \Omega^a_b$.


\reflectcorollary{crl:Lin_Omega_Sucheston}
	Множество $\Omega^a_b$ является разделяющим.
	Т.к. при $a\neq 1$ или $b\neq 0$ множество $\Omega^a_b$ имеет меру нуль~\cite{semenov2010characteristic,connor1990almost},
	то оно является разделяющим множеством нулевой меры.


Пусть $X^a_b = \{x\in\ell_\infty : p(x) = a,~ q(x) = b\}$, $Y^a_b = \{x\in A_0 : p(x) = a, q(x) = b\}$, где $a>b$.
\reflecttheorem{thm:A_0_c_infty_lin}
	Пусть $a\neq -b$.
	Тогда справедливо равенство $\operatorname{Lin} Y^a_b = A_0$.

\reflecttheorem{thm:Lin_ell_infty}
	Справедливо равенство $\operatorname{Lin} X^a_b = \ell_\infty$.

Через $\dim_H E$ будем обозначать хаусдорфову размерность множества $E$.

\reflecttheorem{thm:Hausdorf_measure_1_n}
	Пусть $n\in\N$.
	Тогда существует разделяющее множество $E\subset\Omega$ такое,
	что $\dim_H E = 1/n$.

%Конец главы 4

Глава 5 посвящена связи мультипликативных свойств носителя последовательности из нулей и единиц
и значений, которые могут принимать функционалы Сачестона на такой последовательности.


\reflectcorollary{cor:ac0_powers_finite_set_of_numbers}
	Пусть $\{p_1, ..., p_k\} \subset \N$,
	\begin{equation}
		x_k = \begin{cases}
			1, &\mbox{~если~} k = p_1^{j_1}\cdot p_2^{j_2}\cdot ... \cdot p_k^{j_k} \mbox{~для некоторых~} j_1,...,j_k\in\N,
			\\
			0  &\mbox{~иначе}.
		\end{cases}
	\end{equation}
	Тогда $x\in ac_0$.

\reflectdefinition{def:P-property}
	Будем говорить, что множество $A\subset\N$ обладает $P$-свойством,
	если для любого $n\in\N$ найдётся набор попарно взаимно простых чисел
	\begin{equation}
		\{a_{n,1}, a_{n,2}, ..., a_{n,n}  \} \subset A
		.
	\end{equation}

\reflecttheorem{thm:p_x_infinite_multiples}
	Пусть $A\subset \N\setminus\{1\}$.
	Тогда следующие условия эквивалентны:
	\begin{enumerate}[label=(\roman*)]
		\item
			$A$ обладает $P$-свойством
		\item
			В $A$ существует бесконечное подмножество попарно взаимно простых чисел
		\item
			$p(\chi\mathscr{M}A)=1$.
	\end{enumerate}

\reflectcorollary{cor:ac0_primes_p_psi_A_prod}
	Пусть $A = \{a_1, a_2, ..., a_n,...\}$ "--- бесконечное множество попарно взаимно простых чисел
	и $a_{n+1}>a_1\cdot...\cdot a_n$.
	Тогда
	\begin{equation}
		q(\chi\mathscr{M}A) = 1-\prod_{j=1}^\infty \left(1-\frac{1}{a_j}\right)
		.
	\end{equation}

\reflectlemma{lem:q_x_infinite_Euler}
	Пусть $\varepsilon \in  (0; 1{]}$.
	Существует бесконечное множество попарно непересекающихся подмножеств простых чисел
	$A_i$ такое, что $q(\chi\mathscr{M}A_i)\geq\varepsilon$ для любого $i\in\N$.


\chapter{$\alpha$--функция как асимптотическая характеристика ограниченной последовательности}

	\section{Определение и элементарные свойства $\alpha$--функции}
	На пространстве $\ell_\infty$ определяется $\alpha$--функция следующим равенством:
\begin{equation}
	\alpha(x) = \varlimsup_{i\to\infty} \max_{i<j\leqslant 2i} |x_i - x_j|
	.
\end{equation}
Иногда удобнее использовать одно из нижеследующих равносильных определений:
\begin{equation}
	\alpha(x) = \varlimsup_{i\to\infty} \max_{i \leqslant j\leqslant 2i} |x_i - x_j|
	,
\end{equation}
\begin{equation}
	\alpha(x) = \varlimsup_{i\to\infty} \sup_{i<j\leqslant 2i} |x_i - x_j|
	,
\end{equation}
\begin{equation}
	\alpha(x) = \varlimsup_{i\to\infty} \sup_{i \leqslant j\leqslant 2i} |x_i - x_j|
	.
\end{equation}

Легко видеть, что $\alpha$--функция неотрицательна.
\begin{property}
	\label{thm:alpha_x_triangle_ineq}
	Более того, $\alpha$--функция удовлетворяет неравенству треугольника:
	\begin{equation}
		\alpha(x+y) \leq \alpha(x) + \alpha(y)
		.
	\end{equation}
\end{property}
TODO: доказывать нужно или очевидно?

\begin{property}
	На пространстве $\ell_\infty$ $\alpha$--функция удовлетворяет условию Липшица:
	\begin{equation}\label{alpha_Lipshitz}
		|\alpha(x) - \alpha(y)| \leq 2 \|x-y\|
		.
	\end{equation}
	(и эта оценка точна).
	TODO: доказывать нужно или очевидно?
\end{property}

\begin{property}
	Если $y\in c$, то $\alpha(y) = 0$ и $\alpha(x+y) = \alpha(x)$ для любого $x \in \ell_\infty$.
\end{property}

Кроме того, иногда полезно помнить про следующее очевидное
\begin{property}
	\label{thm:alpha_x_leq_limsup_minus_liminf}
	\begin{equation}
		\alpha(x) \leq \varlimsup_{k\to\infty} x_k - \varliminf_{k\to\infty} x_k
		.
	\end{equation}
\end{property}


	\section{$\alpha$--функция и оператор сдвига $T$}
	Результаты этого пункта опубликованы в~\cite{our-ped-2018-alpha-Tx} и используются в~\cite{our-mz2019ac0}.

Как выясняется, $\alpha$--функция не инвариантна относительно оператора сдвига
\begin{equation}
	T(x_1,x_2,x_3,...) = (x_2, x_3, ...).
\end{equation}

\begin{example}
\label{ex:alpha_x_neq_alpha_Tx}
	Пусть
	\begin{equation}
		x_k = \begin{cases}
			(-1)^n, & \mbox{~если~} k = 2^n
			\\
			0 & \mbox{~иначе~}
		\end{cases}
	\end{equation}
\end{example}

Вычислим $\alpha(x)$.
Заметим сначала, что из принадлежности $x_k\in\{-1,0,1\}$
немедленно следует, что $\alpha(x) \leq 2$.
Оценим теперь $\alpha(x)$ снизу:
\begin{multline}
	\alpha(x)
	=
	\varlimsup_{i\to\infty}\max_{i < j \leqslant 2i} |x_i - x_j|
	\geq
	\\\geq
	\mbox{(переход к частичному верхнему пределу
	}\\ \mbox{
	по индексам специального вида $i=2^n$)}
	\geq
	\\\geq
	\varlimsup_{n\to\infty}\max_{2^n < j \leqslant 2^{n+1}} |x_{2^n} - x_j|
	=
	\varlimsup_{n\to\infty}\max_{2^n < j \leqslant 2^{n+1}} |(-1)^n - x_j|
	\geq
	\\ \geq
	\varlimsup_{n\to\infty} |(-1)^n - x_{2^{n+1}}|
	=
	\varlimsup_{n\to\infty} |(-1)^n - (-1)^{n+1}|
	=
	2
	.
\end{multline}

Итак, $\alpha(x) = 2$.
Вычислим теперь $\alpha(Tx)$:
\begin{multline}
	\alpha(Tx)
	=
	\varlimsup_{i\to\infty}~\max_{i < j \leqslant 2i} |(Tx)_i - (Tx)_j|
	=
	\varlimsup_{i\to\infty}~\max_{i < j \leqslant 2i} |x_{i+1} - x_{j+1}|
	=
	\\=
	(\mbox{замена}~k:=i+1, m:=j+1)
	=
	\\=
	\varlimsup_{k\to\infty}~~\max_{k-1 < m-1 \leqslant 2k-2} |x_k - x_m|
	=
	\varlimsup_{k\to\infty}~~\max_{k < m \leqslant 2k-1} |x_k - x_m|
	=
	\\=
	\max\left\{
		\varlimsup_{k\to\infty, k  =   2^n}~~\max_{k < m \leqslant 2k-1} |x_k - x_m|
		,~~
		\varlimsup_{k\to\infty, k \neq 2^n}~~\max_{k < m \leqslant 2k-1} |x_k - x_m|
	\right\}
	=
	\\=
	\max\left\{
		\varlimsup_{n\to\infty}~~\max_{2^n < m \leqslant 2^{n+1}-1} |x_{2^n} - x_m|
		,~~
		\varlimsup_{k\to\infty, k \neq 2^n}~~\max_{k < m \leqslant 2k-1} |x_k - x_m|
	\right\}
	=
	\\=
	\max\left\{
		\varlimsup_{n\to\infty}~~\max_{2^n < m \leqslant 2^{n+1}-1} |(-1)^n - x_m|
		,~~
		\varlimsup_{k\to\infty, k \neq 2^n}~~\max_{k < m \leqslant 2k-1} |x_k - x_m|
	\right\}
	=
	\\=
	\max\left\{
		\varlimsup_{n\to\infty}~~\max_{2^n < m \leqslant 2^{n+1}-1} |(-1)^n - 0|
		,~~
		\varlimsup_{k\to\infty, k \neq 2^n}~~\max_{k < m \leqslant 2k-1} |x_k - x_m|
	\right\}
	=
	\\=
	\max\left\{
		1
		,~
		\varlimsup_{k\to\infty, k \neq 2^n}~~\max_{k < m \leqslant 2k-1} |x_k - x_m|
	\right\}
	=
	\\=
	\mbox{(если $k \neq 2^n$, то $x_k = 0$)}
	=
	\\=
	\max\left\{
		1
		,~
		\varlimsup_{k\to\infty, k \neq 2^n}~~\max_{k < m \leqslant 2k-1} |0 - x_m|
	\right\}
	=
	1
	.
\end{multline}
Таким образом, $\alpha(Tx) = 1 \neq 2 = \alpha(x)$,
что и требовалось показать.

Верна следующая
\begin{theorem}
	Для любого $x \in \ell_\infty$ выполнено неравенство $\alpha(Tx)\leq \alpha(x)$.
\end{theorem}

\begin{proof}
	\begin{multline}
		\alpha(Tx)
		=
		\varlimsup_{i\to\infty}~\max_{i < j \leqslant 2i} |(Tx)_i - (Tx)_j|
		=
		\varlimsup_{i\to\infty}~\max_{i < j \leqslant 2i} |x_{i+1} - x_{j+1}|
		=
		\\=
		(\mbox{замена}~k:=i+1, m:=j+1)
		=
		\\=
		\varlimsup_{k\to\infty}~~\max_{k-1 < m-1 \leqslant 2k-2} |x_k - x_m|
		=
		\varlimsup_{k\to\infty}~~\max_{k < m \leqslant 2k-1} |x_k - x_m|
		\leq
		\\ \leq
		\mbox{(переход к максимуму по большему множеству)}
		\leq
		\\ \leq
		\varlimsup_{k\to\infty}~~\max_{k < m \leqslant 2k} |x_k - x_m|
		=
		\alpha(x)
		.
	\end{multline}
\end{proof}

Более интересна, однако, следующая оценка.
\begin{theorem}
	Для любого $n\in\N$
	\begin{equation}
		\alpha(T^n x) \geq \frac{1}{2} \alpha(x)
		.
	\end{equation}
\end{theorem}

\begin{proof}
	Зафиксируем $n$.
	Заметим, что
	\begin{equation}
		\alpha(x) = \varlimsup_{i\to\infty} \alpha_i(x),
	\end{equation}
	где
	\begin{equation}
		\alpha_i(x) = \max_{i < j \leqslant 2i} |x_i - x_j|.
	\end{equation}

	Рассмотрим $\alpha_i(x)$ при некотором фиксированном $i$, $i>2n$ (меньшие $i$ не влияют на верхний предел).
	Если
	\begin{equation}
		\max_{i < j \leqslant 2i} |x_i - x_j|
		=
		\max_{i < j \leqslant 2i-n} |x_i - x_j|
		,
	\end{equation}
	то
	\begin{multline}\label{alpha_i(x)_leq_alpha_{i-n}(T_n x)}
		\alpha_i(x)
		=
		\max_{i < j \leqslant 2i} |x_i - x_j|
		=
		\max_{i < j \leqslant 2i-n} |x_i - x_j|
		=
		\\=
		\mbox{(замена $k=i-n$, $m=j-n$)}
		=
		\\=
		\max_{k+n < m+n \leqslant 2k+n} |x_{k+n} - x_{m+n}|
		=
		\max_{k < m \leqslant 2k} |(T^n x)_k - (T^n x)_m|
		=
		\alpha_{i-n}(T^n x)
		.
	\end{multline}
	Иначе
	\begin{equation}
		\max_{i < j \leqslant 2i} |x_i - x_j|
		=
		\max_{2i-n < j \leqslant 2i} |x_i - x_j|
	\end{equation}
	и можно записать, что
	\begin{multline}\label{alpha_i(x)_leq_alpha_{i-n}(T_n x) + alpha_{2i-2n}(T_n x)}
		\alpha_i(x)
		=
		\max_{2i-n < j \leqslant 2i} |x_i - x_j|
		=
		\max_{2i-n < j \leqslant 2i} |x_i - x_{2i-n} + x_{2i-n} - x_j|
		\leq
		\\\leq
		\max_{2i-n < j \leqslant 2i} \left( |x_i - x_{2i-n}| + |x_{2i-n} - x_j| \right)
		=
		\\=
		|x_i - x_{2i-n}| + \max_{2i-n < j \leqslant 2i} |x_{2i-n} - x_j|
		=
		\\=
		|x_{i-n+n} - x_{2(i-n)+n}| + \max_{2i-n < j \leqslant 2i} |x_{2i-n} - x_j|
		=
		\\=
		|(T^n x)_{i-n} - (T^n x)_{2(i-n)}| + \max_{2i-n < j \leqslant 2i} |x_{2i-n} - x_j|
		\leq
		\\ \leq
		\alpha_{i-n}(T^n x) + \max_{2i-n < j \leqslant 2i} |x_{2i-n} - x_j|
		\leq
		\\ \leq
		\alpha_{i-n}(T^n x) + \max_{2i-n < j \leqslant 2i} |(T^n x)_{2i-2n} - (T^n x)_{j-n}|
		=
		\\=
		(\mbox{замена}~ m:=j-n ~)
		=
		\\=
		\alpha_{i-n}(T^n x) + \max_{2i-2n < m \leqslant 2i-n} |(T^n x)_{2i-2n} - (T^n x)_m|
		\leq
		\\ \leq
		\mbox{(т.к. $i>2n$, то $4i-4n > 2i+4n-4n = 2i > 2i-n$)}
		\leq
		\\ \leq
		\alpha_{i-n}(T^n x) + \max_{2i-2n < m \leqslant 4i-4n} |(T^n x)_{2i-2n} - (T^n x)_m|
		=
		\alpha_{i-n}(T^n x) + \alpha_{2i-2n}(T^n x)
		.
	\end{multline}

	Сравнивая \eqref{alpha_i(x)_leq_alpha_{i-n}(T_n x)} и \eqref{alpha_i(x)_leq_alpha_{i-n}(T_n x) + alpha_{2i-2n}(T_n x)},
	делаем вывод, что
	\begin{equation}
		\alpha_i(x) \leq \alpha_{i-n}(T^n x) + \alpha_{2i-2n}(T^n x)
		.
	\end{equation}
	Переходя к верхнему пределу, имеем
	\begin{multline}
		\alpha(x)
		=
		\varlimsup_{i\to\infty} \alpha_i(x)
		\leq
		\varlimsup_{i\to\infty} (\alpha_{i-n}(T^n x) + \alpha_{2i-2n}(T^n x))
		\leq
		\\ \leq
		\varlimsup_{i\to\infty} \alpha_{i-n}(T^n x) + \varlimsup_{i\to\infty} \alpha_{2i-2n}(T^n x)
		=
		\\=
		\varlimsup_{j=i-n, j\to\infty} \alpha_{j}(T^n x) + \varlimsup_{i\to\infty} \alpha_{2i-2n}(T^n x)
		=
		\\=
		\alpha(T^n x) + \varlimsup_{i\to\infty} \alpha_{2i-2n}(T^n x)
		\leq
		\\ \leq
		\mbox{(верхний предел по индексам специального вида
		} \\ \mbox{
		заменим на верхний предел по всем индексам)}
		\leq
		\\ \leq
		\alpha(T^n x) + \varlimsup_{k\to\infty} \alpha_{k}(T^n x)
		=
		\alpha(T^n x) + \alpha(T^n x)
		=
		2 \alpha(T^n x)
		.
	\end{multline}
	Таким образом, $\alpha(T^n x) \geq \frac{1}{2} \alpha(x)$,
	что и требовалось доказать.
\end{proof}

Две предыдущие теоремы немедленно влекут

\begin{corollary}
	\label{thm:est_alpha_Tn_x_full}
	Для любых $x\in\ell_\infty$ и $n \in \N$
	\begin{equation}\label{est_alpha_Tn_x}
		\frac{1}{2}\alpha(x) \leq \alpha(T^n x) \leq \alpha(x)
		.
	\end{equation}
\end{corollary}

\begin{remark}
	Оценки \eqref{est_alpha_Tn_x} точны: нижняя достигается, например,
	в примере выше, верхняя же~--- на любой периодической последовательности.
\end{remark}


	\section{О характере сходимости последовательности $\alpha(T^n x)/ \alpha(x)$}
	В зависимости от выбора $x$ последовательность $\left\{\frac{\alpha(T^n x)}{\alpha(x)}\right\}$
может монотонно сходиться к любому числу из отрезка $\left[\frac{1}{2}; 1\right]$ с любой скоростью
(в том числе и сколь угодно медленно).
Говоря строже, верна следующая

\begin{theorem}
	\label{thm:alpha_beta_T_seq}
	Пусть $\beta_k$~--- монотонная невозрастающая последовательность,
	$\beta_k \to \beta$, $\beta\in\left[\frac{1}{2}; 1\right]$, $\beta_1 \leq 1$.
	Тогда существует такой $x\in\ell_\infty$, что для любого натурального $n$
	\begin{equation}
		\frac{\alpha(T^n x)}{\alpha(x)} = \beta_n.
	\end{equation}
\end{theorem}

\begin{proof}
	Для удобства обозначим $\beta_0 = 1$.
	Это логично, так как
	\begin{equation}
		\frac{\alpha(T^0 x)}{\alpha(x)} = \frac{\alpha(x)}{\alpha(x)} = 1
		.
	\end{equation}

	Пусть $m\geq 3$, $m\in\N$.
	Положим
	\begin{equation}
		x_k = \begin{cases}
			0,  & \mbox{если } k = 2^{2m}     \\
			\beta_l,  & \mbox{если } 2^{2m} < k = 2^{2m+1}-l, l\in\N\cup\{0\}     \\
			\dfrac{1}{2}                    & \mbox{иначе.}
		\end{cases}
	\end{equation}

	Тогда $\alpha(x) = 1$.
	Действительно, $\left| x_{2^{2m}} - x_{2^{2m+1}} \right| =1$,
	а по свойству \ref{thm:alpha_x_leq_limsup_minus_liminf} $\alpha(x) \leq 1$.

	Пусть
	\begin{equation}
		\alpha_{i,n}(x)= \max_{i< j \leq 2i - n} |x_i - x_j|
		,
		\quad
		i>n
		,
	\end{equation}
	тогда
	\begin{equation}
		\alpha(T^n x) = \varlimsup_{i \to \infty} \alpha_{i,n}(x)
		.
	\end{equation}

	Вычислим теперь все $\alpha_{i,n}(x)$.
	Заметим, что
	\begin{multline}
		\alpha_{2^{2m}, n} (x)
		=
		\max_{2^{2m}< j \leq 2^{2m+1} - n} |0 - x_j|
		=
		\max_{2^{2m}< j \leq 2^{2m+1} - n} x_j
		=
		\\=
		\mbox{(замена: $q = 2^{2m+1} - j$, тогда $j = 2^{2m+1} - q$)}
		=
		\\=
		\max_{2^{2m}< 2^{2m+1} - q \leq 2^{2m+1} - n} x_{2^{2m+1} - q}
		=
		\max_{0< 2^{2m} - q \leq 2^{2m} - n} x_{2^{2m+1} - q}
		=
		\max_{n \leq q < 2^{2m}} x_{2^{2m+1} - q}
		=
		\\=
		\max_{n \leq q < 2^{2m}} \beta_q
		=
		%\\=
		\mbox{(в силу невозрастания $\beta_q$)}
		=
		\beta_n
		\geq
		\frac{1}{2}
		.
	\end{multline}

	Пусть теперь $i$ таково, что $2^{2m+1}<i<2^{2m+2}$,
	тогда $x_i = \frac{1}{2}$.
	Так как
	$\forall(j)\left[x_j\in[0;1]\right]$,
	то
	$|x_i - x_j| \leq \frac{1}{2}$
	и, следовательно,
	$\alpha_{i,n}(x)  \leq \frac{1}{2}$.

	Пусть, наконец, $i$ таково, что $2^{2m}<i \leq 2^{2m+2}$,
	тогда
	$x_i = \beta_k \in [1/2;1]$.
	Для таких $j$, что $i<j\leq 2i-n$ и, более того,
	для любых таких $j$, что $2^{2m}<j<2^{2m+2}$
	выполнено $x_j\in[1/2; 1]$
	и, значит, снова $|x_i - x_j| \leq \frac{1}{2}$.

	Таким образом, получаем, что
	\begin{equation}
		\alpha(T^n x) = \varlimsup_{i \to \infty} \alpha_{i,n}(x) = \beta_n
		.
	\end{equation}
\end{proof}


	\section{О множествах $\{x: \alpha(T^n x) = \alpha(x)\}$}
	В данном параграфе обсуждаются некоторые свойства множеств
\begin{equation}
	\label{eq:alpha_T^n_x_equiv_alpha_x}
	\{x \in \ell_\infty : \alpha(T^n x) = \alpha(x) \}, ~n\in\N,
\end{equation}
\begin{equation}
	\label{eq:cap_alpha_T^n_x_equiv_alpha_x}
	\bigcap\limits_{n\in\N}\{x \in \ell_\infty : \alpha(T^n x) = \alpha(x) \}
	,
\end{equation}
\begin{equation}
	\label{eq:cup_alpha_T^n_x_equiv_alpha_x}
	\bigcup_{n\in\N}\{x \in \ell_\infty : \alpha(T^n x) = \alpha(x) \}
	.
\end{equation}

\subsection{Аддитивные свойства}

\begin{theorem}
	Ни одно из множеств
	\eqref{eq:alpha_T^n_x_equiv_alpha_x}, \eqref{eq:cap_alpha_T^n_x_equiv_alpha_x}, \eqref{eq:cup_alpha_T^n_x_equiv_alpha_x}
	не замкнуто по сложению и, следовательно, не является пространством.
\end{theorem}

\begin{proof}
	Построим два таких элемента, принадлежащих множеству \eqref{eq:alpha_T^n_x_equiv_alpha_x} при любых $n\in\N$,
	сумма которых не принадлежит множеству \eqref{eq:alpha_T^n_x_equiv_alpha_x} ни при каких $n\in\N$.
	Пусть $m\in\N_3$.
	Положим

	\begin{equation}
		x_k = \begin{cases}
			\dfrac{1}{2}(-1)^m,  & \mbox{если } k = 2^m     \\
			1,                   & \mbox{если } k = 2^m + 1 \\
			-1,                  & \mbox{если } k = 2^m + 2 \\
			0                    & \mbox{иначе }
		\end{cases}
	\end{equation}

	и

	\begin{equation}
		y_k = \begin{cases}
			\dfrac{1}{2}(-1)^m,  & \mbox{если } k = 2^m     \\
			-1,                  & \mbox{если } k = 2^m + 1 \\
			1,                   & \mbox{если } k = 2^m + 2 \\
			0                    & \mbox{иначе }
		\end{cases}
	\end{equation}

	Так как
	\begin{equation}
		(T^n x)_{2^m-n+1} - (T^n x)_{2^m-n+2} = 2
	\end{equation}
	и
	\begin{equation}
		(T^n y)_{2^m-n+1} - (T^n y)_{2^m-n+2} = -2
		,
	\end{equation}
	то
	\begin{equation}
		\alpha(x) = \alpha(T^n x) = \alpha(y) = \alpha(T^n y) = 2
		.
	\end{equation}
	С другой стороны,
	\begin{equation}
		(x+y)_k = \begin{cases}
			(-1)^m,  & \mbox{если } k = 2^m     \\
			0        & \mbox{иначе }
		\end{cases}
	\end{equation}
	и
	\begin{equation}
		\alpha(x+y) = 2
		,
	\end{equation}
	но в то же время
	\begin{equation}
		\alpha(T^n(x+y)) = 1
	\end{equation}
	(см. пример \ref{ex:alpha_x_neq_alpha_Tx}),
	следовательно, $x+y$ не принадлежит ни одному из множеств
	\eqref{eq:alpha_T^n_x_equiv_alpha_x}, \eqref{eq:cap_alpha_T^n_x_equiv_alpha_x}, \eqref{eq:cup_alpha_T^n_x_equiv_alpha_x}.
\end{proof}




\subsection{Мультипликативные свойства}

\begin{theorem}
	Ни одно из множеств
	\eqref{eq:alpha_T^n_x_equiv_alpha_x}, \eqref{eq:cap_alpha_T^n_x_equiv_alpha_x}, \eqref{eq:cup_alpha_T^n_x_equiv_alpha_x}
	не замкнуто по умножению.
\end{theorem}

\begin{proof}
	Снова построим два таких элемента, принадлежащих множеству \eqref{eq:alpha_T^n_x_equiv_alpha_x} при любых $n\in\N$,
	произведение которых не принадлежит множеству \eqref{eq:alpha_T^n_x_equiv_alpha_x} ни при каких $n\in\N$.
	Пусть $m\in\N_3$.
	Положим

	\begin{equation}
		x_k = \begin{cases}
			(-1)^m,  & \mbox{если } k = 2^m     \\
			1,                   & \mbox{если } k = 2^m + 1 \\
			0                    & \mbox{иначе }
		\end{cases}
	\end{equation}

	и

	\begin{equation}
		y_k = \begin{cases}
			(-1)^{m+1},  & \mbox{если } k = 2^m     \\
			1,                   & \mbox{если } k = 2^m + 2 \\
			0                    & \mbox{иначе }
		\end{cases}
	\end{equation}

	Так как
	\begin{equation}
		(T^n x)_{2^{2m+1}-n} - (T^n x)_{2^{2m+1}-n+1} = -2
	\end{equation}
	и
	\begin{equation}
		(T^n y)_{2^{2m}-n} - (T^n y)_{2^{2m}-n+2} = -2
		,
	\end{equation}
	то
	\begin{equation}
		\alpha(x) = \alpha(T^n x) = \alpha(y) = \alpha(T^n y) = 2
		.
	\end{equation}
	С другой стороны,
	\begin{equation}
		(x\cdot y)_k = \begin{cases}
			(-1)^m,  & \mbox{если } k = 2^m     \\
			0        & \mbox{иначе }
		\end{cases}
	\end{equation}
	и
	\begin{equation}
		\alpha(x+y) = 2
		,
	\end{equation}
	но в то же время
	\begin{equation}
		\alpha(T^n(x \cdot y)) = 1
	\end{equation}
	(см. пример \ref{ex:alpha_x_neq_alpha_Tx}),
	следовательно, $x \cdot y$ не принадлежит ни одному из множеств
	\eqref{eq:alpha_T^n_x_equiv_alpha_x}, \eqref{eq:cap_alpha_T^n_x_equiv_alpha_x}, \eqref{eq:cup_alpha_T^n_x_equiv_alpha_x}.

\end{proof}


	\section{$\alpha$--функция и семейство операторов $\sigma_n$}
	\paragraph{Теорема.}
$$
	\forall(x\in l_\infty) \forall(n\in\mathbb{N})
	[
		\alpha(\sigma_n x) = \alpha(x)
	]
$$

\paragraph{Доказательство.}
По определению
\begin{equation}
	\alpha(x) = \varlimsup_{i\to\infty} \max_{i<j\leqslant 2i} |x_i - x_j|
\end{equation}

Положим
\begin{equation}
	\alpha_i(x) =
	\max_{i<j\leqslant 2i} |x_i - x_j| =
	\max_{i\leqslant j\leqslant 2i} |x_i - x_j|
\end{equation}

Тогда
\begin{equation}
	\alpha(x) = \varlimsup_{i\to\infty} \alpha_i(x)
\end{equation}

Пусть $y = \sigma_n x$.
Тогда для $k=1, ..., n-1$, $a\in\mathbb{N}$ имеем
\begin{multline}
	\alpha_{an-k}(y) =
	\max_{an-k \leqslant j \leqslant 2an-2k} |y_{an-k} - y_j| =
	\\=
	(\mbox{т.к.}~y_{an-(n-1)}=y_{an-(n-2)}=...=y_{an-k}=...=y_{an-1}=y_{an})=
	\\=
	\max_{an \leqslant j \leqslant 2an-2k} |y_{an} - y_j| \leqslant
	\\ \leqslant
	(\mbox{переходим к максимуму по большему множеству}) \leqslant
	\\ \leqslant
	\max_{an \leqslant j \leqslant 2an} |y_{an} - y_j| =
	\alpha_{an}(y)
\end{multline}

С другой стороны,
\begin{multline}
	\alpha_{an}(y) =
	\max_{an \leqslant j \leqslant 2an} |y_{an} - y_j| =
	\\ =
	(\mbox{т.к.}~y=\sigma_n x,~\mbox{можем рассматривать только}~j=kn)=
	\\ =
	\max_{an \leqslant kn \leqslant 2an} |y_{an} - y_{kn}| =
	\max_{a \leqslant k \leqslant 2a} |y_{an} - y_{kn}| =
	\max_{a \leqslant k \leqslant 2a} |x_a - x_k| =
	\alpha_a(x)
\end{multline}

Таким образом, для $k=1, ..., n-1$, $a\in\mathbb{N}$ имеем соотношения:
\begin{gather}
	\alpha_{an}(y) = \alpha_a(x),
\\
	\alpha_{an-k}(y) \leqslant \alpha_a(x),
\end{gather}
откуда немедленно следует, что
\begin{equation}
	\varlimsup_{i\to\infty} \alpha_i(y) =
	\varlimsup_{i\to\infty} \alpha_i(x),
\end{equation}
т.е.
\begin{equation}
	\alpha(\sigma_n x) = \alpha(x),
\end{equation}
что и требовалось доказать.

TODO: ссылка на статью Семёнова

\paragraph{Следствие.}
$$
	\alpha(C\sigma_2 x) =
	\alpha(\sigma_2 Cx) =
	\alpha(Cx)
$$


	\section{$\alpha$--функция и семейство операторов $\sigma_{1/n}$}
	Введём, следуя~\cite[p. 131, prop. 2.b.2]{lindenstrauss1979classical},
на $\ell_\infty$ оператор
\begin{equation}
	\sigma_{1/n} x = n^{-1}
	\left(
		\sum_{i=1}^{n} x_i,
		\sum_{i=n+1}^{2n} x_i,
		\sum_{i=2n+1}^{3n} x_i,
		...
	\right).
\end{equation}

Понятно, что если последовательность $x$~--- периодическая с периодом $n$,
то $\alpha(\sigma_{1/n}x)=0$.
Значит, оценить $\alpha(\sigma_{1/n}x)$ снизу через $\alpha(x)$ не удастся.

Для построения верхней оценки нам потребуется следующая

\begin{lemma}
	\label{thm:distance_from_average}
	Пусть $a=\frac{a_1+...+a_n}{n}$, $a_1 \leq ... \leq a_n$.
	Тогда $a_n - a \leq \frac{n-1}{n} (a_n - a_1)$.
\end{lemma}

\begin{proof}
	\begin{multline}
		a_n - a = a_n - \frac{a_1+...+a_n}{n}
		=
		\frac{n-1}{n}a_n - \frac{a_1+...+a_{n-1}}{n}
		\leq
		\\\leq
		\frac{n-1}{n}a_n - \frac{(n-1)a_1}{n}
		=
		\frac{n-1}{n}(a_n - a_1)
		.
	\end{multline}
\end{proof}

\begin{theorem}
	\label{thm:alpha_sigma_1_n}
	Для любого $n\in\N$ и любого $x\in\ell_\infty$ выполнено
	\begin{equation}
		\alpha(\sigma_{1/n} x) \leq \left( 2- \frac{1}{n} \right) \alpha(x)
		.
	\end{equation}
\end{theorem}

\begin{proof}
	Положим
	\begin{equation}
		\alpha_i(x) =
		\max_{i<j\leqslant 2i} |x_i - x_j| =
		\max_{i\leqslant j\leqslant 2i} |x_i - x_j|
		.
	\end{equation}
	Тогда
	\begin{equation}
		\alpha(x) = \varlimsup_{i\to\infty} \alpha_i(x)
		.
	\end{equation}
	Пусть $y=\sigma_n \sigma_{1/n} x$.
	Из теоремы~\ref{thm:alpha_sigma_n} следует, что $\alpha(\sigma_{1/n} x)=\alpha(y)$.
	Сосредоточим наши усилия на оценке $\alpha(y)$.

	Пусть $1\leq j \leq n$.
	Заметим, что
	\begin{multline}
		\alpha_{kn+j}(y)
		=
		\max_{kn+j \leq i \leq 2kn+2j } |y_{kn+j} - y_i|
		=
		\\=
		\mbox{(т.к. $y_{kn+j}=y_{kn+1}=y_{kn+n} = (\sigma_{1/n}x)_k$)}
		=
		\\=
		\max_{kn+n \leq i \leq 2kn+2j } |y_{kn+n} - y_i|
		\leq
		\\\leq
		\mbox{(переходим к максимуму по не меньшему множеству)}
		\leq
		\\\leq
		\max_{kn+n \leq i \leq 2kn+2n } |y_{kn+n} - y_i|
		=
		\alpha_{kn+n}(y)
		.
	\end{multline}

	Итак, $\alpha_{kn+j}(y) \leq \alpha_{kn+n}(y)$,
	значит,
	\begin{equation}
		\label{eq:alpha_sigma_1_n_subseq_limsup}
		\alpha(\sigma_{1/n}x) = \alpha(y) = \varlimsup_{i\to\infty} \alpha_i(y)
		=
		\varlimsup_{k\to\infty} \alpha_{kn+n}(y)
		.
	\end{equation}

	По лемме~\ref{thm:distance_from_average} имеем
	\begin{multline}
		\label{eq:alpha_sigma_1_n_distance}
		|x_{kn+n}-y_{kn+n}|
		\leq
		\frac{n-1}{n}\max_{1\leq i<j \leq n}|x_{kn+i}-x_{kn+j}|
		\leq
		\\\leq
		\frac{n-1}{n} \max_{1\leq i \leq n} \alpha_{kn+i}(x)
		=
		\frac{n-1}{n}\alpha_{kn+i_k}(x)
		.
	\end{multline}

	Из того, что $y_j = \frac{1}{n}(x_{kn+1}+...+x_{kn+n})$,
	следует, что
	\begin{equation}
		\label{eq:alpha_sigma_1_n_alpha_x}
		\max_{kn+n \leq i \leq 2kn+2n } |x_{kn+n} - y_i|
		\leq
		\max_{kn+n \leq i \leq 2kn+2n } |x_{kn+n} - x_i|
		=
		\alpha_{kn+n}(x)
		.
	\end{equation}

	Оценим:
	\begin{multline}
		\alpha_{kn+n}(y)
		=
		\max_{kn+n \leq i \leq 2kn+2n } |y_{kn+n} - y_i|
		=
		\\=
		\max_{kn+n \leq i \leq 2kn+2n } |y_{kn+n} - x_{kn+n} + x_{kn+n} - y_i|
		\leq
		\\\leq
		|y_{kn+n} - x_{kn+n}| + \max_{kn+n \leq i \leq 2kn+2n } |x_{kn+n} - y_i|
		\mathop{\leq}^{\eqref{eq:alpha_sigma_1_n_distance}}
		\\\leq
		\frac{n-1}{n} \alpha_{kn+i_k}(x) + \max_{kn+n \leq i \leq 2kn+2n } |x_{kn+n} - y_i|
		\mathop{\leq}^{\eqref{eq:alpha_sigma_1_n_alpha_x}}
		\frac{n-1}{n} \alpha_{kn+i_k}(x)+\alpha_{kn+n}(x)
		.
	\end{multline}

	С учётом~\eqref{eq:alpha_sigma_1_n_subseq_limsup} имеем
	\begin{multline}
		\alpha(\sigma_{1/n}x)
		=
		\varlimsup_{k\to\infty} \alpha_{kn+n}(y)
		\leq
		\\\leq
		\varlimsup_{k\to\infty} \left( \frac{n-1}{n} \alpha_{kn+i_k}(x)+\alpha_{kn+n}(x) \right)
		=
		\left(2-\frac{1}{n}\right)\alpha(x)
		.
	\end{multline}
\end{proof}

Точность теоремы~\ref{thm:alpha_sigma_1_n} для $n=1$ очевидна.
Для $n=2$ её показывает
\begin{example}
	Положим для всех $p\in\N$:
	\begin{equation}
		x_k=\begin{cases}
			0, & k \leq 2^3, \\
			0, & k = 2^{3p}+1, \\
			1, & k = 2^{3p}+2, \\
			1, & 2^{3p}+3 \leq k \leq 2^{3p+1}+2, \\
			2, & 2^{3p+1}+3 \leq k \leq 2^{3p+1}+4, \\
			1, & 2^{3p+1}+5 \leq k \leq 2^{3(p+1)},
		\end{cases}
	\end{equation}
	тогда
	\begin{equation}
		(\sigma_{1/2}x)_k=\begin{cases}
			0, & k \leq 2^3, \\
			1/2, & k = 2^{3p}+1, \\
			1/2, & k = 2^{3p}+2, \\
			1, & 2^{3p}+3 \leq k \leq 2^{3p+1}+2, \\
			2, & 2^{3p+1}+3 \leq k \leq 2^{3p+1}+4, \\
			1, & 2^{3p+1}+5 \leq k \leq 2^{3(p+1)}.
		\end{cases}
	\end{equation}
	%\begin{table}
	%	\begin{tabular}{c||c|c|c|c|c|c|}
	%		\hline
	%		$k$     & $0..2^3$ & $2^{3p}+1$ & $2^{3p}+2$ & $2^{3p}+3 .. 2^{3p+1}+2$ & $2^{3p+1}+3 .. 2^{3p+1}+4 $ & $2^{3p+1}+5 .. 2^{3(p+1)}$ \\
	%	\end{tabular}
	%\end{table}
	Очевидно, что $\alpha(x)=1$, но $\alpha(\sigma_{1/2}x)=3/2$
	(достигается на $i=2^{3p}+2$, $j=2^{3p+1}+4$).
\end{example}

\begin{hypothesis}
	Оценка теоремы~\ref{thm:alpha_sigma_1_n} точна для любого $n\in\N$.
\end{hypothesis}


	\section{$\alpha$--функция и оператор Чезаро $C$}
	Ниже приводится расширенная версия материала, опубликованного в
\cite{our-vzms-2018}.

На пространстве ограниченных последовательностей $\ell_\infty$ определяется оператор Чезаро $C$
равенством
\begin{equation}
	(Cx)_n = {1}/{n} \cdot \sum_{k=1}^n x_k
	.
\end{equation}

Можно доказать, что верна
\begin{theorem}
	\label{thm:alpha_Cx_leq_alpha_x}
	%TODO: ссылка?
	$\alpha(Cx) \leqslant \alpha(x)$.
\end{theorem}
Выясняется, что эта оценка достаточно точна.

\subsection{Вспомогательная сумма специального вида}
\begin{lemma}
	Если $p\geq 2$, то
	\begin{equation}\label{summa_drobey}
		\sum_{i=0}^{p-1} \frac{i \cdot 2^i}{p} = \frac{2^p(p-2) + 2}{p}
	\end{equation}
\end{lemma}

% В Демидовиче этого не нашёл

\paragraph{Доказательство.}
Равенство \eqref{summa_drobey} равносильно равенству
\begin{equation}\label{summa_drobey_multiplied}
	\sum_{i=0}^{p-1} i \cdot 2^i = 2^p(p-2) + 2
	.
\end{equation}
Докажем это равенство методом математической индукции.

\paragraph{База индукции.}
Для $p=2$ имеем
\begin{equation}
	\sum_{i=0}^{2-1} i \cdot 2^i = 0 \cdot 2^0 + 1 \cdot 2^1 = 2
\end{equation}
и
\begin{equation}
	2^2(2-2) + 2 = 2
	.
\end{equation}
Видим, что для $p=2$ соотношение \eqref{summa_drobey_multiplied} выполняется.

\paragraph{Шаг индукции.}
Пусть соотношение \eqref{summa_drobey_multiplied} выполняется для $p=m$, $m\geq 2$, т.е.
\begin{equation}\label{summa_drobey_multiplied_m}
	\sum_{i=0}^{m-1} i \cdot 2^i = 2^m(m-2) + 2
	.
\end{equation}

Покажем, что тогда соотношение \eqref{summa_drobey_multiplied} выполняется и для $p=m+1$.
Действительно,
\begin{multline}
	\sum_{i=0}^{(m+1)-1} i \cdot 2^i
	=
	\sum_{i=0}^{m} i \cdot 2^i
	=
	\sum_{i=0}^{m - 1} i \cdot 2^i + m\cdot 2 ^m
	\mathop{=}^{\eqref{summa_drobey_multiplied_m}}
	2^m(m-2) + 2 + m\cdot 2 ^m
	=
	\\=
	m\cdot2^m-2\cdot2^m  + 2 + m\cdot 2 ^m
	=
	m\cdot2^{m+1}-2^{m+1}  + 2
	=
	\\=
	2^{m+1}(m-1)  + 2
	=
	2^{m+1}((m+1)-1)  + 2
	,
\end{multline}
т.е. соотношение \eqref{summa_drobey_multiplied} выполняется и для $p=m+1$,
что и требовалось доказать.



\subsection{Вспомогательный оператор $S$}
Пусть $y\in \ell_\infty$.
Определим оператор $S:\ell_\infty \to \ell_\infty$ следующим образом:
\begin{equation}\label{operator_S}
	(Sy)_k = y_{i+2}, \mbox{ где } 2^i < k \leq 2^i+1
\end{equation}
Этот оператор вводится исключительно для упрощения изложения конструкции.

\begin{example}
	$$
		S(\{1,2,3,4,5,6, ...\}) = \{1,2,3,3,4,4,4,4,5,5,5,5,5,5,5,5,6...\}
	$$
\end{example}

Теперь нам потребуются некоторые свойства оператора $S$.

\begin{lemma}
	\label{thm:alpha_S}
	\begin{equation}\label{alpha_S}
		\alpha(Sx) = \varlimsup_{k\to\infty} |x_{k+1} - x_{k}|
	\end{equation}
\end{lemma}

\paragraph{Доказательство.}

\begin{equation*}
	\alpha(Sx) =
	\varlimsup_{i\to\infty} \sup_{i < j \leq 2i} | (Sx)_i - (Sx)_j | = ...
\end{equation*}
Положим для каждого $i$ число $m_i$ так,
что $m_i = 2^{k_i}$, $i \leq m_i < 2i$
(очевидно, это всегда можно сделать).
\begin{equation*}
	... =
	\varlimsup_{i\to\infty} \max \left\{
		\max_{i   < j \leq m_i} | (Sx)_i - (Sx)_j |,
		\max_{m_i < j \leq 2i } | (Sx)_i - (Sx)_j |
	\right\} =
	...
\end{equation*}
Но при $2^{k_i - 1} < i < j \leq m_i = 2^{k_i}$
имеем $(Sx)_i = (Sx)_j$, и первый модуль обращается в нуль.

\begin{equation*}
	... =
	\varlimsup_{i\to\infty}
		\max_{m_i < j \leq 2i } | (Sx)_i - (Sx)_j |
	=
	...
\end{equation*}
Но при $2^{k_i - 1} < i \leq m_i = 2^{k_i} < j \leq 2^{k_i+1}$
имеем $(Sx)_i = x_{k_i+1}$, $(Sx)_j = x_{k_i+2}$, откуда
\begin{equation}\label{alpha_S_sosedi}
	... =
	\varlimsup_{k\to\infty}
		| x_{k+1} - x_k |
\end{equation}

Лемма доказана.

\begin{lemma}
	\begin{equation}\label{summa_S_less}
		\sum_{k=2}^{2^p} (Sy)_k =
		\sum_{i=0}^{p-1} 2^i y_{i+2}
	\end{equation}
\end{lemma}

\paragraph{Доказательство.}

\begin{equation*}
	\sum_{k=2}^{2^p} (Sy)_k =
	\sum_{i=0}^{p-1} \sum_{k=2^i+1}^{2^{i+1}} (Sy)_k =
	\sum_{i=0}^{p-1} \sum_{k=2^i+1}^{2^{i+1}} y_{i+2} =
	\sum_{i=0}^{p-1} 2^i y_{i+2}
\end{equation*}

Лемма доказана.


\begin{lemma}
	\begin{equation}\label{summa_S}
		\sum_{k=2^i+1}^{2^{i+j+1}} (Sx)_k =
		2^i\sum_{k=2}^{2^{j+1}} (ST^ix)_k
	\end{equation}

	Здесь и далее $(Tx)_n = x_{n+1}$.
\end{lemma}

\paragraph{Доказательство.}

\begin{multline*}
	\sum_{k=2^i+1}^{2^{i+j+1}} (Sx)_k =
	\sum_{m = i}^{i+j}\sum_{k=2^m+1}^{2^{m+1}} (Sx)_k =
	\sum_{m = i}^{i+j}2^m \cdot x_{m+2} =
	\\=
	2^i \cdot \sum_{n = 0}^{j}2^n \cdot x_{n+2+i} =
	2^i \cdot \sum_{n = 0}^{j}2^n (T^i x)_{n+2} =
	2^i \cdot \sum_{k=2}^{2^{j+1}} (ST^i x)_k
\end{multline*}

Лемма доказана.

\subsection{Вспомогательная функция $k_b$}

Введём функцию
\begin{equation}\label{def_k_b}
	k_b(x) = \frac{1}{2b}\left|
		\sum_{k=1}^{b}x_k - \sum_{k=b+1}^{2b}x_k
	\right|
\end{equation}

\begin{lemma}
	\begin{equation}\label{alpha_greater_k_b}
		\alpha (Cx) \geq \varlimsup_{i\to \infty} k_i(x)
	\end{equation}
\end{lemma}

\paragraph{Доказательство.}

\begin{multline*}
	\alpha (Cx) \mathop{=}\limits^{def}
	\varlimsup_{i\to \infty} \sup_{i<j\leq 2i} |(Cx)_i - (Cx)_j| \geq
	\varlimsup_{i\to \infty} |(Cx)_i - (Cx)_{2i}| =
	\\ =
	\varlimsup_{i\to \infty} \left|\frac{1}{i}\sum_{k=1}^i  - \frac{1}{2i}\sum_{k=1}^{2i} \right| =
	\varlimsup_{i\to \infty} \left|\frac{1}{i}\sum_{k=1}^i  - \frac{1}{2i}\sum_{k=1}^{i}- \frac{1}{2i}\sum_{k=i+1}^{2i}\right| =
	\\=
	\varlimsup_{i\to \infty} \left|\frac{1}{2i}\sum_{k=1}^i - \frac{1}{2i}\sum_{k=i+1}^{2i}\right| =
	\varlimsup_{i\to \infty} k_i(x)
\end{multline*}

\paragraph{Примечание.}
Введение функции $k_b(x)$ позволит нам в дальнейшем перейти от работы с оператором Чезаро
к несложным преобразованиям сумм.



\subsection{Основные построения}

Построим вектор $y\in \ell_\infty$ следующим образом:

\begin{equation}\label{y_construction}
	y = \left\{
		0, 0, \frac{1}{p}, \frac{2}{p}, \frac{3}{p},
		...,
		\frac{p-1}{p}, 1, \frac{p-1}{p},
		...,
		\frac{1}{p},
		~~
		\underbrace{
		\phantom{\frac{1}{1}\!\!\!}
			0, 0, 0, ..., 0,
		}_\text{$\phantom{\frac{1}{1}\!\!\!}$\!\!\!$3p+1$ раз}
		~~
		\frac{1}{p}, ...
	\right\}
\end{equation}
так, что
\begin{equation}\label{T_y}
	T^{5p}y = y
\end{equation}
%(0 повторяется $3p+1$ раз или около того, надо будет ещё очень аккуратно пересчитать).


Положим $x = Sy$.
Тогда с учётом (\ref{alpha_S})
\begin{equation}\label{alpha_x}
	\alpha (x) = \alpha (Sy) = \frac{1}{p}
\end{equation}


Оценим $\alpha(Cx)$:

\begin{multline*}
	\alpha (Cx) \mathop{\geq}^{(\ref{alpha_greater_k_b})}
	\varlimsup_{b\to \infty} k_b(x) =
	\varlimsup_{b\to \infty}\frac{1}{2b}\left|
		\sum_{k=1}^{b}x_k - \sum_{k=b+1}^{2b}x_k
	\right| \geq
	\\ \geq
	\varlimsup_{
		i\to \infty,~
		b=2^i~
	}\frac{1}{2^{i+1}}\left|
		\sum_{k=1}^{2^i}(Sy)_k - \sum_{k=2^i+1}^{2^{i+1}}(Sy)_k
	\right| =
	\\=
	\varlimsup_{i\to \infty}\frac{1}{2^{i+1}}\left|
		\sum_{k=1}^{2^i}(Sy)_k - 2^i y_{i+2}
	\right| =
	\varlimsup_{i\to \infty}\left|
		\frac{1}{2^{i+1}}\sum_{k=1}^{2^i}(Sy)_k - \frac{y_{i+2}}{2}
	\right| \geq
\end{multline*}
\begin{multline*}
	\\ \geq
	\varlimsup_{
		m\to \infty,~
		i=5pm+p~
	}\left|
		\frac{1}{2^{5pm+p+1}}\sum_{k=1}^{2^{5pm+p}}(Sy)_k - \frac{y_{5pm+p+2}}{2}
	\right| =
	\\=
	\varlimsup_{m\to \infty}\left|
		\frac{1}{2^{5pm+p+1}}\sum_{k=1}^{2^{5pm+p}}(Sy)_k - \frac{1}{2}
	\right| =
	\\=
	\varlimsup_{m\to \infty}\left|
		\frac{1}{2^{5pm+p+1}}\sum_{k=1}^{2^{5pm}}(Sy)_k
		+
		\frac{1}{2^{5pm+p+1}}\sum_{k=2^{5pm}+1}^{2^{5pm+p}}(Sy)_k
		- \frac{1}{2}
	\right|
	\mathop{=}^{(\ref{summa_S})}
	\\=
	\varlimsup_{m\to \infty}\left|
		\frac{1}{2^{5pm+p+1}}\sum_{k=1}^{2^{5pm}}(Sy)_k
		+
		\frac{1}{2^{5pm+p+1}} \cdot 2^{5pm} \cdot \sum_{k=2}^{2^p}(ST^{5pm}y)_k
		- \frac{1}{2}
	\right|
	\mathop{=}^{(\ref{T_y})}
	\\=
	\varlimsup_{m\to \infty}\left|
		\frac{1}{2^{5pm+p+1}}\sum_{k=1}^{2^{5pm}}(Sy)_k
		+
		\frac{1}{2^{5pm+p+1}} \cdot 2^{5pm} \cdot \sum_{k=2}^{2^p}(Sy)_k
		- \frac{1}{2}
	\right| =
	\\=
	\varlimsup_{m\to \infty}\left|
		\frac{1}{2^{5pm+p+1}}\sum_{k=1}^{2^{5pm}}(Sy)_k
		+
		\frac{1}{2^{p+1}} \sum_{k=2}^{2^p}(Sy)_k
		- \frac{1}{2}
	\right|
	\mathop{=}^{(\ref{summa_S_less})}
	\\=
	\varlimsup_{m\to \infty}\left|
		\frac{1}{2^{5pm+p+1}}\sum_{k=1}^{2^{5pm}}(Sy)_k
		+
		\frac{1}{2^{p+1}} \sum_{i=0}^{p-1}2^i y_{i+2}
		- \frac{1}{2}
	\right| =
	\\=
	\varlimsup_{m\to \infty}\left|
		\frac{1}{2^{5pm+p+1}}\sum_{k=1}^{2^{5pm}}(Sy)_k
		+
		\frac{1}{2^{p+1}} \sum_{i=0}^{p-1}2^i \cdot \frac{i}{p}
		- \frac{1}{2}
	\right|
	\mathop{=}^{(\ref{summa_drobey})}
	\\=
	\varlimsup_{m\to \infty}\left|
		\frac{1}{2^{5pm+p+1}}\sum_{k=1}^{2^{5pm}}(Sy)_k
		+
		\frac{1}{2^{p+1}} \cdot \frac{2^p(p-2)+2}{p}
		- \frac{1}{2}
	\right| =
	\\=
	\varlimsup_{m\to \infty}\left|
		\frac{1}{2^{5pm+p+1}}\sum_{k=1}^{2^{5pm}}(Sy)_k
		+
		\frac{1}{2} \cdot \frac{p-2}{p} + \frac{1}{p 2^p}
		- \frac{1}{2}
	\right| =
	\\=
	\varlimsup_{m\to \infty}\left|
		\frac{1}{2^{5pm+p+1}}\sum_{k=1}^{2^{5pm}}(Sy)_k
		-
		\frac{1}{p} + \frac{1}{p 2^p}
	\right| =
	\\=
	\varlimsup_{m\to \infty}\left|
		\frac{1}{2^{5pm+p+1}}\sum_{k=1}^{2^{5pm-2p}}(Sy)_k
		+
		\frac{1}{2^{5pm+p+1}}\sum_{k=2^{5pm-2p}+1}^{2^{5pm}}(Sy)_k
		-\frac{1}{p} + \frac{1}{p 2^p}
	\right| =
	\\\mbox{но во второй сумме все $(Sy)_k$ --- нули по построению}
\end{multline*}
\begin{multline*}
	\\=
	\varlimsup_{m\to \infty}\left|
		\frac{1}{2^{5pm+p+1}}\sum_{k=1}^{2^{5pm-2p}}(Sy)_k
		-\frac{1}{p} + \frac{1}{p 2^p}
	\right| = h
\end{multline*}

Но $0 \leq (Sy)_k \leq 1$,
значит,
$$
	\frac{1}{2^{5pm+p+1}}\sum_{k=1}^{2^{5pm-2p}}(Sy)_k
	\leq
	\frac{1}{2^{5pm+p+1}} \cdot 2^{5pm-2p}
	=
	\frac{1}{2^{3p+1}}
$$
Модуль раскрываем со знаком ``-''

\begin{multline*}
	h=
	\varlimsup_{m\to \infty} \left(
		\frac{1}{p} (1-2^{-p})
		- \frac{1}{2^{3p+1}}
	\right) =
	\frac{1}{p} (1-2^{-p})
	- \frac{1}{2^{3p+1}}
	= \\ =
	\frac{1}{p} (1-2^{-p})
	- \frac{1}{2^{2p+1}} \cdot 2^{-p}
	>
	\frac{1}{p} (1-2^{-p})
	- \frac{1}{p} \cdot 2^{-p}
	=
	\frac{1}{p} (1-2^{-p+1})
\end{multline*}


Таким образом,
$$
	\frac{\alpha(Cx)}{\alpha(x)} \geq
	\frac{	\frac{1}{p} (1-2^{-p+1}) }{\frac{1}{p}} =
	1-2^{-p+1}
$$

Рассматривая $x$ как функцию от $p$, имеем:
$$
	\sup_{p\in\mathbb{N}} \frac{\alpha(Cx(p))}{\alpha(x(p))} \geq
	\sup_{p\in\mathbb{N}} (1-2^{-p+1}) =
	1
	.
$$
Таким образом, может быть сформулирована следующая

\begin{theorem}
	\label{thm:alpha_Cx_no_gamma}
	\begin{equation}
		\sup_{x\in\ell_\infty, \alpha(x)\neq 0} \frac{\alpha(Cx)}{\alpha(x)}=1
		.
	\end{equation}
\end{theorem}

\subsection{Некоторые гипотезы}

\begin{hypothesis}
	Пусть $x\in\ell_\infty$ и $0 \leq x_k \leq 1$.
	Тогда для любого $n\in\mathbb{N}$
	\begin{equation}
		\alpha(C^n x) \leq \frac{1}{2^n}
		.
	\end{equation}
\end{hypothesis}

\begin{hypothesis}
	Для любых $x\in\ell_\infty$ и $n\in\mathbb{N}$
	\begin{equation}
		\alpha(C^n x) - \alpha(C^{n+1} x) \leq \alpha(C^{n-1} x) - \alpha(C^{n} x)
		.
	\end{equation}
\end{hypothesis}

\begin{hypothesis}
	Для любого $0<a<1$ существует такой $x\in\ell_\infty$, что
	\begin{equation}
		\lim_{m\to\infty} \frac{\alpha(C^m x)}{a^m} = \infty
		.
	\end{equation}
\end{hypothesis}


	\section{$\alpha$--функция и оператор суперпозиции}
	Введём на пространстве $\ell_\infty$ фактор-норму по $c_0$:

TODO: надо ссылку на классиков?

\begin{equation}
	\|x\|_* = \limsup_{k\to\infty} |x_k|
	.
\end{equation}

\begin{theorem}
	\label{thm:alpha_xy}
	Пусть $(x\cdot y)_k = x_k\cdot y_k$.
	Тогда
	$\alpha(x\cdot y)\leq \alpha(x)\cdot \|y\|_* + \alpha(y)\cdot \|x\|_*$.
\end{theorem}

\begin{proof}
	\begin{multline}
		\alpha(x\cdot y)
		=
		\limsup_{i\to\infty} \max_{i\leq j \leq 2i} |x_i y_i - x_j y_j|
		=
		\limsup_{i\to\infty} \max_{i\leq j \leq 2i} |x_i y_i - x_j y_i + x_j y_i - x_j y_j|
		\leq
		\\ \leq
		\limsup_{i\to\infty} \max_{i\leq j \leq 2i} \left(|x_i y_i - x_j y_i| + |x_j y_i - x_j y_j| \right)
		=
		\\=
		\limsup_{i\to\infty} \max_{i\leq j \leq 2i} \left(|y_i|\cdot|x_i - x_j| + |x_j|\cdot|y_i - y_j| \right)
		\leq
		\\ \leq
		\limsup_{i\to\infty} \max_{i\leq j \leq 2i} |y_i|\cdot|x_i - x_j| + \limsup_{i\to\infty} \max_{i\leq j \leq 2i}|x_j|\cdot|y_i - y_j|
		=
		\\ =
		\limsup_{i\to\infty} |y_i| \max_{i\leq j \leq 2i} \cdot|x_i - x_j| + \limsup_{i\to\infty} \max_{i\leq j \leq 2i}|x_j|\cdot|y_i - y_j|
		\leq
		\\ \leq
		\limsup_{i\to\infty} |y_i| \cdot \limsup_{i\to\infty} \max_{i\leq j \leq 2i} \cdot|x_i - x_j| + \limsup_{i\to\infty} \max_{i\leq j \leq 2i}|x_j|\cdot|y_i - y_j|
		=
		\\ =
		\|y\|_* \cdot \limsup_{i\to\infty} \max_{i\leq j \leq 2i} \cdot|x_i - x_j| + \limsup_{i\to\infty} \max_{i\leq j \leq 2i}|x_j|\cdot|y_i - y_j|
		=
		\\ =
		\|y\|_* \cdot \alpha(x) + \limsup_{i\to\infty} \max_{i\leq j \leq 2i}|x_j|\cdot|y_i - y_j|
		\leq
		\\ \leq
		\|y\|_* \cdot \alpha(x) + \limsup_{i\to\infty} \max_{i\leq j \leq 2i}|x_j|\cdot\limsup_{i\to\infty} \max_{i\leq j \leq 2i}|y_i - y_j|
		=
		\\ =
		\|y\|_* \cdot \alpha(x) + \limsup_{i\to\infty} \max_{i\leq j \leq 2i}|x_j|\cdot \alpha(y)
		\leq
		\\ \leq
		\|y\|_* \cdot \alpha(x) + \limsup_{j\to\infty} |x_j|\cdot \alpha(y)
		=
		%\\ =
		\|y\|_* \cdot \alpha(x) + \|x\|_* \cdot \alpha(y)
		.
	\end{multline}
\end{proof}

\begin{example}
	Покажем, что нижнюю оценку на $\alpha(x\cdot y)$ дать нельзя.
	В самом деле,
	пусть $x_k = (-1)^k$.
	Тогда $\alpha(x) = 2$, $\|x\|_* = 1$,
	но $\alpha(x\cdot x) = 0$.
\end{example}

\begin{example}
	\label{ex:alpha-c_not_ideal}
	Покажем, что оценка теоремы~\ref{thm:alpha_xy} точна в том смысле,
	что равенство достижимо.
	В самом деле,
	пусть $x_k = (-1)^k$, $y_k = 1$.
	Тогда $\alpha(x) = 2$, $\alpha(y) = 0$,
	но $\alpha(x\cdot y) = 2$.
\end{example}

\begin{example}
	Пусть
	\begin{equation}\label{y_construction}
		y = \left(
			0, \frac{1}{p}, \frac{2}{p}, \frac{3}{p},
			...,
			\frac{p-1}{p}, 1, \frac{p-1}{p},
			...,
			\frac{1}{p},
			0,
			\frac{1}{p}, \frac{2}{p}, ...
		\right)
	\end{equation}
	Тогда $\alpha(Sy) = \frac{1}{p}$, $\|Sy\|_* = 1$,
	\begin{equation}
		\alpha((Sy)\cdot(Sy)) =
		\left| \frac{(p-1)^2}{p^2} - 1 \right| =
		\frac{2}{p}-\frac{1}{p^2}
		.
	\end{equation}
\end{example}

\begin{example}
	Для того же $y$ и любого $k>0$ имеем
	$\alpha(kSy) = \frac{k}{p}$, $\|kSy\|_* = k$,
	\begin{equation}
		\alpha((kSy)\cdot(kSy)) =
		\left| \frac{k^2(p-1)^2}{p^2} - k^2 \right| =
		\frac{2k^2}{p}-\frac{k^2}{p^2}
		.
	\end{equation}
	Оценка теоремы~\ref{thm:alpha_xy} даёт
	\begin{equation}
		\alpha((kSy)\cdot(kSy)) \leq
		\frac{2k^2}{p}
		.
	\end{equation}
\end{example}


% Увы, неверна - контрпример выше, k=2
%\begin{hypothesis}
%	$\alpha(x\cdot y)\leq \alpha(x)\cdot \|y\|_* + \alpha(y)\cdot \|x\|_* - \alpha(x)\alpha(y) \cdot \|y\|_* \cdot \|x\|_*$.
%\end{hypothesis}
% Убрать фактор-нормы тоже нельзя:

\begin{example}
	Пусть $x_k = (-1)^k$.
	Тогда $\alpha(x) = \alpha(\sigma_2 x) = 2$, $\|x\|_* = \|\sigma_2 x\|_* = 1$,
	но $\alpha(x\cdot \sigma_2 x) = 2 < 4$.
\end{example}

Исходя из построенных примеров может быть выдвинута
\begin{hypothesis}
	Если $\alpha(x\cdot y)= \alpha(x)\cdot \|y\|_* + \alpha(y)\cdot \|x\|_*$,
	то $\alpha(x) = 0$ или $\alpha(y) = 0$.
\end{hypothesis}


	\section{Функционалы $\alpha^*$ и $\alpha_*$}
	Как мы выяснили, $\alpha$--функция не инвариантна относительно сдвига.
Чтобы избавиться от этого недостатка, введём функционалы
\begin{equation}
	\alpha^* = \lim_{n\to\infty} U^n x
\end{equation}
и
\begin{equation}
	\alpha_* = \lim_{n\to\infty} T^n x
\end{equation}
(оба пределы существуют как пределы монотонных ограниченных последовательностей).
Тогда верны следующие соотношения.
\begin{lemma}
	\begin{equation}
		\alpha^*(x) = \alpha^*(\sigma_n x) = \alpha^*(Tx) = \alpha^*(Ux)
		,
		%TODO: \sigma_{1/n}
	\end{equation}
	\begin{equation}
		\alpha_*(x) = \alpha_*(\sigma_n x) = \alpha_*(Tx) = \alpha_*(Ux)
		.
		%TODO: \sigma_{1/n}
	\end{equation}
\end{lemma}

\begin{lemma}
	\begin{equation}
		\frac{1}{2} \alpha(x) \leq \alpha_*(x) \leq \alpha(x) \leq \alpha^*(x) \leq 2 \alpha(x)
		.
	\end{equation}
\end{lemma}

Кроме того, для учёта обоих функционалов и сохранения однородности предлагается ввести функционал
\begin{equation}
	\alpha_*^*(x) = \sqrt{\alpha_*(x)\alpha^*(x)}
	.
\end{equation}

\begin{hypothesis}
	Функционал $\alpha_*^*(x)$ удовлетворяет неравенству треугольника.
\end{hypothesis}

\begin{hypothesis}
	Аналог теоремы~\ref{thm:alpha_Cx_no_gamma} верен для функционалов
	$\alpha^*(x)$, $\alpha_*(x)$ и $\alpha_*^*(x)$
\end{hypothesis}


	\section{Пространство $\{x: \alpha(x) = 0\}$}
	\label{sec:space_A0}

Из свойства~\ref{thm:alpha_x_triangle_ineq} и однородности $\alpha$--функции немедленно вытекает
\begin{theorem}
	\label{thm:A0_is_space}
	Множество $A_0 = \{x: \alpha(x) = 0\}$
	является пространством.
\end{theorem}
Изучим свойства этого пространства.
\begin{property}
	Пространство $A_0$ замкнуто.
\end{property}
\begin{proof}
	Прообраз замкнутого множества $\{0\}$
	при непрерывном отображении $\alpha : \ell_\infty \to \R$
	замкнут.
\end{proof}
\begin{theorem}
	Включение $c \subset A_0$ собственное.
\end{theorem}
\begin{proof}
	Рассмотрим
	\begin{equation}
		x=\left(
			0,1,
			0,\frac{1}{2},1,\frac{1}{2},
			0,\frac{1}{3},\frac{2}{3},1,\frac{2}{3},\frac{1}{3},
			0,
			...
		\right)
		.
	\end{equation}
	Тогда по лемме~\ref{thm:alpha_S} (при определённом тем же образом операторе $S$) имеем
	\begin{equation}
		\alpha(Sx) = \varlimsup_{k\to\infty} |x_{k+1} - x_{k}| = 0
		,
	\end{equation}
	однако, очевидно, $Sx\notin c$.
\end{proof}
Очевидно следующее
\begin{property}
	Если периодическая последовательность принадлежит $A_0$,
	то эта последовательность~--- константа.
\end{property}

Из результатов предыдущих параграфов вытекает
\begin{theorem}
	Пространство $A_0$ замкнуто относительно операторов $T$, $U$, $C$, $\sigma_n$, $\sigma_{1/n}$, $S$.
\end{theorem}

Из теоремы~\ref{thm:alpha_xy} следует
\begin{theorem}
	Пространство $A_0$ замкнуто относительно умножения,
	т.е. если $x,y\in A_0$, то $x\cdot y \in A_0$.
\end{theorem}

Пример~\ref{ex:alpha-c_not_ideal} показывает, что $A_0$
не является идеалом по умножению.

Из теоремы~\ref{thm:rho_x_c_leq_alpha_t_s_x_united} следует
\begin{theorem}
	$ac \cap A_0 = c$.
\end{theorem}

Некоторый интерес представляет следующая теорема,
основанная на идеях~\cite{usachev2009_phd_vsu}.

\begin{theorem}
	Пространство $A_0$ несепарабельно.
\end{theorem}

\begin{proof}
	Напомним, что через $\Omega = \{0;1\}^\N$
	обозначается множество всех последовательностей, состоящих из нулей и единиц.
	Для каждого $\omega\in\Omega$ положим
	\begin{multline}
		\label{eq:x_omega_alpha_c}
		x(\omega)=\left(
			0, 1\omega_1,
			0, \frac{1}{2}\omega_2, 1\omega_2, \frac{1}{2}\omega_2,
			0, \frac{1}{3}\omega_3, \frac{2}{3}\omega_3, 1\omega_3, \frac{2}{3}\omega_3, \frac{1}{3}\omega_3,
			0, ...,
		\right. \\ \left.
			0, \frac{1}{p}\omega_p, \frac{2}{p}\omega_p, ..., \frac{p-1}{p}\omega_p, 1\omega_p,
				\frac{p-1}{p}\omega_p, ..., \frac{2}{p}\omega_p, \frac{1}{p}\omega_p,
			0, \frac{1}{p+1}\omega_{p+1}, ...
		\right).
	\end{multline}
	Тогда по лемме~\ref{thm:alpha_S} (при определённом тем же образом операторе $S$) имеем
	$\alpha(Sx(\omega)) = 0$.
	Заметим, что при $\omega,\omega^* \in \Omega$ и $\omega\neq\omega^*$ выполнено
	$\|Sx(\omega)-Sx(\omega^*)\|=1$ и $|Sx(\Omega)|=\mathfrak{c}$.
	Следовательно, $A_0$ несепарабельно.
\end{proof}

%\begin{theorem}
%	$|A_0| = |\ell_\infty|$.
%\end{theorem}

%\begin{proof}
%	Достаточно заметить, что
%\end{proof}


	\section{О недополняемости некоторых вложений}
	\label{sec:noncomplementarity}

В работе~\cite{phillips1940linear} Филлипс доказал весьма неожиданный (для своего времени) результат:
простанство $c_0$ недополняемо в $\ell_\infty$.
Говоря формально, справедлива следующая


\begin{theorem}[Филлипса]
	\label{thm:phillips}
	Не существует непрерывного линейного оператора $P: \ell_\infty \to c_0$ такого, что для любого
	$x \in c_0$ выполнено равенство $Px =x$.
\end{theorem}
Подобные операторы называются \emph{проекторами}.

Теорема~\ref{thm:phillips} была первым примером недополняемого вложения пространств.
Позже были найдены и другие примеры;
%TODO: краткий обзор - передрать из Conway 2nd ed., p. 94
отсылаем читателя, например, к~\cite{lindenstrauss1979classical}.

Е.А. Алехно привёл~\cite[Theorem 8]{alekhno2006propertiesII} элегантное доказательство
того, что $ac_0$ недополняемо в $\ell_\infty$.
Это доказательство основано на изначальном доказательстве Филлипса теоремы~\ref{thm:phillips}
и использует некоторые леммы из~\cite{phillips1940linear}.

Вложение $c_0 \subset ac_0$ также недополняемо.
Непосредственное явное упоминание этого факта найти не удалось, однако он достаточно легко следует из~\cite[теорема 4]{ASSU2},
доказательство которой опирается на идеи~\cite{whitley1968projecting}~и~\cite[Theorem 6.9]{Carothers}.
Приведём эту теорему (в несколько ослабленной форме, для которой достаточно уже введённой терминологии)
и соотвествующее следствие, предварив одной вспомогательной леммой, доказательство коей является столь же классическим, сколь и кратким,
и приводится здесь исключительно ради полноты изложения.

\begin{definition}
	Семейство множеств $\{A_\lambda\}_{\lambda \in \Lambda}$ называется почти дизъюнктным,
	если для $\lambda, \mu \in \Lambda$, $\lambda \ne \mu$,
	пересечение $A_\lambda \cap A_\mu$ конечно.
\end{definition}

\begin{lemma}
	\label{lem:uncountable_subsets_of_N_with_finite_intersections}
	Существует почти дизъюнктное несчётное семейство подмножеств $\{S_i\}_{i\in I}$, $S_i \subset \mathbb{N}$.
\end{lemma}

\begin{proof}
	Рассмотрим биекцию $\N \leftrightarrow \Q$.
	Пусть $I = \R$.
	Для каждого $i\in I$ положим $S_i = \{q_n\}$,
	где $\{q_n\} \subset \mathbb{Q}$ "--- некоторая последовательность рациональных чисел,
	сходящаяся к $i$.
	%(Proof: switch from $\mathbb{N}$ to $\mathbb{Q}$, pick convergent sequences to irrationals.
	%More details in [this question](https://math.stackexchange.com/q/162387).)
\end{proof}

\begin{theorem}
	\label{thm:Alekhno_noncomplementarity_general}
	Пусть $X$ и $Y$ "--- такие линейные подпространства в $\ell_\infty$,
	что $c_{00} \subseteq Y \subsetneq X \subseteq \ell_\infty$.
	Пусть существует такое несчётное почти дизъюнктное семейство множеств натуральных чисел $\{A_\lambda\}_{\lambda \in \Lambda}$,
	$A_\lambda \subseteq \N$, что для любого $\lambda \in \Lambda$ имеет место включение $\chi_{A\lambda} \in X \setminus Y$.
	Тогда вложение $Y \subset X$ недополняемо.
\end{theorem}

\begin{corollary}
	Вложение $c_0\subsetneq ac_0$ недополняемо.
	Вложение $c_0\subsetneq \Iac$ недополняемо.
\end{corollary}

\begin{proof}
	Возьмём почти дизъюнктное семейство $\{S_i\}_{i\in I}$, $S_i \subset \mathbb{N}$
	из леммы~\ref{lem:uncountable_subsets_of_N_with_finite_intersections}
	и построим новое семейство $\{A_i\}_{i\in I}$, $A_i \subset \mathbb{N}$
	по правилу
	\begin{equation}
		\label{eq:c0_noncomplemented_in_ac0_set_family}
		A_i = \{ 2^n : n\in S_i \}
		.
	\end{equation}
	Легко видеть, что семейство $\{A_i\}_{i\in I}$ снова является почти дизъюнктным.
	Более того, $\chi_{A_i} \in ac_0 \setminus c_0$ и $\chi_{A_i} \in \Iac \setminus c_0$ для любого $i\in I$.

	Таким образом, выполнены все условия теоремы~\ref{thm:Alekhno_noncomplementarity_general}
	для цепочки вложений $c_{00} \subseteq c_0 \subsetneq ac_0 \subseteq \ell_\infty$ и
	для цепочки вложений $c_{00} \subseteq c_0 \subsetneq \Iac \subseteq \ell_\infty$.
\end{proof}

%TODO: а что там с \Iac \subset ac_0 ?

Итак, все три вложения в цепочке
\begin{equation}
	c_0 \subset \ell_\infty,
	\quad
	ac_0 \subset \ell_\infty,
	\quad\mbox{и}\quad
	c_0 \subset ac_0,
\end{equation}
недополняемы.

Перейдём теперь к изучению пространства $A_0$.

\begin{lemma}
	Вложение $A_0 \subset \ell_\infty$ недополняемо.
\end{lemma}

\begin{proof}
	В теореме~\ref{thm:Alekhno_noncomplementarity_general}
	положим $Y=A_0$, $X=\ell_\infty$.
	Тогда для сесмества множеств~\eqref{eq:c0_noncomplemented_in_ac0_set_family}
	выполнены все условия теоремы~\ref{thm:Alekhno_noncomplementarity_general}.
\end{proof}

Однако для вложения $c_0 \subset A_0$ применить теорему~\ref{thm:Alekhno_noncomplementarity_general}
непосредственно не удастся.
Осноная проблема заключается в том, что для любого бесконечного множества $F \subset \N$ такого, что
дополнение $\N \setminus F$ также бесконечно, последовательность $\chi_F \notin A_0$,
поскольку $\alpha(F) = 1$.

Поэтому мы проведём доказательство недополняемости полностью,
во многом опираясь на идеи~\cite{whitley1968projecting} и дискуссию~\cite{mathSE_Phillips}.
Для этого нам потребуется напомнить некоторые вспомогательные конструкции,
с которыми читатель уже встречался в этой главе.


Определим линейный оператор  $F:\ell_\infty \to \ell_\infty$ соотношением
\begin{equation}
	\label{operator_F}
	(Fy)_k = y_{i+2}, \mbox{ for } 2^i < k \leq 2^{i+1}
	,
\end{equation}
т.е.
\begin{equation}
	F(\{x_1,x_2,x_3,x_4,x_5,x_6, ...\}) = \{x_1,x_2,\,x_3,x_3,\,x_4,x_4,x_4,x_4,\,x_5,x_5,x_5,x_5,x_5,x_5,x_5,x_5,\,x_6...\}
\end{equation}

%TODO:ссылки?..
Напомним, что выполнено равенство
\begin{equation}
	\label{eq:alpha_F}
	\alpha(Fx) = \varlimsup_{k\to\infty} |x_{k+1} - x_{k}|
	.
\end{equation}


Определим линейный оператор $M:\ell_\infty \to \ell_\infty$ следующим образом:
\begin{multline*}
	M(\omega_1,\omega_2,...)=\left(
		0, 1\omega_1,
		0, \frac{1}{2}\omega_2, 1\omega_2, \frac{1}{2}\omega_2,
		0, \frac{1}{3}\omega_3, \frac{2}{3}\omega_3, 1\omega_3, \frac{2}{3}\omega_3, \frac{1}{3}\omega_3,
		0, ...,
	\right. \\ \left.
		0, \frac{1}{p}\omega_p, \frac{2}{p}\omega_p, ..., \frac{p-1}{p}\omega_p, 1\omega_p,
			\frac{p-1}{p}\omega_p, ..., \frac{2}{p}\omega_p, \frac{1}{p}\omega_p,
		0, \frac{1}{p+1}\omega_{p+1}, ...
	\right)
	.
\end{multline*}
Заметим, что в силу~\eqref{eq:alpha_F} мы имеем $FM: \ell_\infty \to A_0$.


\begin{lemma}
	\label{lem:c_0_not_complemented_in_A_0}
	Пусть линейный оператор  $Q: A_0 \to A_0$ таков, что $c_0\subseteq \ker Q$.
	Тогда существует счётное подмножество $S \subset \N$ такое, что
	\begin{equation}
		\forall(x \in A_0 : \supp x \subset S)[Qx = 0]
	\end{equation}
	и $x\in A_0\setminus c_0$ такой, что $\supp x \subseteq S$.
\end{lemma}

\begin{proof}
	Пусть $\{U_i\}_{i \in I}$ есть семейство подмножеств $\N$,
	удовлетворяющее условиям леммы~\ref{lem:uncountable_subsets_of_N_with_finite_intersections}.
	Пусть $\{S_i\}_{i \in I}$ есть семейство подмножеств $\N$
	определённое соотношением $S_i = \supp FM\chi_{U_i}$.
	Очевидно, что семество множеств $\{S_i\}_{i \in I}$ также
	удовлетворяет условиям леммы~\ref{lem:uncountable_subsets_of_N_with_finite_intersections}.
	%satisfies the conditions of Lemma~\ref{lem:uncountable_subsets_of_N_with_finite_intersections}.
	Более того, для любого $i\in I$ мы имеем $x = FM\chi_{U_i} \in A_0\setminus c_0$.

	Предположим противное:
	\begin{equation}
		\forall(\mbox{infinite }S\subset\N)\exists(x \in A_0 : \supp x \subset S)[Qx \neq 0]
		.
	\end{equation}
	В частности,
	%In particular,
	\begin{equation}
		\forall(i\in I)\exists(x_i \in A_0 : \supp x_i \subset S_i)[Q(x_i) \neq 0]
		.
	\end{equation}

	Заметим, что $x_i \notin c_0$, поскольку $c_0\subseteq \ker Q$.
	Не теряя общности, будем полагать $\|x_i\|=1$ для всех $i \in I$.
	%Without loss of generality we can assume that $\|x_i\|=1$ for all $i \in I$.

	Положим $I_n = \{i \in I\,:\,(Qx_i)_n \neq 0\}$,
	%Consider $I_n = \{i \in I\,:\,(Qx_i)_n \neq 0\}$,
	тогда $I = \bigcup\limits_{n\in\N} I_n$.
	%then $I = \bigcup\limits_{n\in\N} I_n$.
	Таким образом, найдётся $n$ такой, что $I_n$ также несчётно
	%Thus, we can find $n$ such that $I_n$ is also uncountable
	(иначе $I$ было бы счётным как объединение счётного количества счётных множеств,
	%(otherwise $I$ would be countable as a countable union of countable sets,
	что противоречит условиям леммы~\ref{lem:uncountable_subsets_of_N_with_finite_intersections}).
	%which contradicts to conditions of Lemma~\ref{lem:uncountable_subsets_of_N_with_finite_intersections}).

	Рассмотрим теперь $I_{n,k} = \{i \in I_n\,:\,|(Qx_i)_n| \geq 1/k\}$,
	%Сonsider now $I_{n,k} = \{i \in I_n\,:\,|(Qx_i)_n| \geq 1/k\}$,
	тогда $I_n = \bigcup\limits_{k\in\N} I_{n,k}$.
	%then $I_n = \bigcup\limits_{k\in\N} I_{n,k}$.
	Аналогичные рассуждения показывают, что для некоторого $k$ множество $I_{n,k}$ несчётно.
	%Applying the same argument as above, one can easily see that the set $I_{n,k}$ is uncountable for some $k$.
	Зафиксируем такое $I_{n,k}$ и в дальнейшем будем работать с ним.
	%Let us choose such $I_{n,k}$ and proceed with it.

	Итак, у нас есть несчётное множество $I_{n,k}$ и
	%So, we have an uncountable set $I_{n,k}$ and
	\begin{equation}
		\forall(i\in I_{n,k})\exists(x_i \in ac_0 : \supp x_i \subset S_i)\Bigl[\|x_i\|=1 \mbox{~~и~~} |(Qx_i)_n| \geq 1/k\Bigr]
		.
	\end{equation}

	Рассмотрим конечное множество $J \subset I_{n,k}$, $\#J>1$
	%Consider a finite set $J \subset I_{n,k}$ with $\#J>1$
	(здесь $\#J$ означает мощность множества $J$).
	%(here $\#J$ stands for the cardinality of the set $J$).
	Возьмём
	\begin{equation}
		y = \sum_{j \in J} \operatorname{sign}{(Qx_j)_n} \cdot x_j
		.
	\end{equation}
	Поскольку пересечение $S_i \cap S_j$ конечно для любых $i \neq j$ и
	$\supp x_j \subset S_j$,
	пересечение $\bigcap\limits_{j\in J} \supp x_j$ также конечно.
	Таким образом, $y = f + z$,
	причём $\supp f$ конечен и $\|z\| \leq 1$.

	С другой стороны,
	\begin{equation}
		\label{eq:non_complemented_sum_cardinality}
		(Qy)_n = \sum_{j \in J}
		(\operatorname{sign}(Qx_j)_n)
		\cdot (Qx_j)_n \geq \frac{\# J}{k}
		.
	\end{equation}
	Заметив, что $f\in c_0$, мы получаем $Qf = 0$, поскольку $c_0 \subseteq \ker Q$.
	Значит, $Qy = Q(f+z) = Qf + Qz = Qz$ и
	\begin{equation}
		\label{eq:norm_Q_estimate}
		\frac{\# J}{k} \leq (Qy)_n \leq \|Qy\| = \|Qz\| \leq \|Q\| \cdot \|z\| \leq \|Q\|
		.
	\end{equation}
	С  учётом~\eqref{eq:norm_Q_estimate} получаем $\# J \leq \|Q\| k$ для любого $J\subset I_{n,k}$.
	%Due to~\eqref{eq:norm_Q_estimate}, we obtain $\# J \leq \|Q\| k$ for every  $J\subset I_{n,k}$.
	Таким образом мы получаем противоречие с тем фактом, что $I_{n,k}$ несчётно.
	%This contradicts the fact that $I_{n,k}$ is uncountable,
	%and we are done.
\end{proof}

\begin{theorem}
	Пространство $c_0$ недополняемо в $A_0$.
\end{theorem}

\begin{proof}
	Предположим противное.
	Тогда существует непрерывный проектор $P: A_0 \to c_0$.
	Применим лемму~\ref{lem:c_0_not_complemented_in_A_0} к оператору $I-P$
	и найдём бесконечное подмножество $S\subset\N$ такое,
	что $\forall(x\in A_0 : \supp x \subset S)[(I-P)x = 0]$,
	и $x\in A_0 \setminus c_0$ такой, что $\supp x \subseteq S$.
	Но тогда  $Px = x\notin c_0$,
	что противоречит предположению, что $P$ есть проектор на $c_0$.
\end{proof}




\chapter{Последовательности, почти сходящиеся к нулю (пространство $ac_0$)}

	Почти сходимость является естественным обобщением понятия сходимости.
История исследования почти сходимости начинается с работы Г.Г. Лоренца~\cite{lorentz1948contribution}.
Напомним определение банахова предела.

\begin{definition}
	\label{def:Banach_limit}
	Линейный функционал $B\in \ell_\infty^*$ называется банаховым пределом,
	если
	\begin{enumerate}
		\item
			$B\geq0$, т.~е. $Bx \geq 0$ для $x \geq 0$,
		\item
			$B\one=1$, где $\one =(1,1,\ldots)$,
		\item
			$B(Tx)=B(x)$ для всех $x\in \ell_\infty$, где $T$~---
		оператор сдвига, т.~е. $T(x_1,x_2,\ldots)=(x_2,x_3,\ldots)$.
	\end{enumerate}
\end{definition}
Множество всех банаховых пределов обозначим через $\mathfrak{B}$.
Существование банаховых пределов было анонсировано С. Мазуром \cite{Mazur} и позднее доказано в книге С.~Банаха~\cite{banach2001theory_rus}.


Лоренц установил, что существуют такие последовательности $x\in\ell_\infty$,
что значение выражения $Bx$ не зависит от выбора $B\in\mathfrak{B}$.
Такие последовательности называются почти сходящимися (англ. \textit{almost convergent}).
Пишут: $x\in ac$.

Лоренц доказал следующий критерий почти сходимости,
который оказывается исключительно удобен при проверке последовательности на принадлежность пространству $ac$.

\begin{theorem}[Критерий Лоренца]
	Для заданных $t\in\R$ и $x\in\ell_\infty$ равенство $Bx=t$ выполнено для всех $B\in\mathfrak{B}$
	тогда и только тогда, когда
	\begin{equation}
		\label{eq:crit_Lorentz}
		\lim_{n\to\infty} \frac{1}{n} \sum_{k=m+1}^{m+n} x_k = t
	\end{equation}
	равномерно по $m\in\N$.
\end{theorem}

Если некоторый $x\in\ell_\infty$ удовлетворяет~\eqref{eq:crit_Lorentz},
то мы будем говорить, что $x$ почти сходится к $t$,
и писать: $x\in ac_t$.
Таким образом, очевидно, $ac = \bigcup\limits_{t\in\R} ac_t$.

Равномерный предел в критерии Лоренца можно заменить на двойной~\cite[Теорема 1]{zvol2022ac}:
\begin{theorem}
	Для заданного $t\in\R$ равенство $Bx=t$ выполнено для всех $B\in\B$
	тогда и только тогда, когда
	\begin{equation}
		\label{eq:crit_Lorentz}
		\lim_{n,m\to\infty} \frac{1}{n} \sum_{k=m+1}^{m+n} x_k = t
		.
	\end{equation}
\end{theorem}

(Заметим, что в общем случае равномерный предел и двойной предел "--- это разные объекты;
за подробными комментариями отсылаем к классическим трудам по математическому анализу,
например,~\cite[с. 154]{kudryavcev2004mathanalys}.)
В настоящей главе в целях удобства доказывается модифицированный критерий Лоренца "--- теорема~\ref{thm:Lorentz_mod}.

Приведём важнейшее следствие из критерия Лоренца, позволяющее в ряде случаев без особых усилий показывать почти сходимость последовательности, которое также содержится в~\cite{lorentz1948contribution}.

\begin{corollary}
	\label{thm:period_ac_avg}
	Всякая периодическая последовательность почти сходится к среднему по периоду.
	Иначе говоря, для любого $B\in\mathfrak B$
	\begin{equation}
		B(x_1,x_2, ..., x_n, \; x_1,x_2, ..., x_n, \; x_1,x_2, ..., x_n, \; x_1, ...) = \frac{x_1+x_2+...+x_n}{n}
		.
	\end{equation}
\end{corollary}

Сачестон~\cite{sucheston1967banach} установил, что
для любых $x\in \ell_\infty$ и $B\in\mathfrak{B}$
\begin{equation}\label{Sucheston}
	q(x) \leqslant Bx \leqslant p(x)
	,
\end{equation}
где
\begin{equation*}
	q(x) = \lim_{n\to\infty} \inf_{m\in\N}  \frac{1}{n} \sum_{k=m+1}^{m+n} x_k
	~~~~\mbox{и}~~~~
	p(x) = \lim_{n\to\infty} \sup_{m\in\N}  \frac{1}{n} \sum_{k=m+1}^{m+n} x_k
	.
\end{equation*}
называют нижним и верхним функционалом Сачестона соотвественно.
Заметим, что $p(x) = -q(-x)$.
Неравенства \eqref{Sucheston} точны:
для данного $x$ для любого $r\in[q(x); p(x)]$ найдётся банахов предел
$B\in\mathfrak{B}$ такой, что $Bx = r$.

Множество таких $x\in\ell_\infty$, что $p(x)=q(x)$, и
образует подпространство почти сходящихся последовательностей $ac$.
Таким образом, функционалы Сачестона являются удобным инструментом для доказательства того,
что некоторая последовательность $x$ не является почти сходящейся:
для этого достаточно показать, что $p(x)\ne q(x)$.

В~\cite[теорема 5]{Jerison} показано, что нижний и верхний функционалы Сачестона могут быть переписаны в эквивалентном виде:
\begin{equation*}
	q(x) = \lim_{n\to\infty} \liminf_{m\to\infty}  \frac{1}{n} \sum_{k=m+1}^{m+n} x_k
	~~~~\mbox{и}~~~~
	p(x) = \lim_{n\to\infty} \limsup_{m\to\infty}  \frac{1}{n} \sum_{k=m+1}^{m+n} x_k
	.
\end{equation*}

Пространство $ac$ имеет интересную структуру.
За глобальным обзором его свойств отсылаем читателя к~\cite{semenov2006ac};
ряд интересных фактов можно почерпнуть в~\cite{usachev2008transforms},
а также в диссертации А.С. Усачёва~\cite{usachev2009_phd_vsu}.
%TODO2: включить англоязычную версию!
Стоит также отметить недавнюю работу~\cite{zvolinsky2021subspace},
в которой исследуется почти сходимость последовательностей,
определённых с помощью тригонометрических функций.

Часто мы будем иметь дело не с пространством всех почти сходящихся последовательностей $ac$,
а с его подпространством $ac_0$ последовательностей, почти сходящихся к нулю.
%TODO2: $ac$ не замкнуто относительно умножения.
%Рассмотрим  1/2, 1/2, 1, 0, 1, 0, 1/2, 1/2, 1/2, 1/2, 1, 0, 1, 0, 1, 0, 1, 0, 1/2, ...
% и          1/2, 1/2, 0, 1, 0, 1, 1/2, 1/2, 1/2, 1/2, 0, 1, 0, 1, 0, 1, 0, 1, 1/2, ...
Пространство $ac_0$ имеет ряд особенностей,
существенно отличающих его от пространства $c_0$ последовательностей, сходящихся к нулю.
То, что $ac_0\ne c_0$, показывает следующий классический
\begin{example}
	Пусть
	\begin{equation}
		x_k = \begin{cases}
			1, & ~\mbox{если}~ k = 2^n, ~ n\in\N,
			\\
			0  & ~\mbox{иначе}.
		\end{cases}
	\end{equation}
	Тогда $x\in ac_0 \setminus c_0$.
\end{example}
Более того, в отличие от пространства $c_0$, пространство $ac_0$ не замкнуто относительно
оператора взятия подпоследовательности, относительно покоординатного умножения и относительно возведения в степень.
Три этих свойства показывает
\begin{example}
	Пусть $x_n = (-1)^{n+1}$,
	т.е. $x = (1, -1,1,-1,1, -1,1,-1,...)$.
	Тогда $x\in ac_0$ в силу периодичности (см. следствие~\ref{thm:period_ac_avg}),
	но, очевидно, $x\cdot x = x^2\in ac_1$.
	Если же мы рассмотрим оператор перехода к подпоследовательности с чётными индексами
	\begin{equation}
		E(y_1, y_2, ...)  = (y_2, y_4, y_6, ...)
		,
	\end{equation}
	то обнаружим, что $Ex = (-1,-1,-1,-1,...)\in ac_{-1}$.
\end{example}

Можно привести и пример, когда взятие подпоследовательности выводит из всего пространства $ac$.
\begin{example}
	Пусть $x_n = (-1)^{n+1}$,
	т.е. $x = (1, -1,1,-1,1, -1,1,-1,...)$.
	Рассмотрим подпоследовательность
	\begin{equation}
		y = (1,-1, \; 1,1, -1,-1, \; 1,1,1, -1,-1,-1, \; 1,1,1,1,...)
		.
	\end{equation}
	Легко видеть, что верхний и нижний функционалы Сачестона принимают на последовательности $y$
	различные значения:
	$p(y) =1$, $q(y) = -1$.
	Следовательно, $y\notin ac$.
\end{example}

Е.А. Алехно доказал~\cite{alekhno2012superposition},
что в $ac_0$ существует максимальный идеал по умножению, обозначаемый $\Iac$ и,
более того, этот идеал может быть ёмко описан следующим критерием:
\begin{theorem}
	\label{thm:Iac_criterion_pos_neg}
	Пусть $x\in ac_0$.
	Последовательность $x\in\Iac$ тогда и только тогда,
	$x = x^+ +x^-$, $x^+\geq 0$, $x^- \leq 0$ и $x^+ \in ac_0$
	(последнее включение эквивалентно условию $x^- \in ac_0$).
\end{theorem}

%TODO2: а что там с идеалом по умножению в ac ?

Более того, $\Iac$ является подпространством в $ac_0$.
%TODO2: что там с недополняемостью?

Е.А. Алехно также исследовал~\cite{alekhno2012superposition,alekhno2015banach,ASSU2}
\emph{стабилизатор} пространства $ac_0$:
\begin{equation}
	\Dac = \{x\in\ell_\infty : x\cdot y \in ac_0 \mbox{~для любого~} y \in ac_0\}
\end{equation}
(встречается~\cite{Luxemburg} также обозначение $\operatorname{St} (ac_0)$).

$\Dac$ также является подпространством (уже в $\ell_\infty$),
однако в настоящей работе в дальнейшем не используется,
и потому мы не будем останавливаться на его свойствах;
отсылаем читателя к~\cite{Luxemburg}.
%TODO2: что там с недополняемостью?

Результаты, излагаемые в данной главе, опубликованы в%
~\cite{
our-mz2019ac0,
avdeed2021AandA,
our-mz2019measure,
}.



	\section{Переформулировка критерия Лоренца почти сходимости последовательности к нулю}
	Результаты этого пункта использованы в~\cite{our-mz2019ac0}.

Дадим переформулировку критерия Лоренца
\cite{lorentz1948contribution,bennett1974consistency}
почти сходимости последовательности,
которая иногда позволяет упростить доказательство:
брать предел равномерно не по всем $m\in \N$,
а только по достаточно большим значениям.


\begin{theorem}
	[Модифицированный критерий Лоренца]
	\label{thm:Lorentz_mod}
	Пусть $x\in\ell_\infty$.

	$x\in ac_0$ тогда и только тогда, когда
	\begin{equation}\label{crit_pos_ac0}
		\forall(A_2\in\N)
		\exists(n_0\in\N)
		\exists(m_0\in\N)
		\forall(n\geq n_0)
		\forall(m\geq m_0)
		\\
		\left[
			\left|
			\frac{1}{n}
			\sum_{k=m+1}^{m+n} x_k
			\right|
			<
			\frac{1}{A_2}
		\right]
		.
	\end{equation}

\end{theorem}

\begin{proof}
	По теореме Лоренца $x\in ac_0$ тогда и только тогда, когда
	\begin{equation}\label{Lorencz_ac0}
		\lim_{n\to\infty} \frac{1}{n} \sum_{k=m+1}^{m+n} x_k = 0
	\end{equation}
	равномерно по $m$.

	Или, переводя на язык кванторов,
	\begin{equation}\label{crit_ac0}
		\forall(A_1\in\N)
		\exists(n_1\in\N)
		\forall(n\geq n_0)
		\forall(m \in \N)
		\\
		\left[
			\left|
			\frac{1}{n}
			\sum_{k=m+1}^{m+n} x_k
			\right|
			<
			\frac{1}{A_1}
		\right]
		.
	\end{equation}
	Очевидно, что из \eqref{crit_ac0} следует \eqref{crit_pos_ac0} (например, положив $m_0 = 1$),
	тем самым необходимость \eqref{crit_pos_ac0} доказана.

	\paragraph{Достаточность.}
	Пусть выполнено \eqref{crit_pos_ac0}.
	Покажем, что выполнено \eqref{crit_ac0}.
	Зафиксируем $A_1$.
	Положим $A_2 = 2A_1$ и отыщем $n_0$ и $m_0$ в соотвествии с \eqref{crit_pos_ac0}.
	Положим $n_1 = 2A_2(n_0+m_0)\|x\|$.
	Покажем, что \eqref{crit_ac0} верно для любых $n\geq n_1$, $m\in \N$.
	Зафиксируем $n$ и рассмотрим $m$.

	Пусть сначала $m\geq m_0$.
	Тогда в силу того, что $n\geq n_1 = 2A_2(n_0+m_0)\|x\| > n_0$ имеем $n>n_0$.
	Применим \eqref{crit_pos_ac0}:
	\begin{equation}
		\left|
		\frac{1}{n}
		\sum_{k=m+1}^{m+n} x_k
		\right|
		<
		\frac{1}{A_2}
		=
		\frac{1}{2A_1}
		<
		\frac{1}{A_1}
		,
	\end{equation}
	т.е. \eqref{crit_ac0} выполнено.

	Пусть теперь $m < m_0$.
	Заметим, что
	\begin{equation}
		\left|
			\sum_{k=m+1}^{m+n} x_k
			-
			\sum_{k=m_0+1}^{m_0+n} x_k
		\right|
		\leq 2(m_0 - m) \|x\|
		,
	\end{equation}
	откуда
	\begin{equation}
		\left| \sum_{k=m+1}^{m+n} x_k \right|
		\leq
		2(m_0 - m) \|x\| + \left| \sum_{k=m_0+1}^{m_0+n} x_k \right|
		\leq
		2 m_0 \|x\| + \left| \sum_{k=m_0+1}^{m_0+n} x_k \right|
		.
	\end{equation}


	Тогда
	\begin{multline}
		\left| \frac{1}{n} \sum_{k=m+1}^{m+n} x_k \right|
		\leq
		\frac{2 m_0 \|x\|}{n} + \left| \frac{1}{n} \sum_{k=m_0+1}^{m_0+n} x_k \right|
		\mathop{\leq}^{\mbox{~~в силу \eqref{crit_pos_ac0}~~}}
		\frac{2 m_0 \|x\|}{n} + \frac{1}{A_2}
		\leq
		\\ \leq
		\frac{2 m_0 \|x\|}{n_1} + \frac{1}{A_2}
		\leq
		\frac{2 m_0 \|x\|}{2A_2(n_0+m_0)\|x\|} + \frac{1}{A_2}
		=
		\\=
		\frac{m_0}{A_2(n_0+m_0)} + \frac{1}{A_2}
		<
		\frac{1}{A_2} + \frac{1}{A_2}
		=
		\frac{1}{A_1}
		,
	\end{multline}
	т.е. \eqref{crit_ac0} тоже выполнено.
\end{proof}

Удобство критерия \eqref{crit_pos_ac0} в том,
что можно выбирать $m_0$ в зависимости от $A_2$.


	\section{О почти сходимости к нулю последовательности из нулей и единиц}
	\paragraph{Задача о пасьянсе из нулей и единиц.}
Пусть $n_i$~--- строго возрастающая последовательность натуральных чисел,
\begin{equation}
	M(j) = \liminf_{i\to\infty} n_{i+j} - n_i,
\end{equation}
\begin{equation}
	x_k = \left\{\begin{array}{ll}
		1, & \mbox{~если~} k = n_i
		\\
		0  & \mbox{~иначе~}
	\end{array}\right.
\end{equation}

\paragraph{Замечание.}
Так как $M(j)$ есть нижний предел последовательности натуральных чисел,
то он всегда достигается,
т.е. $M(j)\in\mathbb{N}$.

Более того, для любого $j$ существует лишь конечное количество отрезков длины $M(j)$,
содержащих более $j$ единиц,
и бесконечное количество отрезков длины $M(j)$,
содержащих ровно $j$ единиц.

Через $E_j$ будем обозначать конец последнего отрезка длины $M(j)$,
содержащего более $j$ единиц.

\paragraph{Утверждение.}
Если $x \in ac_0$, то
\begin{equation}\label{lim_M(j)/j}
	\lim_{j \to \infty} \frac{M(j)}{j} = +\infty
	.
\end{equation}

\paragraph{Доказательство.}
Очевидно, что если
\begin{equation}
	\liminf_{j \to \infty} \frac{M(j)}{j} = +\infty
	,
\end{equation}
то выполнено \eqref{lim_M(j)/j}.
Предположим противное:
\begin{equation}
	\liminf_{j \to \infty} \frac{M(j)}{j} = с < +\infty
	.
\end{equation}
Очеивдно, что в таком случае $c>0$.

По определению нижнего предела найдётся счётное множество
$J\subset\mathbb{N}$ такое, что
\begin{equation}
	\forall(j\in J)\left[c \leq \frac{M(j)}{j} \leq c+1 \right],
\end{equation}
т.е. для любого $j\in J$ существует бесконечно много отрезков длины $j\cdot(c+1)$,
на каждом из которых не менее $j$ единиц.

Т.к. $x\in ac_0$, то
\begin{equation}\label{Lorencz_ac0_epsilon}
	\forall(\varepsilon>0)
	\exists(n_0\in\mathbb{N})
	\forall(n \geq n_0)
	\forall(m\in\mathbb{N})
	\left[
		\frac{1}{n} \sum_{k=m+1}^{m+n}x_k < \varepsilon
	\right]
	.
\end{equation}

Положим $\varepsilon = 1/(c+2)$ и отыщем $n_0$.
Положим $n\in J$, $n\geq n_0$.
(Такое $n$ всегда найдётся, т.к. $J$ счётно и $J\subset\mathbb{N}$.)
Выберем $m$ так, чтобы отрезок длины $n\cdot(c+1)$,
содержащий не менее $n$ единиц,
начинался с $m+1$.
Тогда
\begin{equation}
	\frac{1}{n\cdot(c+1)}\sum_{k=m+1}^{m+n\cdot(c+1)}x_k
	\geq
	\frac{1}{n\cdot(c+1)} \cdot n
	=
	\frac{1}{c+1}
	>
	\frac{1}{c+2}
	=
	\varepsilon,
\end{equation}
что противоречит \eqref{Lorencz_ac0_epsilon}.

Полученное противоречие завершает доказательство.


\paragraph{Утверждение.}
Если
\begin{equation}\label{lim_M(j)/j_dost}
	\lim_{j \to \infty} \frac{M(j)}{j} = +\infty
	,
\end{equation}
то $x \in ac_0$.

\paragraph{Доказательство.}

По определению предела \eqref{lim_M(j)/j_dost} означает, что
\begin{equation}\label{lim_M(j)/j_ifty_def}
	\forall(C  \in\mathbb{N})
	\exists(j_0\in\mathbb{N})
	\forall(j \geq j_0)
	\left[
		\frac{M(j)}{j}>C
	\right]
	.
\end{equation}
Покажем, что выполнен модификированный критерий Лоренца почти сходимости последовательности к нулю
\eqref{crit_pos_ac0}, т.е.
\begin{equation}
	\forall(B  \in\mathbb{N})
	\exists(n_0\in\mathbb{N})
	\exists(m_0\in\mathbb{N})
	\forall(n\geq n_0)
	\forall(m\geq m_0)
	\\
	\left[
		\frac{1}{n}
		\sum_{k=m+1}^{m+n} x_k
		<
		\frac{1}{B}
	\right]
	.
\end{equation} Действительно, зафиксируем $B$.
Используя \eqref{lim_M(j)/j_ifty_def} и положив $C=2B$,
отыщем $j_0$ такое, что для любого $j\geq j_0$ выполнено
$M(j)>2Bj$.
Положим $n_0 = 2Bj_0$.
Выберем
$$
	m_0 = 2+\max_{1\leq j \leq j_0} E_j
	.
$$

Тогда для любых $m\geq m_0$ и $n\geq n_0$ имеем
\begin{multline}
	\frac{1}{n} \sum_{k=m+1}^{m+n} x_k
	<
	\frac{1}{n} \cdot \left( \frac{n}{M(j_0)} + 1 \right) j_0
	=
	\frac{j_0}{M(j_0)} + \frac{j_0}{n}
	\leq
	\frac{1}{2B} + \frac{j_0}{n}
	\leq
	\\ \leq
	\frac{1}{2B} + \frac{j_0}{n_0}
	=
	\frac{1}{2B} + \frac{1}{2B}
	=
	\frac{1}{B}
	,
\end{multline}
т.е. условие критерия выполнено
и $x\in ac_0$,
что и требовалось доказать.

Последовательность $\{M(j)\}$, как легко выяснить, удовлетворяет некоторым условиям.


	\section{Замечание о свойствах последовательности $M(j)$}
	
Пусть $n_i$~--- строго возрастающая последовательность натуральных чисел,
\begin{equation}
	C_j = \liminf_{i\to\infty} n_{i+j} - n_i
\end{equation}

Тогда для любых $i$, $j$ имеет место быть
\begin{equation}\label{C_j_addit}
	C_{i+j} \geq C_i + C_j
	.
\end{equation}
\paragraph{Доказательство.}
По определению нижнего предела существует лишь конечное число номеров $k$
таких, что $n_{k+i} - n_k < C_i$ или $n_{k+j} - n_k < C_j$.
Зафиксируем $p$, большее всех таких $k$.

По определению нижнего предела и с учётом того, что выражение под знаком предела
принимает лишь натуральные значения,
существует бесконечное количество номеров $q$ таких, что $n_{q+i+j} - n_q = C_{i+j}$.

Обозначим через $s$ некоторый такой номер, больший $p$.
Тогда
\begin{multline}
	C_{i+j} = n_{s+i+j} - n_s = n_{s+i+j} - n_{s+i} + n_{s+i} - n_s
	\geq \\
	\geq C_j + n_{s+i} - n_s \geq C_j + C_i,
\end{multline}
так как из $s>p$ следует, что $n_{s+i+j} - n_{s+i} \geq C_j$ и $n_{s+i} - n_s \geq C_i$.

\paragraph{Пример.}
Последовательность $C_j = j+1$ не удовлетворяет условию \eqref{C_j_addit}:
действительно, $3=C_2 < C_1+C_1 = 2+2 = 4$.

\paragraph{Примечание.}
Условие \eqref{C_j_addit} является необходимым, но неизвестно, является ли оно достаточным.


	\section{Существование предела последовательности $M(j)$}
	\begin{theorem}
	Пусть $M(j)$~--- последовательность, определённая в~\eqref{eq:definition_M_j}.
	Тогда последовательность $M(j)/j$ имеет предел (конечный или бесконечный).
\end{theorem}

\paragraph{Доказательство.}
Предположим противное.
Пусть
\begin{equation}
	\varliminf_{j\to\infty}\frac{M(j)}{j} = A,
\end{equation}
\begin{equation}
	\varlimsup_{j\to\infty}\frac{M(j)}{j} \geq B
\end{equation}
(неравенство используется для охвата случая, когда верхний предел равен бесконечности)
и
\begin{equation}
	B > A
	.
\end{equation}

Положим $\varepsilon = \frac{B-A}{4}$.
По определению верхнего предела существует такое натуральное $j_0$,
что
\begin{equation}
	\frac{M(j_0)}{j_0} > B - \varepsilon
	.
\end{equation}
По определению нижнего предела существует $\{j_i\}$~--- возрастающая последовательность таких индексов, что
\begin{equation}
	\label{eq:lim_Mj_inf_lim_seq}
	\forall(i\in\mathbb{N})\left[ \frac{M(j_i)}{j_i} < A + \varepsilon \right]
	.
\end{equation}
Определим для каждого $i\in\mathbb{N}$ целое число $C_i$ такое, что
\begin{equation}
	\label{eq:lim_Mj_inf_lim_Ci}
	C_i j_0 \leq j_i < (C_i+1)j_0
	.
\end{equation}
Легко заметить, что последовательность $\{C_i\}$ является возрастающей.

Очевидно, что
\begin{equation}
	M(Cj) \geq C \cdot M(j)
	.
\end{equation}
Таким образом,
\begin{equation}
	M(j_i) \geq M(C_i j_0) \geq C_i M(j_0) > C_i j_0 (B-\varepsilon)
\end{equation}
и
\begin{multline}
	\frac{M(j_i)}{j_i} > \frac{C_i j_0 (B-\varepsilon)}{j_i}
	> \frac{C_i j_0 (B-\varepsilon)}{(C_i+1)j_0}
	=
	\\=
	\frac{C_i}{C_i+1}(B-\varepsilon)
	= B-\varepsilon - \frac{1}{C_i+1}(B-\varepsilon)
	.
\end{multline}
Выберем $i$ настолько большим, что
\begin{equation}
	\frac{1}{C_i+1}(B-\varepsilon) < \varepsilon
	.
\end{equation}
Тогда
\begin{equation}
	\frac{M(j_i)}{j_i} > B - 2 \varepsilon = A + 2 \varepsilon > A + \varepsilon
	,
\end{equation}
что противоречит~\eqref{eq:lim_Mj_inf_lim_seq}.
Полученное противоречие завершает доказательство.

\begin{hypothesis}
	Пусть $M(j)$~--- последовательность натуральных чисел, такая, что существует предел
	\begin{equation}
		\lim_{j\to\infty} \frac{M(j)}{j} \leq +\infty
	\end{equation}
	и для любых $i,j\in\mathbb{N}$ выполнено
	\begin{equation}
		M(i)+M(j) \leq M(i+j)
		.
	\end{equation}
	Тогда существует последовательность $\{x_k\}$,
	удовлетворяющая условию~\eqref{eq:definition_M_j}.
\end{hypothesis}


	\section{Срезочный критерий почти сходимости к нулю неотрицательной последовательности}
	\documentclass[a4paper,14pt]{article} %размер бумаги устанавливаем А4, шрифт 12пунктов
\usepackage[T2A]{fontenc}
\usepackage[utf8]{inputenc}
\usepackage[english,russian]{babel} %используем русский и английский языки с переносами
\usepackage{amssymb,amsfonts,amsmath,mathtext,cite,enumerate,float,amsthm} %подключаем нужные пакеты расширений
\usepackage[unicode,colorlinks=true,citecolor=black,linkcolor=black]{hyperref}
%\usepackage[pdftex,unicode,colorlinks=true,linkcolor=blue]{hyperref}
\usepackage{indentfirst} % включить отступ у первого абзаца
\usepackage[dvips]{graphicx} %хотим вставлять рисунки?
\graphicspath{{illustr/}}%путь к рисункам

\makeatletter
\renewcommand{\@biblabel}[1]{#1.} % Заменяем библиографию с квадратных скобок на точку:
\makeatother %Смысл этих трёх строчек мне непонятен, но поверим "Запискам дебианщика"

\usepackage{geometry} % Меняем поля страницы.
\geometry{left=2cm}% левое поле
\geometry{right=1cm}% правое поле
\geometry{top=2cm}% верхнее поле
\geometry{bottom=2cm}% нижнее поле

\renewcommand{\theenumi}{\arabic{enumi}}% Меняем везде перечисления на цифра.цифра
\renewcommand{\labelenumi}{\arabic{enumi}}% Меняем везде перечисления на цифра.цифра
\renewcommand{\theenumii}{.\arabic{enumii}}% Меняем везде перечисления на цифра.цифра
\renewcommand{\labelenumii}{\arabic{enumi}.\arabic{enumii}.}% Меняем везде перечисления на цифра.цифра
\renewcommand{\theenumiii}{.\arabic{enumiii}}% Меняем везде перечисления на цифра.цифра
\renewcommand{\labelenumiii}{\arabic{enumi}.\arabic{enumii}.\arabic{enumiii}.}% Меняем везде перечисления на цифра.цифра

\sloppy


\renewcommand\normalsize{\fontsize{14}{25.2pt}\selectfont}

\begin{document}
% !!!
% Здесь начинается реальный ТеХ-код
% Всё, что выше - беллетристика

Определим (нелинейный) оператор $\lambda$--срезки $A_\lambda$ на пространстве $\ell_\infty$.
Для $x = (x_1, x_2, ...) \in \ell_\infty$ положим
\begin{equation}
	(A_\lambda x)_k = \begin{cases}
		1, & \mbox{~если~} x_k \geq \lambda
		\\
		0  & \mbox{~иначе.~}
	\end{cases}
\end{equation}

\paragraph{Теорема.}
(Срезочный критерий почти сходимости к нулю неотрицательной последовательности.)
Пусть $x\in\ell_\infty$, $x\geq 0$.
Тогда
\begin{equation}
	x\in ac_0 \Leftrightarrow
	\forall(\lambda>0)[A_\lambda x \in ac_0]
	.
\end{equation}

\paragraph{Необходимость.}
Пусть $x\in ac_0$.
Зафиксируем $\lambda > 0$.
Пусть $y=A_\lambda x$, тогда
\begin{equation}
	(\lambda y)_k = \begin{cases}
		\lambda, & \mbox{~если~} x_k \geq \lambda
		\\
		0  & \mbox{~иначе~}
	\end{cases}
\end{equation}
Таким образом, $0 \leq \lambda y \leq x$.
Следовательно, если $x \in ac_0$,
то $\lambda y \in ac_0$ и $y \in ac_0$,
что и требовалось доказать.

\paragraph{Достаточность.}
Очевидно, что
\begin{equation}
	x\in ac_0 \Leftrightarrow
	\frac{x}{\|x\|}\in ac_0
	.
\end{equation}
Поэтому, не теряя общности, будем полагать $\|x\|\leq 1$.
Более того,
\begin{equation}
	A_\lambda x \in ac_0 \Leftrightarrow
	(1-\lambda)A_\lambda x \in ac_0
	.
\end{equation}

Преположим противное, т.е. $x\notin ac_0$,
но $\forall(\lambda>0)[(1-\lambda))A_\lambda x \in ac_0]$.
Запишем покванторное отрицание критерия Лоренца:
\begin{equation}\label{ac0_lambda_Lorencz_neg}
	\exists(\varepsilon_0 > 0)
	\forall(n_0 \in \mathbb{N})
	\exists(n > n_0)
	\exists(m \in \mathbb{N})
	\left[
		\frac{1}{n}\sum_{k=m+1}^{m+n} x_k \geq \varepsilon_0
	\right]
	.
\end{equation}
Найдём такое $\varepsilon_0$ и положим
\begin{equation}
	\varepsilon = \min\left\{ \frac{\varepsilon_0}{2}, \frac{1}{2} \right\}
	.
\end{equation}
Легко видеть, что
\begin{equation}\label{ac0_lambda_Lorencz_neg_epsilon}
	\forall(n_0 \in \mathbb{N})
	\exists(n > n_0)
	\exists(m \in \mathbb{N})
	\left[
		\frac{1}{n}\sum_{k=m+1}^{m+n} x_k > \varepsilon
	\right]
	.
\end{equation}
(знак неравенства сменился на строгий, это будет играть ключевую роль в дальнейших выкладках).

Построим последовательность
\begin{equation}
	y = \left( 1 - \frac{\varepsilon}{2} \right) A_{\varepsilon/2} x
	.
\end{equation}
Заметим, что $y\in ac_0$, т.е. по критерию Лоренца
\begin{equation}\label{ac0_lambda_Lorencz}
	\forall(\varepsilon_1 > 0)
	\exists(n_1 \in \mathbb{N})
	\forall(n' > n_1)
	\forall(m' \in \mathbb{N})
	\left[
		\frac{1}{n}\sum_{k=m+1}^{m+n} y_k < \varepsilon_1
	\right]
	.
\end{equation}

Положим в \eqref{ac0_lambda_Lorencz} $\varepsilon_1 = \varepsilon/2$
и отыщем $n_1$.
Положим в \eqref{ac0_lambda_Lorencz_neg_epsilon}
$n_0 = n_1$ и отыщем $n$ и $m$.
Положим в \eqref{ac0_lambda_Lorencz} $n' = n$, $m' = m$.
Тогда получим, что по \eqref{ac0_lambda_Lorencz_neg_epsilon}
\begin{equation}
	\frac{1}{n}\sum_{k=m+1}^{m+n} x_k > \varepsilon
	.
\end{equation}
С другой стороны, по \eqref{ac0_lambda_Lorencz}
\begin{equation}
	\frac{1}{n}\sum_{k=m+1}^{m+n} y_k < \varepsilon/2
	.
\end{equation}
Вычитая, получим
\begin{equation}
	\frac{1}{n}\sum_{k=m+1}^{m+n} (x_k - y_k) > \varepsilon/2
	.
\end{equation}
Если среднее арифметическое чисел вида $x_k - y_k$ больше $\varepsilon/2$,
то существует хотя бы один индекс $k$ такой, что $x_k - y_k > \varepsilon/2$.

Предположим, что $k$ таково, что $x_k < \varepsilon/2$.
Тогда $y_k = 0$ и $x_k - y_k < \varepsilon/2$.
Значит, предположение неверно и $x_k \geq \varepsilon/2$
Тогда $y_k = 1-\varepsilon/2$ и с учётом $\|x\|\leq 1$ имеем
\begin{equation}
	x_k - y_k \leq 1- y_k = 1 - (1-\varepsilon/2) = \varepsilon/2
	.
\end{equation}
Следовательно, требуемого индекса $k$ не существует,
и \eqref{ac0_lambda_Lorencz_neg} не выполнено.

Полученное противоречие завершает доказательство.


TODO: сформулировать критерий в терминах $M_\lambda(j)$.

\end{document}


	\section{Пространство $ac_0$ и $\alpha$--функция}
	Докажем теперь ещё одну теорему,
раскрывающую связь между сходимостью, почти сходимостью и $\alpha$--функцией.

\paragraph{Теорема.}
Пусть $x\in ac_0$ и $\alpha(x)=0$.
Тогда $x \in c_0$.

\paragraph{Доказательство.}

Предположим противное, т.е. $x\notin c_0$.
Тогда
\begin{equation}
	\varlimsup_{k\to\infty} x_k \neq 0 ~~\mbox{или}~~ \varliminf_{k\to\infty} x_k \neq 0
	.
\end{equation}

Не теряя общности, положим
\begin{equation}
	\varepsilon = \varlimsup_{k\to\infty} x_k > 0
\end{equation}
(иначе домножим всю последовательность на $-1$, что, очевидно, не повлияет на сходимость к нулю).

Тогда существует бесконечно много таких $n$, что
\begin{equation}\label{alpha_ac0_c0_limsup}
	x_n > \varlimsup_{k\to\infty} x_k - \frac{\varepsilon}{4} = \frac{3\varepsilon}{4}
	.
\end{equation}

Так как
\begin{equation}
	\alpha(x) = \varlimsup_{i\to\infty} \max_{i \leq j \leq 2i} |x_i-x_j| = 0
	,
\end{equation}
то
\begin{equation}
	\exists(N_1\in\mathbb{N})\forall(n > N_1)\left[\max_{n \leq j \leq 2n} |x_n-x_j| < \frac{\varepsilon}{4}\right]
	,
\end{equation}
или, что то же самое,
\begin{equation}\label{alpha_ac0_c0_alpha}
	\exists(N_1\in\mathbb{N})\forall(n > N_1)\forall(j: n \leq j \leq 2n)\left[ |x_n-x_j| < \frac{\varepsilon}{4}\right]
	.
\end{equation}

Поскольку $x \in ac_0$, то  по критерию Лоренца
\begin{equation}
	\exists(N_2 \in\mathbb{N})\forall(n > N_2)\forall(m\in\mathbb{N})
	\left[ \left| \frac{1}{n}\sum_{k=m+1}^{m+n}x_k\right| < \frac{\varepsilon}{4} \right]
	,
\end{equation}
в частности,
\begin{equation}\label{alpha_ac0_c0_Lorencz}
	\exists(N_2 \in\mathbb{N})\forall(n > N_2)
	\left[ \left| \frac{1}{n}\sum_{k=n+1}^{2n}x_k\right| < \frac{\varepsilon}{4} \right]
	,
\end{equation}

Выберем $n$ так, чтобы оно удовлетворяло \eqref{alpha_ac0_c0_limsup} и \eqref{alpha_ac0_c0_alpha}.
Тогда
\begin{multline}
	\left| \frac{1}{n}\sum_{k=n+1}^{2n}x_k\right|
	=
	\\=
	\mbox{(по \eqref{alpha_ac0_c0_alpha} имеем $x_k \geq 3\varepsilon/4 > 0$)}
	=
	\\=
	\frac{1}{n}\sum_{k=n+1}^{2n}x_k
	\geq
	\frac{3\varepsilon}{4}
	>
	\frac{\varepsilon}{4}
	,
\end{multline}
что противоречит \eqref{alpha_ac0_c0_Lorencz}.

Полученное противоречие завершает доказательство.



\paragraph{Следствие.}
Пусть $x\in ac$ и $\alpha(x)=0$.
Тогда $x \in c$.

TODO: доказывать нужно или очевидно?


	\section{$\alpha$--функция на пространстве $ac$ и расстояние до пространства $c$}
	Пусть $\rho(x,c)$ и $\rho(x,c_0)$~--- расстояния от $x$ до пространства сходящихся последовательностей $c$
и пространства сходящихся к нулю последовательностей $c_0$ соответственно.

Из условия Липшица на $\alpha$--функцию \eqref{alpha_Lipshitz}
и того, что на пространстве $c$
всех сходящихся последовательностей
$\alpha$--функция обращается в нуль следует

\begin{lemma}
\label{thm:alpha_x_leq_2_rho_x_c}
	Для любого $x\in\ell_\infty$
	\begin{equation}
		\alpha(x) \leq 2\rho(x, c)
		.
	\end{equation}
\end{lemma}

%TODO2: доказывать или очевидно?

Эта оценка точна.
\begin{example}
\label{ex:alpha_ac_rho_x_c}
	\begin{equation}
	\label{eq:alpha_ac_distance_example_y}
		x_k = \begin{cases}
			(-1)^n, &\mbox{~если~} k = 2^n,
			\\
			0 &\mbox{~иначе.}
		\end{cases}
	\end{equation}
\end{example}
Здесь $\alpha(x) = 2$, $\alpha(T^s x) = 1$ для любого $s\in\N$, $\rho(x,c) = \rho(x, c_0) = 1$.

Как выясняется, верна и оценка с другой стороны.

\begin{lemma}
\label{thm:rho_x_c_leq_alpha_t_s_x}
	Для любого $x\in ac$ и для любого натурального $s$
	\begin{equation}
		\rho(x,c)\leq \alpha(T^s x)
		.
	\end{equation}
\end{lemma}

\begin{proof}
	Зафиксируем $s$.
	Пусть $x\in ac$, $\alpha(T^s x)=\varepsilon$.
	Так как $x\in ac$, то по критерию Лоренца существует такое число $t$,
	что
	\begin{equation}
		\lim_{n\to\infty} \frac{1}{n} \sum_{k=m+1}^{m+n} x_k = t
	\end{equation}
	равномерно по $m$.
	Иначе говоря,
	\begin{equation}
		\forall(p\in\N)
		\exists(n_p \in\N)\forall(n>n_p)\forall(m\in\N)
		\left[
			\left|
				\frac{1}{n}\sum_{k=m+1}^{m+n} x_k
				-t
			\right|
			<\frac{\varepsilon}{p}
		\right]
		.
	\end{equation}

	Так как
	\begin{equation}
		\alpha(T^s x) = \varlimsup_{k\to\infty} \max_{k<j\leq 2k-s} |x_k-x_j| = \varepsilon
		,
	\end{equation}
	то
	\begin{equation}
		\forall(p\in\N)
		\exists(k_p \in\N)\forall(k>k_p)
		\forall(j: k< j \leq 2k-s)
		\left[
			|x_k - x_j|<\varepsilon + \frac{\varepsilon}{p}
		\right]
		.
	\end{equation}
	Положив $q_p = \max\{n_p, k_p\}$, имеем
	\begin{multline}
		\forall(p\in\N)
		\exists(q_p \in\N)
		\forall(q>q_p)
		\left[
			\left|
				\frac{1}{n}\sum_{k=m+1}^{m+n} x_k
				-t
			\right|
			<\frac{\varepsilon}{p}
			\right.\\ \left. \phantom{\sum_0^0}
			\mbox{~~и~~}
			\forall(k:q<k \leq 2q-s)
			\left[
				|x_q - x_k|<\varepsilon + \frac{\varepsilon}{p}
			\right]
		\right]
		.
	\end{multline}
	Т.е. среднее арифметическое чисел $x_q$, $x_{q+1}$, ... , $x_{2q-s}$ отличается от $t$
	не более, чем на $\varepsilon/p$, причём разница любых двух из этих чисел меньше $\varepsilon + \varepsilon/p$.
	Следовательно, любое из чисел $x_q$, $x_{q+1}$, ... , $x_{2q-s}$
	отличается от $t$ менее, чем на $\varepsilon + 2\varepsilon/p$.
	В частности,
	\begin{equation}
		|x_q - t| < \varepsilon + \frac{2\varepsilon}{p}
		.
	\end{equation}
	Таким образом,
	\begin{equation}
		\forall(p\in\N)
		\exists(q_p \in\N)
		\forall(q>q_p)
		\left[
			|x_q - t| < \varepsilon + \frac{2\varepsilon}{p}
		\right]
		,
	\end{equation}
	откуда немедленно следует, что $\rho(x,c) \leq \varepsilon$,
	что и требовалось доказать.
\end{proof}


Из лемм \ref{thm:alpha_x_leq_2_rho_x_c} и \ref{thm:rho_x_c_leq_alpha_t_s_x}
незамедлительно следует
\begin{theorem}
	\label{thm:rho_x_c_leq_alpha_t_s_x_united}
	Для любого $x\in ac$
	\begin{equation}
		\frac{1}{2} \alpha(x) \leq \rho(x,c)\leq \lim_{s\to\infty} \alpha(T^s x)
		.
	\end{equation}
\end{theorem}

\begin{corollary}
	\label{cor:rho_x_c0_leq_alpha_t_s_x_united}
	Для любого $x\in ac_0$
	\begin{equation}
		\frac{1}{2} \alpha(x) \leq \rho(x,c_0)\leq \lim_{s\to\infty} \alpha(T^s x)
		.
	\end{equation}
\end{corollary}

Точность оценок показывают пример \ref{ex:alpha_ac_rho_x_c} и следующие примеры.
\begin{example}
	\begin{equation}
		x_k = \begin{cases}
			1, &\mbox{~если~} k = 2^n,
			\\
			0 &\mbox{~иначе.}
		\end{cases}
	\end{equation}
\end{example}
Здесь $\alpha(T^s x) = 1$ для любого $s\in\N$, $\rho(x,c) = 1/2$, $\rho(x, c_0) = 1$.

\begin{example}
	\begin{equation}
		x_k = \begin{cases}
			1, &\mbox{~если~} k = 2^n,
			\\
			-1, &\mbox{~если~} k = 2^n + 1 > 2
			\\
			0 &\mbox{~иначе.}
		\end{cases}
	\end{equation}
\end{example}
Здесь $\alpha(T^s x) = 2$ для любого $s\in\N$, $\rho(x,c) = \rho(x, c_0) = 1$.


\begin{hypothesis}
	Для любого $x\in ac$
	\begin{equation}
		\frac{1}{2} \lim_{s\to\infty} \alpha(U^s x) \leq \rho(x,c)
		,
	\end{equation}
	где $U$~--- оператор сдвига вправо:
	\begin{equation}
		U(x_1, x_2, x_3, ...) = (0, x_1, x_2, x_3, ...)
		.
	\end{equation}
\end{hypothesis}


	\section{Усиленная теорема Коннора}
	Пусть на множестве $\Omega=\{0,1\}^\mathbb{N}$ задана вероятностная мера <<честной монетки>> $\mu$.
Тогда, согласно~\cite{connor1990almost}, $\mu(\Omega\cap ac)=0$.

Обобщим этот результат.
\begin{theorem}
	Мера множества $F=\{x\in\Omega : q(x) = 0 \wedge p(x)= 1\}$,
	где $p(x)$ и $q(x)$~--- верхний и нижний функционалы Сачестона соответственно,
	равна 1.
\end{theorem}

\begin{proof}
	Пусть $F_1=\{x\in\Omega : p(x) \neq 1\}$, $F_0=\{x\in\Omega : q(x) \neq 0\}$.
	Заметим, что
	\begin{equation}
		\label{eq:Connor_gen}
		\mu F = \mu (\Omega\setminus(F_1 \cup F_0)) = 1 - \mu (F_1 \cup F_0)
		.
	\end{equation}
	Докажем, что $\mu F_1 = 0 $	(для $\mu F_0$  доказательство полностью аналогично).

	Согласно критерию Сачестона,
	\begin{equation}
		p(x) = \lim_{n\to\infty} \sup_{m\in\mathbb{N}} \frac{1}{n} \sum_{k=m+1}^{k=m+n} x_k
		,
	\end{equation}
	откуда следует, что
	для $x\in\Omega$ равенство $p(x) = 1$ выполнено тогда и только тогда,
	когда для любого $n$ последовательность $x$ содержит отрезок из $n$ единиц подряд.
%
	Следовательно, если $x\in F_1$,
	то существует такое $n$,
	что $x$ не содержит $n$ единиц подряд.
%
%
	Обозначим
	\begin{equation}
		A_n^k = \{x\in\Omega : x_{kn+1} = ... = x_{kn+n} = 1\}
		,
		~~~
		B_n^k = \Omega \setminus A_n^k
		%,
		%B_n = \bigcap_{k\in\mathbb{N}} B_n^k
		.
	\end{equation}
	Тогда
	\begin{equation}
		\forall(x\in F_1)\exists(n_x\in\mathbb{N})\forall(k\in\mathbb{N})[x\in B_{n_x}^k]
		,
	\end{equation}
	т.е.
	\begin{equation}
		F_1 \subset \bigcup_{n\in\mathbb{N}} \bigcap_{k\in\mathbb{N}} B_{n}^k
		.
	\end{equation}
	Учитывая, что при $i\neq j$ события $x\in B_n^i$ и $x\in B_n^j$ независимы и $\mu B_n^j = 1-\frac{1}{2^n}$,
	получаем
	\begin{multline}
		\mu F_1 \leq \mu \bigcup_{n\in\mathbb{N}} \bigcap_{k\in\mathbb{N}} B_{n}^k
		=
		\sum_{n\in\mathbb{N}} \mu \bigcap_{k\in\mathbb{N}} B_{n}^k
		=
		\sum_{n\in\mathbb{N}}  \prod_{k\in\mathbb{N}} \mu B_{n}^k
		=
		\sum_{n\in\mathbb{N}}  \prod_{k\in\mathbb{N}} \left( 1-\frac{1}{2^n} \right)
		=0
		.
	\end{multline}
	Тем самым получаем из \eqref{eq:Connor_gen}, что $\mu F = 1$.
\end{proof}


	\section{О мере одного множества}
	Результаты данного пункта получены совместно с А.С. Усачёвым.

Пусть на множестве $\Omega=\{0,1\}^\mathbb{N}$ задана вероятностная мера <<честной монетки>> $\mu$.
Тогда, согласно~\cite{connor1990almost}, $\mu(\Omega\cap ac)=0$.
Применим этот результат к нахождению меры множества $W$,
введённого в~\cite[\S 5]{Semenov2014geomprops}.

Пусть $W$~--- множество всех последовательностей $\chi_e$, где $e =\bigcup_{k=1}^{\infty} [n_{2k-1}, n_{2k} )$
и $\{n_k \}_{k=1}^{\infty}$
удовлетворяет условию
\begin{equation}
	\label{eq:lim_j_n_kj_measure}
	\lim_{j\to\infty}\frac{n_{k+j} - n_k}{j} = \infty
\end{equation}
равномерно по $k \in \mathbb{N}$.

Вероятностная мера <<честной монетки>> означает,
что каждая последовательность из нулей и единиц соответствует бесконечной серии
бросков честной монетки, причём выпадение орла означает нуль, а выпадение решки~--- единицу.
(В практических целях рекомендуем читателю использовать рубль, поскольку он падает быстрее и охотнее.)
%TODO: выпилить!!!!
Тогда мера подмножества $\omega\subset\Omega$
равна вероятности события <<выпала одна из серий монетки, закодированных в $\omega$>>.
Например, $\mu(\Omega)=1$, $\mu(\{x\in\Omega:x_1=1, x_2=0\})=1/4$.

Введём теперь нелинейную биекцию $Q:\Omega\leftrightarrow\Omega$ по следующему правилу:
\begin{equation}
	(Qx)_k = \begin{cases}
		x_k, &\mbox{если~} k = 1,
		\\
		|x_k-x_{k-1}|&\mbox{иначе}.
	\end{cases}
\end{equation}

\begin{lemma}
	Биекция $Q$ сохраняет меру множества.
\end{lemma}

\paragraph{Идея.}
	Пусть последовательность $x$ соответствует серии бросков монетки так, как описано выше.
	Будем теперь интерпретировать ту же самую серию бросков иначе.
	Первый бросок интерпретируем так же,
	а начиная со второго сопоставим нулю событие <<выпала та же сторона монетки, что и в прошлый раз>>,
	а единице~--- противоположное событие.
	Cобытия <<выпала решка>> и <<выпала та же сторона монетки, что и в прошлый раз>> независимы
	и вероятность каждого из них равна $1/2$.
	Осталось заметить, что новая интерпретация той же серии бросков дала нам последовательность $Qx$.
\begin{proof}
	В рамках данного доказательства будем соотносить последовательности из нулей и единиц
	с точками полуинтервала $[0;1)$, представленными в виде двоичных дробей.
	Назовём двоичным отрезком множество последовательностей из $\Omega$,
	в котором первые $k$ координат зафиксированы, а остальные выбираются произвольно.
	Двоичный отрезок действительно соответствует отрезку длины $1/2^k$,
	состоящему из двоичных дробей, в которых первые $k$ цифр после запятой зафиксированы,
	а остальные выбираются произвольно.

	Заметим, что $Q$ отображает двоичный отрезок длины $1/2^k$ в двоичный отрезок длины $1/2^k$.
	Так как любой отрезок может быть представлен в виде объединения не более чем счётного числа
	двоичных отрезков, то $Q$ сохраняет меру любого отрезка.
	В силу биективности $Q$ отсюда следует, что $Q$ сохраняет меру любого борелевского множества
	(так как сигма-алгебра борелевских множеств порождается отрезками).
\end{proof}

\paragraph{Замечание.}
На самом деле соответствие точек отрезка и последовательностей из нулей и единиц не однозначно,
а почти однозначно: например, точка $1/2$ может быть с равным успехом
представлена как $x_1=0.01111...$ и как $x_2=0.10000...$.
Однако $Qx_1 = 0.01010101...$, а $Qx_2=0.100000$.
Представляется целесообразным отказаться не от записи $x_1$ (как это обычно принимается),
а от записи $x_2$, поскольку все последовательности, стабилизирующиеся на нуле,
биекция $Q$ переводит в последовательности, стабилизирующиеся на нуле.
Более того, мера множества всех последовательностей, стабилизирующихся на нуле, равна нулю,
поэтому их исключение из рассмотрения не мешает доказательству.

%NB: это скользкий момент.
%Не всякая биекция сохраняет меру,
%но тут вроде как спасает то, что мы интерпретируем ту же самую последовательность бросков двумя разными способами.

Равномерное стремление~\eqref{eq:lim_j_n_kj_measure}
буквально означает, что
\begin{equation}
	\forall(A>0)\exists(j_A\in\mathbb{N})\forall(j > j_A)\forall(k\in\mathbb{N})
	\left[
		\frac{n_{k+j}-n_k}{j}>A
	\right]
	.
\end{equation}
Так как последнее неравенство выполнено для любого $k$,
то оно верно и для нижнего предела по $k$:
\begin{equation}
	\forall(A>0)\exists(j_A\in\mathbb{N})\forall(j > j_A)
	\left[
		\liminf_{k\to\infty}\frac{n_{k+j}-n_k}{j}>A
	\right]
	.
\end{equation}
<<Свернув>> определение предела по $j$ из кванторной записи, получаем:
\begin{equation}
	\lim_{j\to\infty}\liminf_{k\to\infty}\frac{n_{k+j}-n_k}{j} = \infty
	,
\end{equation}


откуда по лемме~\ref{thm:lim_M(j)/j_dost} немедленно имеем $QW \subset \Omega\cap ac_0$.
Тогда
\begin{equation}
	0 \leq \mu(W) = \mu(QW) \leq \mu(\Omega\cap ac_0) \leq \mu(\Omega\cap ac) = 0
	,
\end{equation}
откуда $\mu(W)=0$.

\begin{hypothesis}
	Включение $QW\subset \Omega\cap ac_0$ собственное.
\end{hypothesis}


	\section{Пространство $ac_0$ и возведение в степень}
	Вопрос о покоординатном возведении в степень почти сходящейся к нулю последовательности сколь-либо системно впервые поднят в~\cite{zvol2022ac}.
Возведение в отрицательную степень может выводить из пространства ограниченных последовательностей $\ell_\infty$ вовсе;
например, очевидна следующая
\begin{lemma}
	Пусть $x\in c_0$, $\alpha< 0$ и $x_k\ne 0 $ для любого $k$.
	Тогда
	\begin{equation*}
		x^\alpha = (x_1^\alpha,x_2^\alpha,x_3^\alpha,...) \notin \ell_\infty
		.
	\end{equation*}
\end{lemma}
(Здесь и далее мы полагаем, что возведение в соответствующую степень определено однозначно и в действительных числах;
то есть, если мы пишем $x_k^\alpha$, то мы неявно предполагаем, что значение этого выражения корректно определено.)


Для $x\in ac_0\setminus c_0$ ситуация, вообще говоря, не столь однозначна.

\begin{example}
	\label{example:ac0_pow_signum_classic}
	Рассмотрим классическую почти сходящуюся к нулю последовательность
	\begin{equation}
		x = (1;-1;1;-1;1;-1;...) \in ac_0
		.
	\end{equation}
	Очевидно, что для любой целой нечётной отрицательной степени $\alpha$ имеем $x^\alpha = x \in \ell_\infty$,
	однако для любой целой чётной отрицательной степени $\alpha$ мы получаем $x^\alpha = (1;1;1;1;1;1;...) \in \ell_\infty$.
\end{example}

\begin{example}
	Рассмотрим почти сходящуюся к нулю последовательность
	\begin{equation}
		y = \left(1;-1;\frac12;-\frac12;1;-1;\frac13;-\frac13;1;-1;\frac14;-\frac14;1;-1;...\right) \in ac_0
		.
	\end{equation}
	Очевидно, что
	\begin{equation}
		y^{-1} = \left(1;-1;2;-2;1;-1;3;-3;1;-1;4;-4;1;-1;...\right)  \notin \ell_\infty
		.
	\end{equation}
\end{example}

Возведение в отрицательную степень мы более обсуждать не будем.

Итак, в недавней статье Р.Е. Зволинского~\cite{zvol2022ac} доказаны два следующих факта (теорема 3 и следствие 2 соответственно):
\begin{theorem}
	\label{thm:Zvol_pow_pos}
	Пусть $x \geqslant 0, x \in a c_0$ и $\alpha>0$, тогда $x^\alpha \in a c_0$.
\end{theorem}

\begin{theorem}
	\label{thm:Zvol_pow_composed}
	Пусть $x\in ac_0$, $x = x^+ +x^-$, $x^+\geq 0$, $x^- \leq 0$ и $x^+ \in ac_0$
	(последнее включение эквивалентно условию $x^- \in ac_0$).
	Пусть $n\in\mathbb{N}$ или $n = \frac1{2k+1}$, $k\in\mathbb{N}$.
	Тогда $x^n \in ac_0$.
\end{theorem}

Возникает закономерный вопрос о том, что происходит при возведении в степень почти сходящейся к нулю последовательности,
которую нельзя разложить в сумму знакопостоянных почти сходящихся к нулю последовательностей.
Следующая теорема показывает, что условия теоремы~\ref{thm:Zvol_pow_composed} существенны.

\begin{theorem}
	\label{thm:ac0_pow_even}
	Пусть $x\in ac_0$, $x = x^+ +x^-$, $x^+\geq 0$, $x^- \leq 0$ и $x^+ \notin ac_0$
	(последнее условие эквивалентно условию $x^- \notin ac_0$).
	Пусть $n = 2k$, $k\in\mathbb{N}$.
	Тогда $x^n \notin ac_0$.
\end{theorem}

\begin{proof}
	Рассмотрим последовательность $y = (x^+)^n \geq 0$.

	Предположим, что $y \in ac_0$.
	Тогда $x^+ = y^{1/n}$ и по теореме~\ref{thm:Zvol_pow_pos} выполнено $x^+\in ac_0$,
	что противоречит условию доказываемой теоремы.
	Значит, $y = (x^+)^n \notin ac_0$ и выполнено неравенство $p\left((x^+)^n\right) > 0$.

	Заметим теперь, что  в силу чётности $n$ выполнено $(x^-)^n \geq 0$, откуда $p\left((x^-)^n\right) \geq 0$.
	(Строго говоря, можно по аналогии с $(x^+)^n$ показать, что $(x^-)^n\notin ac_0$ и, следовательно, $p\left((x^-)^n\right) > 0$.)
	В силу построения $x^+$ и $x^-$ мы имеем $\supp x^+ \cap \supp x^- = \varnothing$,
	откуда
	\begin{equation}
		x^n = (x^+ + x^-)^n = (x^+)^n + (x^-)^n
		.
	\end{equation}
	Очевидно, что для любых ограниченных последовательностей $a\geq0$, $b\geq 0$ выполнено неравенство для верхнего функционала Сачестона $p(a+b) \geq p(a)$.
	Отсюда получаем
	\begin{equation}
		p(x^n) = p\left((x^+)^n + (x^-)^n\right) \geq p\left((x^+)^n\right) > 0
		,
	\end{equation}
	что по теореме Сачестона означает, что $x^n \notin ac_0$.
\end{proof}

Для нечётной степени условие разложение в сумму двух знакопостоянных почти сходящихся последовательностей в теореме~\ref{thm:Zvol_pow_composed} тоже существенно.

\begin{example}
	Напомним,
	%TODO: ссылка!
	что любая периодическая последовательность почти сходится к своему среднему по периоду
	(см. следствие~\ref{thm:period_ac_avg}).
	Пусть
	\begin{equation}
		x = (1;1;-2;\ 1;1;-2;\ 1;1;-2;\ ...) \in ac_0
	\end{equation}
	и пусть $\alpha = 3$.
	Тогда
	\begin{equation}
		x^+ = (1;1;0;\ 1;1;0;\ 1;1;0;\ ...) \notin ac_0, \quad x^+ \in ac_{2/3}
		,
	\end{equation}
	\begin{equation}
		x^- = (0;0;-2;\ 0;0;-2;\ 0;0;-2;\ ...) \notin ac_0, \quad x^- \in ac_{-2/3}
		,
	\end{equation}
	\begin{equation}
		x^\alpha = (1;1;-8;\ 1;1;-8;\ 1;1;-8;\ ...) \notin ac_0, \quad x^\alpha \in ac_{-2}
		.
	\end{equation}
\end{example}

Однако для нечётной степени доказать аналог теоремы~\ref{thm:ac0_pow_even} не удастся "--- это показывает пример~\ref{example:ac0_pow_signum_classic}, в котором $x^+\in ac_1$, $x^-\in ac_{-1}$.

В заключение приведём пример, в котором возведение в нечётную степень выводит не только из пространства $ac_0$,
но и из более широкого пространства $ac$.

\begin{example}
	\label{example:cube_out_of_ac0}
	%Напомним, что $\mathbb{N}_k = \{k, k+1, k+2, k+3,...\}$.
	Пусть
	\begin{equation}
		x_n = \begin{cases}
			 0, & \mbox{~если~} n < 2^{10},
			\\
			 1, & \mbox{~если~} n \ge 2^{10}, 2^k\le n < 2^k+3k \mbox{~и~}  n\neq 2^k + 3m, m\in\mathbb{N},
			\\
			-2, & \mbox{~если~} n \ge 2^{10}, 2^k\le n < 2^k+3k \mbox{~и~}  n  =  2^k + 3m, m\in\mathbb{N}.
		\end{cases}
	\end{equation}
	Таким образом,
	\begin{multline}
		x = (0,0,...,0,0, \; -2, 1, 1, \; -2, 1, 1, \; -2, 1, 1, ..., -2, 1, 1, \; 0, 0, 0, \\ ..., 0, 0, 0, ..., -2, 1, 1, \; -2, 1, 1, ... )
		.
	\end{multline}
	Легко заметить, что в силу критерия Лоренца $x\in ac_0$.
	С другой стороны,
		\begin{equation}
		(x^3)_n = \begin{cases}
			 0, & \mbox{~если~} n < 2^{10},
			\\
			 1, & \mbox{~если~} n \ge 2^{10}, 2^k\le n < 2^k+3k \mbox{~и~}  n\neq 2^k + 3m, m\in\mathbb{N},
			\\
			-8, & \mbox{~если~} n \ge 2^{10}, 2^k\le n < 2^k+3k \mbox{~и~}  n  =  2^k + 3m, m\in\mathbb{N}.
		\end{cases}
	\end{equation}
	Получаем $p(x^3) = 0$, $q(x^3) = -2$, откуда $x^3 \notin ac$.
\end{example}

Аналогично строится и пример, когда возведение в нечётную степень, наоборот, вводит в пространство $ac_0$.

\begin{example}
	%Напомним, что $\mathbb{N}_k = \{k, k+1, k+2, k+3,...\}$.
	Пусть
	\begin{equation}
		y_n = \begin{cases}
			 0, & \mbox{~если~} n < 2^{10},
			\\
			 1, & \mbox{~если~} n \ge 2^{10}, 2^k\le n < 2^k+3k \mbox{~и~}  n\neq 2^k + 3m, m\in\mathbb{N},
			\\
			-\sqrt[3]{2}, & \mbox{~если~} n \ge 2^{10}, 2^k\le n < 2^k+3k \mbox{~и~}  n  =  2^k + 3m, m\in\mathbb{N}.
		\end{cases}
	\end{equation}
	Тогда $p(y) = \frac{2-\sqrt[3]2}{3}$, $q(y) = 0$, и потому $y \notin ac_0$.
	Однако легко заметить, что $y^3 = x$ из примера~\ref{example:cube_out_of_ac0} и потому $y^3\in ac_0$.
\end{example}




Рассуждения данного параграфа подталкивают нас к изучению следующего объекта.

Пусть
\begin{equation}
	ac_0^{(2n+1)} = \{ x\in ac_0 : x^{2n+1} \in ac_0\}, \quad n\in\mathbb N
	.
\end{equation}
(Множества $ac_0^{(2n)}$ мы в рассмотрение не вводим, поскольку каждое из них совпадает с $\Iac$ в силу теоремы~\ref{thm:ac0_pow_even}.)

Очевидна следующая
\begin{lemma}
	Пусть $x \approx y$.
	В этом случае $x \in ac_0^{(2n+1)}$ тогда и только тогда, когда $y \in ac_0^{(2n+1)}$.
\end{lemma}

\begin{lemma}
	Пусть $x - y \in \Iac$.
	В этом случае $x \in ac_0^{(2n+1)}$ тогда и только тогда, когда $y \in ac_0^{(2n+1)}$.
\end{lemma}

%TODO! Через бином Ньютона.
%\begin{proof}
%	\begin{equation}
%		x
%	\end{equation}
%\end{proof}

\begin{hypothesis}
	$ac_0^{(2n+1)}$ замкнуто относительно сложения.
\end{hypothesis}

\begin{hypothesis}
	$ac_0^{(2n+1)}$ замкнуто относительно умножения.
\end{hypothesis}

\begin{hypothesis}
	$ac_0^{(2n+1)}$ замкнуто (топологически).
\end{hypothesis}

\begin{hypothesis}
	$ac_0^{(2n+1)} = ac_0^{(2n+3)}$ для любого $n$ (или же, наоборот, там есть собственное включение).
\end{hypothesis}

\begin{hypothesis}
	Если предыдущая гипотеза неверна, то встаёт вопрос об исследовании множеств
	$\bigcup_{n\in\N}ac_0^{(2n+1)}$ и 	$\bigcap_{n\in\N}ac_0^{(2n+1)}$.
\end{hypothesis}

\begin{hypothesis}
	Пусть $x\in ac_0^{(2n+1)}$.
	Тогда $x^{2n+1} \in ac_0^{(2n+1)}$.
\end{hypothesis}

\begin{hypothesis}
	Пусть $x\in ac_0^{(2n+1)}$.
	Тогда $\sigma_k x \in ac_0^{(2n+1)}$ для любого $k\in \N$
\end{hypothesis}


\chapter{Банаховы пределы, инвариантные относительно операторов}

	\section{Оператор с конечномерным ядром, для которого не существует инвариантного банахова предела}
	В работе \cite{Semenov2010invariant} изучаются условия,
при которых для оператора $H:\ell_\infty\to \ell_\infty$ существует банаховы пределы,
инвариантные относительно данного оператора, то есть такие $B\in\mathfrak{B}$,
что $B(Hx) = Bx$ для любого $x\in\ell_\infty$,
а также приводятся примеры операторов, для которых инвариантные банаховы пределы существуют:
операторы $\sigma_n$ и оператор Чезаро $C$.

Заметим, что операторы $\sigma_n$ и $C$ имеют вырожденное ядро,
однако не для любого оператора с вырожденным ядром существует инвариантный банахов предел.

\begin{example}
	Пусть для $x = (x_1, x_2, ..., x_n, ...)\in \ell_\infty$
	\begin{equation*}
		Ax = (x_1, 0, x_2, 0, x_3, 0, x_4, 0, ...).
	\end{equation*}
	Очевидно, что $\ker A = \{0\}$.
\end{example}

Пусть $B\in\mathfrak{B}$, $BA = B$.
Очевидно, что
\begin{equation*}
	\frac{n-1}{2}\leqslant \sum_{k=m+1}^{m+n} (A\one)_k \leqslant \frac{n+1}{2},
\end{equation*}
где $\one = (1, 1, 1, 1, 1, 1, ...)$.
Тогда по теореме Лоренца
\begin{equation*}
	BA\one =
	\lim_{n\to\infty} \frac{1}{n}\sum_{k=m+1}^{m+n} (A\one)_k = \frac{1}{2}
\end{equation*}
Однако по определению банахова предела
\begin{equation*}
	B\one = 1 \neq \frac{1}{2} = BA\one.
\end{equation*}
Пришли к противоречию, следовательно, банаховых пределов, инвариантных относительно $A$, не существует.


	\section{Оператор с бесконечномерным ядром, относительно которого инвариантен любой банахов предел}
	Конечномерность ядра оператора не является необходимым условием существования
банахова предела, инвариантного относительно данного оператора.

Приведём сначала несложную лемму, а потом соответствующий пример.

\paragraph{Лемма.}
%TODO: а не баян ли?
%\textit{Скорее всего, это настолько очевидно, что это никто не пишет и не доказывает.}

Пусть $Q:l_\infty \to l_\infty$.
\begin{equation}
	\forall(B\in\mathfrak{B})[QB = B]
\end{equation}
тогда и только тогда, когда
\begin{equation}\label{I-Q_to_ac_0}
	I-Q : l_\infty \to ac_0,
\end{equation}
где $I$~--- тождественный оператор на $l_\infty$.

\paragraph{Необходимость.}
\begin{equation}
	B((I-Q)x) =
	B(Ix) - B(Qx) =
	Bx - B(Qx)=
	Bx-Bx
	=
	0
	,
\end{equation}
откуда в силу произвольности выбора $B$ и $x$ вкупе с тем, что $B((I-Q)x)=0$,
имеем $(I-Q)x \in ac_0$ и, следовательно, $I-Q : \ell_\infty \to ac_0$.

\paragraph{Достаточность.}
\begin{equation}
	B(Qx) = B((I-(I-Q))x) =
	B(Ix)-B((I-Q)x) =
	B(x) - 0 = B(x).
\end{equation}


\paragraph{Пример.}

\begin{equation}
	(Qx)_k =
	\begin{cases}
		0,~\mbox{если}~ k = 2^n, n \in\mathbb{N},
		\\
		x_k~\mbox{иначе.}
	\end{cases}
\end{equation}

Очевидно, что $\dim \ker Q = \infty$.

Покажем, что $Q$ удовлетворяет условию (\ref{I-Q_to_ac_0}).
\begin{equation}
	\sum_{k=m+1}^{m+n} x_k - \sum_{k=m+1}^{m+n} (Qx)_k \leqslant (2 + \log_2 n) \|x\|,
\end{equation}
(это очевидно??)\\
откуда немедленно
\begin{equation}
	\frac{1}{n}\sum_{k=m+1}^{m+n} x_k - \frac{1}{n}\sum_{k=m+1}^{m+n} (Qx)_k \leqslant \frac{(2 + \log_2 n) \|x\|}{n} \to 0.
\end{equation}

По предыдущему утверждению отсюда следует, что относительно $Q$ инвариантен любой банахов предел.


	\section{О классах линейных операторов, для которых множества инвариантных банаховых пределов совпадают}
	\begin{lemma}
	Пусть $P,Q:\ell_\infty \to \ell_\infty$~--- линейные операторы и
	\begin{equation}\label{P-Q:ell_infty_to_ac0}
		P-Q : \ell_\infty \to ac_0
		.
	\end{equation}
	Тогда
	\begin{equation}
		\mathfrak{B}(P)=\mathfrak{B}(Q)
		.
	\end{equation}
\end{lemma}

\paragraph{Доказательство.}
Пусть $B\in \mathfrak{B}(P)$.
Тогда
\begin{equation}
	B(Qx) = B((Q-P+P)x) =
	B((Q-P)x)+B(Px) =
	0 + B(Px) =
	Bx
	.
\end{equation}
Значит, $\mathfrak{B}(P) \subset \mathfrak{B}(Q)$.
В силу симметричности утверждения леммы получаем $\mathfrak{B}(Q) \subset \mathfrak{B}(P)$,
откуда и следует требуемое утверждение.

Только что доказанная лемма означает, что множество всех линейных операторов,
действующих из $\ell_\infty$ в $\ell_\infty$, можно разбить на классы по отношению эквивалентности,
задаваемому условием \eqref{P-Q:ell_infty_to_ac0},
и тогда для операторов из одного класса множества инвариантных банаховых пределов будут совпадать.

Обратное, однако, неверно.

\begin{example}
	Пусть
	\begin{equation}
		P(x_1,x_2,x_3) = (x_1, 0, x_3, 0, x_5, 0, ...)
	\end{equation}
	и
	\begin{equation}
		Q(x_1,x_2,x_3) = (0, -x_2, 0, -x_4, 0, -x_6, 0, ...)
		.
	\end{equation}
	Легко видеть, что для любого банахова предела $B\in\mathfrak{B}$ выполнено
	\begin{equation}
		B(P\mathbb{I}) = \frac{1}{2}
	\end{equation}
	и
	\begin{equation}
		B(Q\mathbb{I}) = -\frac{1}{2}
		,
	\end{equation}
	откуда
	\begin{equation}
		\mathfrak{B}(Q) = \mathfrak{B}(P) = \varnothing
		.
	\end{equation}
	Однако разность $P-Q$ есть не что иное, как тождественный оператор $I$,
	который переводит пространство $\ell_\infty$ само в себя,
	а не в пространсво $ac_0$.
\end{example}


\begin{hypothesis}
	Существуют два таких линейных оператора $P, Q : \ell_\infty \to \ell_\infty$,
	что $\mathfrak{B}(P) = \mathfrak{B}(Q) \neq \varnothing$,
	но $(P-Q)(\ell_\infty) \setminus ac \neq \varnothing$.
\end{hypothesis}



	\section{Мощность множества линейных операторов, относительно которых инвариантен банахов предел}
	Очевидна следующая
\begin{lemma}
	Пусть $A:\ell_\infty\to\ell_\infty$,
	\begin{equation}
		(Ax)_k=\begin{cases}
			x_n, & \mbox{если~} k=2^n,
			\\
			0    & \mbox{иначе.}
		\end{cases}
	\end{equation}
	Тогда $A:\ell_\infty\to ac_0$.
\end{lemma}

\begin{corollary}
	$|L(\ell_\infty,\ell_\infty)|=|L(\ell_\infty,ac_0)|$.
\end{corollary}

\begin{proof}
	Пусть $Q\in L(\ell_\infty,\ell_\infty)$.
	Тогда $AQ\in L(\ell_\infty,ac_0)$.
	Из того, что отображение $Q\mapsto AQ$~--- инъекция,
	следует, что $|L(\ell_\infty,ac_0)|\geq|L(\ell_\infty,\ell_\infty)|$.
	Из того, что $ L(\ell_\infty,ac_0) \subsetneq L(\ell_\infty,\ell_\infty)$,
	следует, что $|L(\ell_\infty,ac_0)|\leq|L(\ell_\infty,\ell_\infty)|$.
	Значит, $|L(\ell_\infty,ac_0)|=|L(\ell_\infty,\ell_\infty)|$,
	что и требовалось доказать.
\end{proof}

\begin{corollary}
	Пусть $P:\ell_\infty \to \ell_\infty$ и $\mathfrak{B}(P)\neq\varnothing$.
	Пусть $G = \{Q: \mathfrak{B}(P)= \mathfrak{B}(Q)\}$.
	Тогда $|G| = |L(\ell_\infty,\ell_\infty)|$.
\end{corollary}

\begin{proof}
	Действительно,
	$(P+\ell_\infty,ac_0) \subset \mathfrak{B}(P)$ по лемме~\ref{thm:linear_op_equiv_ac0}.
	Значит,
	\begin{equation}
		|\mathfrak{B}(P)| \geq |L(\ell_\infty,ac_0)|=|L(\ell_\infty,\ell_\infty)|
		.
	\end{equation}
	Из того, что $\mathfrak{B}(P)\subsetneq|L(\ell_\infty,\ell_\infty)|$,
	следует, что $|\mathfrak{B}(P)|\leq|L(\ell_\infty,\ell_\infty)|$.
	Значит, $|\mathfrak{B}(P)|=|L(\ell_\infty,\ell_\infty)|$,
	что и требовалось доказать.
\end{proof}





	\section{Банаховы пределы, инвариантные относительно операторов $\sigma_{1/n}$}
	Введём, вслед за~\cite[с. 131, утверждение 2.b.2]{lindenstrauss1979classical},
на $\ell_\infty$ оператор
\begin{equation}
	\sigma_{1/n} x = n^{-1}
	\left(
		\sum_{i=1}^{n} x_i,
		\sum_{i=n+1}^{2n} x_i,
		\sum_{i=2n+1}^{3n} x_i,
		...
	\right).
\end{equation}

Из критерия Лоренца немедленно следует
\begin{lemma}
	$\sigma_n \sigma_{1/n} - I : \ell_\infty \to ac_0$.
\end{lemma}

\begin{theorem}
	\label{thm:B_sigma_n_eq_B_sigma_1_n}
	Для любого $n\in\N$ выполнено $\mathfrak{B}(\sigma_n) = \mathfrak{B}(\sigma_{1/n})$.
\end{theorem}

\begin{proof}
	Пусть сначала $B \in \mathfrak{B}(\sigma_{1/n})$.
	Тогда
	\begin{equation}
		B(x) = B(Ix) = B(\sigma_{1/n} \sigma_n x) = B(\sigma_n x)
		.
	\end{equation}
	Пусть теперь $B \in \mathfrak{B}(\sigma_n)$.
	Тогда
	\begin{equation}
		B(x) = B(Ix) = B((I+(\sigma_n \sigma_{1/n} - I)) x) = B(\sigma_n \sigma_{1/n} x) = B(\sigma_{1/n} x)
		.
	\end{equation}
\end{proof}

\begin{remark}
	Мы видим, что множество банаховых пределов, инвариантных относительно суперпозиции операторов,
	может быть шире, чем объединение множеств банаховых пределов,
	инвариантных относительно каждого из операторов:
	\begin{equation}
		\mathfrak{B}(\sigma_n) \cup \mathfrak{B}(\sigma_{1/n}) = \mathfrak{B}(\sigma_n) \subsetneq \mathfrak{B}(\sigma_{1/n}\sigma_n) = \mathfrak{B}(I)
		.
	\end{equation}
\end{remark}


	\section{Операторы $\tilde\sigma_k$}
	Введём в рассмотрение семейство операторов
$\tilde\sigma_k : \ell_\infty \to \ell_\infty$, $k>0$,
определяемых следующим образом:
\begin{equation}
	(\tilde\sigma_k x)_n = x_{\left\lceil \dfrac{n}{k}\right\rceil}
	.
\end{equation}

\begin{example}
	\label{example:sigma_3_2}
	\begin{equation}
		\tilde\sigma_{3/2} x =
		(x_1, x_2, x_2, \; x_3, x_4, x_4, \; x_5, ...)
		.
	\end{equation}
\end{example}

\begin{example}
	\label{example:sigma_2_3}
	\begin{equation}
		\tilde\sigma_{2/3} x =
		(x_2, x_3, \; x_5, x_6, \; x_8, ...)
		.
	\end{equation}
\end{example}


Заметим, что для $k\in \N$ выполнено равенство $\sigma_k = \tilde\sigma_k$
(однако использовать обозначение без тильды мы не можем, чтобы избежать путаницы с операторами усреднения $\sigma_{1/k}$).
Однако соотношение $\tilde\sigma_k \tilde\sigma_m = \tilde\sigma_{km}$ для нецелых $k$, вообще говоря, не выполняется.
Чтобы это увидеть, достаточно рассмотреть суперпозиции $\tilde\sigma_{3/2} \tilde\sigma_{2/3}$ и
$\tilde\sigma_{2/3} \tilde\sigma_{3/2}$ (см. примеры~\ref{example:sigma_3_2} и~\ref{example:sigma_2_3} выше).

Таким образом, операторы $\tilde\sigma_k$ обобщают операторы $\sigma_k$,
и возникает закономерный вопрос о соответствующих инвариантных банаховых пределах.

\begin{theorem}
	Пусть $k>0$, $k\in \mathbb{Q} \setminus \N$.
	Тогда $\B(\tilde\sigma_k)=\varnothing$.
\end{theorem}

\begin{proof}
	Пусть $k$ представимо в виде несократимой дроби $k=p/q$, $p\in \N$, $q\in\N_2$.
	Рассмотрим последовательность $x\in \ell_\infty$, заданную соотношением
	\begin{equation}
		x_k = \begin{cases}
			1, \mbox{~если~} m=qn+1, n\in\N,
			\\
			0  \mbox{иначе.}
		\end{cases}
	\end{equation}
	Последовательность $x$ периодична, и её период равен $q$.

	Пусть $B\in \B$, тогда $Bx=\dfrac1q$ , поскольку любой банахов предел на периодической последовательности принимает значение, равное среднему арифметическому по периоду.
	%TODO: ссылка!
	Заметим, что $\tilde\sigma_{p/q}x \in \Omega$.
	Более того, последовательность $\tilde\sigma_{p/q}x \in \Omega$ также периодична и имеет период, равный $p$.

	Действительно,
	\begin{multline}
		(\tilde\sigma_{p/q}x )_{m+p} =
		x_{\left\lceil \dfrac{m+p}{p/q}\right\rceil} =
		x_{\left\lceil \dfrac{qm+qp}{p}\right\rceil} =
		x_{\left\lceil \dfrac{qm}{p}+q\right\rceil} =
		\\=
		x_{q+\left\lceil \dfrac{qm}{p}\right\rceil} =
		x_{\left\lceil \dfrac{qm}{p}\right\rceil} =
		x_{\left\lceil \dfrac{m}{p/q}\right\rceil} =
		(\tilde\sigma_{p/q}x \in)_{m}
		.
	\end{multline}

\end{proof}

\begin{theorem}
	Пусть $k>0$, $k\in \mathbb{Q} \setminus \N$.
	Тогда $\tilde\sigma_k ac_0 \not \subset ac_0$.
\end{theorem}




\begin{theorem}
	Пространство $ac_0$ не замкнуто относительно любого оператора $\tilde\sigma_k$, $k\in\Q^+\setminus \N$.
\end{theorem}

\begin{proof}
	Пусть $k=p/q$ есть несократимая дробь.
	Определим последовательность $x\in\ell_\infty$ соотношением:
	\begin{equation}
		x_m = \begin{cases}
		\end{cases}
	\end{equation}
	TODO: набрать доказательство.
\end{proof}

\begin{hypothesis}
	\label{hyp:tilde_sigma_k_x_notin_ac0}
	Для любого $k \in \R^+\setminus \N$ ($k \in \Q^+\setminus \N$) существует такой $x\in ac_0$, что $\tilde\sigma_{k} x \notin ac$.
\end{hypothesis}

В пользу гипотезы~\ref{hyp:tilde_sigma_k_x_notin_ac0} говорит следующий
\begin{example}
	Определим последовательность $x\in\ell_\infty$ следующим соотношением:
	\begin{equation}
		x_k = \begin{cases}
			(-1)^k, &~\mbox{если}~ 2^{2n} < k \leq 2^{2n+1}, ~ n \in \N_0
			\\
			(-1)^{k+1} &~\mbox{иначе}
			.
		\end{cases}
	\end{equation}
	Тогда $x\in ac_0$, но
	\begin{equation}
		(\tilde\sigma_{1/2} x)_k = \begin{cases}
			-1, &~\mbox{если}~ 2^{2n} < k \leq 2^{2n+1}, ~ n \in \N_0
			\\
			1 &~\mbox{иначе}
			.
		\end{cases}
	\end{equation}
	Имеем $p(\tilde\sigma_{1/2} x) = 1$, $q(\tilde\sigma_{1/2} x) = -1$,
	откуда $\tilde\sigma_{1/2} x \notin ac$.
\end{example}



\begin{hypothesis}
	Для любых натуральных $k > m$ существуют такие $r, s\in\N$, что  $\tilde\sigma_{m/k} T^r \tilde\sigma_{k/m} = T^s$.
\end{hypothesis}


\begin{hypothesis}
	Для любого $k \in \N$ выполнено $(\tilde\sigma_{\sqrt{k}})^2 - \sigma_k : \ell_\infty \to ac_0$.
\end{hypothesis}



\begin{hypothesis}
	Пусть $x\in ac_0$.
	Для того, чтобы $x\in \mathcal{I}(ac_0)$, необходимо и достаточно,
	чтобы $\tilde\sigma_{k} x \in ac_0$ для любого $k\in \Q^+$.
\end{hypothesis}


\begin{hypothesis}
	Пусть $x\in ac_0$.
	Для того, чтобы $x\in \mathcal{I}(ac_0)$, необходимо и достаточно,
	чтобы $\tilde\sigma_{k} x \in ac_0$ для любого $k\in \R^+$.
\end{hypothesis}


	\section{Мера прообраза числа при инвариантности банахова предела относительно оператора Чезаро}
	При изучении банаховых пределов и меры на множестве $\Omega$
возникает закономерный вопрос о мере множества
\begin{equation}
	\{ x \in \Omega : Bx = \beta \}
\end{equation}
для заданного банахова предела $B$ и числа $\beta\in[0;1]$.
(Хаусдорфова размерность этого множества равна 1 в силу леммы~\ref{lem:Hausdorf_measure}.)

\begin{theorem}
	Пусть $B \in \mathfrak{B}(C)$.
	Тогда
	\begin{equation}
		\mes \{ x \in \Omega : Bx = 1/2 \} = 1
		.
	\end{equation}
\end{theorem}

\begin{proof}
	Так как $B \in \mathfrak{B}(C)$, то
	\begin{equation}
		\{ x \in \Omega : Bx = 1/2 \}
		=
		\{ x \in \Omega : BCx = 1/2 \}
		\supset
		\{ x \in \Omega : \lim_{n\to\infty} (Cx)_n = 1/2 \}
		.
	\end{equation}
	Вместе с тем,
	\begin{equation}
		\mes \left\{ x \in \Omega : \lim_{n\to\infty} (Cx)_n  = 1/2 \right\} = 1
	\end{equation}
	(это следует из закона больших чисел~\cite{connor1990almost}).
\end{proof}



	\section{Классификация линейных операторов в свете банаховых пределов}
	При изучении инвариантности банаховых пределов относительно различных непрерывных линейных операторов
возникает закономерный вопрос о классификации этих операторов.

\begin{definition}
	Будем говорить, что оператор $H : \ell_\infty \to \ell_\infty$ \emph{эберлейнов},
	если $\B (H) \ne \varnothing$.
\end{definition}

Выбор именования этого класса операторов обусловлен тем, что именно Эберлейн в работе~\cite{Eberlein}
впервые сколь-либо системно изучил инвариантные банаховы пределы
(хотя отдельные шаги в этом направлении были сделаны ещё в работе~\cite{agnew1938extensions}).

В работе~\cite{alekhno2018invariant} вводится следующее

\begin{definition}
	Оператор $H : \ell_\infty \to \ell_\infty$ называется \emph{B-регулярным},
	если $H*\B \subseteq \B$.
	(Или, что то же самое, $HB\in\B$ для любого $B\in\B$.)
\end{definition}

В той же работе с помощью теоремы о неподвижной точке доказывается,
что любой B-регулярный оператор "--- эберлейнов, приводится следующее необходимое и достаточное условия В-регулярности.

\begin{theorem}
	Оператор $H:\ell_\infty \to \ell_\infty$ является B-регулярным тогда и только тогда, когда:
	\\(i) $H\mathbbm{1} \in ac_1$;
	\\(ii) $q(Hx)\geq 0$ для любого $x\geq 0$;
	\\(iii) $H ac_0 \subseteq ac_0$.
\end{theorem}
Легко заметить, что эти условия являются более слабыми, чем достаточные условия эберлейновости.
%TODO: ссылка


\begin{hypothesis}
	Существует эберлейнов оператор, не являющийся B-регулярным.
\end{hypothesis}

Эту классификацию можно надстроить в обе стороны (по включению) следующими двумя определениями:

\begin{definition}
	Будем называть оператор $A:\ell_\infty \to \ell_\infty$ \emph{дружелюбным (amiable)}, если $BA\in\mathfrak B$ для некоторого $B\in\mathfrak B$.
\end{definition}

\begin{definition}
	Будем называть оператор $A:\ell_\infty \to \ell_\infty$ \emph{существенно дружелюбным}, если $BA\in\mathfrak B$ для любого $B\in\mathfrak B$ и $BA\ne B$ для некоторого $B\in\mathfrak B$.
\end{definition}

Эти четыре класса операторов (существенно дружелюбные, В-регулярные, эберлейновы, дружелюбные "--- в порядке включения)
получены последовательным ослаблением условий, естественных для <<достаточно хороших>> операторов:
$\sigma_k$, $C$
и образуют иерархию по включению.
Возникает закономерный вопрос о совпадении классов.
Ясно, что оператор сдвига $T$ является В-регулярным, но не является существенно дружелюбным.

\begin{hypothesis}
	Существует дружелюбный оператор, не являющийся эберлейновым.
\end{hypothesis}


Далее возникает вопрос о свойствах классов операторов.

Из определения В-регулярности незамедлительно следует

\begin{lemma}
	Множество В-регулярных операторов замкнуто относительно суперпозиции.
\end{lemma}


	\section{Пример дружелюбного не эберлейнового оператора}
	\begin{theorem}
	\label{thm:amiable_but_not_Eberlein_exists}
	Существует дружелюбный оператор, не являющийся эберлейновым.
\end{theorem}

\begin{proof}
	Пусть $B_1, B_2 \in \ext \B$.
	Положим
	\begin{equation}
		\label{eq:am_not_eber_def}
		B_3 = B_1 + 2(B_2-B_1) = 2B_2-B_1,
	\end{equation}
	тогда $B_3 \notin \B$.
	Действительно, если $B_3 \in \B$, то из~\eqref{eq:am_not_eber_def} следует, что
	\begin{equation}
		B_2 = \frac{B_1 + B_3}2 \in \B \setminus \ext \B
		.
	\end{equation}
	Введём в рассмотрение оператор $H:\ell_\infty\to\ell_\infty$, определённый равенством
	\begin{equation}
		2Hx = (x_1 + B_3x, x_2 + B_3x, ...) = x + (B_3 x) \cdot \mathbbm 1
		.
	\end{equation}
	Убедимся, что оператор $H$ дружелюбен.
	Действительно, для произвольного $x\in\ell_\infty$ имеем
	\begin{equation}
		2 B_1 H x = B_1 x + B_1 ((B_3 x) \cdot\mathbbm 1) = B_1 x + B_3 x =
		B_1 x + 2 B_2 x - B_1 x = 2B_2 x
		,
	\end{equation}
	откуда $B_1 H = B_2$.

	Убедимся теперь, что оператор $H$ не эберлейнов.
	Пусть $B = BH$ для некоторого $B\in\B$.
	Тогда для всех $x\in\ell_\infty$ имеем
	\begin{equation}
		2Bx = B (x + (B_3 x) \cdot \mathbbm 1)
		,
	\end{equation}
	т.е.
	\begin{equation}
		Bx =  B((B_3 x) \cdot \mathbbm 1)
		,
	\end{equation}
	откуда незамедлительно следует, что $Bx = B_3x$ и в силу произвольности выбора $x$ имеем $B=B_3$.
	Но ранее мы уже показали, что $B_3\notin \B$.
	Полученное противоречие завершает доказательство.
\end{proof}

\begin{hypothesis}
	$B_2H\notin \B$.
\end{hypothesis}

\begin{hypothesis}
	Существуют такие дружелюбный оператор $H:\ell_\infty\to\ell_\infty$ и банахов предел $B\in \B$,
	что $BH \in \B$, но $B_1 H \notin \B$ для любого $B\in \B\setminus\{B\}$.
	Также интересно наложение дополнительных свойств на $B$ и $BH$, например, принадлежности к $\ext\B$, $\B(C)$ и т.д.
\end{hypothesis}


	\section{Об одном В-регулярном операторе}
	В данном параграфе мы будем изучать оператор $A:\ell_\infty\to\ell_\infty$,
опеределённый равенством:
\begin{multline}
	\label{eq:oper_A_throws_out_2power_blocks}
	Ax = (x_1, x_2, \not x_3, \not x_4, x_5, x_6, x_7, x_8, \not x_9, ..., \not x_{16}, x_{17}, x_{18}, ..., x_{31}, x_{32}, \not x_{33}, \not x_{34}, ..., \not x_{64},
	\\
	x_{65}, x_{66}, ..., x_{128}, \not x_{129}, ...)=
	\\=
	(x_1, x_2, \ x_5, x_6, x_7, x_8, \ x_{17}, x_{18}, ..., x_{31}, x_{32}, \ x_{65}, x_{66}, ..., x_{128}, \ x_{257},
	\\
	..., \ x_{2^{2n} +1}, x_{2^{2n} +2},  x_{2^{2n+1}}, \ \ x_{2^{2(n+1)} +1},  x_{2^{2(n+1)} +2},  x_{2^{2(n+1)+1}}, ...)
\end{multline}

В дальнейшем этот оператор окажется очень полезен при построении других операторов и банаховых пределов.

Очевидна следующая

\begin{lemma}
	\label{lem:supp_I_ac0}
	Пусть $x\in \mathcal I(ac_0)$, $Y \subset \supp x$ и $y = x \cdot \chi_Y$.
	Тогда $y \in \mathcal I(ac_0)$.
\end{lemma}

%TODO2: если совсем прижмёт - аккуратно, со вкусом доказать!

\begin{lemma}
	\label{lem:AT-TA}
	$AT - TA: \ell_\infty \to \mathcal{I}(ac_0)$.
\end{lemma}

\begin{proof}
	Пусть $\varphi(n) = 1 + 2^0 + 2^2 + 2^4 + ... + 2^{2n}$
	и пусть $y = \chi_{k\in\N : k = \varphi(n), n\in \N_0}$.
	Тогда ясно, что $y \in ac_0$ (в силу быстрого роста функции $\varphi(n)$)
	%TODO: аккуратное доказательство с помощью теоремы из матзаметок?
	и, более того, $y\in \mathcal{I}(ac_0)$.
	Заметим, что
	\begin{multline}
		ATx =
		(x_2, x_3, \ x_6, x_7, x_8, x_9, \ x_{18}, x_{19}, ..., x_{32}, x_{33}, \ x_{66}, x_{67}, ..., x_{129}, \ x_{258},
		\\
		..., \ x_{2^{2n} +2}, x_{2^{2n} +3},  x_{ 2^{2n+1} +1}, \ \ x_{2^{2(n+1)} +2},  x_{2^{2(n+1)} +3},  x_{ 2^{2(n+1)+1} +1}, ...)
		.
	\end{multline}
	С другой стороны,
	\begin{multline}
		TAx =
		(x_2, \ x_5, x_6, x_7, x_8, \ x_{17}, x_{18}, ..., x_{31}, x_{32}, \ x_{65}, x_{66}, ..., x_{128}, \ x_{257},
		\\
		..., \ x_{2^{2n} +1}, x_{2^{2n} +2},  x_{2^{2n+1}}, \ \ x_{2^{2(n+1)} +1},  x_{2^{2(n+1)} +2},  x_{2^{2(n+1)+1}}, ...)
		.
	\end{multline}
	Таким образом, для любого $x\in\ell_\infty$
	\begin{equation}
		(ATx-TAx)_k=\begin{cases}
			x_{2^{2n-1}} -x_{2^{2n} + 1}, ~~&\mbox{если}~ k = \varphi(n), ~n\in \N_0,
			\\
			0 ~~&\mbox{иначе}.
		\end{cases}
	\end{equation}
	Ввиду включения $\supp (AT-TA)x \subset \supp y$ имеем $(AT-TA)x \in \mathcal I (ac_0)$ по лемме~\ref{lem:supp_I_ac0}.
\end{proof}

\begin{theorem}
	\label{thm:A_block_thrower_is_B-regular}
	Оператор $A$ является В-регулярным.
\end{theorem}

\begin{proof}
	Пусть $B_1\in \B$.
	Непосредственно проверим определение банахова предела для функционала $B = B_1 A$.

	а) Пусть $x\geq 0$.
	Тогда $Bx = B_1 Ax \geq 0$ в силу того, что $A\geq 0$.

	б) Очевидно, что $A\mathbbm 1 = \mathbbm 1$, поэтому $B\mathbbm 1 = B_1 A \mathbbm 1 = B_1 \mathbbm 1 = 1$.

	в) В силу того, что оператор $A$ является оператором взятия подпоследовательности,
	выполнено включение $Ac_0 \subset c_0$.
	Значит, $Bc_0 = B_1 A c_0 \subset B_1 c_0 \subset c_0$.

	г) В силу леммы~\ref{lem:AT-TA} для любого $x\in \ell_\infty$ выполнено $(AT-TA)x \in \mathcal{I}(ac_0) \subset ac_0$.
	Тогда
	\begin{equation}
		(BT-B)x = BTx - Bx = B_1 ATx - B_1 A x = B_1 ATx - B_1 T A x = B_1 (AT -  T A) x = 0
		.
	\end{equation}
	В силу произвольности выбора $x\in \ell_\infty$ последнее равенство и означает, что $BT=B$.

	Таким образом, $B = B_1 A$ действительно является банаховым пределом.
	В силу произвольности выбора $B_1 \in \B$ мы получаем,
	что оператор $A$ является В-регулярным по определению.
\end{proof}

\begin{lemma}
	Для любого банахова предела $B_1$ выполено $B_1 A \in \mathfrak B \setminus \B (\sigma_{2^{2n-1}})$, $n\in \mathbb N$.
\end{lemma}

\begin{proof}
	Напомним (см. определение~\ref{def:Damerau_Levenshein_distance}), что знак приближенного равенства $ u \approx w$ означает,
	что конечно расстояние Дамерау--Левенштейна между последовательностями
	$u$ и $w$,
	т.е. что $w$ можно получить из $u$ конечным числом вставок, удалений и замен элементов.
	Тогда $Bu = Bw$ для любого $B\in\B$ и $u \approx w$.

	Положим $x = \chi_{\cup_{m=0}^{\infty}\left[2^{2 m}+1, 2^{2 m+1}\right]}$.
	Тогда $\sigma_{2^{2n}} x \approx x$ и $\sigma_{2^{2n-1}} x \approx \mathbbm 1 - x$.
	Заметим, что $Ax \approx \mathbbm 1$.

	Предположим противное утверждению теоремы.
	Пусть $B \in \B(A) \cap \B (\sigma_{2^{2n-1}})$, $n\in \N$.
	Тогда
	\begin{multline}
		1 = B\mathbbm1 = BAx = (BA)x = Bx =
		(B\sigma_{2^{2n-1}}) x = B (\sigma_{2^{2n-1}} x) =
		\\=
		B(\mathbbm1 -x) = 1 - Bx = 1 - (BA)x = 1-1 = 0
		.
	\end{multline}
	Полученное противоречие завершает доказательство.
\end{proof}

\begin{remark}
	Например, для $B_1 \in \mathfrak B (\sigma_2)$ имеем $B_1 \ne B_1 A$, что показывает,
	что оператор $A$ является существенно дружелюбным.
\end{remark}

\begin{remark}
	Оператор $A$ переводит банаховы пределы <<достаточно далеко>>. Действительно, пусть
	\begin{equation}
		y \approx 2\chi_{\cup_{m=0}^{\infty}\left[2^{2 m}+1, 2^{2 m+1}\right]} - \mathbbm 1
		,
	\end{equation}
	тогда
	\begin{equation}
		\sigma_2 y \approx 2\chi_{\cup_{m=0}^{\infty}\left[2^{2 m+1}+1, 2^{2 m+2}\right]} - \mathbbm 1
		.
	\end{equation}
	Пусть $B\in\B$, тогда $BA\in\B$ и $BA\sigma_2\in\B$ в силу теоремы~\ref{thm:A_block_thrower_is_B-regular}.
	Но
	\begin{multline}
		\|BA-BA\sigma_2\| \geq \dfrac{\| BAy-BA\sigma_2y \|}{\|y\|} = \| BAy-BA\sigma_2y \| =
		\\=
		\left\| BA\left(2\chi_{\cup_{m=0}^{\infty}\left[2^{2 m}+1, 2^{2 m+1}\right]} - \mathbbm 1\right)-
		BA\left(2\chi_{\cup_{m=0}^{\infty}\left[2^{2 m+1}+1, 2^{2 m+2}\right]} - \mathbbm 1\right) \right\|=
		\\=
		\left\| 2BA\chi_{\cup_{m=0}^{\infty}\left[2^{2 m}+1, 2^{2 m+1}\right]}-
		2BA\chi_{\cup_{m=0}^{\infty}\left[2^{2 m+1}+1, 2^{2 m+2}\right]}  \right\|=
		\\=
		2\left\| BA\chi_{\cup_{m=0}^{\infty}\left[2^{2 m}+1, 2^{2 m+1}\right]}-
		BA\chi_{\cup_{m=0}^{\infty}\left[2^{2 m+1}+1, 2^{2 m+2}\right]}  \right\|=
		\\=
		2\left\| B \mathbbm 1- B (0\cdot \mathbbm 1)  \right\|=
		2\left\| 1- 0  \right\|= 2 = \diam \B
		.
		\end{multline}
\end{remark}

\begin{remark}
	Оператор $A$ переводит банаховы пределы <<достаточно далеко>> от $\B(\sigma_{2^{2n-1}})$.
	%TODO: \sigma_{2^{2n+1}}
	Пусть снова
	\begin{equation}
		y \approx 2\chi_{\cup_{m=0}^{\infty}\left[2^{2 m}+1, 2^{2 m+1}\right]} - \mathbbm 1
		.
	\end{equation}
	Рассмотрим $B_1 \in \B(\sigma_{2^{2n-1}})$, $n\in\N$.
	Тогда  $y \approx - \sigma_{2^{2n-1}} y$,
	откуда $B_1 y = B_1 \sigma_{2^{2n-1}} y = 0$
	Однако для любого $B_2 \in \B$ имеем $B_2 A y = 1$,
	т.е.
	\begin{equation}
		\|B_1 - B_2 A\| \geq \dfrac{\|B_1y - B_2 Ay\|}{\|y\|} = \|B_1y - B_2 Ay\| = 1
		.
	\end{equation}
\end{remark}

\begin{hypothesis}
	Пусть $B_1 \in \B(\sigma_{2^{2n-1}})$, $n\in\N$, и $B_2\in \B$.
	Тогда $\|B_1 - B_2 A\| = 2 = \diam \B$.
\end{hypothesis}

\begin{hypothesis}
	Множество $\B(A) \cap \B (\sigma_{2^{2n}})$ непусто для любого $n\in \mathbb N$.
\end{hypothesis}

\begin{hypothesis}
	Если $B\in\ext\B$, то $BA \in \ext \B$.
\end{hypothesis}

Более того, реалистичной видится даже существенно более сильная

\begin{hypothesis}
	Для любого $B\in\B$ выполнено $BA \in \ext \B$.
\end{hypothesis}


	\section{Пример эберлейнового не В-регулярного оператора}
	\begin{theorem}
	Существует эберлейнов оператор, не являющийся В-регулярным.
\end{theorem}

\begin{proof}
	Пусть оператор $A$ определён формулой~\eqref{eq:oper_A_throws_out_2power_blocks}.
	Пусть оператор $E:\ell_\infty\to\ell_\infty$ определён формулой
	\begin{equation}
		Ex = x \cdot \chi_{\cup_{m=0}^{\infty}\left[2^{2 m}+1, 2^{2 m+1}\right] \cup\{1\}}
		.
	\end{equation}
	Тогда, очевидно, $AE=A$.
	Поскольку $A$ есть В-регулярный оператор, то он эберлейнов,
	т.е. множество $\B(A)$ непусто.

	Пусть $B\in\B(A)$. Тогда
	\begin{equation}
		B = BA = B(AE) = (BA)E = BE
		,
	\end{equation}
	то есть $B\in\B(E)$.

	Покажем теперь, что оператор $E$ не является В-регулярным.
	Действительно, $q(E\mathbbm 1) =0$, т.е. $ E\mathbbm 1 \notin ac_1$,
	и не выполнено условие (i) критерия В-регулярности (теорема~\ref{thm:crit_B_regularity}).
\end{proof}

\begin{remark}
	В конструкции оператора $E$ вместо блоков из нулей можно вставлять любые другие блоки "---
	они всё равно будут поглощены оператором $A$.
	Эти блоки могут иметь различный знак, например, содержать значительное количество
	элементов вида $-kx_1$, $kx_2$ и т.д.
	Таким образом, можно сконструировать эберлейновы операторы, очень и очень далёкие
	от достаточных условий эберлейновости,
	%TODO: ссылка
	что показывает их значительную избыточность.
\end{remark}


	\section{Обратная задача об инвариантности}
	Ранее мы, как правило, ставили задачу, которую можно назвать прямой задачей об инвариантности:
дан некоторый достаточно хороший оператор $H$, и требуется выяснить, непусто ли множество $\B(H)$
банаховых пределов, инвариантных относительно этого оператора.
(И~если это множество непусто, то исследовать его.)

Возникает закономерный вопрос: для любого ли банахова предела существует нетривиальный оператор,
относительно которого этот банахов предел инвариантен?
Или же такие операторы существуют только для <<достаточно хороших>>
банаховых пределов "--- например, для банаховых пределов, инвариантных относительно какого-нибудь из операторов растяжения?
Существуют ли нетривиальные операторы, инвариантные относительно хотя бы одного $B\in\ext \B$?
(Здесь под тривиальным оператором понимается такой оператор $H:\ell_\infty \to \ell_\infty$, что $H-I:\ell_\infty \to ac_0$.)

Итак, в этом параграфе мы обсудим обратную задачу об инвариантности.
Она имеет неожиданно простое решение.

\begin{theorem}
	Для каждого $B\in \B$ существует такой оператор $G_B:\ell_\infty \to \ell_\infty$,
	что $\B(G_B) = \{B\}$.
\end{theorem}

\begin{proof}
	Определим оператор $G_B$ равенством
	\begin{equation}
		G_B x = (Bx, Bx, Bx, ...) = (Bx)\cdot\mathbbm 1
		.
	\end{equation}
	Легко видеть, что
	\begin{equation}
		B G_B x = B(Bx, Bx, Bx, ...) = B((Bx)\cdot\mathbbm 1) = (Bx)\cdot B(\mathbbm 1)= (Bx)\cdot 1 = Bx
	\end{equation}
	для любого $x\in\ell_\infty$.
	Значит, $B\in \B (G_B)$ и $\B (G_B)$ непусто.

	Рассмотрим теперь $B_1 \in \B \setminus\{B\}$.
	Последнее означает, что на некоторой последовательности $y\in\ell_\infty$ выполнено $B_1y \ne By$.
	Тогда
	\begin{equation}
		B_1 G_B y = B_1(By, By, By, ...) = B_1((By)\cdot\mathbbm 1) = (By)\cdot B_1(\mathbbm 1)= (By)\cdot 1 = By \ne B_1y
		.
	\end{equation}
	Таким образом, $B_1 G_B \ne B_1$, и $B_1 \notin \B (G_B)$.
	Это и означает, что $\B(G_B) = \{B\}$.
\end{proof}

\begin{definition}
	Оператор $G_B$, построенный таким образом, будем называть оператором,
	\emph{порождённым} банаховым пределом $B$.
\end{definition}

\begin{remark}
	Легко заметить, что любой порождённый оператор удовлетворяет достаточным условиям инвариантности.
	%TODO: ссылка на теорему
\end{remark}

\begin{remark}
	Более того, порождённый оператор $G_B$ является в некотором смысле крайним примером В-регулярного оператора,
	поскольку $G_B^* \B = \{B\}$.
\end{remark}

\begin{lemma}
	Пусть $0 \leq \lambda \leq 1$, $B_1, B_2 \in \B$.
	Тогда
	\begin{equation}
		G_{\lambda B_1+(1-\lambda) B_2} =\lambda G_{B_1} + (1-\lambda)G_{B_2}
		.
	\end{equation}
\end{lemma}


\begin{proof}
	В силу выпуклости (на единичной сфере в $\ell_\infty^*$) множества $\B$ оператор $G_{\lambda B_1+(1-\lambda) B_2}$
	действительно определён корректно.
	\begin{multline}
		G_{\lambda B_1+(1-\lambda) B_2} =
		\left((\lambda B_1+(1-\lambda) B_2) x\right)\mathbbm 1 =
		(\lambda B_1 x) \mathbbm 1 +((1-\lambda) B_2 x)\mathbbm 1 =
		\\=
		\lambda (B_1 x) \mathbbm 1 +(1-\lambda) (B_2 x)\mathbbm 1 =
		\lambda G_{B_1} x +(1-\lambda) G_{B_2 x}
		.
	\end{multline}
\end{proof}


\begin{lemma}
	Пусть $B_1, B_2 \in \B$.
	Тогда
	\begin{equation}
		G_{B_1} G_{B_2} = G_{B_2}
		.
	\end{equation}
\end{lemma}
\begin{proof}
	Пусть $x\in\ell_\infty$, тогда
	\begin{equation}
		G_{B_1}G_{B_2}x =
		G_{B_1} ( (B_2 x) \cdot \mathbbm 1 ) =
		(B_2 x) \cdot G_{B_1} ( \mathbbm 1 ) =
		(B_2 x) \cdot  \mathbbm 1 =
		G_{B_2}x
		.
	\end{equation}
\end{proof}


Если же известно, что банахов предел $B$ заведомо обладает дополнительными свойствами инвариантности,
то можно сконструировать и другие примеры операторов, относительно которых $B$ инвариантен.

\begin{example}
	Пусть $B\in\mathfrak B(\sigma_2)$.
	Напомним, что $B(\sigma_2) = B(\sigma_{1/2})$, где
	\begin{equation}
		\sigma_{1/2} x = \left(\dfrac{x_1+x_2}2, \dfrac{x_3+x_4}2, \dfrac{x_5+x_6}2, ...\right)
		.
	\end{equation}
	Рассмотрим оператор $H_B$, $H_Bx = (x_1, Bx, x_2, Bx, x_3, Bx, ...)$.
	Тогда
	\begin{multline}
		B H_B x = B \sigma_{1/2} H_B x = B\left(\dfrac{x_1+Bx}2, \dfrac{x_2+Bx}2, \dfrac{x_3+Bx}2, ...\right) =
		\\=
		B\left(\dfrac{x_1}2, \dfrac{x_2}2, \dfrac{x_3}2, ...\right) + B\left(\dfrac{Bx}2, \dfrac{Bx}2, \dfrac{Bx}2, ...\right)=
		\\=
		\dfrac12 Bx + \dfrac12 B\left(Bx, Bx, Bx, ...\right) = Bx
		.
	\end{multline}

	При этом операторы $H_B$, получаемые таким образом, достаточно <<разнообразны>>.
	Пусть $B_1 x \ne B_2 x$ на некотором $x\in \ell_\infty$.
	Тогда $y = H_{B_1} x - H_{B_2} x = (0, B_1 x - B_2 x, 0, B_1 x - B_2 x, 0, B_1 x - B_2 x, ...)$ и $B_1y = B_2y = \dfrac{B_1 x - B_2 x}2\ne 0$, откуда $(H_{B_1} - H_{B_2})\ell_\infty \not\subseteq ac_0$.


	Покажем  теперь, что $B_2 \notin \mathfrak B (H_{B_1})$ при $B_1 \ne B_2$. Предположим противное. Значит, $B_1 x \ne B_2 x$ на некотором $x\in \ell_\infty$. Пусть $y$ такой же, как выше. Тогда $B_2 y = B_2(H_{B_1} x - H_{B_2} x) = B_2 H_{B_1} x - B_2 H_{B_2} x = B_2 x - B_2 x = 0$, но $B_2 y = \dfrac{B_1x - B_2 x}{2} \ne 0$. Получили противоречие.

	На операторы $\sigma_k$ полученная конструкция обобщается тривиально.
	%TODO: обобщить!
\end{example}



\chapter{Функционалы Сачестона и линейные оболочки}
	При исследовании банаховых пределов особый интерес представляют разделяющие множества~\cite[\S 3]{Semenov2014geomprops}.
Множество $Q\in\ell_\infty$ называют разделяющим, если
для любых неравных $B_1, B_2\in\mathfrak{B}$ существует такая последовательность $x\in Q$,
что $B_1 x \neq B_2 x$.
В частности, разделяющим является~\cite{semenov2010characteristic} множество всех последовательностей из 0 и 1,
которое в дальнейшем мы будем обозначать через $\Omega$
(иногда в литературе встречается также обозначение $\{0;1\}^\mathbb{N}$).

Каждой последовательности $(x_1, x_2, \dots)\in \Omega$ можно поставить в соответствие число
\begin{equation}\label{eq:bijection_omega_0_1}
	\sum_{k=1}^\infty 2^{-k} x_k \in [0,1]
	.
\end{equation}
С точностью до счётного множества это соответствие взаимно однозначно и определяет на множестве $\Omega$ меру,
которую мы будем отождествлять с мерой Лебега на $[0,1]$.

Оказывается, что из $\Omega$ можно выделить некоторые подмножества, которые также будут разделяющими,
например \cite[\S 3, Теорема 11]{Semenov2014geomprops},
\begin{equation}
	U = \{ x\in\Omega: q(x) = 0, p(x) = 1 \}
	.
\end{equation}

Однако множество $U$ имеет меру 1~\cite{semenov2010characteristic}.

В настоящей главе строится пример разделяющего множества,
являющегося подмножеством $\Omega$ и имеющего меру нуль.
Для построения такого множества используется следующий факт.

\begin{lemma}[{\cite[\S 3, замечание 6]{Semenov2014geomprops}}]
	Пусть $X$~--- разделяющее множество и $X \subset \operatorname{Lin} Y$,
	где $\operatorname{Lin} Y$ обозначает линейную оболочку $Y$.
	Тогда $Y$ также является разделяющим множеством.
\end{lemma}

Затем обсуждаются свойства линейных оболочек множеств, определённых с помощью функционалов Сачестона.
В частности, доказывается,
что наряду с использованным при построении разделяющего множества меры нуль включением
\begin{equation}
	\Omega \subset \operatorname{Lin}\{x\in\Omega : p(x) = a,~ q(x) = b\}
\end{equation}
для любых $0\leq b < a \leq 1$,
имеет место равенство
\begin{equation}
	\ell_\infty = \operatorname{Lin}\{x\in\ell_\infty : p(x) = a,~ q(x) = b\}
\end{equation}
для любых $a>b$.

Возникает закономерный вопрос: для каких ещё подмножеств пространства $\ell_\infty$
верны аналогичные соотношения?

Оказывается, что аналогичным свойством обладает и ещё одно подмножество пространства $\ell_\infty$: подпространство
$A_0 = \{ x \in \ell_\infty : \alpha(x) =0 \}$,
где, напомним,
\begin{equation*}
	\alpha(x) = \varlimsup_{i\to\infty} \max_{i<j\leqslant 2i} |x_i - x_j|
	.
\end{equation*}

Пространство $A_0$ обладает рядом интересных свойств.


\begin{theorem}[{\cite[следствие 2]{our-mz2019ac0}}]
	\label{thm:alpha_c_ac_c}
	Пусть $x\in ac$, т.е. $p(x) = q(x)$.
	Тогда $x\in c$ если и только если $\alpha(x) = 0$.
\end{theorem}
Таким образом, $c = ac \cap A_0$.
Включение $c\subset A_0$ собственное.

Некоторые результаты данной главы были анонсированы в~\cite{our-mz2021linearhulls}
и опубликованы в~\cite{avdeev2021vestnik}.


	\section{О хаусдорфовой размерности одного класса множеств}
	На множестве $\Omega$ стандартным образом определяется размерность Хаусдорфа (см. например \cite[Секция 6]{Edgar}).
Для непустого подмножества $F\subset \mathbb R^n$ и $s > 0$ определим $s$-мерную меру Хаусдорфа множества $F$ следующим образом:
$$\mathcal H^s(F) := \lim_{\delta\to0} \inf \left\{\sum_{i=1}^\infty \left({\rm diam} \ U_i \right)^s \ : \ F\subset \bigcup_{i=1}^\infty  U_i, \  0\leqslant {\rm diam} \ U_i \leqslant \delta \right\}.$$

Размерность Хаусдорфа множества  $F\subset \mathbb R^n$ определяется по формуле:
$${\rm dim}_H F := \inf\{ s > 0 \ : \ \mathcal H^s(F)=0\}.$$



Мы приведём определение самоподобных подмножеств множества $\Omega$ (см., например, \cite{falconer1997techniques}).

\begin{definition}
Множество $E\subset\Omega$ называется самоподобным, если существуют $m\in\mathbb{N}$,
$m\geqslant2$, $0< r_1, \dots, r_m<1$ и функции $f_j : \Omega \to \Omega$, $j=1,\dots, m$ такие, что
$$\rho(f_j(x), f_j(y)) = r_j \rho(x,y), \ \forall \ x,y \in \Omega, \ j=1,\dots, m$$
и $E=\bigcup_{j=1}^m f_j(E).$
\end{definition}



\begin{lemma}
	\label{lem:Hausdorf_measure}
	Пусть $E\subset\Omega$ и $TE = E$.
	Тогда размерность Хаусдорфа $E$ равна $1$.
\end{lemma}

\begin{proof}
	Для $j=1,2$ определим функции $f_j : \Omega \to \Omega$ следующим образом:
	$$f_1(x_1, x_2, \dots)=(0, x_1, x_2, \dots), \quad f_2(x_1, x_2, \dots)=(1, x_1, x_2, \dots).$$

	Очевидно, что $E=f_1(E)\cup f_2(E).$

	Теперь мы покажем, что размерность Хаусдорфа множества $E$ равна $1$.

	Т.к. для $j=1,2$ верно
	 $$\rho(f_j(x),f_j(y))=\frac12\rho(x,y), \ \forall \ x, y \in E,$$
	 то функции $f_j$ являются преобразованиями подобия с коэффициентами $r_j=1/2$ для $j=1,2$.


	По~\cite[Теорема 9.3]{Edgar} размерность Хаусдорфа $d$ множества $E$ является решением уравнения:
	$$ r_1^d+r_2^d=1.$$
	Т.к. $r_j=1/2$, то
	$d=1.$
\end{proof}

Лемма~\ref{lem:Hausdorf_measure} позволяет несколько сократить доказательство
[ТОDO: ссылка на ASU\_a\_a].
Действительно, $x\in\Omega\setminus c$ принадлежит $W$ тогда и только тогда, когда
\begin{equation}
	\label{eq:dim_ext_B_W}
	(\ext \mathfrak{B})x = \{0;1\}
\end{equation}
Очевидно, что соотношение~\eqref{eq:dim_ext_B_W} выполнено для $x$ тогда и только тогда, когда оно выполнено для $Tx$.
Следовательно, $W=TW$ и $\dim_H W = 1$.

Аналогично получаем $\dim_H (\Omega \cap ac) = \dim_H (\Omega \cap ac_0) = \dim_H (\Omega \cap c) = \dim_H (\Omega \cap c_0) = 1$.







	\section{О существовании разделяющих множеств малой хаусдорфовой размерности}
	Хорошо известны некоторые примеры разделяющих множеств
[TODO: много ссылок, в т.ч. на новую статью в МЗ].
Применяя лемму~\ref{lem:Hausdorf_measure}, можно показать,
что хаусдорфову размерность 1 имеют множества
TODO: список!

В данном пункте строится пример разделяющего множества,
имеющего малую хаусдорфову размерность.


\begin{theorem}
	\label{thm:Hausdorf_measure_1_n}
	Пусть $n\in\mathbb{N}$.
	Тогда существует разделяющее множество $E\subset\Omega$ такое,
	что $\dim_H E = 1/n$.
\end{theorem}

\begin{proof}
	Пусть
	\begin{equation}
		Е= \{ x \in \Omega : k \neq mn \Rightarrow x_k = 0\}, m\in \mathbb{N}
		,
	\end{equation}
	т.е. у последовательности $x \in E$ равны нулю все элементы, кроме, быть может, $x_n$, $x_{2n}$, $x_{3n}$ и т.д.

	Для $j=1,2$ определим функции $f_j : \Omega \to \Omega$ следующим образом:
	\begin{equation}
		f_1(x_1, x_2, \dots)=(\underbrace{0, ..., 0,}_{\mbox{$n-1$ раз}} 0, x_1, x_2, \dots)
		,
		\quad
		f_2(x_1, x_2, \dots)=(\underbrace{0, ..., 0,}_{\mbox{$n-1$ раз}} 1, x_1, x_2, \dots)
		.
	\end{equation}

	Очевидно, что $E=f_1(E)\cup f_2(E).$

	Теперь мы покажем, что размерность Хаусдорфа множества $E$ равна $2^{-n}$.

	Т.к. для $j=1,2$ верно
	 $$\rho(f_j(x),f_j(y))=2^{-n}\rho(x,y), \ \forall \ x, y \in E,$$
	 то функции $f_j$ являются преобразованиями подобия с коэффициентами $r_j=2^{-n}$ для $j=1,2$.


	По~\cite[Теорема 9.3]{Edgar} размерность Хаусдорфа $d$ множества $E$ является решением уравнения:
	$$ r_1^d+r_2^d=1.$$
	Т.к. $r_j=2^{-n}$, то
	$d=1/n.$
\end{proof}

\begin{remark}
	У читателя может возникнуть закономерный вопрос о свойствах пересечения
	\begin{equation}
		\bigcap_{n\in \mathbb{N}} E_{2^n}
		,
	\end{equation}
	где $E_n$ "--- множество, построенное по теореме~\ref{thm:Hausdorf_measure_1_n}, т.е.
	\begin{equation}
		Е_n = \{ x \in \Omega : k \neq mn \Rightarrow x_k = 0\}, m\in \mathbb{N}
		,
	\end{equation}
	в частности о том, является ли оно разделяющим множеством нулевой хаусдорфовой размерности.
	К сожалению, ответ на этот вопрос положителен только во второй части, а именно
	\begin{equation}
		\bigcap_{n\in \mathbb{N}} E_n = {(0,0,0,...)}
		.
	\end{equation}
	Таким образом, пересечение вложенной последовательности разделяющих множеств может не быть разделяющим множеством.
	Впрочем, существует и более простой пример:
	в качестве вложенной последовательности разделяющих множеств следует взять $\{M_n\} = [0, 2^{-n}]$.
	Очевидно, что $\bigcup\limits M_n = \{0\}$.
\end{remark}


\begin{remark}
	Пусть $n>1$ и множество $E$ построено по теореме~\ref{thm:Hausdorf_measure_1_n}.
	Тогда $p(E)< 1$ и, TODO: ref, мера $E$ равна нулю.
\end{remark}



\chapter{Функционалы Сачестона и мультипликативные свойства носителя последовательности}
	Дальнейшим ослаблением понятия сходимости является сходимость по Чезаро (сходимость в среднем).
Говорят, что последовательность $\{x_n\}\in\ell_\infty$ сходится по Чезаро к $t$, если
\begin{equation}
	\lim_{n\to\infty}\frac1{n}\sum_{i=1}^n x_i = t
	.
\end{equation}
Легко заметить, что обсуждаемые обобщения верхнего и нижнего пределов удовлетворяют соотношению
\begin{equation}
	\label{eq:generalization_of_limits}
	\liminf_{n\to\infty} x_n \leq q(x) \leq \liminf_{n\to\infty}\frac1{n}\sum_{i=1}^n x_i
	\leq
	%\\ \leq
	\limsup_{n\to\infty}\frac1{n}\sum_{i=1}^n x_i
	\leq p(x)
	\leq \limsup_{n\to\infty} x_n
	.
\end{equation}

Отдельный интерес представляет множество всех последовательностей из 0 и 1,
которое, как и выше, мы будем обозначать через $\Omega$.
%(иногда в литературе~\cite{semenov2020geomBL,Semenov2014geomprops} встречается также обозначение $\{0;1\}^\N$).
Понятно, что каждый $x\in \Omega$ можно отождествить с подмножеством множества натуральных чисел
$\supp x \subset \N$.

Вслед за~\cite{hall1992behrend} будем обозначать через $\mathscr{M}A$ множество всех чисел,
кратных элементам множества $A\subset\N$, т.е.
\begin{equation}
	\mathscr{M}A = \{ka: k\in\N, a\in A\}
	,
\end{equation}
через $\chi F$ "--- характеристическую функцию множества $F$.

Так, например,
\begin{gather}
	\chi \mathscr{M}\{2\} = \chi \mathscr{M}\{2, 4\} = \chi \mathscr{M}\{2,4,8,16,...\}
	= (0,1,0,1,0,1,0,1,0,...),
\\
	\chi \mathscr{M}\{3\} = \chi \mathscr{M}\{3,9,27,...\} = (0,0,1,\;0,0,1,\;0,0,1,\;0,0,1,\;0,0,1,\;...),
\\
	\chi \mathscr{M}\{2,3\} = \chi \mathscr{M}\{2,3,6\} = (0,1,1,1,0,1,\;0,1,1,1,0,1,\;0,1,1,1,0,1,...).
\end{gather}

Возникает закономерный вопрос о взаимосвязи структуры множества $A$
и значений, которые принимают обобщения верхнего и нижнего пределов~\eqref{eq:generalization_of_limits}
на последовательности $\chi \mathscr{M}\!A$.
Так, в работах~\cite{davenport1936sequences,davenport1951sequences} доказано, что для любого
$A=\{a_1,a_2,...\}\subset\N$
выполнено
\begin{equation}
	\liminf_{n\to\infty}\frac1{n}\sum_{i=1}^n (\chi\mathscr{M}A)_i =
	\lim_{j\to\infty}\lim_{n\to\infty}\frac1{n}\sum_{i=1}^n (\chi\mathscr{M}\{a_1,a_2,...,a_j\})_i
	.
\end{equation}
В работе~\cite[\S 7]{besicovitch1935density} построено такое множество $A\subset\N$, что
\begin{equation}
	\liminf_{n\to\infty}\frac1{n}\sum_{i=1}^n (\chi\mathscr{M}A)_i \neq
	\limsup_{n\to\infty}\frac1{n}\sum_{i=1}^n (\chi\mathscr{M}A)_i
	.
\end{equation}
За более подробной информацией о множествах типа $\mathscr{M}A$ отсылаем читателя к монографии~\cite{hall1996multiples}.

В этой главе изучается зависимость значений, которые могут принимать функционалы Сачестона
на последовательностях $\chi\mathscr{M}A$, от свойств множества $A$.



\chapter*{Заключение}
\addcontentsline{toc}{chapter}{Заключение}
В работе исследованы такие асимптотические характеристики ограниченных последовательностей,
как $\alpha$--функция (глава 1) и  почти сходимость (глава 2).% и банаховы пределы (глава 3).

Несмотря на то, что $\alpha$--функция оказалась трансляционно неинвариантной,
эта неинвариантность в некотором смысле однородна (см. следствие \ref{thm:est_alpha_Tn_x_full}).
Для элементов пространства почти сходящихся последовательностей $ac $
установлена двусторонняя оценка на расстояние до пространства сходящихся последовательностей $c$,
использующая $\alpha$--функцию.
Примечательно (хотя и ожидаемо), что наряду с трансляционно неинвариантной $\alpha$--функцией
в этой оценке используется и функционал $\lim_{n\to\infty}\alpha(T^n x)$,
который, очевидно, трансляционно инвариантен.
Это вполне логично, поскольку расстояние до пространства $c$ и почти сходимость
сами суть трансляционно инвариантные характеристики.




По итогам исследований, результаты которых составили данную работу,
опубликованы тезисы \cite{our-vvmsh-2018,our-vzms-2018,our-ped-2018-inf-dim-ker,our-ped-2018-alpha-Tx},
а также краткое сообщение~\cite{our-mz2019ac0};
планируется публикация ещё нескольких печатных работ.

Выдвинут ряд гипотез,
работу над доказательством или опровержением которых планируется продолжать в дальнейшем.



\chapter*{Список сформулированных гипотез}
\addcontentsline{toc}{chapter}{Список сформулированных гипотез}

\renewcommand\label[1]{}
\hypotlist


\addcontentsline{toc}{chapter}{Список литературы}

\makeatletter
\ltx@iffilelater{biblatex-gost.def}{2017/02/01}%
{\toggletrue{bbx:gostbibliography}%
\renewcommand*{\revsdnamepunct}{\addcomma}}{}
\makeatother

\printbibliography{}

\end{document}
