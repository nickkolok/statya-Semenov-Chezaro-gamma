\documentclass[a4paper,openbib]{report}
\usepackage{amsmath}
\usepackage[utf8]{inputenc}
\usepackage[english,russian]{babel}
\usepackage{amsfonts,amssymb}
\usepackage{latexsym}
\usepackage{euscript}
\usepackage{enumerate}
\usepackage{graphics}
\usepackage[dvips]{graphicx}
\usepackage{geometry}
\usepackage{wrapfig}
\usepackage[colorlinks=true,allcolors=black]{hyperref}



\righthyphenmin=2

\usepackage[14pt]{extsizes}

\geometry{left=3cm}% левое поле
\geometry{right=1cm}% правое поле
\geometry{top=2cm}% верхнее поле
\geometry{bottom=2cm}% нижнее поле

\renewcommand{\baselinestretch}{1.3}

\renewcommand{\leq}{\leqslant}
\renewcommand{\geq}{\geqslant} % И делись оно всё нулём!

\newcommand{\longcomment}[1]{}

\usepackage[backend=biber,style=gost-numeric,sorting=none]{biblatex}
\addbibresource{../bib/Semenov.bib}
\addbibresource{../bib/my.bib}
\addbibresource{../bib/ext.bib}

\input{../bib/ext.hyphens.bib}

\usepackage{amsthm}
\theoremstyle{definition}
\newtheorem{lemma}{Лемма}[section]
\newtheorem{theorem}[lemma]{Теорема}
\newtheorem{example}[lemma]{Пример}
\newtheorem{property}[lemma]{Свойство}
\newtheorem{corollary}{Следствие}[lemma]

\newtheorem{hhypothesis}[lemma]{Гипотеза}


\newcommand\hypotlist{ }
\newcounter{hypcount}

\makeatletter
\usepackage{environ}
\NewEnviron{hypothesis}{%

	\edef\curlabel{hhypothesis\thehypcount}
    \begin{hhypothesis}
		\label{\curlabel}
		\BODY%
    \end{hhypothesis}
	\edef\curref{\noexpand\ref{\curlabel}}

	\expandafter\g@addto@macro\expandafter\hypotlist\expandafter
	{\paragraph{Гипотеза\!\!\!}}


	\expandafter\g@addto@macro\expandafter\hypotlist\expandafter
	{\expandafter\textbf\expandafter{\curref}}

	\expandafter\g@addto@macro\expandafter\hypotlist\expandafter
	{\textbf{.}~~}

	\expandafter\g@addto@macro\expandafter\hypotlist\expandafter
	{\BODY}

	\addtocounter{hypcount}{1}
}
\makeatother

%Only referenced equations are numbered
\usepackage{mathtools}
\mathtoolsset{showonlyrefs}

%\mathtoolsset{showonlyrefs=false}
% (an equation/multline to be force-numbered)
%\mathtoolsset{showonlyrefs=true}


\begin{document}
\clubpenalty=10000
\widowpenalty=10000
\setcounter{page}{2}
\tableofcontents

\chapter*{Введение}
\addcontentsline{toc}{chapter}{Введение}
Сходящиеся последовательности, т.е. последовательности, имеющие предел в смысле классического математического анализа,
изучены достаточно хорошо.
В частности, любая сходящаяся последовательность является ограниченной.
Пространство ограниченных последовательностей будем, вслед за классиками \cite{wojtaszczyk1996banach,lindenstrauss1973classical},
обозначать через $\ell_\infty$ и снабжать его нормой
\begin{equation*}
	\|x\| = \sup_{n\in\mathbb{N}}|x_n|
	.
\end{equation*}

Однако в приложениях часто возникают ограниченные последовательности,
которые не являются сходящимися.
В таком случае возникает закономерный вопрос:
как измерить <<недостаток сходимости>>?
<<насколько не сходится>> последовательность?

Наиболее очевидным кажется вычисление расстояния $\rho(x,c)$ от заданного элемента $x\in\ell_\infty$
до пространства сходящихся последовательностей $c$
(которое равно половине разности верхнего и нижнего пределов последовательности).
Однако выясняется, что имеют место быть и другие подходы.

Нетрудно заметить, что операция взятия классического предела на пространстве сходящихся последовательностей
является непрерывным (в норме $\ell_\infty$) линейным функционалом.
В 1929 г. С. Мазур анонсировал~\cite{Mazur}, а позже
С. Банах доказал \cite{banach2001theory_rus}, что этот функционал может быть непрерывно продолжен на всё пространство $\ell_\infty$.
На основе этой идеи были определены банаховы пределы
(иногда также называемые пределами Банаха--Мазура \cite{alekhno2012superposition,alekhno2015banach})
следующим образом.

Банаховым пределом называется функционал $B\in \ell_\infty^*$ такой, что:
\begin{enumerate}
	\item
		$B \geqslant 0$
	\item
		$B\one = 1$
	\item
		$B=BT$
\end{enumerate}

Простейшие свойства:
\begin{itemize}
	\item
		$\|B\|_{\ell_\infty^*} = 1$
	\item
		$Bx = \lim\limits_{n\to\infty} x_n$ для любого $x=(x_1, x_2, ...) \in c$.

		Таким образом,
		банахов предел~--- действительно естественное обобщение понятия предела сходящейся последовательности
		на все ограниченные последовательности.
\end{itemize}

Множество банаховых пределов обычно обозначают через $\mathfrak{B}$
(реже через $BM$~--- см., например, \cite{alekhno2012superposition,alekhno2015banach}).

Лоренц \cite{lorentz1948contribution} установил, что существует подпространство $\ell_\infty$,
на котором все банаховы пределы принимают одинаковое значение.
Это пространство названо пространством почти сходящихся последовательностей и обычно обозначается $ac$
(от англ. <<almost convergent>>).
Включение $c \subset ac$ собственное, т.е. $ac \setminus c \neq \varnothing$.


Обобщая критерий Лоренца, Сачестон \cite{sucheston1967banach} доказал, что для любого $x\in\ell_\infty$
и любого $B\in\mathfrak{B}$
\begin{equation*}
	q(x) =
	\lim_{n\to\infty} \sup_{m\in\mathbb{N}} \frac{1}{n} \sum_{k=m+1}^{k=m+n} x_k
	\leq
	Bx
	\leq
	\lim_{n\to\infty} \inf_{m\in\mathbb{N}} \frac{1}{n} \sum_{k=m+1}^{k=m+n} x_k
	= p(x)
\end{equation*}
и, более того,
\begin{equation*}
	\mathfrak{B}x = [q(x), p(x)]
	.
\end{equation*}


За более подробным обзором ранних исследований банаховых пределов отсылаем читателя к~\cite{greenleaf1969invariant,day1973normed,kangro1976theory}.
%Source: https://encyclopediaofmath.org/wiki/Banach_limit
Вскоре после работ Сачестона Дж. Куртц распространил понятие банаховых пределов
на векторные последовательности~\cite{kurtz1970almost},
а затем и на последовательности в произвольных банаховых пространствах~\cite{kurtz1972almost}.
За обсуждением банаховых пределов в векторных пространствах отсылаем читателя
к~\cite{deeds1968summability,hajdukovic1975almost,armario2013vectorvalued_rus,garcia2015extremal,garcia2016fundamental_rus}.
%Тут есть ещё ссылки: https://www.mathnet.ru/php/archive.phtml?wshow=paper&jrnid=faa&paperid=3146&option_lang=rus
%TODO2: В том числе и про то, где применяется.
В недавней работе~\cite{chen2007characterizations} Ч.~Чен и М.~Куо изучают обобщения банаховых пределов
на произвольные гильбертовы пространства и на пространства суммируемых функций $L_p$.
Другим обобщениям банаховых пределов посвящены работы
\cite{hajdukovic1975functionals,koga2016generalization}.
%tanaka2018banach - иероглифы, несовместимые с библиографией

Ещё одним достаточно плодотворным обобщением банаховых пределов оказались их аналоги на двойных последовательностях~\cite{robison1926divergent}, введённые Дж.~Д.~Хиллом в~\cite{hill1965almost}.
За дальнейшими результатами в этом направлении отсылаем читателя
к~\cite{moricz1988almost,bacsarir1995strong,mursaleen2003almost,edely2004almost,mursaleen2004almost}.
Из недавних работ стоит отдельно отметить статью М. Мурсалена и С.А. Мухиддина~\cite{mursaleen2012banach},
в которой с помощью понятия почти сходимости в пространстве ограниченных двойных последовательностей вводится ряд новых интересных подпространств.

Наконец, если исключить из определения банахова предела требование трансляционной инвариантности,
то мы получим объект, называемый обобщённым пределом,
подробно изучавшийся М.~Джерисоном в~\cite{jerison1957set} и многих других работах.


Таким образом, на вопрос: <<Насколько не сходится последовательность?>> %~---
можно дать ответ в терминах почти сходимости, т.е. принадлежности пространству $ac$,
а на вопрос: <<Насколько почти не сходится последовательность?>>~---
назвать длину отрезка $[q(x), p(x)]$.
В дальнейшем пространство почти сходящихся последовательностей неоднократно становилось предметом
различных исследований
\cite{semenov2006ac,usachev2008transforms}.
В частности, в работе~\cite{connor1990almost} доказано,
что последовательность из нулей и единиц почти наверное не принадлежит пространству $ac$.
Этот факт демонстрирует, что почти сходящиеся последовательности <<достаточно редки>>.

%TODO: ссылки! Хватит или ещё?

Банаховы пределы также нашли своё применение в приложениях
\cite{semenov2015banachtraces,SU,strukova2015spectres}.
%Известны обобщения банаховых пределов на двойные последовательности
%\cite{edely2004almost}.

%TODO: ссылки! Хватит или ещё?


В настоящей работе рассматриваются некоторые вопросы асимптотических характеристик ограниченных последовательностей,
в том числе банаховых пределов.
Нумерация приводимых ниже теорем, лемм, определений и следствий совпадает с их нумерацией в диссертации.


В главе 1 обсуждается пространство $ac$ и его подпространство $ac_0$,
даётся критерий почти сходимости к нулю (т.е. принадлежности пространству $ac_0$)
знакопоcтоянной последовательности.

\reflecttheorem{thm:M_j_ac0_inf_lim}
	Пусть $n_i$~--- строго возрастающая последовательность натуральных чисел,
	\begin{equation}
		\label{eq:definition_M_j}
		M(j) = \liminf_{i\to\infty} n_{i+j} - n_i,
	\end{equation}
	\begin{equation}
		x_k = \left\{\begin{array}{ll}
			1, & \mbox{~если~} k = n_i
			\\
			0  & \mbox{~иначе~}
		\end{array}\right.
	\end{equation}
	Тогда следующие условия эквивалентны:
	\\
	(i)   $x \in ac_0$;
	\\\\
	(ii)  $\lim\limits_{j \to \infty} \dfrac{M(j)}{j} = +\infty$;
	\\\\
	(iii) $\inf\limits_{j \in \N}     \dfrac{M(j)}{j} = +\infty$.


\reflecttheorem{thm:crit_ac0_Mj_lambda}
	Пусть $x\in\ell_\infty$, $x \geq 0$, $\lambda>0$.
	Обозначим через $n^{(\lambda)}_i$ возрастающую последовательность
	индексов таких элементов $x$, что $x_k \geq \lambda$ тогда и только тогда,
	когда $k=n^{(\lambda)}_i$ для некоторого $i$.
	Обозначим
	\begin{equation}
		M^{(\lambda)}(j) = \liminf_{i\to\infty} n^{(\lambda)}_{i+j} - n^{(\lambda)}_i
		.
	\end{equation}


	Тогда для того, чтобы $x\in ac_0$, необходимо и достаточно, чтобы
	для любого $\lambda>0$ было выполнено
	\begin{equation}
		\lim_{j \to \infty} \frac{M^{(\lambda)}(j)}{j} = +\infty
		.
	\end{equation}

\reflecttheorem{thm:rho_x_c_leq_alpha_t_s_x_united}
	Для любого $x\in ac$
	\begin{equation}
		\frac{1}{2} \alpha(x) \leq \rho(x,c)\leq \lim_{s\to\infty} \alpha(T^s x)
		.
	\end{equation}

\reflectcorollary{cor:rho_x_c0_leq_alpha_t_s_x_united}
	Для любого $x\in ac_0$
	\begin{equation}
		\frac{1}{2} \alpha(x) \leq \rho(x,c_0)\leq \lim_{s\to\infty} \alpha(T^s x)
		.
	\end{equation}

\reflecttheorem{thm:Connor_generalized}
	Мера множества $F=\{x\in\Omega : q(x) = 0 \wedge p(x)= 1\}$,
	где $p(x)$ и $q(x)$~--- верхний и нижний функционалы Сачестона соответственно,
	равна 1.


В главе 2
%TODO: \ref ???
изучается $\alpha$--функция, введённая в~\cite{our-vzms-2018}:
\begin{equation}
	\alpha(x) = \varlimsup_{i\to\infty} \max_{i<j\leqslant 2i} |x_i - x_j|
	.
\end{equation}
%
%TODO: ссылка на статью Семенова!
%
Поскольку $\alpha(c)=0$,
то $\alpha$--функцию также можно считать <<мерой несходимости>> последовательности;
равенство $\alpha(x) = 0$, однако, вовсе не гарантирует сходимость.

Устанавливается, что $\alpha$--функция не инвариантна относительно оператора сдвига $T$,
и даётся оценка на $\alpha(T^n x)$.
С другой стороны, $\alpha$--функция, в отличие от некоторых банаховых пределов
\cite{Semenov2010invariant,Semenov2011dan},
инвариантна относительно операторов растяжения $\sigma_n$.
Затем выявляется связь между $\alpha$--функцией, расстоянием от заданнной последовательности до пространства $c$
и почти сходимостью.
Рассмотрены и другие свойства $\alpha$--функции.
Приведём основные результаты.

\reflectcorollary{thm:est_alpha_Tn_x_full}
	Для любых $x\in\ell_\infty$ и $n \in \N$
	\begin{equation}\label{est_alpha_Tn_x}
		\frac{1}{2}\alpha(x) \leq \alpha(T^n x) \leq \alpha(x)
		.
	\end{equation}

\reflecttheorem{thm:alpha_beta_T_seq}
	Пусть $\beta_k$~--- монотонная невозрастающая последовательность,
	$\beta_k \to \beta$, $\beta\in\left[\frac{1}{2}; 1\right]$, $\beta_1 \leq 1$.
	Тогда существует такой $x\in\ell_\infty$, что для любого натурального $n$
	\begin{equation}
		\frac{\alpha(T^n x)}{\alpha(x)} = \beta_n.
	\end{equation}

\reflecttheorem{thm:alpha_sigma_n}
	Для любого $x\in\ell_\infty$ и для любого натурального $n$ верно равенство
	\begin{equation}
		\alpha(\sigma_n x) = \alpha(x)
		.
	\end{equation}

\reflecttheorem{thm:alpha_sigma_1_n}
	Для любого $n\in\N$ и любого $x\in\ell_\infty$ выполнено
	\begin{equation}
		\alpha(\sigma_{1/n} x) \leq \left( 2- \frac{1}{n} \right) \alpha(x)
		.
	\end{equation}

\reflecttheorem{thm:alpha_Cx_no_gamma}
	Имеет место равенство
	\begin{equation}
		\sup_{x\in\ell_\infty, \alpha(x)\neq 0} \frac{\alpha(Cx)}{\alpha(x)}=1
		.
	\end{equation}

\reflecttheorem{thm:alpha_xy}
	Пусть $(x\cdot y)_k = x_k\cdot y_k$.
	Тогда
	$\alpha(x\cdot y)\leq \alpha(x)\cdot \|y\|_* + \alpha(y)\cdot \|x\|_*$,
	где
	\begin{equation}
		\|x\|_* = \limsup_{k\to\infty} |x_k|
	\end{equation}
	есть  фактор-норма по $c_0$ на пространстве $\ell_\infty$.

В параграфе~\ref{sec:space_A0} исследуется пространство $A_0 = \{x: \alpha(x) = 0\}$.
Это пространство несепарабельно, замкнуто относительно покоординатного умножения,
операторов левого и правого сдвигов, оператора Чезаро,
операторов растяжения $\sigma_n$ и усредняющего сжатия $\sigma_{1/n}$.

В параграфе~\ref{sec:noncomplementarity} устанавливается, что в цепочке вложений
\begin{equation}
	c_0 \subset A_0 \subset \ell_\infty
\end{equation}
оба подпространства недополняемы.

Как сказано выше, банаховы пределы по определению (как и обычный предел на пространстве $c$) инвариантны относительно оператора сдвига.
Возникает закономерный вопрос: можно ли потребовать от банахова предела сохранять своё значение
при суперпозиции с некоторыми другими операторами на $\ell_\infty$?
Эту проблему исследовал У. Эберлейн в 1950 г. \cite{Eberlein},
т.е. через два года после классической работы Г. Г. Лоренца~\cite{lorentz1948contribution}.
Эберлейн установил, что существуют такие линейные операторы  $A : \ell_\infty\to \ell_\infty$,
для которых $BAx = Bx$ независимо от выбора $x$ и для банаховых пределов специального вида.

Будем говорить, что $B\in\mathfrak B(A)$, $A : \ell_\infty\to \ell_\infty$, если для любого $x\in \ell_\infty$
выполнено равенство $BAx = Bx$.
Такой банахов предел $B$ называют инвариантным относительно оператора $A$.

Можно ли выделить какие-то особые свойства оператора сдвига,
которые необходимы или достаточны оператору, чтобы относительно него были инвариантны все или некоторые банаховы пределы?
Понятно, что если оператор $A$ таков, что для любого $x\in\ell_\infty$ между $Ax$ и $x$
существует (конечное) расстояние Дамерау--Левенштейна \cite{damerau1964technique} (т.е. минимальное количество операций вставки, удаления, замены и перестановки двух соседних элементов последовательности, необходимых для перевода $x$ в $Ax$, причём для разных $x\in\ell_\infty$ эти операции, вообще говоря, не обязаны быть одинаковыми), то относительно данного оператора инвариантен любой банахов предел. Аналогичное утверждение справедливо и в случае, если $Ax -x \in c_0$ для любого $x\in \ell_\infty$.

Следующим по естественности (после сдвига и замены конечного числа элементов) действием, сохраняющем сходимость последовательности, является повторение элементов последовательности, например, оператор
\begin{equation}
	\sigma_2(x_1,x_2,x_3,...) = (x_1,x_1, \; x_2, x_2, \; x_3, x_3, \; ...)
	.
\end{equation}
Однако относительно такого оператора инвариантны не все, а только некоторые банаховы пределы.
Так, в~\cite[теорема 14]{ASSU2} показано, что
\begin{equation}
	\B(\sigma_n) \cap \ext \B = \varnothing  \mbox{~~для любого~~} n\in\N_2
	.
\end{equation}
Заметим, что если мы рассмотрим оператор неравномерного растяжения
\begin{equation}
	\sigma_{1,2}(x_1,x_2,x_3,x_4,x_5,...) = (x_1, \; x_2, x_2, \;  x_3, \; x_4, x_4, \; x_5, ...)
	,
\end{equation}
то увидим, что периодическую последовательность $y_n = (-1)^n$, $y\in ac_0$ оператор $\sigma_{1,2}$
переводит в периодическую последовательность
\begin{equation}
	(-1, 1, 1, \; -1, 1, 1, \; ...) \in ac_{1/3}
	,
\end{equation}
поскольку на периодической последовательности любой банахов предел принимает значение, равное среднему по периоду.
Таким образом, не существует банаховых пределов, инвариантных относительно оператора $\sigma_{1,2}$.

В главе 3 изложены некоторые примеры операторов и найдены множества банаховых пределов,
инвариантных относительно этих операторов.
Затем рассматриваются следующие классы линейных операторов $H:\ell_\infty \to \ell_\infty$:

-- полуэберлейновы: такие, что $B_1 H \in\B$ для некоторого $B_1\in \B$;

-- эберлейновы: такие, что $B_1 H = B_1$ для некоторого $B_1\in \B$;

-- В-регулярные: такие, что $B_1 H \in \B$ для любого   $B_1\in \B$;

-- существенно эберлейновы: такие, что $B_1 H \in \B$ для любого $B_1\in \B$ и $B_2 H \ne B_2$ для некоторого $B_2\in \B$.

Устанавливается (см. теоремы~\ref{thm:amiable_but_not_Eberlein_exists} и~\ref{thm:Eberlein_but_not_B-regular_exists}),
что каждый следующий из этих классов вложен в предыдущий и не совпадает с ним.
Кроме того, доказывается ещё ряд смежных результатов, в частности, решается обратная задача об инвариантности.

\reflecttheorem{thm:generated_operator_G_B}
	Для каждого $B\in \B$ существует такой оператор $G_B:\ell_\infty \to \ell_\infty$,
	что $\B(G_B) = \{B\}$.


Главы 4 и 5 посвящены верхнему и нижнему функционалам Сачестона $p(x)$ и $q(x)$"--- аналогам верхнего и нижнего пределов последовательности.
В главе 4 изучаются разделяющие множества.

Обозначим через $\Omega$ множество всех последовательностей, состоящих из нулей и единиц.

\reflecttheorem{thm:Lin_Omega_Sucheston}
	Пусть
	$1 \geq a > b \geq 0$ и
	$\Omega^a_b = \{x\in\Omega : p(x) = a, q(x) = b\}$,
	где $p(x)$ и $q(x)$~--- верхний и нижний функционалы Сачестона~\cite{sucheston1967banach} соответственно.
	Тогда $\Omega \subset \operatorname{Lin} \Omega^a_b$.


\reflectcorollary{crl:Lin_Omega_Sucheston}
	Множество $\Omega^a_b$ является разделяющим.
	Т.к. при $a\neq 1$ или $b\neq 0$ множество $\Omega^a_b$ имеет меру нуль~\cite{semenov2010characteristic,connor1990almost},
	то оно является разделяющим множеством нулевой меры.


Пусть $X^a_b = \{x\in\ell_\infty : p(x) = a,~ q(x) = b\}$, $Y^a_b = \{x\in A_0 : p(x) = a, q(x) = b\}$, где $a>b$.
\reflecttheorem{thm:A_0_c_infty_lin}
	Пусть $a\neq -b$.
	Тогда справедливо равенство $\operatorname{Lin} Y^a_b = A_0$.

\reflecttheorem{thm:Lin_ell_infty}
	Справедливо равенство $\operatorname{Lin} X^a_b = \ell_\infty$.

Через $\dim_H E$ будем обозначать хаусдорфову размерность множества $E$.

\reflecttheorem{thm:Hausdorf_measure_1_n}
	Пусть $n\in\N$.
	Тогда существует разделяющее множество $E\subset\Omega$ такое,
	что $\dim_H E = 1/n$.

%Конец главы 4

Глава 5 посвящена связи мультипликативных свойств носителя последовательности из нулей и единиц
и значений, которые могут принимать функционалы Сачестона на такой последовательности.


\reflectcorollary{cor:ac0_powers_finite_set_of_numbers}
	Пусть $\{p_1, ..., p_k\} \subset \N$,
	\begin{equation}
		x_k = \begin{cases}
			1, &\mbox{~если~} k = p_1^{j_1}\cdot p_2^{j_2}\cdot ... \cdot p_k^{j_k} \mbox{~для некоторых~} j_1,...,j_k\in\N,
			\\
			0  &\mbox{~иначе}.
		\end{cases}
	\end{equation}
	Тогда $x\in ac_0$.

\reflectdefinition{def:P-property}
	Будем говорить, что множество $A\subset\N$ обладает $P$-свойством,
	если для любого $n\in\N$ найдётся набор попарно взаимно простых чисел
	\begin{equation}
		\{a_{n,1}, a_{n,2}, ..., a_{n,n}  \} \subset A
		.
	\end{equation}

\reflecttheorem{thm:p_x_infinite_multiples}
	Пусть $A\subset \N\setminus\{1\}$.
	Тогда следующие условия эквивалентны:
	\begin{enumerate}[label=(\roman*)]
		\item
			$A$ обладает $P$-свойством
		\item
			В $A$ существует бесконечное подмножество попарно взаимно простых чисел
		\item
			$p(\chi\mathscr{M}A)=1$.
	\end{enumerate}

\reflectcorollary{cor:ac0_primes_p_psi_A_prod}
	Пусть $A = \{a_1, a_2, ..., a_n,...\}$ "--- бесконечное множество попарно взаимно простых чисел
	и $a_{n+1}>a_1\cdot...\cdot a_n$.
	Тогда
	\begin{equation}
		q(\chi\mathscr{M}A) = 1-\prod_{j=1}^\infty \left(1-\frac{1}{a_j}\right)
		.
	\end{equation}

\reflectlemma{lem:q_x_infinite_Euler}
	Пусть $\varepsilon \in  (0; 1{]}$.
	Существует бесконечное множество попарно непересекающихся подмножеств простых чисел
	$A_i$ такое, что $q(\chi\mathscr{M}A_i)\geq\varepsilon$ для любого $i\in\N$.


\chapter{$\alpha$--функция как асимптотическая характеристика ограниченной последовательности}

	\section{Определение и элементарные свойства $\alpha$--функции}
	На пространстве $\ell_\infty$ определяется $\alpha$--функция следующим равенством:
\begin{equation}
	\alpha(x) = \varlimsup_{i\to\infty} \max_{i<j\leqslant 2i} |x_i - x_j|
	.
\end{equation}
Иногда удобнее использовать одно из нижеследующих равносильных определений:
\begin{equation}
	\alpha(x) = \varlimsup_{i\to\infty} \max_{i \leqslant j\leqslant 2i} |x_i - x_j|
	,
\end{equation}
\begin{equation}
	\alpha(x) = \varlimsup_{i\to\infty} \sup_{i<j\leqslant 2i} |x_i - x_j|
	,
\end{equation}
\begin{equation}
	\alpha(x) = \varlimsup_{i\to\infty} \sup_{i \leqslant j\leqslant 2i} |x_i - x_j|
	.
\end{equation}

Легко видеть, что $\alpha$--функция неотрицательна.
\begin{property}
	\label{thm:alpha_x_triangle_ineq}
	Более того, $\alpha$--функция удовлетворяет неравенству треугольника:
	\begin{equation}
		\alpha(x+y) \leq \alpha(x) + \alpha(y)
		.
	\end{equation}
\end{property}
TODO: доказывать нужно или очевидно?

\begin{property}
	На пространстве $\ell_\infty$ $\alpha$--функция удовлетворяет условию Липшица:
	\begin{equation}\label{alpha_Lipshitz}
		|\alpha(x) - \alpha(y)| \leq 2 \|x-y\|
		.
	\end{equation}
	(и эта оценка точна).
	TODO: доказывать нужно или очевидно?
\end{property}

\begin{property}
	Если $y\in c$, то $\alpha(y) = 0$ и $\alpha(x+y) = \alpha(x)$ для любого $x \in \ell_\infty$.
\end{property}

Кроме того, иногда полезно помнить про следующее очевидное
\begin{property}
	\label{thm:alpha_x_leq_limsup_minus_liminf}
	\begin{equation}
		\alpha(x) \leq \varlimsup_{k\to\infty} x_k - \varliminf_{k\to\infty} x_k
		.
	\end{equation}
\end{property}


	\section{$\alpha$--функция и оператор сдвига $T$}
	Результаты этого пункта опубликованы в~\cite{our-ped-2018-alpha-Tx} и используются в~\cite{our-mz2019ac0}.

Как выясняется, $\alpha$--функция не инвариантна относительно оператора сдвига
\begin{equation}
	T(x_1,x_2,x_3,...) = (x_2, x_3, ...).
\end{equation}

\begin{example}
\label{ex:alpha_x_neq_alpha_Tx}
	Пусть
	\begin{equation}
		x_k = \begin{cases}
			(-1)^n, & \mbox{~если~} k = 2^n
			\\
			0 & \mbox{~иначе~}
		\end{cases}
	\end{equation}
\end{example}

Вычислим $\alpha(x)$.
Заметим сначала, что из принадлежности $x_k\in\{-1,0,1\}$
немедленно следует, что $\alpha(x) \leq 2$.
Оценим теперь $\alpha(x)$ снизу:
\begin{multline}
	\alpha(x)
	=
	\varlimsup_{i\to\infty}\max_{i < j \leqslant 2i} |x_i - x_j|
	\geq
	\\\geq
	\mbox{(переход к частичному верхнему пределу
	}\\ \mbox{
	по индексам специального вида $i=2^n$)}
	\geq
	\\\geq
	\varlimsup_{n\to\infty}\max_{2^n < j \leqslant 2^{n+1}} |x_{2^n} - x_j|
	=
	\varlimsup_{n\to\infty}\max_{2^n < j \leqslant 2^{n+1}} |(-1)^n - x_j|
	\geq
	\\ \geq
	\varlimsup_{n\to\infty} |(-1)^n - x_{2^{n+1}}|
	=
	\varlimsup_{n\to\infty} |(-1)^n - (-1)^{n+1}|
	=
	2
	.
\end{multline}

Итак, $\alpha(x) = 2$.
Вычислим теперь $\alpha(Tx)$:
\begin{multline}
	\alpha(Tx)
	=
	\varlimsup_{i\to\infty}~\max_{i < j \leqslant 2i} |(Tx)_i - (Tx)_j|
	=
	\varlimsup_{i\to\infty}~\max_{i < j \leqslant 2i} |x_{i+1} - x_{j+1}|
	=
	\\=
	(\mbox{замена}~k:=i+1, m:=j+1)
	=
	\\=
	\varlimsup_{k\to\infty}~~\max_{k-1 < m-1 \leqslant 2k-2} |x_k - x_m|
	=
	\varlimsup_{k\to\infty}~~\max_{k < m \leqslant 2k-1} |x_k - x_m|
	=
	\\=
	\max\left\{
		\varlimsup_{k\to\infty, k  =   2^n}~~\max_{k < m \leqslant 2k-1} |x_k - x_m|
		,~~
		\varlimsup_{k\to\infty, k \neq 2^n}~~\max_{k < m \leqslant 2k-1} |x_k - x_m|
	\right\}
	=
	\\=
	\max\left\{
		\varlimsup_{n\to\infty}~~\max_{2^n < m \leqslant 2^{n+1}-1} |x_{2^n} - x_m|
		,~~
		\varlimsup_{k\to\infty, k \neq 2^n}~~\max_{k < m \leqslant 2k-1} |x_k - x_m|
	\right\}
	=
	\\=
	\max\left\{
		\varlimsup_{n\to\infty}~~\max_{2^n < m \leqslant 2^{n+1}-1} |(-1)^n - x_m|
		,~~
		\varlimsup_{k\to\infty, k \neq 2^n}~~\max_{k < m \leqslant 2k-1} |x_k - x_m|
	\right\}
	=
	\\=
	\max\left\{
		\varlimsup_{n\to\infty}~~\max_{2^n < m \leqslant 2^{n+1}-1} |(-1)^n - 0|
		,~~
		\varlimsup_{k\to\infty, k \neq 2^n}~~\max_{k < m \leqslant 2k-1} |x_k - x_m|
	\right\}
	=
	\\=
	\max\left\{
		1
		,~
		\varlimsup_{k\to\infty, k \neq 2^n}~~\max_{k < m \leqslant 2k-1} |x_k - x_m|
	\right\}
	=
	\\=
	\mbox{(если $k \neq 2^n$, то $x_k = 0$)}
	=
	\\=
	\max\left\{
		1
		,~
		\varlimsup_{k\to\infty, k \neq 2^n}~~\max_{k < m \leqslant 2k-1} |0 - x_m|
	\right\}
	=
	1
	.
\end{multline}
Таким образом, $\alpha(Tx) = 1 \neq 2 = \alpha(x)$,
что и требовалось показать.

Верна следующая
\begin{theorem}
	Для любого $x \in \ell_\infty$ выполнено неравенство $\alpha(Tx)\leq \alpha(x)$.
\end{theorem}

\begin{proof}
	\begin{multline}
		\alpha(Tx)
		=
		\varlimsup_{i\to\infty}~\max_{i < j \leqslant 2i} |(Tx)_i - (Tx)_j|
		=
		\varlimsup_{i\to\infty}~\max_{i < j \leqslant 2i} |x_{i+1} - x_{j+1}|
		=
		\\=
		(\mbox{замена}~k:=i+1, m:=j+1)
		=
		\\=
		\varlimsup_{k\to\infty}~~\max_{k-1 < m-1 \leqslant 2k-2} |x_k - x_m|
		=
		\varlimsup_{k\to\infty}~~\max_{k < m \leqslant 2k-1} |x_k - x_m|
		\leq
		\\ \leq
		\mbox{(переход к максимуму по большему множеству)}
		\leq
		\\ \leq
		\varlimsup_{k\to\infty}~~\max_{k < m \leqslant 2k} |x_k - x_m|
		=
		\alpha(x)
		.
	\end{multline}
\end{proof}

Более интересна, однако, следующая оценка.
\begin{theorem}
	Для любого $n\in\N$
	\begin{equation}
		\alpha(T^n x) \geq \frac{1}{2} \alpha(x)
		.
	\end{equation}
\end{theorem}

\begin{proof}
	Зафиксируем $n$.
	Заметим, что
	\begin{equation}
		\alpha(x) = \varlimsup_{i\to\infty} \alpha_i(x),
	\end{equation}
	где
	\begin{equation}
		\alpha_i(x) = \max_{i < j \leqslant 2i} |x_i - x_j|.
	\end{equation}

	Рассмотрим $\alpha_i(x)$ при некотором фиксированном $i$, $i>2n$ (меньшие $i$ не влияют на верхний предел).
	Если
	\begin{equation}
		\max_{i < j \leqslant 2i} |x_i - x_j|
		=
		\max_{i < j \leqslant 2i-n} |x_i - x_j|
		,
	\end{equation}
	то
	\begin{multline}\label{alpha_i(x)_leq_alpha_{i-n}(T_n x)}
		\alpha_i(x)
		=
		\max_{i < j \leqslant 2i} |x_i - x_j|
		=
		\max_{i < j \leqslant 2i-n} |x_i - x_j|
		=
		\\=
		\mbox{(замена $k=i-n$, $m=j-n$)}
		=
		\\=
		\max_{k+n < m+n \leqslant 2k+n} |x_{k+n} - x_{m+n}|
		=
		\max_{k < m \leqslant 2k} |(T^n x)_k - (T^n x)_m|
		=
		\alpha_{i-n}(T^n x)
		.
	\end{multline}
	Иначе
	\begin{equation}
		\max_{i < j \leqslant 2i} |x_i - x_j|
		=
		\max_{2i-n < j \leqslant 2i} |x_i - x_j|
	\end{equation}
	и можно записать, что
	\begin{multline}\label{alpha_i(x)_leq_alpha_{i-n}(T_n x) + alpha_{2i-2n}(T_n x)}
		\alpha_i(x)
		=
		\max_{2i-n < j \leqslant 2i} |x_i - x_j|
		=
		\max_{2i-n < j \leqslant 2i} |x_i - x_{2i-n} + x_{2i-n} - x_j|
		\leq
		\\\leq
		\max_{2i-n < j \leqslant 2i} \left( |x_i - x_{2i-n}| + |x_{2i-n} - x_j| \right)
		=
		\\=
		|x_i - x_{2i-n}| + \max_{2i-n < j \leqslant 2i} |x_{2i-n} - x_j|
		=
		\\=
		|x_{i-n+n} - x_{2(i-n)+n}| + \max_{2i-n < j \leqslant 2i} |x_{2i-n} - x_j|
		=
		\\=
		|(T^n x)_{i-n} - (T^n x)_{2(i-n)}| + \max_{2i-n < j \leqslant 2i} |x_{2i-n} - x_j|
		\leq
		\\ \leq
		\alpha_{i-n}(T^n x) + \max_{2i-n < j \leqslant 2i} |x_{2i-n} - x_j|
		\leq
		\\ \leq
		\alpha_{i-n}(T^n x) + \max_{2i-n < j \leqslant 2i} |(T^n x)_{2i-2n} - (T^n x)_{j-n}|
		=
		\\=
		(\mbox{замена}~ m:=j-n ~)
		=
		\\=
		\alpha_{i-n}(T^n x) + \max_{2i-2n < m \leqslant 2i-n} |(T^n x)_{2i-2n} - (T^n x)_m|
		\leq
		\\ \leq
		\mbox{(т.к. $i>2n$, то $4i-4n > 2i+4n-4n = 2i > 2i-n$)}
		\leq
		\\ \leq
		\alpha_{i-n}(T^n x) + \max_{2i-2n < m \leqslant 4i-4n} |(T^n x)_{2i-2n} - (T^n x)_m|
		=
		\alpha_{i-n}(T^n x) + \alpha_{2i-2n}(T^n x)
		.
	\end{multline}

	Сравнивая \eqref{alpha_i(x)_leq_alpha_{i-n}(T_n x)} и \eqref{alpha_i(x)_leq_alpha_{i-n}(T_n x) + alpha_{2i-2n}(T_n x)},
	делаем вывод, что
	\begin{equation}
		\alpha_i(x) \leq \alpha_{i-n}(T^n x) + \alpha_{2i-2n}(T^n x)
		.
	\end{equation}
	Переходя к верхнему пределу, имеем
	\begin{multline}
		\alpha(x)
		=
		\varlimsup_{i\to\infty} \alpha_i(x)
		\leq
		\varlimsup_{i\to\infty} (\alpha_{i-n}(T^n x) + \alpha_{2i-2n}(T^n x))
		\leq
		\\ \leq
		\varlimsup_{i\to\infty} \alpha_{i-n}(T^n x) + \varlimsup_{i\to\infty} \alpha_{2i-2n}(T^n x)
		=
		\\=
		\varlimsup_{j=i-n, j\to\infty} \alpha_{j}(T^n x) + \varlimsup_{i\to\infty} \alpha_{2i-2n}(T^n x)
		=
		\\=
		\alpha(T^n x) + \varlimsup_{i\to\infty} \alpha_{2i-2n}(T^n x)
		\leq
		\\ \leq
		\mbox{(верхний предел по индексам специального вида
		} \\ \mbox{
		заменим на верхний предел по всем индексам)}
		\leq
		\\ \leq
		\alpha(T^n x) + \varlimsup_{k\to\infty} \alpha_{k}(T^n x)
		=
		\alpha(T^n x) + \alpha(T^n x)
		=
		2 \alpha(T^n x)
		.
	\end{multline}
	Таким образом, $\alpha(T^n x) \geq \frac{1}{2} \alpha(x)$,
	что и требовалось доказать.
\end{proof}

Две предыдущие теоремы немедленно влекут

\begin{corollary}
	\label{thm:est_alpha_Tn_x_full}
	Для любых $x\in\ell_\infty$ и $n \in \N$
	\begin{equation}\label{est_alpha_Tn_x}
		\frac{1}{2}\alpha(x) \leq \alpha(T^n x) \leq \alpha(x)
		.
	\end{equation}
\end{corollary}

\begin{remark}
	Оценки \eqref{est_alpha_Tn_x} точны: нижняя достигается, например,
	в примере выше, верхняя же~--- на любой периодической последовательности.
\end{remark}


	\section{О множествах $\{x: \alpha(T^n x) = \alpha(x)\}$}
	В данном параграфе обсуждаются некоторые свойства множеств
\begin{equation}
	\label{eq:alpha_T^n_x_equiv_alpha_x}
	\{x \in \ell_\infty : \alpha(T^n x) = \alpha(x) \}, ~n\in\N,
\end{equation}
\begin{equation}
	\label{eq:cap_alpha_T^n_x_equiv_alpha_x}
	\bigcap\limits_{n\in\N}\{x \in \ell_\infty : \alpha(T^n x) = \alpha(x) \}
	,
\end{equation}
\begin{equation}
	\label{eq:cup_alpha_T^n_x_equiv_alpha_x}
	\bigcup_{n\in\N}\{x \in \ell_\infty : \alpha(T^n x) = \alpha(x) \}
	.
\end{equation}

\subsection{Аддитивные свойства}

\begin{theorem}
	Ни одно из множеств
	\eqref{eq:alpha_T^n_x_equiv_alpha_x}, \eqref{eq:cap_alpha_T^n_x_equiv_alpha_x}, \eqref{eq:cup_alpha_T^n_x_equiv_alpha_x}
	не замкнуто по сложению и, следовательно, не является пространством.
\end{theorem}

\begin{proof}
	Построим два таких элемента, принадлежащих множеству \eqref{eq:alpha_T^n_x_equiv_alpha_x} при любых $n\in\N$,
	сумма которых не принадлежит множеству \eqref{eq:alpha_T^n_x_equiv_alpha_x} ни при каких $n\in\N$.
	Пусть $m\in\N_3$.
	Положим

	\begin{equation}
		x_k = \begin{cases}
			\dfrac{1}{2}(-1)^m,  & \mbox{если } k = 2^m     \\
			1,                   & \mbox{если } k = 2^m + 1 \\
			-1,                  & \mbox{если } k = 2^m + 2 \\
			0                    & \mbox{иначе }
		\end{cases}
	\end{equation}

	и

	\begin{equation}
		y_k = \begin{cases}
			\dfrac{1}{2}(-1)^m,  & \mbox{если } k = 2^m     \\
			-1,                  & \mbox{если } k = 2^m + 1 \\
			1,                   & \mbox{если } k = 2^m + 2 \\
			0                    & \mbox{иначе }
		\end{cases}
	\end{equation}

	Так как
	\begin{equation}
		(T^n x)_{2^m-n+1} - (T^n x)_{2^m-n+2} = 2
	\end{equation}
	и
	\begin{equation}
		(T^n y)_{2^m-n+1} - (T^n y)_{2^m-n+2} = -2
		,
	\end{equation}
	то
	\begin{equation}
		\alpha(x) = \alpha(T^n x) = \alpha(y) = \alpha(T^n y) = 2
		.
	\end{equation}
	С другой стороны,
	\begin{equation}
		(x+y)_k = \begin{cases}
			(-1)^m,  & \mbox{если } k = 2^m     \\
			0        & \mbox{иначе }
		\end{cases}
	\end{equation}
	и
	\begin{equation}
		\alpha(x+y) = 2
		,
	\end{equation}
	но в то же время
	\begin{equation}
		\alpha(T^n(x+y)) = 1
	\end{equation}
	(см. пример \ref{ex:alpha_x_neq_alpha_Tx}),
	следовательно, $x+y$ не принадлежит ни одному из множеств
	\eqref{eq:alpha_T^n_x_equiv_alpha_x}, \eqref{eq:cap_alpha_T^n_x_equiv_alpha_x}, \eqref{eq:cup_alpha_T^n_x_equiv_alpha_x}.
\end{proof}




\subsection{Мультипликативные свойства}

\begin{theorem}
	Ни одно из множеств
	\eqref{eq:alpha_T^n_x_equiv_alpha_x}, \eqref{eq:cap_alpha_T^n_x_equiv_alpha_x}, \eqref{eq:cup_alpha_T^n_x_equiv_alpha_x}
	не замкнуто по умножению.
\end{theorem}

\begin{proof}
	Снова построим два таких элемента, принадлежащих множеству \eqref{eq:alpha_T^n_x_equiv_alpha_x} при любых $n\in\N$,
	произведение которых не принадлежит множеству \eqref{eq:alpha_T^n_x_equiv_alpha_x} ни при каких $n\in\N$.
	Пусть $m\in\N_3$.
	Положим

	\begin{equation}
		x_k = \begin{cases}
			(-1)^m,  & \mbox{если } k = 2^m     \\
			1,                   & \mbox{если } k = 2^m + 1 \\
			0                    & \mbox{иначе }
		\end{cases}
	\end{equation}

	и

	\begin{equation}
		y_k = \begin{cases}
			(-1)^{m+1},  & \mbox{если } k = 2^m     \\
			1,                   & \mbox{если } k = 2^m + 2 \\
			0                    & \mbox{иначе }
		\end{cases}
	\end{equation}

	Так как
	\begin{equation}
		(T^n x)_{2^{2m+1}-n} - (T^n x)_{2^{2m+1}-n+1} = -2
	\end{equation}
	и
	\begin{equation}
		(T^n y)_{2^{2m}-n} - (T^n y)_{2^{2m}-n+2} = -2
		,
	\end{equation}
	то
	\begin{equation}
		\alpha(x) = \alpha(T^n x) = \alpha(y) = \alpha(T^n y) = 2
		.
	\end{equation}
	С другой стороны,
	\begin{equation}
		(x\cdot y)_k = \begin{cases}
			(-1)^m,  & \mbox{если } k = 2^m     \\
			0        & \mbox{иначе }
		\end{cases}
	\end{equation}
	и
	\begin{equation}
		\alpha(x+y) = 2
		,
	\end{equation}
	но в то же время
	\begin{equation}
		\alpha(T^n(x \cdot y)) = 1
	\end{equation}
	(см. пример \ref{ex:alpha_x_neq_alpha_Tx}),
	следовательно, $x \cdot y$ не принадлежит ни одному из множеств
	\eqref{eq:alpha_T^n_x_equiv_alpha_x}, \eqref{eq:cap_alpha_T^n_x_equiv_alpha_x}, \eqref{eq:cup_alpha_T^n_x_equiv_alpha_x}.

\end{proof}


	\section{$\alpha$--функция и семейство операторов $\sigma_n$}
	\paragraph{Теорема.}
$$
	\forall(x\in l_\infty) \forall(n\in\mathbb{N})
	[
		\alpha(\sigma_n x) = \alpha(x)
	]
$$

\paragraph{Доказательство.}
По определению
\begin{equation}
	\alpha(x) = \varlimsup_{i\to\infty} \max_{i<j\leqslant 2i} |x_i - x_j|
\end{equation}

Положим
\begin{equation}
	\alpha_i(x) =
	\max_{i<j\leqslant 2i} |x_i - x_j| =
	\max_{i\leqslant j\leqslant 2i} |x_i - x_j|
\end{equation}

Тогда
\begin{equation}
	\alpha(x) = \varlimsup_{i\to\infty} \alpha_i(x)
\end{equation}

Пусть $y = \sigma_n x$.
Тогда для $k=1, ..., n-1$, $a\in\mathbb{N}$ имеем
\begin{multline}
	\alpha_{an-k}(y) =
	\max_{an-k \leqslant j \leqslant 2an-2k} |y_{an-k} - y_j| =
	\\=
	(\mbox{т.к.}~y_{an-(n-1)}=y_{an-(n-2)}=...=y_{an-k}=...=y_{an-1}=y_{an})=
	\\=
	\max_{an \leqslant j \leqslant 2an-2k} |y_{an} - y_j| \leqslant
	\\ \leqslant
	(\mbox{переходим к максимуму по большему множеству}) \leqslant
	\\ \leqslant
	\max_{an \leqslant j \leqslant 2an} |y_{an} - y_j| =
	\alpha_{an}(y)
\end{multline}

С другой стороны,
\begin{multline}
	\alpha_{an}(y) =
	\max_{an \leqslant j \leqslant 2an} |y_{an} - y_j| =
	\\ =
	(\mbox{т.к.}~y=\sigma_n x,~\mbox{можем рассматривать только}~j=kn)=
	\\ =
	\max_{an \leqslant kn \leqslant 2an} |y_{an} - y_{kn}| =
	\max_{a \leqslant k \leqslant 2a} |y_{an} - y_{kn}| =
	\max_{a \leqslant k \leqslant 2a} |x_a - x_k| =
	\alpha_a(x)
\end{multline}

Таким образом, для $k=1, ..., n-1$, $a\in\mathbb{N}$ имеем соотношения:
\begin{gather}
	\alpha_{an}(y) = \alpha_a(x),
\\
	\alpha_{an-k}(y) \leqslant \alpha_a(x),
\end{gather}
откуда немедленно следует, что
\begin{equation}
	\varlimsup_{i\to\infty} \alpha_i(y) =
	\varlimsup_{i\to\infty} \alpha_i(x),
\end{equation}
т.е.
\begin{equation}
	\alpha(\sigma_n x) = \alpha(x),
\end{equation}
что и требовалось доказать.

TODO: ссылка на статью Семёнова

\paragraph{Следствие.}
$$
	\alpha(C\sigma_2 x) =
	\alpha(\sigma_2 Cx) =
	\alpha(Cx)
$$


	\section{$\alpha$--функция и оператор Чезаро $C$}
	Ниже приводится расширенная версия материала, опубликованного в
\cite{our-vzms-2018}.

На пространстве ограниченных последовательностей $\ell_\infty$ определяется оператор Чезаро $C$
равенством
\begin{equation}
	(Cx)_n = {1}/{n} \cdot \sum_{k=1}^n x_k
	.
\end{equation}

Можно доказать, что верна
\begin{theorem}
	\label{thm:alpha_Cx_leq_alpha_x}
	%TODO: ссылка?
	$\alpha(Cx) \leqslant \alpha(x)$.
\end{theorem}
Выясняется, что эта оценка достаточно точна.

\subsection{Вспомогательная сумма специального вида}
\begin{lemma}
	Если $p\geq 2$, то
	\begin{equation}\label{summa_drobey}
		\sum_{i=0}^{p-1} \frac{i \cdot 2^i}{p} = \frac{2^p(p-2) + 2}{p}
	\end{equation}
\end{lemma}

% В Демидовиче этого не нашёл

\paragraph{Доказательство.}
Равенство \eqref{summa_drobey} равносильно равенству
\begin{equation}\label{summa_drobey_multiplied}
	\sum_{i=0}^{p-1} i \cdot 2^i = 2^p(p-2) + 2
	.
\end{equation}
Докажем это равенство методом математической индукции.

\paragraph{База индукции.}
Для $p=2$ имеем
\begin{equation}
	\sum_{i=0}^{2-1} i \cdot 2^i = 0 \cdot 2^0 + 1 \cdot 2^1 = 2
\end{equation}
и
\begin{equation}
	2^2(2-2) + 2 = 2
	.
\end{equation}
Видим, что для $p=2$ соотношение \eqref{summa_drobey_multiplied} выполняется.

\paragraph{Шаг индукции.}
Пусть соотношение \eqref{summa_drobey_multiplied} выполняется для $p=m$, $m\geq 2$, т.е.
\begin{equation}\label{summa_drobey_multiplied_m}
	\sum_{i=0}^{m-1} i \cdot 2^i = 2^m(m-2) + 2
	.
\end{equation}

Покажем, что тогда соотношение \eqref{summa_drobey_multiplied} выполняется и для $p=m+1$.
Действительно,
\begin{multline}
	\sum_{i=0}^{(m+1)-1} i \cdot 2^i
	=
	\sum_{i=0}^{m} i \cdot 2^i
	=
	\sum_{i=0}^{m - 1} i \cdot 2^i + m\cdot 2 ^m
	\mathop{=}^{\eqref{summa_drobey_multiplied_m}}
	2^m(m-2) + 2 + m\cdot 2 ^m
	=
	\\=
	m\cdot2^m-2\cdot2^m  + 2 + m\cdot 2 ^m
	=
	m\cdot2^{m+1}-2^{m+1}  + 2
	=
	\\=
	2^{m+1}(m-1)  + 2
	=
	2^{m+1}((m+1)-1)  + 2
	,
\end{multline}
т.е. соотношение \eqref{summa_drobey_multiplied} выполняется и для $p=m+1$,
что и требовалось доказать.



\subsection{Вспомогательный оператор $S$}
Пусть $y\in \ell_\infty$.
Определим оператор $S:\ell_\infty \to \ell_\infty$ следующим образом:
\begin{equation}\label{operator_S}
	(Sy)_k = y_{i+2}, \mbox{ где } 2^i < k \leq 2^i+1
\end{equation}
Этот оператор вводится исключительно для упрощения изложения конструкции.

\begin{example}
	$$
		S(\{1,2,3,4,5,6, ...\}) = \{1,2,3,3,4,4,4,4,5,5,5,5,5,5,5,5,6...\}
	$$
\end{example}

Теперь нам потребуются некоторые свойства оператора $S$.

\begin{lemma}
	\label{thm:alpha_S}
	\begin{equation}\label{alpha_S}
		\alpha(Sx) = \varlimsup_{k\to\infty} |x_{k+1} - x_{k}|
	\end{equation}
\end{lemma}

\paragraph{Доказательство.}

\begin{equation*}
	\alpha(Sx) =
	\varlimsup_{i\to\infty} \sup_{i < j \leq 2i} | (Sx)_i - (Sx)_j | = ...
\end{equation*}
Положим для каждого $i$ число $m_i$ так,
что $m_i = 2^{k_i}$, $i \leq m_i < 2i$
(очевидно, это всегда можно сделать).
\begin{equation*}
	... =
	\varlimsup_{i\to\infty} \max \left\{
		\max_{i   < j \leq m_i} | (Sx)_i - (Sx)_j |,
		\max_{m_i < j \leq 2i } | (Sx)_i - (Sx)_j |
	\right\} =
	...
\end{equation*}
Но при $2^{k_i - 1} < i < j \leq m_i = 2^{k_i}$
имеем $(Sx)_i = (Sx)_j$, и первый модуль обращается в нуль.

\begin{equation*}
	... =
	\varlimsup_{i\to\infty}
		\max_{m_i < j \leq 2i } | (Sx)_i - (Sx)_j |
	=
	...
\end{equation*}
Но при $2^{k_i - 1} < i \leq m_i = 2^{k_i} < j \leq 2^{k_i+1}$
имеем $(Sx)_i = x_{k_i+1}$, $(Sx)_j = x_{k_i+2}$, откуда
\begin{equation}\label{alpha_S_sosedi}
	... =
	\varlimsup_{k\to\infty}
		| x_{k+1} - x_k |
\end{equation}

Лемма доказана.

\begin{lemma}
	\begin{equation}\label{summa_S_less}
		\sum_{k=2}^{2^p} (Sy)_k =
		\sum_{i=0}^{p-1} 2^i y_{i+2}
	\end{equation}
\end{lemma}

\paragraph{Доказательство.}

\begin{equation*}
	\sum_{k=2}^{2^p} (Sy)_k =
	\sum_{i=0}^{p-1} \sum_{k=2^i+1}^{2^{i+1}} (Sy)_k =
	\sum_{i=0}^{p-1} \sum_{k=2^i+1}^{2^{i+1}} y_{i+2} =
	\sum_{i=0}^{p-1} 2^i y_{i+2}
\end{equation*}

Лемма доказана.


\begin{lemma}
	\begin{equation}\label{summa_S}
		\sum_{k=2^i+1}^{2^{i+j+1}} (Sx)_k =
		2^i\sum_{k=2}^{2^{j+1}} (ST^ix)_k
	\end{equation}

	Здесь и далее $(Tx)_n = x_{n+1}$.
\end{lemma}

\paragraph{Доказательство.}

\begin{multline*}
	\sum_{k=2^i+1}^{2^{i+j+1}} (Sx)_k =
	\sum_{m = i}^{i+j}\sum_{k=2^m+1}^{2^{m+1}} (Sx)_k =
	\sum_{m = i}^{i+j}2^m \cdot x_{m+2} =
	\\=
	2^i \cdot \sum_{n = 0}^{j}2^n \cdot x_{n+2+i} =
	2^i \cdot \sum_{n = 0}^{j}2^n (T^i x)_{n+2} =
	2^i \cdot \sum_{k=2}^{2^{j+1}} (ST^i x)_k
\end{multline*}

Лемма доказана.

\subsection{Вспомогательная функция $k_b$}

Введём функцию
\begin{equation}\label{def_k_b}
	k_b(x) = \frac{1}{2b}\left|
		\sum_{k=1}^{b}x_k - \sum_{k=b+1}^{2b}x_k
	\right|
\end{equation}

\begin{lemma}
	\begin{equation}\label{alpha_greater_k_b}
		\alpha (Cx) \geq \varlimsup_{i\to \infty} k_i(x)
	\end{equation}
\end{lemma}

\paragraph{Доказательство.}

\begin{multline*}
	\alpha (Cx) \mathop{=}\limits^{def}
	\varlimsup_{i\to \infty} \sup_{i<j\leq 2i} |(Cx)_i - (Cx)_j| \geq
	\varlimsup_{i\to \infty} |(Cx)_i - (Cx)_{2i}| =
	\\ =
	\varlimsup_{i\to \infty} \left|\frac{1}{i}\sum_{k=1}^i  - \frac{1}{2i}\sum_{k=1}^{2i} \right| =
	\varlimsup_{i\to \infty} \left|\frac{1}{i}\sum_{k=1}^i  - \frac{1}{2i}\sum_{k=1}^{i}- \frac{1}{2i}\sum_{k=i+1}^{2i}\right| =
	\\=
	\varlimsup_{i\to \infty} \left|\frac{1}{2i}\sum_{k=1}^i - \frac{1}{2i}\sum_{k=i+1}^{2i}\right| =
	\varlimsup_{i\to \infty} k_i(x)
\end{multline*}

\paragraph{Примечание.}
Введение функции $k_b(x)$ позволит нам в дальнейшем перейти от работы с оператором Чезаро
к несложным преобразованиям сумм.



\subsection{Основные построения}

Построим вектор $y\in \ell_\infty$ следующим образом:

\begin{equation}\label{y_construction}
	y = \left\{
		0, 0, \frac{1}{p}, \frac{2}{p}, \frac{3}{p},
		...,
		\frac{p-1}{p}, 1, \frac{p-1}{p},
		...,
		\frac{1}{p},
		~~
		\underbrace{
		\phantom{\frac{1}{1}\!\!\!}
			0, 0, 0, ..., 0,
		}_\text{$\phantom{\frac{1}{1}\!\!\!}$\!\!\!$3p+1$ раз}
		~~
		\frac{1}{p}, ...
	\right\}
\end{equation}
так, что
\begin{equation}\label{T_y}
	T^{5p}y = y
\end{equation}
%(0 повторяется $3p+1$ раз или около того, надо будет ещё очень аккуратно пересчитать).


Положим $x = Sy$.
Тогда с учётом (\ref{alpha_S})
\begin{equation}\label{alpha_x}
	\alpha (x) = \alpha (Sy) = \frac{1}{p}
\end{equation}


Оценим $\alpha(Cx)$:

\begin{multline*}
	\alpha (Cx) \mathop{\geq}^{(\ref{alpha_greater_k_b})}
	\varlimsup_{b\to \infty} k_b(x) =
	\varlimsup_{b\to \infty}\frac{1}{2b}\left|
		\sum_{k=1}^{b}x_k - \sum_{k=b+1}^{2b}x_k
	\right| \geq
	\\ \geq
	\varlimsup_{
		i\to \infty,~
		b=2^i~
	}\frac{1}{2^{i+1}}\left|
		\sum_{k=1}^{2^i}(Sy)_k - \sum_{k=2^i+1}^{2^{i+1}}(Sy)_k
	\right| =
	\\=
	\varlimsup_{i\to \infty}\frac{1}{2^{i+1}}\left|
		\sum_{k=1}^{2^i}(Sy)_k - 2^i y_{i+2}
	\right| =
	\varlimsup_{i\to \infty}\left|
		\frac{1}{2^{i+1}}\sum_{k=1}^{2^i}(Sy)_k - \frac{y_{i+2}}{2}
	\right| \geq
\end{multline*}
\begin{multline*}
	\\ \geq
	\varlimsup_{
		m\to \infty,~
		i=5pm+p~
	}\left|
		\frac{1}{2^{5pm+p+1}}\sum_{k=1}^{2^{5pm+p}}(Sy)_k - \frac{y_{5pm+p+2}}{2}
	\right| =
	\\=
	\varlimsup_{m\to \infty}\left|
		\frac{1}{2^{5pm+p+1}}\sum_{k=1}^{2^{5pm+p}}(Sy)_k - \frac{1}{2}
	\right| =
	\\=
	\varlimsup_{m\to \infty}\left|
		\frac{1}{2^{5pm+p+1}}\sum_{k=1}^{2^{5pm}}(Sy)_k
		+
		\frac{1}{2^{5pm+p+1}}\sum_{k=2^{5pm}+1}^{2^{5pm+p}}(Sy)_k
		- \frac{1}{2}
	\right|
	\mathop{=}^{(\ref{summa_S})}
	\\=
	\varlimsup_{m\to \infty}\left|
		\frac{1}{2^{5pm+p+1}}\sum_{k=1}^{2^{5pm}}(Sy)_k
		+
		\frac{1}{2^{5pm+p+1}} \cdot 2^{5pm} \cdot \sum_{k=2}^{2^p}(ST^{5pm}y)_k
		- \frac{1}{2}
	\right|
	\mathop{=}^{(\ref{T_y})}
	\\=
	\varlimsup_{m\to \infty}\left|
		\frac{1}{2^{5pm+p+1}}\sum_{k=1}^{2^{5pm}}(Sy)_k
		+
		\frac{1}{2^{5pm+p+1}} \cdot 2^{5pm} \cdot \sum_{k=2}^{2^p}(Sy)_k
		- \frac{1}{2}
	\right| =
	\\=
	\varlimsup_{m\to \infty}\left|
		\frac{1}{2^{5pm+p+1}}\sum_{k=1}^{2^{5pm}}(Sy)_k
		+
		\frac{1}{2^{p+1}} \sum_{k=2}^{2^p}(Sy)_k
		- \frac{1}{2}
	\right|
	\mathop{=}^{(\ref{summa_S_less})}
	\\=
	\varlimsup_{m\to \infty}\left|
		\frac{1}{2^{5pm+p+1}}\sum_{k=1}^{2^{5pm}}(Sy)_k
		+
		\frac{1}{2^{p+1}} \sum_{i=0}^{p-1}2^i y_{i+2}
		- \frac{1}{2}
	\right| =
	\\=
	\varlimsup_{m\to \infty}\left|
		\frac{1}{2^{5pm+p+1}}\sum_{k=1}^{2^{5pm}}(Sy)_k
		+
		\frac{1}{2^{p+1}} \sum_{i=0}^{p-1}2^i \cdot \frac{i}{p}
		- \frac{1}{2}
	\right|
	\mathop{=}^{(\ref{summa_drobey})}
	\\=
	\varlimsup_{m\to \infty}\left|
		\frac{1}{2^{5pm+p+1}}\sum_{k=1}^{2^{5pm}}(Sy)_k
		+
		\frac{1}{2^{p+1}} \cdot \frac{2^p(p-2)+2}{p}
		- \frac{1}{2}
	\right| =
	\\=
	\varlimsup_{m\to \infty}\left|
		\frac{1}{2^{5pm+p+1}}\sum_{k=1}^{2^{5pm}}(Sy)_k
		+
		\frac{1}{2} \cdot \frac{p-2}{p} + \frac{1}{p 2^p}
		- \frac{1}{2}
	\right| =
	\\=
	\varlimsup_{m\to \infty}\left|
		\frac{1}{2^{5pm+p+1}}\sum_{k=1}^{2^{5pm}}(Sy)_k
		-
		\frac{1}{p} + \frac{1}{p 2^p}
	\right| =
	\\=
	\varlimsup_{m\to \infty}\left|
		\frac{1}{2^{5pm+p+1}}\sum_{k=1}^{2^{5pm-2p}}(Sy)_k
		+
		\frac{1}{2^{5pm+p+1}}\sum_{k=2^{5pm-2p}+1}^{2^{5pm}}(Sy)_k
		-\frac{1}{p} + \frac{1}{p 2^p}
	\right| =
	\\\mbox{но во второй сумме все $(Sy)_k$ --- нули по построению}
\end{multline*}
\begin{multline*}
	\\=
	\varlimsup_{m\to \infty}\left|
		\frac{1}{2^{5pm+p+1}}\sum_{k=1}^{2^{5pm-2p}}(Sy)_k
		-\frac{1}{p} + \frac{1}{p 2^p}
	\right| = h
\end{multline*}

Но $0 \leq (Sy)_k \leq 1$,
значит,
$$
	\frac{1}{2^{5pm+p+1}}\sum_{k=1}^{2^{5pm-2p}}(Sy)_k
	\leq
	\frac{1}{2^{5pm+p+1}} \cdot 2^{5pm-2p}
	=
	\frac{1}{2^{3p+1}}
$$
Модуль раскрываем со знаком ``-''

\begin{multline*}
	h=
	\varlimsup_{m\to \infty} \left(
		\frac{1}{p} (1-2^{-p})
		- \frac{1}{2^{3p+1}}
	\right) =
	\frac{1}{p} (1-2^{-p})
	- \frac{1}{2^{3p+1}}
	= \\ =
	\frac{1}{p} (1-2^{-p})
	- \frac{1}{2^{2p+1}} \cdot 2^{-p}
	>
	\frac{1}{p} (1-2^{-p})
	- \frac{1}{p} \cdot 2^{-p}
	=
	\frac{1}{p} (1-2^{-p+1})
\end{multline*}


Таким образом,
$$
	\frac{\alpha(Cx)}{\alpha(x)} \geq
	\frac{	\frac{1}{p} (1-2^{-p+1}) }{\frac{1}{p}} =
	1-2^{-p+1}
$$

Рассматривая $x$ как функцию от $p$, имеем:
$$
	\sup_{p\in\mathbb{N}} \frac{\alpha(Cx(p))}{\alpha(x(p))} \geq
	\sup_{p\in\mathbb{N}} (1-2^{-p+1}) =
	1
	.
$$
Таким образом, может быть сформулирована следующая

\begin{theorem}
	\label{thm:alpha_Cx_no_gamma}
	\begin{equation}
		\sup_{x\in\ell_\infty, \alpha(x)\neq 0} \frac{\alpha(Cx)}{\alpha(x)}=1
		.
	\end{equation}
\end{theorem}

\subsection{Некоторые гипотезы}

\begin{hypothesis}
	Пусть $x\in\ell_\infty$ и $0 \leq x_k \leq 1$.
	Тогда для любого $n\in\mathbb{N}$
	\begin{equation}
		\alpha(C^n x) \leq \frac{1}{2^n}
		.
	\end{equation}
\end{hypothesis}

\begin{hypothesis}
	Для любых $x\in\ell_\infty$ и $n\in\mathbb{N}$
	\begin{equation}
		\alpha(C^n x) - \alpha(C^{n+1} x) \leq \alpha(C^{n-1} x) - \alpha(C^{n} x)
		.
	\end{equation}
\end{hypothesis}

\begin{hypothesis}
	Для любого $0<a<1$ существует такой $x\in\ell_\infty$, что
	\begin{equation}
		\lim_{m\to\infty} \frac{\alpha(C^m x)}{a^m} = \infty
		.
	\end{equation}
\end{hypothesis}


\chapter{Последовательности, почти сходящиеся к нулю (пространство $ac_0$)}

	\section{Переформулировка критерия Лоренца почти сходимости последовательности к нулю}
	Результаты этого пункта использованы в~\cite{our-mz2019ac0}.

Дадим переформулировку критерия Лоренца
\cite{lorentz1948contribution,bennett1974consistency}
почти сходимости последовательности,
которая иногда позволяет упростить доказательство:
брать предел равномерно не по всем $m\in \N$,
а только по достаточно большим значениям.


\begin{theorem}
	[Модифицированный критерий Лоренца]
	\label{thm:Lorentz_mod}
	Пусть $x\in\ell_\infty$.

	$x\in ac_0$ тогда и только тогда, когда
	\begin{equation}\label{crit_pos_ac0}
		\forall(A_2\in\N)
		\exists(n_0\in\N)
		\exists(m_0\in\N)
		\forall(n\geq n_0)
		\forall(m\geq m_0)
		\\
		\left[
			\left|
			\frac{1}{n}
			\sum_{k=m+1}^{m+n} x_k
			\right|
			<
			\frac{1}{A_2}
		\right]
		.
	\end{equation}

\end{theorem}

\begin{proof}
	По теореме Лоренца $x\in ac_0$ тогда и только тогда, когда
	\begin{equation}\label{Lorencz_ac0}
		\lim_{n\to\infty} \frac{1}{n} \sum_{k=m+1}^{m+n} x_k = 0
	\end{equation}
	равномерно по $m$.

	Или, переводя на язык кванторов,
	\begin{equation}\label{crit_ac0}
		\forall(A_1\in\N)
		\exists(n_1\in\N)
		\forall(n\geq n_0)
		\forall(m \in \N)
		\\
		\left[
			\left|
			\frac{1}{n}
			\sum_{k=m+1}^{m+n} x_k
			\right|
			<
			\frac{1}{A_1}
		\right]
		.
	\end{equation}
	Очевидно, что из \eqref{crit_ac0} следует \eqref{crit_pos_ac0} (например, положив $m_0 = 1$),
	тем самым необходимость \eqref{crit_pos_ac0} доказана.

	\paragraph{Достаточность.}
	Пусть выполнено \eqref{crit_pos_ac0}.
	Покажем, что выполнено \eqref{crit_ac0}.
	Зафиксируем $A_1$.
	Положим $A_2 = 2A_1$ и отыщем $n_0$ и $m_0$ в соотвествии с \eqref{crit_pos_ac0}.
	Положим $n_1 = 2A_2(n_0+m_0)\|x\|$.
	Покажем, что \eqref{crit_ac0} верно для любых $n\geq n_1$, $m\in \N$.
	Зафиксируем $n$ и рассмотрим $m$.

	Пусть сначала $m\geq m_0$.
	Тогда в силу того, что $n\geq n_1 = 2A_2(n_0+m_0)\|x\| > n_0$ имеем $n>n_0$.
	Применим \eqref{crit_pos_ac0}:
	\begin{equation}
		\left|
		\frac{1}{n}
		\sum_{k=m+1}^{m+n} x_k
		\right|
		<
		\frac{1}{A_2}
		=
		\frac{1}{2A_1}
		<
		\frac{1}{A_1}
		,
	\end{equation}
	т.е. \eqref{crit_ac0} выполнено.

	Пусть теперь $m < m_0$.
	Заметим, что
	\begin{equation}
		\left|
			\sum_{k=m+1}^{m+n} x_k
			-
			\sum_{k=m_0+1}^{m_0+n} x_k
		\right|
		\leq 2(m_0 - m) \|x\|
		,
	\end{equation}
	откуда
	\begin{equation}
		\left| \sum_{k=m+1}^{m+n} x_k \right|
		\leq
		2(m_0 - m) \|x\| + \left| \sum_{k=m_0+1}^{m_0+n} x_k \right|
		\leq
		2 m_0 \|x\| + \left| \sum_{k=m_0+1}^{m_0+n} x_k \right|
		.
	\end{equation}


	Тогда
	\begin{multline}
		\left| \frac{1}{n} \sum_{k=m+1}^{m+n} x_k \right|
		\leq
		\frac{2 m_0 \|x\|}{n} + \left| \frac{1}{n} \sum_{k=m_0+1}^{m_0+n} x_k \right|
		\mathop{\leq}^{\mbox{~~в силу \eqref{crit_pos_ac0}~~}}
		\frac{2 m_0 \|x\|}{n} + \frac{1}{A_2}
		\leq
		\\ \leq
		\frac{2 m_0 \|x\|}{n_1} + \frac{1}{A_2}
		\leq
		\frac{2 m_0 \|x\|}{2A_2(n_0+m_0)\|x\|} + \frac{1}{A_2}
		=
		\\=
		\frac{m_0}{A_2(n_0+m_0)} + \frac{1}{A_2}
		<
		\frac{1}{A_2} + \frac{1}{A_2}
		=
		\frac{1}{A_1}
		,
	\end{multline}
	т.е. \eqref{crit_ac0} тоже выполнено.
\end{proof}

Удобство критерия \eqref{crit_pos_ac0} в том,
что можно выбирать $m_0$ в зависимости от $A_2$.


	\section{О почти сходимости к нулю последовательности из нулей и единиц}
	\paragraph{Задача о пасьянсе из нулей и единиц.}
Пусть $n_i$~--- строго возрастающая последовательность натуральных чисел,
\begin{equation}
	M(j) = \liminf_{i\to\infty} n_{i+j} - n_i,
\end{equation}
\begin{equation}
	x_k = \left\{\begin{array}{ll}
		1, & \mbox{~если~} k = n_i
		\\
		0  & \mbox{~иначе~}
	\end{array}\right.
\end{equation}

\paragraph{Замечание.}
Так как $M(j)$ есть нижний предел последовательности натуральных чисел,
то он всегда достигается,
т.е. $M(j)\in\mathbb{N}$.

Более того, для любого $j$ существует лишь конечное количество отрезков длины $M(j)$,
содержащих более $j$ единиц,
и бесконечное количество отрезков длины $M(j)$,
содержащих ровно $j$ единиц.

Через $E_j$ будем обозначать конец последнего отрезка длины $M(j)$,
содержащего более $j$ единиц.

\paragraph{Утверждение.}
Если $x \in ac_0$, то
\begin{equation}\label{lim_M(j)/j}
	\lim_{j \to \infty} \frac{M(j)}{j} = +\infty
	.
\end{equation}

\paragraph{Доказательство.}
Очевидно, что если
\begin{equation}
	\liminf_{j \to \infty} \frac{M(j)}{j} = +\infty
	,
\end{equation}
то выполнено \eqref{lim_M(j)/j}.
Предположим противное:
\begin{equation}
	\liminf_{j \to \infty} \frac{M(j)}{j} = с < +\infty
	.
\end{equation}
Очеивдно, что в таком случае $c>0$.

По определению нижнего предела найдётся счётное множество
$J\subset\mathbb{N}$ такое, что
\begin{equation}
	\forall(j\in J)\left[c \leq \frac{M(j)}{j} \leq c+1 \right],
\end{equation}
т.е. для любого $j\in J$ существует бесконечно много отрезков длины $j\cdot(c+1)$,
на каждом из которых не менее $j$ единиц.

Т.к. $x\in ac_0$, то
\begin{equation}\label{Lorencz_ac0_epsilon}
	\forall(\varepsilon>0)
	\exists(n_0\in\mathbb{N})
	\forall(n \geq n_0)
	\forall(m\in\mathbb{N})
	\left[
		\frac{1}{n} \sum_{k=m+1}^{m+n}x_k < \varepsilon
	\right]
	.
\end{equation}

Положим $\varepsilon = 1/(c+2)$ и отыщем $n_0$.
Положим $n\in J$, $n\geq n_0$.
(Такое $n$ всегда найдётся, т.к. $J$ счётно и $J\subset\mathbb{N}$.)
Выберем $m$ так, чтобы отрезок длины $n\cdot(c+1)$,
содержащий не менее $n$ единиц,
начинался с $m+1$.
Тогда
\begin{equation}
	\frac{1}{n\cdot(c+1)}\sum_{k=m+1}^{m+n\cdot(c+1)}x_k
	\geq
	\frac{1}{n\cdot(c+1)} \cdot n
	=
	\frac{1}{c+1}
	>
	\frac{1}{c+2}
	=
	\varepsilon,
\end{equation}
что противоречит \eqref{Lorencz_ac0_epsilon}.

Полученное противоречие завершает доказательство.


\paragraph{Утверждение.}
Если
\begin{equation}\label{lim_M(j)/j_dost}
	\lim_{j \to \infty} \frac{M(j)}{j} = +\infty
	,
\end{equation}
то $x \in ac_0$.

\paragraph{Доказательство.}

По определению предела \eqref{lim_M(j)/j_dost} означает, что
\begin{equation}\label{lim_M(j)/j_ifty_def}
	\forall(C  \in\mathbb{N})
	\exists(j_0\in\mathbb{N})
	\forall(j \geq j_0)
	\left[
		\frac{M(j)}{j}>C
	\right]
	.
\end{equation}
Покажем, что выполнен модификированный критерий Лоренца почти сходимости последовательности к нулю
\eqref{crit_pos_ac0}, т.е.
\begin{equation}
	\forall(B  \in\mathbb{N})
	\exists(n_0\in\mathbb{N})
	\exists(m_0\in\mathbb{N})
	\forall(n\geq n_0)
	\forall(m\geq m_0)
	\\
	\left[
		\frac{1}{n}
		\sum_{k=m+1}^{m+n} x_k
		<
		\frac{1}{B}
	\right]
	.
\end{equation} Действительно, зафиксируем $B$.
Используя \eqref{lim_M(j)/j_ifty_def} и положив $C=2B$,
отыщем $j_0$ такое, что для любого $j\geq j_0$ выполнено
$M(j)>2Bj$.
Положим $n_0 = 2Bj_0$.
Выберем
$$
	m_0 = 2+\max_{1\leq j \leq j_0} E_j
	.
$$

Тогда для любых $m\geq m_0$ и $n\geq n_0$ имеем
\begin{multline}
	\frac{1}{n} \sum_{k=m+1}^{m+n} x_k
	<
	\frac{1}{n} \cdot \left( \frac{n}{M(j_0)} + 1 \right) j_0
	=
	\frac{j_0}{M(j_0)} + \frac{j_0}{n}
	\leq
	\frac{1}{2B} + \frac{j_0}{n}
	\leq
	\\ \leq
	\frac{1}{2B} + \frac{j_0}{n_0}
	=
	\frac{1}{2B} + \frac{1}{2B}
	=
	\frac{1}{B}
	,
\end{multline}
т.е. условие критерия выполнено
и $x\in ac_0$,
что и требовалось доказать.

Последовательность $\{M(j)\}$, как легко выяснить, удовлетворяет некоторым условиям.


	\section{Замечание о свойствах последовательности $M(j)$}
	
Пусть $n_i$~--- строго возрастающая последовательность натуральных чисел,
\begin{equation}
	C_j = \liminf_{i\to\infty} n_{i+j} - n_i
\end{equation}

Тогда для любых $i$, $j$ имеет место быть
\begin{equation}\label{C_j_addit}
	C_{i+j} \geq C_i + C_j
	.
\end{equation}
\paragraph{Доказательство.}
По определению нижнего предела существует лишь конечное число номеров $k$
таких, что $n_{k+i} - n_k < C_i$ или $n_{k+j} - n_k < C_j$.
Зафиксируем $p$, большее всех таких $k$.

По определению нижнего предела и с учётом того, что выражение под знаком предела
принимает лишь натуральные значения,
существует бесконечное количество номеров $q$ таких, что $n_{q+i+j} - n_q = C_{i+j}$.

Обозначим через $s$ некоторый такой номер, больший $p$.
Тогда
\begin{multline}
	C_{i+j} = n_{s+i+j} - n_s = n_{s+i+j} - n_{s+i} + n_{s+i} - n_s
	\geq \\
	\geq C_j + n_{s+i} - n_s \geq C_j + C_i,
\end{multline}
так как из $s>p$ следует, что $n_{s+i+j} - n_{s+i} \geq C_j$ и $n_{s+i} - n_s \geq C_i$.

\paragraph{Пример.}
Последовательность $C_j = j+1$ не удовлетворяет условию \eqref{C_j_addit}:
действительно, $3=C_2 < C_1+C_1 = 2+2 = 4$.

\paragraph{Примечание.}
Условие \eqref{C_j_addit} является необходимым, но неизвестно, является ли оно достаточным.


	\section{Срезочный критерий почти сходимости к нулю неотрицательной последовательности}
	\documentclass[a4paper,14pt]{article} %размер бумаги устанавливаем А4, шрифт 12пунктов
\usepackage[T2A]{fontenc}
\usepackage[utf8]{inputenc}
\usepackage[english,russian]{babel} %используем русский и английский языки с переносами
\usepackage{amssymb,amsfonts,amsmath,mathtext,cite,enumerate,float,amsthm} %подключаем нужные пакеты расширений
\usepackage[unicode,colorlinks=true,citecolor=black,linkcolor=black]{hyperref}
%\usepackage[pdftex,unicode,colorlinks=true,linkcolor=blue]{hyperref}
\usepackage{indentfirst} % включить отступ у первого абзаца
\usepackage[dvips]{graphicx} %хотим вставлять рисунки?
\graphicspath{{illustr/}}%путь к рисункам

\makeatletter
\renewcommand{\@biblabel}[1]{#1.} % Заменяем библиографию с квадратных скобок на точку:
\makeatother %Смысл этих трёх строчек мне непонятен, но поверим "Запискам дебианщика"

\usepackage{geometry} % Меняем поля страницы.
\geometry{left=2cm}% левое поле
\geometry{right=1cm}% правое поле
\geometry{top=2cm}% верхнее поле
\geometry{bottom=2cm}% нижнее поле

\renewcommand{\theenumi}{\arabic{enumi}}% Меняем везде перечисления на цифра.цифра
\renewcommand{\labelenumi}{\arabic{enumi}}% Меняем везде перечисления на цифра.цифра
\renewcommand{\theenumii}{.\arabic{enumii}}% Меняем везде перечисления на цифра.цифра
\renewcommand{\labelenumii}{\arabic{enumi}.\arabic{enumii}.}% Меняем везде перечисления на цифра.цифра
\renewcommand{\theenumiii}{.\arabic{enumiii}}% Меняем везде перечисления на цифра.цифра
\renewcommand{\labelenumiii}{\arabic{enumi}.\arabic{enumii}.\arabic{enumiii}.}% Меняем везде перечисления на цифра.цифра

\sloppy


\renewcommand\normalsize{\fontsize{14}{25.2pt}\selectfont}

\begin{document}
% !!!
% Здесь начинается реальный ТеХ-код
% Всё, что выше - беллетристика

Определим (нелинейный) оператор $\lambda$--срезки $A_\lambda$ на пространстве $\ell_\infty$.
Для $x = (x_1, x_2, ...) \in \ell_\infty$ положим
\begin{equation}
	(A_\lambda x)_k = \begin{cases}
		1, & \mbox{~если~} x_k \geq \lambda
		\\
		0  & \mbox{~иначе.~}
	\end{cases}
\end{equation}

\paragraph{Теорема.}
(Срезочный критерий почти сходимости к нулю неотрицательной последовательности.)
Пусть $x\in\ell_\infty$, $x\geq 0$.
Тогда
\begin{equation}
	x\in ac_0 \Leftrightarrow
	\forall(\lambda>0)[A_\lambda x \in ac_0]
	.
\end{equation}

\paragraph{Необходимость.}
Пусть $x\in ac_0$.
Зафиксируем $\lambda > 0$.
Пусть $y=A_\lambda x$, тогда
\begin{equation}
	(\lambda y)_k = \begin{cases}
		\lambda, & \mbox{~если~} x_k \geq \lambda
		\\
		0  & \mbox{~иначе~}
	\end{cases}
\end{equation}
Таким образом, $0 \leq \lambda y \leq x$.
Следовательно, если $x \in ac_0$,
то $\lambda y \in ac_0$ и $y \in ac_0$,
что и требовалось доказать.

\paragraph{Достаточность.}
Очевидно, что
\begin{equation}
	x\in ac_0 \Leftrightarrow
	\frac{x}{\|x\|}\in ac_0
	.
\end{equation}
Поэтому, не теряя общности, будем полагать $\|x\|\leq 1$.
Более того,
\begin{equation}
	A_\lambda x \in ac_0 \Leftrightarrow
	(1-\lambda)A_\lambda x \in ac_0
	.
\end{equation}

Преположим противное, т.е. $x\notin ac_0$,
но $\forall(\lambda>0)[(1-\lambda))A_\lambda x \in ac_0]$.
Запишем покванторное отрицание критерия Лоренца:
\begin{equation}\label{ac0_lambda_Lorencz_neg}
	\exists(\varepsilon_0 > 0)
	\forall(n_0 \in \mathbb{N})
	\exists(n > n_0)
	\exists(m \in \mathbb{N})
	\left[
		\frac{1}{n}\sum_{k=m+1}^{m+n} x_k \geq \varepsilon_0
	\right]
	.
\end{equation}
Найдём такое $\varepsilon_0$ и положим
\begin{equation}
	\varepsilon = \min\left\{ \frac{\varepsilon_0}{2}, \frac{1}{2} \right\}
	.
\end{equation}
Легко видеть, что
\begin{equation}\label{ac0_lambda_Lorencz_neg_epsilon}
	\forall(n_0 \in \mathbb{N})
	\exists(n > n_0)
	\exists(m \in \mathbb{N})
	\left[
		\frac{1}{n}\sum_{k=m+1}^{m+n} x_k > \varepsilon
	\right]
	.
\end{equation}
(знак неравенства сменился на строгий, это будет играть ключевую роль в дальнейших выкладках).

Построим последовательность
\begin{equation}
	y = \left( 1 - \frac{\varepsilon}{2} \right) A_{\varepsilon/2} x
	.
\end{equation}
Заметим, что $y\in ac_0$, т.е. по критерию Лоренца
\begin{equation}\label{ac0_lambda_Lorencz}
	\forall(\varepsilon_1 > 0)
	\exists(n_1 \in \mathbb{N})
	\forall(n' > n_1)
	\forall(m' \in \mathbb{N})
	\left[
		\frac{1}{n}\sum_{k=m+1}^{m+n} y_k < \varepsilon_1
	\right]
	.
\end{equation}

Положим в \eqref{ac0_lambda_Lorencz} $\varepsilon_1 = \varepsilon/2$
и отыщем $n_1$.
Положим в \eqref{ac0_lambda_Lorencz_neg_epsilon}
$n_0 = n_1$ и отыщем $n$ и $m$.
Положим в \eqref{ac0_lambda_Lorencz} $n' = n$, $m' = m$.
Тогда получим, что по \eqref{ac0_lambda_Lorencz_neg_epsilon}
\begin{equation}
	\frac{1}{n}\sum_{k=m+1}^{m+n} x_k > \varepsilon
	.
\end{equation}
С другой стороны, по \eqref{ac0_lambda_Lorencz}
\begin{equation}
	\frac{1}{n}\sum_{k=m+1}^{m+n} y_k < \varepsilon/2
	.
\end{equation}
Вычитая, получим
\begin{equation}
	\frac{1}{n}\sum_{k=m+1}^{m+n} (x_k - y_k) > \varepsilon/2
	.
\end{equation}
Если среднее арифметическое чисел вида $x_k - y_k$ больше $\varepsilon/2$,
то существует хотя бы один индекс $k$ такой, что $x_k - y_k > \varepsilon/2$.

Предположим, что $k$ таково, что $x_k < \varepsilon/2$.
Тогда $y_k = 0$ и $x_k - y_k < \varepsilon/2$.
Значит, предположение неверно и $x_k \geq \varepsilon/2$
Тогда $y_k = 1-\varepsilon/2$ и с учётом $\|x\|\leq 1$ имеем
\begin{equation}
	x_k - y_k \leq 1- y_k = 1 - (1-\varepsilon/2) = \varepsilon/2
	.
\end{equation}
Следовательно, требуемого индекса $k$ не существует,
и \eqref{ac0_lambda_Lorencz_neg} не выполнено.

Полученное противоречие завершает доказательство.


TODO: сформулировать критерий в терминах $M_\lambda(j)$.

\end{document}


	\section{Пространство $ac_0$ и $\alpha$--функция}
	Докажем теперь ещё одну теорему,
раскрывающую связь между сходимостью, почти сходимостью и $\alpha$--функцией.

\paragraph{Теорема.}
Пусть $x\in ac_0$ и $\alpha(x)=0$.
Тогда $x \in c_0$.

\paragraph{Доказательство.}

Предположим противное, т.е. $x\notin c_0$.
Тогда
\begin{equation}
	\varlimsup_{k\to\infty} x_k \neq 0 ~~\mbox{или}~~ \varliminf_{k\to\infty} x_k \neq 0
	.
\end{equation}

Не теряя общности, положим
\begin{equation}
	\varepsilon = \varlimsup_{k\to\infty} x_k > 0
\end{equation}
(иначе домножим всю последовательность на $-1$, что, очевидно, не повлияет на сходимость к нулю).

Тогда существует бесконечно много таких $n$, что
\begin{equation}\label{alpha_ac0_c0_limsup}
	x_n > \varlimsup_{k\to\infty} x_k - \frac{\varepsilon}{4} = \frac{3\varepsilon}{4}
	.
\end{equation}

Так как
\begin{equation}
	\alpha(x) = \varlimsup_{i\to\infty} \max_{i \leq j \leq 2i} |x_i-x_j| = 0
	,
\end{equation}
то
\begin{equation}
	\exists(N_1\in\mathbb{N})\forall(n > N_1)\left[\max_{n \leq j \leq 2n} |x_n-x_j| < \frac{\varepsilon}{4}\right]
	,
\end{equation}
или, что то же самое,
\begin{equation}\label{alpha_ac0_c0_alpha}
	\exists(N_1\in\mathbb{N})\forall(n > N_1)\forall(j: n \leq j \leq 2n)\left[ |x_n-x_j| < \frac{\varepsilon}{4}\right]
	.
\end{equation}

Поскольку $x \in ac_0$, то  по критерию Лоренца
\begin{equation}
	\exists(N_2 \in\mathbb{N})\forall(n > N_2)\forall(m\in\mathbb{N})
	\left[ \left| \frac{1}{n}\sum_{k=m+1}^{m+n}x_k\right| < \frac{\varepsilon}{4} \right]
	,
\end{equation}
в частности,
\begin{equation}\label{alpha_ac0_c0_Lorencz}
	\exists(N_2 \in\mathbb{N})\forall(n > N_2)
	\left[ \left| \frac{1}{n}\sum_{k=n+1}^{2n}x_k\right| < \frac{\varepsilon}{4} \right]
	,
\end{equation}

Выберем $n$ так, чтобы оно удовлетворяло \eqref{alpha_ac0_c0_limsup} и \eqref{alpha_ac0_c0_alpha}.
Тогда
\begin{multline}
	\left| \frac{1}{n}\sum_{k=n+1}^{2n}x_k\right|
	=
	\\=
	\mbox{(по \eqref{alpha_ac0_c0_alpha} имеем $x_k \geq 3\varepsilon/4 > 0$)}
	=
	\\=
	\frac{1}{n}\sum_{k=n+1}^{2n}x_k
	\geq
	\frac{3\varepsilon}{4}
	>
	\frac{\varepsilon}{4}
	,
\end{multline}
что противоречит \eqref{alpha_ac0_c0_Lorencz}.

Полученное противоречие завершает доказательство.



\paragraph{Следствие.}
Пусть $x\in ac$ и $\alpha(x)=0$.
Тогда $x \in c$.

TODO: доказывать нужно или очевидно?


	\section{$\alpha$--функция на пространстве $ac$ и расстояние до пространства $c$}
	Пусть $\rho(x,c)$ и $\rho(x,c_0)$~--- расстояния от $x$ до пространства сходящихся последовательностей $c$
и пространства сходящихся к нулю последовательностей $c_0$ соответственно.

Из условия Липшица на $\alpha$--функцию \eqref{alpha_Lipshitz}
и того, что на пространстве $c$
всех сходящихся последовательностей
$\alpha$--функция обращается в нуль следует

\begin{lemma}
\label{thm:alpha_x_leq_2_rho_x_c}
	Для любого $x\in\ell_\infty$
	\begin{equation}
		\alpha(x) \leq 2\rho(x, c)
		.
	\end{equation}
\end{lemma}

%TODO2: доказывать или очевидно?

Эта оценка точна.
\begin{example}
\label{ex:alpha_ac_rho_x_c}
	\begin{equation}
	\label{eq:alpha_ac_distance_example_y}
		x_k = \begin{cases}
			(-1)^n, &\mbox{~если~} k = 2^n,
			\\
			0 &\mbox{~иначе.}
		\end{cases}
	\end{equation}
\end{example}
Здесь $\alpha(x) = 2$, $\alpha(T^s x) = 1$ для любого $s\in\N$, $\rho(x,c) = \rho(x, c_0) = 1$.

Как выясняется, верна и оценка с другой стороны.

\begin{lemma}
\label{thm:rho_x_c_leq_alpha_t_s_x}
	Для любого $x\in ac$ и для любого натурального $s$
	\begin{equation}
		\rho(x,c)\leq \alpha(T^s x)
		.
	\end{equation}
\end{lemma}

\begin{proof}
	Зафиксируем $s$.
	Пусть $x\in ac$, $\alpha(T^s x)=\varepsilon$.
	Так как $x\in ac$, то по критерию Лоренца существует такое число $t$,
	что
	\begin{equation}
		\lim_{n\to\infty} \frac{1}{n} \sum_{k=m+1}^{m+n} x_k = t
	\end{equation}
	равномерно по $m$.
	Иначе говоря,
	\begin{equation}
		\forall(p\in\N)
		\exists(n_p \in\N)\forall(n>n_p)\forall(m\in\N)
		\left[
			\left|
				\frac{1}{n}\sum_{k=m+1}^{m+n} x_k
				-t
			\right|
			<\frac{\varepsilon}{p}
		\right]
		.
	\end{equation}

	Так как
	\begin{equation}
		\alpha(T^s x) = \varlimsup_{k\to\infty} \max_{k<j\leq 2k-s} |x_k-x_j| = \varepsilon
		,
	\end{equation}
	то
	\begin{equation}
		\forall(p\in\N)
		\exists(k_p \in\N)\forall(k>k_p)
		\forall(j: k< j \leq 2k-s)
		\left[
			|x_k - x_j|<\varepsilon + \frac{\varepsilon}{p}
		\right]
		.
	\end{equation}
	Положив $q_p = \max\{n_p, k_p\}$, имеем
	\begin{multline}
		\forall(p\in\N)
		\exists(q_p \in\N)
		\forall(q>q_p)
		\left[
			\left|
				\frac{1}{n}\sum_{k=m+1}^{m+n} x_k
				-t
			\right|
			<\frac{\varepsilon}{p}
			\right.\\ \left. \phantom{\sum_0^0}
			\mbox{~~и~~}
			\forall(k:q<k \leq 2q-s)
			\left[
				|x_q - x_k|<\varepsilon + \frac{\varepsilon}{p}
			\right]
		\right]
		.
	\end{multline}
	Т.е. среднее арифметическое чисел $x_q$, $x_{q+1}$, ... , $x_{2q-s}$ отличается от $t$
	не более, чем на $\varepsilon/p$, причём разница любых двух из этих чисел меньше $\varepsilon + \varepsilon/p$.
	Следовательно, любое из чисел $x_q$, $x_{q+1}$, ... , $x_{2q-s}$
	отличается от $t$ менее, чем на $\varepsilon + 2\varepsilon/p$.
	В частности,
	\begin{equation}
		|x_q - t| < \varepsilon + \frac{2\varepsilon}{p}
		.
	\end{equation}
	Таким образом,
	\begin{equation}
		\forall(p\in\N)
		\exists(q_p \in\N)
		\forall(q>q_p)
		\left[
			|x_q - t| < \varepsilon + \frac{2\varepsilon}{p}
		\right]
		,
	\end{equation}
	откуда немедленно следует, что $\rho(x,c) \leq \varepsilon$,
	что и требовалось доказать.
\end{proof}


Из лемм \ref{thm:alpha_x_leq_2_rho_x_c} и \ref{thm:rho_x_c_leq_alpha_t_s_x}
незамедлительно следует
\begin{theorem}
	\label{thm:rho_x_c_leq_alpha_t_s_x_united}
	Для любого $x\in ac$
	\begin{equation}
		\frac{1}{2} \alpha(x) \leq \rho(x,c)\leq \lim_{s\to\infty} \alpha(T^s x)
		.
	\end{equation}
\end{theorem}

\begin{corollary}
	\label{cor:rho_x_c0_leq_alpha_t_s_x_united}
	Для любого $x\in ac_0$
	\begin{equation}
		\frac{1}{2} \alpha(x) \leq \rho(x,c_0)\leq \lim_{s\to\infty} \alpha(T^s x)
		.
	\end{equation}
\end{corollary}

Точность оценок показывают пример \ref{ex:alpha_ac_rho_x_c} и следующие примеры.
\begin{example}
	\begin{equation}
		x_k = \begin{cases}
			1, &\mbox{~если~} k = 2^n,
			\\
			0 &\mbox{~иначе.}
		\end{cases}
	\end{equation}
\end{example}
Здесь $\alpha(T^s x) = 1$ для любого $s\in\N$, $\rho(x,c) = 1/2$, $\rho(x, c_0) = 1$.

\begin{example}
	\begin{equation}
		x_k = \begin{cases}
			1, &\mbox{~если~} k = 2^n,
			\\
			-1, &\mbox{~если~} k = 2^n + 1 > 2
			\\
			0 &\mbox{~иначе.}
		\end{cases}
	\end{equation}
\end{example}
Здесь $\alpha(T^s x) = 2$ для любого $s\in\N$, $\rho(x,c) = \rho(x, c_0) = 1$.


\begin{hypothesis}
	Для любого $x\in ac$
	\begin{equation}
		\frac{1}{2} \lim_{s\to\infty} \alpha(U^s x) \leq \rho(x,c)
		,
	\end{equation}
	где $U$~--- оператор сдвига вправо:
	\begin{equation}
		U(x_1, x_2, x_3, ...) = (0, x_1, x_2, x_3, ...)
		.
	\end{equation}
\end{hypothesis}


\chapter{Банаховы пределы, инвариантные относительно операторов}

	\section{Оператор с конечномерным ядром, для которого не существует инвариантного банахова предела}
	В работе \cite{Semenov2010invariant} изучаются условия,
при которых для оператора $H:\ell_\infty\to \ell_\infty$ существует банаховы пределы,
инвариантные относительно данного оператора, то есть такие $B\in\mathfrak{B}$,
что $B(Hx) = Bx$ для любого $x\in\ell_\infty$,
а также приводятся примеры операторов, для которых инвариантные банаховы пределы существуют:
операторы $\sigma_n$ и оператор Чезаро $C$.

Заметим, что операторы $\sigma_n$ и $C$ имеют вырожденное ядро,
однако не для любого оператора с вырожденным ядром существует инвариантный банахов предел.

\begin{example}
	Пусть для $x = (x_1, x_2, ..., x_n, ...)\in \ell_\infty$
	\begin{equation*}
		Ax = (x_1, 0, x_2, 0, x_3, 0, x_4, 0, ...).
	\end{equation*}
	Очевидно, что $\ker A = \{0\}$.
\end{example}

Пусть $B\in\mathfrak{B}$, $BA = B$.
Очевидно, что
\begin{equation*}
	\frac{n-1}{2}\leqslant \sum_{k=m+1}^{m+n} (A\one)_k \leqslant \frac{n+1}{2},
\end{equation*}
где $\one = (1, 1, 1, 1, 1, 1, ...)$.
Тогда по теореме Лоренца
\begin{equation*}
	BA\one =
	\lim_{n\to\infty} \frac{1}{n}\sum_{k=m+1}^{m+n} (A\one)_k = \frac{1}{2}
\end{equation*}
Однако по определению банахова предела
\begin{equation*}
	B\one = 1 \neq \frac{1}{2} = BA\one.
\end{equation*}
Пришли к противоречию, следовательно, банаховых пределов, инвариантных относительно $A$, не существует.


	\section{Оператор с бесконечномерным ядром, относительно которого инвариантен любой банахов предел}
	Конечномерность ядра оператора не является необходимым условием существования
банахова предела, инвариантного относительно данного оператора.

Приведём сначала несложную лемму, а потом соответствующий пример.

\paragraph{Лемма.}
%TODO: а не баян ли?
%\textit{Скорее всего, это настолько очевидно, что это никто не пишет и не доказывает.}

Пусть $Q:l_\infty \to l_\infty$.
\begin{equation}
	\forall(B\in\mathfrak{B})[QB = B]
\end{equation}
тогда и только тогда, когда
\begin{equation}\label{I-Q_to_ac_0}
	I-Q : l_\infty \to ac_0,
\end{equation}
где $I$~--- тождественный оператор на $l_\infty$.

\paragraph{Необходимость.}
\begin{equation}
	B((I-Q)x) =
	B(Ix) - B(Qx) =
	Bx - B(Qx)=
	Bx-Bx
	=
	0
	,
\end{equation}
откуда в силу произвольности выбора $B$ и $x$ вкупе с тем, что $B((I-Q)x)=0$,
имеем $(I-Q)x \in ac_0$ и, следовательно, $I-Q : \ell_\infty \to ac_0$.

\paragraph{Достаточность.}
\begin{equation}
	B(Qx) = B((I-(I-Q))x) =
	B(Ix)-B((I-Q)x) =
	B(x) - 0 = B(x).
\end{equation}


\paragraph{Пример.}

\begin{equation}
	(Qx)_k =
	\begin{cases}
		0,~\mbox{если}~ k = 2^n, n \in\mathbb{N},
		\\
		x_k~\mbox{иначе.}
	\end{cases}
\end{equation}

Очевидно, что $\dim \ker Q = \infty$.

Покажем, что $Q$ удовлетворяет условию (\ref{I-Q_to_ac_0}).
\begin{equation}
	\sum_{k=m+1}^{m+n} x_k - \sum_{k=m+1}^{m+n} (Qx)_k \leqslant (2 + \log_2 n) \|x\|,
\end{equation}
(это очевидно??)\\
откуда немедленно
\begin{equation}
	\frac{1}{n}\sum_{k=m+1}^{m+n} x_k - \frac{1}{n}\sum_{k=m+1}^{m+n} (Qx)_k \leqslant \frac{(2 + \log_2 n) \|x\|}{n} \to 0.
\end{equation}

По предыдущему утверждению отсюда следует, что относительно $Q$ инвариантен любой банахов предел.


	\section{О классах линейных операторов, для которых множества инвариантных банаховых пределов совпадают}
	\begin{lemma}
	Пусть $P,Q:\ell_\infty \to \ell_\infty$~--- линейные операторы и
	\begin{equation}\label{P-Q:ell_infty_to_ac0}
		P-Q : \ell_\infty \to ac_0
		.
	\end{equation}
	Тогда
	\begin{equation}
		\mathfrak{B}(P)=\mathfrak{B}(Q)
		.
	\end{equation}
\end{lemma}

\paragraph{Доказательство.}
Пусть $B\in \mathfrak{B}(P)$.
Тогда
\begin{equation}
	B(Qx) = B((Q-P+P)x) =
	B((Q-P)x)+B(Px) =
	0 + B(Px) =
	Bx
	.
\end{equation}
Значит, $\mathfrak{B}(P) \subset \mathfrak{B}(Q)$.
В силу симметричности утверждения леммы получаем $\mathfrak{B}(Q) \subset \mathfrak{B}(P)$,
откуда и следует требуемое утверждение.

Только что доказанная лемма означает, что множество всех линейных операторов,
действующих из $\ell_\infty$ в $\ell_\infty$, можно разбить на классы по отношению эквивалентности,
задаваемому условием \eqref{P-Q:ell_infty_to_ac0},
и тогда для операторов из одного класса множества инвариантных банаховых пределов будут совпадать.

Обратное, однако, неверно.

\begin{example}
	Пусть
	\begin{equation}
		P(x_1,x_2,x_3) = (x_1, 0, x_3, 0, x_5, 0, ...)
	\end{equation}
	и
	\begin{equation}
		Q(x_1,x_2,x_3) = (0, -x_2, 0, -x_4, 0, -x_6, 0, ...)
		.
	\end{equation}
	Легко видеть, что для любого банахова предела $B\in\mathfrak{B}$ выполнено
	\begin{equation}
		B(P\mathbb{I}) = \frac{1}{2}
	\end{equation}
	и
	\begin{equation}
		B(Q\mathbb{I}) = -\frac{1}{2}
		,
	\end{equation}
	откуда
	\begin{equation}
		\mathfrak{B}(Q) = \mathfrak{B}(P) = \varnothing
		.
	\end{equation}
	Однако разность $P-Q$ есть не что иное, как тождественный оператор $I$,
	который переводит пространство $\ell_\infty$ само в себя,
	а не в пространсво $ac_0$.
\end{example}


\begin{hypothesis}
	Существуют два таких линейных оператора $P, Q : \ell_\infty \to \ell_\infty$,
	что $\mathfrak{B}(P) = \mathfrak{B}(Q) \neq \varnothing$,
	но $(P-Q)(\ell_\infty) \setminus ac \neq \varnothing$.
\end{hypothesis}



\chapter*{Список сформулированных гипотез}
\addcontentsline{toc}{chapter}{Список сформулированных гипотез}

\renewcommand\label[1]{}
\hypotlist


\addcontentsline{toc}{chapter}{Список литературы}
\printbibliography{}

\end{document}
