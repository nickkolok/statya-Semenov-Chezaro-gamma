Из свойства~\ref{thm:alpha_x_triangle_ineq} и однородности $\alpha$--функции немедленно вытекает
\begin{theorem}
	Множество $\alpha c = \{x: \alpha(x) = 0\}$
	является пространством.
\end{theorem}
Изучим свойства этого пространства.
\begin{property}
	Пространство $\alpha c$ замкнуто.
\end{property}
\begin{proof}
	Прообраз замкнутого множества $\{0\}$
	при непрерывном отображении $\alpha : \ell_\infty \to \mathbb{R}$
	замкнут.
\end{proof}
\begin{theorem}
	Включение $c \subset \alpha c$ собственное.
\end{theorem}
\begin{proof}
	Рассмотрим
	\begin{equation}
		x=\left(
			0,1,
			0,\frac{1}{2},1,\frac{1}{2},
			0,\frac{1}{3},\frac{2}{3},1,\frac{2}{3},\frac{1}{3},
			0,
			...
		\right)
		.
	\end{equation}
	Тогда по лемме~\ref{thm:alpha_S} (при определённом тем же образом операторе $S$) имеем
	\begin{equation}
		\alpha(Sx) = \varlimsup_{k\to\infty} |x_{k+1} - x_{k}| = 0
		,
	\end{equation}
	однако, очевидно, $Sx\notin c$.
\end{proof}
Очевидно следующее
\begin{property}
	Если периодическая последовательность принадлежит $\alpha c$,
	то эта последовательность~--- константа.
\end{property}

Из результатов предыдущих пунктов вытекает
\begin{theorem}
	Пространство $\alpha c$ замкнуто относительно операторов $T$, $U$, $C$, $\sigma_n$, $\sigma_{1/n}$, $S$.
\end{theorem}

Из теоремы~\ref{thm:alpha_xy} следует
\begin{theorem}
	Пространство $\alpha c$ замкнуто относительно умножения,
	т.е. если $x,y\in\alpha c$, то $x\cdot y \in \alpha c$.
\end{theorem}

Пример~\ref{ex:alpha-c_not_ideal} показывает, что $\alpha c$
не является идеалом по умножению.

Из теоремы~\ref{thm:rho_x_c_leq_alpha_t_s_x_united} (см. ниже) следует
\begin{theorem}
	$ac \cap\alpha c = c$.
\end{theorem}

Некоторый интерес представляет следующая теормема,
основанная на идеях~\cite{usachev2009_phd_vsu}.

\begin{theorem}
	$\alpha c$ несепарабельно.
\end{theorem}

\begin{proof}
	Пусть $\Omega = \{0;1\}^{\mathbb{N}}$.
	Для каждого $\omega\in\Omega$ положим
	\begin{multline}
		\label{eq:x_omega_alpha_c}
		x(\omega)=\left(
			0, 1\omega_1,
			0, \frac{1}{2}\omega_2, 1\omega_2, \frac{1}{2}\omega_2,
			0, \frac{1}{3}\omega_3, \frac{2}{3}\omega_3, 1\omega_3, \frac{2}{3}\omega_3, \frac{1}{3}\omega_3,
			0, ...,
		\right. \\ \left.
			0, \frac{1}{p}\omega_p, \frac{2}{p}\omega_p, ..., \frac{p-1}{p}\omega_p, 1\omega_p,
				\frac{p-1}{p}\omega_p, ..., \frac{2}{p}\omega_p, \frac{1}{p}\omega_p,
			0, \frac{1}{p+1}\omega_{p+1}, ...
		\right).
	\end{multline}
	Тогда по лемме~\ref{thm:alpha_S} (при определённом тем же образом операторе $S$) имеем
	$\alpha(Sx(\omega)) = 0$.
	Заметим, что при $\omega,\omega^* \in \Omega$ и $\omega\neq\omega^*$ выполнено
	$\|Sx(\omega)-Sx(\omega^*)\|=1$ и $|Sx(\omega)|=\mathfrak{c}$.
	Следовательно, $\alpha c$ несепарабельно.
\end{proof}

%\begin{theorem}
%	$|\alpha c| = |\ell_\infty|$.
%\end{theorem}

%\begin{proof}
%	Достаточно заметить, что
%\end{proof}
