\begin{theorem}
	Пусть $M(j)$~--- последовательность, определённая в~\eqref{eq:definition_M_j}.
	Тогда последовательность $M(j)/j$ имеет предел (конечный или бесконечный).
\end{theorem}

\paragraph{Доказательство.}
Предположим противное.
Пусть
\begin{equation}
	\varliminf_{j\to\infty}\frac{M(j)}{j} = A,
\end{equation}
\begin{equation}
	\varlimsup_{j\to\infty}\frac{M(j)}{j} \geq B
\end{equation}
(неравенство используется для охвата случая, когда верхний предел равен бесконечности)
и
\begin{equation}
	B > A
	.
\end{equation}

Положим $\varepsilon = \frac{B-A}{4}$.
По определению верхнего предела существует такое натуральное $j_0$,
что
\begin{equation}
	\frac{M(j_0)}{j_0} > B - \varepsilon
	.
\end{equation}
По определению нижнего предела существует $\{j_i\}$~--- возрастающая последовательность таких индексов, что
\begin{equation}
	\label{eq:lim_Mj_inf_lim_seq}
	\forall(i\in\mathbb{N})\left[ \frac{M(j_i)}{j_i} < A + \varepsilon \right]
	.
\end{equation}
Определим для каждого $i\in\mathbb{N}$ целое число $C_i$ такое, что
\begin{equation}
	\label{eq:lim_Mj_inf_lim_Ci}
	C_i j_0 \leq j_i < (C_i+1)j_0
	.
\end{equation}
Легко заметить, что последовательность $\{C_i\}$ является возрастающей.

Очевидно, что
\begin{equation}
	M(Cj) \geq C \cdot M(j)
	.
\end{equation}
Таким образом,
\begin{equation}
	M(j_i) \geq M(C_i j_0) \geq C_i M(j_0) > C_i j_0 (B-\varepsilon)
\end{equation}
и
\begin{multline}
	\frac{M(j_i)}{j_i} > \frac{C_i j_0 (B-\varepsilon)}{j_i}
	> \frac{C_i j_0 (B-\varepsilon)}{(C_i+1)j_0}
	=
	\\=
	\frac{C_i}{C_i+1}(B-\varepsilon)
	= B-\varepsilon - \frac{1}{C_i+1}(B-\varepsilon)
	.
\end{multline}
Выберем $i$ настолько большим, что
\begin{equation}
	\frac{1}{C_i+1}(B-\varepsilon) < \varepsilon
	.
\end{equation}
Тогда
\begin{equation}
	\frac{M(j_i)}{j_i} > B - 2 \varepsilon = A + 2 \varepsilon > A + \varepsilon
	,
\end{equation}
что противоречит~\eqref{eq:lim_Mj_inf_lim_seq}.
Полученное противоречие завершает доказательство.

\begin{hypothesis}
	Пусть $M(j)$~--- последовательность натуральных чисел, такая, что существует предел
	\begin{equation}
		\lim_{j\to\infty} \frac{M(j)}{j} \leq +\infty
	\end{equation}
	и для любых $i,j\in\mathbb{N}$ выполнено
	\begin{equation}
		M(i)+M(j) \leq M(i+j)
		.
	\end{equation}
	Тогда существует последовательность $\{x_k\}$,
	удовлетворяющая условию~\eqref{eq:definition_M_j}.
\end{hypothesis}
