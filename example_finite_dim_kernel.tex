В работе \cite{Semenov2010invariant} изучаются условия,
при которых для оператора $H:\ell_\infty\to \ell_\infty$ существует банаховы пределы,
инвариантные относительно данного оператора, то есть такие $B\in\mathfrak{B}$,
что $B(Hx) = Bx$ для любого $x\in\ell_\infty$,
а также приводятся примеры операторов, для которых инвариантные банаховы пределы существуют:
операторы $\sigma_n$ и оператор Чезаро $C$.

Заметим, что операторы $\sigma_n$ и $C$ имеют вырожденное ядро,
однако не для любого оператора с вырожденным ядром существует инвариантный банахов предел.

\textbf{Пример.}
Пусть для $x = (x_1, x_2, ..., x_n, ...)\in \ell_\infty$
\begin{equation*}
	Ax = (x_1, 0, x_2, 0, x_3, 0, x_4, 0, ...).
\end{equation*}
Очевидно, что $\ker A = \{0\}$.

Пусть $B\in\mathfrak{B}$, $BA = B$.
Очевидно, что
\begin{equation*}
	\frac{n-1}{2}\leqslant \sum_{k=m+1}^{m+n} (A\mathbb{I})_k \leqslant \frac{n+1}{2},
\end{equation*}
где $\mathbb{I} = (1, 1, 1, 1, 1, 1, ...)$.
Тогда по теореме Лоренца
\begin{equation*}
	BA\mathbb{I} =
	\lim_{n\to\infty} \frac{1}{n}\sum_{k=m+1}^{m+n} (A\mathbb{I})_k = \frac{1}{2}
\end{equation*}
Однако по определению банахова предела
\begin{equation*}
	B\mathbb{I} = 1 \neq \frac{1}{2} = BA\mathbb{I}.
\end{equation*}
Пришли к противоречию, следовательно, банаховых пределов, инвариантных относительно $A$, не существует.
