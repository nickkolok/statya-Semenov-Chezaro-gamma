Введём на пространстве $\ell_\infty$ фактор-норму по $c_0$:

TODO: надо ссылку на классиков?

\begin{equation}
	\|x\|_* = \limsup_{k\to\infty} |x_k|
	.
\end{equation}

\begin{theorem}
	\label{thm:alpha_xy}
	Пусть $(x\cdot y)_k = x_k\cdot y_k$.
	Тогда
	$\alpha(x\cdot y)\leq \alpha(x)\cdot \|y\|_* + \alpha(y)\cdot \|x\|_*$.
\end{theorem}

\begin{proof}
	\begin{multline}
		\alpha(x\cdot y)
		=
		\limsup_{i\to\infty} \max_{i\leq j \leq 2i} |x_i y_i - x_j y_j|
		=
		\limsup_{i\to\infty} \max_{i\leq j \leq 2i} |x_i y_i - x_j y_i + x_j y_i - x_j y_j|
		\leq
		\\ \leq
		\limsup_{i\to\infty} \max_{i\leq j \leq 2i} \left(|x_i y_i - x_j y_i| + |x_j y_i - x_j y_j| \right)
		=
		\\=
		\limsup_{i\to\infty} \max_{i\leq j \leq 2i} \left(|y_i|\cdot|x_i - x_j| + |x_j|\cdot|y_i - y_j| \right)
		\leq
		\\ \leq
		\limsup_{i\to\infty} \max_{i\leq j \leq 2i} |y_i|\cdot|x_i - x_j| + \limsup_{i\to\infty} \max_{i\leq j \leq 2i}|x_j|\cdot|y_i - y_j|
		=
		\\ =
		\limsup_{i\to\infty} |y_i| \max_{i\leq j \leq 2i} \cdot|x_i - x_j| + \limsup_{i\to\infty} \max_{i\leq j \leq 2i}|x_j|\cdot|y_i - y_j|
		\leq
		\\ \leq
		\limsup_{i\to\infty} |y_i| \cdot \limsup_{i\to\infty} \max_{i\leq j \leq 2i} \cdot|x_i - x_j| + \limsup_{i\to\infty} \max_{i\leq j \leq 2i}|x_j|\cdot|y_i - y_j|
		=
		\\ =
		\|y\|_* \cdot \limsup_{i\to\infty} \max_{i\leq j \leq 2i} \cdot|x_i - x_j| + \limsup_{i\to\infty} \max_{i\leq j \leq 2i}|x_j|\cdot|y_i - y_j|
		=
		\\ =
		\|y\|_* \cdot \alpha(x) + \limsup_{i\to\infty} \max_{i\leq j \leq 2i}|x_j|\cdot|y_i - y_j|
		\leq
		\\ \leq
		\|y\|_* \cdot \alpha(x) + \limsup_{i\to\infty} \max_{i\leq j \leq 2i}|x_j|\cdot\limsup_{i\to\infty} \max_{i\leq j \leq 2i}|y_i - y_j|
		=
		\\ =
		\|y\|_* \cdot \alpha(x) + \limsup_{i\to\infty} \max_{i\leq j \leq 2i}|x_j|\cdot \alpha(y)
		\leq
		\\ \leq
		\|y\|_* \cdot \alpha(x) + \limsup_{j\to\infty} |x_j|\cdot \alpha(y)
		=
		%\\ =
		\|y\|_* \cdot \alpha(x) + \|x\|_* \cdot \alpha(y)
		.
	\end{multline}
\end{proof}

\begin{example}
	Покажем, что нижнюю оценку на $\alpha(x\cdot y)$ дать нельзя.
	В самом деле,
	пусть $x_k = (-1)^k$.
	Тогда $\alpha(x) = 2$, $\|x\|_* = 1$,
	но $\alpha(x\cdot x) = 0$.
\end{example}

\begin{example}
	\label{ex:alpha-c_not_ideal}
	Покажем, что оценка теоремы~\ref{thm:alpha_xy} точна в том смысле,
	что равенство достижимо.
	В самом деле,
	пусть $x_k = (-1)^k$, $y_k = 1$.
	Тогда $\alpha(x) = 2$, $\alpha(y) = 0$,
	но $\alpha(x\cdot y) = 2$.
\end{example}

\begin{example}
	Пусть
	\begin{equation}
		y = \left(
			0, \frac{1}{p}, \frac{2}{p}, \frac{3}{p},
			...,
			\frac{p-1}{p}, 1, \frac{p-1}{p},
			...,
			\frac{1}{p},
			0,
			\frac{1}{p}, \frac{2}{p}, ...
		\right)
	\end{equation}
	Тогда $\alpha(Sy) = \frac{1}{p}$, $\|Sy\|_* = 1$,
	\begin{equation}
		\alpha((Sy)\cdot(Sy)) =
		\left| \frac{(p-1)^2}{p^2} - 1 \right| =
		\frac{2}{p}-\frac{1}{p^2}
		.
	\end{equation}
\end{example}

\begin{example}
	Для того же $y$ и любого $k>0$ имеем
	$\alpha(kSy) = \frac{k}{p}$, $\|kSy\|_* = k$,
	\begin{equation}
		\alpha((kSy)\cdot(kSy)) =
		\left| \frac{k^2(p-1)^2}{p^2} - k^2 \right| =
		\frac{2k^2}{p}-\frac{k^2}{p^2}
		.
	\end{equation}
	Оценка теоремы~\ref{thm:alpha_xy} даёт
	\begin{equation}
		\alpha((kSy)\cdot(kSy)) \leq
		\frac{2k^2}{p}
		.
	\end{equation}
\end{example}


% Увы, неверна - контрпример выше, k=2
%\begin{hypothesis}
%	$\alpha(x\cdot y)\leq \alpha(x)\cdot \|y\|_* + \alpha(y)\cdot \|x\|_* - \alpha(x)\alpha(y) \cdot \|y\|_* \cdot \|x\|_*$.
%\end{hypothesis}
% Убрать фактор-нормы тоже нельзя:

\begin{example}
	Пусть $x_k = (-1)^k$.
	Тогда $\alpha(x) = \alpha(\sigma_2 x) = 2$, $\|x\|_* = \|\sigma_2 x\|_* = 1$,
	но $\alpha(x\cdot \sigma_2 x) = 2 < 4$.
\end{example}

Исходя из построенных примеров может быть выдвинута
\begin{hypothesis}
	Если $\alpha(x\cdot y)= \alpha(x)\cdot \|y\|_* + \alpha(y)\cdot \|x\|_*$,
	то $\alpha(x) = 0$ или $\alpha(y) = 0$.
\end{hypothesis}
