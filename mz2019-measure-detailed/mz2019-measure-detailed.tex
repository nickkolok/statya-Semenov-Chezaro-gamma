\documentclass[a4paper,14pt]{article} %размер бумаги устанавливаем А4, шрифт 12пунктов
\usepackage[T2A]{fontenc}
\usepackage[utf8]{inputenc}
\usepackage[english,russian]{babel} %используем русский и английский языки с переносами
\usepackage{amssymb,amsfonts,amsmath,mathtext,enumerate,float,amsthm} %подключаем нужные пакеты расширений
\usepackage[unicode,colorlinks=true,citecolor=black,linkcolor=black]{hyperref}
%\usepackage[pdftex,unicode,colorlinks=true,linkcolor=blue]{hyperref}
\usepackage{indentfirst} % включить отступ у первого абзаца
\usepackage[dvips]{graphicx} %хотим вставлять рисунки?
\graphicspath{{illustr/}}%путь к рисункам

\makeatletter
\renewcommand{\@biblabel}[1]{#1.} % Заменяем библиографию с квадратных скобок на точку:
\makeatother %Смысл этих трёх строчек мне непонятен, но поверим "Запискам дебианщика"

\usepackage{geometry} % Меняем поля страницы.
\geometry{left=2cm}% левое поле
\geometry{right=1cm}% правое поле
\geometry{top=2cm}% верхнее поле
\geometry{bottom=2cm}% нижнее поле

\renewcommand{\theenumi}{\arabic{enumi}.}% Меняем везде перечисления на цифра.цифра
\renewcommand{\labelenumi}{\arabic{enumi}.}% Меняем везде перечисления на цифра.цифра
\renewcommand{\theenumii}{\arabic{enumii}}% Меняем везде перечисления на цифра.цифра
\renewcommand{\labelenumii}{\arabic{enumi}.\arabic{enumii}.}% Меняем везде перечисления на цифра.цифра
\renewcommand{\theenumiii}{\arabic{enumiii}}% Меняем везде перечисления на цифра.цифра
\renewcommand{\labelenumiii}{\arabic{enumi}.\arabic{enumii}.\arabic{enumiii}.}% Меняем везде перечисления на цифра.цифра

\sloppy




\renewcommand\normalsize{\fontsize{14}{25.2pt}\selectfont}

\usepackage[backend=biber,style=gost-numeric,sorting=none]{biblatex}
\addbibresource{../bib/ext.bib}
\addbibresource{../bib/my.bib}
\addbibresource{../bib/Semenov.bib}
\input{../bib/ext.hyphens.bib}


\theoremstyle{plain}
\newtheorem{lemma}{Лемма}%[section]
\newtheorem{theorem}[lemma]{Теорема}
\newtheorem{example}[lemma]{Пример}
\newtheorem{property}[lemma]{Свойство}
\newtheorem{remark}[lemma]{Замечание}
\newtheorem{corollary}{Следствие}[lemma]
\newtheorem{hypothesis}[lemma]{Гипотеза}

\usepackage{bbm}


\begin{document}

% The text starts here

В \cite{our-mz2019measure} была анонсирована следующая теорема.
\begin{theorem}[{\cite[Теорема 2]{our-mz2019measure}}]
	\label{thm:mes_W}
	Пусть $W$~--- множество всех последовательностей $\chi_e$, где $e =\bigcup_{k=1}^{\infty} [n_{2k-1}, n_{2k} )$
	и $\{n_k \}_{k=1}^{\infty}$
	удовлетворяет условию
	\begin{equation}
		\label{eq:lim_j_n_kj_measure}
		\lim_{j\to\infty}\frac{n_{k+j} - n_k}{j} = \infty
	\end{equation}
	равномерно по $k \in \mathbb{N}$.
	Тогда $\operatorname{mes} W = 0$.
\end{theorem}

Для доказательства теоремы~\ref{thm:mes_W} нам потребуется ряд вспомогательных построений.


Введём нелинейную биекцию $Q:\Omega\leftrightarrow\Omega$ по следующему правилу:
\begin{equation}
	(Qx)_k = \begin{cases}
		x_k, &\mbox{если~} k = 1,
		\\
		|x_k-x_{k-1}|&\mbox{иначе}.
	\end{cases}
\end{equation}

\begin{remark}
	Может показаться, что оператор $Q$ очень сходен с оператором границы последовательности $\operatorname{bd}$
	~\cite{keller1992invariant},
	однако это не так.
\end{remark}

\begin{example}
	\begin{equation}
		\begin{array}{rl}
			                   x &= (0,0,0,1,1,1,0,0,0,0,0,1,1,1,1,1,0,...)
			\\
			\operatorname{bd}(x) &= (0,0,0,1,0,1,0,0,0,0,0,1,0,0,0,1,0,...)
			\\
			                  Qx &= (0,0,0,1,0,0,1,0,0,0,0,1,0,0,0,0,1,...)
		\end{array}
	\end{equation}
\end{example}

\begin{example}
	Ситуация кардинально меняется, когда в последовательности достаточно часто встречаются блоки $(...,0,1,0,...)$
	(а по закону больших чисел они будут встречаться с большой вероятностью).
	\begin{equation}
		\begin{array}{rl}
			                   x &= (0,0,1,1,0,1,0,1,0,0,0,...)
			\\
			\operatorname{bd}(x) &= (0,0,1,1,0,1,0,1,0,0,0,...)
			\\
			                  Qx &= (0,0,1,0,1,1,1,1,1,0,0,...)
		\end{array}
	\end{equation}
\end{example}


\begin{lemma}
	Биекция $Q$ сохраняет меру множества.
\end{lemma}

\begin{proof}
	В рамках данного доказательства будем соотносить последовательности из нулей и единиц
	с точками полуинтервала $[0;1)$, представленными в виде двоичных дробей.
	Это соответствие почти биективно.
	Назовём двоичным отрезком множество последовательностей из $\Omega$,
	в котором первые $k$ координат зафиксированы, а остальные выбираются произвольно.
	Двоичный отрезок действительно соответствует отрезку длины $1/2^k$,
	состоящему из двоичных дробей, в которых первые $k$ цифр после запятой зафиксированы,
	а остальные выбираются произвольно.

	Заметим, что $Q$ отображает двоичный отрезок длины $1/2^k$ в двоичный отрезок длины $1/2^k$.
	Так как любой отрезок может быть представлен в виде объединения не более чем счётного числа
	двоичных отрезков, то $Q$ сохраняет меру любого отрезка.
	В силу биективности $Q$ отсюда следует, что $Q$ сохраняет меру любого борелевского множества
	(так как сигма-алгебра борелевских множеств порождается отрезками).
\end{proof}

Второй вариант доказательства:
\begin{proof}
	Доказательство из~\cite{connor1990almost} использует наименьшую $\sigma$--алгебру $\mathcal{S}$,
	содержащую множества
	\begin{equation}
		\Omega_i = \{x\in\Omega: x_i = 1\}
		.
	\end{equation}
	Очевидно, что $\mathcal{S}$~--- это также и наименьшая $\sigma$--алгебра, содержащая множества
	$\Omega^*_{i,j}$ последовательностей, в которых зафиксированы первые $i$ координат.
	Заметим, что $\mu(\Omega^*_{i,j}) = \mu(Q\Omega^*_{i,j})$.
	Так как $Q$~--- биекция, то $\mu^*(A) = \mu(QA)$~--- мера.
	Действтельно, неотрицательность $\mu^*(A)$ очевидна.
	Покажем выполнение аксиомы счётной аддитивности.
	Пусть $A_i \cap A_j = \varnothing$ при $i\neq j$,
	тогда в силу биективности $Q$ имеем $QA_i \cap QA_j = \varnothing$ и
	\begin{equation}
		\mu^*\left( \bigcup_i A_i \right)
		=
		\mu\left( Q \bigcup_i A_i \right)
		=
		\mu\left( \bigcup_i QA_i \right)
		=
		\sum_i 	\mu( QA_i )
		=
		\sum_i 	\mu^*( A_i )
		.
	\end{equation}
	Согласно теореме Каратеодори,
	если две меры совпадают на некоторой системе подмножеств $\mathcal{S}^*$,
	то они совпадают и на минимальной $\sigma$--алгебре, содержащей $\mathcal{S}^*$.
	Следовательно, биекция $Q$ действительно сохраняет меру.
\end{proof}

\begin{theorem}[{\cite[Лемма 1]{our-mz2019ac0}}]
	\label{thm:lim_M(j)/j}
	Пусть $n_i$~--- строго возрастающая последовательность чисел из $\mathbb{N}$,
	\begin{equation*}
		x_k = \left\{\begin{array}{ll}
			1, & \mbox{~если~} k = n_i
			\\
			0  & \mbox{~иначе.~}
		\end{array}\right.
	\end{equation*}
	Для того, чтобы $x\in ac_0$,
	необходимо и достаточно, чтобы
	\begin{equation}\label{lim_M(j)/j}
		\lim_{j \to \infty} \frac{M(j)}{j} = \infty
		,
	\end{equation}
	где
	\begin{equation}
		\label{eq:definition_M_j}
		M(j) = \liminf_{i\to\infty} (n_{i+j} - n_i)
		.
	\end{equation}
\end{theorem}


\begin{proof}[Доказательство теоремы~\ref{thm:mes_W}]
	Равномерное стремление~\eqref{eq:lim_j_n_kj_measure}
	буквально означает, что
	\begin{equation}
		\forall(A>0)\exists(j_A\in\mathbb{N})\forall(j > j_A)\forall(k\in\mathbb{N})
		\left[
			\frac{n_{k+j}-n_k}{j}>A
		\right]
		.
	\end{equation}
	Так как последнее неравенство выполнено для любого $k$,
	то оно верно и для нижнего предела по $k$:
	\begin{equation}
		\forall(A>0)\exists(j_A\in\mathbb{N})\forall(j > j_A)
		\left[
			\liminf_{k\to\infty}\frac{n_{k+j}-n_k}{j}>A
		\right]
		.
	\end{equation}
	<<Свернув>> определение предела по $j$ из кванторной записи, получаем:
	\begin{equation}
		\lim_{j\to\infty}\liminf_{k\to\infty}\frac{n_{k+j}-n_k}{j} = \infty
		,
	\end{equation}
	откуда по лемме~\ref{thm:lim_M(j)/j} немедленно имеем $QW \subset \Omega\cap ac_0$.
	Тогда
	\begin{equation}
		0 \leq \operatorname{mes}(W) = \operatorname{mes}(QW) \leq \operatorname{mes}(\Omega\cap ac_0) \leq \operatorname{mes}(\Omega\cap ac) = 0
		,
	\end{equation}
	откуда $\operatorname{mes}(W)=0$.
\end{proof}

Оказывается, что теорему~\ref{thm:lim_M(j)/j} можно в некотором смысле усилить.

\begin{theorem}
	Пусть $M(j)$~--- последовательность, определённая в~\eqref{thm:lim_M(j)/j}.
	Тогда последовательность $M(j)/j$ имеет предел (конечный или бесконечный).
\end{theorem}

\begin{proof}
	Предположим противное.
	Пусть
	\begin{equation}
		\liminf_{j\to\infty}\frac{M(j)}{j} = A,
	\end{equation}
	\begin{equation}
		\limsup_{j\to\infty}\frac{M(j)}{j} \geq B
	\end{equation}
	(неравенство используется для охвата случая, когда верхний предел равен бесконечности)
	и
	\begin{equation}
		B > A
		.
	\end{equation}

	Положим $\varepsilon = \frac{B-A}{4}$.
	По определению верхнего предела существует такое натуральное $j_0$,
	что
	\begin{equation}
		\frac{M(j_0)}{j_0} > B - \varepsilon
		.
	\end{equation}
	По определению нижнего предела существует $\{j_i\}$~--- возрастающая последовательность таких индексов, что
	\begin{equation}
		\label{eq:lim_Mj_inf_lim_seq}
		\forall(i\in\mathbb{N})\left[ \frac{M(j_i)}{j_i} < A + \varepsilon \right]
		.
	\end{equation}
	Определим для каждого $i\in\mathbb{N}$ целое число $C_i$ такое, что
	\begin{equation}
		\label{eq:lim_Mj_inf_lim_Ci}
		C_i j_0 \leq j_i < (C_i+1)j_0
		.
	\end{equation}
	Легко заметить, что последовательность $\{C_i\}$ является неубывающей.

	Очевидно, что
	\begin{equation}
		M(Cj) \geq C \cdot M(j)
		.
	\end{equation}
	Таким образом,
	\begin{equation}
		M(j_i) \geq M(C_i j_0) \geq C_i M(j_0) > C_i j_0 (B-\varepsilon)
	\end{equation}
	и
	\begin{multline}
		\frac{M(j_i)}{j_i} > \frac{C_i j_0 (B-\varepsilon)}{j_i}
		> \frac{C_i j_0 (B-\varepsilon)}{(C_i+1)j_0}
		=
		\\=
		\frac{C_i}{C_i+1}(B-\varepsilon)
		= B-\varepsilon - \frac{1}{C_i+1}(B-\varepsilon)
		.
	\end{multline}
	Выберем $i$ настолько большим, что
	\begin{equation}
		\frac{1}{C_i+1}(B-\varepsilon) < \varepsilon
		.
	\end{equation}
	Тогда
	\begin{equation}
		\frac{M(j_i)}{j_i} > B - 2 \varepsilon = A + 2 \varepsilon > A + \varepsilon
		,
	\end{equation}
	что противоречит~\eqref{eq:lim_Mj_inf_lim_seq}.
	Полученное противоречие завершает доказательство.
\end{proof}

\begin{hypothesis}
	Включение $QW\subset \Omega\cap ac_0$ собственное.
\end{hypothesis}

\begin{hypothesis}
	Пусть $M(j)$~--- последовательность натуральных чисел, такая,
	что для любых $i,j\in\mathbb{N}$ выполнено
	\begin{equation}
		M(i)+M(j) \leq M(i+j)
		.
	\end{equation}
	Тогда существует последовательность $\{x_k\}$,
	удовлетворяющая условию~\eqref{eq:definition_M_j}.
\end{hypothesis}
TODO: думаю, что конструкцию выписать несложно.


\printbibliography

\end{document}
