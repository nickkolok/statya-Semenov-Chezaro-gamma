\documentclass[a4paper,openbib]{report}
\usepackage{amsmath}
\usepackage[utf8]{inputenc}
\usepackage[english,russian]{babel}
\usepackage{amsfonts,amssymb}
\usepackage{latexsym}
\usepackage{euscript}
\usepackage{enumerate}
\usepackage{graphics}
\usepackage[dvips]{graphicx}
\usepackage{geometry}
\usepackage{wrapfig}
\usepackage[colorlinks=true,allcolors=black]{hyperref}
\usepackage{bbm}



\righthyphenmin=2

\usepackage[14pt]{extsizes}

\geometry{left=4cm}% левое поле
\geometry{right=4cm}% правое поле
\geometry{top=4cm}% верхнее поле
\geometry{bottom=4cm}% нижнее поле
\geometry{papersize={420mm,594mm}}% A2, вертикальный


\renewcommand{\baselinestretch}{1.17}

\renewcommand{\leq}{\leqslant}
\renewcommand{\geq}{\geqslant} % И делись оно всё нулём!

\newcommand{\longcomment}[1]{}

\usepackage[backend=biber,style=gost-numeric,sorting=none]{biblatex}
\addbibresource{../bib/my.bib}

\usepackage{../../../biblatex2bibitem/biblatex2bibitem}

\usepackage{amsthm}
\theoremstyle{definition}
\newtheorem{lemma}{Лемма}%[section]
\newtheorem{theorem}[lemma]{Теорема}
\newtheorem{example}[lemma]{Пример}
\newtheorem{property}[lemma]{Свойство}
\newtheorem{remark}[lemma]{Замечание}
\newtheorem{corollary}{Следствие}[lemma]

\newtheorem{hypothesis}[lemma]{Гипотеза}

\usepackage{varwidth}


%Only referenced equations are numbered
\usepackage{mathtools}
\mathtoolsset{showonlyrefs}

%\mathtoolsset{showonlyrefs=false}
% (an equation/multline to be force-numbered)
%\mathtoolsset{showonlyrefs=true}

\newcommand{\frametitle}[1]{\begin{center}\textbf{\textsc{#1}}\end{center}}


\makeatletter
\def\thm@space@setup{\thm@preskip=0pt
\thm@postskip=0pt}
\makeatother


\begin{document}
\thispagestyle{empty}
\clubpenalty=10000
\widowpenalty=10000
	\Large
	\begin{center}
		ФГБОУ ВО
		«ВОРОНЕЖСКИЙ ГОСУДАРСТВЕННЫЙ УНИВЕРСИТЕТ»
		\\
		Математический факультет
	\end{center}
	\begin{center}
		\Huge
		Банаховы пределы и асимптотические свойства \\ ограниченных последовательностей
	\end{center}
	Автор: Николай Николаевич Авдеев, аспирант третьего года обучения
	\\
	Научный руководитель: д.ф.-м.н., проф. Евгений Михайлович Семенов
	\\
	\nocite{avdeev2021subsets,avdeev2019banach,avdeev2019space,avdeed2021AandA,our-ped-2018-alpha-Tx,our-vzms-2018}

	\noindent
	\par
\begin{varwidth}[t]{0.3\linewidth}
	\frametitle{Пространство $\ell_\infty$}
	$\ell_\infty$~--- пространство всех ограниченных последовательностей
	$x=(x_1, x_2, ..., x_n, ...)$
	с нормой
	$$
		\|x\|_{\ell_\infty} = \sup_{k\in\mathbb{N}} |x_k|
	$$
	$c$~--- пространство всех сходящихся последовательностей.
\end{varwidth}
\hfill
\begin{varwidth}[t]{0.25\linewidth}
	\frametitle{Банаховы пределы}
	Банаховым пределом называется функционал $B\in \ell_\infty^*$ такой, что:
	\begin{enumerate}
		\item
			$B \geqslant 0$
		\item
			$B(1,1,1,1,1,...) = 1$
		\item
			$B=BT$, где $T$ "--- сдвиг.
	\end{enumerate}
\end{varwidth}
\hfill
\begin{varwidth}[t]{0.35\linewidth}
\frametitle{Теорема Лоренца}
	Для заданного $r\in\mathbb{R}$ равенство $Bx=r$ выполнено для всех $B\in\mathfrak{B}$
	тогда и только тогда, когда
	\begin{equation*}
		\lim_{n\to\infty} \frac{1}{n} \sum_{k=m+1}^{m+n} x_k = r
	\end{equation*}
	сходится равномерно по $m\in\mathbb{N}$.
	\\
	$ac$~---пространство всех таких $x \in \ell_\infty$.
\end{varwidth}
\par
\vfill

	\vfill
	\quad$\uparrow$ Классические понятия и результаты, использованные в работе
	\hrule
	\quad$\downarrow$ Основные результаты в области фундаментального функционального анализа, полученные автором
	\vfill
	\noindent
	\begin{varwidth}[v]{0.455\linewidth}
	\frametitle{Критерии принадлежности $ac_0$~\cite{avdeev2019space}}


	\begin{lemma}
		Пусть $n_i$~--- строго возрастающая последовательность чисел из $\mathbb{N}$,
		$\displaystyle
			x_k = \left\{\begin{array}{ll}
				1, & \mbox{~если~} k = n_i
				\\
				0  & \mbox{~иначе.~}
			\end{array}\right.
		$
		\\
		$x\in ac_0$
		тогда и только тогда, когда
		$\displaystyle
			\lim_{j \to \infty} {M(j)}/{j} = +\infty
			,
		$
		где
		$\displaystyle
			%\quad\mbox{~где~\quad}
			M(j) = \liminf_{i\to\infty} (n_{i+j} - n_i)
			.
		$
	\end{lemma}


	\vspace{1.2em}
	Определим нелинейный оператор $\lambda$--срезки $A_\lambda$
	для $x\in\ell_\infty$ как
	$\displaystyle
		(A_\lambda x)_k = \begin{cases}
			1, & \mbox{~если~} x_k \geq \lambda
			\\
			0  & \mbox{~иначе.~}
		\end{cases}
	$

	\vspace{0.5em}
	\begin{theorem}
		\label{thm:lambda_prelim}
		Пусть $x\in\ell_\infty$, $x\geq 0$.
		Тогда
		$
			x\in ac_0
		$
		если и только если
		для любого $\lambda > 0$
		выполнено
		$
			A_\lambda x \in ac_0
		$
		.
	\end{theorem}
	\begin{theorem}
		Пусть $x\in\ell_\infty$, $x \geq 0$,
		$\displaystyle
			0<\lambda < \limsup_{k\to\infty} x_k
		$,
		$\{k: x_k \geq \lambda \} = \{n(\lambda,1),n(\lambda,2),...\}$.
		Обозначим
		$\displaystyle
			M_{\lambda}(j) = \liminf_{i\to\infty} n(\lambda,i+j) - n(\lambda,i)
			.
		$
		$x\in ac_0$ тогда и только тогда, когда
		для всех $\lambda$ выполнено
		$\displaystyle
			\lim_{j \to \infty} {M_{\lambda}(j)}/{j} = +\infty
			.
		$
	\end{theorem}
\end{varwidth}

	\hfill
	\begin{varwidth}[v]{0.52\linewidth}
	\frametitle{Функция $\alpha(x)$ и пространство $A_0=\{x\in\ell_\infty:\alpha(x)=0\}$~\cite{avdeev2019space,our-ped-2018-alpha-Tx,our-vzms-2018}}

	Рассмотрим функцию
	$\displaystyle
		\alpha(x) = \limsup_{i\to\infty} \max_{i \leq j \leq 2i} |x_i-x_j|
	$.
	\\
	Условие $\alpha(x) = 0$ есть ослабленное условие Коши.
	\vspace{0.25em}

	\begin{lemma}
		Для любого $x\in\ell_\infty$
		выполнено
		$
			\alpha(x) \leq 2\rho(x, c)
		$,
		\\
		где $\rho(x,c)$ "--- расстояние от $x$ до $c$.
		\\
		Для любых $x\in ac$,~$s\in\mathbb{N}$ верно
		$
			\rho(x,c)\leq \alpha(T^s x)
		$.
		\\
		Обе оценки точны;
		%\\
		условие $x\in ac$ существенно.
	\end{lemma}
	\begin{theorem}
	%\label{thm:rho_x_c_leq_alpha_t_s_x}
		Для любого $x\in ac$ верно
		$\displaystyle
			\frac{1}{2} \alpha(x) \leq \rho(x,c)\leq \lim_{s\to\infty} \alpha(T^s x) \leq \alpha(x)
			;
		$
		в частности, $A_0 \cap ac = c$.
	\end{theorem}
	\vspace{0.2em}
	\begin{theorem}
		Ни по сложению, ни по умножению не замкнуто ни  одно из множеств
		$\displaystyle
			\{x \in \ell_\infty : \alpha(T^n x) = \alpha(x) \};%, ~n\in\mathbb{N};
		$~
		$$\displaystyle
			\bigcap\limits_{n\in\mathbb{N}}\{x \in \ell_\infty : \alpha(T^n x) = \alpha(x) \};
		\hfill
		\displaystyle
			\bigcup_{n\in\mathbb{N}}\{x \in \ell_\infty : \alpha(T^n x) = \alpha(x) \}
			.
		$$
		\vspace{-0.7em}
	\end{theorem}

	\begin{theorem}
		Пространство $A_0$ замкнуто относительно
		(покоординатного) умножения,
		операторов сдвига $T$, Чезаро $C$,
		растяжения $\sigma_n$ и сжатия $\sigma_{1/n}$,
		несепарабельно и недополняемо в $\ell_\infty$.
	\end{theorem}


\end{varwidth}



	\vfill

	%Представленная НИР содержит новые результаты в области фундаментального функционального анализа
	\begin{center}
		Список основных публикаций с участием автора по теме НИР:
	\end{center}

	\renewcommand{\section}[2]{}%
	\renewcommand{\chapter}[2]{}% for other classes
	\renewcommand{\baselinestretch}{0.95}
	\printbibliography{}
	\newpage
	\printbibitembibliography{}

\end{document}
