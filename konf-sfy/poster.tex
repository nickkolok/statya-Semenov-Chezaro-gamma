\documentclass[a4paper,openbib]{report}
\usepackage{amsmath}
\usepackage[utf8]{inputenc}
\usepackage[english,russian]{babel}
\usepackage{amsfonts,amssymb}
\usepackage{latexsym}
\usepackage{euscript}
\usepackage{enumerate}
\usepackage{graphics}
\usepackage[dvips]{graphicx}
\usepackage{geometry}
\usepackage{wrapfig}
\usepackage[colorlinks=true,allcolors=black]{hyperref}
\usepackage{bbm}



\righthyphenmin=2

\usepackage[14pt]{extsizes}

\geometry{left=4cm}% левое поле
\geometry{right=4cm}% правое поле
\geometry{top=4cm}% верхнее поле
\geometry{bottom=4cm}% нижнее поле
\geometry{papersize={420mm,594mm}}% A2, вертикальный


\renewcommand{\baselinestretch}{1.3}

\renewcommand{\leq}{\leqslant}
\renewcommand{\geq}{\geqslant} % И делись оно всё нулём!

\newcommand{\longcomment}[1]{}

\usepackage[backend=biber,style=gost-numeric,sorting=none]{biblatex}
\addbibresource{../bib/my.bib}


\usepackage{amsthm}
\theoremstyle{definition}
\newtheorem{lemma}{Лемма}[section]
\newtheorem{theorem}[lemma]{Теорема}
\newtheorem{example}[lemma]{Пример}
\newtheorem{property}[lemma]{Свойство}
\newtheorem{remark}[lemma]{Замечание}
\newtheorem{corollary}{Следствие}[lemma]

\newtheorem{hypothesis}[lemma]{Гипотеза}

\usepackage{varwidth}


%Only referenced equations are numbered
\usepackage{mathtools}
\mathtoolsset{showonlyrefs}

%\mathtoolsset{showonlyrefs=false}
% (an equation/multline to be force-numbered)
%\mathtoolsset{showonlyrefs=true}

\newcommand{\frametitle}[1]{\begin{center}#1\end{center}}


\begin{document}
\thispagestyle{empty}
\clubpenalty=10000
\widowpenalty=10000
	\Large
	\begin{center}
		ФЕДЕРАЛЬНОЕ ГОСУДАРСТВЕННОЕ БЮДЖЕТНОЕ ОБРАЗОВАТЕЛЬНОЕ
		УЧРЕЖДЕНИЕ ВЫСШЕГО ОБРАЗОВАНИЯ
		«ВОРОНЕЖСКИЙ ГОСУДАРСТВЕННЫЙ УНИВЕРСИТЕТ»
		\\
		Математический факультет
	\end{center}
	\begin{center}
		\Huge
		Банаховы пределы и асимптотические свойства \\ ограниченных последовательностей
	\end{center}
	Автор: Николай Николаевич Авдеев, аспирант третьего года обучения
	\\
	Научный руководитель: д.ф.-м.н., проф. Е. М. Семенов.
	\\
	\nocite{avdeev2021subsets,our-mz2021linearhulls}
	\nocite{*}

	\noindent
	\par
\begin{varwidth}[t]{0.3\linewidth}
	\frametitle{Пространство $\ell_\infty$}
	$\ell_\infty$~--- пространство всех ограниченных последовательностей
	$x=(x_1, x_2, ..., x_n, ...)$
	с нормой
	$$
		\|x\|_{\ell_\infty} = \sup_{k\in\mathbb{N}} |x_k|
	$$
	$c$~--- пространство всех сходящихся последовательностей.
\end{varwidth}
\hfill
\begin{varwidth}[t]{0.25\linewidth}
	\frametitle{Банаховы пределы}
	Банаховым пределом называется функционал $B\in \ell_\infty^*$ такой, что:
	\begin{enumerate}
		\item
			$B \geqslant 0$
		\item
			$B(1,1,1,1,1,...) = 1$
		\item
			$B=BT$, где $T$ "--- сдвиг.
	\end{enumerate}
\end{varwidth}
\hfill
\begin{varwidth}[t]{0.35\linewidth}
\frametitle{Теорема Лоренца}
	Для заданного $r\in\mathbb{R}$ равенство $Bx=r$ выполнено для всех $B\in\mathfrak{B}$
	тогда и только тогда, когда
	\begin{equation*}
		\lim_{n\to\infty} \frac{1}{n} \sum_{k=m+1}^{m+n} x_k = r
	\end{equation*}
	сходится равномерно по $m\in\mathbb{N}$.
	\\
	$ac$~---пространство всех таких $x \in \ell_\infty$.
\end{varwidth}
\par
\vfill


	\qquad$\uparrow$ Классические понятия и результаты, использованные в работе
	\hrule
	\qquad$\downarrow$ Основные результаты, полученные автором

	\vfill
	\begin{center}
		Список публикаций
	\end{center}

	\renewcommand{\section}[2]{}%
	\renewcommand{\chapter}[2]{}% for other classes
	\printbibliography{}

\end{document}
