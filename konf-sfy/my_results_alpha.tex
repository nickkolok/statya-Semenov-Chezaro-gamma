\begin{varwidth}[v]{0.52\linewidth}
	\frametitle{Функция $\alpha(x)$ и пространство $A_0=\{x\in\ell_\infty:\alpha(x)=0\}$~\cite{avdeev2019space,our-ped-2018-alpha-Tx,our-vzms-2018}}

	Рассмотрим функцию
	$\displaystyle
		\alpha(x) = \limsup_{i\to\infty} \max_{i \leq j \leq 2i} |x_i-x_j|
	$.
	\\
	Условие $\alpha(x) = 0$ есть ослабленное условие Коши.
	\vspace{0.25em}



	\begin{theorem}
		Пусть $(x\cdot y)_k = x_k\cdot y_k$.
		Тогда
		$\alpha(x\cdot y)\leq \alpha(x)\cdot \|y\|_* + \alpha(y)\cdot \|x\|_*$,
		где
		$\displaystyle\|x\|_* = \limsup_{k\to\infty} |x_k|$
		есть фактор"=норма $\ell_\infty$ по $c_0$.
	\end{theorem}
	\begin{lemma}
		Для любого $x\in\ell_\infty$
		выполнено
		$
			\alpha(x) \leq 2\rho(x, c)
		$,
		%\\
		где $\rho(x,c)$ "--- расстояние от $x$ до $c$.
	\end{lemma}
	%	\\
	%	Для любых $x\in ac$,~$s\in\mathbb{N}$ верно
	%	$
	%		\rho(x,c)\leq \alpha(T^s x)
	%	$.
	%	\\
	%	Обе оценки точны;
	%	%\\
	%	условие $x\in ac$ существенно.
	%\end{lemma}
	\begin{theorem}
	%\label{thm:rho_x_c_leq_alpha_t_s_x}
		Для любого $x\in ac$ верно
		$\displaystyle
			\frac{1}{2} \alpha(x) \leq \rho(x,c)\leq \lim_{s\to\infty} \alpha(T^s x) \leq \alpha(x)
			;
		$
		в частности, $A_0 \cap ac = c$.
	\end{theorem}
	\vspace{0.2em}
	\begin{theorem}
		Ни по сложению, ни по умножению не замкнуто ни  одно из множеств
		$\displaystyle
			\{x \in \ell_\infty : \alpha(T^n x) = \alpha(x) \};%, ~n\in\mathbb{N};
		$
		$$\displaystyle
			\mathop{\cap}\limits_{n\in\mathbb{N}}\{x \in \ell_\infty {:\,} \alpha(T^n x) = \alpha(x) \};
		\hfill
		\displaystyle
			\mathop{\cup}_{n\in\mathbb{N}}\{x \in \ell_\infty {:\,} \alpha(T^n x) = \alpha(x) \}
			.
		$$
		\vspace{-1.2em}
	\end{theorem}

	\begin{theorem}
		Пространство $A_0$ замкнуто относительно
		(покоординатного) умножения,
		операторов сдвига $T$, Чезаро $C$,
		растяжения $\sigma_n$ и сжатия $\sigma_{1/n}$,
		несепарабельно и недополняемо в $\ell_\infty$.
	\end{theorem}


\end{varwidth}
