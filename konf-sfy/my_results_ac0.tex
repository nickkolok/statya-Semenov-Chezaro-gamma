\begin{varwidth}[v]{0.455\linewidth}
	\frametitle{Критерии принадлежности $ac_0$~\cite{avdeev2019space}}


	\begin{lemma}
		Пусть $n_i$~--- строго возрастающая последовательность чисел из $\mathbb{N}$,
		$\displaystyle
			x_k = \left\{\begin{array}{ll}
				1, & \mbox{~если~} k = n_i
				\\
				0  & \mbox{~иначе.~}
			\end{array}\right.
		$
		\\
		$x\in ac_0$
		тогда и только тогда, когда
		$\displaystyle
			\lim_{j \to \infty} {M(j)}/{j} = +\infty
			,
		$
		где
		$\displaystyle
			%\quad\mbox{~где~\quad}
			M(j) = \liminf_{i\to\infty} (n_{i+j} - n_i)
			.
		$
	\end{lemma}


	\vspace{1.2em}
	Определим нелинейный оператор $\lambda$--срезки $A_\lambda$
	для $x\in\ell_\infty$ как
	$\displaystyle
		(A_\lambda x)_k = \begin{cases}
			1, & \mbox{~если~} x_k \geq \lambda
			\\
			0  & \mbox{~иначе.~}
		\end{cases}
	$

	\vspace{0.5em}
	\begin{theorem}
		\label{thm:lambda_prelim}
		Пусть $x\in\ell_\infty$, $x\geq 0$.
		Тогда
		$
			x\in ac_0
		$
		если и только если
		для любого $\lambda > 0$
		выполнено
		$
			A_\lambda x \in ac_0
		$
		.
	\end{theorem}
	\begin{theorem}
		Пусть $x\in\ell_\infty$, $x \geq 0$,
		$\displaystyle
			0<\lambda < \limsup_{k\to\infty} x_k
		$,
		$\{k: x_k \geq \lambda \} = \{n(\lambda,1),n(\lambda,2),...\}$.
		Обозначим
		$\displaystyle
			M_{\lambda}(j) = \liminf_{i\to\infty} n(\lambda,i+j) - n(\lambda,i)
			.
		$
		$x\in ac_0$ тогда и только тогда, когда
		для всех $\lambda$ выполнено
		$\displaystyle
			\lim_{j \to \infty} {M_{\lambda}(j)}/{j} = +\infty
			.
		$
	\end{theorem}
\end{varwidth}
