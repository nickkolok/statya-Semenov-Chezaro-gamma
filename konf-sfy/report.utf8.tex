\documentclass[10pt,pdf,hyperref={unicode},aspectratio=169]{beamer}
\usepackage{amsmath}
\usepackage[utf8]{inputenc}
\usepackage[english,russian]{babel}
\usepackage{amsfonts}
\usepackage{amsfonts,amssymb}
\usepackage{amssymb}
\usepackage{latexsym}
\usepackage{euscript}
\usepackage{enumerate}
\usepackage{graphics}
\usepackage{graphicx}
\usepackage{geometry}
\usepackage{wrapfig}

\usepackage{bbm}

\righthyphenmin=2

% https://superuser.com/questions/517025/how-can-i-append-two-pdfs-that-have-links
\usepackage{pdfpages}% http://ctan.org/pkg/pdfpages

\usepackage{varwidth}


\usepackage[backend=biber,style=gost-numeric,sorting=none]{biblatex}
\addbibresource{../bib/Semenov.bib}
\addbibresource{../bib/my.bib}
\addbibresource{../bib/ext.bib}

\input{../bib/ext.hyphens.bib}


\usepackage{amsthm}
\theoremstyle{definition}
\newtheorem{llemma}{Лемма}
\newtheorem{ttheorem}[llemma]{Теорема}
\newtheorem{eexample}[llemma]{Пример}
\newtheorem{property}[llemma]{Свойство}
\newtheorem{remark}[llemma]{Замечание}
\newtheorem{ccorollary}{Следствие}[llemma]


%\usetheme{Antibes}
\usefonttheme{professionalfonts} % using non standard fonts for beamer
\usefonttheme[onlymath]{serif} % default family is serif
%\usepackage{fontspec}
%\setmainfont{Liberation Serif}
\setbeamertemplate{footline}[frame number]
\setbeamertemplate{navigation symbols}{}

\setbeamertemplate{frametitle}[default][center]

\begin{document}
{
	\setbeamercolor{background canvas}{bg=}
	\includepdf{pres_Avdeev_title.pdf}
}
%\setcounter{page}{2}
\setbeamertemplate{background}
{\includegraphics[width=\paperwidth,height=\paperheight,keepaspectratio]{bg.jpg}}

\setbeamertemplate{navigation symbols}{реферативная часть}

\begin{frame}
	\frametitle{Обобщения понятия сходимости в $\ell_\infty$}
	%\hspace{1.468em}
	\begin{varwidth}[t]{\linewidth}
		\centering
		Верхний и нижний
		\\
		пределы:
		\\
		$\displaystyle \limsup_{n\to\infty} x_n$
		,~
		$\displaystyle \liminf_{n\to\infty} x_n$
		\\~\\
		\emph{
			не характеризуют
			\\
			распределение элементов
		}
	\end{varwidth}
	\hfill
	\begin{varwidth}[t]{\linewidth}
		\centering
		Сходимость в среднем
		\\
		(по Чезаро):
		$\displaystyle \lim_{n\to\infty} (Cx)_n$
		\\
		$\displaystyle (Cx)_n = \frac1n\sum_{i=1}^n x_i$
		\\~\\
		\emph{слишком универсальна}
	\end{varwidth}
	\hfill
	\begin{varwidth}[t]{\linewidth}
		\centering
		Банаховы пределы $B\in \mathfrak{B}$:
		\\
		т. Хана--Банаха к $\lim: c\to\mathbb R$
		\\
			$B \geqslant 0$
		\\
			$B\mathbbm{1} = 1$
		\\
			$B=BT$
		\\
		\emph{почти сходимость}
	\end{varwidth}
	\\~\\~\\
	\begin{varwidth}[t]{\linewidth}
		\centering
		\emph{Критерий Лоренца:}
		$\displaystyle
			x\in ac_t
			\Leftrightarrow
			\forall(B\in\mathfrak{B})[Bx = t]
			\Leftrightarrow
			\lim_{n\to\infty}  \sum_{k=m+1}^{m+n} \frac{x_k}n = t
		$
		равном. по $m\in\mathbb{N}$.
	\end{varwidth}
	\\~\\
	\vspace{1em}
	\begin{varwidth}[t]{\linewidth}
		$\displaystyle ac = \mathop{\cup}\limits_{t\in\mathbb R} ac_t$ "--- пространство почти сходящихся последовательностей
	\end{varwidth}
	\\
	\vspace{1em}
	\begin{varwidth}[t]{\linewidth}
		Пример $x\in ac_0$:~~
		$\displaystyle
			x_k=\begin{cases}
				1, & k=2^j,~j\in\mathbb N
				\\
				0 & \mbox{для прочих~} k
			\end{cases}
		$
	\end{varwidth}
\end{frame}

\setbeamertemplate{navigation symbols}{результаты конкурсанта}


\begin{frame}
	\frametitle{Пространства $ac$ и $ac_0$}



	\begin{ttheorem}
		Пусть ~$x\in\ell_\infty$,~~ $x \geq 0$,
		~~
		$\displaystyle
			0<\lambda < \limsup_{k\to\infty} x_k
			.
		$
		~~
		Пусть $\{k: x_k \geq \lambda \} = \{n(\lambda,1),n(\lambda,2),...\}$.
		Обозначим
		$\displaystyle
			M_{\lambda}(j) = \liminf_{i\to\infty} n(\lambda,i+j) - n(\lambda,i)
			.
		$
		Для того, чтобы $x\in ac_0$, необходимо и достаточно, чтобы
		независимо от выбора $\lambda$ было выполнено
		$\displaystyle
			\lim_{j \to \infty} \frac{M_{\lambda}(j)}{j} = \infty
			.
		$
	\end{ttheorem}



\end{frame}


\begin{frame}
	\frametitle{Функция $\alpha(x)$ и пространство $ac$~\cite{our-mz2019ac0}}
	Пусть $\rho(x,c)$ и $\rho(x,c_0)$~--- расстояния от $x$ до пространств $c$
	и $c_0$ соотв.
	Легко видеть, что функция
	\begin{equation*}
		\alpha(x) = \limsup_{i\to\infty} \max_{i \leq j \leq 2i} |x_i-x_j|
	\end{equation*}
	удовлетворяет условию Липшица с константой 2
	и для $x\in c$ выполнено
	$\alpha(x)=0$.

	\begin{llemma}
	\label{thm:alpha_x_leq_2_rho_x_c}
		Для любого $x\in\ell_\infty$
		выполнено неравенство
		$
			\alpha(x) \leq 2\rho(x, c)
		$.
	\end{llemma}


	\begin{llemma}
	\label{thm:rho_x_c_leq_alpha_t_s_x}
		Для любого $x\in ac$ и для любого натурального $s$ верно
		$
			\rho(x,c)\leq \alpha(T^s x)
		$,
	%	где $T$~--- оператор сдвига влево, т.е.
	%	$T(x_1,x_2,...) = (x_2,x_3,...)$.
	\end{llemma}

	Обе оценки точны.
	Условие $x\in ac$ существенно.

\end{frame}


\begin{frame}
	\frametitle{Функция $\alpha(x)$ и пространство $ac$~\cite{our-mz2019ac0}}
	\begin{ttheorem}
	%\label{thm:rho_x_c_leq_alpha_t_s_x}
		Для любого $x\in ac$
		\begin{equation*}
			\frac{1}{2} \alpha(x) \leq \rho(x,c)\leq \lim_{s\to\infty} \alpha(T^s x) \leq \alpha(x)
			.
		\end{equation*}
	\end{ttheorem}

	\begin{ccorollary}
		Для любого $x\in ac_0$
		\begin{equation*}
			\frac{1}{2} \alpha(x) \leq \rho(x,c_0)\leq \lim_{s\to\infty} \alpha(T^s x) \leq \alpha(x)
			.
		\end{equation*}
	\end{ccorollary}

	Обе оценки точны~\cite{our-ped-2018-alpha-Tx}.

	\begin{ccorollary}
		Пусть $x\in ac$.
		Тогда $x\in c$ если и только если $\alpha(x) = 0$.
	\end{ccorollary}

\end{frame}




\begin{frame}
	\huge\centering
	Спасибо за внимание
\end{frame}


\end{document}
