\documentclass[10pt,pdf,hyperref={unicode},aspectratio=169,color={usenames, dvipsnames}]{beamer}
\usepackage{amsmath}
\usepackage[utf8]{inputenc}
\usepackage[english,russian]{babel}
\usepackage{amsfonts,amssymb}
\usepackage{amssymb}
\usepackage{latexsym}
\usepackage{euscript}
\usepackage{enumerate}
\usepackage{graphics}
\usepackage{graphicx}
\usepackage{geometry}
\usepackage{wrapfig}

\usepackage{bbm}

\righthyphenmin=2

% https://superuser.com/questions/517025/how-can-i-append-two-pdfs-that-have-links
\usepackage{pdfpages}% http://ctan.org/pkg/pdfpages

\usepackage{varwidth}


\usepackage[backend=biber,style=gost-numeric,sorting=none]{biblatex}
\addbibresource{../bib/Semenov.bib}
\addbibresource{../bib/my.bib}
\addbibresource{../bib/ext.bib}

\input{../bib/ext.hyphens.bib}


\usepackage{amsthm}
\theoremstyle{definition}
\newtheorem{llemma}{Лемма}
\newtheorem{ttheorem}[llemma]{Теорема}
\newtheorem{eexample}[llemma]{Пример}
\newtheorem{property}[llemma]{Свойство}
\newtheorem{remark}[llemma]{Замечание}
\newtheorem{ccorollary}{Следствие}[llemma]


%\usetheme{Antibes}
\usefonttheme{professionalfonts} % using non standard fonts for beamer
\usefonttheme[onlymath]{serif} % default family is serif
%\usepackage{fontspec}
%\setmainfont{Liberation Serif}
\setbeamertemplate{footline}[frame number]
\setbeamertemplate{navigation symbols}{}

\setbeamertemplate{frametitle}[default][center]

\begin{document}
{
	\setbeamercolor{background canvas}{bg=}
	\includepdf{pres_Avdeev_title.pdf}
}
%\setcounter{page}{2}
\setbeamertemplate{background}
{\includegraphics[width=\paperwidth,height=\paperheight,keepaspectratio]{bg.jpg}}

\setbeamertemplate{navigation symbols}{реферативная часть}

\begin{frame}
	\frametitle{Обобщения понятия сходимости в $\ell_\infty$}
	%\hspace{1.468em}
	\begin{varwidth}[t]{\linewidth}
		\centering
		Верхний и нижний
		\\
		пределы:
		\\
		$\displaystyle \limsup_{n\to\infty} x_n$
		,~
		$\displaystyle \liminf_{n\to\infty} x_n$
		\\~\\
		\emph{
			не характеризуют
			\\
			распределение элементов
		}
	\end{varwidth}
	\hfill
	\begin{varwidth}[t]{\linewidth}
		\centering
		Сходимость в среднем
		\\
		(по Чезаро):
		$\displaystyle \lim_{n\to\infty} (Cx)_n$
		\\
		$\displaystyle (Cx)_n = \frac1n\sum_{i=1}^n x_i$
		\\~\\
		\emph{слишком универсальна}
	\end{varwidth}
	\hfill
	\begin{varwidth}[t]{\linewidth}
		\centering
		Банаховы пределы $B\in \mathfrak{B}$:
		\\
		т. Хана--Банаха к $\lim: c\to\mathbb R$
		\\
			$B \geqslant 0$
		\\
			$B\mathbbm{1} = 1$
		\\
			$B=BT$
		\\
		\emph{почти сходимость}
	\end{varwidth}
	\\~\\~\\
	\begin{varwidth}[t]{\linewidth}
		\centering
		\emph{Критерий Лоренца:}
		$\displaystyle
			x\in ac_t
			\Leftrightarrow
			\forall(B\in\mathfrak{B})[Bx = t]
			\Leftrightarrow
			\lim_{n\to\infty}  \sum_{k=m+1}^{m+n} \frac{x_k}n = t
		$
		равном. по $m\in\mathbb{N}$.
	\end{varwidth}
	\\~\\
	\vspace{1em}
	\begin{varwidth}[t]{\linewidth}
		$\displaystyle ac = \mathop{\cup}\limits_{t\in\mathbb R} ac_t$ "--- пространство почти сходящихся последовательностей
	\end{varwidth}
	\\
	\vspace{1em}
	\begin{varwidth}[t]{\linewidth}
		Пример $x\in ac_0$:~~
		$\displaystyle
			x_k=\begin{cases}
				1, & k=2^j,~j\in\mathbb N
				\\
				0 & \mbox{для прочих~} k
			\end{cases}
		$
	\end{varwidth}
\end{frame}

\setbeamertemplate{navigation symbols}{результаты конкурсанта}


\begin{frame}
	\frametitle{Пространства $ac$ и $ac_0$}


	\vspace{0.4em}
	\begin{ttheorem}
		Пусть ~$x\in\ell_\infty$,~~ $x \geq 0$,
		~~
		$\displaystyle
			0<\lambda < \limsup_{k\to\infty} x_k
			.
		$
		~~
		Пусть $\{k: x_k \geq \lambda \} = \{n(\lambda,1),n(\lambda,2),...\}$.
		Обозначим
		$\displaystyle
			M_{\lambda}(j) = \liminf_{i\to\infty} n(\lambda,i+j) - n(\lambda,i)
			.
		$
		Для того, чтобы $x\in ac_0$, необходимо и достаточно, чтобы
		независимо от выбора $\lambda$ было выполнено
		$\displaystyle
			\lim_{j \to \infty} \frac{M_{\lambda}(j)}{j} = \infty
			.
		$
	\end{ttheorem}


	\vspace{0.8em}
	Рассмотрим функцию
	$\displaystyle
		\alpha(x) = \limsup_{i\to\infty} \max_{i \leq j \leq 2i} |x_i-x_j|
	$.
	%(условие $\alpha(x) = 0$ есть ослабленное определение фундаментальности).


	\begin{llemma}
		Для любого $x\in\ell_\infty$
		выполнено
		$
			\alpha(x) \leq 2\rho(x, c)
		$,
		\\
		где $\rho(x,c)$ "--- расстояние от $x$ до $c$.
		\\
		Для любых $x\in ac$,~$s\in\mathbb{N}$ верно
		$
			\rho(x,c)\leq \alpha(T^s x)
		$,
		\\
		где $T$~--- оператор сдвига влево.%, т.е.
		%$T(x_1,x_2,...) = (x_2,x_3,...)$.
		\\
		Обе оценки точны;
		%\\
		условие $x\in ac$ важно.
	\end{llemma}

\end{frame}


\begin{frame}
	\frametitle{Функция $\alpha(x)$}
	\begin{equation*}
		\alpha(x) = \limsup_{i\to\infty} \max_{i \leq j \leq 2i} |x_i-x_j|
		;\qquad
		A_0 = \{x\in\ell_\infty : \alpha(x) = 0\}
	\end{equation*}
	Условие $\alpha(x) = 0$ есть ослабленное условие Коши (определение фундаментальности).

	\begin{ttheorem}
	%\label{thm:rho_x_c_leq_alpha_t_s_x}
		Для любого $x\in ac$ верно
		$\displaystyle
			\frac{1}{2} \alpha(x) \leq \rho(x,c)\leq \lim_{s\to\infty} \alpha(T^s x) \leq \alpha(x)
			;
		$
		в частности, $A_0 \cap ac = c$.
	\end{ttheorem}


	\begin{ttheorem}
		Ни одно из множеств
		$\displaystyle
			\{x \in \ell_\infty : \alpha(T^n x) = \alpha(x) \}, ~n\in\mathbb{N};
		$~
		$\displaystyle
			\bigcap\limits_{n\in\mathbb{N}}\{x \in \ell_\infty : \alpha(T^n x) = \alpha(x) \};
		$ ~
		$\displaystyle
			\bigcup_{n\in\mathbb{N}}\{x \in \ell_\infty : \alpha(T^n x) = \alpha(x) \}
		$ не замкнуто ни по сложению, ни по умножению.
	\end{ttheorem}

	\begin{ttheorem}
		Пространство $A_0$ замкнуто относительно
		\\
		(покоординатного) умножения,
		\\
		а также операторов сдвига $T$, Чезаро $C$,
		\\
		растяжения $\sigma_n$ и сжатия $\sigma_{1/n}$,
		\\
		несепарабельно и недополняемо в $\ell_\infty$.
	\end{ttheorem}

\end{frame}



\def\rinc{\color[rgb]{0.5,0,0}{(РИНЦ)~}}
\def\vak{\color[rgb]{0.5,0.5,0}{(ВАК)~}}
\def\mzm{\color[rgb]{0,0.5,0}{(WoS)~(Scopus)~}}

\setbeamertemplate{bibliography item}{\insertbiblabel\small}

\begin{frame}
	\frametitle{Основные публикации с участием конкурсанта}
	\vspace{-1.2em}
	\hfill
	\begin{varwidth}[t]{0.95\linewidth}
	\setlength\itemsep{0pt}
	\begin{thebibliography}{99}
		\addtobeamertemplate{block begin}{\vspace*{-30pt}}{}
		\addtobeamertemplate{block end}{}{\vspace*{-30pt}}
		\setlength{\parskip}{0pt}%
		\setlength{\itemsep}{0pt}
		{}
		\bibitem{avdeev2021subsets}\small
			\emph{Avdeev N.} On Subsets of the Space of Bounded Sequences // Mathematical Notes. — 2021. — т. 109, No 1. — с. 150—154.
			\mzm
		{}
		\vspace{-0.5em}
		\bibitem{avdeev2019banach}\small
			\emph{Avdeev} \emph{N.}, \emph{Semenov} \emph{E.}, \emph{Usachev} \emph{A.} Banach Limits and a Measure on the Set of 0-1-Sequences //
			Mathematical Notes. — 2019. — т. 106, No 5/6. — с. 833—836.
			\mzm
		{}
		\vspace{-0.5em}
		\bibitem{avdeev2019space}\small
			\emph{Avdeev} \emph{N.} On the Space of Almost Convergent Sequences // Mathematical Notes. — 2019. — т. 105, No 3/4. — с. 464—468.
			%— DOI: \href
			%{https://doi.org/10.1134/S0001434619030179} {\nolinkurl {10.1134/S0001434619030179}}. — URL: \url
			%{https://www.scopus.com/record/display.uri?origin=inward&eid=2-s2.0-85065492318}.
			\mzm
		{}
		\vspace{-0.5em}
		\bibitem{avdeed2021AandA}\small
			\emph{Авдеев} \emph{Н. Н.}, \emph{Семёнов} \emph{Е. М.}, \emph{Усачев} \emph{А. С.}
			Банаховы пределы: экстремальные свойства,
			инвариантность и теорема Фубини // Алгебра и анализ. — 2021. — т. 33, No 4. — с. 32—48.
			\vak
		{}
		\vspace{-0.5em}
		\bibitem{our-ped-2018-alpha-Tx}\footnotesize
			\emph{Авдеев} \emph{Н. Н.} О суперпозиции оператора сдвига и одной функции на пространстве ограниченных последовательностей //
			Некоторые вопросы анализа, алгебры, геометрии и математического образования. — 2018. — с. 20—21.
			\rinc
		{}
		\vspace{-0.6em}
		\bibitem{our-vzms-2018}\footnotesize
			\emph{Авдеев} \emph{Н. Н.}, \emph{Семенов} \emph{Е. М.}
			Об асимптотических свойствах оператора Чезаро \\// Материалы Воронежской Зимней
			Математической\\школы С.Г. Крейна – 2018.  Под ред.\\В.А. Костина. — 2018. — с. 107—109.
			\rinc
	\end{thebibliography}
	\end{varwidth}
	\vspace{10em}
\end{frame}


\begin{frame}
	{
		\huge\centering
		~\\~\\~\\~\\
		Спасибо за внимание
	}
	~\\
	\vspace{6.28em}
	nickkolok@mail.ru, avdeev@math.vsu.ru
	\\
	github.com/nickkolok
	\\
	arxiv.org/a/avdeev\_n\_1.html
\end{frame}


\end{document}
