\begin{varwidth}[V]{0.3\linewidth}
	\frametitle{Пространство $\ell_\infty$}
	$\ell_\infty$~--- пространство всех ограниченных последовательностей
	$x=(x_1, x_2, ..., x_n, ...)$
	с нормой
	$$
		\|x\|_{\ell_\infty} = \sup_{k\in\mathbb{N}} |x_k|
	$$
	$c$~--- пространство всех сходящихся последовательностей.
\end{varwidth}
 \hfill
\begin{varwidth}[V]{0.25\linewidth}
	\frametitle{Банаховы пределы}
	Банаховым пределом называется функционал $B\in \ell_\infty^*$ такой, что:
	\begin{enumerate}
		\item
			$B \geqslant 0$
		\item
			$B(1,1,1,1,1,...) = 1$
		\item
			$B=BT$, где $T$ "--- сдвиг.
	\end{enumerate}
\end{varwidth}
 \hfill
\begin{varwidth}[V]{0.35\linewidth}
\frametitle{Теорема Лоренца}
	Для заданного $r\in\mathbb{R}$ равенство $Bx=r$ выполнено для всех $B\in\mathfrak{B}$
	тогда и только тогда, когда
	\begin{equation*}
		\lim_{n\to\infty} \frac{1}{n} \sum_{k=m+1}^{m+n} x_k = r
	\end{equation*}
	сходится равномерно по $m\in\mathbb{N}$.
	\\
	$ac$~---пространство всех таких $x \in \ell_\infty$.
\end{varwidth}
