\documentclass[a4paper,openbib]{report}
\usepackage{amsmath}
\usepackage[utf8]{inputenc}
\usepackage[english,russian]{babel}
\usepackage{amsfonts,amssymb}
\usepackage{latexsym}
\usepackage{euscript}
\usepackage{enumerate}
\usepackage{graphics}
\usepackage[dvips]{graphicx}
\usepackage{geometry}
\usepackage{wrapfig}
\usepackage[colorlinks=true,allcolors=black]{hyperref}



\righthyphenmin=2

\usepackage[14pt]{extsizes}

\geometry{left=3cm}% левое поле
\geometry{right=1cm}% правое поле
\geometry{top=2cm}% верхнее поле
\geometry{bottom=2cm}% нижнее поле

\renewcommand{\baselinestretch}{1.3}

\renewcommand{\leq}{\leqslant}
\renewcommand{\geq}{\geqslant} % И делись оно всё нулём!

\newcommand{\longcomment}[1]{}

\usepackage[backend=biber,style=gost-numeric,sorting=none]{biblatex}
\addbibresource{../bib/Semenov.bib}
\addbibresource{../bib/my.bib}
\addbibresource{../bib/ext.bib}

\begin{document}
\clubpenalty=10000
\widowpenalty=10000

\centering{
	{\large\bf Программа}
	\\
	учебной практики студента Авдеева Н.Н.
}

\begin{enumerate}
	\item
		Ознакомиться с главами II.C.--II.E. монографии \cite{wojtaszczyk1996banach}.
	\item
		Изучить статью \cite{Semenov2010invariant}.
	\item
		Пусть для $x\in\ell_\infty$
		$$
			\alpha(x) = \varlimsup_{i\to\infty} \max_{i < j \leq 2i} |x_i-x_j|
			.
		$$

		Получить оценки на $\alpha(Tx)$ и $\alpha(\sigma_n x)$ (в обозначениях \cite{Semenov2010invariant})
		через $\alpha(x)$
		и исследовать их точность.
\end{enumerate}

\begingroup
	\let\clearpage\relax
	\printbibliography[title={\large Список литературы}]
\endgroup

\end{document}
