\documentclass[a5paper,12pt,openbib]{report}
\usepackage{amsmath}
\usepackage[utf8]{inputenc}
\usepackage[english,russian]{babel}
\usepackage{amsfonts}
\usepackage{amsfonts,amssymb}
\usepackage{amssymb}
\usepackage{latexsym}
\usepackage{euscript}
\usepackage{enumerate}
\usepackage{graphics}
\usepackage[dvips]{graphicx}
\usepackage{geometry}
\usepackage{wrapfig}

\geometry{verbose,a5paper,tmargin=1.75cm,bmargin=2.1cm,lmargin=1.75cm,rmargin=1.75cm}

\righthyphenmin=2


\begin{document}

\clubpenalty=10000
\widowpenalty=10000



\begin{center}{ \bf  Об асимптотических свойствах оператора Чезаро }\\
{\it Н.Н. Авдеев } \\
(Воронежский госуниверситет; {\it nickkolok@mail.ru} ) \\
{\it Е.М. Семёнов } \\
(Воронежский госуниверситет; %{\it mail@mail.ru}
)
\end{center}
\addcontentsline{toc}{section}{Авдеев Н.Н., Семёнов Е.М.}
%Или тут раздельно?


Обозначим через $\mathbb{N}$ множество натуральных чисел,
через $l_\infty$ --- пространство ограниченных последовательностей $\{x_n\}_{n\in\mathbb{N}}$,
т.~е.
$$
	l_\infty = \left\{
		\{x_n\}_{n \in \mathbb{N}} \left|
			\sup_{n\in\mathbb{N}} x_n < \infty
		\right.
	\right\}.
$$

Определим на $l_\infty$ оператор Чезаро $C$
%TODO: ссылка на JFA
равенством
$$
	(Cx)_n = \frac{1}{n}\sum_{k=1}^n x_k.
$$

Определим на $l_\infty$ $\alpha$--функцию,
характеризующую асимпототические свойства элемента,
равенством
$$
	\alpha(x) = \varlimsup_{i\to\infty}\sup_{i < j \leqslant 2i} |x_i - x_j|.
$$

Пусть $A = {x\in l_\infty | 0 \leqslant x_n \leqslant 1}$.
Асимптотические свойства оператора Чезаро удобнее сначала изучать на множестве $A$,
а затем перенормировкой распространять на всё $l_\infty$.

Можно легко доказать, что для $x\in A$ выполнено соотношение $\alpha(Cx) \leqslant 1/2$.
Значительно сложнее доказывается, что для $x\in A$ выполнена оценка
$$
	\alpha(Cx) \leqslant \alpha(x) \cdot
	\left(
		1 - 2 ^ {-\frac{1}{\alpha(x)} - 4}
	\right)
$$ 
%TODO: проверить, -4 ли там

Далее приводится схема доказательства следующего утверждения.


\textbf{Теорема~1.}
{\it
	Не существует такого $\gamma < 1$,
	что для любого $x\in A$ выполнена мультипликативная оценка
	$$
		\alpha(Cx) \leqslant \gamma \cdot \alpha(x).
	$$
}

Очевидно, что теорема~1 может быть эквиваленто переформулирована:

\textbf{Теорема~1'.}
{\it
	Не существует такого $p\in \mathbb{N}$,
	что для любого $x\in A$ выполнена мультипликативная оценка
	$$
		\alpha(Cx) \leqslant (1-2^p)\cdot \alpha(x).
	$$
}

Для доказательства теоремы~1' потребуются вспомогательные построения.


\textbf{Доказательство теоремы~1.}


Объем статей --- 3 страницы (для лекторов --- до 5 стр.),
выполненная в редакторе  LaTeX по  образцу, который можно скачать на
сайте vzmsh2018.ru.

При оформлении текста тезисов просим не  переопределять команды и не
вводить свои макросы, а также не использовать автоматическую
нумерацию формул, библиографии, теорем и пр. Для нумерации формул
используйте, пожалуйста, команду \verb"\eqno".

Утверждения типа теорем и лемм следует оформлять по следующему
образцу.

\textbf{Теорема~1.} {\it Пример оформления определений и утверждений
типа теорем, лемм.}

\begin{center}
Файл должен компилироваться без ошибок \\
и переполнений ("overfull").
\end{center}

\textbf{Электронную версию тезисов необходимо выслать  по
электронному адресу vzms@mail.ru.}

%%%%  ОФОРМЛЕНИЕ СПИСКА ЛИТЕРАТУРЫ %%%
\smallskip \centerline{\bf Литература}\nopagebreak

1. {\it Крейн С.Г.} Линейные дифференциальные уравнения в банаховом пространстве. М.: Наука, 1987. 408 с.

2. {\it Львовский С. М.} Набор и верстка в системе LATEX. М.: МЦНМО, 2003. — 448 с.

\end{document} 
