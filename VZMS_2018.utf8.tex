\documentclass[a5paper,12pt,openbib]{report}
\usepackage{amsmath}
\usepackage[utf8]{inputenc}
\usepackage[english,russian]{babel}
\usepackage{amsfonts}
\usepackage{amsfonts,amssymb}
\usepackage{amssymb}
\usepackage{latexsym}
\usepackage{euscript}
\usepackage{enumerate}
\usepackage{graphics}
\usepackage[dvips]{graphicx}
\usepackage{geometry}
\usepackage{wrapfig}

\geometry{verbose,a5paper,tmargin=1.75cm,bmargin=2.1cm,lmargin=1.75cm,rmargin=1.75cm}

\righthyphenmin=2


\begin{document}

\clubpenalty=10000
\widowpenalty=10000



\begin{center}{ \bf  Об асимптотических свойствах оператора Чезаро}%
\footnote{Работа выполнена за счёт гранта РНФ, проект 16-11-10125}
\\
{\it Н.Н. Авдеев } \\
(Воронеж, ВГУ; {\it nickkolok@mail.ru} ) \\
{\it Е.М. Семенов } \\
(Воронеж, ВГУ; {\it nadezhka\_ssm@geophys.vsu.ru}
)
\end{center}
\addcontentsline{toc}{section}{Авдеев Н.Н., Семёнов Е.М.}
%Или тут раздельно?

%%%%%%%%%%%%%%%%%%%%%%%%%%%%%%%%%%%%%%%%%%%%%%%%%%%%%%%
% Немного магии
% Мне ОЧЕНЬ хочется использовать автонумерацию формул.
% Если причина только в том, что при автонумерации
% номера формул в статье будут начинаться не с 1,
% то это можно побороть вот так:

\setcounter{equation}{0}

% Достаточно включить предыдущую строчку в образец оформления,
% и автонумерацией формул можно будет смело пользоваться.
% С теоремами это тоже делается при желании,
% но перед набором тезисов автор обычно знает,
% сколько теорем будет в тексте. А с формулами это не так :(
% Библиографию не трогаю, это вообще отдельная песня с непотребным припевом.

На пространстве ограниченных последовательностей $l_\infty$ определяется оператор Чезаро $C$
равенством
$
	(Cx)_n = {1}/{n} \cdot \sum_{k=1}^n x_k.
$
Определим на $l_\infty$ $\alpha$--функцию,
характеризующую асимптотические свойства последовательности,
равенством
$$
	\alpha(x) = \limsup_{i\to\infty}\sup_{i < j \leqslant 2i} |x_i - x_j|.
$$
Пусть $A = \{x\in l_\infty | 0 \leqslant x_n \leqslant 1\}$.
Асимптотические свойства оператора Чезаро удобнее сначала изучать на множестве $A$,
а затем перенормировкой распространять на всё $l_\infty$.
В [1] изучается ряд свойств оператора Чезаро.
Можно легко доказать, что для $x\in A$ выполнено соотношение $\alpha(Cx) \leqslant 1/2$
и $\alpha(Cx) \leqslant \alpha(x)$.
Возникает естественный вопрос о справедливости более точной оценки.

\textbf{Теорема~1.}
{\it
	Не существует такого $\gamma < 1$,
	что для любого $x\in A$ выполнена мультипликативная оценка
	$$
		\alpha(Cx) \leqslant \gamma \cdot \alpha(x)
	,
	$$
или, что то же самое, не существует такого $p\in \mathbb{N}$,
	что для любого $x\in A$ выполнено неравенство
	$
		\alpha(Cx) \leqslant (1-2^{-p+1})\cdot \alpha(x).
	$
}

Для доказательства теоремы~1 потребуются вспомогательные построения.
\begin{equation}\label{summa_drobey}
	\sum_{i=0}^{p-1} \frac{i \cdot 2^i}{p} = \frac{2^p(p-2) + 2}{p}
\end{equation}
Введём вспомогательный оператор $S:l_\infty \to l_\infty$:
\begin{equation*}\label{operator_S}
	(Sy)_k = y_{i+2}, \mbox{ где } 2^i < k \leqslant 2^i+1
\end{equation*}
Нам потребуются следующие свойства оператора $S$.
\begin{equation}\label{alpha_S}
	\alpha(Sx) = \varlimsup_{k\to\infty} |x_{k+1} - x_{k}|
\end{equation}
\begin{equation}\label{summa_S_less}
	\sum_{k=2}^{2^p} (Sy)_k =
	\sum_{i=0}^{p-1} 2^i y_{i+2}
\end{equation}
Здесь и далее $(Tx)_n = x_{n+1}$.
\begin{equation}\label{summa_S}
	\sum_{k=2^i+1}^{2^{i+j+1}} (Sx)_k =
	2^i\sum_{k=2}^{2^{j+1}} (ST^ix)_k
\end{equation}
Введём вспомогательную функцию
%\vspace{-2.28em}
\begin{equation*}\label{def_k_b}
	k_b(x) = (2b)^{-1} \left|
		\sum\nolimits_{k=1}^{b}x_k - \sum\nolimits_{k=b+1}^{2b}x_k
	\right|
\end{equation*}
Тогда
\begin{equation}\label{alpha_greater_k_b}
	\alpha (Cx) \geqslant \varlimsup_{i\to \infty} k_i(x)
\end{equation}

\textbf{Схема доказательства теоремы~1.}
Зафиксируем $p$ и построим $y\in l_\infty$:
\begin{equation*}\label{y_construction}
	y = \left\{
		0, 0, \frac{1}{p}, \frac{2}{p}, %\frac{3}{p},
		...,
		\frac{p-1}{p}, 1, \frac{p-1}{p},
		...,
		\frac{1}{p},
		0, ..., 0,
		\frac{1}{p}, ...
	\right\}
\end{equation*}
так, что
\begin{equation}\label{T_y}
	T^{5p}y = y
\end{equation}
Положим $x = Sy$, тогда с учётом (\ref{alpha_S})
$
	\alpha (x) = \alpha (Sy) = \frac{1}{p}
$.
Оценим $\alpha(Cx)$:
\begin{multline*}
	\alpha (Cx) \mathop{\geqslant}^{(\ref{alpha_greater_k_b})}
	\varlimsup_{b\to \infty} k_b(x) \geqslant
	\varlimsup_{
		i\to \infty,~
		b=2^i~
	}\frac{1}{2^{i+1}}\left|
		\sum_{k=1}^{2^i}(Sy)_k - \sum_{k=2^i+1}^{2^{i+1}}(Sy)_k
	\right| \geqslant
	\\ \geqslant
	\varlimsup_{
		m\to \infty,~
		i=5pm+p~
	}\left|
		\frac{1}{2^{5pm+p+1}}\sum_{k=1}^{2^{5pm+p}}(Sy)_k - \frac{y_{5pm+p+2}}{2}
	\right| =
	\\=
	\varlimsup_{m\to \infty}\left|
		\frac{1}{2^{5pm+p+1}}\sum_{k=1}^{2^{5pm}}(Sy)_k
		+
		\frac{1}{2^{5pm+p+1}}\sum_{k=2^{5pm}+1}^{2^{5pm+p}}(Sy)_k
		- \frac{1}{2}
	\right|
	\mathop{=}^{(\ref{summa_S})}
	\\=
	\varlimsup_{m\to \infty}\left|
		\frac{1}{2^{5pm+p+1}}\sum_{k=1}^{2^{5pm}}(Sy)_k
		+
		\frac{2^{5pm}}{2^{5pm+p+1}} \sum_{k=2}^{2^p}(ST^{5pm}y)_k
		- \frac{1}{2}
	\right|
	\mathop{=}^{(\ref{T_y})}
	\\=
	\varlimsup_{m\to \infty}\left|
		\frac{1}{2^{5pm+p+1}}\sum_{k=1}^{2^{5pm}}(Sy)_k
		+
		\frac{1}{2^{p+1}} \sum_{k=2}^{2^p}(Sy)_k
		- \frac{1}{2}
	\right|
	\mathop{=}^{(\ref{summa_S_less})}
	\\=
	\varlimsup_{m\to \infty}\left|
		\frac{1}{2^{5pm+p+1}}\sum_{k=1}^{2^{5pm}}(Sy)_k
		+
		\frac{1}{2^{p+1}} \sum_{i=0}^{p-1}2^i \cdot \frac{i}{p}
		- \frac{1}{2}
	\right|
	\mathop{=}^{(\ref{summa_drobey})}
%\end{multline*}
%\begin{multline*}
	\\=
	\varlimsup_{m\to \infty}\left|
		\frac{1}{2^{5pm+p+1}}\sum_{k=1}^{2^{5pm-2p}}(Sy)_k
		-\frac{1}{p} + \frac{1}{p 2^p}
	\right| \geqslant
	\\ \geqslant
	\varlimsup_{m\to \infty} \left(
		\frac{1}{p} (1-2^{-p})
		- \frac{1}{2^{3p+1}}
	\right) >
	\frac{1}{p} (1-2^{-p+1})
\end{multline*}


Таким образом,
$
	\alpha(Cx) >
	(1-2^{-p+1}) \cdot \alpha(x)
$.

%%%%  ОФОРМЛЕНИЕ СПИСКА ЛИТЕРАТУРЫ %%%
\smallskip \centerline{\bf Литература}\nopagebreak

1. {\it Semenov E. M., Sukochev F. A.}
 Invariant Banach limits and applications //Journal of Functional Analysis. – 2010. – Т. 259. – №. 6. – С. 1517-1541.

\end{document}
