\documentclass[a5paper,12pt,openbib]{report}
\usepackage{amsmath}
\usepackage[cp1251]{inputenc}
\usepackage[english,russian]{babel}
\usepackage{amsfonts}
\usepackage{amsfonts,amssymb}
\usepackage{amssymb}
\usepackage{latexsym}
\usepackage{euscript}
\usepackage{enumerate}
\usepackage{graphics}
\usepackage[dvips]{graphicx}
\usepackage{geometry}
\usepackage{wrapfig}

\geometry{verbose,a5paper,tmargin=1.75cm,bmargin=2.1cm,lmargin=1.75cm,rmargin=1.75cm}

\righthyphenmin=2


\begin{document}

\clubpenalty=10000
\widowpenalty=10000



\begin{center}{ \bf  НАЗВАНИЕ СТАТЬИ}\\
{\it И.И. Иванов } \\
(Москва; {\it mail@mail.ru} )
\end{center}
\addcontentsline{toc}{section}{Иванов И.И.}


Текст статьи



Объем статей --- 3 страницы (для лекторов --- до 5 стр.),
выполненная в редакторе  LaTeX по  образцу, который можно скачать на
сайте vzmsh2018.ru.

При оформлении текста тезисов просим не  переопределять команды и не
вводить свои макросы, а также не использовать автоматическую
нумерацию формул, библиографии, теорем и пр. Для нумерации формул
используйте, пожалуйста, команду \verb"\eqno".

Утверждения типа теорем и лемм следует оформлять по следующему
образцу.

\textbf{Теорема~1.} {\it Пример оформления определений и утверждений
типа теорем, лемм.}

\begin{center}
Файл должен компилироваться без ошибок \\
и переполнений ("overfull").
\end{center}

\textbf{Электронную версию тезисов необходимо выслать  по
электронному адресу vzms@mail.ru.}

%%%%  ОФОРМЛЕНИЕ СПИСКА ЛИТЕРАТУРЫ %%%
\smallskip \centerline{\bf Литература}\nopagebreak

1. {\it Крейн С.Г.} Линейные дифференциальные уравнения в банаховом пространстве. М.: Наука, 1987. 408 с.

2. {\it Львовский С. М.} Набор и верстка в системе LATEX. М.: МЦНМО, 2003. — 448 с.

\end{document} 
