\documentclass[a4paper,14pt]{article} %размер бумаги устанавливаем А4, шрифт 12пунктов
\usepackage[T2A]{fontenc}
\usepackage[utf8]{inputenc}
\usepackage[english,russian]{babel} %используем русский и английский языки с переносами
\usepackage{amssymb,amsfonts,amsmath,mathtext,cite,enumerate,float,amsthm} %подключаем нужные пакеты расширений
\usepackage[unicode,colorlinks=true,citecolor=black,linkcolor=black]{hyperref}
%\usepackage[pdftex,unicode,colorlinks=true,linkcolor=blue]{hyperref}
\usepackage{indentfirst} % включить отступ у первого абзаца
\usepackage[dvips]{graphicx} %хотим вставлять рисунки?
\graphicspath{{illustr/}}%путь к рисункам

\makeatletter
\renewcommand{\@biblabel}[1]{#1.} % Заменяем библиографию с квадратных скобок на точку:
\makeatother %Смысл этих трёх строчек мне непонятен, но поверим "Запискам дебианщика"

\usepackage{geometry} % Меняем поля страницы.
\geometry{left=2cm}% левое поле
\geometry{right=1cm}% правое поле
\geometry{top=2cm}% верхнее поле
\geometry{bottom=2cm}% нижнее поле

\renewcommand{\theenumi}{\arabic{enumi}}% Меняем везде перечисления на цифра.цифра
\renewcommand{\labelenumi}{\arabic{enumi}}% Меняем везде перечисления на цифра.цифра
\renewcommand{\theenumii}{.\arabic{enumii}}% Меняем везде перечисления на цифра.цифра
\renewcommand{\labelenumii}{\arabic{enumi}.\arabic{enumii}.}% Меняем везде перечисления на цифра.цифра
\renewcommand{\theenumiii}{.\arabic{enumiii}}% Меняем везде перечисления на цифра.цифра
\renewcommand{\labelenumiii}{\arabic{enumi}.\arabic{enumii}.\arabic{enumiii}.}% Меняем везде перечисления на цифра.цифра

\sloppy


\renewcommand\normalsize{\fontsize{14}{25.2pt}\selectfont}

\begin{document}
% !!!
% Здесь начинается реальный ТеХ-код
% Всё, что выше - беллетристика

\paragraph{Теорема.}
$$
	\forall(x\in l_\infty) \forall(n\in\mathbb{N})
	[
		\alpha(\sigma_n x) = \alpha(x)
	]
$$

\paragraph{Доказательство.}
По определению
\begin{equation}
	\alpha(x) = \varlimsup_{i\to\infty} \max_{i<j\leqslant 2i} |x_i - x_j|
\end{equation}

Положим
\begin{equation}
	\alpha_i(x) =
	\max_{i<j\leqslant 2i} |x_i - x_j| =
	\max_{i\leqslant j\leqslant 2i} |x_i - x_j|
\end{equation}

Тогда
\begin{equation}
	\alpha(x) = \varlimsup_{i\to\infty} \alpha_i(x)
\end{equation}

Пусть $y = \sigma_n x$.
Тогда для $k=1, ..., n-1$, $a\in\mathbb{N}$ имеем
\begin{multline}
	\alpha_{an-k}(y) =
	\max_{an-k \leqslant j \leqslant 2an-2k} |y_{an-k} - y_j| =
	\\=
	(\mbox{т.к.}~y_{an-(n-1)}=y_{an-(n-2)}=...=y_{an-k}=...=y_{an-1}=y_{an})=
	\\=
	\max_{an \leqslant j \leqslant 2an-2k} |y_{an} - y_j| \leqslant
	\\ \leqslant
	(\mbox{переходим к максимуму по большему множеству}) \leqslant
	\\ \leqslant
	\max_{an \leqslant j \leqslant 2an} |y_{an} - y_j| =
	\alpha_{an}(y)
\end{multline}

С другой стороны,
\begin{multline}
	\alpha_{an}(y) =
	\max_{an \leqslant j \leqslant 2an} |y_{an} - y_j| =
	\\ =
	(\mbox{т.к.}~y=\sigma_n x,~\mbox{можем рассматривать только}~j=kn)=
	\\ =
	\max_{an \leqslant kn \leqslant 2an} |y_{an} - y_{kn}| =
	\max_{a \leqslant k \leqslant 2a} |y_{an} - y_{kn}| =
	\max_{a \leqslant k \leqslant 2a} |x_a - x_k| =
	\alpha_a(x)
\end{multline}

Таким образом, для $k=1, ..., n-1$, $a\in\mathbb{N}$ имеем соотношения:
\begin{gather}
	\alpha_{an}(y) = \alpha_a(x),
\\
	\alpha_{an-k}(y) \leqslant \alpha_a(x),
\end{gather}
откуда немедленно следует, что
\begin{equation}
	\varlimsup_{i\to\infty} \alpha_i(y) =
	\varlimsup_{i\to\infty} \alpha_i(x),
\end{equation}
т.е.
\begin{equation}
	\alpha(\sigma_n x) = \alpha(x),
\end{equation}
что и требовалось доказать.

\paragraph{Следствие.}
$$
	\alpha(C\sigma_2 x) =
	\alpha(\sigma_2 Cx) =
	\alpha(Cx)
$$

\end{document}
