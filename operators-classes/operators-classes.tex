\documentclass[a4paper,14pt]{article} %размер бумаги устанавливаем А4, шрифт 12пунктов
\usepackage[T2A]{fontenc}
\usepackage[utf8]{inputenc}
\usepackage[croatian,german,french,english,russian]{babel}
\usepackage{amssymb,amsfonts,amsmath,mathtext,enumerate,amsthm} %подключаем нужные пакеты расширений
\usepackage[unicode,colorlinks=true,citecolor=black,linkcolor=black]{hyperref}
%\usepackage[pdftex,unicode,colorlinks=true,linkcolor=blue]{hyperref}
\usepackage{indentfirst} % включить отступ у первого абзаца
%\usepackage[dvips]{graphicx} %хотим вставлять рисунки?
%\graphicspath{{illustr/}}%путь к рисункам

\makeatletter
\renewcommand{\@biblabel}[1]{#1.} % Заменяем библиографию с квадратных скобок на точку:
\makeatother %Смысл этих трёх строчек мне непонятен, но поверим "Запискам дебианщика"

\usepackage{geometry} % Меняем поля страницы.
\geometry{left=2cm}% левое поле
\geometry{right=1cm}% правое поле
\geometry{top=2cm}% верхнее поле
\geometry{bottom=2cm}% нижнее поле

\renewcommand{\theenumi}{\arabic{enumi}.}% Меняем везде перечисления на цифра.цифра
\renewcommand{\labelenumi}{\arabic{enumi}.}% Меняем везде перечисления на цифра.цифра
\renewcommand{\theenumii}{\arabic{enumii}}% Меняем везде перечисления на цифра.цифра
\renewcommand{\labelenumii}{\arabic{enumi}.\arabic{enumii}.}% Меняем везде перечисления на цифра.цифра
\renewcommand{\theenumiii}{\arabic{enumiii}}% Меняем везде перечисления на цифра.цифра
\renewcommand{\labelenumiii}{\arabic{enumi}.\arabic{enumii}.\arabic{enumiii}.}% Меняем везде перечисления на цифра.цифра

\sloppy


\DeclareMathOperator{\ext}{ext}
\DeclareMathOperator{\mes}{mes}
\DeclareMathOperator{\supp}{supp}
\DeclareMathOperator{\conv}{conv}
\DeclareMathOperator{\diam}{diam}

\newcommand{\N}{\ensuremath{\mathbb{N}}}
\newcommand{\Q}{\ensuremath{\mathbb{Q}}}
\newcommand{\R}{\ensuremath{\mathbb{R}}}
\newcommand{\B}{\ensuremath{\mathfrak{B}}}
\newcommand{\Iac}{\mathcal{I}(ac_0)}
\newcommand{\Dac}{\mathcal{D}(ac_0)}
\newcommand{\one}{\ensuremath{\mathbbm 1}}




\renewcommand\normalsize{\fontsize{14}{25.2pt}\selectfont}

%\usepackage{cite}


\theoremstyle{plain}
\newtheorem{lemma}{Лемма}[section]
\newtheorem{theorem}[lemma]{Теорема}
\newtheorem{example}[lemma]{Пример}
\newtheorem{property}[lemma]{Свойство}
\newtheorem{remark}[lemma]{Замечание}
\newtheorem{corollary}{Следствие}[lemma]
\newtheorem{definition}[lemma]{Определение}
\newtheorem{proposition}[lemma]{Утверждение}
\newtheorem{hypothesis}[lemma]{Гипотеза}

\usepackage{bbm}


%Only referenced equations are numbered
\let\etoolboxforlistloop\forlistloop % save the good meaning of \forlistloop
\usepackage{mathtools}
\mathtoolsset{showonlyrefs}
\let\forlistloop\etoolboxforlistloop % restore the good meaning of \forlistloop

% https://tex.stackexchange.com/questions/220268/biblatex-and-autonum-dont-work-together

%\mathtoolsset{showonlyrefs=false}
% (write an equation/multline to be force-numbered here)
%\mathtoolsset{showonlyrefs=true}


\newcommand{\longcomment}[1]{}

\usepackage[backend=biber,style=gost-numeric,sorting=none]{biblatex}
\addbibresource{../bib/Semenov.bib}
\addbibresource{../bib/my.bib}
\addbibresource{../bib/ext.bib}
\addbibresource{../bib/classic.bib}
\addbibresource{../bib/Damerau-Levenstein.bib}
\addbibresource{../bib/general_monographies.bib}
\addbibresource{../bib/Bibliography_from_Usachev.bib}

\input{../bib/ext.hyphens.bib}



\begin{document}

%\renewcommand{\bibname}{Список цитированной литературы}
%\renewcommand\refname{\bibname}
% !!!
% The text starts here

\title{
	Classes of linear operators defined via Banach limits
	\footnote{
		TODO - grant
	}
}

\author{
	A.S. Usachev
	\footnote{TODO},
	R.E. Zvolinsky
	\footnote{TODO},
	N.N. Avdeev
	\footnote{nickkolok@mail.ru, avdeev@math.vsu.ru}
}

\maketitle

УДК 517.982.276 % Пространства последовательностей и матриц

\paragraph{Abstract}
TODO


\paragraph{Keywords}
	space of bounded sequences,
	Banach limits,
	Sucheston functional,




\section{Введение}

Рассмотрим пространство ограниченных последовательностей $\ell_\infty$ с обычной нормой
\begin{equation*}
	\|x\| = \sup_{k\in\mathbb{N}} |x_k|
	.
\end{equation*}
и обычной полуупорядоченностью, где $\mathbb{N}$ "--- множество натуральных чисел.
Через $c$ будем обозначать пространство сходящихся последовательностей,
через $c_0$ "--- пространство последовательностей, сходящихся к нулю.


\begin{definition}
	Линейный функционал $B\in \ell_\infty^*$ называется банаховым пределом,
	если
	\begin{enumerate}
		\item
			$B\geq0$, т.~е. $Bx \geq 0$ для $x \geq 0$,
		\item
			$B\mathbbm{1}=1$, где $\mathbbm{1} =(1,1,\ldots)$,
		\item
			$B(Tx)=B(x)$ для всех $x\in \ell_\infty$, где $T$~---
		оператор сдвига, т.~е. $T(x_1,x_2,\ldots)=(x_2,x_3,\ldots)$.
	\end{enumerate}
\end{definition}
Множество всех банаховых пределов обозначим через $\mathfrak{B}$.
Существование банаховых пределов было анонсировано С. Мазуром \cite{Mazur} и позднее доказано в книге С.~Банаха~\cite{B}.



	Сачестон~\cite{sucheston1967banach} установил, что
	для любых $x\in \ell_\infty$ и $B\in\mathfrak{B}$
	\begin{equation}\label{Sucheston}
		q(x) \leqslant Bx \leqslant p(x)
		,
	\end{equation}
	где
	\begin{equation*}
		q(x) = \lim_{n\to\infty} \inf_{m\in\mathbb{N}}  \frac{1}{n} \sum_{k=m+1}^{m+n} x_k
		~~~~\mbox{и}~~~~
		p(x) = \lim_{n\to\infty} \sup_{m\in\mathbb{N}}  \frac{1}{n} \sum_{k=m+1}^{m+n} x_k
		.
	\end{equation*}
	называют нижним и верхним функционалом Сачестона соотвественно.
	Заметим, что $p(x) = -q(-x)$.
	Неравенства \eqref{Sucheston} точны:
	для данного $x$ для любого $r\in[q(x); p(x)]$ найдётся банахов предел
	$B\in\mathfrak{B}$ такой, что $Bx = r$.

Множество таких $x\in\ell_\infty$, что $p(x)=q(x)$,
образует подпространство почти сходящихся последовательностей $ac$~~\cite{lorentz1948contribution}.
На почти сходящейся последовательности все банаховы пределы принимают одинаковые значения.

Через $ac_\lambda$ будем обозначать множество последовательностей, почти сходящихся к $\lambda$,
т.е. таких $x\in\ell_\infty$, что $p(x)=q(x)=\lambda$.

	Е.~А. Алехно  доказал \cite{alekhno2012superposition}, что в пространстве $ac_0$
	существует максимальный (по включению) идеал~-- идеальный
	стабилизатор пространства почти сходящихся к нулю последовательностей,
	обозначаемый $\mathcal I(ac_0)$.


Итак, каждый банахов предел по определению инвариантн относительно оператора сдвига.
Возникает закономерный вопрос: относительно каких ещё операторов инвариантны банаховы пределы?


Уже для совершенно естественного "--- если бы мы говорили об обычной сходимости последовательностей "---
оператора растяжения $\sigma_k$, который
просто повторяет каждый элемент последовательности $k\geq 2$ раз:
\begin{equation}
	\sigma_n (x_1, x_2, x_3, ...) = (
		\underbrace{x_1,...,x_1}_{n~\text{раз}},
		\underbrace{x_2,...,x_2}_{n~\text{раз}},
		\underbrace{x_3,...,x_3}_{n~\text{раз}},
		...)
\end{equation}
выясняется, что относительно этого оператора инварианты не все банаховы пределы~\cite{TODO}!
Более того, инвариантность конкретного банахова предела зависит от выбора $k\in\N$.

Для оператора Чезаро
\begin{equation}
	C (x_1, x_2, x_3, ...) = \left(
	x_1,
	\dfrac{x_1+x_2}2,
	\dfrac{x_1+x_2 + x_3}3,
	\dfrac{x_1+x_2+x_3+x_4}4,
	...,
	\dfrac{x_1+...+x_n}n,
	...\right)
	.
\end{equation}
ситуация с инвариантностью банаховых пределов обстоит ещё <<хуже>>.
Через $\B(H)$ мы обозначаем множество банаховых пределов,
инвариантных относительно оператора $H$,
т.е. таких $B\in \B$, что $B=BH$.
Так, в~\cite[\S2, Theorem 4]{semenov2020dilation} показано, что
\begin{equation}
	\label{eq:B_C_subsetneq_B_sigma_n}
	\B(C) \subsetneq \bigcap_{n\in \N}\B(\sigma_n)
	.
\end{equation}
(это утверждение обобщает~\cite[Theorem 3]{semenov2020invariant_noncommutative}
и~\cite[Theorem 4.8]{ASSU4}).

Важной вехой в изучении инвариантных банаховых пределов явилась следующая теорема,
доказанная в~\cite[\S2]{Semenov2010invariant}.

\begin{theorem}
	\label{thm:Semenov_Sukochev_conditions}
	Пусть линейный оператор $H:\ell_\infty\to\ell_\infty$ таков, что:
	\\(i)   $H \geqslant 0, H \one=\one$;
	\\(ii)  $H c_0 \subset c_0$;
	\\(iii) $\limsup _{j \rightarrow \infty}(A(I-T) x)_j \geqslant 0$ для всех $x \in \ell_{\infty}, A \in G$,
	\\где
	\begin{equation}
		G=G(H):=\operatorname{conv}\left\{H^n, n=1,2, \ldots\right\}
		.
	\end{equation}
	Тогда $\B(H) \ne\varnothing$.
\end{theorem}

In the present paper in Theorem~\ref{thm:Eberlein_but_not_B-regular_exists}
строится пример такого оператора $E$, что $\B(E)\ne\varnothing$,
но $E\one \ne \one$, что показывает существенную избыточность условий теоремы~\ref{thm:Semenov_Sukochev_conditions}.


Достаточно <<плохими>> в смысле инвариантности оказываются крайние точки множества банаховых пределов $\ext \B$.
Это множество не является слабо$^*$ замкнутым~\cite{Nillsen,Talagrand},
и его мощность совпадает~\cite{Chou} с мощностью всего пространства $\ell_\infty^*$
и составляет $2^{\mathfrak c}$.
Из~\cite[Corollary 6]{semenov2020dilation} непосредственно следует, что для любых
$n\in\N_2$, $B_1 \in \B(\sigma_n)$, $B_2 \in \ext\B$ выполнено равенство
$\rho(B_1,B_2) = 2$, т.е. расстояние является максимально возможным.

TODO: reverse invariance (со ссылкой)

%TODO2: тут можно сказать больше про расстояние от $\ext \B$ до $\B(C)$ и прочих там всяких $\B(\sigma_n)$ (а также их объединения?).

Тем не менее, в настоящей главе удалось использовать элементы $\ext\B$ для исследования инвариантных банаховых пределов
"--- в теореме~\ref{thm:amiable_but_not_Eberlein_exists}.


Инвариантные банаховы пределы находят приложения в некоммутативной геометрии
и теории сингулярных следов, где используются в качестве элемента конструкции
различных подклассов следов Диксмье
~\cite{carey2003spectral,lord2012singular,sukochev2015characterization,sukochev2016dixmier}.


При изучении инвариантности банаховых пределов относительно различных непрерывных линейных операторов
возникает закономерный вопрос о выделении некоторых классов этих операторов.

\begin{definition}
	Будем говорить, что оператор $H : \ell_\infty \to \ell_\infty$ \emph{эберлейнов},
	если $\B (H) \ne \varnothing$.
\end{definition}

Выбор именования этого класса операторов обусловлен тем, что именно Эберлейн в работе~\cite{Eberlein}
впервые сколь-либо системно изучил инвариантные банаховы пределы
(хотя отдельные шаги в этом направлении были сделаны ещё Эгнью и Морсом~\cite{agnew1938linear,agnew1938extensions}).

В работе~\cite{alekhno2018invariant} вводится следующее

\begin{definition}
	\label{def:B-regular_operator}
	Оператор $H : \ell_\infty \to \ell_\infty$ называется \emph{B-регулярным},
	если $H^*\B \subseteq \B$.
	(Или, что то же самое, $BH\in\B$ для любого $B\in\B$.)
\end{definition}

В той же работе с помощью теоремы Брауэра--Шаудера--Тихонова о неподвижной точке~\cite[Corollary  17.56]{aliprantis2006infinite}
доказывается следующая
\begin{theorem}
	\label{thm:B-regular_is_Eberlein}
	Любой B-регулярный оператор "--- эберлейнов.
\end{theorem}

%TODO: strong B-regularity; is sprawling  operator strong-B-regular?
%Является, но лучше не надо.

Там же приводится следующее необходимое и достаточное условие В-регулярности.

\begin{theorem}
	\label{thm:crit_B_regularity}
	Оператор $H:\ell_\infty \to \ell_\infty$ является B-регулярным тогда и только тогда, когда:
	\\(i) $H\one \in ac_1$;
	\\(ii) $q(Hx)\geq 0$ для любого $x\geq 0$;
	\\(iii) $H ac_0 \subseteq ac_0$.
\end{theorem}
Легко заметить, что эти условия являются в целом более слабыми, чем достаточные условия эберлейновости
(теорема~\ref{thm:Semenov_Sukochev_conditions}).

\begin{corollary}
	Пусть $x\in ac_\lambda$ и оператор $H$ является В-регулярным.
	Тогда $Hx \in ac_\lambda$.
\end{corollary}

Кроме того, в той же работе~\cite{alekhno2018invariant}
с помощью теоремы Крейна--Мильмана~\cite[Theorem  9.14]{aliprantis2006positive}
доказывается следующая
\begin{lemma}
	Пусть оператор $H:\ell_\infty\to\ell_\infty$ является B-регулярным.
	Тогда множество $\ext\B(H)$ непусто.
\end{lemma}

Возникает закономерный вопрос: существует ли эберлейнов оператор, не являющийся B-регулярным?
Интуитивно кажется, что существует, однако в вопросах, связанных с банаховыми пределами,
многие факты оказываются контринтуитивными.
Пример эберлейнова не В-регулярного оператора строится ниже
в теореме~\ref{thm:Eberlein_but_not_B-regular_exists}.

Подобные рассуждения можно продолжить в обе стороны, а именно "--- следующими двумя определениями:

\begin{definition}
	Будем называть оператор $A:\ell_\infty \to \ell_\infty$ \emph{полуэберлейновым}, если $BA\in\mathfrak B$ для некоторого $B\in\mathfrak B$.
\end{definition}

\begin{definition}
	Будем называть оператор $A:\ell_\infty \to \ell_\infty$ \emph{существенно эберлейновым}, если $BA\in\mathfrak B$ для любого $B\in\mathfrak B$ и $BA\ne B$ для некоторого $B\in\mathfrak B$.
\end{definition}

Эти четыре класса операторов (существенно эберлейновы, В-регулярные, эберлейновы, полуэберлейновы "--- в порядке включения)
получены последовательным ослаблением условий, естественных для <<достаточно хороших>> операторов:
$\sigma_k$, $C$
и образуют иерархию по включению.
Возникает закономерный вопрос о совпадении классов.
Ясно, что оператор сдвига $T$ является В-регулярным, но не является существенно эберлейновым, поскольку относительно него любой банахов предел инвариантен по определению.
Более того, ясно, что операторы растяжения $\sigma_k$, $k\in\N_2$, и оператор Чезаро $C$ являются существенно эберлейновыми,
поскольку для каждого из них существуют как инвариантные, так и неинвариантные банаховы пределы.

Существует ли полуэберлейнов оператор, не являющийся эберлейновым?
Положительный (и конструктивный) ответ на этот вопрос даёт теорема~\ref{thm:amiable_but_not_Eberlein_exists} ниже.



Из определения В-регулярности незамедлительно следует

\begin{lemma}
	\label{lem:B-regular_superposition_and_addition}
	Множество В-регулярных операторов замкнуто относительно суперпозиции и выпуклой комбинации.
\end{lemma}

\begin{lemma}
	Let $H_1$ be a B-regular operator and $H_2$ be a semi-Eberlein operator.
	Then $H_2 \circ H_1$ is a semi-Eberlein operator.
\end{lemma}
\begin{proof}
	Since $H_2$ is Eberlein, $B_1H_2 = B_2$ for some $B_1,B_2\in\B$.
	Then
	\begin{equation}
		B_1 H_2 \circ H_1 = B_2 \circ H_1 = B_3
	\end{equation}
	by the definition of B-regular operator $H_1$,
	that is exactly the definition of semi-Eberlein operator for $H_2 \circ H_1$.
\end{proof}



\section{Sprawling operators}



	Докажем следующую вспомогательную лемму.

	\begin{lemma}
		\label{lem:suff_B_reg}
		Пусть оператор $H$ удовлетворяет следующим условиям:
		\\i)   $H \geq 0$;
		\\ii)  $H\one\in ac_1$;
		\\iii) $H c_0 \subset ac_0$;
		\\iv)  $HT-TH : \ell_\infty \to ac_0$.% для некоторых $k,m \in \N_0$.

		Тогда оператор $H$ является В-регулярным.
	\end{lemma}

	\begin{proof}
		Доказательство проведём непосредственной проверкой определения банахова предела
		(определение~\ref{def:Banach_limit})
		для суперпозиции $B = B_1 H$,
		где $B_1 \in \B$ "--- произвольный банахов предел.

		Действительно, если $H\geq 0$, то $B = B_1 H \geq 0$,
		поскольку $B_1 \geq 0$ по определению банахова предела $B_1$.
		Далее, $B\one = B_1 H \one = 1$, поскольку $H\one \in ac_1$,
		а $B_1(ac_1) = 1$ по определению банахова предела $B_1$.
		Заметим теперь, что
		\begin{equation}
			Bc_0 = B_1 H c_0 = B_1 ac_0 = 0
		\end{equation}
		снова в силу того, что $B_1\in\B$.
		Наконец, для любого $x\in\ell_\infty$ выполнено
		\begin{equation}
			(BT-B)x = BTx - Bx = B_1 HTx - B_1 H x = B_1 HTx - B_1 T H x = B_1 (HT -  T H) x = 0
			.
		\end{equation}
		В силу произвольности выбора $x\in \ell_\infty$ последнее равенство и означает, что $BT=B$.

		Значит, для любого банахова предела $B_1$ функционал $B=B_1 H$ снова есть банахов предел,
		что по определению~\ref{def:B-regular_operator}
		и означает В-регулярность оператора $H$.
	\end{proof}

	\begin{remark}
		Условия леммы~\ref{lem:suff_B_reg} не являются необходимыми;
		так, контрпример к условию (i) очевиден:
		\begin{equation}
			H_1(x_1,x_2,x_3,x_4...) = (-x_1, x_2, x_3, x_4, ...)
			.
		\end{equation}
		Оператор $H_1$ является В-регулярным и, более того, $\B(H_1) = \B$.
	\end{remark}


	Before we introduce the next notion, let us remind the following result~\cite[теорема 1]{usachev2008transforms} (приводим её в эквивалентной формулировке и более удобных обозначениях).


	\begin{theorem}
		Определим линейный оператор $H:\ell_\infty\to\ell_\infty$ равенством
		\begin{equation}
			H(x_1, x_2, ...) = \left(x_{m_1}, x_{m_1+1}, \ldots, x_{m_1+n_1} ; x_{m_2}, x_{m_2+1}, \ldots, x_{m_2+n_2} ; \ldots\right)
			,
		\end{equation}
		где $m_k, n_k$ "--- последовательности натуральных чисел и $\lim_{n\to\infty} n_k = \infty$.
		Пусть $x\in ac_s$, $s\in\R$. Тогда $Hx \in ac_s$.
	\end{theorem}


	\begin{definition}
		Разреженным (sprawling) назовем оператор $H: \ell_\infty \to \ell_\infty$,
		определяемый следующим образом
		$$
		H(x_1, x_2, x_3, \ldots) = (x_{m_1}, x_{m_1 + 1}, \ldots, x_{n_1 - 1}, x_{n_1};
		x_{m_2}, x_{m_2 + 1}, \ldots, x_{n_2 - 1}, x_{n_2}; \ldots),
		$$
		где для всех
		$k \geqslant k_0 \in \mathbb N$
		выполнено
		\begin{equation}
			\label{eq:sprawling_mk_nk}
			m_k \leqslant n_k < m_{k+1}\leqslant n_{k+1},
		\end{equation}
		при этом
		$$
			\lim\limits_{k \to \infty} (n_k - m_k) = \infty.
		$$
	\end{definition}



	\begin{theorem}
		Оператор $H$ является $B$-регулярным.
	\end{theorem}

	\begin{proof}
		Непосредственно проверим условия леммы 3.10.3.
		\begin{enumerate}
			\item[i)] From the definition, one can easily see that $H\geq 0$.
			\item[ii)] $H \one = \one \in ac_1$.
			\item[iii)] The inclusion $Hc_0 \subset c_0 \subset ac_0$ holds due to the inequality~\eqref{eq:sprawling_mk_nk}.
			\item[iv)] Наконец, покажем, что $HT - TH : \ell_\infty \to ac_0$. Имеем
			\begin{align*}
				THx &= T(x_{m_1}, x_{m_1 + 1}, \ldots, x_{n_1 - 1}, x_{n_1};
				x_{m_2}, x_{m_2 + 1}, \ldots, x_{n_2 - 1}, x_{n_2}; \ldots) = \\
				&= (x_{m_1 + 1}, x_{m_1 + 2}, \ldots, x_{n_1 - 1}, x_{n_1};
				x_{m_2}, x_{m_2 + 1}, \ldots, x_{n_2 - 1}, x_{n_2}; x_{m_3}, x_{m_3 + 1}\ldots), \\
				HTx &= ((Tx)_{m_1}, (Tx)_{m_1 + 1}, \ldots, (Tx)_{n_1 - 1}, (Tx)_{n_1}; \\
				&(Tx)_{m_2}, (Tx)_{m_2 + 1}, \ldots, (Tx)_{n_2 - 1}, (Tx)_{n_2}; \ldots) = \\
				&= (x_{m_1 + 1}, x_{m_1 + 2}, \ldots, x_{n_1 - 1}, x_{n_1}, x_{n_1 + 1}; x_{m_2 + 1},
				\ldots, x_{n_2 - 1}, x_{n_2}, x_{n_2 + 1}; x_{m_3 + 1} \ldots),
			\end{align*}
			где $x \in \ell_\infty$. Отсюда
			\begin{align*}
				&(HT - TH)x = HTx - THx = \\
				&= (\underbrace{0, 0, \ldots, 0, 0}_{n_1 - m_1}, x_{n_1 + 1} - x_{m_2}, \underbrace{0, \ldots,
					0, 0}_{m_2 - n_2}, x_{n_2 + 1} - x_{m_3}, 0, \ldots) \in \mathcal I(ac_0) \subset ac_0,
			\end{align*}
			т. к. $\lim_{k \to \infty} (n_k - m_k) = \infty$.
		\end{enumerate}

		Таким образом, все условия леммы 3.10.3 выполнены, и оператор $H$ действительно является $B$-регулярным.
	\end{proof}


	{\bf Следствие 3.8.4.1.} Пусть $x \in ac_\lambda$ и оператор $H$ является $B$-регулярным. Тогда $Hx \in ac_\lambda$.

	\begin{corollary}
		Оператор $H$ переводит $ac_\lambda$ в $ac_\lambda$, $\lambda \in \mathbb R$.
	\end{corollary}

	\begin{proof}
		Вытекает непосредственно из следствия 3.8.4.1 в силу $B$-регулярности оператора $H$.
	\end{proof}

	We finish the section with a brief discussion on a non-sprawling operator
	that nevertheless turns to be B-regular due to Lemma~\ref{TODO}.

	Рассмотрим теперь оператор,
	в чём-то похожий на операторы растяжения,
	но при этом растяжения <<неравномерного>>.
	Пусть
	\begin{equation}
		\sigma_\triangle x =
		(x_1; \; x_1; x_2; \; x_1; x_2; x_3; ... ; \underbrace{x_1; ...; x_n}_{n~\mbox{элементов}}; ...)
		.
	\end{equation}

	\begin{remark}
		Оператор $\sigma_\triangle x$ естественным образом возник в работах А.С. Усачёва~\cite{usachev2009_phd_vsu},
		где обозначался $\overline{x}$ и
		использовался для изучения коэффициентов Фурье--Хаара
		и расстояния от произвольной последовательности $x\in\ell_\infty$
		до пространства $ac$.
	\end{remark}

	Нам потребуется один факт об операторе $\sigma_\triangle$,
	доказанный А.С. Усачёвым в~\cite[теорема 19]{usachev2009_phd_vsu}.
	Мы приведём его эквивалентную формулировку, более удобную для нас в дальнейшем.

	\begin{theorem}
		\label{thm:usachev_overline_x_ac_s}
		Пусть $x\in\ell_\infty$, $s\in\R$.
		Тогда условия $x\in ac_s$ и $\sigma_\triangle x \in ac_s$ эквивалентны.
	\end{theorem}

	\begin{theorem}
		Оператор $\sigma_\triangle$ является В-регулярным.
	\end{theorem}

	\begin{proof}
		Снова воспользуемся леммой~\ref{lem:suff_B_reg}.
		Легко заметить, что $\sigma_\triangle \geq 0$ и $\sigma_\triangle \one  = \one$,
		что обеспечивает выполнение условий (i) и (ii) соответственно.

		Выполнение условия (iii) непосредственно следует из теоремы~\ref{thm:usachev_overline_x_ac_s}.

		Наконец, покажем, что $\sigma_\triangle T - T \sigma_\triangle  : \ell_\infty \to ac_0$.
		Действительно, для произвольного $x\in\ell_\infty$
		\begin{multline}
		(\sigma_\triangle T - T \sigma_\triangle)x =
		\sigma_\triangle Tx - T \sigma_\triangle x =
		\\
		(x_2; \; x_2, x_3; \; x_2, x_3,   x_4; \; x_2, x_3, x_4, x_5; \; ... )-
		\\
		(x_1, x_2; \; x_1,    x_2, x_3; \; x_1,   x_2, x_3, x_4; \; x_1, ... )=
		\\
		(x_2-x_1; 0; x_3-x_1; 0, 0, x_4-x_1; 0,0,0; x_5-x_1; ...) \in\Iac\subset ac_0
		.
		\end{multline}

		Таким образом, все условия леммы~\ref{lem:suff_B_reg} выполнены,
		и оператор $\sigma_\triangle$ действительно является В-регулярным.
	\end{proof}



\section{A semi-Eberlein operator that is not Eberlein}
	\begin{theorem}
		\label{thm:amiable_but_not_Eberlein_exists}
		Существует полуэберлейнов оператор, не являющийся эберлейновым.
	\end{theorem}

	\begin{proof}
		Пусть $B_1, B_2 \in \ext \B$, $B_1 \ne B_2$.
		Положим
		\begin{equation}
			\label{eq:am_not_eber_def}
			B_3 = B_1 + 2(B_2-B_1) = 2B_2-B_1,
		\end{equation}
		тогда $B_3 \notin \B$.
		Действительно, если $B_3 \in \B$, то из~\eqref{eq:am_not_eber_def} следует, что
		\begin{equation}
			B_2 = \frac{B_1 + B_3}2 \in \B \setminus \ext \B
			.
		\end{equation}
		Введём в рассмотрение оператор $H:\ell_\infty\to\ell_\infty$, определённый равенством
		\begin{equation}
			2Hx = (x_1 + B_3x, x_2 + B_3x, ...) = x + (B_3 x) \cdot \one
			.
		\end{equation}
		Убедимся, что оператор $H$ дружелюбен.
		Действительно, для произвольного $x\in\ell_\infty$ имеем
		\begin{equation}
			2 B_1 H x = B_1 x + B_1 ((B_3 x) \cdot\one) = B_1 x + B_3 x =
			B_1 x + 2 B_2 x - B_1 x = 2B_2 x
			,
		\end{equation}
		откуда $B_1 H = B_2$.

		Убедимся теперь, что оператор $H$ не эберлейнов.
		Пусть $B = BH$ для некоторого $B\in\B$.
		Тогда для всех $x\in\ell_\infty$ имеем
		\begin{equation}
			2Bx = B (x + (B_3 x) \cdot \one)
			,
		\end{equation}
		т.е.
		\begin{equation}
			Bx =  B((B_3 x) \cdot \one)
			,
		\end{equation}
		откуда незамедлительно следует, что $Bx = B_3x$ и в силу произвольности выбора $x$ имеем $B=B_3$.
		Но ранее мы уже показали, что $B_3\notin \B$.
		Полученное противоречие завершает доказательство.
	\end{proof}

	\begin{remark}
		Интересно, что $B_2H\notin \B$.
		Действительно,
		\begin{equation}
			\label{eq:amiable_B2}
			2B_2 H x = B_2 x + B_2((B_3 x)\cdot \one) = B_2 x + B_3 x  =
			(B_2 + B_3) x  = (B_2+2B_2-B_1)x = (3B_2-B_1)x
			.
		\end{equation}
		Так как равенство~\eqref{eq:amiable_B2} верно для любого $x\in\ell_\infty$,
		то мы можем сделать вывод, что
		\begin{equation}
			2B_2H = 3B_2-B_1
			,
		\end{equation}
		откуда
		\begin{equation}
			3B_2 = 2B_2H  + B_1
			,
		\end{equation}
		т.е. если $B_2H\in\B$, то
		\begin{equation}
			B_2 = \frac23 B_2H  + \frac13 B_1 \notin \ext\B
			,
		\end{equation}
		что противоречит нашему построению.
	\end{remark}

	\begin{hypothesis}
		Существуют такие полуэберлейнов оператор $H:\ell_\infty\to\ell_\infty$ и банахов предел $B\in \B$,
		что $BH \in \B$, но $B_1 H \notin \B$ для любого $B_1\in \B\setminus\{B\}$.
		Также интересно наложение дополнительных свойств на $B$ и $BH$, например, принадлежности к $\ext\B$, $\B(C)$ и т.д.
	\end{hypothesis}


\section{An Eberlein operator that is not B-regular}

\begin{lemma}
	\label{lem:R_is_sprawling}
	Let $R$ be a subsequence operator defined as
	\begin{multline}
		\label{eq:sprawling_R}
		Rx = (x_2; \; x_4; x_5; \; x_{2^k}; x_{2^k + 1}; ... ; x_{2^k + 2^{k-1} - 1}; \; x_{2^{k+1}}; x_{2^{k+1} + 1};...)
		=\\
		(x_2; \; x_4; x_5; \; x_8; x_9; x_{10}; x_{11}; \; x_{16}; x_{17}; ...)
		.
	\end{multline}
	Operator $R$ is sprawling.
\end{lemma}

\begin{proof}
	In the definition of the sprawling operator, we take $m_k = 2^k$ and $n_k = 2^k + 2^{k-1} - 1$.
\end{proof}

\begin{theorem}
	\label{thm:Eberlein_but_not_B-regular_exists}
	Существует эберлейнов оператор, не являющийся В-регулярным.
\end{theorem}

\begin{proof}
	Пусть оператор $R$ определён формулой~\eqref{eq:sprawling_R}.
	Пусть оператор $E:\ell_\infty\to\ell_\infty$ определён формулой
	\begin{equation}
		Ex = x \cdot \chi\cup_{k=1}^{\infty}\left[2^{k}, 2^k + 2^{k-1} - 1\right]
		.
	\end{equation}
	Тогда, очевидно, $RE=R$.
	Поскольку $R$ есть В-регулярный оператор, то он эберлейнов,
	т.е. множество $\B(R)$ непусто.

	Пусть $B\in\B(R)$. Тогда
	\begin{equation}
		B = BR = B(RE) = (BR)E = BE
		,
	\end{equation}
	то есть $B\in\B(E)$.

	Покажем теперь, что оператор $E$ не является В-регулярным.
	Действительно, $q(E\one) =0$, т.е. $ E\one \notin ac_1$,
	и не выполнено условие (i) критерия В-регулярности (теорема~\ref{thm:crit_B_regularity}).
\end{proof}

\begin{hypothesis}
	Выполнено равенство
	\begin{equation}
		\B(E) = \B(R)
		.
	\end{equation}
\end{hypothesis}

\begin{remark}
	В конструкции оператора $E$ вместо блоков из нулей можно вставлять любые другие блоки "---
	они всё равно будут поглощены оператором $R$.
	Эти блоки могут иметь различный знак, например, содержать значительное количество
	элементов вида $-kx_1$, $kx_2$ и т.д.
	Таким образом, можно сконструировать эберлейновы операторы, очень и очень далёкие
	от достаточных условий эберлейновости (теорема~\ref{thm:Semenov_Sukochev_conditions}),
	что показывает значительную избыточность этих условий.
\end{remark}




\section{The signature of a Banach limit}

{Профиль банахова предела. Построение оператора, инвариантные БП для которого имеют профиль почти-ноль или почти-единицу (вроде есть в черновиках, надо набрать).}

In~\cite{TODO}, the following functional characteristic of a Banach limit was introduced:

 Обозначим через $\Gamma$ множество всех неубывающих на $[0, 1]$ функций $f$ таких,
    что $f(0) = 0$ и $f(1) = 1$. Каждому $B \in \mathfrak B$
    ставится в соответствие следующая функция, определенная на $[0, 1]$:
    $$
		\gamma(B, t) = B \chi\Bigg(\bigcup^\infty_{n = 1} [2^n, 2^{n + t})\Bigg)
		.
    $$
    Функция $\gamma(B,t)$ была введена в статье \cite{semenov2019main}.

    Легко увидеть, что $\gamma (B, t) \in \Gamma$ для всех $B \in \mathfrak B$.
    Справедливо и обратное. Для любой $f \in \Gamma$ существует такой
	$B \in \mathfrak B$, что $\gamma(B, t) = f(t)$ для всех
	$t \in [0, 1]$~\cite[Предложение 2]{semenov2019main}.
    For the sake of brevity, we will can $\gamma(B,t)$ the \emph{signature}
    of Banach limit $B$.

	Другая функциональная характеристика -- аналог функции распределения -- была введена в работе Е.~А.~Алехно~\cite{Alekhno2}.

	В силу монотонности сигнатура имеет не более чем счётное количество точек разрыва.
	In~\cite[Теорема 23]{semenov2019main}, it is proved that
	для любых $B \in \mathfrak B(C)$, $t \in [0, 1]$ верно равенство $\gamma (B, t) = t$.




In this section, we provide the examples of Banach limits for that the signature is almost constant.

\begin{lemma}
	Let R be the subsequence operator
	defined by~\eqref{eq:sprawling_r}.
	Then $\B(R) \ne\varnothing$ and for every $B\in \B(R)$ we have
	\begin{equation}
		\gamma(B, t) = \chi(0;1]
		.
	\end{equation}
\end{lemma}

\begin{proof}
	Due to Lemma~\ref{lem:R_is_sprawling}, $R$ is sprawling.
	Thus, $R$ is B-regular by Theorem~\ref{TODO} and thus Eberlein due to Theorem~\ref{TODO},
	so $\B(R)\ne \varnothing$.

	Take any $B \in \mathfrak{B}(R)$ and
	\begin{equation}
		y = \chi \cup_{n=1}^\infty \left[ 2^n; 2^{n-1+\log_2 3}\right) =
		\chi \{ 2; \; 4;5; \; 8;9;10;11; \; 16;17;... \}\quad]
		,
	\end{equation}
	then $Ry=\one$.

	Следовательно, по определению сигнатуры, $\gamma(B,t) = 1$ для всех $t \geq \log_2 3 - 1 = \log_2 \frac32$.

	Далее, $R^2 x = (x_4; \; x_8; x_9;\; x_{16}; x_{17}; x_{18}; x_{19}; \; x_{32}; x_{33};...)$.
	В силу вложения $\mathfrak{B}(R)\subset\mathfrak{B}(R^2)$

	TODO: всегда ли оно строгое?

	имеем $B\in\mathfrak{B}(R^2)$.
	Легко убедиться, что $\gamma(B,t) = 1$ для всех $t \geq \log_2 \frac54$.

	TODO: Дальше через всякие там пределы доказывается, что

	\begin{equation}
		\label{eq:gamma_chi_0_1}
		\gamma(B,t) = \chi{[}0;1)
	\end{equation}


	%Ясно, что $B\in\mathfrak{B}(R)$ ~--- достаточное условие для~\eqref{eq:gamma_chi_0_1}.
	%TODO: является ли оно необходимым?



\end{proof}

%\begin{theorem}
%НЕВЕРНО!!!
%	Вложение $ac(R) \subset \ell_\infty$ является собственным.
%\end{theorem}

%\begin{proof}
	Определим последовательность $e\in\ell_\infty$ соотношением
	\begin{equation}
		e_k = (-1)^n, \quad 2^n \leq k < 2^{n+1}
		.
	\end{equation}
	Тогда $R^n e = (-1)^n e \notin ac$.
%\end{proof}

\begin{remark}
	\B(R)(e) =\{0\}
	.
\end{remark}

TODO: а можно ли скачок в произвольной точке?

TODO: про диадический фрактал

\printbibliography{}

\end{document}
