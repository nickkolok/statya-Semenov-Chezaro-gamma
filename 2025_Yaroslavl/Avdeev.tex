%эти команды менять нельзя
\documentclass[14pt, a4paper]{extbook}
\usepackage{amsfonts}
\usepackage[utf8]{inputenc} %кодировка должна быть такой
\usepackage[T1,T2A]{fontenc}
\usepackage{mathtext}
\usepackage{amsmath, amsfonts, amssymb}
\usepackage[english, russian]{babel}
\usepackage[body={14cm, 22.7cm},left= 3.53cm, top=3.53cm]{geometry}
\usepackage{indentfirst}
\usepackage[]{graphicx}
\usepackage{floatflt}
\usepackage{amsthm}



%команды, которыми надо пользоваться для оформления теорем, лемм, утверждений, предположений, гипотез и определений
\newtheorem{theorem}{Теорема}
\newtheorem{lemma}{Лемма}
\newtheorem{statement}{Утверждение}
\newtheorem{assumption}{Предположение}
\newtheorem{hypothesis}{Гипотеза}
\newtheorem{definition}{Определение}

%сам документ с тезисами
\begin{document}

\begin{center}%в этой шапке (заголовке) надо вставить данные. Оформление менять НЕ надо.
\textbf{\large{} Диадические последовательности и~почти~сходимость} %название работы

\textit{\textbf{Авдеев Н.~Н. (ВГУ, Воронеж)}} %Фамилия И.О. автора1 (ВУЗ, Город)

\textit{Научный руководитель: д.ф.-м.н., профессор Семенов~Е.~М.} %степень, звание (которое присудила ВАК) Фамилия И.О. научного руководителя
\end{center}


Всякая сходящаяся последовательность ограничена,
но не всякая ограниченная последовательность имеет предел.
Этот факт, хорошо известный из курса математического анализа,
приводит к вопросу об операции, подобной взятию предела,
но применимой к более широкому множеству последовательностей.


Пусть $\ell_\infty$ есть пространство ограниченных последовательностей с обычной нормой
$
	\|x\| = \sup_{k\in\mathbb{N}} |x_k|
	,
$
где $\mathbb{N}$ "--- множество натуральных чисел, и обычной полуупорядоченностью.
Одним из естественных обобщений операции взятия предела с пространства сходящихся последовательностей $c$ сразу на всё пространство $\ell_\infty$
является понятие банахова предела.

\begin{definition}
	Линейный функционал $B\in \ell_\infty^*$ есть банахов предел (пишем: $B \in \mathfrak{B}$),
	если
	\\(1)
		$B\geq0$, т.~е. $Bx \geq 0$ для $x \geq 0$,
	\\(2)
		$B\mathbb{I}=1$, где $\mathbb{I} =(1,1,\ldots)$,
	\\(3)
		$B(Tx)=B(x)$ для всех $x\in \ell_\infty$, где $T$~---
		оператор сдвига, т.~е. $T(x_1,x_2,\ldots)=(x_2,x_3,\ldots)$.
\end{definition}

Существование банаховых пределов анонсировано в 1929\,г. С.\,Мазуром~[1];
доказательство приведено в книге С.\,Банаха~[2].
Банаховы пределы строятся на основе теоремы Хана--Банаха,
их более чем континуум.
Однако Лоренцем [3] было установлено,
что существуют последовательности,
на которых все банаховы пределы принимают одинаковое значение.



\begin{theorem}[{Критерий Лоренца~[3]}]
%\textbf{Критерий Лоренца.}~\cite{lorentz1948contribution}
%{\sl
	Для заданных вещественного $t$ и последовательности $x\in\ell_\infty$ равенство $Bx=t$ выполнено для всех $B\in\mathfrak{B}$
	(для краткости пишут: $x\in ac$ и при необходимости уточняют: $x\in ac_t$)
	тогда и только тогда, когда
	\begin{equation*}
		\label{eq:crit_Lorentz}
		\lim_{n\to\infty} \frac{1}{n} \sum_{k=m+1}^{m+n} x_k = t
		\qquad\qquad\mbox{равномерно по $m\in\mathbb N$.}
	\end{equation*}
%}
\end{theorem}

Равномерный предел в критерии Лоренца можно заменить на двойной~[4, Теорема 1].
Последовательности, принадлежащие пространству $ac$, называют почти сходящимися.

Ту же роль, которую для сходимости играют нижний и верхний пределы,
для почти сходимости играют функционалы Сачестона:
	\begin{equation*}
		q(x) = \lim_{n\to\infty} \inf_{m\in\mathbb{N}}  \frac{1}{n} \sum_{k=m+1}^{m+n} x_k,
		~~~~~~~~
		p(x) = \lim_{n\to\infty} \sup_{m\in\mathbb{N}}  \frac{1}{n} \sum_{k=m+1}^{m+n} x_k.
	\end{equation*}

\begin{theorem}[{Сачестон, [5]}]
	Для любых $x\in \ell_\infty$ и $B\in\mathfrak{B}$
	выполнено
	$
		q(x) \leqslant Bx \leqslant p(x)
	$;
	для данного $x$ для любого $r\in[q(x); p(x)]$ найдётся банахов предел
	$B\in\mathfrak{B}$ такой, что $Bx = r$.
\end{theorem}



Представляет интерес вопрос об объектах, в некотором смысле двойственных пространству $ac$.
Множество $Q\subset\ell_\infty$ называют разделяющим~[6, \S 3], %\cite{Semenov2014geomprops},
если для любых неравных $B_1, B_2\in\mathfrak{B}$ существует такая последовательность $x\in Q$,
что $B_1 x \neq B_2 x$.

В частности, разделяющим является~[7] %\cite{semenov2010characteristic}
множество всех последовательностей из 0 и 1
(обозначим его через $\Omega$).
Последовательности, элементы которых могут принимать только два значения,
часто рассматриваются с точки зрения почти сходимости.
Такие последовательности называют диадическими.
Связь между диадическими последовательностями и почти сходимостью
устанавливает следующая
\begin{lemma}
	\label{Avdeev_decomposition_lemma}
	Пусть $x\in\ell_\infty$, $\|x\|\leq 1$.
	Тогда существует такая последовательность $h\in ac_0$, что $(x+h)_n = \pm 1$ для всех $n\in\mathbb N$.
\end{lemma}

Из леммы~\ref{Avdeev_decomposition_lemma} результат из [7] о том, что множество $\Omega$
является разделяющим, получается непосредственно.
За более подробным обсуждением разделяющих множеств и функционалов Сачестона
отсылаем читателя к [8].

Ещё одним классическим примером диадической последовательности
является характеристическая функция множества кратных.
	Каждый $x\in \Omega$ отождествим с подмножеством множества натуральных чисел
	$\operatorname{supp} x \subset \mathbb{N}$.
	Вслед за~[9] обозначим %{hall1992behrend}
	$
		\mathcal{M}A = \{ka: k\in\mathbb{N}, a\in A\}
		,
	$
	через $\chi F$ "--- характеристическую функцию множества $F$.

	\begin{definition}
		Будем говорить, что множество $A\subset\mathbb{N}$ обладает $P$-свойством,
		если для любого $n\in\mathbb{N}$ найдётся набор попарно взаимно простых чисел
		$
			\{a_{n,1}, a_{n,2}, ..., a_{n,n}  \} \subset A
			.
		$
	\end{definition}

	\begin{theorem}
		Пусть $A\subset \mathbb{N}\setminus\{1\}$.
		Тогда эквивалентны условия:
		\\(i)
			$A$ обладает $P$-свойством;
		\\(ii)
			В $A$ существует бесконечное подмножество попарно взаимно простых чисел;
		\\(iii)
			$p(\chi\mathcal{M}A)=1$.
	\end{theorem}

За подробностями отсылаем читателя к [10].%{avdeev2021vmzprimes}



{\small Исследование выполнено при финансовой поддержке РНФ в рамках научного проекта No 24-21-00220.} %если исследование поддержано грантом, то здесь указывается ссылка

%Для оформления списка литературы, пожалуйста, используйте следующий образец.

\begin{center}
{\small \bf Литература}
\end{center}
\small
	{[1]}
	%\bibitem{Mazur}
	{Mazur~S.} O metodach sumowalności. // Ann. Soc. Polon. Math.
	(Supple\-ment). 1929. с. 102--107.
	\\
	{[2]}
	%\bibitem{banach2001theory_rus}
	{Банах~С.} Теория линейных операций. Рег. и хаот.
	динамика. М., 2001.
	\\
	{[3]}
	%\bibitem{lorentz1948contribution}
	{Lorentz~G.\,G.} A contribution to the theory of divergent sequences.
	// Acta Mathematica. 1948. т. 80, No 1. с. 167--190.
	DOI: 10.1007/BF02393648
	\\
	{[4]}
	%\bibitem{zvol2022ac}
	Зволинский~Р.\,Е. Почти сходящиеся последовательности и операция
	возведения в положительную степень. // Вестн. ВГУ. Сер.: Физика.
	Математика. 2022. т. 3. с. 76--81.
	\\
	{[5]}
	%\bibitem{sucheston1967banach}
	{Sucheston~L.} Banach limits. // Amer. Math. Monthly. 1967.
	т. 74. с. 308—311. ISSN 0002-9890. DOI: 10.2307/2316038
	\\
	{[6]}
	%\bibitem{Semenov2014geomprops}
	{Семёнов~Е. М.}, {Сукочев~Ф. А.},
	{Усачев~А. С.} Геометрические свойства множества банаховых
	пределов. // Известия Российской академии наук. Серия математическая.
	2014. т. 78, No 3. с. 177—204.
	\\
	{[7]}
	%\bibitem{semenov2010characteristic}
	{Семенов~Е. М.}, {Сукочев~Ф. А.}
	Характеристические функции банаховых пределов. // Сибирский математический
	журнал. 2010. т. 51, No 4. с. 904—910.
	\\
	{[8]}
	%\bibitem{avdeev2021vestnik}
	{Авдеев~Н. Н.} О разделяющих множествах меры нуль и
	функционалах Сачестона. // Вестн. ВГУ. Сер.: Физика. Математика. 2021.
	т. 4. с. 38—50.
	\\
	{[9]}
	%\bibitem{hall1992behrend}
	{Hall~R. R.}, {Tenenbaum~G.} On Behrend sequences.
	// Mathematical Proc. of the Cambridge Philosophical Society. т. 112.
	Cambridge University Press. 1992. с. 467—482. DOI: 10.1017/S0305004100071140
	\\
	%\bibitem{avdeev2021vmzprimes}
	{[10]}
	Авдеев~Н.\,Н. Почти сходящиеся последовательности из 0 и 1 и
	простые числа. // Владикавказский математический журнал. 2021. т. 23,
	No 4. с. 5—14. DOI: 10.46698/p9825-1385-3019-c

\end{document}



