%эти команды менять нельзя
\documentclass[14pt, a4paper]{extbook}
\usepackage{amsfonts}
\usepackage[utf8]{inputenc} %кодировка должна быть такой
\usepackage[T1,T2A]{fontenc}
\usepackage{mathtext}
\usepackage{amsmath, amsfonts, amssymb}
\usepackage[english, russian]{babel}
\usepackage[body={14cm, 22.7cm},left= 3.53cm, top=3.53cm]{geometry}
\usepackage{indentfirst}
\usepackage[]{graphicx}
\usepackage{floatflt}
\usepackage{amsthm}



%команды, которыми надо пользоваться для оформления теорем, лемм, утверждений, предположений, гипотез и определений
\newtheorem{theorem}{Теорема}
\newtheorem{lemma}{Лемма}
\newtheorem{statement}{Утверждение}
\newtheorem{assumption}{Предположение}
\newtheorem{hypothesis}{Гипотеза}
\newtheorem{definition}{Определение}

%сам документ с тезисами
\begin{document}

\begin{center}%в этой шапке (заголовке) надо вставить данные. Оформление менять НЕ надо.
\textbf{\large{} Почти сходимость и возведение в степень} %название работы

\textit{\textbf{Зволинский Р.~Е. (ВГУ, Воронеж)}} %Фамилия И.О. автора1 (ВУЗ, Город)

\textit{\textbf{Авдеев Н.~Н. (ВГУ, Воронеж)}} %Фамилия И.О. автора1 (ВУЗ, Город)

\textit{Научный руководитель: д.ф.-м.н., профессор Семенов~Е.~М.} %степень, звание (которое присудила ВАК) Фамилия И.О. научного руководителя
\end{center}


Из курса математического анализа хорошо известно,
что если последовательность $x=(x_1, x_2, x_3, ...)$ сходится
(пишут: $x\in c$), то она ограничена (пишут: $x\in \ell_\infty$);
обратное неверно, т.е. $\ell_\infty\setminus c \ne \varnothing$.
Это несложный факт подталкивает к изучению всевозможных пространств $Y$, где $c \subsetneq Y \subsetneq \ell_\infty$;
или же, учитывая тривиальное разложение в прямую сумму $c=c_0 \oplus \mathbb R$,
где $c_0$ есть пространство последовательностей, сходящихся к нулю,
а $\mathbb R$ есть множество вещественных чисел ~---
к изучению пространств $Y$, удовлетворяющих соотношению $c_0 \subsetneq Y \subsetneq \ell_\infty$, $Y\ne c$.

Одним из классических пространств такого типа является пространство почти сходящихся последовательностей $ac$,
введённое Лоренцем в [1].
Для того, чтобы дать этому пространству естественное определение, нам потребуется понятие банахова предела.

	Линейный функционал $B\in \ell_\infty^*$ есть банахов предел (пишем: $B \in \mathfrak{B}$),
	если
	\begin{enumerate}
		\item
			$B\geq0$, т.~е. $Bx \geq 0$ для $x \geq 0$,
		\item
			$B\mathbb{I}=1$, где $\mathbb{I} =(1,1,\ldots)$,
		\item
			$B(Tx)=B(x)$ для всех $x\in \ell_\infty$, где $T$~---
			оператор сдвига, т.~е. $T(x_1,x_2,\ldots)=(x_2,x_3,\ldots)$.
	\end{enumerate}

Существование банаховых пределов анонсировано в 1929\,г. С.\,Мазуром~[2];
доказательство приведено в книге С.\,Банаха~[3].

\begin{theorem}[{Критерий Лоренца~[1]}]
%\textbf{Критерий Лоренца.}~\cite{lorentz1948contribution}
%{\sl
	Для заданных $t\in\mathbb R$ и $x\in\ell_\infty$ равенство $Bx=t$ выполнено для всех $B\in\mathfrak{B}$
	(для краткости пишут: $x\in ac$ и при необходимости уточняют: $x\in ac_t$)
	тогда и только тогда, когда
	\begin{equation*}
		\label{eq:crit_Lorentz}
		\lim_{n\to\infty} \frac{1}{n} \sum_{k=m+1}^{m+n} x_k = t
		\qquad\qquad\mbox{равномерно по $m\in\mathbb N$.}
	\end{equation*}
%}
\end{theorem}

Равномерный предел в критерии Лоренца можно заменить на двойной~[4, Теорема 1].

Для любой сходящейся последовательности $x\in c$ и любого $\lambda > 0$ верно, что
$x^\lambda = (x_1^\lambda,x_2^\lambda,x_3^\lambda,...) \in c$.
Для пространства $ac$, однако, верны лишь существенно более слабые утверждения.

	%Следующие результаты опубликованы в~\cite{zvol2022ac}.

	\begin{theorem}
		\label{thm:Zvol_pow_pos}
		Пусть $x \geqslant 0, x \in a c_0$ и $\lambda>0$, тогда $x^\lambda \in a c_0$.
	\end{theorem}

	\begin{theorem}
		\label{thm:Zvol_pow_composed}
		Пусть $x\in \mathcal{I}(ac_0)$.
		Пусть $n\in\mathbb{N}$ или $n = \frac1{2k+1}$, $k\in\mathbb{N}$.
		Тогда $x^n \in ac_0$.
	\end{theorem}

	Напомним, что $x\in \mathcal{I}(ac_0)$ тогда и только тогда, когда $x = x^+ +x^-$, $x^+\geq 0$, $x^- \leq 0$ и $x^+ \in ac_0$
	(последнее включение эквивалентно условию $x^- \in ac_0$).

	Условия предыдущей теоремы существенны.

	\begin{lemma}
		\label{thm:ac0_pow_even}
		Пусть $x\in ac_0 \setminus \mathcal{I}(ac_0)$, т.е. $x = x^+ +x^-$, $x^+\geq 0$, $x^- \leq 0$ и $x^+ \notin ac_0$
		(последнее условие эквивалентно условию $x^- \notin ac_0$).
		Пусть $n = 2k$, $k\in\mathbb{N}$.
		Тогда $x^n \notin ac_0$.
	\end{lemma}






		Для нечётной степени условие разложения в сумму двух знакопостоянных
		почти сходящихся последовательностей в теореме тоже существенно.
		Например, пусть
		\begin{equation*}
			x = (1;1;-2;\ 1;1;-2;\ 1;1;-2;\ ...) \in ac_0
			,
			\quad
			\lambda = 3
			.
		\end{equation*}
		Тогда
		\begin{equation*}
			x^+ = (1;1;0;\ 1;1;0;\ 1;1;0;\ ...) \notin ac_0, \quad x^+ \in ac_{2/3}
			,
		\end{equation*}
		\begin{equation*}
			x^- = (0;0;-2;\ 0;0;-2;\ 0;0;-2;\ ...) \notin ac_0, \quad x^- \in ac_{-2/3}
			,
		\end{equation*}
		\begin{equation*}
			x^\lambda = (1;1;-8;\ 1;1;-8;\ 1;1;-8;\ ...) \notin ac_0, \quad x^\lambda \in ac_{-2}
			.
		\end{equation*}
	%\end{example}



Возведение в нечётную степень может выводить не только из $ac_0$,
но и из $ac$.

Пример.
	%\label{example:cube_out_of_ac0}
	%Напомним, что $\mathbb{N}_k = \{k, k+1, k+2, k+3,...\}$.
	Пусть
	$
		x_n = \begin{cases}
			 0, & \mbox{~если~} n < 2^{10},
			\\
			 1, & \mbox{~если~} n \ge 2^{10}, 2^k\le n < 2^k+3k
			\\
				& \mbox{~и~}  n\neq 2^k + 3m, m\in\mathbb{N},
			\\
			-2, & \mbox{~если~} n \ge 2^{10}, 2^k\le n < 2^k+3k
			\\
				& \mbox{~и~}  n  =  2^k + 3m, m\in\mathbb{N}.
		\end{cases}
	$
	%Таким образом,
	\begin{multline*}
		x = (0,0,...,0,0, \; -2, 1, 1, \; -2, 1, 1, \; -2, 1, 1, ..., -2, 1, 1, \; 0, 0, 0, \\ ..., 0, 0, 0, ..., -2, 1, 1, \; -2, 1, 1, ... ) \in ac_0
		.
	\end{multline*}
	С другой стороны,
		\begin{equation*}
		(x^3)_n = \begin{cases}
			 0, & \mbox{~если~} n < 2^{10},
			\\
			 1, & \mbox{~если~} n \ge 2^{10}, 2^k\le n < 2^k+3k
			\\
				& \mbox{~и~}  n\neq 2^k + 3m, m\in\mathbb{N},
			\\
			-8, & \mbox{~если~} n \ge 2^{10}, 2^k\le n < 2^k+3k
			\\
				& \mbox{~и~}  n  =  2^k + 3m, m\in\mathbb{N}.
		\end{cases}
	\end{equation*}
	Непосредственным вычислением можно показать, что $x^3 \notin ac$.
%\end{eexample}





{\small Исследование выполнено при финансовой поддержке РНФ в рамках научного проекта No XXX.} %если исследование поддержано грантом, то здесь указывается ссылка

%Для оформления списка литературы, пожалуйста, используйте следующий образец.

\begin{center}
{\small \bf Литература}
\end{center}
\small
	{[1]}
	%\bibitem{lorentz1948contribution}
	{Lorentz~G.\,G.} A contribution to the theory of divergent sequences.
	// Acta Mathematica. 1948. т. 80, No 1. с. 167--190. ISSN 0001-5962.
	DOI: 10.1007/BF02393648
	\\
	{[2]}
	%\bibitem{Mazur}
	{Mazur~S.} O metodach sumowalności. // Ann. Soc. Polon. Math.
	(Supplement). 1929. с. 102--107.
	\\
	{[3]}
	%{}
	%\bibitem{banach2001theory_rus}
	{Банах~С.} Теория линейных операций. Регуляр. и хаот.
	динамика. М., 2001.
	\\
	{[4]}
	%\bibitem{zvol2022ac}
	Зволинский~Р.\,Е. Почти сходящиеся последовательности и операция
	возведения в положительную степень. // Вестник ВГУ. Серия: Физика.
	Математика. 2022. т. 3. с. 76--81.
	%\bibitem{sucheston1967banach}
	%L. Sucheston, Banach limits, {Amer. Math. Monthly} 74 (1967), 308--311, \textsc{issn}: 0002-9890,
	%\textsc{doi}: {10.2307/2316038}.
	%{}
\end{document}


{[1]} Ладыженская~О.\,А., Уральцева~Н.\,Н. Линейные и квазилинейные уравнения эллиптического типа. М.: Наука, 1973.\\ %пример оформления ссылки на книгу
{[2]} Данченко~В.\,И., Данченко~Д.\,Я. О приближении
наипростейшими дробями // Матем. заметки. 2001. Т. 70. №4. С. 553--559.\\ %пример оформления ссылки на статью в журнале
{[3]} Данченко~В.\,И., Данченко~Д.\,Я. О единственности
наипростейшей дроби наилучшего приближения // Тезисы докладов
Международной конференции по дифференциальным уравнениям и
динамическим системам (Суздаль, 2010). М.: МИАН, 2010. С. 71--72.\\ %пример оформления ссылки на тезисы доклада на конференции
{[4]} Название интернет-ресурса: http://полный, но максимально короткий адрес




