\documentclass[a4paper,openbib]{report}
\usepackage{amsmath}
\usepackage[utf8]{inputenc}
\usepackage[english,russian]{babel}
\usepackage{amsfonts,amssymb}
\usepackage{latexsym}
\usepackage{euscript}
\usepackage{enumerate}
\usepackage{graphics}
\usepackage[dvips]{graphicx}
\usepackage{geometry}
\usepackage{wrapfig}
\usepackage[colorlinks=true,allcolors=black]{hyperref}

\usepackage{bbm}


\righthyphenmin=2

\usepackage[14pt]{extsizes}

\geometry{left=3cm}% левое поле
\geometry{right=1cm}% правое поле
\geometry{top=2cm}% верхнее поле
\geometry{bottom=2cm}% нижнее поле

\renewcommand{\baselinestretch}{1.3}

\renewcommand{\leq}{\leqslant}
\renewcommand{\geq}{\geqslant} % И делись оно всё нулём!

\newcommand{\longcomment}[1]{}

\usepackage[backend=biber,style=gost-numeric,sorting=none]{biblatex}
\addbibresource{../bib/general_monographies.bib}
\addbibresource{../bib/ext.bib}
\addbibresource{../bib/my.bib}
\addbibresource{../bib/Semenov.bib}
\addbibresource{../bib/Bibliography_from_Usachev.bib}
\addbibresource{../bib/classic.bib}

\addbibresource{../../sct-statya/common/my.bib}


\input{../bib/ext.hyphens.bib}

\usepackage{amsthm}
\theoremstyle{definition}
\newtheorem{lemma}{Лемма}%[section]
\newtheorem{theorem}[lemma]{Теорема}
\newtheorem{example}[lemma]{Пример}
\newtheorem{property}[lemma]{Свойство}
\newtheorem{corollary}{Следствие}[lemma]
\newtheorem{definition}[lemma]{Определение}
\newtheorem{remark}[lemma]{Замечание}

%Only referenced equations are numbered
\usepackage{mathtools}
\mathtoolsset{showonlyrefs}

%\mathtoolsset{showonlyrefs=false}
% (an equation/multline to be force-numbered)
%\mathtoolsset{showonlyrefs=true}


\begin{document}
\clubpenalty=10000
\widowpenalty=10000

\setcounter{page}{2}


За прошедший период в соответствии с планом исследованы подмножества пространства ограниченных последовательностей,
определяемые функционалам Сачестона,
и линейные оболочки таких последовательностей.
Доказано, что пространство ограниченных последовательностей и некоторые его специальные подмножества
содержатся в линейных оболочках своих срезов по функционалам Сачестона.

Приведём краткий обзор полученных результатов.

Одним из основных объектов научно-исследовательской работы, выполняемой при поддержке гранта,
является пространство ограниченных последовательностей $\ell_\infty$ с обычной нормой
\begin{equation*}
	\|x\| = \sup_{k\in\mathbb{N}} |x_k|
	.
\end{equation*}
и обычной полуупорядоченностью, где $\mathbb{N}$ "--- множество натуральных чисел.
Естественным обобщением предела с пространства сходящихся последовательностей $c$ на пространство $\ell_\infty$
является понятие банахова предела.


\begin{definition}
	Линейный функционал $B\in l_\infty^*$ называется банаховым пределом,
	если
	\begin{enumerate}
		\item
			$B\geq0$, т.~е. $Bx \geq 0$ для $x \geq 0$,
		\item
			$B\mathbbm{1}=1$, где $\mathbbm{1} =(1,1,\ldots)$,
		\item
			$B(Tx)=B(x)$ для всех $x\in l_\infty$, где $T$~---
		оператор сдвига, т.~е. $T(x_1,x_2,\ldots)=(x_2,x_3,\ldots)$.
	\end{enumerate}
\end{definition}
Множество всех банаховых пределов обозначим через $\mathfrak{B}$.
Существование банаховых пределов было анонсировано С. Мазуром \cite{Mazur} и позднее доказано в книге С.~Банаха~\cite{B}.


Сачестон~\cite{sucheston1967banach} установил, что
для любых $x\in l_\infty$ и $B\in\mathfrak{B}$
\begin{equation}\label{Sucheston}
	q(x) \leqslant Bx \leqslant p(x)
	,
\end{equation}
где
\begin{equation*}
	q(x) = \lim_{n\to\infty} \inf_{m\in\mathbb{N}}  \frac{1}{n} \sum_{k=m+1}^{m+n} x_k
	~~~~\mbox{и}~~~~
	p(x) = \lim_{n\to\infty} \sup_{m\in\mathbb{N}}  \frac{1}{n} \sum_{k=m+1}^{m+n} x_k
	.
\end{equation*}
называют нижним и верхним функционалом Сачестона соотвественно.
Заметим, что $p(x) = -q(-x)$.
Неравенства \eqref{Sucheston} точны:
для данного $x$ для любого $r\in[q(x); p(x)]$ найдётся банахов предел
$B\in\mathfrak{B}$ такой, что $Bx = r$.


При исследовании банаховых пределов особый интерес представляют разделяющие множества~\cite[\S 3]{Semenov2014geomprops}.
Множество $Q\in\ell_\infty$ называют разделяющим, если
для любых неравных $B_1, B_2\in\mathfrak{B}$ существует такая последовательность $x\in Q$,
что $B_1 x \neq B_2 x$.
В частности, разделяющим является~\cite{semenov2010characteristic} множество всех последовательностей из 0 и 1,
которое в дальнейшем мы будем обозначать через $\Omega$
(иногда в литературе встречается также обозначение $\{0;1\}^\mathbb{N}$).

Каждой последовательности $(x_1, x_2, \dots)\in \Omega$ можно поставить в соответствие число
\begin{equation}\label{eq:bijection_omega_0_1}
	\sum_{k=1}^\infty 2^{-k} x_k \in [0,1]
	.
\end{equation}
С точностью до счётного множества это соответствие взаимно однозначно и определяет на множестве $\Omega$ меру,
которую мы будем отождествлять с мерой Лебега на $[0,1]$.

Оказывается, что из $\Omega$ можно выделить некоторые подмножества, которые также будут разделяющими,
например \cite[\S 3, Теорема 11]{Semenov2014geomprops},
\begin{equation}
	U = \{ x\in\Omega: q(x) = 0, p(x) = 1 \}
	.
\end{equation}

Однако множество $U$ имеет меру 1~\cite{semenov2010characteristic}.

В ходе выполнения работ по гранту построен пример разделяющего множества,
являющегося подмножеством $\Omega$ и имеющего меру нуль.
Для построения такого множества используется следующий факт.

\begin{lemma}[{\cite[\S 3, замечание 6]{Semenov2014geomprops}}]
	Пусть $X$~--- разделяющее множество и $X \subset \operatorname{Lin} Y$,
	где $\operatorname{Lin} Y$ обозначает линейную оболочку $Y$.
	Тогда $Y$ также является разделяющим множеством.
\end{lemma}

Следующим важным результатом работ является исследование свойств линейных оболочек множеств,
определённых с помощью функционалов Сачестона.
В частности, доказано
что наряду с использованным при построении разделяющего множества меры нуль включением
\begin{equation}
	\Omega \subset \operatorname{Lin}\{x\in\Omega : p(x) = a,~ q(x) = b\}
\end{equation}
для любых $0\leq b < a \leq 1$,
имеет место равенство
\begin{equation}
	\ell_\infty = \operatorname{Lin}\{x\in\ell_\infty : p(x) = a,~ q(x) = b\}
\end{equation}
для любых $a>b$.

Возникает закономерный вопрос: для каких ещё подмножеств пространства $\ell_\infty$
верны аналогичные соотношения?

Установлено, что аналогичным свойством обладает и ещё одно подмножество пространства $\ell_\infty$: подпространство
$A_0 = \{ x \in \ell_\infty : \alpha(x) =0 \}$,
где~\cite{our-vzms-2018}
\begin{equation*}
	\alpha(x) = \varlimsup_{i\to\infty} \max_{i<j\leqslant 2i} |x_i - x_j|
	.
\end{equation*}

Кроме того, были исследованы свойства пространства $A_0$, в частности, инвариантность относительно класических операторов.

Статья с подробными результатами работ направлена в журнал <<Вестник ВГУ>> (входит в Перечень ВАК) и ожидает рецензирования.





Таким образом, запланированные работы по гранту выполнены в полном объёме.




\addcontentsline{toc}{chapter}{Список литературы}
\small
\printbibliography{}

\end{document}
