\documentclass[a4paper,openbib]{report}
\usepackage{amsmath}
\usepackage[utf8]{inputenc}
\usepackage[english,russian]{babel}
\usepackage{amsfonts,amssymb}
\usepackage{latexsym}
\usepackage{euscript}
\usepackage{enumerate}
\usepackage{graphics}
\usepackage[dvips]{graphicx}
\usepackage{geometry}
\usepackage{wrapfig}
\usepackage[colorlinks=true,allcolors=black]{hyperref}

\usepackage{bbm}
\usepackage{mathrsfs}


\righthyphenmin=2

\usepackage[14pt]{extsizes}

\geometry{left=3cm}% левое поле
\geometry{right=1cm}% правое поле
\geometry{top=2cm}% верхнее поле
\geometry{bottom=2cm}% нижнее поле

\renewcommand{\baselinestretch}{1.3}

\renewcommand{\leq}{\leqslant}
\renewcommand{\geq}{\geqslant} % И делись оно всё нулём!

\newcommand{\longcomment}[1]{}

\usepackage[backend=biber,style=gost-numeric,sorting=none]{biblatex}
\addbibresource{../bib/general_monographies.bib}
\addbibresource{../bib/ext.bib}
\addbibresource{../bib/my.bib}
\addbibresource{../bib/Semenov.bib}
\addbibresource{../bib/Bibliography_from_Usachev.bib}
\addbibresource{../bib/classic.bib}

\addbibresource{../../sct-statya/common/my.bib}


\input{../bib/ext.hyphens.bib}

\usepackage{amsthm}
\theoremstyle{definition}
\newtheorem{lemma}{Лемма}%[section]
\newtheorem{theorem}[lemma]{Теорема}
\newtheorem{example}[lemma]{Пример}
\newtheorem{property}[lemma]{Свойство}
\newtheorem{corollary}{Следствие}[lemma]
\newtheorem{definition}[lemma]{Определение}
\newtheorem{remark}[lemma]{Замечание}

%Only referenced equations are numbered
\usepackage{mathtools}
\mathtoolsset{showonlyrefs}

%\mathtoolsset{showonlyrefs=false}
% (an equation/multline to be force-numbered)
%\mathtoolsset{showonlyrefs=true}


\begin{document}
\clubpenalty=10000
\widowpenalty=10000

\setcounter{page}{2}


За прошедший период в соответствии с планом исследованы последовательности из нулей и единиц,
носитель которых определяется через набор допустимых множителей.
Доказано, что конечность набора допустимых множителей влечёт почти сходимость
порождённой последовательности.

Приведём краткий обзор полученных результатов.

Одним из основных объектов научно-исследовательской работы, выполняемой при поддержке гранта,
является пространство ограниченных последовательностей $\ell_\infty$ с обычной нормой
\begin{equation*}
	\|x\| = \sup_{k\in\mathbb{N}} |x_k|
	.
\end{equation*}
и обычной полуупорядоченностью, где $\mathbb{N}$ "--- множество натуральных чисел.
Естественным обобщением предела с пространства сходящихся последовательностей $c$ на пространство $\ell_\infty$
является понятие банахова предела.

\begin{definition}
	Линейный функционал $B\in \ell_\infty^*$ называется банаховым пределом,
	если
	\begin{enumerate}%[label=(\roman*)]
		\item
			$B\geq0$, т.~е. $Bx \geq 0$ для $x \geq 0$,
		\item
			$B\mathbb{I}=1$, где $\mathbb{I} =(1,1,\ldots)$,
		\item
			$B(Tx)=B(x)$ для всех $x\in \ell_\infty$, где $T$~---
		оператор сдвига, т.~е. $T(x_1,x_2,\ldots)=(x_2,x_3,\ldots)$.
	\end{enumerate}
\end{definition}
Множество всех банаховых пределов обозначим через $\mathfrak{B}$.
Существование банаховых пределов было анонсировано С. Мазуром \cite{Mazur} и позднее доказано в книге С.~Банаха~\cite{banach1993theorie}.
%
Cуществуют почти сходящиеся последовательности ---
такие последовательности из $\ell_\infty$,
на которых все банаховы пределы принимают одинаковые значения.
\begin{theorem}[критерий Лоренца~\cite{lorentz1948contribution}]
	Для заданного $t\in\mathbb{R}$ равенство $Bx=t$ выполнено для всех $B\in\mathfrak{B}$
	тогда и только тогда, когда
	\begin{equation}
		\label{eq:crit_Lorentz}
		\lim_{n\to\infty} \frac{1}{n} \sum_{k=m+1}^{m+n} x_k = t
	\end{equation}
	равномерно по $m\in\mathbb{N}$.
\end{theorem}


Пространство почти сходящихся последовательностей будем обозначать через $ac$.

Уточняя результат Лоренца, Сачестон установил~\cite{sucheston1967banach}, что
для любых $x\in \ell_\infty$ и $B\in\mathfrak{B}$
\begin{equation}\label{Sucheston}
	q(x) \leqslant Bx \leqslant p(x)
	,
\end{equation}
где
\begin{equation*}
	q(x) = \lim_{n\to\infty} \inf_{m\in\mathbb{N}}  \frac{1}{n} \sum_{k=m+1}^{m+n} x_k
	\mbox{~~~~и~~~~}
	p(x) = \lim_{n\to\infty} \sup_{m\in\mathbb{N}}  \frac{1}{n} \sum_{k=m+1}^{m+n} x_k
\end{equation*}
называют нижним и верхним функционалом Сачестона соответственно.
Заметим, что $p(x) = -q(-x)$.
Неравенства \eqref{Sucheston} точны:
для данного $x$ для любого $t\in[q(x); p(x)]$ найдётся банахов предел
$B\in\mathfrak{B}$ такой, что $Bx = t$.

Легко видеть, что множество таких $x\in\ell_\infty$, что $p(x)=q(x)$,
в точности является пространством почти сходящихся последовательностей $ac$.
Будем говорить, что $x\in ac_t$, если $x\in ac$ и $p(x)=q(x)=t$.
Таким образом, функционалы Сачестона для почти сходимости играют такую же роль,
как верхний и нижний пределы для <<обычной>> сходимости.
Более подробно о свойствах почти сходящихся последовательностей и банаховых пределах см.~\cite{semenov2020geometry,avdeev2019space}.

Дальнейшим ослаблением понятия сходимости является сходимость по Чезаро (сходимость в среднем).
Говорят, что последовательность $\{x_n\}\in\ell_\infty$ сходится по Чезаро к $t$, если
\begin{equation}
	\lim_{n\to\infty}\frac1{n}\sum_{i=1}^n x_i = t
	.
\end{equation}
Легко заметить, что обсуждаемые обобщения верхнего и нижнего пределов удовлетворяют соотношению
\begin{multline}
	\label{eq:generalization_of_limits}
	\liminf_{n\to\infty} x_n \leq q(x) \leq \liminf_{n\to\infty}\frac1{n}\sum_{i=1}^n x_i
	\leq
	\\ \leq
	\limsup_{n\to\infty}\frac1{n}\sum_{i=1}^n x_i
	\leq p(x)
	\leq \limsup_{n\to\infty} x_n
	.
\end{multline}

Существенно место в исследованиях, ыполняемых при поддержке гранта,
занимает множество всех последовательностей из 0 и 1,
которое в дальнейшем мы будем обозначать через $\Omega$.
Понятно, что каждый $x\in \Omega$ можно отождествить с подмножеством множества натуральных чисел
$\operatorname{supp} x \subset \mathbb{N}$.

Вслед за~\cite{hall1992behrend} будем обозначать через $\mathscr{M}A$ множество всех чисел,
кратных элементам множества $A\subset\mathbb{N}$, т.е.
\begin{equation}
	\mathscr{M}A = \{ka: k\in\mathbb{N}, a\in A\}
	,
\end{equation}
через $\chi A$ "--- характеристическую функцию множества $A$.

Так, например,
\begin{gather}
	\chi \mathscr{M}\!A(\{2\}) = \chi \mathscr{M}\!A(\{2, 4\}) = \chi \mathscr{M}\!A(\{2,4,8,16,...\})
	= (0,1,0,1,0,1,0,1,0,...),
\\
	\chi \mathscr{M}\!A(\{3\}) = \chi \mathscr{M}\!A(\{3,9,27,...\}) = (0,0,1,\;0,0,1,\;0,0,1,\;0,0,1,\;0,0,1,\;...),
\\
	\chi \mathscr{M}\!A(\{2,3\}) = \chi \mathscr{M}\!A(\{2,3,6\}) = (0,1,1,1,0,1,\;0,1,1,1,0,1,\;0,1,1,1,0,1,...).
\end{gather}

Возникает закономерный вопрос о взаимосвязи структуры множества $A$
и значений, которые принимают обобщения верхнего и нижнего пределов~\eqref{eq:generalization_of_limits}
на последовательности $\chi \mathscr{M}\!A$.
Так, в работах~\cite{davenport1936sequences,davenport1951sequences} доказано, что для любого
$A=\{a_1,a_2,...\}\subset\mathbb{N}$
выполнено
\begin{equation}
	\liminf_{n\to\infty}\frac1{n}\sum_{i=1}^n (\chi\mathscr{M}A)_i =
	\lim_{j\to\infty}\lim_{n\to\infty}\frac1{n}\sum_{i=1}^n (\chi\mathscr{M}\{a_1,a_2,...,a_j\})_i
	.
\end{equation}
В работе~\cite[\S 7]{besicovitch1935density} построено такое множество $A\subset\mathbb{N}$, что
\begin{equation}
	\liminf_{n\to\infty}\frac1{n}\sum_{i=1}^n (\chi\mathscr{M}A)_i \neq
	\limsup_{n\to\infty}\frac1{n}\sum_{i=1}^n (\chi\mathscr{M}A)_i
	.
\end{equation}

В рамках работ по гранту проводится изучение аналогов классических результатов,
известных для верхнего и нижнего пределов,
на функционалы Сачестона
изучается зависимость значений, которые могут принимать функционалы Сачестона
на последовательностях $\chi\mathscr{M}A$, от свойств множества $A$.

Доказано, что для любого конечного множества $A=\{a_1,...,a_k\}$
функционалы Сачестона для последовательности $\chi\mathscr{M}A$ совпадают и обращаются в нуль,
и, как следствие, $\chi\mathscr{M}A\in ac_0$.


Статья с подробными результатами работ направлена во
<<Владикавказский математический журнал>> (входит в Перечень ВАК) и ожидает рецензирования.





Таким образом, запланированные работы по гранту выполнены в полном объёме.




\addcontentsline{toc}{chapter}{Список литературы}
\small
\printbibliography{}

\end{document}
