\documentclass[a4paper,openbib]{report}
\usepackage{amsmath}
\usepackage[utf8]{inputenc}
\usepackage[english,russian]{babel}
\usepackage{amsfonts,amssymb}
\usepackage{latexsym}
\usepackage{euscript}
\usepackage{enumerate}
\usepackage{graphics}
\usepackage[dvips]{graphicx}
\usepackage{geometry}
\usepackage{wrapfig}
\usepackage[colorlinks=true,allcolors=black]{hyperref}

\usepackage{bbm}


\righthyphenmin=2

\usepackage[14pt]{extsizes}

\geometry{left=3cm}% левое поле
\geometry{right=1cm}% правое поле
\geometry{top=2cm}% верхнее поле
\geometry{bottom=2cm}% нижнее поле

\renewcommand{\baselinestretch}{1.3}

\renewcommand{\leq}{\leqslant}
\renewcommand{\geq}{\geqslant} % И делись оно всё нулём!

\newcommand{\longcomment}[1]{}

\usepackage[backend=biber,style=gost-numeric,sorting=none]{biblatex}
\addbibresource{../bib/general_monographies.bib}
\addbibresource{../bib/ext.bib}
\addbibresource{../bib/my.bib}
\addbibresource{../bib/Semenov.bib}
\addbibresource{../bib/Bibliography_from_Usachev.bib}
\addbibresource{../bib/classic.bib}

\addbibresource{../../sct-statya/common/my.bib}


\input{../bib/ext.hyphens.bib}

\usepackage{amsthm}
\theoremstyle{definition}
\newtheorem{lemma}{Лемма}%[section]
\newtheorem{theorem}[lemma]{Теорема}
\newtheorem{example}[lemma]{Пример}
\newtheorem{property}[lemma]{Свойство}
\newtheorem{corollary}{Следствие}[lemma]
\newtheorem{definition}[lemma]{Определение}
\newtheorem{remark}[lemma]{Замечание}

%Only referenced equations are numbered
\usepackage{mathtools}
\mathtoolsset{showonlyrefs}

%\mathtoolsset{showonlyrefs=false}
% (an equation/multline to be force-numbered)
%\mathtoolsset{showonlyrefs=true}


\begin{document}
\clubpenalty=10000
\widowpenalty=10000

\setcounter{page}{2}

Одним из основных объектов научно-исследовательской работы, выполняемой при поддержке гранта,
является пространство ограниченных последовательностей $\ell_\infty$ с обычной нормой
\begin{equation*}
	\|x\| = \sup_{k\in\mathbb{N}} |x_k|
	.
\end{equation*}
и обычной полуупорядоченностью, где $\mathbb{N}$ "--- множество натуральных чисел.
Естественным обобщением предела с пространства сходящихся последовательностей $c$ на пространство $\ell_\infty$
является понятие банахова предела.


\begin{definition}
	Линейный функционал $B\in l_\infty^*$ называется банаховым пределом,
	если
	\begin{enumerate}
		\item
			$B\geq0$, т.~е. $Bx \geq 0$ для $x \geq 0$,
		\item
			$B\mathbbm{1}=1$, где $\mathbbm{1} =(1,1,\ldots)$,
		\item
			$B(Tx)=B(x)$ для всех $x\in l_\infty$, где $T$~---
		оператор сдвига, т.~е. $T(x_1,x_2,\ldots)=(x_2,x_3,\ldots)$.
	\end{enumerate}
\end{definition}
Множество всех банаховых пределов обозначим через $\mathfrak{B}$.
Существование банаховых пределов было анонсировано С. Мазуром \cite{Mazur} и позднее доказано в книге С.~Банаха~\cite{B}. 


Сачестон~\cite{sucheston1967banach} установил, что
для любых $x\in l_\infty$ и $B\in\mathfrak{B}$
\begin{equation}\label{Sucheston}
	q(x) \leqslant Bx \leqslant p(x)
	,
\end{equation}
где
\begin{equation*}
	q(x) = \lim_{n\to\infty} \inf_{m\in\mathbb{N}}  \frac{1}{n} \sum_{k=m+1}^{m+n} x_k
	~~~~\mbox{и}~~~~
	p(x) = \lim_{n\to\infty} \sup_{m\in\mathbb{N}}  \frac{1}{n} \sum_{k=m+1}^{m+n} x_k
	.
\end{equation*}
называют нижним и верхним функционалом Сачестона соотвественно.
Заметим, что $p(x) = -q(-x)$.
Неравенства \eqref{Sucheston} точны:
для данного $x$ для любого $r\in[q(x); p(x)]$ найдётся банахов предел
$B\in\mathfrak{B}$ такой, что $Bx = r$.

Множество таких $x\in\ell_\infty$, что $p(x)=q(x)$,
образует подпространство почти сходящихся последовательностей $ac$~~\cite{lorentz1948contribution}.
На почти сходящейся последовательности все банаховы пределы принимают одинаковые значения.
Основным результатом, полученным в ходе исследования коммутации банаховых пределов и интеграла Лебега,
стала невозможность аппроксимации произвольного банахова предела частичными функционалами Сачестона.

Определим на пространстве $\ell_\infty$ функционал
\begin{equation}
	F_{i,j} (x) =\frac{1}{i} \sum_{k=j+1}^{j+i} x_k
	.
\end{equation}

Можно легко доказать следующее утверждение:

\begin{lemma}
	Для любых $x\in\ell_\infty$, $B\in\mathfrak{B}$ и $\varepsilon>0$ существуют такие $i$ и $j$,
	что
	\begin{equation}
		\label{eq:sucheston_approx_epsilon}
		|Bx - F_{i,j}(x)| < \varepsilon
		.
	\end{equation}
\end{lemma}

Однако, как выясняется, можно найти такие две последовательности $x$ и $y$,
банахов предел $B$ и число $\varepsilon>0$, что <<общих>> индексов в предыдущей лемме найти не удастся.

\begin{example}
	Пусть
	\begin{equation}
		x_k = \begin{cases}
			1, \mbox{~если~} 2^{2m} \leq k < 2^{2m} + m, m > 6
			\\
			0 \mbox{~иначе,~}
		\end{cases}
	\end{equation}
	\begin{equation}
		y_k = \begin{cases}
			1, \mbox{~если~} 2^{2m+1} \leq k < 2^{2m+1} + m, m > 6
			\\
			0 \mbox{~иначе.}
		\end{cases}
	\end{equation}
	Пусть $L = B|_{ac}$, где $B\in\mathfrak{B}$ ($L$, очевидно, не зависит от выбора $B$).
	Непосредственно используя конструкции из доказательства теоремы Сачестона~\cite{sucheston1967banach},
	можно последовательно продлить $L$ сначала на $A_1 = \operatorname{Lin}(ac, x)$
	так, что $L(x) = 1 = p(x)$,
	а затем и на $A_2 = \operatorname{Lin}(ac, x,y)$
	так, что $L(y) = 1 = p(y)$
	(это возможно, так как $y\notin A_1$).
	Продлив $L$ на всё $\ell_\infty$ с сохранением неравенства $Lx\leq p(x)$,
	мы получим банахов предел.
	(Здесь мы обозначаем все продолжения функционала и полученный в результате банахов предел одной и той же буквой
	для краткости.)

	Пусть теперь $\varepsilon = 1/2$.
	Попробуем найти общие $i$ и $j$.
	Для $x$ условие~\eqref{eq:sucheston_approx_epsilon} принимает вид
	\begin{equation}
		|1 - F_{i,j}(x)| < 1/2
		,
	\end{equation}
	или, с учётом неравенства $0 \leq F_{i,j}(x) \leq 1$,
	\begin{equation}
		\label{eq:suchecton_approx_x_1_2}
		F_{i,j}(x) > 1/2
		.
	\end{equation}
	Для выполнения условия~\eqref{eq:suchecton_approx_x_1_2} необходимо (хотя и не достаточно),
	чтобы выполнялись неравенства
	\begin{equation}
		\label{eq:sucheston_approx_cases_j_x}
		\begin{cases}
			 2^{2m} - m \leq j < 2^{2m} + m,
			 \\
			 i \leq 2m
			 .
		\end{cases}
	\end{equation}
	Аналогично для $y$ получаем условия
	\begin{equation}
		\label{eq:sucheston_approx_cases_j_y}
		\begin{cases}
			 2^{2n+1} - n \leq j < 2^{2n+1} + n,
			 \\
			 i \leq 2n
			 .
		\end{cases}
	\end{equation}
	Легко видеть, что требования, накладываемые на $j$ в~\eqref{eq:sucheston_approx_cases_j_x} и~\eqref{eq:sucheston_approx_cases_j_y},
	несовместимы.
	Следовательно, <<общих>> индексов действительно не существует.
\end{example}


Для множеств точек с целочисленными расстояниями в евклидовом пространстве
(англ. \textit{integral point sets}, общепринятый лаконичный перевод на русский язык отсутствует ввиду того,
что подавляющая часть исследований по данному вопросу ведётся за границей)
исследованы свойства, присущие множествам полуобщего положения
(т.е. таким множествам, в которых никакие три точки не лежат на прямой).
Установлено, что для таких множеств можно дать нижнюю оценку на диаметр в зависимости от мощности,
более точную чем линейную, а именно полиномиальную с показателем $5/4$.
Этот результат опубликован на arXiv~\cite{my-semi-general-5-4-bound-2019}.
Ценность этого результата в том, что все полученные ранее другими учёными
были линейными, а также в том, что полученная оценка является первой оценкой,
специфичной для множест полуобщего положения.



Таким образом, запланированные работы по гранту выполнены в полном объёме.




\addcontentsline{toc}{chapter}{Список литературы}
\small
\printbibliography{}

\end{document}
