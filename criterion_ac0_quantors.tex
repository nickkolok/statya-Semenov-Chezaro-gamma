Дадим переформулировку критерия Лоренца почти сходимости последовательности,
которая иногда позволяет упростить доказательство.


\paragraph{Теорема.}
(Критерий почти сходимости к нулю для неотрицательной последовательности).
Пусть $x\in\ell_{\infty}$,
$\forall(i\in\mathbb{N})[x_i\geq 0]$.

$x\in ac_0$ тогда и только тогда, когда
\begin{equation}\label{crit_pos_ac0}
	\forall(A_0\in\mathbb{N})
	\exists(n_0\in\mathbb{N})
	\exists(m_0\in\mathbb{N})
	\forall(n\geq n_0)
	\forall(m\geq m_0)
	\\
	\left[
		\frac{1}{n}
		\sum_{k=m+1}^{m+n} x_k
		<
		\frac{1}{A_0}
	\right]
	.
\end{equation}

\paragraph{Доказательство.}
По теореме Лоренца $x\in ac_0$ тогда и только тогда, когда
\begin{equation}\label{Lorencz_ac0}
	\lim_{n\to\infty} \frac{1}{n} \sum_{k=m+1}^{m+n} x_k = 0
\end{equation}
равномерно по $m$.

Или, переводя на язык кванторов,
\begin{equation}\label{crit_ac0}
	\forall(A_1\in\mathbb{N})
	\exists(n_1\in\mathbb{N})
	\forall(n\geq n_0)
	\forall(m \in \mathbb{N})
	\\
	\left[
		\frac{1}{n}
		\sum_{k=m+1}^{m+n} x_k
		<
		\frac{1}{A_1}
	\right]
	.
\end{equation}
Очевидно, что из \eqref{crit_ac0} следует \eqref{crit_pos_ac0} (например, положив $m_0 = 1$),
тем самым необходимость \eqref{crit_pos_ac0} доказана.

\paragraph{Достаточность.}
Пусть выполнено \eqref{crit_pos_ac0}.
Покажем, что выполнено \eqref{crit_ac0}.
Зафиксируем $A_1$.
Положим $A_0 = 2A_1$ и отыщем $n_0$ и $m_0$ в соотвествии с \eqref{crit_pos_ac0}.
Положим $n_1 = 2A_0(n_0+m_0)\|x\|$.
Покажем, что \eqref{crit_ac0} верно для любых $n\geq n_1$, $m\in \mathbb{N}$.
Зафиксируем $n$ и рассмотрим $m$.

Пусть сначала $m\geq m_0$.
Тогда в силу того, что $n\geq n_1 = 2A_0(n_0+m_0)\|x\| > n_0$ имеем $n>n_0$.
Применим \eqref{crit_pos_ac0}:
\begin{equation}
	\frac{1}{n}
	\sum_{k=m+1}^{m+n} x_k
	<
	\frac{1}{A_0}
	=
	\frac{1}{2A_1}
	<
	\frac{1}{A_1}
	,
\end{equation}
т.е. \eqref{crit_ac0} выполнено.

Пусть теперь $m < m_0$.
Тогда
\begin{multline}
	\frac{1}{n} \sum_{k=m+1}^{m+n} x_k
	<
	\frac{1}{n} \sum_{k=1}^{m_0+n} x_k
	=
	\\=
	\frac{1}{n} \sum_{k=1}^{m_0} x_k + \frac{1}{n} \sum_{k=m_0+1}^{m_0+n} x_k
	%\leq
	\mathop{\leq}^{\mbox{~~в силу \eqref{crit_pos_ac0}~~}}
	\frac{1}{n} \sum_{k=1}^{m_0} x_k + \frac{1}{A_0}
	\leq
	\\ \leq
	\frac{1}{n} \sum_{k=1}^{m_0} \|x\| + \frac{1}{A_0}
	\leq
	\frac{1}{n_1} \sum_{k=1}^{m_0} \|x\| + \frac{1}{A_0}
	=
	\frac{1}{2A_0(n_0+m_0)\|x\|} \sum_{k=1}^{m_0} \|x\| + \frac{1}{A_0}
	=
	\\=
	\frac{m_0\|x\|}{2A_0(n_0+m_0)\|x\|} + \frac{1}{A_0}
	=
	\frac{m_0}{2A_0(n_0+m_0)} + \frac{1}{A_0}
	\leq
	\\ \leq
	\frac{m_0}{2A_0m_0} + \frac{1}{A_0}
	=
	\frac{1}{2A_0} + \frac{1}{A_0}
	=
	\frac{3}{2A_0}
	=
	\frac{3}{4A_1}
	<
	\frac{1}{A_1},
\end{multline}
т.е. \eqref{crit_ac0} тоже выполнено.

\paragraph{Примечание.}
Для знакопеременных последовательностей условие \eqref{crit_pos_ac0} является необходимым,
но неизвестно, является ли оно достаточным.

\paragraph{Примечание.}
Удобство критерия \eqref{crit_pos_ac0} в том,
что можно выбирать $m_0$ в зависимости от $A_0$.
