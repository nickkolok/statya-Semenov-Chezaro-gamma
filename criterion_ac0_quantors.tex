Результаты этого пункта использованы в~\cite{our-mz2019ac0}.

Дадим переформулировку критерия Лоренца
\cite{lorentz1948contribution,bennett1974consistency}
почти сходимости последовательности,
которая иногда позволяет упростить доказательство.


\begin{theorem}
	[Модифицированный критерий Лоренца]
	Пусть $x\in\ell_\infty$.

	$x\in ac_0$ тогда и только тогда, когда
	\begin{equation}\label{crit_pos_ac0}
		\forall(A_0\in\mathbb{N})
		\exists(n_0\in\mathbb{N})
		\exists(m_0\in\mathbb{N})
		\forall(n\geq n_0)
		\forall(m\geq m_0)
		\\
		\left[
			\left|
			\frac{1}{n}
			\sum_{k=m+1}^{m+n} x_k
			\right|
			<
			\frac{1}{A_0}
		\right]
		.
	\end{equation}

\end{theorem}

\paragraph{Доказательство.}
По теореме Лоренца $x\in ac_0$ тогда и только тогда, когда
\begin{equation}\label{Lorencz_ac0}
	\lim_{n\to\infty} \frac{1}{n} \sum_{k=m+1}^{m+n} x_k = 0
\end{equation}
равномерно по $m$.

Или, переводя на язык кванторов,
\begin{equation}\label{crit_ac0}
	\forall(A_1\in\mathbb{N})
	\exists(n_1\in\mathbb{N})
	\forall(n\geq n_0)
	\forall(m \in \mathbb{N})
	\\
	\left[
		\left|
		\frac{1}{n}
		\sum_{k=m+1}^{m+n} x_k
		\right|
		<
		\frac{1}{A_1}
	\right]
	.
\end{equation}
Очевидно, что из \eqref{crit_ac0} следует \eqref{crit_pos_ac0} (например, положив $m_0 = 1$),
тем самым необходимость \eqref{crit_pos_ac0} доказана.

\paragraph{Достаточность.}
Пусть выполнено \eqref{crit_pos_ac0}.
Покажем, что выполнено \eqref{crit_ac0}.
Зафиксируем $A_1$.
Положим $A_0 = 2A_1$ и отыщем $n_0$ и $m_0$ в соотвествии с \eqref{crit_pos_ac0}.
Положим $n_1 = 2A_0(n_0+m_0)\|x\|$.
Покажем, что \eqref{crit_ac0} верно для любых $n\geq n_1$, $m\in \mathbb{N}$.
Зафиксируем $n$ и рассмотрим $m$.

Пусть сначала $m\geq m_0$.
Тогда в силу того, что $n\geq n_1 = 2A_0(n_0+m_0)\|x\| > n_0$ имеем $n>n_0$.
Применим \eqref{crit_pos_ac0}:
\begin{equation}
	\left|
	\frac{1}{n}
	\sum_{k=m+1}^{m+n} x_k
	\right|
	<
	\frac{1}{A_0}
	=
	\frac{1}{2A_1}
	<
	\frac{1}{A_1}
	,
\end{equation}
т.е. \eqref{crit_ac0} выполнено.

Пусть теперь $m < m_0$.
Заметим, что
\begin{equation}
	\left|
		\sum_{k=m+1}^{m+n} x_k
		-
		\sum_{k=m_0+1}^{m_0+n} x_k
	\right|
	\leq 2(m_0 - m) \|x\|
	,
\end{equation}
откуда
\begin{equation}
	\left| \sum_{k=m+1}^{m+n} x_k \right|
	\leq
	2(m_0 - m) \|x\| + \left| \sum_{k=m_0+1}^{m_0+n} x_k \right|
	\leq
	2 m_0 \|x\| + \left| \sum_{k=m_0+1}^{m_0+n} x_k \right|
	.
\end{equation}


Тогда
\begin{multline}
	\left| \frac{1}{n} \sum_{k=m+1}^{m+n} x_k \right|
	\leq
	\frac{2 m_0 \|x\|}{n} + \left| \frac{1}{n} \sum_{k=m_0+1}^{m_0+n} x_k \right|
	\mathop{\leq}^{\mbox{~~в силу \eqref{crit_pos_ac0}~~}}
	\frac{2 m_0 \|x\|}{n} + \frac{1}{A_0}
	\leq
	\\ \leq
	\frac{2 m_0 \|x\|}{n_1} + \frac{1}{A_0}
	\leq
	\frac{2 m_0 \|x\|}{2A_0(n_0+m_0)\|x\|} + \frac{1}{A_0}
	=
	\\=
	\frac{m_0}{A_0(n_0+m_0)} + \frac{1}{A_0}
	<
	\frac{1}{A_0} + \frac{1}{A_0}
	=
	\frac{1}{A_1}
	,
\end{multline}
т.е. \eqref{crit_ac0} тоже выполнено.

\paragraph{Примечание.}
Удобство критерия \eqref{crit_pos_ac0} в том,
что можно выбирать $m_0$ в зависимости от $A_0$.
