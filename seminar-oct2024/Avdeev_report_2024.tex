\documentclass[10pt,pdf,hyperref={unicode},aspectratio=169,color={usenames, dvipsnames}]{beamer}
\usepackage{amsmath}
\usepackage[utf8]{inputenc}
\usepackage[english,russian]{babel}
\usepackage{amsfonts}
\usepackage{amsfonts,amssymb}
\usepackage{amssymb}
\usepackage{latexsym}
\usepackage{euscript}
\usepackage{enumerate}
\usepackage{graphics}
\usepackage{graphicx}
\usepackage{geometry}
\usepackage{wrapfig}

\usepackage{bbm}
\usepackage{mathrsfs}



\DeclareMathOperator{\ext}{ext}
\DeclareMathOperator{\mes}{mes}
\DeclareMathOperator{\supp}{supp}
\DeclareMathOperator{\conv}{conv}
\DeclareMathOperator{\diam}{diam}

\newcommand{\N}{\ensuremath{\mathbb{N}}}
\newcommand{\Q}{\ensuremath{\mathbb{Q}}}
\newcommand{\R}{\ensuremath{\mathbb{R}}}
\newcommand{\B}{\ensuremath{\mathfrak{B}}}
\newcommand{\Iac}{\mathcal{I}(ac_0)}
\newcommand{\Dac}{\mathcal{D}(ac_0)}
\newcommand{\one}{\ensuremath{\mathbbm 1}}


\newcommand{\longcomment}[1]{}


%Only referenced equations are numbered
\usepackage{mathtools}
\mathtoolsset{showonlyrefs}

%\mathtoolsset{showonlyrefs=false}
% (an equation/multline to be force-numbered)
%\mathtoolsset{showonlyrefs=true}


\usepackage{varwidth}
\usepackage{substr}

\usepackage{amsthm}
\theoremstyle{definition}
\newtheorem{llemma}{Лемма}
\newtheorem{ttheorem}[llemma]{Теорема}
\newtheorem{eexample}[llemma]{Пример}
\newtheorem{property}[llemma]{Свойство}
\newtheorem{remark}[llemma]{Замечание}
\newtheorem{ccorollary}{Следствие}[llemma]
\newtheorem{hhypothesis}{Гипотеза}[llemma]
\newtheorem{ddefinition}{Определение}[llemma]


\usepackage[backend=biber,style=gost-numeric,sorting=none]{biblatex}
\addbibresource{../bib/Semenov.bib}
\addbibresource{../bib/my.bib}
\addbibresource{../bib/ext.bib}
\addbibresource{../bib/classic.bib}
\addbibresource{../bib/Damerau-Levenstein.bib}
\addbibresource{../bib/general_monographies.bib}
\addbibresource{../bib/Bibliography_from_Usachev.bib}


%\usepackage{../../../biblatex2bibitem/biblatex2bibitem}


\righthyphenmin=2

%\usetheme{Antibes}
\usefonttheme{professionalfonts} % using non standard fonts for beamer
\usefonttheme[onlymath]{serif} % default family is serif
%\usepackage{fontspec}
%\setmainfont{Liberation Serif}
\begin{document}


\setbeamertemplate{navigation symbols}{\large{}}

\title{
	Банаховы пределы и некоторые классы линейных операторов
}
\author{докладчики: асп. Р.Е.~Зволинский, асп. Н.Н.~Авдеев\\науч. рук.: д.ф.-м.н., проф. Е.М. Семенов}
\institute{ Воронежский Государственный Университет \\
		{Работа выполнена при поддержке Фонда БАЗИС, проект No 22-7-2-27-3.
	}}
\date{2025}

\maketitle


\nocite{
%Солидные статьи
our-mz2019ac0,
our-mz2019measure,
our-mz2021linearhulls,
avdeed2021AandA,
avdeev2021vmzprimes,
avdeev2024decomposition,
avdeev2024set_DAN_eng,
avdeev2021vestnik,
%Тезисы
our-vvmsh-2018,
our-vzms-2018,
our-ped-2018-inf-dim-ker,
our-ped-2018-alpha-Tx,
avdeev2022measure,
%TODO: если будут ещё работы
}


\IfSubStringInString{\detokenize{_withmeta}}{\jobname}{
	\begin{frame}\frametitle{Метаданные: оппоненты и ведущая организация}
	\begin{itemize}
		\item
			Оппонент: Бородин Пётр Анатольевич, д.ф.-м.н., профессор РАН,
			\\
			профессор кафедры теории функций и функционального анализа Механико-математического факультета
			ФГБОУ ВО «Московский государственный университет им. М. В. Ломоносова»
			\\
			pborodin@inbox.ru , +7 (926) 217 1347
		\item
			Оппонент: Ушакова Елена Павловна, д.ф.-м.н., проф.,
			\\
			вед.н.с. лаборатории №45,
			ФГБУН «Институт проблем управления им.~В.А.~Трапезникова Российской Академии Наук»
			\\
			elenau@inbox.ru , +7 (962) 586 8768
		\item
			Ведущая организация:
			ФГБОУ ВО «Санкт-Петербургский государственный университет»
			\\
			Представитель:
			Скопина Мария Александровна, д.ф.-м.н., проф.,
			\\
			профессор факультета прикладной математики - процессов управления
			\\
			skopinama@gmail.com , skopina@MS1167.spb.edu
	\end{itemize}
\end{frame}

\begin{frame}\frametitle{Метаданные: публикации соискателя в Scopus по теме диссертации}
	\begin{enumerate}
		\small
		\item
		\emph{Авдеев} \emph{Н. Н.} — О пространстве почти сходящихся
		последовательностей //Математические заметки. — 2019. — т. 105, No 3. —
		с. 462—466. — DOI: \href {https://doi.org/10.4213/mzm12298} {\nolinkurl
		{10.4213/mzm12298}}.
		{}
		\item
		\emph{Авдеев} \emph{Н. Н.}, \emph{Cеменов} \emph{Е. М.},
		\emph{Усачев} \emph{А. С.} — Банаховы пределы и мера на множестве
		последовательностей из 0 и 1 //Математические заметки. — 2019. — т. 106,
		No 5. — с. 784—787.
		{}
		\item
		\emph{Авдеев} \emph{Н. Н.} — О подмножествах пространства ограниченных
		последовательностей //Математические заметки. — 2021. — т. 109, No 1. —
		с. 150—154.
		{}
		\item
		\emph{Авдеев} \emph{Н. Н.}, \emph{Семёнов} \emph{Е. М.},
		\emph{Усачев} \emph{А. С.} — Банаховы пределы: экстремальные свойства,
		инвариантность и теорема Фубини //Алгебра и анализ. — 2021. — т. 33, No 4. —
		с. 32—48.
		{}
		\item
		\emph{Авдеев} \emph{Н. Н.} — Почти сходящиеся последовательности из 0 и 1 и
		простые числа //Владикавказский математический журнал. — 2021. — т. 23,
		No 4. — с. 5—14. — DOI: \href {https://doi.org/10.46698/p9825-1385-3019-c}
		{\nolinkurl {10.46698/p9825-1385-3019-c}}.
		{}
		\item
		Decomposition of the set of Banach limits into discrete and continuous subsets. /. —
		N. Avdeev [et al.] //Annals of Functional Analysis. — 2024. — т. 15, No 81. — DOI: \href
		{https://doi.org/10.1007/s43034-024-00382-5} {\nolinkurl
		{10.1007/s43034-024-00382-5}}.
		{}
		\item
		The set of Banach limits and its discrete and continuous subsets. /. — N. Avdeev
		[et al.] //Doklady Mathematics. — 2024. — т. 110, No 1. — с. 346—348. — DOI: \href
		{https://doi.org/10.1134/S106456242470217X} {\nolinkurl
		{10.1134/S106456242470217X}}.
	\end{enumerate}

\end{frame}

\begin{frame}\frametitle{Метаданные: прочие публикации соискателя в Scopus}



	\begin{block}{Integral point sets (\textit{комбинаторная геометрия}):}
		\begin{enumerate}
			\item
				Avdeev N. N. On existence of integral point sets and their diameter bounds
				//Australasian Journal of Combinatorics. – 2020. – V. 77. – Pp. 100-116.
			\item
				Авдеев Н. Н. Об оценках на диаметр плоских множеств с целочисленными расстояниями полуобщего положения
				//Чебышевский сборник. – 2021. – Т. 22. – №.~4 (80). – С. 343-350.
			\item
				Авдеев Н. Н., Семёнов Е. М. О множествах точек на плоскости с целочисленными расстояниями
				//Математические заметки. – 2016. – Т. 100. – №.~5. – С. 757-761.
		\end{enumerate}
	\end{block}



	\begin{block}{Теория аттракторов:}
		\begin{enumerate}
			\setcounter{enumi}{3}
			\item
				Звягин В. Г., Авдеев Н. Н. Пример системы, минимальный траекторный аттрактор которой не содержит решений системы //Математические заметки. – 2018. – Т. 104. – №. 6. – С. 937-941.
		\end{enumerate}
	\end{block}

\end{frame}


}{
}


%\setbeamertemplate{footline}[frame number]
\addtobeamertemplate{navigation symbols}{}{%
    \usebeamerfont{footline}%
    \usebeamercolor[fg]{darkblue}%
    \hspace{1em}%
    \huge\bf
    \insertframenumber/\pageref{slide:final}\!\!
}

\begin{frame}\frametitle{Пространство $\ell_\infty$}
	$\ell_\infty$~--- пространство всех ограниченных последовательностей
	$x=(x_1, x_2, ..., x_n, ...)$
	с~нормой
	$$
		\|x\|_{\ell_\infty} = \sup_{k\in\mathbb{N}} |x_k|
	$$
	{Свойства:}

	\begin{itemize}
		\item
			$\ell_\infty$~--- линейное пространство над полем $\mathbb{R}$
		\item
			$\ell_\infty$  несепарабельно
		\item
			$\ell_1 \subset \ell_2 \subset \dots \subset \ell_\infty$
		\item
			$\ell_1^* = \ell_\infty$
		\item
			$\ell_\infty^* \neq \ell_1$
	\end{itemize}
\end{frame}


\begin{frame}\frametitle{Обобщения понятия сходимости в $\ell_\infty$}
	%\hspace{1.468em}
	\begin{varwidth}[t]{\linewidth}
		\centering
		Верхний и нижний
		\\
		пределы:
		\\
		$\displaystyle \limsup_{n\to\infty} x_n$
		,~
		$\displaystyle \liminf_{n\to\infty} x_n$
		\\~\\
		\emph{
			не характеризуют
			\\
			распределение элементов
		}
	\end{varwidth}
	\hfill
	\begin{varwidth}[t]{\linewidth}
		\centering
		Сходимость в среднем
		\\
		(по Чезаро):
		$\displaystyle \lim_{n\to\infty} (Cx)_n$
		\\
		$\displaystyle (Cx)_n = \frac1n\sum_{i=1}^n x_i$
		\\~\\
		\emph{слишком универсальна}
	\end{varwidth}
	\hfill
	\begin{varwidth}[t]{\linewidth}
		\centering
		Банаховы пределы $B\in \mathfrak{B}$:
		\\
		т. Хана--Банаха к $\lim: c\to\mathbb R$
		\\
			$B \geqslant 0$
		\\
			$B\mathbbm{1} = 1$
		\\
			$B=BT$
		\\
		\emph{почти сходимость}
	\end{varwidth}
	\\~\\~\\
	\begin{varwidth}[t]{\linewidth}
		\centering
		\emph{Критерий Лоренца:}
		$\displaystyle
			x\in ac_t
			\Leftrightarrow
			\forall(B\in\mathfrak{B})[Bx = t]
			\Leftrightarrow
			\lim_{n\to\infty}  \sum_{k=m+1}^{m+n} \frac{x_k}n = t
		$
		равном. по $m\in\mathbb{N}$.
	\end{varwidth}
	\\~\\
	\vspace{0.8em}
	\begin{varwidth}[t]{\linewidth}
		$\displaystyle ac = \mathop{\cup}\limits_{t\in\mathbb R} ac_t$ "--- пространство почти сходящихся последовательностей
	\end{varwidth}
\end{frame}


\begin{frame}\frametitle{Банаховы пределы}
	Банаховым пределом называется функционал $B\in \ell_\infty^*$ такой, что:
	\begin{enumerate}
		\item
			$B \geqslant 0$
		\item
			$B\mathbbm{1} = 1$
		\item
			$B=BT$
	\end{enumerate}
	Здесь $\mathbbm{1}=(1,1,1,1,1,...)$,
	$T$~--- оператор сдвига: $T(x_1, x_2, x_3, ...) = (x_2, x_3, ...)$.

	Простейшие свойства:
	\begin{itemize}
		\item
			$\|B\|_{\ell_\infty^*} = 1$
		\item
			$Bx = \lim_{n\to\infty} x_n$ для любого $x=(x_1, x_2, ...) \in c$,
			\\
			где $c$~--- множество сходящихся последовательностей.

			Таким образом,
			банахов предел~--- естественное обобщение понятия предела сходящейся последовательности
			на все ограниченные последовательности.
	\end{itemize}
\end{frame}

\longcomment{
\begin{frame}\frametitle{Свойства множества банаховых пределов $\mathfrak{B}$}
	\begin{enumerate}
		\item
			$\mathfrak{B}$~--- замкнутое выпуклое подмножество $ S_{\ell_\infty^*}$~---
			единичной сферы пространства $\ell_\infty^*$
		\item
			$d(\mathfrak{B}) = 2$
		\item
			$\mathfrak{B}$ слабо *-компактно
	\end{enumerate}
\end{frame}
}

\begin{frame}\frametitle{Теорема Лоренца~\cite{L}}
	Для заданного $r\in\mathbb{R}$ равенство $Bx=r$ выполнено для всех $B\in\mathfrak{B}$
	тогда и только тогда, когда
	\begin{equation*}
		\lim_{n\to\infty} \frac{1}{n} \sum_{k=m+1}^{m+n} x_k = r
	\end{equation*}
	сходится равномерно по всем $m\in\mathbb{N}$.

	Множество всех таких $x \in \ell_\infty$ обозначается $ac$.
\end{frame}

\begin{frame}\frametitle{Теорема Сачестона~\cite{S}}
	является уточнением теоремы Лоренца.
	Пусть
	\begin{equation*}
		q(x) = \lim_{n\to\infty} \inf_{m\in\mathbb{N}}  \frac{1}{n} \sum_{k=m+1}^{m+n} x_k,
		~~~~~~~~
		p(x) = \lim_{n\to\infty} \sup_{m\in\mathbb{N}}  \frac{1}{n} \sum_{k=m+1}^{m+n} x_k.
	\end{equation*}
	Тогда для любых $x\in \ell_\infty$ и $B\in\mathfrak{B}$
	\begin{equation}\label{Sucheston}
		q(x) \leqslant Bx \leqslant p(x)
	\end{equation}
	Неравенства (\ref{Sucheston}) точны:
	для данного $x$ для любого $r\in[q(x); p(x)]$ найдётся банахов предел
	$B\in\mathfrak{B}$ такой, что $Bx = r$.
\end{frame}

\begin{frame}\frametitle{Что такое почти сходимость?}
	\begin{varwidth}[t]{\linewidth}
		\hspace{3em}
		Пример $x\in ac_0$, но $\displaystyle0=\liminf_{n\to\infty}x_n < \limsup_{n\to\infty}x_n=1$:~~
		$$\displaystyle
			x_k=\begin{cases}
				1, & k=2^j,~j\in\mathbb N
				\\
				0 & \mbox{для прочих~} k
			\end{cases}
		$$
		$x=(0,1,0,1,\;0,0,0,1,\;\;0,0,0,0,\;0,0,0,1,\;\;0,0,0,0,\;0,0,0,0,\;\;0,0,0,0,\;0,0,0,1,\;\;0,0,0,0,\;...)$
	\end{varwidth}
	\\
	\vspace{2em}
	\begin{varwidth}[t]{\linewidth}
		Пример сходящейся по Чезаро к нулю последовательности $x\notin ac$:~~
		$$\displaystyle
			x_k=\begin{cases}
				1, & 2^j-j < k \leq 2^j,~j\in\mathbb N
				\\
				0 & \mbox{для прочих~} k
			\end{cases}
		$$
		$x=(0,1,1,1,\;0,1,1,1,\;\;0,0,0,0,\;1,1,1,1,$\\
		$\phantom{x=(}0,0,0,0,\;0,0,0,0,\;\;0,0,0,1,\;1,1,1,1,$\\
		$\phantom{x=(}0,0,0,0,\;0,0,0,0,\;\;0,0,0,0,\;0,0,0,0,$\\
		$\phantom{x=(}0,0,0,0,\;0,0,0,0,\;\;0,0,1,1,\;1,1,1,1,...)$
	\end{varwidth}
\end{frame}





\begin{frame}\frametitle{Структура работы}
	\begin{enumerate}
		\item
			Последовательности, почти сходящиеся к нулю
			(пространство $ac_0$)
		\item
			$\alpha$--функция как асимптотическая характеристика ограниченной последовательности
		\item
			Банаховы пределы, инвариантные относительно некоторых операторов
		\item
			Функционалы Сачестона и линейные оболочки
		\item
			Функционалы Сачестона и мультипликативные свойства носителя
	\end{enumerate}

	~\\~\\
	Заголовки слайдов с результатами, полученными лично докладчиком, \underline{подчёркнуты}.
\end{frame}


\begin{frame}\frametitle{\underline{Диадическая последовательность}}
	\begin{ddefinition}
		Последовательность $x\in\ell_\infty$
		называется \emph{диадической}, если существует такое $X=\{a;b\}$,
		что $x_k \in X$ для всех $k\in \mathbb N$.
	\end{ddefinition}

	Например, диадическими являются все последовательности из нулей и единиц.


	\begin{llemma}
		Пусть $x\in\ell_\infty$, $\|x\|\leq 1$.
		Тогда найдётся такая $h\in ac_0$, что $(x+h)_n = \pm 1$ для всех $n\in\mathbb N$.
	\end{llemma}

	\begin{ccorollary}
		Пусть $x\in\ell_\infty$.
		Тогда найдётся такая $h\in ac_0$, что для всех $n\in\mathbb N$
		\begin{equation*}
			(x+h)_n \in \{\inf_{n\in\mathbb N} x_n,\sup_{n\in\mathbb N} x_n\}
			.
		\end{equation*}
	\end{ccorollary}
	\vspace{-2em}
	\begin{ccorollary}
		Пусть $x\in\ell_\infty$.
		Тогда найдётся такая $h\in ac_0$, что $(x+h)_n =\pm \|x\|$ для всех $n\in\mathbb N$.
	\end{ccorollary}

	\vfill
\end{frame}




\begin{frame}\frametitle{Разделяющие множества}
	\label{page:p_q_linhulls}
	Множество $M\subset \ell_\infty$ называется разделяющим,
	если для любых двух различных банаховых пределов $B_1$ и $B_2$
	найдётся такой $x\in M$, что $B_1 x \ne B_2 x$.
	\\~\\
	Интересны <<маленькие>> разделяющие множества.
	\\~\\
	Например, $\Omega$ разделяющее~\cite{semenov2010characteristic}.
	\\~\\
	Этот результат можно получить непосредственно из теоремы о разложении.
\end{frame}





\begin{frame}\frametitle{\underline{Интервальный критерий}}
	%\begin{theorem}
		\label{thm:M_j_ac0_inf_lim}
		Пусть $n_i$~--- строго возрастающая последовательность натуральных чисел,
		\begin{equation}
			\label{eq:definition_M_j}
			M(j) = \liminf_{i\to\infty} \left(n_{i+j} - n_i\right),
		\end{equation}
		\begin{equation}
			x_k = \left\{\begin{array}{ll}
				1, & \mbox{~если~} k = n_i
				\\
				0  & \mbox{~иначе~}
			\end{array}\right.
		\end{equation}
		Тогда следующие условия эквивалентны:
		\\~\\
		(i)   $x \in ac_0$;
		\\~\\
		(ii)  $\lim\limits_{j \to \infty} \dfrac{M(j)}{j} = +\infty$;
		\\~\\
		(iii) $\inf\limits_{j \in \N}     \dfrac{M(j)}{j} = +\infty$.
	%\end{theorem}
\end{frame}



\begin{frame}\frametitle{\underline{Усиленная теорема Коннора}}
	Пусть на множестве $\Omega=\{0,1\}^\N$ задана вероятностная мера <<честной монетки>> $\mu$.
	Коннор доказал~\cite{connor1990almost}, что $\mu(\Omega\cap ac)=0$.

	Обобщим этот результат.

	\begin{ttheorem}
	%\label{thm:Connor_generalized}
	Мера множества $F=\{x\in\Omega : q(x) = 0 \wedge p(x)= 1\}$,
	где $p(x)$ и $q(x)$~--- верхний и нижний функционалы Сачестона соответственно,
	равна 1.
	\end{ttheorem}
\end{frame}



\begin{frame}\frametitle{$\alpha$--функция}
	\begin{equation}
		\alpha(x) = \limsup_{i\to\infty} \max_{i<j\leqslant 2i} |x_i - x_j|
		.
	\end{equation}
	\vfill
	\begin{equation}
		\alpha(c)=0,
		\qquad
		\alpha(x+y) \leq \alpha(x) + \alpha(y),
		\qquad
		|\alpha(x) - \alpha(y)| \leq 2 \|x-y\|
		.
	\end{equation}
	\vfill
	$\alpha$--функция --- <<мера несходимости>> последовательности;
	\\
	равенство $\alpha(x) = 0$, однако, не гарантирует сходимость.

\end{frame}

\begin{frame}\frametitle{\underline{$\alpha$--функция и оператор сдвига}}
	\vfill
	$\alpha$--функция не инвариантна относительно оператора сдвига $T$
	\\
	(в отличие от банаховых пределов).

	\begin{ttheorem}
		Для любых $x\in\ell_\infty$ и $n \in \N$
		выполнено
		$\displaystyle\qquad
			\frac{1}{2}\alpha(x) \leq \lim_{m\to\infty} \alpha(T^m x)\leq\alpha(T^n x) \leq \alpha(x)
			.
		$
	\end{ttheorem}

	\vfill

	\begin{ttheorem}
		Пусть $\beta_k$~--- монотонная невозрастающая последовательность,
		$\beta_k \to \beta \in\left[\frac{1}{2}; 1\right]$, $\beta_1 \leq 1$.
		Тогда существует такой $x\in\ell_\infty$, что для любого натурального $n$
		\begin{equation}
			\frac{\alpha(T^n x)}{\alpha(x)} = \beta_n.
		\end{equation}
	\end{ttheorem}
	%\vfill
	%
	\vspace{-1.46em}
	\begin{ttheorem}
		\label{thm:rho_x_c_leq_alpha_t_s_x}
		Для любого $x\in ac$ и $n \in \N$
		выполнено
		$\displaystyle
			\frac{1}{2}\alpha(x) \leq \rho(x,c) \leq
			\lim_{m\to\infty} \alpha(T^m x)\leq \alpha(T^n x)
			\leq
			\alpha(x)
			.
		$
	\end{ttheorem}
	Аналогично для $ac_0$ и $c_0$. Условие почти сходимости существенно.

	\vfill
\end{frame}


\begin{frame}\frametitle{\underline{$\alpha$--функция и другие классические операторы}}
	\begin{ttheorem}
		Для любого $n\in\N$ и любого $x\in\ell_\infty$ выполнено
		\begin{equation}
			\alpha(\sigma_n x) = \alpha(x)
			\qquad\mbox{и}\qquad
			\alpha(\sigma_{1/n} x) \leq \left( 2- \frac{1}{n} \right) \alpha(x)
			.
		\end{equation}
	\end{ttheorem}

	\begin{ttheorem}
		Имеет место равенство
		\begin{equation}
			\sup_{x\in\ell_\infty, \alpha(x)\neq 0} \frac{\alpha(Cx)}{\alpha(x)}=1
			.
		\end{equation}
	\end{ttheorem}

	\begin{ttheorem}
		Пусть $(x\cdot y)_k = x_k\cdot y_k$.
		Тогда
		$\alpha(x\cdot y)\leq \alpha(x)\cdot \|y\|_* + \alpha(y)\cdot \|x\|_*$,
		где
		\begin{equation}
			\|x\|_* = \limsup_{k\to\infty} |x_k|
		\end{equation}
		есть  фактор-норма по $c_0$ на пространстве $\ell_\infty$.
	\end{ttheorem}
\end{frame}


\begin{frame}\frametitle{\underline{Пространство $A_0$}}

	Пространство $A_0 = \{x: \alpha(x) = 0\}$
	несепарабельно, замкнуто относительно покоординатного умножения,
	операторов левого и правого сдвигов $T$ и $U$,
	оператора Чезаро $C$,
	операторов растяжения $\sigma_n$ и усредняющего сжатия $\sigma_{1/n}$.

	\begin{equation}
		A_0 \cap ac_0 = c_0,
		\qquad
		A_0 \cap ac\, = c,
	\end{equation}

	\vfill

	Оба вложения
	\qquad$
		c_0 \subset A_0 \subset \ell_\infty
	$\qquad
	недополняемы.

	\vfill

	Более того, недополняемо вложение $c_0 \subset ac_0$
	\\
	(недополняемость вложения $ac_0 \subset \ell_\infty$ доказана Е.А. Алехно~\cite[{Theorem 8}]{alekhno2006propertiesII})
\end{frame}


\begin{frame}\frametitle{Инвариантные банаховы пределы}
	\label{page:invariant_BL}

	\vfill
	Для любого банахова предела $B$ выполнено $B=BT$.
	\vfill
	Относительно каких ещё операторов банаховы пределы инвариантны?
	\vfill
	Даже для естественного оператора растяжения $\sigma_n$, $n\in\N_2$,
	инвариантны не все банаховы пределы.
	\vfill
	Банаховых пределов, инвариантных относительно оператора Чезаро $C$,
	ещё <<меньше>>:
	\begin{equation}
		B = BC \ \Rightarrow \ B = B\sigma_n \quad \mbox{для любого} \quad n \in \N
	\end{equation}
	\vfill


\end{frame}







\begin{frame}\frametitle{Инвариантные банаховы пределы }
	\begin{ddefinition}
		Банахов предел $B$ называется инвариантным относительно оператора $H:\ell_\infty \to \ell_\infty$,
		если $Bx = BHx$ для любого $x\in\ell_\infty$. Обозначение: $B\in\B(H)$.
	\end{ddefinition}
	\vfill
	\begin{ttheorem}[Признак Семенова--Сукочева,~\cite{Semenov2010invariant}]
		\label{thm:Semenov_Sukochev_conditions}
		Пусть линейный оператор $H:\ell_\infty\to\ell_\infty$ таков, что:
		\\(i)   $H \geqslant 0, H \one=\one$;
		\\(ii)  $H c_0 \subset c_0$;
		\\(iii) $\limsup _{j \rightarrow \infty}(A(I-T) x)_j \geqslant 0$ для всех $x \in \ell_{\infty}, A \in R$,
		\\где
		\begin{equation}
			R=R(H):=\operatorname{conv}\left\{H^n, n=1,2, \ldots\right\}
			.
		\end{equation}
		Тогда $\B(H) \ne\varnothing$.
	\end{ttheorem}
	\vfill
	Эти достаточные условия очень сильны и избыточны.
	\vfill
	Им удовлетворяют, например, операторы растяжения $\sigma_n$ и Чезаро $C$.
\end{frame}



\begin{frame}\frametitle{B-регулярные операторы}
	\begin{ddefinition}
		Оператор $H : \ell_\infty \to \ell_\infty$ называется \emph{эберлейновым},
		если $\B(H) \ne\varnothing$.
	\end{ddefinition}
	\begin{ddefinition}[\cite{alekhno2018invariant}]
		\label{def:B-regular_operator}
		Оператор $H : \ell_\infty \to \ell_\infty$ называется \emph{B-регулярным},
		если $H^*\B \subseteq \B$.
		\\
		(Или, что то же самое, $BH\in\B$ для любого $B\in\B$.)
	\end{ddefinition}
	\begin{ttheorem}[\cite{alekhno2018invariant}]
		\label{thm:B-regular_is_Eberlein}
		Любой B-регулярный оператор "--- эберлейнов.
	\end{ttheorem}
	\begin{ttheorem}[\cite{alekhno2018invariant}]
		Оператор $H:\ell_\infty \to \ell_\infty$ является B-регулярным тогда и только тогда, когда:
		\\(i) $H\one \in ac_1$;
		\\(ii) $q(Hx)\geq 0$ для любого $x\geq 0$;
		\\(iii) $H ac_0 \subseteq ac_0$.
	\end{ttheorem}
\end{frame}


\begin{frame}\frametitle{{Новые классы операторов}}
	\vfill
	\begin{ddefinition}
		Будем называть оператор $H:\ell_\infty \to \ell_\infty$ \emph{полуэберлейновым}, если $BH\in\mathfrak B$ для некоторого $B\in\mathfrak B$.
	\end{ddefinition}
	\vfill
	\begin{ddefinition}
		Будем называть оператор $H:\ell_\infty \to \ell_\infty$ \emph{Q-регулярным}, если
		для любого $x\in\ell_\infty$ выполнено равенство $q(Hx) = q(x)$.
	\end{ddefinition}
	\vfill
\end{frame}


\begin{frame}\frametitle{{Новые классы операторов}}
	\vfill
	\begin{ddefinition}
		Разреженным назовём оператор
		взятия подпоследовательности
		$H: \ell_\infty \to \ell_\infty$,
		определяемый следующим образом:
		$$
		H(x_1, x_2, x_3, \ldots) = (x_{m_1}, x_{m_1 + 1}, \ldots, x_{n_1 - 1}, x_{n_1};
		x_{m_2}, x_{m_2 + 1}, \ldots, x_{n_2 - 1}, x_{n_2}; \ldots),
		$$
		где для всех
		$k \geqslant k_0 \in \mathbb N$
		выполнено
		$
		m_k \leqslant n_k < m_{k+1}\leqslant n_{k+1}
		$,
		при этом
		$$
		\lim\limits_{k \to \infty} (n_k - m_k) = \infty
		.
		$$

		\begin{itemize}
			\item
				Всякий разреженный оператор является В-регулярным.
			\item
				Обратное, очевидно, неверно.
			\item
				Существует разреженный оператор, не являющийся Q-регулярным.
		\end{itemize}
	\end{ddefinition}
	\vfill
\end{frame}

\begin{frame}\frametitle{{Классы операторов}}
	$H:\ell_\infty \to \ell_\infty$
	\begin{itemize}
		\item
			полуэберлейновы: такие, что $B_1 H \in\B$ для некоторого $B_1\in \B$;
		\item
			эберлейновы: такие, что $B_1 H = B_1$ для некоторого $B_1\in \B$;
		\item
			В-регулярные: такие, что $B_1 H \in \B$ для любого   $B_1\in \B$;
		\item
			Q-регулярные: такие, что $q(Hx) = q(x)$ для любого   $x\in \ell_\infty$.
	\end{itemize}
	\begin{ttheorem}
		Каждый следующий из этих классов вложен в предыдущий и не совпадает с ним.
	\end{ttheorem}
\end{frame}


\begin{frame}\frametitle{{Полуэберлейнов оператор, не являющийся эберлейновым}}
	Пусть $B_1, B_2 \in \ext \B$.
	Положим
	\begin{equation}
		\label{eq:am_not_eber_def}
		B_3 = B_1 + 2(B_2-B_1) = 2B_2-B_1,
	\end{equation}
	тогда $B_3 \notin \B$.
	Положим $Hx = (B_3 x) \cdot\one$, тогда
	\begin{equation}
		2 B_1 H x = B_1 x + B_1 ((B_3 x) \cdot\one) = B_1 x + B_3 x =
		B_1 x + 2 B_2 x - B_1 x = 2B_2 x
		,
	\end{equation}
	т.е. $B_1 H = B_2 \in \B$ и оператор $H$ полуэберлейнов.
	Пусть $B = BH$ для некоторого $B\in\B$.
	Тогда для всех $x\in\ell_\infty$ имеем
	\begin{equation}
		2Bx = B (x + (B_3 x) \cdot \one)
		\Rightarrow
		Bx =  B((B_3 x) \cdot \one)
		\Rightarrow
		Bx = B_3x
		.
	\end{equation}
\end{frame}

\begin{frame}\frametitle{{Эберлейнов оператор, не являющийся В-регулярным}}
	\vspace{-3em}
	\begin{multline}
		\label{eq:oper_A_throws_out_2power_blocks}
		Ax = (x_1, x_2, \not x_3, \not x_4, x_5, x_6, x_7, x_8, \not x_9, ..., \not x_{16}, x_{17}, x_{18}, ..., x_{31}, x_{32}, \not x_{33}, \not x_{34}, ..., \not x_{64},
		\\
		x_{65}, x_{66}, ..., x_{128}, \not x_{129}, ...)=
		\\=
		(x_1, x_2, \ x_5, x_6, x_7, x_8, \ x_{17}, x_{18}, ..., x_{31}, x_{32}, \ x_{65}, x_{66}, ..., x_{128}, \ x_{257},
		\\
		..., \ x_{2^{2n} +1}, x_{2^{2n} +2},  x_{2^{2n+1}}, \ \ x_{2^{2(n+1)} +1},  x_{2^{2(n+1)} +2},  x_{2^{2(n+1)+1}}, ...)
	\end{multline}
	Оператор $A$ является В-регулярным, разреженным, но не Q-регулярным.
	\vfill

	\begin{equation}
		Ex = x \cdot \chi_{\cup_{m=0}^{\infty}\left[2^{2 m}+1, 2^{2 m+1}\right] \cup\{1\}}
		.
	\end{equation}
	\vfill
	Очевидно, $AE=A$.
	Поскольку $A$ есть В-регулярный оператор, то он эберлейнов,
	и $\B(A)\ne\varnothing$.
	Пусть $B\in\B(A)$. Тогда
	\begin{equation}
		B = BA = B(AE) = (BA)E = BE
		,
	\end{equation}
	то есть $B\in\B(E)$.
	Но оператор $E$ не является В-регулярным.
	Действительно, $q(E\one) =q(A\one) =0$, т.е. $ E\one \notin ac_1$,
	и не выполнено условие (i) критерия В-регулярности.
	\vfill
\end{frame}


\begin{frame}\frametitle{Поиск <<самого хорошего>> банахова предела}
	\begin{equation}
		\sigma_n (x_1, x_2, x_3, ...) = (
		\underbrace{x_1,...,x_1}_{n~\text{раз}},
		\underbrace{x_2,...,x_2}_{n~\text{раз}},
		\underbrace{x_3,...,x_3}_{n~\text{раз}},
		...)
	\end{equation}

	\begin{equation}
		C (x_1, x_2, x_3, ...) = \left(
		x_1,
		\dfrac{x_1+x_2}2,
		\dfrac{x_1+x_2 + x_3}3,
		\dfrac{x_1+x_2+x_3+x_4}4,
		...,
		\dfrac{x_1+...+x_n}n,
		...\right)
		.
	\end{equation}

	\begin{equation}
		\mathfrak{B}(C) \subsetneq \bigcap_{n=2}^\infty \mathfrak{B}(\sigma_n)
		\qquad
		\qquad
		\diam\mathfrak{B}(C) = \operatorname{rad} \mathfrak{B}(C) = 2
	\end{equation}
\end{frame}


\begin{frame}\frametitle{{Обратная задача об инвариантности}}
	Для всякого ли банахова предела существует нетривиальный оператор, относительно которого он инвариантен?
	\vfill
	\begin{ddefinition}
		Оператор $G_B x = (Bx, Bx, Bx, ...) = (Bx)\cdot\one$
		назовём порождённым банаховым пределом $B$.
	\end{ddefinition}
	\vfill
	\begin{ttheorem}
		$\B(G_B) = \{B\}$ и $G_B^* \B = \{B\}$.
	\end{ttheorem}
	\vfill
	\begin{llemma}
		Пусть $0 \leq \lambda \leq 1$, $B_1, B_2 \in \B$.
		Тогда
		\begin{equation}
			G_{\lambda B_1+(1-\lambda) B_2} =\lambda G_{B_1} + (1-\lambda)G_{B_2}
			,
		\end{equation}
		\begin{equation}
			G_{B_1} G_{B_2} = G_{B_2}
			.
		\end{equation}
	\end{llemma}
\end{frame}


\begin{frame}\frametitle{{Мультипорождённые операторы}}
	\begin{ddefinition}
		Линейный непрерывный оператор $G_{\{B_k\}}$, определённый равенством
		\begin{equation}
			(G_{\{B_k\}}x)_n = B_n x
			,
		\end{equation}
		будем называть мультипорождённым последовательностью банаховых пределов $\{B_k\}\subset \B$.
	\end{ddefinition}
	\vfill

	\begin{llemma}
		Всякий мультипорождённый оператор $G_{\{B_k\}}$ является В-регулярным оператором.
	\end{llemma}
	\vfill

	\begin{llemma}
		Пусть множество значений, которые принимают элементы последовательности ${\{B_k\}}$, конечно.
		Тогда оператор $G_{\{B_k\}}$ компактен.
	\end{llemma}
	\vfill

	\begin{ttheorem}
		Мультипорождённый оператор не может иметь матричного представления.
	\end{ttheorem}

\end{frame}


\begin{frame}\frametitle{{Мультипорождённые операторы и инвариантность}}
	\begin{multline}
		Ax = (x_1, x_2, \not x_3, \not x_4, x_5, x_6, x_7, x_8, \not x_9, ..., \not x_{16}, x_{17}, x_{18}, ..., x_{31}, x_{32}, \not x_{33}, \not x_{34}, ..., \not x_{64},
		\\
		x_{65}, x_{66}, ..., x_{128}, \not x_{129}, ...)=
		\\=
		(x_1, x_2, \ x_5, x_6, x_7, x_8, \ x_{17}, x_{18}, ..., x_{31}, x_{32}, \ x_{65}, x_{66}, ..., x_{128}, \ x_{257},
		\\
		..., \ x_{2^{2n} +1}, x_{2^{2n} +2},  x_{2^{2n+1}}, \ \ x_{2^{2(n+1)} +1},  x_{2^{2(n+1)} +2},  x_{2^{2(n+1)+1}}, ...)
	\end{multline}
	Операторы $A$ и $A\sigma_2$ В-регулярны; $\B(A) \ne\varnothing$, $\B(A\sigma_2) \ne\varnothing$,
	$\B(A) \cap \B(A\sigma_2) =\varnothing$.

	\vfill

	Выберем $B_1 \in \B (A)$ и $B_2 \in \B(A\sigma_2)$ и определим оператор $H$ соотношением
	\begin{equation}
		Hx = (B_1 x, B_1 x, \ B_2 x, B_2 x, \ B_1 x, B_1 x, B_1 x, B_1 x, \ \underbrace{B_2 x, ..., B_2 x}_{8~\mbox{раз}}, ...)
		.
	\end{equation}

	Тогда $[B_1; B_2]\subset \B(H)$.
	Оператор $(\sigma_2 H)^*$ переводит отрезок $[B_1; B_2]$ в $[B_2; B_1]$,
	т.е. <<меняет местами>> два банаховых предела,
	и $\dfrac{B_1 + B_2}2 \in \B(\sigma_2 H)$.
\end{frame}



\begin{frame}\frametitle{Разделяющие множества}
	Множество $M\subset \ell_\infty$ называется разделяющим,
	если для любых двух различных банаховых пределов $B_1$ и $B_2$
	найдётся такой $x\in M$, что $B_1 x \ne B_2 x$.
	\\~\\
	Интересны <<маленькие>> разделяющие множества.
	\\~\\
	Например, $\Omega$ разделяющее~\cite{semenov2010characteristic}.
	\\~\\
	А можно ли ещё меньше?
	\vfill
	Счётных разделяющих множеств не бывает~\cite[{следствие 22}]{Semenov2014geomprops}.
	\vfill
	\begin{llemma}[{\cite[{\S 3, замечание 6}]{Semenov2014geomprops}}]
		Пусть $X$~--- разделяющее множество и $X \subset \operatorname{Lin} Y$,
		где $\operatorname{Lin} Y$ обозначает линейную оболочку $Y$.
		Тогда $Y$ также является разделяющим множеством.
	\end{llemma}
\end{frame}


\begin{frame}\frametitle{\underline{Разделяющие множества}}

	\begin{ttheorem}
		Пусть
		$1 \geq a > b \geq 0$ и
		$\Omega^a_b = \{x\in\Omega : p(x) = a, q(x) = b\}$.
		\\
		Тогда $\Omega \subset \operatorname{Lin} \Omega^a_b$.
	\end{ttheorem}

	\vfill
	$\operatorname{Lin}$ ~--- линейная оболочка.
	\vfill
	\begin{ttheorem}
		Множество $\Omega^a_b$ является разделяющим.
		Т.к. при $a\neq 1$ или $b\neq 0$ множество $\Omega^a_b$ имеет меру нуль~\cite{semenov2010characteristic,connor1990almost},
		то оно является разделяющим множеством нулевой меры.
	\end{ttheorem}
\end{frame}


\begin{frame}\frametitle{\underline{Линейные оболочки}}

	Пусть
	\\
	$X^a_b = \{x\in\ell_\infty : p(x) = a,~ q(x) = b\}$,
	\\
	$Y^a_b\, = \{x\in A_0 : p(x) = a,~ q(x) = b\}$, где $a>b$.
	\vfill
	\begin{ttheorem}
		Пусть $a\neq -b$.
		Тогда справедливо равенство $\operatorname{Lin} Y^a_b = A_0$.
	\end{ttheorem}
	\vfill
	\begin{ttheorem}
		Справедливо равенство $\operatorname{Lin} X^a_b = \ell_\infty$.
	\end{ttheorem}

\end{frame}



\begin{frame}\frametitle{Размерность Хаусдорфа на $\Omega$}


	$s$-мерная мера Хаусдорфа непустого подмножества $F\subset \R^n$, $s > 0$ :
	$$\mathcal H^s(F) := \lim_{\delta\to0} \inf \left\{\sum_{i=1}^\infty \left({\rm diam} \ U_i \right)^s \ : \ F\subset \bigcup_{i=1}^\infty  U_i, \  0\leqslant {\rm diam} \ U_i \leqslant \delta \right\}$$

	\vfill

	Размерность Хаусдорфа множества  $F\subset \R^n$:
	$${\rm dim}_H F := \inf\{ s > 0 \ : \ \mathcal H^s(F)=0\}.$$

	\vfill

	\begin{ddefinition}
		Множество $E\subset\Omega$ называется самоподобным, если существуют $m\in\N_2$,
		$0< r_1, \dots, r_m<1$ и функции $f_j : \Omega \to \Omega$, $j=1,\dots, m$ такие, что
		$$
			\rho(f_j(x), f_j(y)) = r_j \rho(x,y), \ \forall \ x,y \in \Omega, \ j=1,\dots, m
			\mbox{~~и~~} E=\bigcup_{j=1}^m f_j(E).
		$$

	\end{ddefinition}
\end{frame}







\begin{frame}\frametitle{\underline{Размерность Хаусдорфа}}

	\begin{llemma}
		\label{lem:Hausdorf_measure}
		Пусть $E\subset\Omega$ и $TE = E$.
		Тогда размерность Хаусдорфа $E$ равна $1$.
	\end{llemma}

	\vfill

	Для разделяющего множества
	\begin{equation}
		\Omega^a_b = \{x\in\Omega : p(x) = a, q(x) = b\}
		,
		\quad 0 \leq b < a \leq 1
		,
	\end{equation}
	%являющегося разделяющим согласно следствию~\ref{crl:Lin_Omega_Sucheston},
	имеем $\mu\Omega^a_b = 0$, но  $\dim_H \Omega^a_b = 1$. А можно ещё <<меньше>>?

	\vfill


	\begin{ttheorem}
		Пусть $n\in\N$.
		Тогда существует разделяющее множество $E\subset\Omega$ такое,
		что $\dim_H E = 1/n$.
	\end{ttheorem}
\end{frame}


\begin{frame}\frametitle{Сходимость по Чезаро}
	\label{page:p_q_multiples}
	Дальнейшее ослабление понятия сходимости~---
	\\
	сходимость по Чезаро (сходимость в среднем).
	\vfill
	Последовательность $\{x_n\}\in\ell_\infty$ сходится по Чезаро к $t$, если
	\begin{equation}
		\lim_{n\to\infty}\frac1{n}\sum_{i=1}^n x_i = t
	\end{equation}

	Напомним: последовательность $\{x_n\}\in\ell_\infty$ почти сходится к $t$, если

	\begin{equation}
			\lim_{n\to\infty}  \frac1{n}\sum_{i=m+1}^{m+n} x_i = t
			\qquad
			\mbox{равномерно по~~}
			m\in\mathbb{N}
	\end{equation}

	\vfill

	\begin{equation}
		\label{eq:generalization_of_limits}
		\liminf_{n\to\infty} x_n \leq q(x) \leq \liminf_{n\to\infty}\frac1{n}\sum_{i=1}^n x_i
		\leq
		%\\ \leq
		\limsup_{n\to\infty}\frac1{n}\sum_{i=1}^n x_i
		\leq p(x)
		\leq \limsup_{n\to\infty} x_n
	\end{equation}

\end{frame}

\begin{frame}\frametitle{Sets of multiples}

	Каждый $x\in \Omega$ отождествим с подмножеством множества натуральных чисел
	$\supp x \subset \N$.

	Вслед за~\cite{hall1992behrend} обозначим
	$
		\mathscr{M}A = \{ka: k\in\N, a\in A\}
		,
	$
	через $\chi F$ "--- характеристическую функцию множества $F$.

	Например,
	\begin{gather}
		\chi \mathscr{M}\{2\} = \chi \mathscr{M}\{2, 4\} = \chi \mathscr{M}\{2,4,8,16,...\}
		= (0,1,0,1,0,1,0,1,0,...),
	\\
		\chi \mathscr{M}\{3\} = \chi \mathscr{M}\{3,9,27,...\} = (0,0,1,\;0,0,1,\;0,0,1,\;0,0,1,\;0,0,1,\;...),
	\\
		\chi \mathscr{M}\{2,3\} = \chi \mathscr{M}\{2,3,6\} = (0,1,1,1,0,1,\;0,1,1,1,0,1,\;0,1,1,1,0,1,...).
	\end{gather}

	Как связана структура множества $A$
	и  обобщения верхнего и нижнего пределов на последовательности $\chi \mathscr{M}\!A$ ?
	В~\cite{davenport1936sequences,davenport1951sequences} доказано, что для любого
	$A=\{a_1,a_2,...\}\subset\N$
	\begin{equation}
		\liminf_{n\to\infty}\frac1{n}\sum_{i=1}^n (\chi\mathscr{M}A)_i =
		\lim_{j\to\infty}\lim_{n\to\infty}\frac1{n}\sum_{i=1}^n (\chi\mathscr{M}\{a_1,a_2,...,a_j\})_i
		.
	\end{equation}
	В работе~\cite[{\S 7}]{besicovitch1935density} построено такое множество $A\subset\N$, что
	\begin{equation}
		\liminf_{n\to\infty}\frac1{n}\sum_{i=1}^n (\chi\mathscr{M}A)_i \neq
		\limsup_{n\to\infty}\frac1{n}\sum_{i=1}^n (\chi\mathscr{M}A)_i
		.
	\end{equation}

\end{frame}


\begin{frame}\frametitle{\underline{Верхний функционал Сачестона}}


	\begin{llemma}
		Пусть $\{p_1, ..., p_k\} \subset \N$,
		\begin{equation}
			x_k = \begin{cases}
				1, &\mbox{~если~} k = p_1^{j_1}\cdot p_2^{j_2}\cdot ... \cdot p_k^{j_k} \mbox{~для некоторых~} j_1,...,j_k\in\N,
				\\
				0  &\mbox{~иначе}.
			\end{cases}
		\end{equation}
		Тогда $x\in ac_0$.
	\end{llemma}
\end{frame}


\begin{frame}\frametitle{\underline{Верхний функционал Сачестона}}

	\begin{ddefinition}
		Будем говорить, что множество $A\subset\N$ обладает $P$-свойством,
		если для любого $n\in\N$ найдётся набор попарно взаимно простых чисел
		\begin{equation}
			\{a_{n,1}, a_{n,2}, ..., a_{n,n}  \} \subset A
			.
		\end{equation}
	\end{ddefinition}

	\begin{ttheorem}
		Пусть $A\subset \N\setminus\{1\}$.
		Тогда следующие условия эквивалентны:
		\begin{enumerate}%[label=(\roman*)]
			\item
				$A$ обладает $P$-свойством
			\item
				В $A$ существует бесконечное подмножество попарно взаимно простых чисел
			\item
				$p(\chi\mathscr{M}A)=1$.
		\end{enumerate}
	\end{ttheorem}

\end{frame}


\begin{frame}\frametitle{\underline{Нижний функционал Сачестона}}


	\begin{ttheorem}
		Пусть $A = \{a_1, a_2, ..., a_n,...\}$ "--- бесконечное множество попарно взаимно простых чисел
		и $a_{n+1}>a_1\cdot...\cdot a_n$.
		Тогда
		\begin{equation}
			q(\chi\mathscr{M}A) = 1-\prod_{j=1}^\infty \left(1-\frac{1}{a_j}\right)
			.
		\end{equation}
	\end{ttheorem}

	\begin{llemma}
		Пусть $\varepsilon \in  (0; 1{]}$.
		Существует бесконечное множество попарно непересекающихся подмножеств простых чисел
		$A_i$ такое, что $q(\chi\mathscr{M}A_i)\geq\varepsilon$ для любого $i\in\N$.
	\end{llemma}
\end{frame}





\setbeamertemplate{bibliography item}{\insertbiblabel\small}

\begin{frame}[allowframebreaks]
	\printbibliography{}
\end{frame}

%\begin{frame}[allowframebreaks]
%	\tiny
%	\printbibitembibliography{}
%\end{frame}



\setbeamertemplate{navigation symbols}{}

\begin{frame}
	{
		\huge\centering
		~\\~\\~\\~\\
		Спасибо за внимание
	}
	~\\
	\vspace{6.28em}
	nickkolok@mail.ru, avdeev@math.vsu.ru
	\\
	github.com/nickkolok
	\\
	arxiv.org/a/avdeev\_n\_1.html
	\label{slide:final}
\end{frame}


\addtobeamertemplate{navigation symbols}{}{%
    \usebeamerfont{footline}%
    \usebeamercolor[fg]{darkblue}%
    \hspace{1em}%
    \huge\bf
    \insertframenumber/\pageref{slide:final}\!\!
}


\begin{frame}\frametitle{Пустой слайд}

	Здесь ничего нет!
\end{frame}

\begin{frame}\frametitle{Пустой слайд}

	...только следы...
\end{frame}




\begin{frame}\frametitle{Следы}
	Пусть $(F, +, \cdot)$~--- некоторое кольцо.
	\begin{ddefinition}
		Следом называется такой функционал $\tau : F \to \R$,
		что для любых $A,B\in F$
		\begin{equation}
			\tau (A\cdot B) = \tau (B\cdot A)
			.
		\end{equation}
	\end{ddefinition}
	\vfill
	Пример: классический след на кольце квадратных матриц
	\\
	(сумма диагональных элементов).


\end{frame}


\begin{frame}\frametitle{Следы на операторах}
	Пусть $H$~--- сепарабельное гильбертово пространство,
	$B(H)$~--- алгебра всех ограниченных линейных операторов на $H$.

	\bigskip

	Классический след:
	$${\rm Tr} (A)= \sum_{k=0}^\infty \lambda(k,A),$$
	где $\{\lambda(k,A)\}_{k=0}^\infty$ --- последовательность собственных значений компактного оператора $A$.

	\bigskip

	След называется нормальным, если
	\\
	для любой монотонно сходящейся $A_k\to A$ верно $\operatorname{Tr} A_k \to\operatorname{Tr} A$.

	\vfill
	Классический след является единственным нормальным (с точностью до константы).

	А есть ли другие, анормальные?


\end{frame}



 \begin{frame}\frametitle{Конструкция Диксмье (1966)}

	Сингулярные значения $\mu(k,A)=\lambda(k,|A|)$ (в убывающем порядке),
	где $|A|=\sqrt{A^*A}$.

	Пространство Лоренца:
	$$\mathcal M_{1,\infty} := \left\{ A \ \text{компактен} \ : \ \sup_{n> 0}  \frac{1}{\log(1+n)} \sum_{k=0}^{n-1} \mu(k,A) < \infty \right \}.$$

	Положим
	$${\rm Tr}_\omega (A):= \omega \left( \frac{1}{\log(1+n)} \sum_{k=0}^{n-1} \lambda(k,A) \right), \quad 0\le A \in \mathcal M_{1,\infty},$$
	где $\omega\in \ell_\infty^*$ такое, что
	$$\omega(x_0, x_1, \dots)=\omega(x_0, x_0, x_1, x_1, \dots).$$


	\bigskip
	1. ${\rm Tr}_\omega$ --- след;

	2. ${\rm Tr}_\omega$ нетривиален: ${\rm Tr}_\omega ({\rm diag} \left\{ \frac1{n+1} \right\})=1$;

	3. ${\rm Tr}_\omega(A)=0$, если $A$ любой оператор конечного ранга.
	\bigskip

	\begin{center}
	Т.е. ${\rm Tr}_\omega$ --- нетривиальный, анормальный след.
	\end{center}
\end{frame}





\begin{frame}\frametitle{Следы и банаховы пределы}
	 Для любого банахова предела $B$ функционал
	 $$A \mapsto B \left( \frac{1}{1+n} \sum_{k=0}^{2^n-1} \lambda(k,A) \right), \quad 0\le A \in \mathcal M_{1,\infty}$$
	продолжается по линейности на все пространство $\mathcal M_{1,\infty}$ до некоторого следа Диксмье\footnote{П. Г. Доддс, Б. де Пагтер, А. А. Седаев, Е. М. Семенов, Ф. А. Сукочев, Изв. РАН. Сер. матем., 67:6 (2003),  111–136.}.

	\vfill

		Отображение $\omega \mapsto {\rm Tr}_\omega$ из множества $\sigma_2$-инвариантных обобщённых пределов на $\ell_\infty$ во множество сингулярных следов на $\mathcal M_{1,\infty}$ не является ни инъективным,\footnote{Sukochev, F., Usachev, A., and Zanin, D., Adv. Math. 239 (2013), 164--189.
	} ни сюръективным\footnote{Kalton, N., and Sukochev, F., Canad. Math. Bull. 51, 1 (2008), 67--80.}.
\end{frame}




\begin{frame}\frametitle{Альтернативная конструкция сингулярных следов (Е.М. Семенов и др.)}

	Обозначим через $\mathcal L_{1,\infty}$ линейное пространство всех компактных операторов на фиксированном гильбертовом прострастве $H$, для которых квази-норма
	$$\|A\|_{\mathcal L_{1,\infty}} = \sup_{n \ge 0} (n+1) \mu(n,A)$$
	конечна. Легко видеть, что $\mathcal L_{1,\infty} \subsetneq \mathcal M_{1,\infty}$.

	\vfill

	Зафиксируем некий базис $\{e_n\}$ в гильбертовом пространстве $H$ и обозначим через ${\rm diag}$ диагональный оператор в $B(H)$, соответствующий этому базису.

\end{frame}

\begin{frame}\frametitle{Альтернативная конструкция сингулярных следов (Е.М. Семенов и др.)}

	Определим оператор $D: \ell_\infty \to \mathcal L_{1,\infty}$ следующим образом
	\begin{align*}
	D(x_0, x_1, \dots) := \log 2 \cdot {\rm diag} \sum_{n=0}^\infty \frac{x_n}{2^n} \chi_{[2^n-1,2^{n+1}-2]}.
	\end{align*}

	\begin{ttheorem}
		(i) Для любого инвариантного относительно сдвига функционала $\theta$ на $\ell_\infty$ функционал
	\begin{equation}\label{ssec:Gen:formula1}
	\tau(A) = \theta\left(\sum_{i=2^n-1}^{2^{n+1}-2} \lambda(i,A) \right) , \quad A\ge0,
	\end{equation}
	продляется по линейности до сингулярного следа $\tau$ на $\mathcal L_{1,\infty}$.

	(ii) Для любого сингулярного следа $\tau$ на $\mathcal L_{1,\infty}$ и инвариантного относительно сдвига функционала $\theta=\tau\circ D$ на $\ell_\infty$ справедлива формула ~\eqref{ssec:Gen:formula1}.
	\end{ttheorem}

\end{frame}





\begin{frame}\frametitle{Система Хаара}


Обозначим через $F$ множество индексов $(n,k)$ таких, что
$k=0,1$ для $n=0$ и $1\leq k \leq 2^n$ для $n\geq 1$, а для любых
$(n,k)\in F$ через $\Delta_n^k$ обозначим интервал $
(\frac{k-1}{2^n},\frac{k}{2^n})$.

\emph{Система Хаара} ~--- последовательность функций $\varphi_{0,0}=1$,
$$\varphi_{n,k} (t)=\begin{cases} 2^\frac{n}2, &t\in \Delta_{n+1}^{2k-1}\\
-2^\frac{n}2, &t\in \Delta_{n+1}^{2k}\\0,  &\text{для} \,\, t \,\,
\text{из остальных} \, \Delta_i^j,
\end{cases}$$
Значения в концах отрезка $[0,1]$ и в
точках разрыва выбираются из условий
\begin{equation}
	\varphi_{n,k} (0)= \!\lim_{\delta \to 0+}\!\varphi_{n,k}
	(\delta);
	\hfill
	\varphi_{n,k} (1)= \!\lim_{\delta \to
	0+}\!\varphi_{n,k} (1-\delta);
	\hfill
	\varphi_{n,k} (t)= \lim_{\delta \to 0}\frac{\varphi_{n,k}
	(t+\delta)+\varphi_{n,k} (t-\delta)}2
\end{equation}


$f_{n,k}(x)$ ~--- коэффициенты Фурье по
системе Хаара функции $x\in~L_1 [0,1]$.

Биекция $F \leftrightarrow \N$ задаётся формулой
$m=2^n+k$, где
$(n,k)\in F$.

\vfill

Система Хаара не является равномерно ограниченной.

\end{frame}

\begin{frame}\frametitle{Пространства $L_{p,\infty}$}

	Пусть $h(t)$ "--- некоторая измеримая на $[0;1]$ функция.
	Тогда её функция распределения:
	\begin{equation}
		\eta_h(\tau) = \mes \{t : h(t) > \tau \}
		.
	\end{equation}

	Невозрастающая перестановка $h^* : [0;1] \to \R$:
	\begin{equation}
		h^*(t) = \inf\{\tau: \eta_{|h|}(\tau) \leq t \}.
	\end{equation}


	$L_{p,\infty}\quad(1<p<\infty)$ "--- пространство
	функций с нормой
	\begin{equation}
		\| x \|_{p,\infty}=\sup\limits_{0\leq \tau \leq
		1} \tau^{\frac{1}{p}-1}\int_0^\tau  x^*(t)dt =
		%\\=
		%\| x \|_{p,\infty}=
		\sup_{e\subset[0,1], \ \mes\, e > 0}
		\{(\mes\, e)^{\frac{1}{p}-1}\int_e \mid x(t)\mid dt\}<\infty
		.
	\end{equation}


\end{frame}

\begin{frame}\frametitle{Коэффициенты Фурье-Хаара (Усачев, 2009)}


	\begin{ttheorem}[Мерсера]
		Коэффициенты Фурье любой суммируемой функции
		по равномерно ограниченной ортонормированной системе
		стремятся к нулю.
	\end{ttheorem}

	\vfill

	Система Хаара не является равномерно ограниченной.

	Существует функция $h\in L_p$, $1<p<2$, коэффициенты Фурье-Хаара
	которой не стремятся к нулю.


	\begin{ttheorem}
		Если $1<p<\infty$, $h\in
		L_{p,\infty}$, то $\left\{m^{\frac{1}2-\frac{1}p}\,
		f_m(h)\right\}_{m=1}^\infty \in ac_0$.
	\end{ttheorem}

	\begin{ccorollary}
		Если $h\in L_{2,\infty}$, то
		$\left\{f_m(h)\right\}_{m=1}^\infty\in ac_0$.
	\end{ccorollary}

	\vfill

	Это аналог теоремы Мерсера для
	пространства $L_{2,\infty}$ и системы Хаара.

\end{frame}




\end{document}
