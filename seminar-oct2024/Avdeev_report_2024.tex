\documentclass[10pt,pdf,hyperref={unicode},aspectratio=169,color={usenames, dvipsnames}]{beamer}
\usepackage{amsmath}
\usepackage[utf8]{inputenc}
\usepackage[english,russian]{babel}
\usepackage{amsfonts}
\usepackage{amsfonts,amssymb}
\usepackage{amssymb}
\usepackage{latexsym}
\usepackage{euscript}
\usepackage{enumerate}
\usepackage{graphics}
\usepackage{graphicx}
\usepackage{geometry}
\usepackage{wrapfig}

\usepackage{bbm}
\usepackage{mathrsfs}



\DeclareMathOperator{\ext}{ext}
\DeclareMathOperator{\mes}{mes}
\DeclareMathOperator{\supp}{supp}
\DeclareMathOperator{\conv}{conv}
\DeclareMathOperator{\diam}{diam}

\newcommand{\N}{\ensuremath{\mathbb{N}}}
\newcommand{\Q}{\ensuremath{\mathbb{Q}}}
\newcommand{\R}{\ensuremath{\mathbb{R}}}
\newcommand{\B}{\ensuremath{\mathfrak{B}}}
\newcommand{\Iac}{\mathcal{I}(ac_0)}
\newcommand{\Dac}{\mathcal{D}(ac_0)}
\newcommand{\one}{\ensuremath{\mathbbm 1}}


\newcommand{\longcomment}[1]{}


%Only referenced equations are numbered
\usepackage{mathtools}
\mathtoolsset{showonlyrefs}

%\mathtoolsset{showonlyrefs=false}
% (an equation/multline to be force-numbered)
%\mathtoolsset{showonlyrefs=true}


\usepackage{varwidth}

\usepackage{amsthm}
\theoremstyle{definition}
\newtheorem{llemma}{Лемма}
\newtheorem{ttheorem}[llemma]{Теорема}
\newtheorem{eexample}[llemma]{Пример}
\newtheorem{property}[llemma]{Свойство}
\newtheorem{remark}[llemma]{Замечание}
\newtheorem{ccorollary}{Следствие}[llemma]
\newtheorem{hhypothesis}{Гипотеза}[llemma]
\newtheorem{ddefinition}{Определение}[llemma]


\usepackage[backend=biber,style=gost-numeric,sorting=none]{biblatex}
\addbibresource{../bib/Semenov.bib}
\addbibresource{../bib/my.bib}
\addbibresource{../bib/ext.bib}
\addbibresource{../bib/classic.bib}
\addbibresource{../bib/Damerau-Levenstein.bib}
\addbibresource{../bib/general_monographies.bib}
\addbibresource{../bib/Bibliography_from_Usachev.bib}


\righthyphenmin=2

%\usetheme{Antibes}
\usefonttheme{professionalfonts} % using non standard fonts for beamer
\usefonttheme[onlymath]{serif} % default family is serif
%\usepackage{fontspec}
%\setmainfont{Liberation Serif}
\begin{document}


\setbeamertemplate{navigation symbols}{\large{}}

\title{
	Инвариантные банаховы пределы
}
\author{докладчик: асп. Н.Н.~Авдеев\\науч. рук.: д.ф.-м.н., проф. Е.М. Семенов}
\institute{ Воронежский Государственный Университет \\
		{Работа выполнена при поддержке Фонда БАЗИС, проект No 22-7-2-27-3.
	}}
\date{Москва 2024}

\maketitle


\setbeamertemplate{footline}[frame number]
\setbeamertemplate{navigation symbols}{\large{}}


\begin{frame}\frametitle{Пространство $\ell_\infty$}
	$\ell_\infty$~--- пространство всех ограниченных последовательностей
	$x=(x_1, x_2, ..., x_n, ...)$
	с~нормой
	$$
		\|x\|_{\ell_\infty} = \sup_{k\in\mathbb{N}} |x_k|
	$$
	{Свойства:}

	\begin{itemize}
		\item
			$\ell_\infty$~--- линейное пространство над полем $\mathbb{R}$
		\item
			$\ell_\infty$  несепарабельно
		\item
			$\ell_1 \subset \ell_2 \subset \dots \subset \ell_\infty$
		\item
			$\ell_1^* = \ell_\infty$
		\item
			$\ell_\infty^* \neq \ell_1$
	\end{itemize}
\end{frame}


\begin{frame}\frametitle{Обобщения понятия сходимости в $\ell_\infty$}
	%\hspace{1.468em}
	\begin{varwidth}[t]{\linewidth}
		\centering
		Верхний и нижний
		\\
		пределы:
		\\
		$\displaystyle \limsup_{n\to\infty} x_n$
		,~
		$\displaystyle \liminf_{n\to\infty} x_n$
		\\~\\
		\emph{
			не характеризуют
			\\
			распределение элементов
		}
	\end{varwidth}
	\hfill
	\begin{varwidth}[t]{\linewidth}
		\centering
		Сходимость в среднем
		\\
		(по Чезаро):
		$\displaystyle \lim_{n\to\infty} (Cx)_n$
		\\
		$\displaystyle (Cx)_n = \frac1n\sum_{i=1}^n x_i$
		\\~\\
		\emph{слишком универсальна}
	\end{varwidth}
	\hfill
	\begin{varwidth}[t]{\linewidth}
		\centering
		Банаховы пределы $B\in \mathfrak{B}$:
		\\
		т. Хана--Банаха к $\lim: c\to\mathbb R$
		\\
			$B \geqslant 0$
		\\
			$B\mathbbm{1} = 1$
		\\
			$B=BT$
		\\
		\emph{почти сходимость}
	\end{varwidth}
	\\~\\~\\
	\begin{varwidth}[t]{\linewidth}
		\centering
		\emph{Критерий Лоренца:}
		$\displaystyle
			x\in ac_t
			\Leftrightarrow
			\forall(B\in\mathfrak{B})[Bx = t]
			\Leftrightarrow
			\lim_{n\to\infty}  \sum_{k=m+1}^{m+n} \frac{x_k}n = t
		$
		равном. по $m\in\mathbb{N}$.
	\end{varwidth}
	\\~\\
	\vspace{0.8em}
	\begin{varwidth}[t]{\linewidth}
		$\displaystyle ac = \mathop{\cup}\limits_{t\in\mathbb R} ac_t$ "--- пространство почти сходящихся последовательностей
	\end{varwidth}
\end{frame}


\begin{frame}\frametitle{Банаховы пределы}
	Банаховым пределом называется функционал $B\in \ell_\infty^*$ такой, что:
	\begin{enumerate}
		\item
			$B \geqslant 0$
		\item
			$B\mathbbm{1} = 1$
		\item
			$B=BT$
	\end{enumerate}
	Здесь $\mathbbm{1}=(1,1,1,1,1,...)$,
	$T$~--- оператор сдвига: $T(x_1, x_2, x_3, ...) = (x_2, x_3, ...)$.

	Простейшие свойства:
	\begin{itemize}
		\item
			$\|B\|_{\ell_\infty^*} = 1$
		\item
			$Bx = \lim_{n\to\infty} x_n$ для любого $x=(x_1, x_2, ...) \in c$,
			\\
			где $c$~--- множество сходящихся последовательностей.

			Таким образом,
			банахов предел~--- естественное обобщение понятия предела сходящейся последовательности
			на все ограниченные последовательности.
	\end{itemize}
\end{frame}

\longcomment{
\begin{frame}\frametitle{Свойства множества банаховых пределов $\mathfrak{B}$}
	\begin{enumerate}
		\item
			$\mathfrak{B}$~--- замкнутое выпуклое подмножество $ S_{\ell_\infty^*}$~---
			единичной сферы пространства $\ell_\infty^*$
		\item
			$d(\mathfrak{B}) = 2$
		\item
			$\mathfrak{B}$ слабо *-компактно
	\end{enumerate}
\end{frame}
}

\begin{frame}\frametitle{Теорема Лоренца~\cite{L}}
	Для заданного $r\in\mathbb{R}$ равенство $Bx=r$ выполнено для всех $B\in\mathfrak{B}$
	тогда и только тогда, когда
	\begin{equation*}
		\lim_{n\to\infty} \frac{1}{n} \sum_{k=m+1}^{m+n} x_k = r
	\end{equation*}
	сходится равномерно по всем $m\in\mathbb{N}$.

	Множество всех таких $x \in \ell_\infty$ обозначается $ac$.
\end{frame}

\begin{frame}\frametitle{Теорема Сачестона~\cite{S}}
	является уточнением теоремы Лоренца.
	Пусть
	\begin{equation*}
		q(x) = \lim_{n\to\infty} \inf_{m\in\mathbb{N}}  \frac{1}{n} \sum_{k=m+1}^{m+n} x_k,
		~~~~~~~~
		p(x) = \lim_{n\to\infty} \sup_{m\in\mathbb{N}}  \frac{1}{n} \sum_{k=m+1}^{m+n} x_k.
	\end{equation*}
	Тогда для любых $x\in \ell_\infty$ и $B\in\mathfrak{B}$
	\begin{equation}\label{Sucheston}
		q(x) \leqslant Bx \leqslant p(x)
	\end{equation}
	Неравенства (\ref{Sucheston}) точны:
	для данного $x$ для любого $r\in[q(x); p(x)]$ найдётся банахов предел
	$B\in\mathfrak{B}$ такой, что $Bx = r$.
\end{frame}

\begin{frame}\frametitle{Что такое почти сходимость?}
	\begin{varwidth}[t]{\linewidth}
		\hspace{3em}
		Пример $x\in ac_0$, но $\displaystyle0=\liminf_{n\to\infty}x_n < \limsup_{n\to\infty}x_n=1$:~~
		$$\displaystyle
			x_k=\begin{cases}
				1, & k=2^j,~j\in\mathbb N
				\\
				0 & \mbox{для прочих~} k
			\end{cases}
		$$
		$x=(0,1,0,1,\;0,0,0,1,\;\;0,0,0,0,\;0,0,0,1,\;\;0,0,0,0,\;0,0,0,0,\;\;0,0,0,0,\;0,0,0,1,\;\;0,0,0,0,\;...)$
	\end{varwidth}
	\\
	\vspace{2em}
	\begin{varwidth}[t]{\linewidth}
		Пример сходящейся по Чезаро к нулю последовательности $x\notin ac$:~~
		$$\displaystyle
			x_k=\begin{cases}
				1, & 2^j-j < k \leq 2^j,~j\in\mathbb N
				\\
				0 & \mbox{для прочих~} k
			\end{cases}
		$$
		$x=(0,1,1,1,\;0,1,1,1,\;\;0,0,0,0,\;1,1,1,1,$\\
		$\phantom{x=(}0,0,0,0,\;0,0,0,0,\;\;0,0,0,1,\;1,1,1,1,$\\
		$\phantom{x=(}0,0,0,0,\;0,0,0,0,\;\;0,0,0,0,\;0,0,0,0,$\\
		$\phantom{x=(}0,0,0,0,\;0,0,0,0,\;\;0,0,1,1,\;1,1,1,1,...)$
	\end{varwidth}
\end{frame}





\begin{frame}\frametitle{Структура работы}
	\begin{enumerate}
		\item
			Последовательности, почти сходящиеся к нулю
			(пространство $ac_0$)
		\item
			$\alpha$--функция как асимптотическая характеристика ограниченной последовательности
		\item
			Банаховы пределы, инвариантные относительно некоторых операторов
		\item
			Функционалы Сачестона и линейные оболочки
		\item
			Функционалы Сачестона и мультипликативные свойства носителя последовательности
	\end{enumerate}
\end{frame}


\begin{frame}\frametitle{\underline{Диадическая последовательность}}
	\begin{ddefinition}
		Последовательность $x\in\ell_\infty$
		называется \emph{диадической}, если существует такое $X=\{a;b\}$,
		что $x_k \in X$ для всех $k\in \mathbb N$.
	\end{ddefinition}

	Например, диадическими являются все последовательности из нулей и единиц.


	\begin{llemma}
		Пусть $x\in\ell_\infty$, $\|x\|\leq 1$.
		Тогда найдётся такая $h\in ac_0$, что $(x+h)_n = \pm 1$ для всех $n\in\mathbb N$.
	\end{llemma}

	\begin{ccorollary}
		Пусть $x\in\ell_\infty$.
		Тогда найдётся такая $h\in ac_0$, что для всех $n\in\mathbb N$
		\begin{equation*}
			(x+h)_n \in \{\inf_{n\in\mathbb N} x_n,\sup_{n\in\mathbb N} x_n\}
			.
		\end{equation*}
	\end{ccorollary}
	\vspace{-2em}
	\begin{ccorollary}
		Пусть $x\in\ell_\infty$.
		Тогда найдётся такая $h\in ac_0$, что $(x+h)_n =\pm \|x\|$ для всех $n\in\mathbb N$.
	\end{ccorollary}

	\vfill
\end{frame}




\begin{frame}\frametitle{Разделяющие множества}
	\label{page:p_q_linhulls}
	Множество $M\subset \ell_\infty$ называется разделяющим,
	если для любых двух различных банаховых пределов $B_1$ и $B_2$
	найдётся такой $x\in M$, что $B_1 x \ne B_2 x$.
	\\~\\
	Интересны <<маленькие>> разделяющие множества.
	\\~\\
	Например, $\Omega$ разделяющее~\cite{semenov2010characteristic}.
	\\~\\
	Этот результат можно получить непосредственно из теоремы о разложении.
\end{frame}





\begin{frame}\frametitle{\underline{Интервальный критерий}}
	%\begin{theorem}
		\label{thm:M_j_ac0_inf_lim}
		Пусть $n_i$~--- строго возрастающая последовательность натуральных чисел,
		\begin{equation}
			\label{eq:definition_M_j}
			M(j) = \liminf_{i\to\infty} \left(n_{i+j} - n_i\right),
		\end{equation}
		\begin{equation}
			x_k = \left\{\begin{array}{ll}
				1, & \mbox{~если~} k = n_i
				\\
				0  & \mbox{~иначе~}
			\end{array}\right.
		\end{equation}
		Тогда следующие условия эквивалентны:
		\\~\\
		(i)   $x \in ac_0$;
		\\~\\
		(ii)  $\lim\limits_{j \to \infty} \dfrac{M(j)}{j} = +\infty$;
		\\~\\
		(iii) $\inf\limits_{j \in \N}     \dfrac{M(j)}{j} = +\infty$.
	%\end{theorem}
\end{frame}



\begin{frame}\frametitle{\underline{Усиленная теорема Коннора}}
	Пусть на множестве $\Omega=\{0,1\}^\N$ задана вероятностная мера <<честной монетки>> $\mu$.
	Коннор доказал~\cite{connor1990almost}, что $\mu(\Omega\cap ac)=0$.

	Обобщим этот результат.

	\begin{ttheorem}
	%\label{thm:Connor_generalized}
	Мера множества $F=\{x\in\Omega : q(x) = 0 \wedge p(x)= 1\}$,
	где $p(x)$ и $q(x)$~--- верхний и нижний функционалы Сачестона соответственно,
	равна 1.
	\end{ttheorem}
\end{frame}



\begin{frame}\frametitle{$\alpha$--функция}
	\begin{equation}
		\alpha(x) = \limsup_{i\to\infty} \max_{i<j\leqslant 2i} |x_i - x_j|
		.
	\end{equation}
	\vfill
	\begin{equation}
		\alpha(c)=0,
		\qquad
		\alpha(x+y) \leq \alpha(x) + \alpha(y),
		\qquad
		|\alpha(x) - \alpha(y)| \leq 2 \|x-y\|
		.
	\end{equation}
	\vfill
	$\alpha$--функция --- <<мера несходимости>> последовательности;
	\\
	равенство $\alpha(x) = 0$, однако, не гарантирует сходимость.

\end{frame}

\begin{frame}\frametitle{\underline{$\alpha$--функция и оператор сдвига}}

	$\alpha$--функция не инвариантна относительно оператора сдвига $T$
	\\
	(в отличие от банаховых пределов).

	\begin{ttheorem}
		Для любых $x\in\ell_\infty$ и $n \in \N$
		\begin{equation}\label{est_alpha_Tn_x}
			\frac{1}{2}\alpha(x) \leq \alpha(T^n x) \leq \alpha(x)
			.
		\end{equation}
	\end{ttheorem}

	\begin{ttheorem}
		Пусть $\beta_k$~--- монотонная невозрастающая последовательность,
		$\beta_k \to \beta$, $\beta\in\left[\frac{1}{2}; 1\right]$, $\beta_1 \leq 1$.
		Тогда существует такой $x\in\ell_\infty$, что для любого натурального $n$
		\begin{equation}
			\frac{\alpha(T^n x)}{\alpha(x)} = \beta_n.
		\end{equation}
	\end{ttheorem}
\end{frame}


\begin{frame}\frametitle{\underline{$\alpha$--функция и другие классические операторы}}
	\begin{ttheorem}
		Для любого $n\in\N$ и любого $x\in\ell_\infty$ выполнено
		\begin{equation}
			\alpha(\sigma_n x) = \alpha(x)
			\qquad\mbox{и}\qquad
			\alpha(\sigma_{1/n} x) \leq \left( 2- \frac{1}{n} \right) \alpha(x)
			.
		\end{equation}
	\end{ttheorem}

	\begin{ttheorem}
		Имеет место равенство
		\begin{equation}
			\sup_{x\in\ell_\infty, \alpha(x)\neq 0} \frac{\alpha(Cx)}{\alpha(x)}=1
			.
		\end{equation}
	\end{ttheorem}

	\begin{ttheorem}
		Пусть $(x\cdot y)_k = x_k\cdot y_k$.
		Тогда
		$\alpha(x\cdot y)\leq \alpha(x)\cdot \|y\|_* + \alpha(y)\cdot \|x\|_*$,
		где
		\begin{equation}
			\|x\|_* = \limsup_{k\to\infty} |x_k|
		\end{equation}
		есть  фактор-норма по $c_0$ на пространстве $\ell_\infty$.
	\end{ttheorem}
\end{frame}


\begin{frame}\frametitle{\underline{Пространство $A_0$}}

	Пространство $A_0 = \{x: \alpha(x) = 0\}$
	несепарабельно, замкнуто относительно покоординатного умножения,
	операторов левого и правого сдвигов $T$ и $U$,
	оператора Чезаро $C$,
	операторов растяжения $\sigma_n$ и усредняющего сжатия $\sigma_{1/n}$.

	\begin{equation}
		A_0 \cap ac_0 = c_0,
		\qquad
		A_0 \cap ac\, = c,
	\end{equation}

	\vfill

	Оба вложения
	\qquad$
		c_0 \subset A_0 \subset \ell_\infty
	$\qquad
	недополняемы.

	\vfill

	Более того, недополняемо вложение $c_0 \subset ac_0$
	\\
	(недополняемость вложения $ac_0 \subset \ell_\infty$ доказана Е.А. Алехно~\cite[{Theorem 8}]{alekhno2006propertiesII})
\end{frame}


\begin{frame}\frametitle{Инвариантные банаховы пределы}
	\label{page:invariant_BL}

	\vfill
	Для любого банахова предела $B$ выполнено $B=BT$.
	\vfill
	Относительно каких ещё операторов банаховы пределы инвариантны?
	\vfill
	Даже для естественного оператора растяжения $\sigma_n$, $n\in\N_2$,
	инвариантны не все банаховы пределы.
	\vfill
	Банаховых пределов, инвариантных относительно оператора Чезаро $C$,
	ещё <<меньше>>:
	\begin{equation}
		B = BC \ \Rightarrow \ B = B\sigma_n \quad \mbox{для любого} \quad n \in \N
	\end{equation}
	\vfill


\end{frame}







\begin{frame}\frametitle{Инвариантные банаховы пределы }
	\begin{ddefinition}
		Банахов предел $B$ называется инвариантным относительно оператора $H:\ell_\infty \to \ell_\infty$,
		если $Bx = BHx$ для любого $x\in\ell_\infty$. Обозначение: $B\in\B(H)$.
	\end{ddefinition}
	\vfill
	\begin{ttheorem}[Признак Семенова--Сукочева,~\cite{Semenov2010invariant}]
		\label{thm:Semenov_Sukochev_conditions}
		Пусть линейный оператор $H:\ell_\infty\to\ell_\infty$ таков, что:
		\\(i)   $H \geqslant 0, H \one=\one$;
		\\(ii)  $H c_0 \subset c_0$;
		\\(iii) $\limsup _{j \rightarrow \infty}(A(I-T) x)_j \geqslant 0$ для всех $x \in \ell_{\infty}, A \in R$,
		\\где
		\begin{equation}
			R=R(H):=\operatorname{conv}\left\{H^n, n=1,2, \ldots\right\}
			.
		\end{equation}
		Тогда $\B(H) \ne\varnothing$.
	\end{ttheorem}
	\vfill
	Эти достаточные условия очень сильны и избыточны.
	\vfill
	Им удовлетворяют, например, операторы растяжения $\sigma_n$ и Чезаро $C$.
\end{frame}



\begin{frame}\frametitle{B-регулярные операторы}
	\begin{ddefinition}
		Оператор $H : \ell_\infty \to \ell_\infty$ называется \emph{эберлейновым},
		если $\B(H) \ne\varnothing$.
	\end{ddefinition}
	\begin{ddefinition}[\cite{alekhno2018invariant}]
		\label{def:B-regular_operator}
		Оператор $H : \ell_\infty \to \ell_\infty$ называется \emph{B-регулярным},
		если $H^*\B \subseteq \B$.
		\\
		(Или, что то же самое, $BH\in\B$ для любого $B\in\B$.)
	\end{ddefinition}
	\begin{ttheorem}[\cite{alekhno2018invariant}]
		\label{thm:B-regular_is_Eberlein}
		Любой B-регулярный оператор "--- эберлейнов.
	\end{ttheorem}
	\begin{ttheorem}[\cite{alekhno2018invariant}]
		Оператор $H:\ell_\infty \to \ell_\infty$ является B-регулярным тогда и только тогда, когда:
		\\(i) $H\one \in ac_1$;
		\\(ii) $q(Hx)\geq 0$ для любого $x\geq 0$;
		\\(iii) $H ac_0 \subseteq ac_0$.
	\end{ttheorem}
\end{frame}


\begin{frame}\frametitle{{Новые классы операторов}}
	\vfill
	\begin{llemma}
		\label{lem:suff_B_reg}
		Пусть оператор $H$ удовлетворяет следующим условиям:
		\\(i)   $(Hx)^- \in ac_0$ для любого $x\geq 0$;
		\\(ii)  $H\one\in ac_1$;
		\\(iii) $H c_0 \subset ac_0$;
		\\(iv)  $HT-TH : \ell_\infty \to ac_0$.% для некоторых $k,m \in \N_0$.

		Тогда оператор $H$ является В-регулярным.
	\end{llemma}
    \vfill
	\begin{ddefinition}
		Будем называть оператор $H:\ell_\infty \to \ell_\infty$ \emph{полуэберлейновым}, если $BH\in\mathfrak B$ для некоторого $B\in\mathfrak B$.
	\end{ddefinition}
	\vfill
	\begin{ddefinition}
		Будем называть оператор $H:\ell_\infty \to \ell_\infty$ \emph{Q-регулярным}, если
		для любого $x\in\ell_\infty$ выполнено равенство $q(Hx) = q(x)$.
	\end{ddefinition}
	\vfill
\end{frame}


\begin{frame}\frametitle{{Новые классы операторов}}
	\vfill
	\begin{ddefinition}
		Разреженным назовём оператор
		взятия подпоследовательности
		$H: \ell_\infty \to \ell_\infty$,
		определяемый следующим образом:
		$$
		H(x_1, x_2, x_3, \ldots) = (x_{m_1}, x_{m_1 + 1}, \ldots, x_{n_1 - 1}, x_{n_1};
		x_{m_2}, x_{m_2 + 1}, \ldots, x_{n_2 - 1}, x_{n_2}; \ldots),
		$$
		где для всех
		$k \geqslant k_0 \in \mathbb N$
		выполнено
		\begin{equation} \label{eq:sprawling_mk_nk}
		m_k \leqslant n_k < m_{k+1}\leqslant n_{k+1},
	    \end{equation}
		при этом
		$$
		\lim\limits_{k \to \infty} (n_k - m_k) = \infty
		.
		$$

		\begin{itemize}
			\item
				Всякий разреженный оператор является В-регулярным.
			\item
				Обратное, очевидно, неверно.
			\item
				Существует разреженный оператор, не являющийся Q-регулярным.
		\end{itemize}
	\end{ddefinition}
	\vfill
\end{frame}

\begin{frame}\frametitle{{Новые классы операторов}}
	\begin{ttheorem}
		\label{thm:sprawling_is_B-regular}
		Любой разреженный оператор $H$ является $B$-регулярным.
	\end{ttheorem}

		Непосредственно проверим условия леммы \ref{lem:suff_B_reg}.
		\begin{enumerate}
			\item[i)] From the definition, one can easily see that $H\geq 0$
			and thus $(Hx)^- = (0,0,0...)\in ac_0$ for any $x\geq 0$.
			\item[ii)] $H \one = \one \in ac_1$.
			\item[iii)] The inclusion $Hc_0 \subset c_0 \subset ac_0$ holds due to the inequality~\eqref{eq:sprawling_mk_nk}.
		\end{enumerate}
\end{frame}

\begin{frame}
	    \begin{enumerate}
	        \item[iv)] Наконец, покажем, что $HT - TH : \ell_\infty \to ac_0$. Имеем
			\begin{align*}
				THx &= T(x_{m_1}, x_{m_1 + 1}, \ldots, x_{n_1 - 1}, x_{n_1};
				x_{m_2}, x_{m_2 + 1}, \ldots, x_{n_2 - 1}, x_{n_2}; \ldots) = \\
				&= (x_{m_1 + 1}, x_{m_1 + 2}, \ldots, x_{n_1 - 1}, x_{n_1};
				x_{m_2}, x_{m_2 + 1}, \ldots, x_{n_2 - 1}, x_{n_2}; x_{m_3}, x_{m_3 + 1}\ldots), \\
				HTx &= ((Tx)_{m_1}, (Tx)_{m_1 + 1}, \ldots, (Tx)_{n_1 - 1}, (Tx)_{n_1}; \\
				&(Tx)_{m_2}, (Tx)_{m_2 + 1}, \ldots, (Tx)_{n_2 - 1}, (Tx)_{n_2}; \ldots) = \\
				&= (x_{m_1 + 1}, x_{m_1 + 2}, \ldots, x_{n_1 - 1}, x_{n_1}, x_{n_1 + 1}; x_{m_2 + 1},
				\ldots, x_{n_2 - 1}, x_{n_2}, x_{n_2 + 1}; x_{m_3 + 1} \ldots),
			\end{align*}
			где $x \in \ell_\infty$. Отсюда
			\begin{align*}
				&(HT - TH)x = HTx - THx = \\
				&= (\underbrace{0, 0, \ldots, 0, 0}_{n_1 - m_1}, x_{n_1 + 1} - x_{m_2}, \underbrace{0, \ldots,
					0, 0}_{m_2 - n_2}, x_{n_2 + 1} - x_{m_3}, 0, \ldots) \in \mathcal I(ac_0) \subset ac_0,
			\end{align*}
			т. к. $\lim_{k \to \infty} (n_k - m_k) = \infty$.
		\end{enumerate}

		Таким образом, все условия леммы \ref{lem:suff_B_reg} выполнены, и оператор $H$ действительно является $B$-регулярным.

\end{frame}

\begin{frame}\frametitle{{Классы операторов}}
	$H:\ell_\infty \to \ell_\infty$
	\begin{itemize}
		\item
			полуэберлейновы: такие, что $B_1 H \in\B$ для некоторого $B_1\in \B$;
		\item
			эберлейновы: такие, что $B_1 H = B_1$ для некоторого $B_1\in \B$;
		\item
			В-регулярные: такие, что $B_1 H \in \B$ для любого   $B_1\in \B$;
		\item
			Q-регулярные: такие, что $q(Hx) = q(x)$ для любого   $x\in \ell_\infty$.
	\end{itemize}
	\begin{ttheorem}
		Каждый следующий из этих классов вложен в предыдущий и не совпадает с ним.
	\end{ttheorem}
\end{frame}


\begin{frame}\frametitle{{Полуэберлейнов оператор, не являющийся эберлейновым}}
	Пусть $B_1, B_2 \in \ext \B$.
	Положим
	\begin{equation}
		\label{eq:am_not_eber_def}
		B_3 = B_1 + 2(B_2-B_1) = 2B_2-B_1,
	\end{equation}
	тогда $B_3 \notin \B$.
	Положим $Hx = (B_3 x) \cdot\one$, тогда
	\begin{equation}
		2 B_1 H x = B_1 x + B_1 ((B_3 x) \cdot\one) = B_1 x + B_3 x =
		B_1 x + 2 B_2 x - B_1 x = 2B_2 x
		,
	\end{equation}
	т.е. $B_1 H = B_2 \in \B$ и оператор $H$ полуэберлейнов.
	Пусть $B = BH$ для некоторого $B\in\B$.
	Тогда для всех $x\in\ell_\infty$ имеем
	\begin{equation}
		2Bx = B (x + (B_3 x) \cdot \one)
		\Rightarrow
		Bx =  B((B_3 x) \cdot \one)
		\Rightarrow
		Bx = B_3x
		.
	\end{equation}
\end{frame}

\begin{frame}\frametitle{{Эберлейнов оператор, не являющийся В-регулярным}}
	\vspace{-3em}
	\begin{multline}
		\label{eq:oper_A_throws_out_2power_blocks}
		Ax = (x_1, x_2, \not x_3, \not x_4, x_5, x_6, x_7, x_8, \not x_9, ..., \not x_{16}, x_{17}, x_{18}, ..., x_{31}, x_{32}, \not x_{33}, \not x_{34}, ..., \not x_{64},
		\\
		x_{65}, x_{66}, ..., x_{128}, \not x_{129}, ...)=
		\\=
		(x_1, x_2, \ x_5, x_6, x_7, x_8, \ x_{17}, x_{18}, ..., x_{31}, x_{32}, \ x_{65}, x_{66}, ..., x_{128}, \ x_{257},
		\\
		..., \ x_{2^{2n} +1}, x_{2^{2n} +2},  x_{2^{2n+1}}, \ \ x_{2^{2(n+1)} +1},  x_{2^{2(n+1)} +2},  x_{2^{2(n+1)+1}}, ...)
	\end{multline}
	Оператор $A$ является В-регулярным, разреженным, но не Q-регулярным.
	\vfill

	\begin{equation}
		Ex = x \cdot \chi_{\cup_{m=0}^{\infty}\left[2^{2 m}+1, 2^{2 m+1}\right] \cup\{1\}}
		.
	\end{equation}
	\vfill
	Очевидно, $AE=A$.
	Поскольку $A$ есть В-регулярный оператор, то он эберлейнов,
	и $\B(A)\ne\varnothing$.
	Пусть $B\in\B(A)$. Тогда
	\begin{equation}
		B = BA = B(AE) = (BA)E = BE
		,
	\end{equation}
	то есть $B\in\B(E)$.
	Но оператор $E$ не является В-регулярным.
	Действительно, $q(E\one) =q(A\one) =0$, т.е. $ E\one \notin ac_1$,
	и не выполнено условие (i) критерия В-регулярности.
	\vfill
\end{frame}


\begin{frame}\frametitle{Поиск <<самого хорошего>> банахова предела}
	\begin{equation}
		\sigma_n (x_1, x_2, x_3, ...) = (
		\underbrace{x_1,...,x_1}_{n~\text{раз}},
		\underbrace{x_2,...,x_2}_{n~\text{раз}},
		\underbrace{x_3,...,x_3}_{n~\text{раз}},
		...)
	\end{equation}

	\begin{equation}
		C (x_1, x_2, x_3, ...) = \left(
		x_1,
		\dfrac{x_1+x_2}2,
		\dfrac{x_1+x_2 + x_3}3,
		\dfrac{x_1+x_2+x_3+x_4}4,
		...,
		\dfrac{x_1+...+x_n}n,
		...\right)
		.
	\end{equation}

	\begin{equation}
		\mathfrak{B}(C) \subsetneq \bigcap_{n=2}^\infty \mathfrak{B}(\sigma_n)
		\qquad
		\qquad
		\diam\mathfrak{B}(C) = \operatorname{rad} \mathfrak{B}(C) = 2
	\end{equation}
\end{frame}


\begin{frame}\frametitle{{Обратная задача об инвариантности}}
	Для всякого ли банахова предела существует нетривиальный оператор, относительно которого он инвариантен?
	\vfill
	\begin{ddefinition}
		Оператор $G_B x = (Bx, Bx, Bx, ...) = (Bx)\cdot\one$
		назовём порождённым банаховым пределом $B$.
	\end{ddefinition}
	\vfill
	\begin{ttheorem}
		$\B(G_B) = \{B\}$ и $G_B^* \B = \{B\}$.
	\end{ttheorem}
	\vfill
	\begin{llemma}
		Пусть $0 \leq \lambda \leq 1$, $B_1, B_2 \in \B$.
		Тогда
		\begin{equation}
			G_{\lambda B_1+(1-\lambda) B_2} =\lambda G_{B_1} + (1-\lambda)G_{B_2}
			,
		\end{equation}
		\begin{equation}
			G_{B_1} G_{B_2} = G_{B_2}
			.
		\end{equation}
	\end{llemma}
\end{frame}


\begin{frame}\frametitle{{Мультипорождённые операторы}}
	\begin{ddefinition}
		Линейный непрерывный оператор $G_{\{B_k\}}$, определённый равенством
		\begin{equation}
			(G_{\{B_k\}}x)_n = B_n x
			,
		\end{equation}
		будем называть мультипорождённым последовательностью банаховых пределов $\{B_k\}\subset \B$.
	\end{ddefinition}
	\vfill

	\begin{llemma}
		Всякий мультипорождённый оператор $G_{\{B_k\}}$ является В-регулярным оператором.
	\end{llemma}
	\vfill

	\begin{llemma}
		Пусть множество значений, которые принимают элементы последовательности ${\{B_k\}}$, конечно.
		Тогда оператор $G_{\{B_k\}}$ компактен.
	\end{llemma}
	\vfill

	\begin{ttheorem}
		Мультипорождённый оператор не может иметь матричного представления.
	\end{ttheorem}

\end{frame}


\begin{frame}\frametitle{{Мультипорождённые операторы и инвариантность}}
	\begin{multline}
		Ax = (x_1, x_2, \not x_3, \not x_4, x_5, x_6, x_7, x_8, \not x_9, ..., \not x_{16}, x_{17}, x_{18}, ..., x_{31}, x_{32}, \not x_{33}, \not x_{34}, ..., \not x_{64},
		\\
		x_{65}, x_{66}, ..., x_{128}, \not x_{129}, ...)=
		\\=
		(x_1, x_2, \ x_5, x_6, x_7, x_8, \ x_{17}, x_{18}, ..., x_{31}, x_{32}, \ x_{65}, x_{66}, ..., x_{128}, \ x_{257},
		\\
		..., \ x_{2^{2n} +1}, x_{2^{2n} +2},  x_{2^{2n+1}}, \ \ x_{2^{2(n+1)} +1},  x_{2^{2(n+1)} +2},  x_{2^{2(n+1)+1}}, ...)
	\end{multline}
	Операторы $A$ и $A\sigma_2$ В-регулярны; $\B(A) \ne\varnothing$, $\B(A\sigma_2) \ne\varnothing$,
	$\B(A) \cap \B(A\sigma_2) =\varnothing$.

	\vfill

	Выберем $B_1 \in \B (A)$ и $B_2 \in \B(A\sigma_2)$ и определим оператор $H$ соотношением
	\begin{equation}
		Hx = (B_1 x, B_1 x, \ B_2 x, B_2 x, \ B_1 x, B_1 x, B_1 x, B_1 x, \ \underbrace{B_2 x, ..., B_2 x}_{8~\mbox{раз}}, ...)
		.
	\end{equation}

	Тогда $[B_1; B_2]\subset \B(H)$.
	Оператор $(\sigma_2 H)^*$ переводит отрезок $[B_1; B_2]$ в $[B_2; B_1]$,
	т.е. <<меняет местами>> два банаховых предела,
	и $\dfrac{B_1 + B_2}2 \in \B(\sigma_2 H)$.
\end{frame}












\setbeamertemplate{bibliography item}{\insertbiblabel\small}

\begin{frame}
	\printbibliography{}
\end{frame}

\setbeamertemplate{navigation symbols}{}

\begin{frame}
	{
		\huge\centering
		~\\~\\~\\~\\
		Спасибо за внимание
	}
	~\\
	\vspace{6.28em}
	nickkolok@mail.ru, avdeev@math.vsu.ru
	\\
	github.com/nickkolok
	\\
	arxiv.org/a/avdeev\_n\_1.html
\end{frame}

\end{document}
