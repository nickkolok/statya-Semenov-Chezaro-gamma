\documentclass[10pt,pdf,hyperref={unicode},aspectratio=169,color={usenames, dvipsnames}]{beamer}
\usepackage{amsmath}
\usepackage[utf8]{inputenc}
\usepackage[english,russian]{babel}
\usepackage{amsfonts}
\usepackage{amsfonts,amssymb}
\usepackage{amssymb}
\usepackage{latexsym}
\usepackage{euscript}
\usepackage{enumerate}
\usepackage{graphics}
\usepackage{graphicx}
\usepackage{geometry}
\usepackage{wrapfig}

\usepackage{bbm}
\usepackage{mathrsfs}



\DeclareMathOperator{\ext}{ext}
\DeclareMathOperator{\mes}{mes}
\DeclareMathOperator{\supp}{supp}
\DeclareMathOperator{\conv}{conv}
\DeclareMathOperator{\diam}{diam}

\newcommand{\N}{\ensuremath{\mathbb{N}}}
\newcommand{\Q}{\ensuremath{\mathbb{Q}}}
\newcommand{\R}{\ensuremath{\mathbb{R}}}
\newcommand{\B}{\ensuremath{\mathfrak{B}}}
\newcommand{\Iac}{\mathcal{I}(ac_0)}
\newcommand{\Dac}{\mathcal{D}(ac_0)}
\newcommand{\one}{\ensuremath{\mathbbm 1}}


\newcommand{\longcomment}[1]{}


%Only referenced equations are numbered
\usepackage{mathtools}
\mathtoolsset{showonlyrefs}

%\mathtoolsset{showonlyrefs=false}
% (an equation/multline to be force-numbered)
%\mathtoolsset{showonlyrefs=true}


\usepackage{varwidth}

\usepackage{amsthm}
\theoremstyle{definition}
\newtheorem{llemma}{Лемма}
\newtheorem{ttheorem}[llemma]{Теорема}
\newtheorem{eexample}[llemma]{Пример}
\newtheorem{property}[llemma]{Свойство}
\newtheorem{remark}[llemma]{Замечание}
\newtheorem{ccorollary}{Следствие}[llemma]
\newtheorem{hhypothesis}{Гипотеза}[llemma]
\newtheorem{ddefinition}{Определение}[llemma]


\usepackage[backend=biber,style=gost-numeric,sorting=none]{biblatex}
\addbibresource{../bib/Semenov.bib}
\addbibresource{../bib/my.bib}
\addbibresource{../bib/ext.bib}
\addbibresource{../bib/classic.bib}
\addbibresource{../bib/Damerau-Levenstein.bib}
\addbibresource{../bib/general_monographies.bib}
\addbibresource{../bib/Bibliography_from_Usachev.bib}


\righthyphenmin=2

%\usetheme{Antibes}
\usefonttheme{professionalfonts} % using non standard fonts for beamer
\usefonttheme[onlymath]{serif} % default family is serif
%\usepackage{fontspec}
%\setmainfont{Liberation Serif}
\begin{document}


\setbeamertemplate{navigation symbols}{\large{}}

\title{
	Банаховы пределы и некоторые классы линейных операторов
}
\author{докладчики: асп. Р.Е.~Зволинский, асп. Н.Н.~Авдеев\\науч. рук.: д.ф.-м.н., проф. Е.М. Семенов}
\institute{ Воронежский Государственный Университет \\
		{Работа выполнена при поддержке Фонда БАЗИС, проект No 22-7-2-27-3.
	}}
\date{2025}

\maketitle


\setbeamertemplate{footline}[frame number]
\setbeamertemplate{navigation symbols}{\large{}}


\begin{frame}\frametitle{Пространство $\ell_\infty$}
	$\ell_\infty$~--- пространство всех ограниченных последовательностей
	$x=(x_1, x_2, ..., x_n, ...)$
	с~нормой
	$$
		\|x\|_{\ell_\infty} = \sup_{k\in\mathbb{N}} |x_k|
	$$
	{Свойства:}

	\begin{itemize}
		\item
			$\ell_\infty$~--- линейное пространство над полем $\mathbb{R}$
		\item
			$\ell_\infty$  несепарабельно
		\item
			$\ell_1 \subset \ell_2 \subset \dots \subset \ell_\infty$
		\item
			$\ell_1^* = \ell_\infty$
		\item
			$\ell_\infty^* \neq \ell_1$
	\end{itemize}
\end{frame}


\begin{frame}\frametitle{Обобщения понятия сходимости в $\ell_\infty$}
	%\hspace{1.468em}
	\begin{varwidth}[t]{\linewidth}
		\centering
		Верхний и нижний
		\\
		пределы:
		\\
		$\displaystyle \limsup_{n\to\infty} x_n$
		,~
		$\displaystyle \liminf_{n\to\infty} x_n$
		\\~\\
		\emph{
			не характеризуют
			\\
			распределение элементов
		}
	\end{varwidth}
	\hfill
	\begin{varwidth}[t]{\linewidth}
		\centering
		Сходимость в среднем
		\\
		(по Чезаро):
		$\displaystyle \lim_{n\to\infty} (Cx)_n$
		\\
		$\displaystyle (Cx)_n = \frac1n\sum_{i=1}^n x_i$
		\\~\\
		\emph{слишком универсальна}
	\end{varwidth}
	\hfill
	\begin{varwidth}[t]{\linewidth}
		\centering
		Банаховы пределы $B\in \mathfrak{B}$:
		\\
		т. Хана--Банаха к $\lim: c\to\mathbb R$
		\\
			$B \geqslant 0$
		\\
			$B\mathbbm{1} = 1$
		\\
			$B=BT$
		\\
		\emph{почти сходимость}
	\end{varwidth}
	\\~\\~\\
	\begin{varwidth}[t]{\linewidth}
		\centering
		\emph{Критерий Лоренца:}
		$\displaystyle
			x\in ac_t
			\Leftrightarrow
			\forall(B\in\mathfrak{B})[Bx = t]
			\Leftrightarrow
			\lim_{n\to\infty}  \sum_{k=m+1}^{m+n} \frac{x_k}n = t
		$
		равном. по $m\in\mathbb{N}$.
	\end{varwidth}
	\\~\\
	\vspace{0.8em}
	\begin{varwidth}[t]{\linewidth}
		$\displaystyle ac = \mathop{\cup}\limits_{t\in\mathbb R} ac_t$ "--- пространство почти сходящихся последовательностей
	\end{varwidth}
\end{frame}


\begin{frame}\frametitle{Банаховы пределы}
	Банаховым пределом называется функционал $B\in \ell_\infty^*$ такой, что:
	\begin{enumerate}
		\item
			$B \geqslant 0$
		\item
			$B\mathbbm{1} = 1$
		\item
			$B=BT$
	\end{enumerate}
	Здесь $\mathbbm{1}=(1,1,1,1,1,...)$,
	$T$~--- оператор сдвига: $T(x_1, x_2, x_3, ...) = (x_2, x_3, ...)$.

	Простейшие свойства:
	\begin{itemize}
		\item
			$\|B\|_{\ell_\infty^*} = 1$
		\item
			$Bx = \lim_{n\to\infty} x_n$ для любого $x=(x_1, x_2, ...) \in c$,
			\\
			где $c$~--- множество сходящихся последовательностей.

			Таким образом,
			банахов предел~--- естественное обобщение понятия предела сходящейся последовательности
			на все ограниченные последовательности.
	\end{itemize}
\end{frame}

\longcomment{
\begin{frame}\frametitle{Свойства множества банаховых пределов $\mathfrak{B}$}
	\begin{enumerate}
		\item
			$\mathfrak{B}$~--- замкнутое выпуклое подмножество $ S_{\ell_\infty^*}$~---
			единичной сферы пространства $\ell_\infty^*$
		\item
			$d(\mathfrak{B}) = 2$
		\item
			$\mathfrak{B}$ слабо *-компактно
	\end{enumerate}
\end{frame}
}

\begin{frame}\frametitle{Теорема Лоренца~\cite{L}}
	Для заданного $r\in\mathbb{R}$ равенство $Bx=r$ выполнено для всех $B\in\mathfrak{B}$
	тогда и только тогда, когда
	\begin{equation*}
		\lim_{n\to\infty} \frac{1}{n} \sum_{k=m+1}^{m+n} x_k = r
	\end{equation*}
	сходится равномерно по всем $m\in\mathbb{N}$.

	Множество всех таких $x \in \ell_\infty$ обозначается $ac$.
\end{frame}

\begin{frame}\frametitle{Теорема Сачестона~\cite{S}}
	является уточнением теоремы Лоренца.
	Пусть
	\begin{equation*}
		q(x) = \lim_{n\to\infty} \inf_{m\in\mathbb{N}}  \frac{1}{n} \sum_{k=m+1}^{m+n} x_k,
		~~~~~~~~
		p(x) = \lim_{n\to\infty} \sup_{m\in\mathbb{N}}  \frac{1}{n} \sum_{k=m+1}^{m+n} x_k.
	\end{equation*}
	Тогда для любых $x\in \ell_\infty$ и $B\in\mathfrak{B}$
	\begin{equation}\label{Sucheston}
		q(x) \leqslant Bx \leqslant p(x)
	\end{equation}
	Неравенства (\ref{Sucheston}) точны:
	для данного $x$ для любого $r\in[q(x); p(x)]$ найдётся банахов предел
	$B\in\mathfrak{B}$ такой, что $Bx = r$.
\end{frame}

\begin{frame}\frametitle{Что такое почти сходимость?}
	\begin{varwidth}[t]{\linewidth}
		\hspace{3em}
		Пример $x\in ac_0$, но $\displaystyle0=\liminf_{n\to\infty}x_n < \limsup_{n\to\infty}x_n=1$:~~
		$$\displaystyle
			x_k=\begin{cases}
				1, & k=2^j,~j\in\mathbb N
				\\
				0 & \mbox{для прочих~} k
			\end{cases}
		$$
		$x=(0,1,0,1,\;0,0,0,1,\;\;0,0,0,0,\;0,0,0,1,\;\;0,0,0,0,\;0,0,0,0,\;\;0,0,0,0,\;0,0,0,1,\;\;0,0,0,0,\;...)$
	\end{varwidth}
	\\
	\vspace{2em}
	\begin{varwidth}[t]{\linewidth}
		Пример сходящейся по Чезаро к нулю последовательности $x\notin ac$:~~
		$$\displaystyle
			x_k=\begin{cases}
				1, & 2^j-j < k \leq 2^j,~j\in\mathbb N
				\\
				0 & \mbox{для прочих~} k
			\end{cases}
		$$
		$x=(0,1,1,1,\;0,1,1,1,\;\;0,0,0,0,\;1,1,1,1,$\\
		$\phantom{x=(}0,0,0,0,\;0,0,0,0,\;\;0,0,0,1,\;1,1,1,1,$\\
		$\phantom{x=(}0,0,0,0,\;0,0,0,0,\;\;0,0,0,0,\;0,0,0,0,$\\
		$\phantom{x=(}0,0,0,0,\;0,0,0,0,\;\;0,0,1,1,\;1,1,1,1,...)$
	\end{varwidth}
\end{frame}






\begin{frame}\frametitle{Инвариантные банаховы пределы }
	$$H:\ell_\infty \to \ell_\infty$$
	$$B = BH \quad \Leftrightarrow \quad B\in\B(H)$$
	\vspace{1em}
	\begin{ttheorem}[Признак Семенова--Сукочева,~\cite{Semenov2010invariant}]
		\label{thm:Semenov_Sukochev_conditions}
		Пусть линейный оператор $H:\ell_\infty\to\ell_\infty$ таков, что:
		\\(i)   $H \geqslant 0, H \one=\one$;
		\\(ii)  $H c_0 \subset c_0$;
		\\(iii) $\limsup _{j \rightarrow \infty}(A(I-T) x)_j \geqslant 0$ для всех $x \in \ell_{\infty}, A \in R$,
		\\где
		\begin{equation}
			R=R(H):=\operatorname{conv}\left\{H^n, n=1,2, \ldots\right\}
			.
		\end{equation}
		Тогда $\B(H) \ne\varnothing$.
	\end{ttheorem}
\end{frame}


\begin{frame}\frametitle{B-регулярные операторы}
	\begin{ddefinition}[\cite{alekhno2018invariant}]
		\label{def:B-regular_operator}
		Оператор $H : \ell_\infty \to \ell_\infty$ называется \emph{B-регулярным},
		если $H^*\B \subseteq \B$.
		(Или, что то же самое, $BH\in\B$ для любого $B\in\B$.)
	\end{ddefinition}
	\begin{ttheorem}[\cite{alekhno2018invariant}]
		\label{thm:B-regular_is_Eberlein}
		Любой B-регулярный оператор "--- эберлейнов.
	\end{ttheorem}
\end{frame}


\begin{frame}\frametitle{Введём новые классы операторов}
	\begin{ddefinition}
		Будем называть оператор $H:\ell_\infty \to \ell_\infty$ \emph{полуэберлейновым}, если $BH\in\mathfrak B$ для некоторого $B\in\mathfrak B$.
	\end{ddefinition}

	\begin{ddefinition}
		Будем называть оператор $H:\ell_\infty \to \ell_\infty$ \emph{существенно эберлейновым}, если $BH\in\mathfrak B$ для любого $B\in\mathfrak B$ и $BH\ne B$ для некоторого $B\in\mathfrak B$.
	\end{ddefinition}
\end{frame}

\begin{frame}\frametitle{Классы операторов}
	$H:\ell_\infty \to \ell_\infty$
	\begin{itemize}
		\item
			полуэберлейновы: такие, что $B_1 H \in\B$ для некоторого $B_1\in \B$;
		\item
			эберлейновы: такие, что $B_1 H = B_1$ для некоторого $B_1\in \B$;
		\item
			В-регулярные: такие, что $B_1 H \in \B$ для любого   $B_1\in \B$;
		\item
			существенно эберлейновы: такие, что $B_1 H \in \B$ для любого $B_1\in \B$ и $B_2 H \ne B_2$ для некоторого $B_2\in \B$.
	\end{itemize}
	Каждый следующий из этих классов вложен в предыдущий и не совпадает с ним.
\end{frame}


\begin{frame}\frametitle{Полуэберлейнов оператор, не являющийся эберлейновым}
	Пусть $B_1, B_2 \in \ext \B$.
	Положим
	\begin{equation}
		\label{eq:am_not_eber_def}
		B_3 = B_1 + 2(B_2-B_1) = 2B_2-B_1,
	\end{equation}
	тогда $B_3 \notin \B$.
	\begin{equation}
		2 B_1 H x = B_1 x + B_1 ((B_3 x) \cdot\one) = B_1 x + B_3 x =
		B_1 x + 2 B_2 x - B_1 x = 2B_2 x
		,
	\end{equation}
	Пусть $B = BH$ для некоторого $B\in\B$.
	Тогда для всех $x\in\ell_\infty$ имеем
	\begin{equation}
		2Bx = B (x + (B_3 x) \cdot \one)
		\Rightarrow
		Bx =  B((B_3 x) \cdot \one)
		\Rightarrow
		Bx = B_3x
		.
	\end{equation}
\end{frame}

\begin{frame}\frametitle{Эберлейнов оператор, не являющийся В-регулярным}
	\begin{multline}
		\label{eq:oper_A_throws_out_2power_blocks}
		Ax = (x_1, x_2, \not x_3, \not x_4, x_5, x_6, x_7, x_8, \not x_9, ..., \not x_{16}, x_{17}, x_{18}, ..., x_{31}, x_{32}, \not x_{33}, \not x_{34}, ..., \not x_{64},
		\\
		x_{65}, x_{66}, ..., x_{128}, \not x_{129}, ...)=
		\\=
		(x_1, x_2, \ x_5, x_6, x_7, x_8, \ x_{17}, x_{18}, ..., x_{31}, x_{32}, \ x_{65}, x_{66}, ..., x_{128}, \ x_{257},
		\\
		..., \ x_{2^{2n} +1}, x_{2^{2n} +2},  x_{2^{2n+1}}, \ \ x_{2^{2(n+1)} +1},  x_{2^{2(n+1)} +2},  x_{2^{2(n+1)+1}}, ...)
	\end{multline}
	Оператор $A$ является В-регулярным.
	Пусть оператор $E:\ell_\infty\to\ell_\infty$ определён формулой
	\begin{equation}
		Ex = x \cdot \chi_{\cup_{m=0}^{\infty}\left[2^{2 m}+1, 2^{2 m+1}\right] \cup\{1\}}
		.
	\end{equation}
	Тогда, очевидно, $AE=A$.
	Поскольку $A$ есть В-регулярный оператор, то он эберлейнов,
	т.е. множество $\B(A)$ непусто.
	Пусть $B\in\B(A)$. Тогда
	\begin{equation}
		B = BA = B(AE) = (BA)E = BE
		,
	\end{equation}
	то есть $B\in\B(E)$.
	Но оператор $E$ не является В-регулярным.
	Действительно, $q(E\one) =0$, т.е. $ E\one \notin ac_1$,
	и не выполнено условие (i) критерия В-регулярности.
\end{frame}




\setbeamertemplate{bibliography item}{\insertbiblabel\small}

\begin{frame}
	\printbibliography{}
\end{frame}

\setbeamertemplate{navigation symbols}{}

\begin{frame}
	{
		\huge\centering
		~\\~\\~\\~\\
		Спасибо за внимание
	}
	~\\
	\vspace{6.28em}
	nickkolok@mail.ru, avdeev@math.vsu.ru
	\\
	github.com/nickkolok
	\\
	arxiv.org/a/avdeev\_n\_1.html
\end{frame}

\end{document}
