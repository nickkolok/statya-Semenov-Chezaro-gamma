\begin{frame}\frametitle{\underline{Срезочный критерий}}
	Определим (нелинейный) оператор $\lambda$--срезки $A_\lambda$ на пространстве $\ell_\infty$.
	Для $x = (x_1, x_2, ...) \in \ell_\infty$ положим
	\begin{equation}
		(A_\lambda x)_k = \begin{cases}
			1, & \mbox{~если~} x_k \geq \lambda
			\\
			0  & \mbox{~иначе.~}
		\end{cases}
	\end{equation}

	\vfill

	\begin{ttheorem}
		Пусть $x\in\ell_\infty$, $x\geq 0$.
		Тогда
		\begin{equation}
			x\in ac_0 \Leftrightarrow
			\forall(\lambda>0)[A_\lambda x \in ac_0]
			.
		\end{equation}
	\end{ttheorem}
\end{frame}




\begin{frame}\frametitle{\underline{Интервальный критерий}}
	%\begin{theorem}
		\label{thm:M_j_ac0_inf_lim}
		Пусть $n_i$~--- строго возрастающая последовательность натуральных чисел,
		\begin{equation}
			\label{eq:definition_M_j}
			M(j) = \liminf_{i\to\infty} n_{i+j} - n_i,
		\end{equation}
		\begin{equation}
			x_k = \left\{\begin{array}{ll}
				1, & \mbox{~если~} k = n_i
				\\
				0  & \mbox{~иначе~}
			\end{array}\right.
		\end{equation}
		Тогда следующие условия эквивалентны:
		\\~\\
		(i)   $x \in ac_0$;
		\\~\\
		(ii)  $\lim\limits_{j \to \infty} \dfrac{M(j)}{j} = +\infty$;
		\\~\\
		(iii) $\inf\limits_{j \in \N}     \dfrac{M(j)}{j} = +\infty$.
	%\end{theorem}
\end{frame}


\begin{frame}\frametitle{\underline{Срезочно-интервальный критерий}}
	\begin{ttheorem}
		%\label{thm:crit_ac0_Mj_lambda}
		Пусть $x\in\ell_\infty$, $x \geq 0$, $\lambda>0$.
		Обозначим через $n^{(\lambda)}_i$ возрастающую последовательность
		индексов таких элементов $x$, что $x_k \geq \lambda$ тогда и только тогда,
		когда $k=n^{(\lambda)}_i$ для некоторого $i$.
		Обозначим
		\begin{equation}
			M^{(\lambda)}(j) = \liminf_{i\to\infty} n^{(\lambda)}_{i+j} - n^{(\lambda)}_i
			.
		\end{equation}


		Тогда для того, чтобы $x\in ac_0$, необходимо и достаточно, чтобы
		для любого $\lambda>0$ было выполнено
		\begin{equation}
			\lim_{j \to \infty} \frac{M^{(\lambda)}(j)}{j} = +\infty
			.
		\end{equation}
	\end{ttheorem}
\end{frame}


\begin{frame}\frametitle{\underline{Усиленная теорема Коннора}}
	Пусть на множестве $\Omega=\{0,1\}^\N$ задана вероятностная мера <<честной монетки>> $\mu$.
	Коннор доказал~\cite{connor1990almost}, что $\mu(\Omega\cap ac)=0$.

	Обобщим этот результат.

	\begin{ttheorem}
	%\label{thm:Connor_generalized}
	Мера множества $F=\{x\in\Omega : q(x) = 0 \wedge p(x)= 1\}$,
	где $p(x)$ и $q(x)$~--- верхний и нижний функционалы Сачестона соответственно,
	равна 1.
	\end{ttheorem}
\end{frame}


\begin{frame}\frametitle{Немультипликативность банаховых пределов}
	Хорошо известно,
	что банаховы пределы не обладают свойством мультипликативности
	\\
	(в отличие от обычного предела).

	Предположим, что некоторый $B\in \B$ мультипликативен,
	то есть для любых $x,y\in B$ выполнено
	\begin{equation}
		B(x\cdot y) = (Bx)\cdot (By)
		.
	\end{equation}

	Пусть $x_n = (-1)^n$, $y=x$.
	Тогда в силу критерия Лоренца $Bx = 0$
	(банахов предел на периодической последовательности равен среднему по периоду),
	но
	\begin{equation}
		B(x\cdot x) = B\one = 1 \ne 0 = (Bx) \cdot (Bx)
		.
	\end{equation}


\end{frame}


\begin{frame}\frametitle{Множество $W$}


	Пусть $W$~--- множество всех последовательностей $\chi e$, где $e =\bigcup_{k=1}^{\infty} [n_{2k-1}, n_{2k} )$
	и $\{n_k \}_{k=1}^{\infty}$
	удовлетворяет условию
	\begin{equation}
		\label{eq:lim_j_n_kj_measure}
		\lim_{j\to\infty}\frac{n_{k+j} - n_k}{j} = \infty
	\end{equation}
	равномерно по $k \in \N$.

	\vfill

	Множество $W$ возникает естественным образом при изучении крайних точек множества банаховых пределов.

	\begin{llemma}[{\cite[Лемма 30]{Semenov2014geomprops}}]
		Для любого $e\in\N$ такого, что $\chi e \in W$,
		любого $x\in\ell_\infty$
		и любого $B\in\ext\B$
		справедливо равенство
		\begin{equation}
			B(x \cdot \chi e) = (Bx) \cdot (B\chi e)
			.
		\end{equation}
	\end{llemma}
\end{frame}




\begin{frame}\frametitle{\underline{Множество $W$}}
	Введём нелинейную биекцию $Q:\Omega\leftrightarrow\Omega$ по следующему правилу:
	\begin{equation}
		(Qx)_k = \begin{cases}
			x_k, &\mbox{если~} k = 1,
			\\
			|x_k-x_{k-1}|&\mbox{иначе}.
		\end{cases}
	\end{equation}

	\begin{llemma}
		Биекция $Q$ сохраняет меру множества.
	\end{llemma}

	\begin{llemma}
		Имеет место быть включение $QW\subsetneq \Omega\cap ac_0$ и,
		более того,
		$Q^{-2} ac_0 \not \subset ac_0$.
	\end{llemma}

	Доказательство: интервальный критерий почти сходимости;
	конструктивные контрпримеры.


	\begin{ttheorem}
		Мера множества $W$ равна нулю.
	\end{ttheorem}


\end{frame}





\begin{frame}\frametitle{Возведение в степень (покоординатно)}
	Покоординатное возведение в степень выводит даже из $c_0$:
	\vfill
	\begin{equation}
		x_n = \left(\frac12\right)^{2^n} \to 0, \qquad
		y_n = \frac1{2^n} \to 0,
	\end{equation}
	\vfill
	\begin{equation}
		(x_n)^{y_n} = \left(\left(\frac12\right)^{2^n}\right)^{1/2^n}  = \frac12 \not \to 0
		.
	\end{equation}


\end{frame}



\begin{frame}\frametitle{Возведение в отрицательную степень}
	\begin{llemma}
		Пусть $x\in c_0$, $\lambda< 0$ и $x_k\ne 0 $ для любого $k$.
		Тогда
		$
			x^\lambda = (x_1^\lambda,x_2^\lambda,x_3^\lambda,...) \notin \ell_\infty
			.
		$
	\end{llemma}

	\begin{eexample}
		\begin{equation}
			x = (1;-1;1;-1;1;-1;...) \in ac_0 \setminus c_0
			.
		\end{equation}
		Для любой целой нечётной отрицательной степени $\lambda$:\quad $x^\lambda = x \in ac_0$
		\\
		Для любой целой чётной отрицательной степени $\lambda$:\quad\quad
		$
			x^\lambda  = \one \in ac_1 \subset ac \setminus ac_0
		$
	\end{eexample}

	\begin{eexample}
		\begin{equation}
			y = \left(1;-1;\frac12;-\frac12;1;-1;\frac13;-\frac13;1;-1;\frac14;-\frac14;1;-1;...\right) \in ac_0
			.
		\end{equation}
		Очевидно, что
		\begin{equation}
			y^{-1} = \left(1;-1;2;-2;1;-1;3;-3;1;-1;4;-4;1;-1;...\right)  \notin \ell_\infty
			.
		\end{equation}
	\end{eexample}

\end{frame}




\begin{frame}\frametitle{Возведение в положительную степень}
	Следующие результаты доказаны Р.Е. Зволинским~\cite{zvol2022ac}.

	\begin{ttheorem}
		\label{thm:Zvol_pow_pos}
		Пусть $x \geqslant 0, x \in a c_0$ и $\lambda>0$, тогда $x^\lambda \in a c_0$.
	\end{ttheorem}

	\begin{ttheorem}
		\label{thm:Zvol_pow_composed}
		Пусть $x\in \Iac$.
		Пусть $n\in\N$ или $n = \frac1{2k+1}$, $k\in\N$.
		Тогда $x^n \in ac_0$.
	\end{ttheorem}

	Примечание: $x\in \Iac$ тогда и только тогда, когда $x = x^+ +x^-$, $x^+\geq 0$, $x^- \leq 0$ и $x^+ \in ac_0$
	(последнее включение эквивалентно условию $x^- \in ac_0$).

\end{frame}



\begin{frame}\frametitle{\underline{Возведение в положительную степень}}
	Покажем, что условия предыдущей теоремы существенны.

	\begin{llemma}
		\label{thm:ac0_pow_even}
		Пусть $x\in ac_0 \setminus \Iac$, т.е. $x = x^+ +x^-$, $x^+\geq 0$, $x^- \leq 0$ и $x^+ \notin ac_0$
		(последнее условие эквивалентно условию $x^- \notin ac_0$).
		Пусть $n = 2k$, $k\in\N$.
		Тогда $x^n \notin ac_0$.
	\end{llemma}





	\begin{eexample}
		Для нечётной степени условие разложения в сумму двух знакопостоянных
		почти сходящихся последовательностей в теореме тоже существенно.
		Пусть
		\begin{equation}
			x = (1;1;-2;\ 1;1;-2;\ 1;1;-2;\ ...) \in ac_0
			,
			\quad
			\lambda = 3
			.
		\end{equation}
		Тогда
		\begin{equation}
			x^+ = (1;1;0;\ 1;1;0;\ 1;1;0;\ ...) \notin ac_0, \quad x^+ \in ac_{2/3}
			,
		\end{equation}
		\begin{equation}
			x^- = (0;0;-2;\ 0;0;-2;\ 0;0;-2;\ ...) \notin ac_0, \quad x^- \in ac_{-2/3}
			,
		\end{equation}
		\begin{equation}
			x^\lambda = (1;1;-8;\ 1;1;-8;\ 1;1;-8;\ ...) \notin ac_0, \quad x^\lambda \in ac_{-2}
			.
		\end{equation}
	\end{eexample}



\end{frame}



\begin{frame}\frametitle{\underline{Возведение в положительную степень}}

Возведение в нечётную степень может выводить не только из $ac_0$,
но и из $ac$.

\begin{eexample}
	\label{example:cube_out_of_ac0}
	%Напомним, что $\N_k = \{k, k+1, k+2, k+3,...\}$.
	Пусть
	$
		x_n = \begin{cases}
			 0, & \mbox{~если~} n < 2^{10},
			\\
			 1, & \mbox{~если~} n \ge 2^{10}, 2^k\le n < 2^k+3k \mbox{~и~}  n\neq 2^k + 3m, m\in\N,
			\\
			-2, & \mbox{~если~} n \ge 2^{10}, 2^k\le n < 2^k+3k \mbox{~и~}  n  =  2^k + 3m, m\in\N.
		\end{cases}
	$
	%Таким образом,
	\begin{multline}
		x = (0,0,...,0,0, \; -2, 1, 1, \; -2, 1, 1, \; -2, 1, 1, ..., -2, 1, 1, \; 0, 0, 0, \\ ..., 0, 0, 0, ..., -2, 1, 1, \; -2, 1, 1, ... ) \in ac_0
		.
	\end{multline}
	С другой стороны,
		\begin{equation}
		(x^3)_n = \begin{cases}
			 0, & \mbox{~если~} n < 2^{10},
			\\
			 1, & \mbox{~если~} n \ge 2^{10}, 2^k\le n < 2^k+3k \mbox{~и~}  n\neq 2^k + 3m, m\in\N,
			\\
			-8, & \mbox{~если~} n \ge 2^{10}, 2^k\le n < 2^k+3k \mbox{~и~}  n  =  2^k + 3m, m\in\N.
		\end{cases}
	\end{equation}
	Получаем $p(x^3) = 0$, $q(x^3) = -2$, откуда $x^3 \notin ac$.
\end{eexample}

\end{frame}



\begin{frame}\frametitle{\underline{Возведение в положительную степень}}

	\begin{eexample}
		%Напомним, что $\N_k = \{k, k+1, k+2, k+3,...\}$.
		Пусть
		\begin{equation}
			y_n = \begin{cases}
				 0, & \mbox{~если~} n < 2^{10},
				\\
				 1, & \mbox{~если~} n \ge 2^{10}, 2^k\le n < 2^k+3k \mbox{~и~}  n\neq 2^k + 3m, m\in\N,
				\\
				-\sqrt[3]{2}, & \mbox{~если~} n \ge 2^{10}, 2^k\le n < 2^k+3k \mbox{~и~}  n  =  2^k + 3m, m\in\N.
			\end{cases}
		\end{equation}
		Тогда $p(y) = \frac{2-\sqrt[3]2}{3}$, $q(y) = 0$, и потому $y \notin ac_0$.
		Однако легко заметить, что $y^3 = x$ из примера выше и потому $y^3\in ac_0$.
	\end{eexample}


\end{frame}


