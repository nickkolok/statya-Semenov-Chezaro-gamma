\begin{frame}\frametitle{Инвариантные банаховы пределы}

	\vfill
	Для любого банахова предела $B$ выполнено $B=BT$.
	\vfill
	Относительно каких ещё операторов банаховы пределы инвариантны?
	\vfill
	Даже для естественного оператора растяжения $\sigma_n$, $n\in\N_2$,
	инвариантны не все банаховы пределы.
	\vfill
	Банаховых пределов, инвариантных относительно оператора Чезаро $C$,
	ещё <<меньше>>:
	\begin{equation}
		B = BC \ \Rightarrow \ B = B\sigma_n \quad \mbox{для любого} \quad n \in \N
	\end{equation}
	\vfill


\end{frame}







\begin{frame}\frametitle{Инвариантные банаховы пределы }
	\begin{ddefinition}
		Банахов предел $B$ называется инвариантным относительно оператора $H:\ell_\infty \to \ell_\infty$,
		если $Bx = BHx$ для любого $x\in\ell_\infty$. Обозначение: $B\in\B(H)$.
	\end{ddefinition}
	\vfill
	\begin{ttheorem}[Признак Семенова--Сукочева,~\cite{Semenov2010invariant}]
		\label{thm:Semenov_Sukochev_conditions}
		Пусть линейный оператор $H:\ell_\infty\to\ell_\infty$ таков, что:
		\\(i)   $H \geqslant 0, H \one=\one$;
		\\(ii)  $H c_0 \subset c_0$;
		\\(iii) $\limsup _{j \rightarrow \infty}(A(I-T) x)_j \geqslant 0$ для всех $x \in \ell_{\infty}, A \in R$,
		\\где
		\begin{equation}
			R=R(H):=\operatorname{conv}\left\{H^n, n=1,2, \ldots\right\}
			.
		\end{equation}
		Тогда $\B(H) \ne\varnothing$.
	\end{ttheorem}
	\vfill
	Эти достаточные условия очень сильны и избыточны.
	\vfill
	Им удовлетворяют, например, операторы растяжения $\sigma_n$ и Чезаро $C$.
\end{frame}


\begin{frame}\frametitle{\underline{Конечномерность ядра}}
	Операторы растяжения $\sigma_n$ и Чезаро $C$ имеют конечномерное (даже нульмерное) ядро.
	\vfill
	Важно ли это? Нет.
	\begin{eexample}
		Пусть для $x = (x_1, x_2, ..., x_n, ...)\in \ell_\infty$
		\begin{equation*}
			Ax = (x_1, 0, x_2, 0, x_3, 0, x_4, 0, ...).
		\end{equation*}
		Очевидно, что $\ker A = \{0\}$.
		Но $\B(A)=\varnothing$.
	\end{eexample}
	\vfill

	\begin{eexample}
		\begin{equation}
			(Qx)_k =
			\begin{cases}
				0,~\mbox{если}~ k = 2^n, n \in\N,
				\\
				x_k~\mbox{иначе.}
			\end{cases}
		\end{equation}
		Очевидно, что $\dim \ker Q = \infty$.
		Однако $Q-I : \ell_\infty \to ac_0$ и потому $\B(Q) = \B$.
	\end{eexample}

\end{frame}


\begin{frame}\frametitle{B-регулярные операторы}
	\begin{ddefinition}
		Оператор $H : \ell_\infty \to \ell_\infty$ называется \emph{эберлейновым},
		если $\B(H) \ne\varnothing$.
	\end{ddefinition}
	\begin{ddefinition}[\cite{alekhno2018invariant}]
		\label{def:B-regular_operator}
		Оператор $H : \ell_\infty \to \ell_\infty$ называется \emph{B-регулярным},
		если $H^*\B \subseteq \B$.
		\\
		(Или, что то же самое, $BH\in\B$ для любого $B\in\B$.)
	\end{ddefinition}
	\begin{ttheorem}[\cite{alekhno2018invariant}]
		\label{thm:B-regular_is_Eberlein}
		Любой B-регулярный оператор "--- эберлейнов.
	\end{ttheorem}
	\begin{ttheorem}[\cite{alekhno2018invariant}]
		Оператор $H:\ell_\infty \to \ell_\infty$ является B-регулярным тогда и только тогда, когда:
		\\(i) $H\one \in ac_1$;
		\\(ii) $q(Hx)\geq 0$ для любого $x\geq 0$;
		\\(iii) $H ac_0 \subseteq ac_0$.
	\end{ttheorem}
\end{frame}


\begin{frame}\frametitle{\underline{Новые классы операторов}}
	\vfill
	\begin{ddefinition}
		Будем называть оператор $H:\ell_\infty \to \ell_\infty$ \emph{полуэберлейновым}, если $BH\in\mathfrak B$ для некоторого $B\in\mathfrak B$.
	\end{ddefinition}
	\vfill
	\begin{ddefinition}
		Будем называть оператор $H:\ell_\infty \to \ell_\infty$ \emph{существенно эберлейновым}, если $BH\in\mathfrak B$ для любого $B\in\mathfrak B$ и $BH\ne B$ для некоторого $B\in\mathfrak B$.
	\end{ddefinition}
	\vfill
\end{frame}

\begin{frame}\frametitle{\underline{Классы операторов}}
	$H:\ell_\infty \to \ell_\infty$
	\begin{itemize}
		\item
			полуэберлейновы: такие, что $B_1 H \in\B$ для некоторого $B_1\in \B$;
		\item
			эберлейновы: такие, что $B_1 H = B_1$ для некоторого $B_1\in \B$;
		\item
			В-регулярные: такие, что $B_1 H \in \B$ для любого   $B_1\in \B$;
		\item
			существенно эберлейновы: такие, что $B_1 H \in \B$ для любого $B_1\in \B$ и $B_2 H \ne B_2$ для некоторого $B_2\in \B$.
	\end{itemize}
	\begin{ttheorem}
		Каждый следующий из этих классов вложен в предыдущий и не совпадает с ним.
	\end{ttheorem}
\end{frame}


\begin{frame}\frametitle{\underline{Полуэберлейнов оператор, не являющийся эберлейновым}}
	Пусть $B_1, B_2 \in \ext \B$.
	Положим
	\begin{equation}
		\label{eq:am_not_eber_def}
		B_3 = B_1 + 2(B_2-B_1) = 2B_2-B_1,
	\end{equation}
	тогда $B_3 \notin \B$.
	Положим $Hx = (B_3 x) \cdot\one$, тогда
	\begin{equation}
		2 B_1 H x = B_1 x + B_1 ((B_3 x) \cdot\one) = B_1 x + B_3 x =
		B_1 x + 2 B_2 x - B_1 x = 2B_2 x
		,
	\end{equation}
	т.е. $B_1 H = B_2 \in \B$ и оператор $H$ полуэберлейнов.
	Пусть $B = BH$ для некоторого $B\in\B$.
	Тогда для всех $x\in\ell_\infty$ имеем
	\begin{equation}
		2Bx = B (x + (B_3 x) \cdot \one)
		\Rightarrow
		Bx =  B((B_3 x) \cdot \one)
		\Rightarrow
		Bx = B_3x
		.
	\end{equation}
\end{frame}

\begin{frame}\frametitle{\underline{Эберлейнов оператор, не являющийся В-регулярным}}
	\begin{multline}
		\label{eq:oper_A_throws_out_2power_blocks}
		Ax = (x_1, x_2, \not x_3, \not x_4, x_5, x_6, x_7, x_8, \not x_9, ..., \not x_{16}, x_{17}, x_{18}, ..., x_{31}, x_{32}, \not x_{33}, \not x_{34}, ..., \not x_{64},
		\\
		x_{65}, x_{66}, ..., x_{128}, \not x_{129}, ...)=
		\\=
		(x_1, x_2, \ x_5, x_6, x_7, x_8, \ x_{17}, x_{18}, ..., x_{31}, x_{32}, \ x_{65}, x_{66}, ..., x_{128}, \ x_{257},
		\\
		..., \ x_{2^{2n} +1}, x_{2^{2n} +2},  x_{2^{2n+1}}, \ \ x_{2^{2(n+1)} +1},  x_{2^{2(n+1)} +2},  x_{2^{2(n+1)+1}}, ...)
	\end{multline}
	Оператор $A$ является В-регулярным.
	Пусть оператор $E:\ell_\infty\to\ell_\infty$ определён формулой
	\begin{equation}
		Ex = x \cdot \chi_{\cup_{m=0}^{\infty}\left[2^{2 m}+1, 2^{2 m+1}\right] \cup\{1\}}
		.
	\end{equation}
	Тогда, очевидно, $AE=A$.
	Поскольку $A$ есть В-регулярный оператор, то он эберлейнов,
	т.е. множество $\B(A)$ непусто.
	Пусть $B\in\B(A)$. Тогда
	\begin{equation}
		B = BA = B(AE) = (BA)E = BE
		,
	\end{equation}
	то есть $B\in\B(E)$.
	Но оператор $E$ не является В-регулярным.
	Действительно, $q(E\one) =0$, т.е. $ E\one \notin ac_1$,
	и не выполнено условие (i) критерия В-регулярности.
\end{frame}


\begin{frame}\frametitle{\underline{Обратная задача об инвариантности}}
	Для всякого ли банахова предела существует нетривиальный оператор, относительно которого он инвариантен?
	\vfill
	\begin{ddefinition}
		Оператор $G_B x = (Bx, Bx, Bx, ...) = (Bx)\cdot\one$
		назовём порождённым банаховым пределом $B$.
	\end{ddefinition}
	\vfill
	\begin{ttheorem}
		$\B(G_B) = \{B\}$ и $G_B^* \B = \{B\}$.
	\end{ttheorem}
	\vfill
	\begin{llemma}
		Пусть $0 \leq \lambda \leq 1$, $B_1, B_2 \in \B$.
		Тогда
		\begin{equation}
			G_{\lambda B_1+(1-\lambda) B_2} =\lambda G_{B_1} + (1-\lambda)G_{B_2}
			,
		\end{equation}
		\begin{equation}
			G_{B_1} G_{B_2} = G_{B_2}
			.
		\end{equation}
	\end{llemma}
\end{frame}


\begin{frame}\frametitle{\underline{Мультипорождённые операторы}}
	\begin{ddefinition}
		Линейный непрерывный оператор $G_{\{B_k\}}$, определённый равенством
		\begin{equation}
			(G_{\{B_k\}}x)_n = B_n x
			,
		\end{equation}
		будем называть мультипорождённым последовательностью банаховых пределов $\{B_k\}\subset \B$.
	\end{ddefinition}
	\vfill

	\begin{llemma}
		Всякий мультипорождённый оператор $G_{\{B_k\}}$ является В-регулярным оператором.
	\end{llemma}
	\vfill

	\begin{llemma}
		Пусть множество значений, которые принимают элементы последовательности ${\{B_k\}}$, конечно.
		Тогда оператор $G_{\{B_k\}}$ компактен.
	\end{llemma}
	\vfill

	\begin{ttheorem}
		Мультипорождённый оператор не может иметь матричного представления.
	\end{ttheorem}

\end{frame}


\begin{frame}\frametitle{\underline{Мультипорождённые операторы и инвариантность}}
	\begin{multline}
		Ax = (x_1, x_2, \not x_3, \not x_4, x_5, x_6, x_7, x_8, \not x_9, ..., \not x_{16}, x_{17}, x_{18}, ..., x_{31}, x_{32}, \not x_{33}, \not x_{34}, ..., \not x_{64},
		\\
		x_{65}, x_{66}, ..., x_{128}, \not x_{129}, ...)=
		\\=
		(x_1, x_2, \ x_5, x_6, x_7, x_8, \ x_{17}, x_{18}, ..., x_{31}, x_{32}, \ x_{65}, x_{66}, ..., x_{128}, \ x_{257},
		\\
		..., \ x_{2^{2n} +1}, x_{2^{2n} +2},  x_{2^{2n+1}}, \ \ x_{2^{2(n+1)} +1},  x_{2^{2(n+1)} +2},  x_{2^{2(n+1)+1}}, ...)
	\end{multline}
	Операторы $A$ и $A\sigma_2$ В-регулярны; $\B(A) \ne\varnothing$, $\B(A\sigma_2) \ne\varnothing$,
	$\B(A) \cap \B(A\sigma_2) =\varnothing$.

	\vfill

	Выберем $B_1 \in \B (A)$ и $B_2 \in \B(A\sigma_2)$ и определим оператор $H$ соотношением
	\begin{equation}
		Hx = (B_1 x, B_1 x, \ B_2 x, B_2 x, \ B_1 x, B_1 x, B_1 x, B_1 x, \ \underbrace{B_2 x, ..., B_2 x}_{8~\mbox{раз}}, ...)
		.
	\end{equation}

	Тогда $[B_1; B_2]\subset \B(H)$.
	Оператор $(\sigma_2 H)^*$ переводит отрезок $[B_1; B_2]$ в $[B_2; B_1]$,
	т.е. <<меняет местами>> два банаховых предела,
	и $\dfrac{B_1 + B_2}2 \in \B(\sigma_2 H)$.
\end{frame}

