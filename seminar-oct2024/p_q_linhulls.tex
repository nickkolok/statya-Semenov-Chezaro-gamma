\begin{frame}\frametitle{Разделяющие множества}
	\label{page:p_q_linhulls}
	Множество $M\subset \ell_\infty$ называется разделяющим,
	если для любых двух различных банаховых пределов $B_1$ и $B_2$
	найдётся такой $x\in M$, что $B_1 x \ne B_2 x$.
	\\~\\
	Интересны <<маленькие>> разделяющие множества.
	\\~\\
	Например, $\Omega$ разделяющее~\cite{semenov2010characteristic}.
	\\~\\
	А можно ли ещё меньше?
	\vfill
	Счётных разделяющих множеств не бывает~\cite{Semenov2014geomprops}.
	\vfill
	\begin{llemma}[{\cite[{\S 3, замечание 6}]{Semenov2014geomprops}}]
		Пусть $X$~--- разделяющее множество и $X \subset \operatorname{Lin} Y$,
		где $\operatorname{Lin} Y$ обозначает линейную оболочку $Y$.
		Тогда $Y$ также является разделяющим множеством.
	\end{llemma}
\end{frame}


\begin{frame}\frametitle{\underline{Разделяющие множества}}

	\begin{ttheorem}
		Пусть
		$1 \geq a > b \geq 0$ и
		$\Omega^a_b = \{x\in\Omega : p(x) = a, q(x) = b\}$.
		\\
		Тогда $\Omega \subset \operatorname{Lin} \Omega^a_b$.
	\end{ttheorem}

	\vfill
	$\operatorname{Lin}$ ~--- линейная оболочка.
	\vfill
	\begin{ttheorem}
		Множество $\Omega^a_b$ является разделяющим.
		Т.к. при $a\neq 1$ или $b\neq 0$ множество $\Omega^a_b$ имеет меру нуль~\cite{semenov2010characteristic,connor1990almost},
		то оно является разделяющим множеством нулевой меры.
	\end{ttheorem}
\end{frame}


\begin{frame}\frametitle{\underline{Линейные оболочки}}

	Пусть
	\\
	$X^a_b = \{x\in\ell_\infty : p(x) = a,~ q(x) = b\}$,
	\\
	$Y^a_b\, = \{x\in A_0 : p(x) = a,~ q(x) = b\}$, где $a>b$.
	\vfill
	\begin{ttheorem}
		Справедливо равенство $\operatorname{Lin} Y^a_b = A_0$.
	\end{ttheorem}
	\vfill
	\begin{ttheorem}
		Справедливо равенство $\operatorname{Lin} X^a_b = \ell_\infty$.
	\end{ttheorem}

\end{frame}



\begin{frame}\frametitle{Размерность Хаусдорфа на $\Omega$}


	$s$-мерная мера Хаусдорфа непустого подмножества $F\subset \R^n$, $s > 0$ :
	$$\mathcal H^s(F) := \lim_{\delta\to0} \inf \left\{\sum_{i=1}^\infty \left({\rm diam} \ U_i \right)^s \ : \ F\subset \bigcup_{i=1}^\infty  U_i, \  0\leqslant {\rm diam} \ U_i \leqslant \delta \right\}$$

	\vfill

	Размерность Хаусдорфа множества  $F\subset \R^n$:
	$${\rm dim}_H F := \inf\{ s > 0 \ : \ \mathcal H^s(F)=0\}.$$

	\vfill

	\begin{ddefinition}
		Множество $E\subset\Omega$ называется самоподобным, если существуют $m\in\N_2$,
		$0< r_1, \dots, r_m<1$ и функции $f_j : \Omega \to \Omega$, $j=1,\dots, m$ такие, что
		$$
			\rho(f_j(x), f_j(y)) = r_j \rho(x,y), \ \forall \ x,y \in \Omega, \ j=1,\dots, m
			\mbox{~~и~~} E=\bigcup_{j=1}^m f_j(E).
		$$

	\end{ddefinition}
\end{frame}







\begin{frame}\frametitle{\underline{Размерность Хаусдорфа}}

	\begin{llemma}
		\label{lem:Hausdorf_measure}
		Пусть $E\subset\Omega$ и $TE = E$.
		Тогда размерность Хаусдорфа $E$ равна $1$.
	\end{llemma}

	\vfill

	Для разделяющего множества
	\begin{equation}
		\Omega^a_b = \{x\in\Omega : p(x) = a, q(x) = b\}
		,
		\quad 0 \leq b < a \leq 1
		,
	\end{equation}
	%являющегося разделяющим согласно следствию~\ref{crl:Lin_Omega_Sucheston},
	имеем $\mu\Omega^a_b = 0$, но  $\dim_H \Omega^a_b = 1$. А можно ещё <<меньше>>?

	\vfill


	\begin{ttheorem}
		Пусть $n\in\N$.
		Тогда существует разделяющее множество $E\subset\Omega$ такое,
		что $\dim_H E = 1/n$.
	\end{ttheorem}
\end{frame}
