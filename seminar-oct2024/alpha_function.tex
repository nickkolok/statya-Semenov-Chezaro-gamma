\begin{frame}\frametitle{$\alpha$--функция}
	\label{page:alpha_function}
	\begin{equation}
		\alpha(x) = \limsup_{i\to\infty} \max_{i<j\leqslant 2i} |x_i - x_j|
		.
	\end{equation}
	\vfill
	\begin{equation}
		\alpha(c)=0,
		\qquad
		\alpha(x+y) \leq \alpha(x) + \alpha(y),
		\qquad
		|\alpha(x) - \alpha(y)| \leq 2 \|x-y\|
		.
	\end{equation}
	\vfill
	$\alpha$--функция --- <<мера несходимости>> последовательности;
	\\
	равенство $\alpha(x) = 0$, однако, не гарантирует сходимость.

\end{frame}

\begin{frame}\frametitle{\underline{$\alpha$--функция и оператор сдвига}}
	\vfill
	$\alpha$--функция не инвариантна относительно оператора сдвига $T$
	\\
	(в отличие от банаховых пределов).

	\begin{ttheorem}
		Для любых $x\in\ell_\infty$ и $n \in \N$
		выполнено
		$\displaystyle\qquad
			\frac{1}{2}\alpha(x) \leq \lim_{m\to\infty} \alpha(T^m x)\leq\alpha(T^n x) \leq \alpha(x)
			.
		$
	\end{ttheorem}

	\vfill

	\begin{ttheorem}
		Пусть $\beta_k$~--- монотонная невозрастающая последовательность,
		$\beta_k \to \beta \in\left[\frac{1}{2}; 1\right]$, $\beta_1 \leq 1$.
		Тогда существует такой $x\in\ell_\infty$, что для любого натурального $n$
		\begin{equation}
			\frac{\alpha(T^n x)}{\alpha(x)} = \beta_n.
		\end{equation}
	\end{ttheorem}
	%\vfill
	%
	\vspace{-1.46em}
	\begin{ttheorem}
		\label{thm:rho_x_c_leq_alpha_t_s_x}
		Для любого $x\in ac$ и $n \in \N$
		выполнено
		$\displaystyle
			\frac{1}{2}\alpha(x) \leq \rho(x,c) \leq
			\lim_{m\to\infty} \alpha(T^m x)\leq \alpha(T^n x)
			\leq
			\alpha(x)
			.
		$
	\end{ttheorem}
	Аналогично для $ac_0$ и $c_0$. Условие почти сходимости существенно.

	\vfill
\end{frame}


\begin{frame}\frametitle{\underline{$\alpha$--функция и другие классические операторы}}
	\begin{ttheorem}
		Для любого $n\in\N$ и любого $x\in\ell_\infty$ выполнено
		\begin{equation}
			\alpha(\sigma_n x) = \alpha(x)
			\qquad\mbox{и}\qquad
			\alpha(\sigma_{1/n} x) \leq \left( 2- \frac{1}{n} \right) \alpha(x)
			.
		\end{equation}
	\end{ttheorem}

	\begin{ttheorem}
		Имеет место равенство
		\begin{equation}
			\sup_{x\in\ell_\infty, \alpha(x)\neq 0} \frac{\alpha(Cx)}{\alpha(x)}=1
			.
		\end{equation}
	\end{ttheorem}

	\begin{ttheorem}
		Пусть $(x\cdot y)_k = x_k\cdot y_k$.
		Тогда
		$\alpha(x\cdot y)\leq \alpha(x)\cdot \|y\|_* + \alpha(y)\cdot \|x\|_*$,
		где
		\begin{equation}
			\|x\|_* = \limsup_{k\to\infty} |x_k|
		\end{equation}
		есть  фактор-норма по $c_0$ на пространстве $\ell_\infty$.
	\end{ttheorem}
\end{frame}


\begin{frame}\frametitle{\underline{Пространство $A_0$}}

	Пространство $A_0 = \{x: \alpha(x) = 0\}$
	несепарабельно, замкнуто относительно покоординатного умножения,
	операторов левого и правого сдвигов $T$ и $U$,
	оператора Чезаро $C$,
	операторов растяжения $\sigma_n$ и усредняющего сжатия $\sigma_{1/n}$.

	\begin{equation}
		A_0 \cap ac_0 = c_0,
		\qquad
		A_0 \cap ac\, = c,
	\end{equation}

	\vfill

	Оба вложения
	\qquad$
		c_0 \subset A_0 \subset \ell_\infty
	$\qquad
	недополняемы.

	\vfill

	Более того, недополняемо вложение $c_0 \subset ac_0$
	\\
	(недополняемость вложения $ac_0 \subset \ell_\infty$ доказана Е.А. Алехно~\cite[{Theorem 8}]{alekhno2006propertiesII})
\end{frame}
