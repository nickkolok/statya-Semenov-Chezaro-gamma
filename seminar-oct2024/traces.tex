\begin{frame}\frametitle{Пустой слайд}

	Здесь ничего нет!
\end{frame}

\begin{frame}\frametitle{Пустой слайд}

	...только следы...
\end{frame}




\begin{frame}\frametitle{Следы}
	Пусть $(F, +, \cdot)$~--- некоторое кольцо.
	\begin{ddefinition}
		Следом называется такой функционал $\tau : F \to \R$,
		что для любых $A,B\in F$
		\begin{equation}
			\tau (A\cdot B) = \tau (B\cdot A)
			.
		\end{equation}
	\end{ddefinition}
	\vfill
	Пример: классический след на кольце квадратных матриц
	\\
	(сумма диагональных элементов).


\end{frame}


\begin{frame}\frametitle{Следы на операторах}
	Пусть $H$~--- сепарабельное гильбертово пространство,
	$B(H)$~--- алгебра всех ограниченных линейных операторов на $H$.

	\bigskip

	Классический след:
	$${\rm Tr} (A)= \sum_{k=0}^\infty \lambda(k,A),$$
	где $\{\lambda(k,A)\}_{k=0}^\infty$ --- последовательность собственных значений компактного оператора $A$.

	\bigskip

	След называется нормальным, если
	\\
	для любой монотонно сходящейся $A_k\to A$ верно $\operatorname{Tr} A_k \to\operatorname{Tr} A$.

	\vfill
	Классический след является единственным нормальным (с точностью до константы).

	А есть ли другие, анормальные?


\end{frame}



 \begin{frame}\frametitle{Конструкция Диксмье (1966)}

	Сингулярные значения $\mu(k,A)=\lambda(k,|A|)$ (в убывающем порядке),
	где $|A|=\sqrt{A^*A}$.

	Пространство Лоренца:
	$$\mathcal M_{1,\infty} := \left\{ A \ \text{компактен} \ : \ \sup_{n> 0}  \frac{1}{\log(1+n)} \sum_{k=0}^{n-1} \mu(k,A) < \infty \right \}.$$

	Положим
	$${\rm Tr}_\omega (A):= \omega \left( \frac{1}{\log(1+n)} \sum_{k=0}^{n-1} \lambda(k,A) \right), \quad 0\le A \in \mathcal M_{1,\infty},$$
	где $\omega\in \ell_\infty^*$ такое, что
	$$\omega(x_0, x_1, \dots)=\omega(x_0, x_0, x_1, x_1, \dots).$$


	\bigskip
	1. ${\rm Tr}_\omega$ --- след;

	2. ${\rm Tr}_\omega$ нетривиален: ${\rm Tr}_\omega ({\rm diag} \left\{ \frac1{n+1} \right\})=1$;

	3. ${\rm Tr}_\omega(A)=0$, если $A$ любой оператор конечного ранга.
	\bigskip

	\begin{center}
	Т.е. ${\rm Tr}_\omega$ --- нетривиальный, анормальный след.
	\end{center}
\end{frame}





\begin{frame}\frametitle{Следы и банаховы пределы}
	 Для любого банахова предела $B$ функционал
	 $$A \mapsto B \left( \frac{1}{1+n} \sum_{k=0}^{2^n-1} \lambda(k,A) \right), \quad 0\le A \in \mathcal M_{1,\infty}$$
	продолжается по линейности на все пространство $\mathcal M_{1,\infty}$ до некоторого следа Диксмье\footnote{П. Г. Доддс, Б. де Пагтер, А. А. Седаев, Е. М. Семенов, Ф. А. Сукочев, Изв. РАН. Сер. матем., 67:6 (2003),  111–136.}.

	\vfill

		Отображение $\omega \mapsto {\rm Tr}_\omega$ из множества $\sigma_2$-инвариантных обобщённых пределов на $\ell_\infty$ во множество сингулярных следов на $\mathcal M_{1,\infty}$ не является ни инъективным,\footnote{Sukochev, F., Usachev, A., and Zanin, D., Adv. Math. 239 (2013), 164--189.
	} ни сюръективным\footnote{Kalton, N., and Sukochev, F., Canad. Math. Bull. 51, 1 (2008), 67--80.}.
\end{frame}




\begin{frame}\frametitle{Альтернативная конструкция сингулярных следов (Е.М. Семенов и др.)}

	Обозначим через $\mathcal L_{1,\infty}$ линейное пространство всех компактных операторов на фиксированном гильбертовом прострастве $H$, для которых квази-норма
	$$\|A\|_{\mathcal L_{1,\infty}} = \sup_{n \ge 0} (n+1) \mu(n,A)$$
	конечна. Легко видеть, что $\mathcal L_{1,\infty} \subsetneq \mathcal M_{1,\infty}$.

	\vfill

	Зафиксируем некий базис $\{e_n\}$ в гильбертовом пространстве $H$ и обозначим через ${\rm diag}$ диагональный оператор в $B(H)$, соответствующий этому базису.

\end{frame}

\begin{frame}\frametitle{Альтернативная конструкция сингулярных следов (Е.М. Семенов и др.)}

	Определим оператор $D: \ell_\infty \to \mathcal L_{1,\infty}$ следующим образом
	\begin{align*}
	D(x_0, x_1, \dots) := \ln 2 \cdot {\rm diag} \sum_{n=0}^\infty \frac{x_n}{2^n} \chi_{[2^n-1,2^{n+1}-2]}.
	\end{align*}

	\begin{ttheorem}
		(i) Для любого инвариантного относительно сдвига функционала $\theta$ на $\ell_\infty$ функционал
	\begin{equation}\label{ssec:Gen:formula1}
	\tau(A) = \theta\left(\sum_{i=2^n-1}^{2^{n+1}-2} \lambda(i,A) \right) , \quad A\ge0,
	\end{equation}
	продолжается по линейности до сингулярного следа $\tau$ на $\mathcal L_{1,\infty}$.

	(ii) Для любого сингулярного следа $\tau$ на $\mathcal L_{1,\infty}$ и инвариантного относительно сдвига функционала $\theta=\tau\circ D$ на $\ell_\infty$ справедлива формула ~\eqref{ssec:Gen:formula1}.
	\end{ttheorem}

\end{frame}



