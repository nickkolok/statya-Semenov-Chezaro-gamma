\documentclass[11pt,a4paper,openbib,twocolumn,landscape]{report}
\usepackage{amsmath}
\usepackage[utf8]{inputenc}
\usepackage[english,russian]{babel}
\usepackage{amsfonts,amssymb}
\usepackage{latexsym}
\usepackage{euscript}
\usepackage{enumerate}
\usepackage{graphics}
\usepackage[dvips]{graphicx}
\usepackage{geometry}
\usepackage{wrapfig}
\usepackage[colorlinks=true,allcolors=black]{hyperref}
\usepackage{bbm}
\usepackage{enumitem}
\usepackage{mathrsfs}


% https://tex.stackexchange.com/questions/634509/show-hide-thumbnail-sidebar-by-default-in-pdf
\hypersetup{pdfpagemode=UseNone}


\righthyphenmin=2

%\usepackage[14pt]{extsizes}

\geometry{left=1cm}% левое поле
\geometry{right=1cm}% правое поле
\geometry{top=1cm}% верхнее поле
\geometry{bottom=1cm}% нижнее поле

\renewcommand{\baselinestretch}{1.3}

\renewcommand{\le}{\leqslant}
\renewcommand{\ge}{\geqslant}
\renewcommand{\leq}{\leqslant}
\renewcommand{\geq}{\geqslant} % И делись оно всё нулём!

\renewcommand{\varlimsup}{\limsup}
\renewcommand{\varliminf}{\liminf} % ... по самую асимптоту!



\DeclareMathOperator{\ext}{ext}
\DeclareMathOperator{\mes}{mes}
\DeclareMathOperator{\supp}{supp}
\DeclareMathOperator{\conv}{conv}
\DeclareMathOperator{\diam}{diam}

\newcommand{\N}{\ensuremath{\mathbb{N}}}
\newcommand{\Q}{\ensuremath{\mathbb{Q}}}
\newcommand{\R}{\ensuremath{\mathbb{R}}}
\newcommand{\B}{\ensuremath{\mathfrak{B}}}
\newcommand{\Iac}{\mathcal{I}(ac_0)}
\newcommand{\Dac}{\mathcal{D}(ac_0)}
\newcommand{\one}{\ensuremath{\mathbbm 1}}


\newcommand{\longcomment}[1]{}

\usepackage[backend=biber,style=gost-numeric,sorting=none]{biblatex}
\addbibresource{../bib/Semenov.bib}
\addbibresource{../bib/my.bib}
\addbibresource{../bib/ext.bib}
\addbibresource{../bib/classic.bib}
\addbibresource{../bib/Damerau-Levenstein.bib}
\addbibresource{../bib/general_monographies.bib}
\addbibresource{../bib/Bibliography_from_Usachev.bib}

\input{../bib/ext.hyphens.bib}

\usepackage{amsthm}
\theoremstyle{definition}
\newtheorem{lemma}{Лемма}[section]
\newtheorem{theorem}[lemma]{Теорема}
\newtheorem{example}[lemma]{Пример}
\newtheorem{property}[lemma]{Свойство}
\newtheorem{remark}[lemma]{Замечание}
\newtheorem{definition}[lemma]{Определение}
\newtheorem{proposition}[lemma]{Утверждение}
\newtheorem{corollary}[lemma]{Следствие}

\newtheorem{hhypothesis}[lemma]{Гипотеза}


\newcommand\hypotlist{ }
\newcounter{hypcount}

\makeatletter
\usepackage{environ}
\NewEnviron{hypothesis}{%

	\edef\curlabel{hhypothesis\thehypcount}
    \begin{hhypothesis}
		\label{\curlabel}
		\BODY%
    \end{hhypothesis}
	\edef\curref{\noexpand\ref{\curlabel}}

	\expandafter\g@addto@macro\expandafter\hypotlist\expandafter
	{\paragraph{Гипотеза\!\!\!}}


	\expandafter\g@addto@macro\expandafter\hypotlist\expandafter
	{\expandafter\textbf\expandafter{\curref}}

	\expandafter\g@addto@macro\expandafter\hypotlist\expandafter
	{\textbf{.}~~}

	\expandafter\g@addto@macro\expandafter\hypotlist\expandafter
	{\BODY}

	\addtocounter{hypcount}{1}
}
\makeatother

%For the introduction
\newcommand\reflecttheorem[1]{\paragraph{Теорема \ref{#1}}}
\newcommand\reflectlemma[1]{\paragraph{Лемма \ref{#1}}}
\newcommand\reflectcorollary[1]{\paragraph{Следствие \ref{#1}}}
\newcommand\reflectdefinition[1]{\paragraph{Определение \ref{#1}}}

%For the main text, we do not make the difference between
%self-citation and external citation
%(for autoref we do really do!)
\def\selfcite{\cite}

%Only referenced equations are numbered
\usepackage{mathtools}
\mathtoolsset{showonlyrefs}

%\mathtoolsset{showonlyrefs=false}
% (an equation/multline to be force-numbered)
%\mathtoolsset{showonlyrefs=true}

% https://superuser.com/questions/517025/how-can-i-append-two-pdfs-that-have-links
\usepackage{pdfpages}% http://ctan.org/pkg/pdfpages

\begin{document}
\clubpenalty=10000
\widowpenalty=10000
\pagenumbering{gobble}





\paragraph{Числовые множества.}
$\N = \{1;2;3;4;...\}$~--- множество натуральных чисел.
$\N_k$ ~--- множество целых чисел, не меньших $k$. Так,
\begin{equation}
	\N_1 = \N
	,
	\quad
	\N_0 = \N \cup\{0\} = \{0;1;2;3;4;...\}
	,
	\quad
	\N_3 = \{3;4;5;6;7;...\}
	.
\end{equation}

$\Q$ и $\R$~--- множества рациональных и вещественных чисел соотв.;
$\Q^+$, $\Q^-$, $\R^+$ и $\R^-$ ~--- множества положительных рациональных, отрицательных рациональных,
положительных вещественных и отрицательных вещественных чисел соотв.

\paragraph{Пространства и множества последовательностей.}

Основное пространство~--- $\ell_\infty$ ограниченных последовательностей со стандартной нормой
\begin{equation}
	\|x\| = \sup_{n\in\N} |x_n|
	,
	~~\mbox{где}~~
	x=(x_1, x_2, x_3, x_4, ...)
	.
\end{equation}
Эта же норма будет используется и в остальных пространствах и множествах:
\begin{itemize}
	\setlength\itemsep{0.1em}
	\item
		$c$ ~--- пространство сходящихся последовательностей;
	\item
		$c_\lambda$ ~--- множество последовательностей, сходящихся к $\lambda \in \R$, в т.ч.,
	\item
		$c_0$ ~--- пространство последовательностей, сходящихся к нулю;
	\item
		$c_{00}$ ~--- пространство последовательностей с конечным носителем;
	\item
		$ac$ ~--- пространство почти сходящихся последовательностей;
	\item
		$ac_\lambda$ ~--- множество последовательностей, почти сходящихся к $\lambda \in \R$, в т.ч.,
	\item
		$ac_0$ ~--- пространство последовательностей, почти сходящихся к нулю;
	\item
		$\Iac$ ~--- максимальный (по включению) идеал по умножению в $ac_0$;
	\item
		$A_0 = \{ x\in\ell_\infty : \alpha(x) = 0\}$ (см. ниже);
	\item
		$\Omega = \{0;1\}^\N$ ~--- множество последовательностей из нулей и единиц.
\end{itemize}



\paragraph{Нормы.}
Запись $\|\cdot\|$ без уточнений будет обозначать норму в пространстве $\ell_\infty$, $\ell_\infty^*$
или в пространстве  $\mathcal L (\ell_\infty, \ell_\infty)$ линейных ограниченных операторов, действующих из $\ell_\infty$ в $\ell_\infty$ в зависимости от природы аргумента.
Использование других норм (например, фактор-нормы $\ell_\infty / c_0$)
оговаривается явно.


\paragraph{Последовательности.} Константная единица: $\one = (1, 1, 1, 1, ...)$.
%
\\
$x\geq 0$, если $x_n \geq 0$ для всех $n\in \N$;\hfill~$x\leq 0$, если $x_n \leq 0$ для всех $n\in \N$.
%TODO: периодические последовательности и конкатенация

$\mathscr{M}A$ ~--- множество всех чисел,
кратных элементам множества $A\subset\N$, т.е.
\begin{equation}
	\mathscr{M}A = \{ka: k\in\N, a\in A\}
	;
\end{equation}
$\chi F$ "--- характеристическая функция множества $F$.
Например,
\begin{gather}
	\chi \mathscr{M}\!A(\{2\}) = \chi \mathscr{M}\!A(\{2, 4\}) = \chi \mathscr{M}\!A(\{2,4,8,16,...\})
	= (0,1,0,1,0,1,0,1,0,...),
\\
	\chi \mathscr{M}\!A(\{3\}) = \chi \mathscr{M}\!A(\{3,9,27,...\}) = (0,0,1,\;0,0,1,\;0,0,1,\;0,0,1,\;0,0,1,\;...),
\\
	\chi \mathscr{M}\!A(\{2,3\}) = \chi \mathscr{M}\!A(\{2,3,6\}) = (0,1,1,1,0,1,\;0,1,1,1,0,1,\;0,1,1,1,0,1,...).
\end{gather}


\paragraph{Операторы.}
\begin{itemize}
	\item
		$T(x_1, x_2, x_3, x_4, ...) = (x_2, x_3, x_4, ...)$ ~--- оператор сдвига влево
	\item
		$
			U(x_1, x_2, x_3, x_4, ...) = (0, x_1, x_2, x_3, x_4, ...)
		$~--- оператор сдвига вправо
	\item
		$
			\sigma_n (x_1, x_2, x_3, ...) = (
				\underbrace{x_1,...,x_1}_{n~\text{раз}},
				\underbrace{x_2,...,x_2}_{n~\text{раз}},
				%\underbrace{x_3,...,x_3}_{n~\text{раз}},
				...)
		$~--- оператор растяжения ($n\in\N$)
	\item
		$
			\sigma_{1/n} x = n^{-1}
			\left(
				\sum_{i=1}^{n} x_i,
				\sum_{i=n+1}^{2n} x_i,
				%\sum_{i=2n+1}^{3n} x_i,
				...
			\right)
		$~--- усредняющее сжатие ($n\in\N$).
	\item
		$
			(Cx)_n = n^{-1} \cdot \sum_{k=1}^n x_k
		$~---оператор Чезаро,
		т.е.
		\\~\quad
		$
			C (x_1, x_2, x_3, ...) = \left(
			x_1,
			\dfrac{x_1+x_2}2,
			\dfrac{x_1+x_2 + x_3}3,
			%\dfrac{x_1+x_2+x_3+x_4}4,
			...,
			\dfrac{x_1+...+x_n}n,
			...\right)
		$
	\item
		$
			x \cdot y = (x_1\cdot y_1; ~~x_2\cdot y_2; ~...)
		$~---суперпозиция (покоординатное умножение)

\end{itemize}


\paragraph{Специальные функции и множества.}

$\B$~--- множество всех банаховых пределов;
$\B(H)$~--- множество всех банаховых пределов, инвариантных относительно оператора $H$;
%TODO: ссылку на теорему/определние?
$\ext A$~--- множество крайних точек множества $A$.

Нижний и верхний функционалы Сачестона соответственно:
\begin{equation*}
	q(x) = \lim_{n\to\infty} \inf_{m\in\N}  \frac{1}{n} \sum_{k=m+1}^{m+n} x_k
	~~~~\mbox{и}~~~~
	p(x) = \lim_{n\to\infty} \sup_{m\in\N}  \frac{1}{n} \sum_{k=m+1}^{m+n} x_k
	.
\end{equation*}

Кроме того, \quad
$
	\displaystyle
	\alpha(x) = \varlimsup_{i\to\infty} \max_{i<j\leqslant 2i} |x_i - x_j|
$.


\end{document}
