\begin{frame}\frametitle{Метаданные: оппоненты и ведущая организация}
	\begin{itemize}
		\item
			Оппонент: Бородин Пётр Анатольевич, д.ф.-м.н., профессор РАН,
			\\
			профессор кафедры теории функций и функционального анализа Механико-математического факультета
			ФГБОУ ВО «Московский государственный университет им. М. В. Ломоносова»
			\\
			pborodin@inbox.ru , +7 (926) 217 1347
		\item
			Оппонент: Ушакова Елена Павловна, д.ф.-м.н., проф.,
			\\
			вед.н.с. лаборатории №45,
			ФГБУН «Институт проблем управления им.~В.А.~Трапезникова Российской Академии Наук»
			\\
			elenau@inbox.ru , +7 (962) 586 8768
		\item
			Ведущая организация:
			ФГБОУ ВО «Санкт-Петербургский государственный университет»
			\\
			Представитель:
			Скопина Мария Александровна, д.ф.-м.н., проф.,
			\\
			профессор факультета прикладной математики - процессов управления
			\\
			skopinama@gmail.com , skopina@MS1167.spb.edu
	\end{itemize}
\end{frame}

\begin{frame}\frametitle{Метаданные: публикации соискателя в Scopus по теме диссертации}
	\begin{enumerate}
		\small
		\item
		\emph{Авдеев} \emph{Н. Н.} — О пространстве почти сходящихся
		последовательностей //Математические заметки. — 2019. — т. 105, No 3. —
		с. 462—466. — DOI: \href {https://doi.org/10.4213/mzm12298} {\nolinkurl
		{10.4213/mzm12298}}.
		{}
		\item
		\emph{Авдеев} \emph{Н. Н.}, \emph{Cеменов} \emph{Е. М.},
		\emph{Усачев} \emph{А. С.} — Банаховы пределы и мера на множестве
		последовательностей из 0 и 1 //Математические заметки. — 2019. — т. 106,
		No 5. — с. 784—787.
		{}
		\item
		\emph{Авдеев} \emph{Н. Н.} — О подмножествах пространства ограниченных
		последовательностей //Математические заметки. — 2021. — т. 109, No 1. —
		с. 150—154.
		{}
		\item
		\emph{Авдеев} \emph{Н. Н.}, \emph{Семёнов} \emph{Е. М.},
		\emph{Усачев} \emph{А. С.} — Банаховы пределы: экстремальные свойства,
		инвариантность и теорема Фубини //Алгебра и анализ. — 2021. — т. 33, No 4. —
		с. 32—48.
		{}
		\item
		\emph{Авдеев} \emph{Н. Н.} — Почти сходящиеся последовательности из 0 и 1 и
		простые числа //Владикавказский математический журнал. — 2021. — т. 23,
		No 4. — с. 5—14. — DOI: \href {https://doi.org/10.46698/p9825-1385-3019-c}
		{\nolinkurl {10.46698/p9825-1385-3019-c}}.
		{}
		\item
		Decomposition of the set of Banach limits into discrete and continuous subsets. /. —
		N. Avdeev [et al.] //Annals of Functional Analysis. — 2024. — т. 15, No 81. — DOI: \href
		{https://doi.org/10.1007/s43034-024-00382-5} {\nolinkurl
		{10.1007/s43034-024-00382-5}}.
		{}
		\item
		The set of Banach limits and its discrete and continuous subsets. /. — N. Avdeev
		[et al.] //Doklady Mathematics. — 2024. — т. 110, No 1. — с. 346—348. — DOI: \href
		{https://doi.org/10.1134/S106456242470217X} {\nolinkurl
		{10.1134/S106456242470217X}}.
	\end{enumerate}

\end{frame}

\begin{frame}\frametitle{Метаданные: прочие публикации соискателя в Scopus}



	\begin{block}{Integral point sets (\textit{комбинаторная геометрия}):}
		\begin{enumerate}
			\item
				Avdeev N. N. On existence of integral point sets and their diameter bounds
				//Australasian Journal of Combinatorics. – 2020. – V. 77. – Pp. 100-116.
			\item
				Авдеев Н. Н. Об оценках на диаметр плоских множеств с целочисленными расстояниями полуобщего положения
				//Чебышевский сборник. – 2021. – Т. 22. – №.~4 (80). – С. 343-350.
			\item
				Авдеев Н. Н., Семёнов Е. М. О множествах точек на плоскости с целочисленными расстояниями
				//Математические заметки. – 2016. – Т. 100. – №.~5. – С. 757-761.
		\end{enumerate}
	\end{block}



	\begin{block}{Теория аттракторов:}
		\begin{enumerate}
			\setcounter{enumi}{3}
			\item
				Звягин В. Г., Авдеев Н. Н. Пример системы, минимальный траекторный аттрактор которой не содержит решений системы //Математические заметки. – 2018. – Т. 104. – №. 6. – С. 937-941.
		\end{enumerate}
	\end{block}

\end{frame}

