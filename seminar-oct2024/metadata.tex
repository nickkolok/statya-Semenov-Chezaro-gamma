\begin{frame}\frametitle{Метаданные: оппоненты и ведущая организация}
	\begin{itemize}
		\item
			Оппонент: Бородин Пётр Анатольевич, д.ф.-м.н., профессор РАН,
			\\
			профессор кафедры теории функций и функционального анализа Механико-математического факультета
			ФГБОУ ВО «Московский государственный университет им. М. В. Ломоносова»
			\\
			pborodin@inbox.ru , +7 (926) 217 1347
		\item
			Оппонент: Ушакова Елена Павловна, д.ф.-м.н., проф.,
			\\
			вед.н.с. лаборатории №45,
			ФГБУН «Институт проблем управления им.~В.А.~Трапезникова Российской Академии Наук»
			\\
			elenau@inbox.ru , +7 (962) 586 8768
		\item
			Ведущая организация:
			ФГБОУ ВО «Санкт-Петербургский государственный университет»
			\\
			Представитель:
			Скопина Мария Александровна, д.ф.-м.н., проф.,
			\\
			профессор факультета прикладной математики - процессов управления
			\\
			skopinama@gmail.com , skopina@MS1167.spb.edu
	\end{itemize}
\end{frame}

\begin{frame}\frametitle{Метаданные: прочие публикации соискателя в Scopus}



	\begin{block}{Integral point sets (\textit{комбинаторная геометрия}):}
		\begin{enumerate}
			\item
				Avdeev N. N. On existence of integral point sets and their diameter bounds
				//Australasian Journal of Combinatorics. – 2020. – V. 77. – Pp. 100-116.
			\item
				Авдеев Н. Н. Об оценках на диаметр плоских множеств с целочисленными расстояниями полуобщего положения
				//Чебышевский сборник. – 2021. – Т. 22. – №.~4 (80). – С. 343-350.
			\item
				Авдеев Н. Н., Семёнов Е. М. О множествах точек на плоскости с целочисленными расстояниями
				//Математические заметки. – 2016. – Т. 100. – №.~5. – С. 757-761.
		\end{enumerate}
	\end{block}



	\begin{block}{Теория аттракторов:}
		\begin{enumerate}
			\setcounter{enumi}{3}
			\item
				Звягин В. Г., Авдеев Н. Н. Пример системы, минимальный траекторный аттрактор которой не содержит решений системы //Математические заметки. – 2018. – Т. 104. – №. 6. – С. 937-941.
		\end{enumerate}
	\end{block}

\end{frame}

