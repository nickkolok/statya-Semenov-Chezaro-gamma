\begin{frame}\frametitle{Метаданные: оппоненты и ведущая организация}
	\label{page:metadata}
	\begin{itemize}
		\item
			Оппонент: Бородин Петр Анатольевич, д.ф.-м.н., профессор РАН,
			\\
			профессор кафедры теории функций и функционального анализа механико-математического факультета
			ФГБОУ ВО «Московский государственный университет им. М. В. Ломоносова»
			\\
			pborodin@inbox.ru , +7 (926) 217 1347
		\item
			Оппонент: Ушакова Елена Павловна, д.ф.-м.н.,
			\\
			вед.н.с. лаборатории №45,
			ФГБУН «Институт проблем управления им.~В.А.~Трапезникова Российской Академии Наук»
			\\
			elenau@inbox.ru , +7 (962) 586 8768
		\item
			Ведущая организация:
			\\
			ФГБОУ ВО «Санкт-Петербургский государственный университет»
			\\
			Представитель:
			Скопина Мария Александровна, д.ф.-м.н., проф.,
			\\
			профессор факультета прикладной математики -- процессов управления
			\\
			skopinama@gmail.com
	\end{itemize}
\end{frame}

\begin{frame}\frametitle{Метаданные: публикации соискателя в Scopus по теме диссертации}
	\begin{enumerate}
		\small
		\item
		\emph{Авдеев} \emph{Н. Н.} — О пространстве почти сходящихся
		последовательностей //Математические заметки. — 2019. — т. 105, No 3. —
		с. 462—466. — DOI: \href {https://doi.org/10.4213/mzm12298} {\nolinkurl
		{10.4213/mzm12298}}.
		{}
		\item
		\emph{Авдеев} \emph{Н. Н.}, \emph{Cеменов} \emph{Е. М.},
		\emph{Усачев} \emph{А. С.} — Банаховы пределы и мера на множестве
		последовательностей из 0 и 1 //Математические заметки. — 2019. — т. 106,
		No 5. — с. 784—787.
		{}
		\item
		\emph{Авдеев} \emph{Н. Н.} — О подмножествах пространства ограниченных
		последовательностей //Математические заметки. — 2021. — т. 109, No 1. —
		с. 150—154.
		{}
		\item
		\emph{Авдеев} \emph{Н. Н.}, \emph{Семёнов} \emph{Е. М.},
		\emph{Усачев} \emph{А. С.} — Банаховы пределы: экстремальные свойства,
		инвариантность и теорема Фубини //Алгебра и анализ. — 2021. — т. 33, No 4. —
		с. 32—48.
		{}
		\item
		\emph{Авдеев} \emph{Н. Н.} — Почти сходящиеся последовательности из 0 и 1 и
		простые числа //Владикавказский математический журнал. — 2021. — т. 23,
		No 4. — с. 5—14. — DOI: \href {https://doi.org/10.46698/p9825-1385-3019-c}
		{\nolinkurl {10.46698/p9825-1385-3019-c}}.
		{}
		\item
		Decomposition of the set of Banach limits into discrete and continuous subsets. /. —
		N. Avdeev [et al.] //Annals of Functional Analysis. — 2024. — т. 15, No 81. — DOI: \href
		{https://doi.org/10.1007/s43034-024-00382-5} {\nolinkurl
		{10.1007/s43034-024-00382-5}}.
		{}
		\item
		The set of Banach limits and its discrete and continuous subsets. /. — N. Avdeev
		[et al.] //Doklady Mathematics. — 2024. — т. 110, No 1. — с. 346—348. — DOI: \href
		{https://doi.org/10.1134/S106456242470217X} {\nolinkurl
		{10.1134/S106456242470217X}}.
	\end{enumerate}
\end{frame}



\begin{frame}\frametitle{Метаданные: прочие публикации соискателя по теме диссертации}
	\begin{enumerate}
		\setcounter{enumi}{7}
		\item
		\emph{Авдеев} \emph{Н. Н.} — О разделяющих множествах меры нуль и
		функционалах Сачестона //Вестн. ВГУ. Серия: Физика. Математика. — 2021. —
		т. 4. — с. 38—50.
		\hfill\textbf{(ВАК)}
		{}
		\item
		\emph{Авдеев} \emph{Н. Н.}, \emph{Семенов} \emph{Е. М.} — Об
		асимптотических свойствах оператора Чезаро //Воронежская Зимняя
		Математическая школа С.Г. Крейна – 2018. Материалы международной
		конференции. Под ред. В.А. Костина. — 2018. — с. 107—109.
		{}
		\item
		\emph{Авдеев} \emph{Н. Н.} — Оператор с бесконечномерным ядром, для которого
		существует инвариантный банахов предел //Некоторые вопросы анализа,
		алгебры, геометрии и математического образования. — 2018. — с. 18—19.
		{}
		\item
		\emph{Авдеев} \emph{Н. Н.} — Замечание об инвариантных банаховых пределах
		//Современные методы теории краевых задач : материалы международной
		конференции «Понтрягинские чтения — XXIX», посвященной 90-летию Владимира
		Александровича Ильина (2–6 мая 2018 г.) — 2018. — с. 31—32.
		{}
		\item
		\emph{Авдеев} \emph{Н. Н.} — О суперпозиции оператора сдвига и одной функции
		на пространстве ограниченных последовательностей //Некоторые вопросы
		анализа, алгебры, геометрии и математического образования. — 2018. — с. 20—21.
		{}
		\item
		\emph{Авдеев} \emph{Н. Н.} — О мере множеств, разделяющих банаховы пределы
		 //Современные проблемы математики и её приложений. — Институт математики
		и механики УрО РАН им. Н.Н. Красовского, 2022. — с. 56—57.



	\end{enumerate}

\end{frame}



\begin{frame}\frametitle{Метаданные: прочие публикации соискателя в Scopus}



	\begin{block}{Integral point sets (\textit{комбинаторная геометрия}):}
		\begin{enumerate}
			\item
				Avdeev N. N. On existence of integral point sets and their diameter bounds
				//Australasian Journal of Combinatorics. – 2020. – V. 77. – Pp. 100-116.
			\item
				Авдеев Н. Н. Об оценках на диаметр плоских множеств с целочисленными расстояниями полуобщего положения
				//Чебышевский сборник. – 2021. – Т. 22. – №.~4 (80). – С. 343-350.
			\item
				Авдеев Н. Н., Семёнов Е. М. О множествах точек на плоскости с целочисленными расстояниями
				//Математические заметки. – 2016. – Т. 100. – №.~5. – С. 757-761.
		\end{enumerate}
	\end{block}



	\begin{block}{Теория аттракторов:}
		\begin{enumerate}
			\setcounter{enumi}{3}
			\item
				Звягин В. Г., Авдеев Н. Н. Пример системы, минимальный траекторный аттрактор которой не содержит решений системы //Математические заметки. – 2018. – Т. 104. – №. 6. – С. 937-941.
		\end{enumerate}
	\end{block}

\end{frame}



\begin{frame}\frametitle{Метаданные: доклады соискателя по теме диссертации}
	\begin{enumerate}
		\item
			16.10.2024 г. на семинаре в МИАН под рук. чл.-корр. РАН О.В. Бесова
		\item
			15.10.2024 г. на семинаре в МГУ под рук. проф. РАН П.А. Бородина
		\item
			13.11.2024 г. на семинаре под рук. проф. С.В. Асташкина (Самара)
		\item
			19.11.2024 г. на семинаре под рук. проф. А.Л. Скубачевского (Москва)
		\item
			29.01.2025 г. на семинаре под рук. проф. А.Г. Кусраева и М.А. Плиева (Владикавказ)
		\item
			на конкурсе научно-исследовательских работ студентов и аспирантов российских вузов
			<<Наука будущего --- наука молодых>> в секции <<Информационные технологии и математика>>
			(присуждено II место среди аспирантов) в ноябре 2021~г.;
		\item
			на международной (53-й Всероссийской) молодёжной школе-конференции
			<<Современные проблемы математики и её приложений>>
			(Екатеринбург, 2022~г.);
		\item
			на международной (56-й Всероссийской) молодёжной школе-конференции
			<<Современные проблемы математики и её приложений>>
			(Екатеринбург, 2025~г.);
		\item
			на международной конференции <<Воронежская Зимняя Математическая школа С.Г. Крейна>> в 2018, 2022, 2025~гг.;
		\item
			на международной конференции «Понтрягинские чтения — XXIX», посвященной 90-летию Владимира Александровича Ильина (Воронеж, 2018~г.);
		\item
			на международной молодежной научной школе «Актуальные направления математического анализа и смежные вопросы» (Воронеж, 2018~г.);
		\item
			на научной сессии ВГУ в 2020, 2021, 2022, 2024 гг.

	\end{enumerate}
\end{frame}
