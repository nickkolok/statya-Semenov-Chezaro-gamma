\documentclass[a4paper,14pt]{article}
\usepackage[utf8]{inputenc}
\usepackage[T2A]{fontenc}
\usepackage{float}
\usepackage{cite}
\usepackage[english,russian]{babel}
\usepackage{amssymb,amsmath,amsfonts}
\usepackage{color}
\usepackage{enumerate}
\usepackage[dvips]{graphicx}
\usepackage{setspace}
\usepackage{xcolor}
\usepackage{fancyhdr}

\textheight=220mm
\textwidth=160mm
\oddsidemargin=0.1in
\evensidemargin=0.1in

\begin{document}

\pagestyle{fancy}
\fancyhead{}
\fancyhead[L]{\footnotesize{Современные проблемы математики и ее приложений}}
\fancyhead[R]{\footnotesize{Тезисы конференции}}
\fancyfoot{}
\fancyfoot[L]{\footnotesize{Екатеринбург, Россия}}
\fancyfoot[R]{\footnotesize{}}
\renewcommand{\footrulewidth}{0.1 mm}

\begin{center}
\textbf{Банаховы пределы и некоторые классы линейных операторов}%
%TODO:
\footnote{Исследование выполнено за счет гранта Российского научного фонда (проект № 19-11-00197).}\\
\vspace{\baselineskip}
Авдеев Н.Н.\\
\emph{Воронежский госуниверситет, Воронеж, Россия}\\nickkolok@mail.ru, avdeev@math.vsu.ru
\vspace{\baselineskip}\\
Зволинский Р.Е.\\
\emph{Воронежский госуниверситет, Воронеж, Россия}\\roman.zvolinskiy@gmail.com, roman-zvolinskiy@mail.ru
\vspace{\baselineskip}\\
Усачев А.С.\\
\emph{TODO}\\alex.usachev.ru@gmail.com, usa-alexandr@yandex.ru
\vspace{\baselineskip}\\
\end{center}
\vspace{\baselineskip}

%На коллег надейся, а сам не плошай!
\setcounter{equation}{0}
\setcounter{figure}{0}

Обозначим через $\ell_\infty$ пространство ограниченных последовательностей с обычной нормой
$
	\|x\| = \sup_{k\in\mathbb{N}} |x_k|
	,
$
где $\mathbb{N}$ "--- множество натуральных чисел, и обычной полуупорядоченностью.
Естественным обобщением предела с пространства сходящихся последовательностей $c$ на $\ell_\infty$
является понятие банахова предела.

\textbf{Определение.}
	Линейный функционал $B\in \ell_\infty^*$ называется банаховым пределом,
	если
	\begin{enumerate}
		\item
			$B\geq0$, т.~е. $Bx \geq 0$ для $x \geq 0$,
		\item
			$B\mathbb{I}=1$, где $\mathbb{I} =(1,1,\ldots)$,
		\item
			$B(Tx)=B(x)$ для всех $x\in \ell_\infty$, где $T$~---
			оператор сдвига, т.~е. $T(x_1,x_2,\ldots)=(x_2,x_3,\ldots)$.
	\end{enumerate}

Кратко пишем: $B \in \mathfrak{B}$.
Существование банаховых пределов было анонсировано С. Мазуром~\cite{Mazur} и доказано С.~Банахом~\cite{banach1993theorie}.
Сачестон установил~\cite{sucheston1967banach}, что
для любых $x\in \ell_\infty$ и $B\in\mathfrak{B}$
\begin{equation*}\label{Sucheston}
	q(x) \leqslant Bx \leqslant p(x)
	,
	\quad\mbox{~~где~~}\quad
	q(x) = \lim_{n\to\infty} \inf_{m\in\mathbb{N}}  \frac{1}{n} \sum_{k=m+1}^{m+n} x_k
	~~~~\mbox{и}~~~~
	p(x) = \lim_{n\to\infty} \sup_{m\in\mathbb{N}}  \frac{1}{n} \sum_{k=m+1}^{m+n} x_k
\end{equation*}
суть нижний и верхний функционалы Сачестона соотв.
Множество таких $x\in\ell_\infty$, что $p(x)=q(x)$,
образует~\cite{lorentz1948contribution} пространство почти сходящихся последовательностей $ac$.
На каждом $x\in ac$ все $B\in \mathfrak{B}$ принимают одинаковые значения.

Все банаховы пределы инвариантны относительно сдвига по определению.
Однако уже для совершенно естественного "--- если бы мы говорили об обычной сходимости последовательностей "---
оператора растяжения $\sigma_k$, $k\in \mathbb N$, $k\ge2$, который просто повторяет каждый элемент последовательности $k$ раз,
выясняется, что относительно этого оператора инварианты не все банаховы пределы и,
более того, инвариантность конкретного банахова предела зависит от выбора $k$.

В настоящей работе мы строим классы линейных операторов,
основываясь на их свойствах инвариантности относительно банаховых пределов и функционалов Сачестона,
начиная с самых широких.

\textbf{Определение.}
Полуэберлейновым назовём такой оператор $A:\ell_\infty \to \ell_\infty$,
что $BA\in\mathfrak B$ для некоторого $B\in\mathfrak B$.

Класс полуэберлейновых операторов впервые введён авторами.

\textbf{Определение.}
Эберлейновым назовём такой оператор $A:\ell_\infty \to \ell_\infty$,
что $BA=B$ для некоторого $B\in\mathfrak B$.

Этот класс операторов изучался в работах многих авторов
TODO:cite,
в т.ч. Эберлейна
TODO:cite,
в силу чего авторы и считают возможным дать ему такое название.
%
Всякий эберлейнов оператор является полуэберлейновым по определению.
Обратное неверно.
	Пусть $B_1, B_2$ есть различные крайние точки $ \mathfrak B$
	и $Hx = ((2B_2-B_1) x) \cdot\mathbb I$.
	Тогда оператор $H$ полуэберлейнов,
	но не эберлейнов.

\textbf{Определение.}
В-регулярным назовём такой оператор $H:\ell_\infty \to \ell_\infty$,
что $B H \in \mathfrak B$ для любого $B\in \mathfrak B$.

Этот класс введён Е.А.Алехно в
TODO:cite.
Там же доказано, что всякий В-регулярный является эберлейновым.
Можно доказать, что обратное неверно.
Например, эберлейновым, но не В-регулярным является оператор
$
	Ex = x \cdot \chi_{\cup_{m=0}^{\infty}\left[2^{2 m}+1, 2^{2 m+1}\right] \cup\{1\}}
$.

Заметим, что для любого В-регулярного оператора $H$ и любого $x\in \ell_\infty$ выполнено
$p(x) \geq p(Hx) \geq q(Hx) \geq q(x)$.
Это приводит нас к следующему классу операторов.

\textbf{Определение.} Q-регулярным называется такой оператор $H:\ell_\infty\to \ell_\infty$,
что для любого $x\in\ell_\infty$ выполнено равенство $q(Hx) = q(x)$.


Этот класс в данной работе вводится впервые.
Всякий Q-регулярный оператор является В-регулярным;
обратное неверно.
Например, В-регулярным, но не Q-регулярным является оператор Чезаро (Харди):
$
	C (x_1, x_2, x_3, ...) = \left(
	x_1,
	\dfrac{x_1+x_2}2,
	\dfrac{x_1+x_2 + x_3}3,
	%\dfrac{x_1+x_2+x_3+x_4}4,
	...,
	\dfrac{x_1+...+x_n}n,
	...\right)
	.
$


Примерами Q-регулярных операторов являются операторы $\sigma_k$, $k\ge 2$.
Но даже для таких операторов не все банаховы пределы инвариантны.

\textbf{Определение.}
Разреженным назовём оператор $H: \ell_\infty \to \ell_\infty$,
определяемый следующим образом
$$
H(x_1, x_2, x_3, \ldots) = (x_{m_1}, x_{m_1 + 1}, \ldots, x_{n_1 - 1}, x_{n_1};
x_{m_2}, x_{m_2 + 1}, \ldots, x_{n_2 - 1}, x_{n_2}; \ldots),
$$
где для всех
$k \geqslant k_0 \in \mathbb N$
выполнено
$
	m_k \leqslant n_k < m_{k+1}\leqslant n_{k+1}
$,
при этом
$$
	\lim\limits_{k \to \infty} (n_k - m_k) = \infty
	.
$$

Этот класс в данной работе также вводится впервые.
Всякий разреженный оператор является В-регулярным.
Обратное, очевидно, неверно.
Существует разреженный оператор, не являющийся Q-регулярным.


\setcounter{equation}{0}
\setcounter{figure}{0}

\small
\begin{thebibliography}{99}
	{}
	\bibitem{Mazur}
	S. Mazur, O metodach sumowalno\'sci,
	\emph{Ann. Soc. Polon. Math. (Supplement)}, 1929, 102--107.
	{}
	\bibitem{banach1993theorie}
	S. Banach, Th\'eorie des op\'erations lin\'eaires, Reprint of the 1932 original, Sceaux: \'Editions Jacques Gabay,
	1993, iv+128, \textsc{isbn}: 2-87647-148-5.
	{}
	\bibitem{sucheston1967banach}
	L. Sucheston, Banach limits, \emph{Amer. Math. Monthly} 74 (1967), 308--311, \textsc{issn}: 0002-9890,
	\textsc{doi}: {10.2307/2316038}.
	{}
	\bibitem{lorentz1948contribution}
	G. G. Lorentz, A contribution to the theory of divergent sequences, \emph{Acta Math.} \textbf{80}:~{1}
	(1948), 167--190, \textsc{issn}: 0001-5962, \textsc{doi}: {10.1007/BF02393648}.
	{}
	\bibitem{Semenov2014geomprops}
	Е. М. Семёнов, Ф. А. Сукочев, А. С. Усачев, Геометрические свойства множества банаховых пределов,
	\emph{Известия Российской академии наук. Серия математическая} \textbf{78}:~{3} (2014), 177--204.
	{}
	\bibitem{semenov2010characteristic}
	Е. М. Семенов, Ф. А. Сукочев, Характеристические функции банаховых пределов, \emph{Сибирский
	математический журнал} \textbf{51}:~4 (2010), 904--910.
	{}
	\bibitem{our-mz2019ac0}
	Н. Н. Авдеев, О пространстве почти сходящихся последовательностей, \emph{Математические заметки}
	105.\textbf{3} (2019), 462--466, \textsc{doi}: {10.4213/mzm12298}.
	{}
	\bibitem{avdeev2021vestnik}
	Н. Н. Авдеев, О разделяющих множествах меры нуль и функционалах Сачестона,
	\emph{Вестн. ВГУ. Серия: Физика. Математика} 4 (2021), 38--50.
	{}
	\bibitem{avdeev2021vmzprimes}
	Н. Н. Авдеев, Почти сходящиеся последовательности из 0 и 1 и
	простые числа, \emph{Владикавказский математический журнал} {\bf 23}:~4 (2021), 5--14,
	\textsc{doi}: {10.46698/p9825-1385-3019-c}.
\end{thebibliography}
\normalsize
\end{document}




\bibitem{Pi1}
M.~Hagie. The prime graph of a sporadic simple group. {\it Comm. Algebra}, {\bf 31}:~9 (2003), 4405--4424.

\bibitem{Pi2}
V.~D.~Mazurov, W.~J.~Shi. A note to the characterization of sporadic simple groups. {\it Algebra Colloq.},
{\bf 5}:~3 (1998), 285--288.

\bibitem{Pi3}
P.~Poto\v{c}nik, P.~Spiga, G.~Verret. Cubic vertex--transitive graphs on up to 1280 vertices. arXiv:1201.5317v1 [math.CO].

\bibitem{Pi4}
J.~H.~Conway, R.~T.~Curtis, S.~P.~Norton, R.~A.~Parker, R.~A.~Wilson. Atlas of finite groups.
Oxford: Clarendon Press, 1985.

\bibitem{Pi5}
А.~С.~Кондратьев, Н.~А.~Минигулов. О конечных неразрешимых $4$-примарных $3'$-группах. {\it Тезисы международной конференции "Алгебра, теория чисел и математическое моделирование динамических систем"\,, посвященной 70-летию А.Х. Журтова.} Нальчик: издательство КБГУ, 2019, 56--57.
\end{thebibliography}
\normalsize
\end{document}
