\documentclass[a4paper,14pt]{article}
\usepackage[utf8]{inputenc}
\usepackage[T2A]{fontenc}
\usepackage{float}
\usepackage{cite}
\usepackage[english,russian]{babel}
\usepackage{amssymb,amsmath,amsfonts}
\usepackage{color}
\usepackage{enumerate}
\usepackage[dvips]{graphicx}
\usepackage{setspace}
\usepackage{xcolor}
\usepackage{fancyhdr}

\textheight=220mm
\textwidth=160mm
\oddsidemargin=0.1in
\evensidemargin=0.1in

\begin{document}

\pagestyle{fancy}
\fancyhead{}
\fancyhead[L]{\footnotesize{Современные проблемы математики и ее приложений}}
\fancyhead[R]{\footnotesize{Тезисы конференции}}
\fancyfoot{}
\fancyfoot[L]{\footnotesize{Екатеринбург, Россия}}
\fancyfoot[R]{\footnotesize{}}
\renewcommand{\footrulewidth}{0.1 mm}

\begin{center}
\textbf{О разложении ограниченной последовательности в сумму почти сходящейся к нулю и диадической}%
%TODO:
\footnote{Исследование выполнено за счет гранта Российского научного фонда (проект № 24-21-00220).}\\
\vspace{\baselineskip}
Авдеев Н.Н.\\
\emph{Воронежский госуниверситет, Воронеж, Россия}\\nickkolok@mail.ru, avdeev@math.vsu.ru
\vspace{\baselineskip}\\
\end{center}
\vspace{\baselineskip}

%На коллег надейся, а сам не плошай!
\setcounter{equation}{0}
\setcounter{figure}{0}


Через $\ell_\infty$ обозначим пространство ограниченных последовательностей с обычной нормой
$
	\|x\| = \sup_{k\in\mathbb{N}} |x_k|
	,
$
где $\mathbb{N}$ "--- множество натуральных чисел, и обычной полуупорядоченностью.
Естественным обобщением предела с пространства сходящихся последовательностей $c$ на $\ell_\infty$
является понятие банахова предела.

\textbf{Определение.}
	Линейный функционал $B\in \ell_\infty^*$ есть банахов предел (пишем: $B \in \mathfrak{B}$),
	если
	\begin{enumerate}
		\item
			$B\geq0$, т.~е. $Bx \geq 0$ для $x \geq 0$,
		\item
			$B\mathbb{I}=1$, где $\mathbb{I} =(1,1,\ldots)$,
		\item
			$B(Tx)=B(x)$ для всех $x\in \ell_\infty$, где $T$~---
			оператор сдвига, т.~е. $T(x_1,x_2,\ldots)=(x_2,x_3,\ldots)$.
	\end{enumerate}

Существование банаховых пределов было анонсировано С. Мазуром~\cite{Mazur} и доказано С.~Банахом~\cite{banach1993theorie}.

\textbf{Критерий Лоренца.}~\cite{lorentz1948contribution}
{\sl
	Для заданных $t\in\mathbb R$ и $x\in\ell_\infty$ равенство $Bx=t$ выполнено для всех $B\in\mathfrak{B}$
	(для краткости пишут: $x\in ac$ и при необходимости уточняют: $x\in ac_t$)
	тогда и только тогда, когда
	\begin{equation*}
		\label{eq:crit_Lorentz}
		\lim_{n\to\infty} \frac{1}{n} \sum_{k=m+1}^{m+n} x_k = t
		\qquad\qquad\mbox{равномерно по $m\in\mathbb N$.}
	\end{equation*}
}

Равномерный предел в критерии Лоренца можно заменить на двойной~\cite[Теорема 1]{zvol2022ac}.


Представляет интерес вопрос об объектах, в некотором смысле двойственных пространству $ac$.
Множество $Q\in\ell_\infty$ называют разделяющим~\cite[\S 3]{Semenov2014geomprops}, если
для любых неравных $B_1, B_2\in\mathfrak{B}$ существует такая последовательность $x\in Q$,
что $B_1 x \neq B_2 x$.
В частности, разделяющим является~\cite{semenov2010characteristic} множество всех последовательностей из 0 и 1.

Очень часто при изучении банаховых пределов и смежных вопросов
рассматриваются последовательности, например, состоящие только из 0 и 1
(см., например,~\cite{connor1990almost,our-mz2019ac0,avdeev2021vestnik,avdeev2021vmzprimes}).


Ниже мы конструктивно докажем результат,
в некотором смысле обосновывающий такой подход.


\medskip

\textbf{Лемма 1.}
{\sl
	Пусть $x\in\ell_\infty$, $\|x\|\leq 1$.
	Тогда существует такая $h\in ac_0$, что $(x+h)_n = \pm 1$ для всех $n\in\mathbb N$.
}

\textbf{Доказательство.}
Построим последовательность $h$ согласно следующим соотношениям.
Положим $h_1 = 1-x_1$.
Обозначим $s_n = \sum_{k=1}^n h_k$. Будем полагать

$$
	h_k = \begin{cases}
		1-x_k, \quad\mbox{если}~~ s_{k-1} < 0,
		\\
		-1 - x_k,\quad\mbox{если}~~ s_{k-1} \geq 0
		.
	\end{cases}
$$

Тогда $|h_k| \leq 2$ и, более того, по индукции следует, что $|s_k| \leq 2$. Осталось применить критерий Лоренца к последовательности $h_k$:

\begin{multline*}
	\left|\lim_{m,n\to \infty} \frac{1}{n} \sum_{k=m+1}^{m+n} h_k\right| \leq
	\lim_{m,n\to \infty} \frac{1}{n} \left|\sum_{k=m+1}^{m+n} h_k\right| \leq
	\lim_{m,n\to \infty} \frac{1}{n} \left|\sum_{k=1}^{m+n} h_k - \sum_{k=1}^{m} h_k  \right| =
	\\=
	\lim_{m,n\to \infty} \frac{1}{n} \left|s_{m+n} - s_{m}  \right| \leq
	\lim_{m,n\to \infty} \frac{1}{n} (|s_{m+n}| + |s_{m} |) \leq
	\lim_{m,n\to \infty} \frac{1}{n} (2 + 2) =0
\end{multline*}

\medskip

\textbf{Следствие 1.} {\sl
	Пусть $x\in\ell_\infty$.
	Тогда существует такая $h\in ac_0$, что для всех $n\in\mathbb N$
	\begin{equation*}
		(x+h)_n \in \{\inf_{n\in\mathbb N} x_n,\sup_{n\in\mathbb N} x_n\}
		.
	\end{equation*}
}

\medskip

\textbf{Следствие 2.} {\sl
	Пусть $x\in\ell_\infty$.
	Тогда существует такая $h\in ac_0$, что $(x+h)_n =\pm \|x\|$ для всех $n\in\mathbb N$.
}

\medskip

Полученный результат интересен в первую очередь тем, что для банаховых пределов и связанных с ними понятий
очень редки теоремы о приближении или о декомпозиции.
Более того, из полученного результата непосредственно следует, например,
доказанное в~\cite{semenov2010characteristic} утверждение о том, что множество всех последовательностей из 0 и 1 является разделяющим.

Предложенная выше процедура, однако, в общем случае не сохраняет последовательность из $ac_0$,
а превращает её в сумму двух последовательностей совсем другой структуры.
%Этот эффект, по всей видимости, связан с тем, что пространство $ac_0$ недополняемо в $\ell_\infty$.



\setcounter{equation}{0}
\setcounter{figure}{0}

\small
\begin{thebibliography}{99}
	{}
	\bibitem{Mazur}
	S. Mazur, O metodach sumowalno\'sci,
	\emph{Ann. Soc. Polon. Math. (Supplement)}, 1929, 102--107.
	{}
	\bibitem{banach1993theorie}
	S. Banach, Th\'eorie des op\'erations lin\'eaires, Reprint of the 1932 original, Sceaux: \'Editions Jacques Gabay,
	1993, iv+128, \textsc{isbn}: 2-87647-148-5.
	{}
	%\bibitem{sucheston1967banach}
	%L. Sucheston, Banach limits, \emph{Amer. Math. Monthly} 74 (1967), 308--311, \textsc{issn}: 0002-9890,
	%\textsc{doi}: {10.2307/2316038}.
	%{}
	\bibitem{lorentz1948contribution}
	G. G. Lorentz, A contribution to the theory of divergent sequences, \emph{Acta Math.} \textbf{80}:~{1}
	(1948), 167--190, \textsc{issn}: 0001-5962, \textsc{doi}: {10.1007/BF02393648}.
	{}
	\bibitem{zvol2022ac}
	Р. Е. Зволинский,
	Почти сходящиеся последовательности и операция возведения в положительную степень,
	\emph{Вестник ВГУ. Серия: Физика. Математика.} 3 (2022), 76--81.
	{}
	\bibitem{Semenov2014geomprops}
	Е. М. Семёнов, Ф. А. Сукочев, А. С. Усачев, Геометрические свойства множества банаховых пределов,
	\emph{Известия Российской академии наук. Серия математическая} \textbf{78}:~{3} (2014), 177--204.
	{}
	\bibitem{semenov2010characteristic}
	Е. М. Семенов, Ф. А. Сукочев, Характеристические функции банаховых пределов, \emph{Сибирский
	математический журнал} \textbf{51}:~4 (2010), 904--910.
	{}
	\bibitem{connor1990almost}
	J. Connor, Almost none of the sequences of 0’s and 1’s are almost
	convergent.
	\emph{International Journal of Mathematics and Mathematical Sciences}, \textbf{13}:~4 (1990), 775--777.
	{}
	\bibitem{our-mz2019ac0}
	Н. Н. Авдеев, О пространстве почти сходящихся последовательностей, \emph{Математические заметки}
	105.\textbf{3} (2019), 462--466, \textsc{doi}: {10.4213/mzm12298}.
	{}
	\bibitem{avdeev2021vestnik}
	Н. Н. Авдеев, О разделяющих множествах меры нуль и функционалах Сачестона,
	\emph{Вестн. ВГУ. Серия: Физика. Математика} 4 (2021), 38--50.
	{}
	\bibitem{avdeev2021vmzprimes}
	Н. Н. Авдеев, Почти сходящиеся последовательности из 0 и 1 и
	простые числа, \emph{Владикавказский математический журнал} {\bf 23}:~4 (2021), 5--14,
	\textsc{doi}: {10.46698/p9825-1385-3019-c}.
\end{thebibliography}
\normalsize
\end{document}




\bibitem{Pi1}
M.~Hagie. The prime graph of a sporadic simple group. {\it Comm. Algebra}, {\bf 31}:~9 (2003), 4405--4424.

\bibitem{Pi2}
V.~D.~Mazurov, W.~J.~Shi. A note to the characterization of sporadic simple groups. {\it Algebra Colloq.},
{\bf 5}:~3 (1998), 285--288.

\bibitem{Pi3}
P.~Poto\v{c}nik, P.~Spiga, G.~Verret. Cubic vertex--transitive graphs on up to 1280 vertices. arXiv:1201.5317v1 [math.CO].

\bibitem{Pi4}
J.~H.~Conway, R.~T.~Curtis, S.~P.~Norton, R.~A.~Parker, R.~A.~Wilson. Atlas of finite groups.
Oxford: Clarendon Press, 1985.

\bibitem{Pi5}
А.~С.~Кондратьев, Н.~А.~Минигулов. О конечных неразрешимых $4$-примарных $3'$-группах. {\it Тезисы международной конференции "Алгебра, теория чисел и математическое моделирование динамических систем"\,, посвященной 70-летию А.Х. Журтова.} Нальчик: издательство КБГУ, 2019, 56--57.
\end{thebibliography}
\normalsize
\end{document}
