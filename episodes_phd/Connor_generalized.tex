Пусть на множестве $\Omega=\{0,1\}^\mathbb{N}$ задана вероятностная мера <<честной монетки>> $\mu$.
Тогда, согласно~\cite{connor1990almost}, $\mu(\Omega\cap ac)=0$.

Обобщим этот результат.
\begin{theorem}
	Мера множества $F=\{x\in\Omega : q(x) = 0 \wedge p(x)= 1\}$,
	где $p(x)$ и $q(x)$~--- верхний и нижний функционалы Сачестона соответственно,
	равна 1.
\end{theorem}

\begin{proof}
	Пусть $F_1=\{x\in\Omega : p(x) \neq 1\}$, $F_0=\{x\in\Omega : q(x) \neq 0\}$.
	Заметим, что
	\begin{equation}
		\label{eq:Connor_gen}
		\mu F = \mu (\Omega\setminus(F_1 \cup F_0)) = 1 - \mu (F_1 \cup F_0)
		.
	\end{equation}
	Докажем, что $\mu F_1 = 0 $	(для $\mu F_0$  доказательство полностью аналогично).

	Согласно критерию Сачестона,
	\begin{equation}
		p(x) = \lim_{n\to\infty} \sup_{m\in\mathbb{N}} \frac{1}{n} \sum_{k=m+1}^{k=m+n} x_k
		,
	\end{equation}
	откуда следует, что
	для $x\in\Omega$ равенство $p(x) = 1$ выполнено тогда и только тогда,
	когда для любого $n$ последовательность $x$ содержит отрезок из $n$ единиц подряд.
%
	Следовательно, если $x\in F_1$,
	то существует такое $n$,
	что $x$ не содержит $n$ единиц подряд.
%
%
	Обозначим
	\begin{equation}
		A_n^k = \{x\in\Omega : x_{kn+1} = ... = x_{kn+n} = 1\}
		,
		~~~
		B_n^k = \Omega \setminus A_n^k
		%,
		%B_n = \bigcap_{k\in\mathbb{N}} B_n^k
		.
	\end{equation}
	Тогда
	\begin{equation}
		\forall(x\in F_1)\exists(n_x\in\mathbb{N})\forall(k\in\mathbb{N})[x\in B_{n_x}^k]
		,
	\end{equation}
	т.е.
	\begin{equation}
		F_1 \subset \bigcup_{n\in\mathbb{N}} \bigcap_{k\in\mathbb{N}} B_{n}^k
		.
	\end{equation}
	Учитывая, что при $i\neq j$ события $x\in B_n^i$ и $x\in B_n^j$ независимы и $\mu B_n^j = 1-\frac{1}{2^n}$,
	получаем
	\begin{multline}
		\mu F_1 \leq \mu \bigcup_{n\in\mathbb{N}} \bigcap_{k\in\mathbb{N}} B_{n}^k
		=
		\sum_{n\in\mathbb{N}} \mu \bigcap_{k\in\mathbb{N}} B_{n}^k
		=
		\sum_{n\in\mathbb{N}}  \prod_{k\in\mathbb{N}} \mu B_{n}^k
		=
		\sum_{n\in\mathbb{N}}  \prod_{k\in\mathbb{N}} \left( 1-\frac{1}{2^n} \right)
		=0
		.
	\end{multline}
	Тем самым получаем из \eqref{eq:Connor_gen}, что $\mu F = 1$.
\end{proof}
