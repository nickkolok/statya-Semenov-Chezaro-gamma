\begin{theorem}
	\label{thm:Lin_Omega_Sucheston}
	Пусть
	$1 \geq a > b \geq 0$ и
	$\Omega^a_b = \{x\in\Omega : p(x) = a, q(x) = b\}$,
	где $p(x)$ и $q(x)$~--- верхний и нижний функционалы Сачестона~\cite{sucheston1967banach} соответственно.
	Тогда $\Omega \subset \operatorname{Lin} \Omega^a_b$.
\end{theorem}

\begin{proof}
	Выберем $n\in\N$ таким образом, что
	\begin{equation}
		\label{eq:Omega_a_b_gap}
		a - b > \frac{3}{2^n}
	\end{equation}
	и $n$ чётно.

	Очевидно, что существует разложение
	\begin{equation}
		x = \sum_{i=0}^{k-1} T^i x_i, \quad x_i \in \Omega
		,
	\end{equation}
	где $k\in\N$ и все элементы последовательностей $x_i$,
	кроме имеющих индексы $km+1$, $m\in\N_0$, являются нулевыми.
	Пусть $k=2^n$; зафиксируем $i$ и в дальнейшем для удобства записи положим $w=x_i$.
	Наша задача~--- построить конечную линейную комбинацию элементов из $\Omega^a_b$, равную $w$.
	%
	Положим
	%\begin{equation}
	%	v = T^{2^n} \operatorname{Br}(2^ n   ,a) + T^{2^{n+1}} \operatorname{Br}(2^{n+1},b) +
	%	T^{2^{n+2}} \operatorname{Br}(2^{n+2},a) + T^{2^{n+3}} \operatorname{Br}(2^{n+3},b) + ...
	%\end{equation}
	%Эта сумма является формальной, т.е. не сходится в смысле ряда по норме,
	%однако носители слагаемых попарно не пересекаются.
	%Иначе говоря,
	\begin{equation}
		v_j = \begin{cases}
			0,  & \mbox{~если~} j \leq 2^n,
			\\
			(\operatorname{Br}(2^{2k  },a))_{j-2^{2k  }},  & \mbox{~если~} 2^{2k  } < j \leq 2^{2k+1}, 2k   \geq n,
			\\
			(\operatorname{Br}(2^{2k+1},b))_{j-2^{2k+1}},  & \mbox{~если~} 2^{2k+1} < j \leq 2^{2k+2}, 2k+1 \geq n
			.
		\end{cases}
	\end{equation}
	Иначе говоря, сначала <<резервируется>> $2^n$ нулевых элементов
	(большей частью для удобства записи, поскольку конечное количество членов в начале последовательности
	не влияет на функционалы Сачестона),
	а затем по очереди приписываются блоки "--- от первого элемента (нулевого) до последнего ненулевого элемента
	(конца носителя).
	Положим далее
	\begin{equation}
		u_j = \begin{cases}
			v_j + w_j,  & \mbox{~если~} j \leq 2^n
			\\
			            & \mbox{~или~} 2^{4k+3} < j \leq 2^{4k+4} \mbox{~и~} 4k + 3 \geq n,
			\\
			v_j         & \mbox{~для остальных~} j
			.
		\end{cases}
	\end{equation}

	В силу утверждения~\ref{prop:Br_has_nulls} все элементы, к которым прибавляются ненулевые элементы $w_j$, равны нулю.
	Кроме того, с учётом леммы~\ref{lem:sum_Br_k_c} и утверждения~\ref{prop:Sucheston_partial_limit}
	имеем $p(u)=p(w)=a$ и $q(u)=q(w)=b$.
	(На <<возмущённом>> блоке $u$ среднее, соответствующее функционалу $q$,
	увеличивается не более чем на $2^{-n}$ и не влияет на значение функционала $p$,
	в силу условия~\eqref{eq:Omega_a_b_gap}.)
	Следовательно, $u,v\in\Omega^a_b$.
	Заметим теперь, что
	\begin{equation}
		(u-v)_j = \begin{cases}
			w_j,  & \mbox{~если~} j \leq 2^n,
			\\
			0,  & \mbox{~если~} 2^{2k  } < j \leq 2^{2k+1}, 2k    \geq n,
			\\
			0,  & \mbox{~если~} 2^{4k+1} < j \leq 2^{4k+2}, 4k + 1 \geq n,
			\\
			w_j,  & \mbox{~если~} 2^{4k+3} < j \leq 2^{4k+4}, 4k + 3 \geq n
			.
		\end{cases}
	\end{equation}

	Аналогично строятся пары элементов, разность которых равна $w_j$ на $2^{4k+i} < j \leq 2^{4k+i+1}, 4k + i \geq n$ для $i=0,1,2$
	(требуется только обнулить первые $2^n$ элементов).
	Складывая полученные таким образом $4\cdot 2^n$ разностей элементов из $\Omega^a_b$, получаем требуемый элемент $x$.

\end{proof}

\begin{corollary}
	\label{crl:Lin_Omega_Sucheston}
	Множество $\Omega^a_b$ является разделяющим.
	Т.к. при $a\neq 1$ или $b\neq 0$ множество $\Omega^a_b$ имеет меру нуль~\cite{semenov2010characteristic,connor1990almost},
	то оно является разделяющим множеством нулевой меры.
\end{corollary}
