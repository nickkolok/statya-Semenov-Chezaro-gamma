Пользуясь критерием Лоренца~\eqref{eq:crit_Lorentz},
нетрудно доказать, что для $x_n = m^n$, $m\in\N$, $m\geq 2$ выполнено $\chi x \in ac_0$.

\begin{lemma}
	Пусть $y = \{y_n\}$ --- строго возрастающая последовательность,
	$\chi y\in\Omega \cap ac_0$.
	Пусть $m \in \N$
	и последовательность $x=\{x_k\}$ определена соотношением
	\begin{equation}
		x_k = \begin{cases}
			1, &\mbox{~если~} k = y_i \cdot m^j \mbox{~для некоторых~} i,j\in\N,
			\\
			0  &\mbox{~иначе}
			.
		\end{cases}
	\end{equation}
	Тогда $x\in ac_0$.
\end{lemma}

\begin{proof}
	Зафиксируем некоторое $K \in \N$ и покажем, что $p(x) < m^{-K}$.
	Действительно, представим $x$ в виде суммы
	\begin{equation}
		\label{eq:ac0_powers_x_as_sum}
		x \leq z_1 + z_2 + \dots + z_K + z'_{K+1}
		,
	\end{equation}
	где каждое слагаемое $z_j$ соответствует умножению индексов на $m^j$:
	\begin{equation}
		(z_j)_k = \begin{cases}
			1, &\mbox{~если~} k = y_i \cdot m^j \mbox{~для некоторого~} i\in\N,
			\\
			0  &\mbox{~иначе}
			,
		\end{cases}
	\end{equation}
	а слагаемое $z'_j$ соответствует умножению индексов на $m^{j+1}$, $m^{j+2},...$:
	\begin{equation}
		\label{eq:ac0_powers_residue}
		(z'_j)_k = \begin{cases}
			1, &\mbox{~если~} k = y_i \cdot m^{j'} \mbox{~для некоторых~} i,j'\in\N,~~ j' > j
			,%\\ &\mbox{~и ни для какого~} j' \leq j \mbox{~не выполнено~} k = y_i \cdot m^{j'},
			\\
			0  &\mbox{~иначе}
			.
		\end{cases}
	\end{equation}
	(Знак неравенства в~\eqref{eq:ac0_powers_x_as_sum} возникает ввиду того, что возможен случай
	$y_i \cdot m^j = y_{i'} \cdot m^{j'}$ для $j\neq j'$.
	Например, в случае $y_1 = 3$, $y_2 = 6$ и $m=2$ имеем $y_1 \cdot m^2 = y_2 \cdot m^1$.)
	Понятно, что $p(z_j)=0$.
	Таким образом,
	\begin{equation}
		p(x) \leq p(z_1) + p(z_2) + \dots + p(z_K) + p(z'_{K+1}) = p(z'_{K+1})
		.
	\end{equation}
	Заметим, что в силу определения~\eqref{eq:ac0_powers_residue} каждый отрезок $z'_j$ из $m^{j+1}$ элементов
	содержит не более одной единицы, и потому $p(z'_j) \leq m^{-j-1} < m^{-j}$.
	Таким образом, для любого $K\in\N$ выполнена оценка $p(x) < m^{-K}$,
	откуда $p(x) = 0$.
\end{proof}

\begin{corollary}
	\label{cor:ac0_powers_finite_set_of_numbers}
	Пусть $\{p_1, ..., p_k\} \subset \N$,
	\begin{equation}
		x_k = \begin{cases}
			1, &\mbox{~если~} k = p_1^{j_1}\cdot p_2^{j_2}\cdot ... \cdot p_k^{j_k} \mbox{~для некоторых~} j_1,...,j_k\in\N,
			\\
			0  &\mbox{~иначе}.
		\end{cases}
	\end{equation}
	Тогда $x\in ac_0$.
\end{corollary}


%\begin{hypothesis}
%	Пусть $y=\{y_n\}$ и $z=\{z_n\}$ --- строго возрастающие последовательности,
%	$\chi_y,\chi_z\in\{0;1\}^\N \cap ac_0$.
%	Тогда почти сходится к нулю последовательность $x=\{x_k\}$, определённая соотношением
%	\begin{equation}
%		x_k = \begin{cases}
%			1, &\mbox{~если~} k = y_i \cdot z_j \mbox{~для некоторых~} i,j\in\N,
%			\\
%			0  &\mbox{~иначе}
%			.
%		\end{cases}
%	\end{equation}
%\end{hypothesis}
