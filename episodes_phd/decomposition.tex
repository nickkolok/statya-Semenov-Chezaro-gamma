Очень часто при изучении банаховых пределов и смежных вопросов
рассматриваются последовательности,
элементы которых могут принимать только два значения
(см., например,~\cite{connor1990almost,our-mz2019ac0,avdeev2021vestnik,avdeev2021vmzprimes}).


Мы начнём с того,
что конструктивно докажем лаконичный результат,
в некотором смысле обосновывающий такой подход.


\begin{lemma}
	Пусть $x\in\ell_\infty$, $\|x\|\leq 1$.
	Тогда существует такая $h\in ac_0$, что $(x+h)_n = \pm 1$ для всех $n\in\mathbb N$.
\end{lemma}

\begin{proof}
	Построим последовательность $h$ согласно следующим соотношениям.
	Положим $h_1 = 1-x_1$.
	Обозначим $s_n = \sum_{k=1}^n h_k$. Будем полагать

	$$
		h_k = \begin{cases}
			1-x_k, \quad\mbox{если}~~ s_{k-1} < 0,
			\\
			-1 - x_k,\quad\mbox{если}~~ s_{k-1} \geq 0
			.
		\end{cases}
	$$

	Тогда $|h_k| \leq 2$ и, более того, по индукции нетрудно показать, что $|s_k| \leq 2$. Осталось применить критерий Лоренца к последовательности $h_k$:

	\begin{multline*}
		\left|\lim_{m,n\to \infty} \frac{1}{n} \sum_{k=m+1}^{m+n} h_k\right| \leq
		\lim_{m,n\to \infty} \frac{1}{n} \left|\sum_{k=m+1}^{m+n} h_k\right| \leq
		\lim_{m,n\to \infty} \frac{1}{n} \left|\sum_{k=1}^{m+n} h_k - \sum_{k=1}^{m} h_k  \right| =
		\\=
		\lim_{m,n\to \infty} \frac{1}{n} \left|s_{m+n} - s_{m}  \right| \leq
		\lim_{m,n\to \infty} \frac{1}{n} (|s_{m+n}| + |s_{m} |) \leq
		\lim_{m,n\to \infty} \frac{1}{n} (2 + 2) =0
	\end{multline*}
\end{proof}

\begin{corollary}
	Пусть $x\in\ell_\infty$.
	Тогда существует такая $h\in ac_0$, что для всех $n\in\mathbb N$
	\begin{equation*}
		(x+h)_n \in \{\inf_{n\in\mathbb N} x_n,\sup_{n\in\mathbb N} x_n\}
		.
	\end{equation*}
\end{corollary}


\begin{corollary}
	Пусть $x\in\ell_\infty$.
	Тогда существует такая $h\in ac_0$, что $(x+h)_n =\pm \|x\|$ для всех $n\in\mathbb N$.
\end{corollary}

Полученный результат интересен в первую очередь тем, что для банаховых пределов и связанных с ними понятий
очень редки теоремы о приближении или о декомпозиции.
Более того, из полученного результата непосредственно следует, например,
доказанное в~\cite{semenov2010characteristic} утверждение о том, что множество всех последовательностей из 0 и 1 является разделяющим.
(Приведённое в~\cite{semenov2010characteristic} доказательство достаточно длинное.)

Предложенная выше процедура, однако, в общем случае не сохраняет последовательность из $ac_0$,
а превращает её в сумму двух последовательностей совсем другой структуры.
