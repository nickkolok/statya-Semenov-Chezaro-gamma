\begin{lemma}
	Для любого непустого $A\subset \mathbb{N} $ выполнено $\chi\mathscr{M}A \notin ac_0$.
\end{lemma}
\begin{proof}
	Пусть $a_1\in A$.
	Тогда из каждых идущих подряд $a_1$ элементов последовательности $\chi\mathscr{M}A$
	хотя бы один равен единице,
	следовательно,
	\begin{equation}
		q(\chi\mathscr{M}A) \geq \frac{1}{a_1} > 0
		.
	\end{equation}
\end{proof}

При дополнительных ограничениях верно и более сильное утверждение.

\begin{theorem}
	\label{lem:ac0_primes_infinity_mutually_prime_subset}
	Пусть $A'$ "--- бесконечное подмножество попарно взаимно простых чисел
	(т.е. для любых двух чисел $a_1, a_2 \in A'$ их наибольший общий делитель равен единице).
	Тогда для любого $A \supset A' $ выполнено $p(\chi\mathscr{M}A)=1$.
\end{theorem}
\begin{proof}
	Пусть $A' = \{ a_1, a_2, ..., a_j, ... \}$ и
	\begin{equation}
		\label{eq:ac0_primes_A_j_prod_des}
		A_j = \prod_{i=1}^j a_i
		.
	\end{equation}

	Для каждого $k$ найдём такие номера $n_k$, что
	\begin{equation}
		(\chi\mathscr{M}A)_{n_k+1} = (\chi\mathscr{M}A)_{n_k+2} = \dots = (\chi\mathscr{M}A)_{n_k+k} = 1
		.
	\end{equation}
	Тем самым мы докажем, что отрезок из любого наперёд заданного количества единиц подряд
	встречается в последовательности $\chi\mathscr{M}A$ и, следовательно, $p(\chi\mathscr{M}A) = 1$.

	Действительно,
	пусть $n_1 = a_1 - 1$.
	Рассмотрим множество  $F_1 = \{ n_1 + A_1, n_1 + 2A_1, n_1 + 3A_1, \dots, n_1 + a_2A_1 \}$
	и отметим два следующих факта.

	Во-первых, пусть $f \in F_1$,
	тогда
	\begin{equation}
		f \equiv n_1 \mod a_1
		.
	\end{equation}
	Во-вторых, числа $a_2$ и $A_1$ взаимно просты.
	Следовательно, все $a_2$ чисел из множества $F_1$ дают разные остатки при делении на $a_2$.

	В качестве $n_2$ возьмём такое $f\in F_1$, что
	\begin{equation}
		f \equiv a_2 - 2 \mod a_2
		.
	\end{equation}
	Заметим, что тогда
	\begin{equation}
		n_2 + 1 \equiv n_1 + 1 \equiv 0 \mod a_1
	\end{equation}
	и
	\begin{equation}
		n_2 + 2 \equiv 0 \mod a_2
		,
	\end{equation}
	следовательно,
	$(\chi\mathscr{M}A)_{n_2 + 1} = (\chi\mathscr{M}A)_{n_2 + 2} = 1$.

	Полученные рассуждения несложно продолжить по индукции.

	Рассмотрим множество  $F_{j} = \{ n_j + A_j, n_j + 2A_j, n_j + 3A_j, \dots, n_j + a_{j+1}A_j \}$
	и отметим два следующих факта.

	Во-первых, пусть $f \in F_{j}$,
	тогда
	\begin{equation}
		f \equiv n_j \mod A_j
	\end{equation}
	и, в силу~\eqref{eq:ac0_primes_A_j_prod_des},
	\begin{equation}
		\begin{array}{rl}
		f &\equiv n_j \mod a_1
		\\
		f &\equiv n_j \mod a_2
		\\
		&\dots
		\\
		f &\equiv n_j \mod a_j
		.
		\end{array}
	\end{equation}
	Во-вторых, числа $a_{j+1}$ и $A_j$ взаимно просты, поскольку $a_{j+1}$ взаимно просто с каждым из чисел $a_1,...,a_j$.
	Следовательно, все $a_{j+1}$ чисел из множества $F_j$ дают разные остатки при делении на $a_{j+1}$.

	В качестве $n_{j+1}$ возьмём такое $f\in F_j$, что
	\begin{equation}
		f \equiv a_{j+1} - (j+1) \mod a_{j+1}
		.
	\end{equation}
	Заметим, что тогда
	\begin{equation}
		\begin{array}{l}
			n_{j+1} + 1 \equiv n_j + 1 \equiv n_{j-1} + 1 \equiv \dots \equiv n_2 + 1 \equiv n_1 + 1 \equiv 0 \mod a_1
			\\
			n_{j+1} + 2 \equiv n_j + 2 \equiv n_{j-1} + 2 \equiv \dots \equiv n_2 + 2 \equiv 0 \mod a_2
			\\
			\dots
			\\
			n_{j+1} + j \equiv n_j + j  \equiv 0 \mod a_j
		\end{array}
	\end{equation}
	и
	\begin{equation}
		n_{j+1} + (j+1) \equiv 0 \mod a_{j+1}
		,
	\end{equation}
	следовательно,
	\begin{equation}
		(\chi\mathscr{M}A)_{n_{j+1} + j+1} = (\chi\mathscr{M}A)_{n_{j} + j} =
		\dots = (\chi\mathscr{M}A)_{n_{j} + 2} = (\chi\mathscr{M}A)_{n_{j} + 1} = 1
		.
	\end{equation}


\end{proof}

\begin{remark}
	Понятно, что в качестве примера бесконечного множества
	попарно взаимно простых чисел можно взять любое бесконечное множество простых чисел.
	Однако бывают бесконечные множества попарно взаимно простых чисел,
	не содержащие простых чисел вовсе, например множество
	\begin{equation}
		A = \{ 2\cdot 3,~5 \cdot 7,~11 \cdot 13,~17\cdot 19,~23\cdot29,~31\cdot 37,~...\},
	\end{equation}
	элементами которого являются произведения пар соседних простых чисел.
\end{remark}

\begin{definition}
	Будем говорить, что множество $A\subset\mathbb{N}$ обладает $P$-свойством,
	если для любого $n\in\mathbb{N}$ найдётся набор попарно взаимно простых чисел
	\begin{equation}
		\{a_{n,1}, a_{n,2}, ..., a_{n,n}  \} \subset A
		.
	\end{equation}
\end{definition}

Из доказательства теоремы~\ref{lem:ac0_primes_infinity_mutually_prime_subset} понятно,
что для множества $A$ в условии теоремы достаточно потребовать $P$-свойства.
Интересно, что на самом деле $P$-свойство эквивалентно условиям,
наложенным на множество $A$ в теореме~\ref{lem:ac0_primes_infinity_mutually_prime_subset}.

\begin{lemma}
	\label{lem:ac0_primes_infinity_mutually_prime_subset_equiv_to_P_property}
	Пусть множество $A$ обладает $P$-свойством.
	Тогда существует бесконечное подмножество $A'\subset A$ попарно взаимно простых чисел.
\end{lemma}
\begin{proof}
	Зафиксируем $f_0\in A$, $f_0 \neq 1$ и представим $A$ в виде объединения трёх попарно непересекающихся множеств:
	\begin{equation}
		A = \{f_0\} \cup E \cup F
		,
	\end{equation}
	где
	\begin{alignat*}{4}
		& E &&= \{ a \in A \mid a \mbox{~не~}&&\mbox{взаимно просто с~} f_0 \mbox{~и~} a\neq f_0\}
		,
		\\
		& F &&= \{ a \in A \mid a            &&\mbox{взаимно просто с~} f_0 \}
		.
	\end{alignat*}

	Пусть разложение $f_0$ на простые множители имеет вид
	\begin{equation}
		f_0 = p_1^{j_1} \cdot p_2^{j_2} \cdot ... \cdot p_k^{j_k}
		.
	\end{equation}

	Тогда множество $E$ можно представить в виде объединения (возможно пересекающихся) множеств:
	\begin{equation}
		E = \bigcup_{i=1}^{k} E_i,\quad E_i = \{a\in E \mid a \mbox{~кратно~} p_i\}
		.
	\end{equation}

	Покажем, что множество $F$ обладает $P$-свойством.
	Действительно, зафиксируем $n\in\mathbb{N}$.
	Так как множество $A$ обладает $P$-свойством,
	то в нём найдётся подмножество попарно взаимно простых чисел
	$$G=\{a_1, a_2, ..., a_{n+k-1}, a_{n+k}\}\subset A.$$

	Если $f_0\in G$, то $G\setminus f_0 \subset F$ в силу построения множеств $G$ и $F$, и требуемый набор попарно взаимно простых чисел предъявлен.

	Пусть теперь $f_0\notin G$.
	Заметим, что в каждое из множеств $E_i$ может входить не более одного элемента множества $G$
	в силу того, что при фиксированном $i$ все элементы множества $E_i$ имеют нетривиальный общий делитель.
	Следовательно, как минимум $n$ элементов из $G$ принадлежат множеству $F$,
	и требуемый набор попарно взаимно простых чисел снова предъявлен.

	%Поскольку множество $F$ обладает $P$-свойством, то оно, очевидно, счётно.

	Итак, нам удалось получить число $f_0\in A$ и бесконечное множество $F$, обладающее $P$-свойством
	и состоящее из чисел, взаимно простых с $f_0$.
	Продолжая по индукции, получим требуемое бесконечное множество попарно взаимно простых чисел.
\end{proof}

%Условие теоремы~\ref{lem:ac0_primes_infinity_mutually_prime_subset}
%является не только достаточным, но и необходимым.

\begin{lemma}
	\label{lem:ac0_primes_q_psi_A_0_causes_P}
	Пусть для множества $A\subset\mathbb{N}\setminus\{1\}$ выполнено $p(\chi\mathscr{M}A)=1$.
	Тогда $A$ обладает $P$-свойством.
\end{lemma}

\begin{proof}
	Предположим противное: пусть $A$ не обладает $P$-свойством.
	Тогда существует $n\in\mathbb{N}$ такое, что из множества $A$ можно выбрать $n$
	попарно взаимно простых чисел, но нельзя выбрать $n+1$.

	Пусть $\{a_1, a_2, ..., a_n\}\subset A$ "--- набор попарно взаимно простых чисел.
	Так как $p(\chi\mathscr{M}A)=1$, то в последовательности $\chi\mathscr{M}A$ найдётся отрезок, состоящий сплошь из единиц,
	любой наперёд заданной длины.
	Выберем $k$ таким, что
	\begin{equation}
		(\chi\mathscr{M}A)_{k+1} = (\chi\mathscr{M}A)_{k+2} = ... = (\chi\mathscr{M}A)_{k+a_1a_2\cdots a_n} = 1
		.
	\end{equation}
	Тогда существует такое число $k_0$, $k+1 \leq k_0 \leq k+a_1a_2\cdots a_n$,
	что $k_0$ даёт в остатке $1$ при делении на $a_1a_2\cdots a_n$.
	Так как $(\chi\mathscr{M}A)_{k_0} = 1$, то $k_0 = m\cdot a_0$ для некоторых $m\in\mathbb{N}$ и $a_0\in A$.
	С другой стороны, $k_0$ взаимно просто с каждым из чисел $a_1, a_2, ..., a_n$.
	Следовательно, $a_0$ также взаимно просто с каждым из чисел $a_1, a_2, ..., a_n$,
	и $\{a_0, a_1, a_2, ..., a_n\}\subset A$ "--- набор из $n+1$ попарно взаимно простых чисел.
	Полученное противоречие завершает доказательство.
\end{proof}

Таким образом,
из теоремы~\ref{lem:ac0_primes_infinity_mutually_prime_subset}
и лемм~\ref{lem:ac0_primes_infinity_mutually_prime_subset_equiv_to_P_property},~\ref{lem:ac0_primes_q_psi_A_0_causes_P}
незамедлительно следует
\begin{theorem}
	Пусть $A\subset \mathbb{N}\setminus\{1\}$.
	Тогда следующие условия эквивалентны:
	\begin{enumerate}[label=(\roman*)]
		\item
			$A$ обладает $P$-свойством
		\item
			В $A$ существует бесконечное подмножество попарно взаимно простых чисел
		\item
			$p(\chi\mathscr{M}A)=1$.
	\end{enumerate}
\end{theorem}
