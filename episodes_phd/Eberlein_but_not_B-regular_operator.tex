\begin{theorem}
	\label{thm:Eberlein_but_not_B-regular_exists}
	Существует эберлейнов оператор, не являющийся В-регулярным.
\end{theorem}

\begin{proof}
	Пусть оператор $A$ определён формулой~\eqref{eq:oper_A_throws_out_2power_blocks}.
	Пусть оператор $E:\ell_\infty\to\ell_\infty$ определён формулой
	\begin{equation}
		Ex = x \cdot \chi{\cup_{m=0}^{\infty}\left[2^{2 m}+1, 2^{2 m+1}\right] \cup\{1\}}
		.
	\end{equation}
	Тогда, очевидно, $AE=A$.
	Поскольку $A$ есть В-регулярный оператор, то он эберлейнов,
	т.е. множество $\B(A)$ непусто.

	Пусть $B\in\B(A)$. Тогда
	\begin{equation}
		B = BA = B(AE) = (BA)E = BE
		,
	\end{equation}
	то есть $B\in\B(E)$.

	Покажем теперь, что оператор $E$ не является В-регулярным.
	Действительно, $q(E\one) =0$, т.е. $ E\one \notin ac_1$,
	и не выполнено условие (i) критерия В-регулярности (теорема~\ref{thm:crit_B_regularity}).
\end{proof}

\begin{hypothesis}
	Выполнено равенство
	\begin{equation}
		\B(E) = \B(A)
		.
	\end{equation}
\end{hypothesis}

\begin{remark}
	В конструкции оператора $E$ вместо блоков из нулей можно вставлять любые другие блоки "---
	они всё равно будут поглощены оператором $A$.
	Эти блоки могут иметь различный знак, например, содержать значительное количество
	элементов вида $-kx_1$, $kx_2$ и т.д.
	Таким образом, можно сконструировать эберлейновы операторы, очень и очень далёкие
	от достаточных условий эберлейновости (теорема~\ref{thm:Semenov_Sukochev_conditions}),
	что показывает значительную избыточность этих условий.
\end{remark}
