
Пусть $n_i$~--- строго возрастающая последовательность натуральных чисел,
\begin{equation}
	C_j = \liminf_{i\to\infty} n_{i+j} - n_i
\end{equation}

Тогда для любых $i$, $j$ имеет место быть
\begin{equation}\label{C_j_addit}
	C_{i+j} \geq C_i + C_j
	.
\end{equation}
\paragraph{Доказательство.}
По определению нижнего предела существует лишь конечное число номеров $k$
таких, что $n_{k+i} - n_k < C_i$ или $n_{k+j} - n_k < C_j$.
Зафиксируем $p$, большее всех таких $k$.

По определению нижнего предела и с учётом того, что выражение под знаком предела
принимает лишь натуральные значения,
существует бесконечное количество номеров $q$ таких, что $n_{q+i+j} - n_q = C_{i+j}$.

Обозначим через $s$ некоторый такой номер, больший $p$.
Тогда
\begin{multline}
	C_{i+j} = n_{s+i+j} - n_s = n_{s+i+j} - n_{s+i} + n_{s+i} - n_s
	\geq \\
	\geq C_j + n_{s+i} - n_s \geq C_j + C_i,
\end{multline}
так как из $s>p$ следует, что $n_{s+i+j} - n_{s+i} \geq C_j$ и $n_{s+i} - n_s \geq C_i$.

\begin{example}
	Последовательность $C_j = j+1$ не удовлетворяет условию \eqref{C_j_addit}:
	действительно, $3=C_2 < C_1+C_1 = 2+2 = 4$.
\end{example}

\paragraph{Примечание.}
Условие \eqref{C_j_addit} является необходимым, но неизвестно, является ли оно достаточным.
