\begin{lemma}
	\label{lem:convex_sequence_S_j}
	Пусть $n_i$~--- строго возрастающая последовательность натуральных чисел,
	тогда последовательность
	\begin{equation}
		S_j = \liminf_{i\to\infty} \left(n_{i+j} - n_i\right)
	\end{equation}
	является вогнутой, т.е. для любых $i$, $j$ имеет место неравенство
	\begin{equation}\label{S_j_addit}
		S_{i+j} \geq S_i + S_j
		.
	\end{equation}
\end{lemma}

\begin{proof}
	По определению нижнего предела существует лишь конечное число номеров $k$
	таких, что $n_{k+i} - n_k < S_i$ или $n_{k+j} - n_k < S_j$.
	Зафиксируем $p$, большее всех таких $k$.

	По определению нижнего предела и с учётом того, что выражение под знаком предела
	принимает лишь натуральные значения,
	существует бесконечное количество номеров $q$ таких, что $n_{q+i+j} - n_q = S_{i+j}$.

	Обозначим через $s$ некоторый такой номер, больший $p$.
	Тогда
	\begin{equation}
		S_{i+j} = n_{s+i+j} - n_s = n_{s+i+j} - n_{s+i} + n_{s+i} - n_s
		%\geq \\
		\geq S_j + n_{s+i} - n_s \geq S_j + S_i,
	\end{equation}
	так как из $s>p$ следует, что $n_{s+i+j} - n_{s+i} \geq S_j$ и $n_{s+i} - n_s \geq S_i$.
\end{proof}

\begin{example}
	Последовательность $S_j = j+1$ не удовлетворяет условию \eqref{S_j_addit}:
	действительно, $3=S_2 < S_1+S_1 = 2+2 = 4$.
\end{example}

\begin{remark}
	Условие \eqref{S_j_addit} является необходимым, но неизвестно, является ли оно достаточным
	для того, чтобы последовательность натуральных чисел $n_i$ была строго возрастающей.
\end{remark}

\begin{remark}
	Предел в формулировке лемм~\ref{thm:lim_M(j)/j_neobh}~и~\ref{thm:lim_M(j)/j_dost}
	существует для любой строго возрастающей последовательности натуральных чисел $n_i$.
	Легко видеть, что последовательность $\{M(j)\}_{j=1}^\infty$ вогнута.
	Однако, используя результат из работы \cite{Fekete} (см. также \cite[I, Задача 98]{polia1978zadachi}) получаем,
	что в таком случае предел выражения $M(j)/j$ существует и справедливо равенство
	\begin{equation}
		\lim_{j\to\infty}\frac{M(j)}{j} =\inf_{j\in\N}\frac{M(j)}{j}
		.
	\end{equation}
\end{remark}
