%Хорошо известны некоторые примеры разделяющих множеств
%[TODO2: много ссылок, в т.ч. на новую статью в МЗ].
%Применяя лемму~\ref{lem:Hausdorf_measure}, можно показать,
%что хаусдорфову размерность 1 имеют множества
%TODO2: список?

Выше в этой главе приведены различные примеры разделяющих множеств.
В данном параграфе строится пример разделяющего множества,
имеющего меру нуль и сколь угодно малую хаусдорфову размерность.

\begin{lemma}
	\label{lem:Omega_Hausdorf_separators_sum}
	Пусть множество $F \subset \ell_\infty$ "--- разделяющее
	и пусть множество $F' \subset \ell_\infty$ таково, что для некоторого $n\in\N$
	\begin{equation}
		F \subset TF' + T^2F' + ... + T^n F'
		,
	\end{equation}
	то есть для любого $x\in F$ найдутся такие $x_k\in F'$, $j=k,..,n$,
	что
	\begin{equation}
		\label{eq:Omega_Hausdorf_separators_sum}
		x = \sum_{k=1}^{n} T^k x_k
		.
	\end{equation}

	Тогда $F'$ "--- также разделяющее множество.
\end{lemma}

\begin{proof}
	Предположим противное, т.е. что $F'$ не является разделяющим.

	Тогда найдутся такие два банаховых предела $B_1\ne B_2$, что для любого
	$y \in F'$ выполнено $B_1y=B_2y$, т.е. $(B_1-B_2)y = 0$.
	Тогда для любого $x\in F$, применяя разложение~\eqref{eq:Omega_Hausdorf_separators_sum}, имеем:
	\begin{multline}
		(B_1 - B_2)x =
		(B_1 - B_2)\left(\sum_{k=1}^{n} T^k x_k\right) =
		\sum_{k=1}^{n}\left( (B_1 - B_2) T^k x_k\right) =
		\\=
		\sum_{k=1}^{n}\left( B_1 T^k x_k - B_2 T^k x_k\right) =
		\sum_{k=1}^{n}\left( B_1 x_k - B_2 x_k\right) =
		\sum_{k=1}^{n}\left( (B_1 - B_2)x_k\right) =
		0
		,
	\end{multline}
	т.е. для любого $x\in F$ выполнено $B_1 x = B_2 x$.

	С другой стороны, поскольку множество $F$ является разделяющим,
	то существует такой $x\in F$, что $B_1 x \ne B_2 x$.

	Полученное противоречие завершает доказательство.
\end{proof}

\begin{theorem}
	\label{thm:Hausdorf_measure_1_n}
	Пусть $n\in\N$.
	Тогда существует разделяющее множество $E\subset\Omega$ такое,
	что $\dim_H E = 1/n$.
\end{theorem}

\begin{proof}
	Пусть
	\begin{equation}
		E= \{ x \in \Omega : k \neq mn \Rightarrow x_k = 0\}, m\in \N
		,
	\end{equation}
	т.е. у последовательности $x \in E$ равны нулю все элементы, кроме, быть может, $x_n$, $x_{2n}$, $x_{3n}$ и т.д.

	Для $j=1,2$ определим функции $f_j : \Omega \to \Omega$ следующим образом:
	\begin{equation}
		f_1(x_1, x_2, \dots)=(\underbrace{0, ..., 0,}_{\mbox{$n-1$ раз}} 0, x_1, x_2, \dots)
		,
		\quad
		f_2(x_1, x_2, \dots)=(\underbrace{0, ..., 0,}_{\mbox{$n-1$ раз}} 1, x_1, x_2, \dots)
		.
	\end{equation}

	Очевидно, что $E=f_1(E)\cup f_2(E).$

	Теперь мы покажем, что размерность Хаусдорфа множества $E$ равна $2^{-n}$.

	Т.к. для $j=1,2$ верно
	 $$\rho(f_j(x),f_j(y))=2^{-n}\rho(x,y), \ \forall \ x, y \in E,$$
	 то функции $f_j$ являются преобразованиями подобия с коэффициентами $r_j=2^{-n}$ для $j=1,2$.


	По~\cite[Теорема 9.3]{Edgar} размерность Хаусдорфа $d$ множества $E$ является решением уравнения:
	$$ r_1^d+r_2^d=1.$$
	Т.к. $r_j=2^{-n}$, то
	$d=1/n.$

	Применяя лемму~\ref{lem:Omega_Hausdorf_separators_sum}
	для $n$, $F'=E$, $F=\Omega$, получаем, что множество $E$ является разделяющим.
\end{proof}

\begin{remark}
	У читателя может возникнуть закономерный вопрос о свойствах пересечения
	\begin{equation}
		\bigcap_{n\in \N} E_{n}
		,
	\end{equation}
	где $E_n$ "--- множество, построенное по теореме~\ref{thm:Hausdorf_measure_1_n}, т.е.
	\begin{equation}
		E_n = \{ x \in \Omega : k \neq mn \Rightarrow x_k = 0\}, m\in \N
		,
	\end{equation}
	в частности о том, является ли оно разделяющим множеством нулевой хаусдорфовой размерности.
	К сожалению, ответ на этот вопрос положителен только во второй части, а именно
	\begin{equation}
		\bigcap_{n\in \N} E_n = {(0,0,0,...)}
		.
	\end{equation}
	Таким образом, пересечение вложенной последовательности разделяющих множеств может не быть разделяющим множеством.
	Впрочем, существует и более простой пример:
	в качестве вложенной последовательности разделяющих множеств следует взять $\{M_n\}$, $M_n = [0, 2^{-n}]$.
	Очевидно, что $\bigcup\limits M_n = \{0\}$.
\end{remark}


\begin{remark}
	Пусть $n>1$ и множество $E$ построено по теореме~\ref{thm:Hausdorf_measure_1_n}.
	Тогда $p(E)< 1$ и по теореме~\ref{thm:Connor_generalized} мера $E$ равна нулю.
\end{remark}
