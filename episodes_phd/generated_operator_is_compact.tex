В свете предложенной классификации операторов возникает вопрос о том,
как обсуждаемый свойства (дружелюбность, эберлейновость и т.д.)
соотносятся с классическими свойствами линейных операторов.

Например, существует ли компактный эберлейнов оператор?
Положительный ответ на этот вопрос даёт

\begin{theorem}
	Пусть $B\in\B$.
	Тогда порождённый оператор $G_B$ компактен.
\end{theorem}

\begin{proof}
	Пусть $X$ "--- единичный шар в $\ell_\infty$,
	тогда $BX = [-1;1]$
	и множество $G_B X = [-1;1] \cdot \mathbbm 1$
	компактно,
	поскольку сходимость любой его подпоследовательности $\{y_j = G_B x_j = (Bx_j) \cdot \mathbbm\}_{j\in\N}$
	в норме пространства $\ell_\infty$ эквивалентна сходимости последовательности
	$\{Bx_j\}_{j\in\N}$ в норме пространства $\R$,
	в котором любой отрезок "--- компакт.
\end{proof}

Таким образом, мы предъявили компактный существенно дружелюбный оператор.
Заметим, что многие часто встречаемые операторы некомпактны
(что и привело к постановке обсуждаемого вопроса).
Так, очевидно, некомпактны операторы растяжения $\sigma_k$, $\sigma_{1/k}$ и $\tilde\sigma_k$;
с оператором Чезаро $C$ ситуация обстоит несколько сложнее.

Хорошо известен (и находит обширное применение)
тот факт, что оператор Чезаро компактен в пространствах $\ell_p$, $1\leq p < \infty$.
%TODO: ссылки!
Утверждение о том, что оператор $C:\ell_\infty\to\ell_\infty$ некомпактен,
выглядит как классический результат, но найти его доказательство не удалось.
Поэтому мы для полноты изложения вынуждены привести краткое доказательство этого факта,
никоим образом не претендуя на его новизну.

\begin{lemma}
	Оператор $C:\ell_\infty\to\ell_\infty$ некомпактен.
\end{lemma}

\begin{proof}
	Определим оператор $F:\ell_\infty \to \ell_\infty$ соотношением
	\begin{equation}
		(Fx)_k = x_{n} ~~\mbox{при}~~ 4^{n-1} < k \leq 4^{n}, ~n\in\N_0
		.
	\end{equation}
	Иначе говоря,
	\begin{equation}
		F(x_1, x_2, x_3, ...) =
		(x_1, \ x_2, x_2, x_2, \ \underbrace{x_3, ..., x_3}_{12~\mbox{раз}}, x_4, ...)
		.
	\end{equation}
	Рассмотрим $x=e_j$, т.е. $j$-й базисный вектор.
	Тогда $(CFe_j)_{4^{j-1}} = 3/4$, но $(CFe_j)_{4^{k}} = 0$ для $k<j-1$.

	Пусть $X$ "--- единичный шар в $\ell_\infty$, тогда $\{Fe_j\}_{j\in \N} \subset X$.
	Рассмотрим последовательность $\{CFe_j\}_{j\in \N} \subset CX$.
	Для любых $2 < k < j-1$ имеем
	\begin{equation}
		\|CFe_j - CFe_k\|_{\infty} \geq |(CFe_j)_{4^{k-1}} - (CFe_k)_{4^{k-1}}| = \left|0 - \frac 34\right| = \frac34
		.
	\end{equation}
	Таким образом, образ единичного шара $CХ$ не является относительно компактным,
	поскольку из последовательности $\{CFe_j\}_{j\in \N} \subset CX$ нельзя выделить сходящуюся подпоследовательность.

	Это и означает, что оператор $C:\ell_\infty\to\ell_\infty$ некомпактен.

\end{proof}
