\begin{definition}
	Линейный непрерывный оператор $G_{\{B_k\}}$, определённый равенством
	\begin{equation}
		(G_{\{B_k\}}x)_n = B_n x
		,
	\end{equation}
	будем называть мультипорождённым последовательностью банаховых пределов $\{B_k\}\subset \B$.
\end{definition}

Понятно, что порождённый оператор является частным случаем мультипорождённого.

\begin{lemma}
	\label{lem:multigen_is_B-regular}
	Всякий мультипорождённый оператор $G_{\{B_k\}}$ является В-регулярным оператором.
\end{lemma}

\begin{proof}
	Проверим условия критерия В-регулярности (теорема~\ref{thm:crit_B_regularity}).
	Действительно,
	\begin{equation}
		G_{\{B_k\}} \one = \one \in ac_1
		,
	\end{equation}
	\begin{equation}
		G_{\{B_k\}}\geq 0, \ \mbox{т.к.}\ (G_{\{B_k\}}x)_n \geq q(x) \ \mbox{для любого}~n\in\N
		,
	\end{equation}
	\begin{equation}
		G_{\{B_k\}} ac_0 = \{0\cdot \one \} \subset ac_0
		.
	\end{equation}
\end{proof}

\begin{corollary}
	Для всякого мультипорождённого оператора $G_{\{B_k\}}$ непусто множество $\B(G_{\{B_k\}})$
\end{corollary}

\begin{hypothesis}
	Всякий мультипорождённый оператор является существенно дружелюбным.
\end{hypothesis}
В пользу этой гипотезы говорит тот факт~\cite{Chou},
что $|\ext B| = 2^{\mathfrak c}$.

\begin{lemma}
	Пусть множество ${\{B_k\}}$ конечно, тогда оператор $G_{\{B_k\}}$ компактен.
\end{lemma}

\begin{proof}
	Достаточно заметить, что в таком случае ранг подпространства $G_{\{B_k\}} \ell_\infty$ конечен.
\end{proof}

\begin{hypothesis}
	Оператор $G_{\{B_k\}}$ компактен тогда и только тогда, когда множество ${\{B_k\}}$ конечно.
\end{hypothesis}

\begin{hypothesis}
	Для мультипорождённого оператора $G_{\{B_k\}}$ имеет место включение
	\begin{equation}
		\B (G_{\{B_k\}}) \subset \conv {\{B_k\}}
		.
	\end{equation}
	(Та же гипотеза "--- для равенства или для обратного включения)
\end{hypothesis}

\begin{remark}
	Мультипорождённые операторы по своему определению в некотором смысле противоположны матричным операторам
	(к последним относятся, например, оператор Чезаро $C$, операторы растяжения $\sigma_k$, усредняющего сжатия $\sigma_{1/k}$,
	операторы неравномерного растяжения $\tilde\sigma_k$, оператор прореживания $A$ (равенство~\eqref{eq:oper_A_throws_out_2power_blocks}) и оператор покоординатного умножения $E$ из теоремы~\ref{thm:Eberlein_but_not_B-regular_exists}).
\end{remark}

\begin{theorem}
	Мультипорождённый оператор не может иметь матричного представления.
\end{theorem}

\begin{proof}
	Справедливость теоремы непосредствено вытекает из нижеследующего результата А.А. Седаева~\cite[\S 6.3]{sedaev2009_doc_vgasu}:
	\begin{equation}
		\ell_\infty^* = \ell_1 \oplus \{\varphi \in\ell_\infty^* : \varphi c_0 = \{0\}\}
		.
	\end{equation}
\end{proof}

Перейдём теперь к примеру мультипорождённого оператора, важному для понимания инвариантности банаховых пределов.

\begin{example}
	\label{ex:multigen_invariant_interval}
	Пусть оператор $A:\ell_\infty\to\ell_\infty$
	опеределён равенством~\eqref{eq:oper_A_throws_out_2power_blocks}:
	\begin{multline}
		Ax = (x_1, x_2, \not x_3, \not x_4, x_5, x_6, x_7, x_8, \not x_9, ..., \not x_{16}, x_{17}, x_{18}, ..., x_{31}, x_{32}, \not x_{33}, \not x_{34}, ..., \not x_{64},
		\\
		x_{65}, x_{66}, ..., x_{128}, \not x_{129}, ...)=
		\\=
		(x_1, x_2, \ x_5, x_6, x_7, x_8, \ x_{17}, x_{18}, ..., x_{31}, x_{32}, \ x_{65}, x_{66}, ..., x_{128}, \ x_{257},
		\\
		..., \ x_{2^{2n} +1}, x_{2^{2n} +2},  x_{2^{2n+1}}, \ \ x_{2^{2(n+1)} +1},  x_{2^{2(n+1)} +2},  x_{2^{2(n+1)+1}}, ...)
	\end{multline}
	Тогда оператор $A$ В-регулярен по теореме~\ref{thm:A_block_thrower_is_B-regular}.
	Более того, в силу В-регулярности оператора $\sigma_2$ и того факта,
	что суперпозиция двух В-регулярных операторов снова В-регулярна (лемма~\ref{lem:B-regular_superposition_and_addition}),
	оператор $A\sigma_2$ также В-регулярен.

	В силу теоремы~\ref{thm:B-regular_is_Eberlein} мы можем выбрать $B_1 \in \B (A)$ и $B_2 \in \B(A\sigma_2)$.
	Убедимся, что $B_1 \ne B_2$, для этого достаточно предъявить одну последовательность, на которой
	значения этих банаховых пределов не совпадают.
	Пусть
	\begin{equation}
		y = \chi_{\cup_{n\in\N}(2^{2n}+1, 2^{2n+1}]}
		,
	\end{equation}
	тогда
	\begin{equation}
		\sigma_2 y \approx \chi_{\cup_{n\in\N}(2^{2n+1}+1, 2^{2n+2}]}
		.
	\end{equation}
	Следовательно, $B_1y = B_1Ay = B_1\one = 1$, но $B_2y = B_2A\sigma_2 y = B_2(0\cdot \one) = 0$.

	Определим оператор $H$ соотношением
	\begin{equation}
		Hx = (B_1 x, B_1 x, \ B_2 x, B_2 x, \ B_1 x, B_1 x, B_1 x, B_1 x, \ \underbrace{B_2 x, ..., B_2 x}_{8~\mbox{раз}}, ...)
		.
	\end{equation}

	Оператор $H$ является мультипорождённым и потому В-регулярным в силу леммы~\ref{lem:multigen_is_B-regular}.
	Значит, множество $\B(H)$ непусто.
	Однако в нашем случае мы можем дополнительно охарактеризовать его.

	Покажем, что $[B_1; B_2]\subset \B(H)$.
	Действительно,
	\begin{equation}
		B_1 Hx  = (B_1 A) Hx = B_1 (AHx)  = B_1 ((B_1 x) \cdot \one) = B_1 x
	\end{equation}
	и
	\begin{equation}
		B_2 Hx  = (B_2 A\sigma_2) Hx = B_2 (A\sigma_2 Hx)  = B_2 ((B_2 x) \cdot \one) = B_2 x
		,
	\end{equation}
	откуда и следует включение $[B_1; B_2]\subset \B(H)$.
\end{example}

\begin{hypothesis}
	Верно равенство $[B_1; B_2] = \B(H)$.
\end{hypothesis}

\begin{hypothesis}
	Верно равенство $[B_1; B_2] \cap \B(\sigma_2) = \varnothing$.
\end{hypothesis}

\begin{hypothesis}
	Верно равенство $B(H) \cap \B(\sigma_2) = \varnothing$.
\end{hypothesis}

\begin{remark}
	Оператор $(\sigma_2 H)^*$ переводит отрезок $[B_1; B_2]$ в $[B_2; B_1]$,
	т.е. <<меняет местами>> два банаховых предела.
	Действительно,
	\begin{equation}
		B_1 (\sigma_2 H)x  = (B_1 A) (\sigma_2 H)x = B_1 (A(\sigma_2 H)x)  = B_1 ((B_2 x) \cdot \one) = B_2 x
	\end{equation}
	и
	\begin{equation}
		B_2 (\sigma_2 H)x  = (B_2 A\sigma_2) (\sigma_2 H)x = B_2 (A\sigma_4 Hx)  = B_2 ((B_1 x) \cdot \one) = B_1 x
		.
	\end{equation}
	Тогда, очевидно, $\dfrac{B_1 + B_2}2 \in \B(\sigma_2 H)$.
\end{remark}

\begin{hypothesis}
	Пример~\ref{ex:multigen_invariant_interval} может быть обобщён на симплекс произвольной конечной размерности
	(треугольник, тетраэдр и т.д.).
\end{hypothesis}
