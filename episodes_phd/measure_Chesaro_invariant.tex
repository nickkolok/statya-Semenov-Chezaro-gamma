При изучении банаховых пределов и меры на множестве $\Omega$
возникает закономерный вопрос о мере множества
\begin{equation}
	\{ x \in \Omega : Bx = \beta \}
\end{equation}
для заданного банахова предела $B$ и числа $\beta\in[0;1]$.
(Хаусдорфова размерность этого множества равна 1 в силу леммы~\ref{lem:Hausdorf_measure}.)

\begin{theorem}
	Пусть $B \in \mathfrak{B}(C)$.
	Тогда
	\begin{equation}
		\mes \{ x \in \Omega : Bx = 1/2 \} = 1
		.
	\end{equation}
\end{theorem}

\begin{proof}
	Так как $B \in \mathfrak{B}(C)$, то
	\begin{equation}
		\{ x \in \Omega : Bx = 1/2 \}
		=
		\{ x \in \Omega : BCx = 1/2 \}
		\supset
		\{ x \in \Omega : \lim_{n\to\infty} (Cx)_n = 1/2 \}
		.
	\end{equation}
	Вместе с тем,
	\begin{equation}
		\mes \left\{ x \in \Omega : \lim_{n\to\infty} (Cx)_n  = 1/2 \right\} = 1
	\end{equation}
	(это следует из закона больших чисел~\cite{connor1990almost}).
\end{proof}
