При исследовании банаховых пределов интерес представляют разделяющие множества~\cite[\S 3]{Semenov2014geomprops}.
Множество $Q\subset\ell_\infty$ называют разделяющим, если
для любых неравных $B_1, B_2\in\mathfrak{B}$ существует такая последовательность $x\in Q$,
что $B_1 x \neq B_2 x$.
В частности, разделяющим является множество всех последовательностей из 0 и 1~\cite{semenov2010characteristic},
которое, как и выше, мы будем обозначать через $\Omega$
(иногда в литературе встречается также обозначение $\{0;1\}^\N$).

Каждой последовательности $(x_1, x_2, \dots)\in \Omega$ можно поставить в соответствие число
\begin{equation}\label{eq:bijection_omega_0_1}
	\sum_{k=1}^\infty 2^{-k} x_k \in [0,1]
	.
\end{equation}
С точностью до счётного множества это соответствие взаимно однозначно и определяет на множестве $\Omega$ меру,
которую мы будем отождествлять с мерой Лебега на $[0,1]$.

Оказывается, что из $\Omega$ можно выделить некоторые подмножества, которые также будут разделяющими,
например \cite[\S 3, Теорема 11]{Semenov2014geomprops},
\begin{equation}
	U = \{ x\in\Omega: q(x) = 0, p(x) = 1 \}
	.
\end{equation}

Однако множество $U$ имеет меру 1~\cite{semenov2010characteristic}.
Счётных разделяющих множеств не существует~\cite[{следствие 22}]{Semenov2014geomprops}.

В настоящей главе строится пример разделяющего множества,
являющегося подмножеством $\Omega$ и имеющего меру нуль.
Для построения такого множества используется следующий факт.

\begin{lemma}[{\cite[\S 3, замечание 6]{Semenov2014geomprops}}]
	Пусть $X$~--- разделяющее множество и $X \subset \operatorname{Lin} Y$,
	где $\operatorname{Lin} Y$ обозначает линейную оболочку $Y$.
	Тогда $Y$ также является разделяющим множеством.
\end{lemma}

В теореме~\ref{thm:Lin_Omega_Sucheston} ниже доказывается, что
\begin{equation}
	\label{eq:Omega_Lin_Omega_repeat}
	\Omega \subset \operatorname{Lin}\{x\in\Omega : p(x) = a,~ q(x) = b\}
\end{equation}
для любых $0\leq b < a \leq 1$.

Затем обсуждаются свойства линейных оболочек множеств, определённых с помощью функционалов Сачестона.
В частности, доказывается,
что наряду с включением~\eqref{eq:Omega_Lin_Omega_repeat}
имеет место равенство
\begin{equation}
	\ell_\infty = \operatorname{Lin}\{x\in\ell_\infty : p(x) = a,~ q(x) = b\}
\end{equation}
для любых $a>b$.

Возникает закономерный вопрос: для каких ещё подмножеств пространства $\ell_\infty$
верны аналогичные соотношения?

Выясняется, что аналогичным свойством обладает подпространство
$A_0 = \{ x \in \ell_\infty : \alpha(x) =0 \}$,
где, напомним,
\begin{equation*}
	\alpha(x) = \varlimsup_{i\to\infty} \max_{i<j\leqslant 2i} |x_i - x_j|
	,
\end{equation*}
уже знакомое нам, например, по теореме~\ref{thm:A0_is_space}.

Результаты, излагаемые в данной главе, опубликованы в%
~\cite{
our-mz2021linearhulls,
avdeev2021vestnik,
avdeev2022measure,
}.
