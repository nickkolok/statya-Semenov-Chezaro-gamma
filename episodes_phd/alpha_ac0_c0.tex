Докажем теперь ещё одну теорему,
раскрывающую связь между сходимостью, почти сходимостью и $\alpha$--функцией.

\begin{theorem}
	Пусть $x\in ac_0$ и $\alpha(x)=0$.
	Тогда $x \in c_0$.
\end{theorem}

\begin{proof}
	Предположим противное, т.е. $x\notin c_0$.
	Тогда
	\begin{equation}
		\varlimsup_{k\to\infty} x_k \neq 0 ~~\mbox{или}~~ \varliminf_{k\to\infty} x_k \neq 0
		.
	\end{equation}

	Не теряя общности, положим
	\begin{equation}
		\varepsilon = \varlimsup_{k\to\infty} x_k > 0
	\end{equation}
	(иначе домножим всю последовательность на $-1$, что, очевидно, не повлияет на сходимость к нулю).

	Тогда существует бесконечно много таких $n$, что
	\begin{equation}\label{alpha_ac0_c0_limsup}
		x_n > \varlimsup_{k\to\infty} x_k - \frac{\varepsilon}{4} = \frac{3\varepsilon}{4}
		.
	\end{equation}

	Так как
	\begin{equation}
		\alpha(x) = \varlimsup_{i\to\infty} \max_{i \leq j \leq 2i} |x_i-x_j| = 0
		,
	\end{equation}
	то
	\begin{equation}
		\exists(N_1\in\N)\forall(n > N_1)\left[\max_{n \leq j \leq 2n} |x_n-x_j| < \frac{\varepsilon}{4}\right]
		,
	\end{equation}
	или, что то же самое,
	\begin{equation}\label{alpha_ac0_c0_alpha}
		\exists(N_1\in\N)\forall(n > N_1)\forall(j: n \leq j \leq 2n)\left[ |x_n-x_j| < \frac{\varepsilon}{4}\right]
		.
	\end{equation}

	Поскольку $x \in ac_0$, то  по критерию Лоренца
	\begin{equation}
		\exists(N_2 \in\N)\forall(n > N_2)\forall(m\in\N)
		\left[ \left| \frac{1}{n}\sum_{k=m+1}^{m+n}x_k\right| < \frac{\varepsilon}{4} \right]
		,
	\end{equation}
	в частности,
	\begin{equation}\label{alpha_ac0_c0_Lorencz}
		\exists(N_2 \in\N)\forall(n > N_2)
		\left[ \left| \frac{1}{n}\sum_{k=n+1}^{2n}x_k\right| < \frac{\varepsilon}{4} \right]
		,
	\end{equation}

	Выберем $n$ так, чтобы оно удовлетворяло \eqref{alpha_ac0_c0_limsup} и \eqref{alpha_ac0_c0_alpha}.
	Тогда
	\begin{multline}
		\left| \frac{1}{n}\sum_{k=n+1}^{2n}x_k\right|
		=
		\\=
		\mbox{(по \eqref{alpha_ac0_c0_alpha} имеем $x_k \geq 3\varepsilon/4 > 0$)}
		=
		\\=
		\frac{1}{n}\sum_{k=n+1}^{2n}x_k
		\geq
		\frac{3\varepsilon}{4}
		>
		\frac{\varepsilon}{4}
		,
	\end{multline}
	что противоречит \eqref{alpha_ac0_c0_Lorencz}.

	Полученное противоречие завершает доказательство.
\end{proof}


\begin{corollary}
	Пусть $x\in ac$ и $\alpha(x)=0$.
	Тогда $x \in c$.
\end{corollary}

%TODO: доказывать нужно или очевидно?

Однако, как выясняется, справедлив и более общий результат.
