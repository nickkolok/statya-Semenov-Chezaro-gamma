Дадим переформулировку критерия Лоренца
\cite{lorentz1948contribution,bennett1974consistency}
почти сходимости последовательности,
которая иногда позволяет упростить доказательство:
брать предел равномерно не по всем $m\in \N$,
а только по достаточно большим значениям.


\begin{theorem}
	[Модифицированный критерий Лоренца]
	\label{thm:Lorentz_mod}
	Пусть $x\in\ell_\infty$.

	$x\in ac_0$ тогда и только тогда, когда
	\begin{equation}\label{crit_pos_ac0}
		\forall(A_2\in\N)
		\exists(n_0\in\N)
		\exists(m_0\in\N)
		\forall(n\geq n_0)
		\forall(m\geq m_0)
		\\
		\left[
			\left|
			\frac{1}{n}
			\sum_{k=m+1}^{m+n} x_k
			\right|
			<
			\frac{1}{A_2}
		\right]
		.
	\end{equation}

\end{theorem}

\begin{proof}
	По теореме Лоренца $x\in ac_0$ тогда и только тогда, когда
	\begin{equation}\label{Lorencz_ac0}
		\lim_{n\to\infty} \frac{1}{n} \sum_{k=m+1}^{m+n} x_k = 0
	\end{equation}
	равномерно по $m$.

	Или, переводя на язык кванторов,
	\begin{equation}\label{crit_ac0}
		\forall(A_1\in\N)
		\exists(n_1\in\N)
		\forall(n\geq n_0)
		\forall(m \in \N)
		\\
		\left[
			\left|
			\frac{1}{n}
			\sum_{k=m+1}^{m+n} x_k
			\right|
			<
			\frac{1}{A_1}
		\right]
		.
	\end{equation}
	Очевидно, что из \eqref{crit_ac0} следует \eqref{crit_pos_ac0} (например, положив $m_0 = 1$),
	тем самым необходимость \eqref{crit_pos_ac0} доказана.

	\paragraph{Достаточность.}
	Пусть выполнено \eqref{crit_pos_ac0}.
	Покажем, что выполнено \eqref{crit_ac0}.
	Зафиксируем $A_1$.
	Положим $A_2 = 2A_1$ и отыщем $n_0$ и $m_0$ в соотвествии с \eqref{crit_pos_ac0}.
	Положим $n_1 = 2A_2(n_0+m_0)\|x\|$.
	Покажем, что \eqref{crit_ac0} верно для любых $n\geq n_1$, $m\in \N$.
	Зафиксируем $n$ и рассмотрим $m$.

	Пусть сначала $m\geq m_0$.
	Тогда в силу того, что $n\geq n_1 = 2A_2(n_0+m_0)\|x\| > n_0$ имеем $n>n_0$.
	Применим \eqref{crit_pos_ac0}:
	\begin{equation}
		\left|
		\frac{1}{n}
		\sum_{k=m+1}^{m+n} x_k
		\right|
		<
		\frac{1}{A_2}
		=
		\frac{1}{2A_1}
		<
		\frac{1}{A_1}
		,
	\end{equation}
	т.е. \eqref{crit_ac0} выполнено.

	Пусть теперь $m < m_0$.
	Заметим, что
	\begin{equation}
		\left|
			\sum_{k=m+1}^{m+n} x_k
			-
			\sum_{k=m_0+1}^{m_0+n} x_k
		\right|
		\leq 2(m_0 - m) \|x\|
		,
	\end{equation}
	откуда
	\begin{equation}
		\left| \sum_{k=m+1}^{m+n} x_k \right|
		\leq
		2(m_0 - m) \|x\| + \left| \sum_{k=m_0+1}^{m_0+n} x_k \right|
		\leq
		2 m_0 \|x\| + \left| \sum_{k=m_0+1}^{m_0+n} x_k \right|
		.
	\end{equation}


	Тогда
	\begin{multline}
		\left| \frac{1}{n} \sum_{k=m+1}^{m+n} x_k \right|
		\leq
		\frac{2 m_0 \|x\|}{n} + \left| \frac{1}{n} \sum_{k=m_0+1}^{m_0+n} x_k \right|
		\mathop{\leq}^{\mbox{~~в силу \eqref{crit_pos_ac0}~~}}
		\frac{2 m_0 \|x\|}{n} + \frac{1}{A_2}
		\leq
		\\ \leq
		\frac{2 m_0 \|x\|}{n_1} + \frac{1}{A_2}
		\leq
		\frac{2 m_0 \|x\|}{2A_2(n_0+m_0)\|x\|} + \frac{1}{A_2}
		=
		\\=
		\frac{m_0}{A_2(n_0+m_0)} + \frac{1}{A_2}
		<
		\frac{1}{A_2} + \frac{1}{A_2}
		=
		\frac{1}{A_1}
		,
	\end{multline}
	т.е. \eqref{crit_ac0} тоже выполнено.
\end{proof}

Удобство критерия \eqref{crit_pos_ac0} в том,
что можно выбирать $m_0$ в зависимости от $A_2$.
