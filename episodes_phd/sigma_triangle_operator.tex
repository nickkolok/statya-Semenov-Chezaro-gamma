Рассмотрим теперь оператор,
в чём-то похожий на операторы растяжения,
но при этом растяжения <<неравномерного>>.
Пусть
\begin{equation}
	\sigma_\triangle x =
	(x_1; \; x_1; x_2; \; x_1; x_2; x_3; ... ; \underbrace{x_1; ...; x_n}_{n~\mbox{элементов}}; ...)
	.
\end{equation}

\begin{remark}
	Оператор $\sigma_\triangle x$ естественным образом возник в работах А.С. Усачёва~\cite{usachev2009_phd_vsu},
	%TODO1: другие статьи?
	где обозначался $\overline{x}$ и
	использовался для изучения коэффициентов Фурье--Хаара
	и расстояния от произвольной последовательности $x\in\ell_\infty$
	до пространства $ac$.
\end{remark}

Нам потребуется один факт об операторе $\sigma_\triangle$,
доказанный А.С. Усачёвым в~\cite[теорема 19]{usachev2009_phd_vsu}.
Мы приведём его эквивалентную формулировку, более удобную для нас в дальнейшем.

\begin{theorem}
	\label{thm:usachev_overline_x_ac_s}
	Пусть $x\in\ell_\infty$, $s\in\R$.
	Тогда условия $x\in ac_s$ и $\sigma_\triangle x \in ac_s$ эквивалентны.
\end{theorem}

Проницательный читатель уже наверняка догадался,
какой факт об операторе $\sigma_\triangle$ мы сейчас докажем.

\begin{theorem}
	Оператор $\sigma_\triangle$ является В-регулярным.
\end{theorem}

\begin{proof}
	Снова воспользуемся леммой~\ref{lem:suff_B_reg}.
	Легко заметить, что $\sigma_\triangle \geq 0$ и $\sigma_\triangle \one  = \one$,
	что обеспечивает выполнение условий (i) и (ii) соответственно.

	Выполнение условия (iii) непосредственно следует из теоремы~\ref{thm:usachev_overline_x_ac_s}.

	Наконец, покажем, что $\sigma_\triangle T - T \sigma_\triangle  : \ell_\infty \to ac_0$.
	Действительно, для произвольного $x\in\ell_\infty$
	\begin{multline}
	(\sigma_\triangle T - T \sigma_\triangle)x =
	\sigma_\triangle Tx - T \sigma_\triangle x =
	\\
	(x_2; \; x_2, x_3; \; x_2, x_3,   x_4; \; x_2, x_3, x_4, x_5; \; ... )-
	\\
	(x_1, x_2; \; x_1,    x_2, x_3; \; x_1,   x_2, x_3, x_4; \; x_1, ... )=
	\\
	(x_2-x_1; 0; x_3-x_1; 0, 0, x_4-x_1; 0,0,0; x_5-x_1; ...) \in\Iac\subset ac_0
	.
	\end{multline}

	Таким образом, все условия леммы~\ref{lem:suff_B_reg} выполнены,
	и оператор $\sigma_\triangle$ действительно является В-регулярным.
\end{proof}

\begin{hypothesis}
	\label{hyp:sprawling_ac_B-regular}
	Размашистым (sprawling)
	% https://github.com/ualexandr/Problems-for-our-grant/issues/30
	назовём оператор взятия подпоследовательности $H:\ell_\infty \to \ell_\infty$ такой, что
	\begin{equation}
		H(x_1,x_2, x_3, ...) =
		(
		x_{m_1},x_{m_1+1},...x_{n_1-1},x_{n_1};\;
		x_{m_2},x_{m_2+1},...x_{n_2-1},x_{n_2};\;
		...)
		,
	\end{equation}
	где для всех $k\in\N$ выполнено $m_k\leq n_k$,
	и для всех $k > k_0 \in \N$ выполнено
	$m_{k+1} \geq m_k$, $n_{k+1} \geq n_k$,
	при этом
	\begin{equation}
		\lim_{k\to\infty} n_k = \infty, \quad \lim_{k\to\infty} n_k - m_k = \infty
		.
	\end{equation}
	Гипотеза заключается в том, что любой размашистый оператор
	(а) переводит $ac_0$ в $ac_0$ (или, слабее "--- $c_0$ в $ac_0$) и
	(б) является В-регулярным.
	Операторы $A$ и $\sigma_\triangle$ из этого параграфа являются размашистыми.
\end{hypothesis}

В пользу гипотезы~\ref{hyp:sprawling_ac_B-regular} говорит~\cite[теорема 1]{usachev2008transforms} (приводим её в эквивалентной формулировке и более удобных обозначениях).

\begin{theorem}
	Определим линейный оператор $H:\ell_\infty\to\ell_\infty$ равенством
	\begin{equation}
		H(x_1, x_2, ...) = \left(x_{m_1}, x_{m_1+1}, \ldots, x_{m_1+n_1} ; x_{m_2}, x_{m_2+1}, \ldots, x_{m_2+n_2} ; \ldots\right)
		,
	\end{equation}
	где $m_k, n_k$ "--- последовательности натуральных чисел и $\lim_{n\to\infty} n_k = \infty$.
	Пусть $x\in ac_s$, $s\in\R$. Тогда $Hx \in ac_s$.
\end{theorem}
