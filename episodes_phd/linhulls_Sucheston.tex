Итак, $\Omega \subset \operatorname{Lin}\{x\in\Omega : p(x) = a,~ q(x) = b\}$, где $1\geq a>b\geq 0$.


Пусть $Y^a_b = \{x\in A_0 : p(x) = a, q(x) = b\}$, где $a>b$.
Подготовим сначала вспомогательные леммы о константе.

\begin{lemma}
	\label{lem:const_Lin_alpha_0}
	Справедливо включение
	$\one\in \operatorname{Lin} Y^a_b$.
\end{lemma}

\begin{proof}
	Не теряя общности, будем полагать, что $a>0$.

	Определим оператор $S:\ell_\infty \to \ell_\infty$ следующим образом:
	\begin{equation}\label{operator_S}
		(Sy)_k = y_{i+2}, \mbox{ где } 2^i < k \leq 2^i+1
		.
	\end{equation}
	\begin{lemma}[{\cite{our-vzms-2018}}]
		Для любого $x\in \ell_\infty$ выполнено равенство
		\begin{equation}\label{alpha_S}
			\alpha(Sx) = \varlimsup_{k\to\infty} |x_{k+1} - x_{k}|
			.
		\end{equation}
	\end{lemma}
	Положим
	\begin{equation}
		\label{eq:y_for_s_alpha}
		y = \left(0,1,0,\frac{1}{2},1,\frac{1}{2},0,\frac{1}{3},\frac{2}{3},1,\frac{2}{3},\frac{1}{3},0,...\right)
		,
	\end{equation}
	тогда $Sy\in A_0$ и $\one-Sy = S(\one-y)\in A_0$.

	Пусть $x=(a-b)Sy+b\one$, $z=(a-b)(\one-Sy)+b\one$.
	Тогда $p(x)=p(z)=a$, $q(x)=q(z)=b$ и, следовательно, $x,z\in Y^a_b$.
	Кроме того, заметим, что
	\begin{equation}
		x+z = (a-b)Sy+b\one + (a-b)(\one-Sy)+b\one
		=
		(a-b)(Sy-Sy+\one) + 2 b\one = (a+b)\one
		,
	\end{equation}
	откуда и следует, что $\one\in Y^a_b$.
\end{proof}


\begin{lemma}
	\label{lem:const_Lin_alpha_0_a_eq_-b}
	Справедливо включение
	$\one\in \operatorname{Lin} Y^a_{-a}$.
\end{lemma}

\begin{proof}
	Определим линейный оператор $M:\ell_\infty \to \ell_\infty$ следующим образом:
	\begin{multline}
		M\omega=
		M(\omega_1,\omega_2,\omega_3,...)=
		\left(
			0, 1\omega_1,
			0, \frac{1}{2}\omega_2, 1\omega_2, \frac{1}{2}\omega_2,
			0, \frac{1}{3}\omega_3, \frac{2}{3}\omega_3, 1\omega_3, \frac{2}{3}\omega_3, \frac{1}{3}\omega_3,
			0, ...,
		\right. \\ \left.
			0, \frac{1}{p}\omega_p, \frac{2}{p}\omega_p, ..., \frac{p-1}{p}\omega_p, 1\omega_p,
				\frac{p-1}{p}\omega_p, ..., \frac{2}{p}\omega_p, \frac{1}{p}\omega_p,
			0, \frac{1}{p+1}\omega_{p+1}, ...
		\right)
		.
	\end{multline}
	Тогда $SM: \ell_\infty \to A_0$.
	Положим
	\begin{gather}
		x=aS(2M(\one)-\one),
		\\
		y=-aS(2M(1,0,1,0,1,0,1,0,...)-\one),
		\\
		z=-aS(2M(0,1,0,1,0,1,0,1,...)-\one).
	\end{gather}

	Тогда, очевидно, каждая из последовательностей $x,y,z$ содержит отрезки сколь угодно большой длины,
	состоящие из $a$ (равно как и из $-a$), при этом $-a \leq x,y,z \leq a$.
	Следовательно, $p(x)=p(y)=p(z) = a$ и $q(x)=q(y)=q(z) = -a$,
	откуда $x,y,z \in Y^a_{-a}$.

	Заметим теперь, что
	\begin{multline}
		x + y + z
		=
		\\=
		aS(2M(\one)-\one) - aS(2M(1,0,1,0,1,0,...)-\one) - aS(2M(0,1,0,1,0,1,...)-\one)
		=
		\\=
		aS(2M(\one)-\one  -    2M(1,0,1,0,1,0,...)+\one  -    2M(0,1,0,1,0,1,...)+\one)
		=
		\\=
		aS(2M(\one) - 2M(1,0,1,0,1,0,...) - 2M(0,1,0,1,0,1,...)+\one+\one-\one)
		=
		\\=
		aS(2M(\one) - 2M(1,0,1,0,1,0,...) - 2M(0,1,0,1,0,1,...)+\one)
		=
		aS\one
		=
		a\one
		,
	\end{multline}
	откуда $\one \in \operatorname{Lin} Y^a_{-a}$.
\end{proof}


\begin{lemma}
	\label{lem:c_0_Lin_alpha_0}
	Справедливо включение $c_0 \subset Y^a_b$.
\end{lemma}

\begin{proof}
	Зафиксируем $z\in c_0$.
	Выберем произвольный $x \in Y^a_b$.
	Тогда $x+z\in Y^a_b$ и, очевидно, $z=(x+z)-x$.
\end{proof}

\begin{theorem}
	\label{thm:A_0_c_infty_lin}
	Пусть $a\neq -b$.
	Тогда справедливо равенство $\operatorname{Lin} Y^a_b = A_0$.
\end{theorem}

\begin{proof}
	Зафиксируем $x \in A_0$.

	Пусть сначала $p(x) = q(x)$.
	Тогда, согласно теореме~\ref{thm:alpha_c_ac_c}, $x\in c$
	и $x$ может быть представлен в виде суммы константы и последовательности из $c_0$.
	Утверждение теоремы следует из лемм~\ref{lem:const_Lin_alpha_0}, ~\ref{lem:const_Lin_alpha_0_a_eq_-b} и~\ref{lem:c_0_Lin_alpha_0}.

	Пусть теперь $p(x) > q(x)$.
	Положим
	\begin{equation}
		y=k\cdot x + C\cdot\one,
		\quad \mbox{где} \quad
		k=\dfrac{a-b}{p(x)-q(x)},
		\quad
		C=\dfrac{bp(x)-aq(x)}{p(x)-q(x)}
		.
	\end{equation}
	Тогда, очевидно,
	\begin{equation}
		\label{eq:x_representation}
		x=(y-C\cdot\one)/k
		.
	\end{equation}
	Представление~\eqref{eq:x_representation} искомое.
	Действительно, в силу лемм~\ref{lem:const_Lin_alpha_0}~и~\ref{lem:const_Lin_alpha_0_a_eq_-b} выполнено
	$C\cdot\one\in Y^a_b$; кроме того,
	\begin{equation}
		p(y) = k\cdot p(x) + C
		=
		%\\=
		\frac{ap(x)-bp(x)+bp(x)-aq(x)}{p(x)-q(x)}
		=
		a
		,
	\end{equation}
	\begin{equation}
		q(y) = k\cdot q(x) + C
		=
		%\\=
		\frac{aq(x)-bq(x)+bp(x)-aq(x)}{p(x)-q(x)}
		=
		b
		.
	\end{equation}


\end{proof}

Факт, аналогичный теоремам~\ref{thm:Lin_Omega_Sucheston} и~\ref{thm:A_0_c_infty_lin}, верен и для
всего пространства $\ell_\infty$:
$\ell_\infty\subset \operatorname{Lin} X^a_b$, где
$X^a_b = \{x\in\ell_\infty : p(x) = a,~ q(x) = b\}$, $a>b$.


\begin{lemma}
	\label{lem:const_Lin_ell_infty}
	Справедливо включение
	$\one\in \operatorname{Lin} X^a_b$.
\end{lemma}

\begin{proof}
	В самом деле,
	$Y^a_b \subset X^a_b$
	и, следовательно,
	\begin{equation}
		\one \in \operatorname{Lin} Y^a_b \subset \operatorname{Lin} X^a_b
		.
	\end{equation}
\end{proof}

\begin{theorem}
	\label{thm:Lin_ell_infty}
	Справедливо равенство $\operatorname{Lin} X^a_b = \ell_\infty$.
\end{theorem}

\begin{proof}
	Зафиксируем $x \in \ell_\infty$ и представим его в виде линейной комбинации последовательностей из $X^a_b$.

	Не теряя общности, положим $x\geq 0$
	(иначе представим сначала $x$ в виде $x = y - z$, где $y \geq 0$, $z \geq 0$.
	и найдём представления для $y$ и $z$).

	Если $p(x) = q(x)$, то возьмём некоторый $y\in\ell_\infty$,
	такой, что $p(y) > p(x) = q(x)  \geq q(y) \geq 0$.
	Тогда в силу выпуклости функционала $p$ имеем
	\begin{equation}
		p(x+y) \geq p(y) > p(x) = q(x)
		,
	\end{equation}


	\begin{equation}
		q(x+y) = -p(-x-y) \leq -p(-x) -p(-y) = q(x) + q(y) \leq q(x) < p(x+y)
		,
	\end{equation}
	и задача сведена к отысканию представлений для $y$ и $x+y$.
	Таким образом, случай $p(x) = q(x)$ можно исключить,
	и, не теряя общности, рассматривать только такие $x$, что $p(x) > q(x)$.

	Снова, как и в доказательстве теоремы~\ref{thm:A_0_c_infty_lin},
	положим
	\begin{equation}
		y=k\cdot x + C\cdot\one,
		\quad\mbox{где}\quad
		k=\frac{a-b}{p(x)-q(x)},
		\quad
		C=\frac{bp(x)-aq(x)}{p(x)-q(x)}
		.
	\end{equation}
	Дальнейшее доказательство переносится дословно.
\end{proof}
