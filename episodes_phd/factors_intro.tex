Дальнейшим ослаблением понятия сходимости является сходимость по Чезаро (сходимость в среднем).
Говорят, что последовательность $\{x_n\}\in\ell_\infty$ сходится по Чезаро к $t$, если
\begin{equation}
	\lim_{n\to\infty}\frac1{n}\sum_{i=1}^n x_i = t
	.
\end{equation}
Легко заметить, что обсуждаемые обобщения верхнего и нижнего пределов удовлетворяют соотношению
\begin{equation}
	\label{eq:generalization_of_limits}
	\liminf_{n\to\infty} x_n \leq q(x) \leq \liminf_{n\to\infty}\frac1{n}\sum_{i=1}^n x_i
	\leq
	%\\ \leq
	\limsup_{n\to\infty}\frac1{n}\sum_{i=1}^n x_i
	\leq p(x)
	\leq \limsup_{n\to\infty} x_n
	.
\end{equation}

Отдельный интерес представляет множество всех последовательностей из 0 и 1,
которое в дальнейшем мы будем обозначать через $\Omega$.
%(иногда в литературе~\cite{semenov2020geomBL,Semenov2014geomprops} встречается также обозначение $\{0;1\}^\N$).
Понятно, что каждый $x\in \Omega$ можно отождествить с подмножеством множества натуральных чисел
$\supp x \subset \N$.

Вслед за~\cite{hall1992behrend} будем обозначать через $\mathscr{M}A$ множество всех чисел,
кратных элементам множества $A\subset\N$, т.е.
\begin{equation}
	\mathscr{M}A = \{ka: k\in\N, a\in A\}
	,
\end{equation}
через $\chi F$ "--- характеристическую функцию множества $F$.

Так, например,
\begin{gather}
	\chi \mathscr{M}\!A(\{2\}) = \chi \mathscr{M}\!A(\{2, 4\}) = \chi \mathscr{M}\!A(\{2,4,8,16,...\})
	= (0,1,0,1,0,1,0,1,0,...),
\\
	\chi \mathscr{M}\!A(\{3\}) = \chi \mathscr{M}\!A(\{3,9,27,...\}) = (0,0,1,\;0,0,1,\;0,0,1,\;0,0,1,\;0,0,1,\;...),
\\
	\chi \mathscr{M}\!A(\{2,3\}) = \chi \mathscr{M}\!A(\{2,3,6\}) = (0,1,1,1,0,1,\;0,1,1,1,0,1,\;0,1,1,1,0,1,...).
\end{gather}

Возникает закономерный вопрос о взаимосвязи структуры множества $A$
и значений, которые принимают обобщения верхнего и нижнего пределов~\eqref{eq:generalization_of_limits}
на последовательности $\chi \mathscr{M}\!A$.
Так, в работах~\cite{davenport1936sequences,davenport1951sequences} доказано, что для любого
$A=\{a_1,a_2,...\}\subset\N$
выполнено
\begin{equation}
	\liminf_{n\to\infty}\frac1{n}\sum_{i=1}^n (\chi\mathscr{M}A)_i =
	\lim_{j\to\infty}\lim_{n\to\infty}\frac1{n}\sum_{i=1}^n (\chi\mathscr{M}\{a_1,a_2,...,a_j\})_i
	.
\end{equation}
В работе~\cite[\S 7]{besicovitch1935density} построено такое множество $A\subset\N$, что
\begin{equation}
	\liminf_{n\to\infty}\frac1{n}\sum_{i=1}^n (\chi\mathscr{M}A)_i \neq
	\limsup_{n\to\infty}\frac1{n}\sum_{i=1}^n (\chi\mathscr{M}A)_i
	.
\end{equation}
За более подробной информацией о множествах типа $\mathscr{M}A$ отсылаем читателя к монографии~\cite{hall1996multiples}.

В этой главе изучается зависимость значений, которые могут принимать функционалы Сачестона
на последовательностях $\chi\mathscr{M}A$, от свойств множества $A$.
