Введём в рассмотрение семейство операторов
$\tilde\sigma_k : \ell_\infty \to \ell_\infty$, $k>0$,
определяемых следующим образом:
\begin{equation}
	(\tilde\sigma_k x)_n = x_{\left\lceil \dfrac{n}{k}\right\rceil}
	.
\end{equation}

\begin{example}
	\label{example:sigma_3_2}
	\begin{equation}
		\tilde\sigma_{3/2} x =
		(x_1, x_2, x_2, \; x_3, x_4, x_4, \; x_5, ...)
		.
	\end{equation}
\end{example}

\begin{example}
	\label{example:sigma_2_3}
	\begin{equation}
		\tilde\sigma_{2/3} x =
		(x_2, x_3, \; x_5, x_6, \; x_8, ...)
		.
	\end{equation}
\end{example}


Заметим, что для $k\in \N$ выполнено равенство $\sigma_k = \tilde\sigma_k$
(однако использовать обозначение без тильды мы не можем, чтобы избежать путаницы с операторами усреднения $\sigma_{1/k}$).
Однако соотношение $\tilde\sigma_k \tilde\sigma_m = \tilde\sigma_{km}$ для нецелых $k$, вообще говоря, не выполняется.
Чтобы это увидеть, достаточно рассмотреть суперпозиции $\tilde\sigma_{3/2} \tilde\sigma_{2/3}$ и
$\tilde\sigma_{2/3} \tilde\sigma_{3/2}$ (см. примеры~\ref{example:sigma_3_2} и~\ref{example:sigma_2_3} выше).

Таким образом, операторы $\tilde\sigma_k$ обобщают операторы $\sigma_k$,
и возникает закономерный вопрос о соответствующих инвариантных банаховых пределах.

\begin{theorem}
	Пусть $k>0$, $k\in \Q \setminus \N$.
	Тогда $\B(\tilde\sigma_k)=\varnothing$.
\end{theorem}

\begin{proof}
	Пусть $k$ представимо в виде несократимой дроби $k=p/q$, $p\in \N$, $q\in\N_2$.
	Рассмотрим последовательность $x\in \ell_\infty$, заданную соотношением
	\begin{equation}
		x_k = \begin{cases}
			1, \mbox{~если~} m=qn+1, n\in\N,
			\\
			0  \mbox{~иначе.}
		\end{cases}
	\end{equation}
	Последовательность $x$ периодична, и её период равен $q$.

	Пусть $B\in \B$, тогда $Bx=\dfrac1q$ , поскольку любой банахов предел на периодической последовательности принимает значение, равное среднему арифметическому по периоду.
	%TODO: ссылка!
	Заметим, что $\tilde\sigma_{p/q}x \in \Omega$.
	Более того, последовательность $\tilde\sigma_{p/q}x \in \Omega$ также периодична и имеет период, равный $p$.

	Действительно,
	\begin{multline}
		(\tilde\sigma_{p/q}x )_{m+p} =
		x_{\left\lceil \dfrac{m+p}{p/q}\right\rceil} =
		x_{\left\lceil \dfrac{qm+qp}{p}\right\rceil} =
		x_{\left\lceil \dfrac{qm}{p}+q\right\rceil} =
		\\=
		x_{q+\left\lceil \dfrac{qm}{p}\right\rceil} =
		x_{\left\lceil \dfrac{qm}{p}\right\rceil} =
		x_{\left\lceil \dfrac{m}{p/q}\right\rceil} =
		(\tilde\sigma_{p/q}x \in)_{m}
		.
	\end{multline}

\end{proof}

\begin{theorem}
	Пусть $k>0$, $k\in \Q \setminus \N$.
	Тогда $\tilde\sigma_k ac_0 \not \subset ac_0$.
\end{theorem}




\begin{theorem}
	Пространство $ac_0$ не замкнуто относительно любого оператора $\tilde\sigma_k$, $k\in\Q^+\setminus \N$.
\end{theorem}

\begin{proof}
	Пусть $k=p/q$ есть несократимая дробь.
	Определим последовательность $x\in\ell_\infty$ соотношением:
	\begin{equation}
		x_m = \begin{cases}
		\end{cases}
	\end{equation}
	TODO: набрать доказательство.
\end{proof}

\begin{hypothesis}
	\label{hyp:tilde_sigma_k_x_notin_ac0}
	Для любого $k \in \R^+\setminus \N$ ($k \in \Q^+\setminus \N$) существует такой $x\in ac_0$, что $\tilde\sigma_{k} x \notin ac$.
\end{hypothesis}

В пользу гипотезы~\ref{hyp:tilde_sigma_k_x_notin_ac0} говорит следующий
\begin{example}
	Определим последовательность $x\in\ell_\infty$ следующим соотношением:
	\begin{equation}
		x_k = \begin{cases}
			(-1)^k, &~\mbox{если}~ 2^{2n} < k \leq 2^{2n+1}, ~ n \in \N_0
			\\
			(-1)^{k+1} &~\mbox{иначе}
			.
		\end{cases}
	\end{equation}
	Тогда $x\in ac_0$, но
	\begin{equation}
		(\tilde\sigma_{1/2} x)_k = \begin{cases}
			-1, &~\mbox{если}~ 2^{2n} < k \leq 2^{2n+1}, ~ n \in \N_0
			\\
			1 &~\mbox{иначе}
			.
		\end{cases}
	\end{equation}
	Имеем $p(\tilde\sigma_{1/2} x) = 1$, $q(\tilde\sigma_{1/2} x) = -1$,
	откуда $\tilde\sigma_{1/2} x \notin ac$.
\end{example}



\begin{hypothesis}
	Для любых натуральных $k > m$ существуют такие $r, s\in\N$, что  $\tilde\sigma_{m/k} T^r \tilde\sigma_{k/m} = T^s$.
\end{hypothesis}


\begin{hypothesis}
	Для любого $k \in \N$ выполнено $(\tilde\sigma_{\sqrt{k}})^2 - \sigma_k : \ell_\infty \to ac_0$.
\end{hypothesis}



\begin{hypothesis}
	Пусть $x\in ac_0$.
	Для того, чтобы $x\in \mathcal{I}(ac_0)$, необходимо и достаточно,
	чтобы $\tilde\sigma_{k} x \in ac_0$ для любого $k\in \Q^+$.
\end{hypothesis}


\begin{hypothesis}
	Пусть $x\in ac_0$.
	Для того, чтобы $x\in \mathcal{I}(ac_0)$, необходимо и достаточно,
	чтобы $\tilde\sigma_{k} x \in ac_0$ для любого $k\in \R^+$.
\end{hypothesis}
