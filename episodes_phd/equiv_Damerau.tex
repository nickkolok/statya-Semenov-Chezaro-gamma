Этот параграф носит вспомогательный характер.
Основная его цель "--- ввести полезное отношение эквивалентности,
в некотрых случаях упрощающее запись.

В математической лингвистике широко известно расстояние Дамерау--Левенштейна
"---
мера разницы двух строк символов, определяемая как минимальное количество операций вставки,
удаления, замены и транспозиции (перестановки двух соседних символов),
необходимых для перевода одной строки в другую~\cite{damerau1964technique,wagner1974string,gasfield2003strings}.
Расстояние Дамерау--Левенштейна является модификацией расстояния Левенштейна (в сторону невозрастания):
к операциям вставки, удаления и замены символов, определённых в расстоянии Левенштейна~\cite{levenstein1965binary},
добавлена операция транспозиции (перестановки) двух соседних символов.

Эти расстояния действительно являются метриками на множестве всех конечных слов из букв заданного алфавита;
основное их современное приложение "--- это, например, определение опечаток при наборе текста
и выявление помех при передаче алфавитных сигналов~\cite{oommen1997pattern,brill2000improved,bard2006spelling,li2006exploring}.
%TODO: если нужно, больше ссылок из
%https://en.wikipedia.org/wiki/Damerau%E2%80%93Levenshtein_distance#References


Нам, однако, расстояние Дамерау--Левенштейна интересно в другом контексте.
Совершенно очевидно, что его определение можно обобщить на бесконечные последовательности чисел,
при этом метрикой полученный объект быть перестаёт,
так как может принимать значение, равное бесконечности.

\begin{definition}
	\label{def:Damerau_Levenshein_distance}
	Для $x, y \in \ell_\infty$ будем говорить,
	что расстояние Дамерау--Левенштейна между $x$ и $y$ конечно,
	и писать $x\approx y$,
	если последовательность $x$ можно получить из последовательности $y$
	конечным числом вставок элементов и конечным числом удалений элементов.
\end{definition}

\begin{remark}
	Здесь мы опускаем упоминание операций замены элемента и транспозиции элементов,
	поскольку каждая из этих операций представима в виде комбинации операции вставки и операции удаления.
	Это обусловлено тем, что численное значение самого расстояния нам безразлично "---
	важен лишь факт его конечности.
\end{remark}

\begin{lemma}
	Пусть $B\in\B$, $x\approx y$.
	Тогда $Bx=By$.
\end{lemma}

\begin{proof}
	Предположим сначала, что последовательность $x$ получена из последовательности $y$ удалением одного элемента $y_k$:
	\begin{equation}
		(x_1, x_2, x_3,...) = (y_1, y_2, ..., y_{k-1}, y_{k+1}, ...)
		.
	\end{equation}
	Тогда, очевидно,
	\begin{equation}
		T^{k-1} x = T^{k}y = (y_{k+1}, y_{k+2}, y_{k+3}, ...)
		,
	\end{equation}
	поэтому
	\begin{equation}
		Bx = BT^{k-1}x = BT^{k}y = By
		.
	\end{equation}
	Если же последовательность $x$ получена из последовательности $y$ вставкой одного элемента $x_k$,
	то в вышеприведённых рассуждениях последовательности $x$ и $y$ меняются местами.

	Наконец, по индукции приведённые рассуждения можно повторить для любого числа удалений и вставок,
	то есть для любых последовательностей, расстояние Дамерау--Левенштейна между которыми конечно.
\end{proof}

Более того, очевидна следующая
\begin{lemma}
	Соотношение $x\approx y$ выполнено тогда и только тогда, когда для некоторых $m,n\in\N$ верно равенство $T^m x = T^n y$.
\end{lemma}

Непосредственной проверкой аксиом (рефлексивности, симметричности и транзитивности) легко убедиться,
отношение $\approx$ является отношением эквивалентности (будем называть его \emph{эквивалентностью Дамерау--Левенштейна}) разбивает $\ell_\infty$ на классы эквивалентности.
Эти классы оказываются достаточно узкими.
Так, например, в разные классы попадут последовательности $x$ и $y$, связанные соотношением $x_n = y_n + \frac1n$,
хотя $x-y \in c_0$ и для любого $B\in\B$ выполнено $Bx=By$.

Мы будем пользоваться эквивалентностью Дамерау--Левенштейна, чтобы упростить запись равенств,
в ряде случаев избегая оператора сдвига и пренебрегая конечным количеством элементов последовательности.
