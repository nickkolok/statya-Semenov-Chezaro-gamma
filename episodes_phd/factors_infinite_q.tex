Перейдём теперь к изучению нижнего функционала Сачестона $q(\chi\mathscr{M}A)$.

\begin{theorem}
	\label{thm:ac0_primes_p_psi_A_prod}
	Пусть $A = \{a_1, a_2, ..., a_n,...\}$ "--- бесконечное множество попарно взаимно простых чисел.
	Тогда
	\begin{equation}
		q(\chi\mathscr{M}A) \geq 1-\prod_{j=1}^\infty \left(1-\frac{1}{a_j}\right)
		.
	\end{equation}
\end{theorem}

\begin{proof}
	%Зафиксируем $k\in\mathbb{N}$.

	Заметим, что нижний функционал Сачестона можно представить в виде
	\begin{equation}
		\label{eq:ac0_primes_px_lim_pnx}
		q(x) = \lim_{n\to\infty} q_n(x)
		,
	\end{equation}
	где
	\begin{equation}
		q_n(x) = \inf_{m\in\mathbb{N}}  \frac{1}{n} \sum_{j=m}^{m+n-1} x_j
		.
	\end{equation}

	Поскольку предел~\eqref{eq:ac0_primes_px_lim_pnx} существует, то для его оценки можно использовать предел подпоследовательности
	$q_{n_k}(x)$, где $n_k = a_1\cdot a_2 \cdot ... \cdot a_k$.

	Заметим теперь, что в любом отрезке последовательности $\chi\mathscr{M}A$ длины $n_k$
	содержится не более $\prod_{j=1}^k (a_j-1)$ нулей
	(могут попадаться <<дополнительные>> единицы "--- элементы с индексами, кратными $a_j$ для $j>k$).
	Значит,
	\begin{equation}
		q_{n_k}(\chi\mathscr{M}A) =
		\sup_{m\in\mathbb{N}}  \frac{1}{n_k} \sum_{j=m}^{m+n_k-1} (\chi\mathscr{M}A)_j
		\geq
		1-\prod_{i=1}^k \frac{a_i-1}{a_i}
		.
	\end{equation}
	Снова перейдя к пределу по $k$, получим
	\begin{equation}
		\label{eq:ac0_primes_p_psi_A_upper_bound}
		q(\chi\mathscr{M}A) \geq \lim_{k\to \infty} \prod_{i=1}^k \frac{a_i-1}{a_i}
		=
		1-\prod_{i=1}^\infty \frac{a_i-1}{a_i}
		.
	\end{equation}

\end{proof}




\begin{corollary}
	\label{cor:ac0_primes_p_psi_A_prod}
	Пусть $A = \{a_1, a_2, ..., a_n,...\}$ "--- бесконечное множество попарно взаимно простых чисел
	и $a_{n+1}>a_1\cdot...\cdot a_n$.
	Тогда
	\begin{equation}
		q(\chi\mathscr{M}A) = 1-\prod_{j=1}^\infty \left(1-\frac{1}{a_j}\right)
		.
	\end{equation}
\end{corollary}

\begin{proof}
	%Зафиксируем $k\in\mathbb{N}$.

	Заметим, что нижний функционал Сачестона можно представить в виде
	\begin{equation}
		\label{eq:ac0_primes_px_lim_pnx}
		q(x) = \lim_{n\to\infty} q_n(x)
		,
	\end{equation}
	где
	\begin{equation}
		q_n(x) = \inf_{m\in\mathbb{N}}  \frac{1}{n} \sum_{j=m}^{m+n-1} x_j
		.
	\end{equation}

	Поскольку предел~\eqref{eq:ac0_primes_px_lim_pnx} существует, то для его оценки можно использовать предел подпоследовательности
	$q_{n_k}(x)$, где $n_k = a_1\cdot a_2 \cdot ... \cdot a_k$.


	Cреди первых $n_k$ элементов последовательности $\chi\mathscr{M}A$
	ровно $\prod_{j=1}^k (a_j-1)$ нулей, поскольку комбинации остатков от деления
	на взаимно простые числа $a_1, a_2, ..., a_k$ <<не успевают>> повторяться.
	Следовательно,
	\begin{equation}
		q_{n_k}(\chi\mathscr{M}A) =
		\inf_{m\in\mathbb{N}}  \frac{1}{n_k} \sum_{j=m}^{m+n_k-1} (\chi\mathscr{M}A)_j
		\leq
		\frac{1}{n_k} \sum_{j=1}^{n_k} (\chi\mathscr{M}A)_j
		=
		1-\prod_{i=1}^k \frac{a_i-1}{a_i}
		.
	\end{equation}
	Переходя к пределу по $k$, имеем
	\begin{equation}
		\label{eq:ac0_primes_p_psi_A_lower_bound}
		q(\chi\mathscr{M}A) \leq 1-\lim_{k\to \infty} \prod_{i=1}^k \frac{a_i-1}{a_i}
		=
		1-\prod_{i=1}^\infty \frac{a_i-1}{a_i}
		.
	\end{equation}

	Сопоставив~\eqref{eq:ac0_primes_p_psi_A_lower_bound} и~\eqref{eq:ac0_primes_p_psi_A_upper_bound}, получим утверждение следствия.
\end{proof}

\begin{corollary}
	В случае, когда $\mathbb{N}\setminus A$ конечно,
	из следствия~\ref{cor:ac0_powers_finite_set_of_numbers} непосредственно вытекает, что $\chi\mathscr{M}A\in ac_1$
	и, соответственно, $q(\chi\mathscr{M}A)=1$.
\end{corollary}



Классическая теорема Эйлера~\cite{euler1737variae} говорит о том, что
ряд обратных простых чисел
\begin{equation}
	\frac{1}{2} + \frac{1}{3} + \frac{1}{5} + \frac{1}{7} + ...
	=
	\sum_j \frac{1}{j},
\end{equation}
где $j$ пробегает все простые числа, расходится.

С учётом этого факта
из теоремы~\ref{thm:ac0_primes_p_psi_A_prod} вытекает
\begin{lemma}
	Пусть $\varepsilon \in  (0; 1{]}$.
	Существует бесконечное множество попарно непересекающихся подмножеств простых чисел
	$A_i$ такое, что $q(\chi\mathscr{M}A)\geq\varepsilon$ для любого $i\in\mathbb{N}$.
\end{lemma}



\begin{theorem}
	Пусть $A=\{a_1, a_2, ..., a_n, ...\}\subset\mathbb{N}$ есть бесконечное множество попарно взаимно простых чисел
	и
	\begin{equation}
		\label{eq:prod_causes_q}
		\prod_{j=1}^\infty \left(1-\frac{1}{a_j}\right) = 0
	\end{equation}
	Тогда $\chi\mathscr{M}A\in ac$ и, более того, $\chi\mathscr{M}A\in ac_1$.
\end{theorem}

\begin{proof}
	По теореме~\ref{thm:ac0_primes_p_psi_A_prod} условие~\eqref{eq:prod_causes_q} влечёт равенство $q(\chi\mathscr{M}A)=1$.
	По теореме~\ref{lem:ac0_primes_infinity_mutually_prime_subset} $p(\chi\mathscr{M}A)=1$,
	откуда и следует требуемое.
\end{proof}
