По определению, каждый банахов предел инвариантен относительно сдвига: $B=BT$.
Возникает закономерный вопрос: относительно каких ещё операторов инвариантны банаховы пределы?

Мы не будем рассматривать здесь операторы обобщённого сдвига~\cite{marchenko2006generalized,lewitan1945normed};
%TODO2: при критической необходимости ссылок можно набрать тут:
%https://encyclopediaofmath.org/index.php?title=Generalized_displacement_operators#References
достаточно естественным обобщением обычного сдвига являются <<движения>> "--- инъективные отображения $\N$ в себя без периодических точек; за их обсуждением в свете банаховых пределов отсылаем читателя к работе Р.\,Нильсена~\cite{Nillsen}.

Уже для совершенно естественного "--- если бы мы говорили об обычной сходимости последовательностей "---
оператора растяжения $\sigma_k$
(напомним, что этот оператор просто повторяет каждый элемент последовательности $k$ раз)
выясняется, что относительно этого оператора инварианты не все банаховы пределы!
Более того, инвариантность конкретного банахова предела зависит от выбора $k\in\N_2$.

Для оператора Чезаро
\begin{equation}
	C (x_1, x_2, x_3, ...) = \left(
	x_1,
	\dfrac{x_1+x_2}2,
	\dfrac{x_1+x_2 + x_3}3,
	\dfrac{x_1+x_2+x_3+x_4}4,
	...,
	\dfrac{x_1+...+x_n}n,
	...\right)
	.
\end{equation}
ситуация с инвариантностью банаховых пределов обстоит ещё <<хуже>>.
Напомним, что через $\B(H)$ мы обозначаем множество банаховых пределов,
инвариантных относительно оператора $H$,
т.е. таких $B\in \B$, что $B=BH$.
Так, в~\cite[\S2, Theorem 4]{semenov2020dilation} показано, что
\begin{equation}
	\label{eq:B_C_subsetneq_B_sigma_n}
	\B(C) \subsetneq \bigcap_{n\in \N}\B(\sigma_n)
	.
\end{equation}
(это утверждение обобщает~\cite[Theorem 3]{semenov2020invariant_noncommutative}
и~\cite[Theorem 4.8]{ASSU4}).

Важной вехой в изучении инвариантных банаховых пределов явилась следующая теорема,
доказанная в~\cite[\S2]{Semenov2010invariant}.
Мы ещё не раз будем возвращаться к её обсуждению.

\begin{theorem}
	\label{thm:Semenov_Sukochev_conditions}
	Пусть линейный оператор $H:\ell_\infty\to\ell_\infty$ таков, что:
	\\(i)   $H \geqslant 0, H \one=\one$;
	\\(ii)  $H c_0 \subset c_0$;
	\\(iii) $\limsup _{j \rightarrow \infty}(A(I-T) x)_j \geqslant 0$ для всех $x \in \ell_{\infty}, A \in R$,
	\\где
	\begin{equation}
		R=R(H):=\operatorname{conv}\left\{H^n, n=1,2, \ldots\right\}
		.
	\end{equation}
	Тогда $\B(H) \ne\varnothing$.
\end{theorem}

В настоящей главе в теореме~\ref{thm:Eberlein_but_not_B-regular_exists}
строится пример такого оператора $E$, что $\B(E)\ne\varnothing$,
но $E\one \ne \one$, что показывает существенную избыточность условий теоремы~\ref{thm:Semenov_Sukochev_conditions}.


Достаточно <<плохими>> в смысле инвариантности оказываются крайние точки множества банаховых пределов $\ext \B$.
Это множество не является слабо$^*$ замкнутым~\cite{Nillsen,Talagrand},
и его мощность совпадает~\cite{Chou} с мощностью всего пространства $\ell_\infty^*$
и составляет $2^{\mathfrak c}$.

%TODO2: про расстояние от $\ext \B$ до $\B(C)$ и прочих там всяких $\B(\sigma_n)$ (а также их объединения?).

Тем не менее, в настоящей главе удалось использовать элементы $\ext\B$ для построения инвариантных банаховых пределов.
%TODO1: ссылки на теоремы

При взгляде на включение~\eqref{eq:B_C_subsetneq_B_sigma_n} возникает закономерный вопрос:
а можно ли найти такой <<самый лучший>> банахов предел (или класс банаховых пределов),
который был бы инвариантен относительно всех линейных операторов,
относительно которых инвариантен хотя бы один банахов предел?
Отрицательный ответ на этот вопрос даёт теорема~\ref{thm:generated_operator_G_B}, в которой
для произвольного $B\in \B$ строится такой оператор $G_B:\ell_\infty \to \ell_\infty$,
что $\B(G_B) = \{B\}$.

Инвариантные банаховы пределы находят приложения в некоммутативной геометрии
и теории сингулярных следов, где используются в качестве элемента конструкции
различных подклассов следов Диксмье
~\cite{carey2003spectral,lord2012singular,sukochev2015characterization,sukochev2016dixmier}.
В недавних работах~\cite{astashkin2015constants_rus_DAN,astashkin2016constants_rus_SMJ} инвариантные банаховы пределы были применены для исследования
асимптотики констант Лебега для системы Уолша.

%Invariant Banach limits have also
%found important applications in noncommutative geometry and the theory
%of singular traces, where they were employed in the construction of various
%subclasses of Dixmier traces [12, 19, 28, 29]. Recently, invariant Banach
%limits have been used to study the asymptotics of the Lebesgue constants
%of the Walsh system [9].
