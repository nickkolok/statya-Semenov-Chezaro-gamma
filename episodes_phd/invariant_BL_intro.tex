По определению, каждый банахов предел инвариантен относительно сдвига: $B=BT$.
Возникает закономерный вопрос: относительно каких ещё операторов инвариантны банаховы пределы?

Уже для совершенно естественного "--- если бы мы говорили об обычной сходимости последовательностей "---
оператора растяжения $\sigma_k$
(напомним, что этот оператор просто повторяет каждый элемент последовательности $k$ раз)
выясняется, что относительно этого оператора инварианты не все банаховы пределы!
Более того инвариантность конкретного банахова предела зависит от выбора $k\in\N_2$.


%TODO: больше!


Достаточно <<плохими>> в смысле инвариантности оказываются крайние точки множества банаховых пределов $\ext \B$.
Это множество не является слабо$^*$ замкнутым~\cite{Nillsen,Talagrand},
и его мощность совпадает~\cite{Chou} с мощностью всего пространства $\ell_\infty^*$
и составляет $2^{\mathfrak c}$.

TODO: про расстояние от $\ext \B$ до $\B(C)$ и прочих там всяких $\B(\sigma_n)$ (а также их объединения?).

Тем не менее, в настоящей главе удалось использовать элементы $\ext\B$ для построения инвариантных банаховых пределов.
%TODO: ссылки на теоремы


Инвариантные банаховы пределы находят приложения в некоммутативной геометрии
и теории сингулярных следов, где используются в качестве элемента конструкции
различных подклассов следов Диксмье
~\cite{carey2003spectral,lord2012singular,sukochev2015characterization,sukochev2016dixmier}.
В недавних работах~\cite{astashkin2015constants_rus_DAN,astashkin2016constants_rus_SMJ} инвариантные банаховы пределы были применены для исследования
асимптотики констант Лебега для системы Уолша.

%Invariant Banach limits have also
%found important applications in noncommutative geometry and the theory
%of singular traces, where they were employed in the construction of various
%subclasses of Dixmier traces [12, 19, 28, 29]. Recently, invariant Banach
%limits have been used to study the asymptotics of the Lebesgue constants
%of the Walsh system [9].
