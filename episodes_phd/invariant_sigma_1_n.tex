Введём, вслед за~\cite[с. 131, утверждение 2.b.2]{lindenstrauss1979classical},
на $\ell_\infty$ оператор
\begin{equation}
	\sigma_{1/n} x = n^{-1}
	\left(
		\sum_{i=1}^{n} x_i,
		\sum_{i=n+1}^{2n} x_i,
		\sum_{i=2n+1}^{3n} x_i,
		...
	\right).
\end{equation}

Следующая лемма была впервые сформулирована в~\cite{ASSU2},
однако для полноты изложения мы приводим здесь более подробное доказательство.

\begin{lemma}[{\cite[{Лемма~4}]{ASSU2}}]
	Для любого $k\in\N_2$ выполнено
	\begin{equation}
		\sigma_k \sigma_{1/k} - I : \ell_\infty \to ac_0
		.
	\end{equation}
\end{lemma}

\begin{proof}
	Покажем, что для любого $x\in\ell_\infty$ выполнен критерий Лоренца~\eqref{eq:crit_Lorentz},
	т.е.
	\begin{equation}
		\label{eq:Lorentz_sigma_1_k_I}
		\lim_{n\to\infty} \frac{1}{n} \sum_{i=m+1}^{m+n} ((\sigma_k \sigma_{1/k} - I)x)_i = 0
	\end{equation}
	равномерно по $m\in\N$.

	Зафиксируем сначала $m$ и выберем $a,b\in\N$
	таким образом, что
	\begin{equation}
		k(a-1) <    m+1 \leq ka
		,
		\quad
		kb     \leq m+n <    k(b+1)
		.
	\end{equation}
	Тогда
	\begin{multline}
		0 \leq \left|
		\sum_{i=m     +1}^{m+n} \left( (\sigma_k \sigma_{1/k} - I)x \right)_i
		\right|
		\leq
		\\\leq
		\left|\sum_{i=m     +1}^{ka    } \left( (\sigma_k \sigma_{1/k} - I)x \right)_i\right| +
		\left|\sum_{i=ka    +1}^{kb    } \left( (\sigma_k \sigma_{1/k} - I)x \right)_i\right| +
		\left|\sum_{i=kb    +1}^{m+n   } \left( (\sigma_k \sigma_{1/k} - I)x \right)_i\right|
		\leq
		\\\leq
		\left|\sum_{i=m     +1}^{ka    } \left\| (\sigma_k \sigma_{1/k} - I)x \right\|\right| +
		\left|\sum_{i=ka    +1}^{kb    } \left( (\sigma_k \sigma_{1/k} - I)x \right)_i\right| +
		\left|\sum_{i=kb    +1}^{m+n   } \left\| (\sigma_k \sigma_{1/k} - I)x \right\|\right|
		\leq
		\\\leq
		      \sum_{i=k(a-1)+1}^{ka    } \left\| (\sigma_k \sigma_{1/k} - I)x \right\| +
		\left|\sum_{i=ka    +1}^{kb    } \left( (\sigma_k \sigma_{1/k} - I)x \right)_i\right| +
		      \sum_{i=kb    +1}^{k(b+1)} \left\| (\sigma_k \sigma_{1/k} - I)x \right\|
		=
		\\=
		2k \left\| (\sigma_k \sigma_{1/k} - I)x \right\| +
		\left|\sum_{i=ka    +1}^{kb    } \left( (\sigma_k \sigma_{1/k} - I)x \right)_i\right|
		\leq
		\\\leq
		2k \left(\left\| (\sigma_k \sigma_{1/k} - I)\| \cdot \|x \right\|\right) +
		\left|\sum_{i=ka    +1}^{kb    } \left( (\sigma_k \sigma_{1/k} - I)x \right)_i\right|
		\leq
		\\\leq
		2k \left( (\left\|\sigma_k\| \cdot \| \sigma_{1/k}\| + \|I\|) \cdot \|x \right\|\right) +
		\left|\sum_{i=ka    +1}^{kb    } \left( (\sigma_k \sigma_{1/k} - I)x \right)_i\right|
		\leq
		\\\leq
		4k  \|x \| +
		\left|\sum_{i=ka    +1}^{kb    } \left( (\sigma_k \sigma_{1/k} - I)x \right)_i\right|
		=
		4k  \|x \| +
		\left|\sum_{i=ka    +1}^{kb    } (\sigma_k \sigma_{1/k}x)_i - \sum_{i=ka    +1}^{kb    } x_i\right|
		=
		\\=
		4k  \|x \| +
		\left|
			\underbrace{\frac{x_{ka+1} + x_{ka+2} + ... + x_{ka+k}}{k} + ... + \frac{x_{ka+1} + x_{ka+2} + ... + x_{ka+k}}{k}}_{k~\text{~раз}}
			+
			...+
		\right.
		\\
		\left.
			+
			\underbrace{\frac{x_{k(b-1)+1}  + ... + x_{k(b-1)+k}}{k} + ... + \frac{x_{k(b-1)+1} + ... + x_{k(b-1)+k}}{k}}_{k~\text{~раз}}
			-\sum_{i=ka    +1}^{kb    } x_i
		\right|
		=
		4k\|x\|
		.
	\end{multline}

	Заметим, что полученное выражение не зависит от $m$.
	Следовательно,
	\begin{equation}
		\left|\lim_{n\to\infty} \frac{1}{n} \sum_{i=m+1}^{m+n} ((\sigma_k \sigma_{1/k} - I)x)_i\right|
		=
		%\\=
		\lim_{n\to\infty} \frac{1}{n} \left|\sum_{i=m+1}^{m+n} ((\sigma_k \sigma_{1/k} - I)x)_i\right|
		\leq
		\lim_{n\to\infty} \left( \frac{1}{n} \cdot 4k \|x\|\right)
		= 0
	\end{equation}
	равномерно по $m$,
	откуда и вытекает~\eqref{eq:Lorentz_sigma_1_k_I}.

	%Осталось заметить, что для знакопеременной последовательности $x = x^+ + x^-$, $x^+ \geq 0$, $x^- \leq 0$
	%мы имеем $(\sigma_k \sigma_{1/k} - I)x^+ \in ac_0$ и $(\sigma_k \sigma_{1/k} - I)(-x^-) \in ac_0$,
	%а потому $(\sigma_k \sigma_{1/k} - I)x \in ac_0$.
\end{proof}

\begin{theorem}
	\label{thm:B_sigma_n_eq_B_sigma_1_n}
	Для любого $n\in\N$ выполнено $\mathfrak{B}(\sigma_n) = \mathfrak{B}(\sigma_{1/n})$.
\end{theorem}

\begin{proof}
	Пусть сначала $B \in \mathfrak{B}(\sigma_{1/n})$.
	Тогда
	\begin{equation}
		B(x) = B(Ix) = B(\sigma_{1/n} \sigma_n x) = B(\sigma_n x)
		.
	\end{equation}
	Пусть теперь $B \in \mathfrak{B}(\sigma_n)$.
	Тогда
	\begin{equation}
		B(x) = B(Ix) = B((I+(\sigma_n \sigma_{1/n} - I)) x) = B(\sigma_n \sigma_{1/n} x) = B(\sigma_{1/n} x)
		.
	\end{equation}
\end{proof}

\begin{remark}
	Мы видим, что множество банаховых пределов, инвариантных относительно суперпозиции операторов,
	может быть шире, чем объединение множеств банаховых пределов,
	инвариантных относительно каждого из операторов:
	\begin{equation}
		\mathfrak{B}(\sigma_n) \cup \mathfrak{B}(\sigma_{1/n}) = \mathfrak{B}(\sigma_n) \subsetneq \mathfrak{B}(\sigma_{1/n}\sigma_n) = \mathfrak{B}(I)
		.
	\end{equation}
\end{remark}
