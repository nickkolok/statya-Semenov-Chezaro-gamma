Изучим сначала оператор $A:\ell_\infty\to\ell_\infty$,
определённый равенством:
\begin{multline}
	\label{eq:oper_A_throws_out_2power_blocks}
	Ax = (x_1, x_2, \not x_3, \not x_4, x_5, x_6, x_7, x_8, \not x_9, ..., \not x_{16}, x_{17}, x_{18}, ..., x_{31}, x_{32}, \not x_{33}, \not x_{34}, ..., \not x_{64},
	\\
	x_{65}, x_{66}, ..., x_{128}, \not x_{129}, ...)=
	\\=
	(x_1, x_2, \ x_5, x_6, x_7, x_8, \ x_{17}, x_{18}, ..., x_{31}, x_{32}, \ x_{65}, x_{66}, ..., x_{128}, \ x_{257},
	\\
	..., \ x_{2^{2n} +1}, x_{2^{2n} +2},  x_{2^{2n+1}}, \ \ x_{2^{2(n+1)} +1},  x_{2^{2(n+1)} +2},  x_{2^{2(n+1)+1}}, ...)
\end{multline}
(Зачёркивание, т.е. запись вида $\not x_k$,
применяется при записи оператора взятия подпоследовательности для наглядности и означает,
что $k$-й элемент последовательности $x$ не включается в последовательность $Ax$.)
В дальнейшем этот оператор окажется очень полезен при построении других операторов и банаховых пределов.

Очевидна следующая

\begin{lemma}
	\label{lem:supp_I_ac0}
	Пусть $x\in \mathcal I(ac_0)$, $Y \subset \supp x$ и $y = x \cdot \chi Y$.
	Тогда $y \in \mathcal I(ac_0)$.
\end{lemma}

%TODO2: если совсем прижмёт - аккуратно, со вкусом доказать!

\begin{lemma}
	\label{lem:AT-TA}
	$AT - TA: \ell_\infty \to \mathcal{I}(ac_0)$.
\end{lemma}

\begin{proof}
	Пусть $\varphi(n) = 1 + 2^0 + 2^2 + 2^4 + ... + 2^{2n}$
	и пусть $y = \chi{\{k\in\N : k = \varphi(n), n\in \N\}}$.
	Тогда ясно, что $y \in ac_0$ (в силу быстрого роста функции $\varphi(n)$)
	%TODO2: аккуратное доказательство с помощью теоремы из матзаметок?
	и, более того, $y\in \mathcal{I}(ac_0)$.
	Заметим, что
	\begin{multline}
		ATx =
		(x_2, x_3, \ x_6, x_7, x_8, x_9, \ x_{18}, x_{19}, ..., x_{32}, x_{33}, \ x_{66}, x_{67}, ..., x_{129}, \ x_{258},
		\\
		..., \ x_{2^{2n} +2}, x_{2^{2n} +3},  x_{ 2^{2n+1} +1}, \ \ x_{2^{2(n+1)} +2},  x_{2^{2(n+1)} +3},  x_{ 2^{2(n+1)+1} +1}, ...)
		.
	\end{multline}
	С другой стороны,
	\begin{multline}
		TAx =
		(x_2, \ x_5, x_6, x_7, x_8, \ x_{17}, x_{18}, ..., x_{31}, x_{32}, \ x_{65}, x_{66}, ..., x_{128}, \ x_{257},
		\\
		..., \ x_{2^{2n} +1}, x_{2^{2n} +2},  x_{2^{2n+1}}, \ \ x_{2^{2(n+1)} +1},  x_{2^{2(n+1)} +2},  x_{2^{2(n+1)+1}}, ...)
		.
	\end{multline}
	Таким образом, для любого $x\in\ell_\infty$
	\begin{equation}
		(ATx-TAx)_k=\begin{cases}
			x_{2^{2n-1}} -x_{2^{2n} + 1}, ~~&\mbox{если}~ k = \varphi(n), ~n\in \N_0,
			\\
			0 ~~&\mbox{иначе}.
		\end{cases}
	\end{equation}
	Ввиду включения $\supp (AT-TA)x \subset \supp y$ имеем $(AT-TA)x \in \mathcal I (ac_0)$ по лемме~\ref{lem:supp_I_ac0}.
\end{proof}

Докажем следующую вспомогательную лемму.

\begin{lemma}
	\label{lem:suff_B_reg}
	Пусть оператор $H$ удовлетворяет следующим условиям:
	\\i)   $H \geq 0$;
	\\ii)  $H\one\in ac_1$;
	\\iii) $H c_0 \subset ac_0$;
	\\iv)  $HT-TH : \ell_\infty \to ac_0$.% для некоторых $k,m \in \N_0$.

	Тогда оператор $H$ является В-регулярным.
\end{lemma}

\begin{proof}
	Доказательство проведём непосредственной проверкой определения банахова предела
	(определение~\ref{def:Banach_limit})
	для суперпозиции $B = B_1 H$,
	где $B_1 \in \B$ "--- произвольный банахов предел.

	Действительно, если $H\geq 0$, то $B = B_1 H \geq 0$,
	поскольку $B_1 \geq 0$ по определению банахова предела $B_1$.
	Далее, $B\one = B_1 H \one = 1$, поскольку $H\one \in ac_1$,
	а $B_1(ac_1) = 1$ по определению почти сходимости.
	Заметим теперь, что
	\begin{equation}
		Bc_0 = B_1 H c_0 = B_1 ac_0 = 0
	\end{equation}
	снова в силу того, что $B_1\in\B$.
	Наконец, для любого $x\in\ell_\infty$ выполнено
	\begin{equation}
		(BT-B)x = BTx - Bx = B_1 HTx - B_1 H x = B_1 HTx - B_1 T H x = B_1 (HT -  T H) x = 0
		.
	\end{equation}
	В силу произвольности выбора $x\in \ell_\infty$ последнее равенство и означает, что $BT=B$.

	Значит, для любого банахова предела $B_1$ функционал $B=B_1 H$ снова есть банахов предел,
	что по определению~\ref{def:B-regular_operator}
	и означает В-регулярность оператора $H$.
\end{proof}

\begin{remark}
	Условия леммы~\ref{lem:suff_B_reg} не являются необходимыми;
	так, контрпример к условию (i) очевиден:
	\begin{equation}
		H_1(x_1,x_2,x_3,x_4...) = (-x_1, x_2, x_3, x_4, ...)
		.
	\end{equation}
	Оператор $H_1$ является В-регулярным и, более того, $\B(H_1) = \B$.
\end{remark}


\begin{theorem}
	\label{thm:A_block_thrower_is_B-regular}
	Оператор $A$ является В-регулярным.
\end{theorem}

\begin{proof}
	Непосредственно проверим условия леммы~\ref{lem:suff_B_reg}.

	i) Всякий оператор взятия подпоследовательности является неотрицательным.

	ii) $A\one = \one \in ac_1$.

	iii)  В силу того, что оператор $A$ является оператором взятия подпоследовательности,
	выполнено включение $Ac_0 \subset c_0 \subset ac_0$.

	iv) В силу леммы~\ref{lem:AT-TA} для любого $x\in \ell_\infty$ выполнено $(AT-TA)x \in \mathcal{I}(ac_0) \subset ac_0$.

	Таким образом, оператор $A$ действительно является В-регулярным.
\end{proof}

\begin{lemma}
	Для любого банахова предела $B_1$ выполено $B_1 A \in \mathfrak B \setminus \B (\sigma_{2^{2n-1}})$, $n\in \N$.
\end{lemma}

\begin{proof}
	Напомним (см. определение~\ref{def:Damerau_Levenshein_distance}), что знак приближенного равенства $ u \approx w$ означает,
	что конечно расстояние Дамерау--Левенштейна между последовательностями
	$u$ и $w$,
	т.е. что $w$ можно получить из $u$ конечным числом вставок, удалений и замен элементов.
	Тогда $Bu = Bw$ для любого $B\in\B$ и любых $u \approx w$.

	Положим $x = \chi {\cup_{m=0}^{\infty}\left[2^{2 m}+1, 2^{2 m+1}\right]}$.
	Тогда $\sigma_{2^{2n}} x \approx x$ и $\sigma_{2^{2n-1}} x \approx \one - x$.
	Заметим, что $Ax \approx \one$.

	Предположим противное утверждению леммы.
	Пусть $B \in \B(A) \cap \B (\sigma_{2^{2n-1}})$, $n\in \N$.
	Тогда
	\begin{multline}
		1 = B\one = BAx = (BA)x = Bx =
		(B\sigma_{2^{2n-1}}) x = B (\sigma_{2^{2n-1}} x) =
		\\=
		B(\one -x) = 1 - Bx = 1 - (BA)x = 1-1 = 0
		.
	\end{multline}
	Полученное противоречие завершает доказательство.
\end{proof}

\begin{remark}
	Например, для $B_1 \in \mathfrak B (\sigma_2)$ имеем $B_1 \ne B_1 A$, что показывает,
	что оператор $A$ является существенно эберлейновым.
\end{remark}

\begin{theorem}
	Для любого $B\in\B$ выполнено
	\begin{equation}
		\|BA-BA\sigma_2\| = 2
		.
	\end{equation}
\end{theorem}

\begin{proof}
	Заметим, что $BA\in\B$ и $BA\sigma_2\in\B$ в силу теоремы~\ref{thm:A_block_thrower_is_B-regular}, и потому
	\begin{equation}
		\|BA-BA\sigma_2\| \leq \|BA\|+\|BA\sigma_2\| = 1+1 = 2
		.
	\end{equation}
	Пусть теперь
	\begin{equation}
		y \approx 2\chi {\cup_{m=0}^{\infty}\left[2^{2 m}+1, 2^{2 m+1}\right]} - \one
		,
	\end{equation}
	тогда
	\begin{equation}
		\sigma_2 y \approx 2\chi {\cup_{m=0}^{\infty}\left[2^{2 m+1}+1, 2^{2 m+2}\right]} - \one
	\end{equation}
	и
	\begin{multline}
		\|BA-BA\sigma_2\| \geq \dfrac{\| BAy-BA\sigma_2y \|}{\|y\|} = \| BAy-BA\sigma_2y \| =
		\\=
		\left\| BA\left(2\chi {\cup_{m=0}^{\infty}\left[2^{2 m}+1, 2^{2 m+1}\right]} - \one\right)-
		BA\left(2\chi {\cup_{m=0}^{\infty}\left[2^{2 m+1}+1, 2^{2 m+2}\right]} - \one\right) \right\|=
		\\=
		\left\| 2BA\chi {\cup_{m=0}^{\infty}\left[2^{2 m}+1, 2^{2 m+1}\right]}-
		2BA\chi {\cup_{m=0}^{\infty}\left[2^{2 m+1}+1, 2^{2 m+2}\right]}  \right\|=
		\\=
		2\left\| BA\chi {\cup_{m=0}^{\infty}\left[2^{2 m}+1, 2^{2 m+1}\right]}-
		BA\chi {\cup_{m=0}^{\infty}\left[2^{2 m+1}+1, 2^{2 m+2}\right]}  \right\|=
		\\=
		2\left\| B \one- B (0\cdot \one)  \right\|=
		2\cdot| 1- 0  |= 2 = \diam \B
		.
		\end{multline}
\end{proof}

\begin{remark}
	Оператор $A$ переводит банаховы пределы <<достаточно далеко>> от $\B(\sigma_{2^{2n-1}})$.
	%TODO2: \sigma_{2^{2n}}
	Пусть снова
	\begin{equation}
		y \approx 2\chi {\cup_{m=0}^{\infty}\left[2^{2 m}+1, 2^{2 m+1}\right]} - \one
		.
	\end{equation}
	Рассмотрим $B_1 \in \B(\sigma_{2^{2n-1}})$, $n\in\N$.
	Тогда  $y \approx - \sigma_{2^{2n-1}} y$,
	откуда $B_1 y = B_1 \sigma_{2^{2n-1}} y = 0$.
	Однако для любого $B_2 \in \B$ имеем $B_2 A y = 1$,
	т.е.
	\begin{equation}
		\|B_1 - B_2 A\| \geq \dfrac{\|B_1y - B_2 Ay\|}{\|y\|} = \|B_1y - B_2 Ay\| = 1
		.
	\end{equation}
\end{remark}

\begin{hypothesis}
	Пусть $B_1 \in \B(\sigma_{2^{2n-1}})$, $n\in\N$, и $B_2\in \B$.
	Тогда $\|B_1 - B_2 A\| = 2 = \diam \B$.
\end{hypothesis}

\begin{hypothesis}
	Множество $\B(A) \cap \B (\sigma_{2^{2n}})$ непусто для любого $n\in \N$.
\end{hypothesis}

\begin{hypothesis}
	Если $B\in\ext\B$, то $BA \in \ext \B$.
\end{hypothesis}

Более того, реалистичной видится даже существенно более сильная

\begin{hypothesis}
	Для любого $B\in\B$ выполнено $BA \in \Lin \ext \B$, где $\Lin$ обозначает линейную оболочку,
	т.е. множество всевозможных конечных линейных комбинаций.
\end{hypothesis}
