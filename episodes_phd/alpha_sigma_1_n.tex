Введём, следуя~\cite[p. 131, prop. 2.b.2]{lindenstrauss1979classical},
на $\ell_\infty$ оператор
\begin{equation}
	\sigma_{1/n} x = n^{-1}
	\left(
		\sum_{i=1}^{n} x_i,
		\sum_{i=n+1}^{2n} x_i,
		\sum_{i=2n+1}^{3n} x_i,
		...
	\right).
\end{equation}

Понятно, что если последовательность $x$~--- периодическая с периодом $n$,
то $\alpha(\sigma_{1/n}x)=0$.
Значит, оценить $\alpha(\sigma_{1/n}x)$ снизу через $\alpha(x)$ не удастся.

Для построения верхней оценки нам потребуется следующая

\begin{lemma}
	\label{thm:distance_from_average}
	Пусть
	\begin{equation}
		a=\frac{a_1+...+a_n}{n}, \quad a_1 \leq ... \leq a_n
		.
	\end{equation}
	Тогда
	\begin{equation}
		a_n - a \leq \frac{n-1}{n} (a_n - a_1)
		.
	\end{equation}
\end{lemma}

\begin{proof}
	\begin{multline}
		a_n - a = a_n - \frac{a_1+...+a_n}{n}
		=
		\frac{n-1}{n}a_n - \frac{a_1+...+a_{n-1}}{n}
		\leq
		\\\leq
		\frac{n-1}{n}a_n - \frac{(n-1)a_1}{n}
		=
		\frac{n-1}{n}(a_n - a_1)
		.
	\end{multline}
\end{proof}

\begin{theorem}
	\label{thm:alpha_sigma_1_n}
	Для любого $n\in\N$ и любого $x\in\ell_\infty$ выполнено
	\begin{equation}
		\alpha(\sigma_{1/n} x) \leq \left( 2- \frac{1}{n} \right) \alpha(x)
		.
	\end{equation}
\end{theorem}

\begin{proof}
	Положим
	\begin{equation}
		\alpha_i(x) =
		\max_{i<j\leqslant 2i} |x_i - x_j| =
		\max_{i\leqslant j\leqslant 2i} |x_i - x_j|
		.
	\end{equation}
	Тогда
	\begin{equation}
		\alpha(x) = \varlimsup_{i\to\infty} \alpha_i(x)
		.
	\end{equation}
	Пусть $y=\sigma_n \sigma_{1/n} x$.
	Из теоремы~\ref{thm:alpha_sigma_n} следует, что $\alpha(\sigma_{1/n} x)=\alpha(y)$.
	Сосредоточим наши усилия на оценке $\alpha(y)$.

	Пусть $1\leq j \leq n$.
	Заметим, что
	\begin{multline}
		\alpha_{kn+j}(y)
		=
		\max_{kn+j \leq i \leq 2kn+2j } |y_{kn+j} - y_i|
		=
		\\=
		\mbox{(т.к. $y_{kn+j}=y_{kn+1}=y_{kn+n} = (\sigma_{1/n}x)_k$)}
		=
		\\=
		\max_{kn+n \leq i \leq 2kn+2j } |y_{kn+n} - y_i|
		\leq
		\\\leq
		\mbox{(переходим к максимуму по не меньшему множеству)}
		\leq
		\\\leq
		\max_{kn+n \leq i \leq 2kn+2n } |y_{kn+n} - y_i|
		=
		\alpha_{kn+n}(y)
		.
	\end{multline}

	Итак, $\alpha_{kn+j}(y) \leq \alpha_{kn+n}(y)$,
	значит,
	\begin{equation}
		\label{eq:alpha_sigma_1_n_subseq_limsup}
		\alpha(\sigma_{1/n}x) = \alpha(y) = \varlimsup_{i\to\infty} \alpha_i(y)
		=
		\varlimsup_{k\to\infty} \alpha_{kn+n}(y)
		.
	\end{equation}

	По лемме~\ref{thm:distance_from_average} имеем
	\begin{multline}
		\label{eq:alpha_sigma_1_n_distance}
		|x_{kn+n}-y_{kn+n}|
		\leq
		\frac{n-1}{n}\max_{1\leq i<j \leq n}|x_{kn+i}-x_{kn+j}|
		\leq
		\\\leq
		\frac{n-1}{n} \max_{1\leq i \leq n} \alpha_{kn+i}(x)
		=
		\frac{n-1}{n}\alpha_{kn+i_k}(x)
		.
	\end{multline}

	Из того, что $y_j = \frac{1}{n}(x_{kn+1}+...+x_{kn+n})$,
	следует, что
	\begin{equation}
		\label{eq:alpha_sigma_1_n_alpha_x}
		\max_{kn+n \leq i \leq 2kn+2n } |x_{kn+n} - y_i|
		\leq
		\max_{kn+n \leq i \leq 2kn+2n } |x_{kn+n} - x_i|
		=
		\alpha_{kn+n}(x)
		.
	\end{equation}

	Оценим:
	\begin{multline}
		\alpha_{kn+n}(y)
		=
		\max_{kn+n \leq i \leq 2kn+2n } |y_{kn+n} - y_i|
		=
		\\=
		\max_{kn+n \leq i \leq 2kn+2n } |y_{kn+n} - x_{kn+n} + x_{kn+n} - y_i|
		\leq
		\\\leq
		|y_{kn+n} - x_{kn+n}| + \max_{kn+n \leq i \leq 2kn+2n } |x_{kn+n} - y_i|
		\mathop{\leq}^{\eqref{eq:alpha_sigma_1_n_distance}}
		\\\leq
		\frac{n-1}{n} \alpha_{kn+i_k}(x) + \max_{kn+n \leq i \leq 2kn+2n } |x_{kn+n} - y_i|
		\mathop{\leq}^{\eqref{eq:alpha_sigma_1_n_alpha_x}}
		\frac{n-1}{n} \alpha_{kn+i_k}(x)+\alpha_{kn+n}(x)
		.
	\end{multline}

	С учётом~\eqref{eq:alpha_sigma_1_n_subseq_limsup} имеем
	\begin{multline}
		\alpha(\sigma_{1/n}x)
		=
		\varlimsup_{k\to\infty} \alpha_{kn+n}(y)
		\leq
		\\\leq
		\varlimsup_{k\to\infty} \left( \frac{n-1}{n} \alpha_{kn+i_k}(x)+\alpha_{kn+n}(x) \right)
		=
		\left(2-\frac{1}{n}\right)\alpha(x)
		.
	\end{multline}
\end{proof}

Точность теоремы~\ref{thm:alpha_sigma_1_n} для $n=1$ очевидна.
Для $n=2$ её показывает
\begin{example}
	Положим для всех $p\in\N$:
	\begin{equation}
		x_k=\begin{cases}
			0, & k \leq 2^3, \\
			0, & k = 2^{3p}+1, \\
			1, & k = 2^{3p}+2, \\
			1, & 2^{3p}+3 \leq k \leq 2^{3p+1}+2, \\
			2, & 2^{3p+1}+3 \leq k \leq 2^{3p+1}+4, \\
			1, & 2^{3p+1}+5 \leq k \leq 2^{3(p+1)},
		\end{cases}
	\end{equation}
	тогда
	\begin{equation}
		(\sigma_{1/2}x)_k=\begin{cases}
			0, & k \leq 2^3, \\
			1/2, & k = 2^{3p}+1, \\
			1/2, & k = 2^{3p}+2, \\
			1, & 2^{3p}+3 \leq k \leq 2^{3p+1}+2, \\
			2, & 2^{3p+1}+3 \leq k \leq 2^{3p+1}+4, \\
			1, & 2^{3p+1}+5 \leq k \leq 2^{3(p+1)}.
		\end{cases}
	\end{equation}
	%\begin{table}
	%	\begin{tabular}{c||c|c|c|c|c|c|}
	%		\hline
	%		$k$     & $0..2^3$ & $2^{3p}+1$ & $2^{3p}+2$ & $2^{3p}+3 .. 2^{3p+1}+2$ & $2^{3p+1}+3 .. 2^{3p+1}+4 $ & $2^{3p+1}+5 .. 2^{3(p+1)}$ \\
	%	\end{tabular}
	%\end{table}
	Очевидно, что $\alpha(x)=1$, но $\alpha(\sigma_{1/2}x)=3/2$
	(достигается на $i=2^{3p}+2$, $j=2^{3p+1}+4$).
\end{example}

\begin{hypothesis}
	Оценка теоремы~\ref{thm:alpha_sigma_1_n} точна для любого $n\in\N$.
\end{hypothesis}
