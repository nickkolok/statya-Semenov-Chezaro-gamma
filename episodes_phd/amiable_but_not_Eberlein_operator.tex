\begin{theorem}
	\label{thm:amiable_but_not_Eberlein_exists}
	Существует дружелюбный оператор, не являющийся эберлейновым.
\end{theorem}

\begin{proof}
	Пусть $B_1, B_2 \in \ext \B$.
	Положим
	\begin{equation}
		\label{eq:am_not_eber_def}
		B_3 = B_1 + 2(B_2-B_1) = 2B_2-B_1,
	\end{equation}
	тогда $B_3 \notin \B$.
	Действительно, если $B_3 \in \B$, то из~\eqref{eq:am_not_eber_def} следует, что
	\begin{equation}
		B_2 = \frac{B_1 + B_3}2 \in \B \setminus \ext \B
		.
	\end{equation}
	Введём в рассмотрение оператор $H:\ell_\infty\to\ell_\infty$, определённый равенством
	\begin{equation}
		2Hx = (x_1 + B_3x, x_2 + B_3x, ...) = x + (B_3 x) \cdot \one
		.
	\end{equation}
	Убедимся, что оператор $H$ дружелюбен.
	Действительно, для произвольного $x\in\ell_\infty$ имеем
	\begin{equation}
		2 B_1 H x = B_1 x + B_1 ((B_3 x) \cdot\one) = B_1 x + B_3 x =
		B_1 x + 2 B_2 x - B_1 x = 2B_2 x
		,
	\end{equation}
	откуда $B_1 H = B_2$.

	Убедимся теперь, что оператор $H$ не эберлейнов.
	Пусть $B = BH$ для некоторого $B\in\B$.
	Тогда для всех $x\in\ell_\infty$ имеем
	\begin{equation}
		2Bx = B (x + (B_3 x) \cdot \one)
		,
	\end{equation}
	т.е.
	\begin{equation}
		Bx =  B((B_3 x) \cdot \one)
		,
	\end{equation}
	откуда незамедлительно следует, что $Bx = B_3x$ и в силу произвольности выбора $x$ имеем $B=B_3$.
	Но ранее мы уже показали, что $B_3\notin \B$.
	Полученное противоречие завершает доказательство.
\end{proof}

\begin{hypothesis}
	$B_2H\notin \B$.
\end{hypothesis}

\begin{hypothesis}
	Существуют такие дружелюбный оператор $H:\ell_\infty\to\ell_\infty$ и банахов предел $B\in \B$,
	что $BH \in \B$, но $B_1 H \notin \B$ для любого $B\in \B\setminus\{B\}$.
	Также интересно наложение дополнительных свойств на $B$ и $BH$, например, принадлежности к $\ext\B$, $\B(C)$ и т.д.
\end{hypothesis}
