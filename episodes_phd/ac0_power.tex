Вопрос о покоординатном возведении в степень почти сходящейся к нулю последовательности сколь-либо системно впервые поднят в~\cite{zvol2022ac}.
Возведение в отрицательную степень может выводить из пространства ограниченных последовательностей $\ell_\infty$ вовсе;
например, очевидна следующая
\begin{lemma}
	Пусть $x\in c_0$, $\alpha< 0$ и $x_k\ne 0 $ для любого $k$.
	Тогда
	\begin{equation*}
		x^\alpha = (x_1^\alpha,x_2^\alpha,x_3^\alpha,...) \notin \ell_\infty
		.
	\end{equation*}
\end{lemma}
(Здесь и далее мы полагаем, что возведение в соответствующую степень определено однозначно и в действительных числах;
то есть, если мы пишем $x_k^\alpha$, то мы неявно предполагаем, что значение этого выражения корректно определено.)


Для $x\in ac_0\setminus c_0$ ситуация, вообще говоря, не столь однозначна.

\begin{example}
	\label{example:ac0_pow_signum_classic}
	Рассмотрим классическую почти сходящуюся к нулю последовательность
	\begin{equation}
		x = (1;-1;1;-1;1;-1;...) \in ac_0
		.
	\end{equation}
	Очевидно, что для любой целой нечётной отрицательной степени $\alpha$ имеем $x^\alpha = x \in \ell_\infty$,
	однако для любой целой чётной отрицательной степени $\alpha$ мы получаем $x^\alpha = (1;1;1;1;1;1;...) \in \ell_\infty$.
\end{example}

\begin{example}
	Рассмотрим почти сходящуюся к нулю последовательность
	\begin{equation}
		y = \left(1;-1;\frac12;-\frac12;1;-1;\frac13;-\frac13;1;-1;\frac14;-\frac14;1;-1;...\right) \in ac_0
		.
	\end{equation}
	Очевидно, что
	\begin{equation}
		y^{-1} = \left(1;-1;2;-2;1;-1;3;-3;1;-1;4;-4;1;-1;...\right)  \notin \ell_\infty
		.
	\end{equation}
\end{example}

Возведение в отрицательную степень мы более обсуждать не будем.

Итак, в недавней статье Р.Е. Зволинского~\cite{zvol2022ac} доказаны два следующих факта (теорема 3 и следствие 2 соответственно):
\begin{theorem}
	\label{thm:Zvol_pow_pos}
	Пусть $x \geqslant 0, x \in a c_0$ и $\alpha>0$, тогда $x^\alpha \in a c_0$.
\end{theorem}

\begin{theorem}
	\label{thm:Zvol_pow_composed}
	Пусть $x\in ac_0$, $x = x^+ +x^-$, $x^+\geq 0$, $x^- \leq 0$ и $x^+ \in ac_0$
	(последнее включение эквивалентно условию $x^- \in ac_0$).
	Пусть $n\in\N$ или $n = \frac1{2k+1}$, $k\in\N$.
	Тогда $x^n \in ac_0$.
\end{theorem}

Возникает закономерный вопрос о том, что происходит при возведении в степень почти сходящейся к нулю последовательности,
которую нельзя разложить в сумму знакопостоянных почти сходящихся к нулю последовательностей.
Следующая теорема показывает, что условия теоремы~\ref{thm:Zvol_pow_composed} существенны.

\begin{theorem}
	\label{thm:ac0_pow_even}
	Пусть $x\in ac_0$, $x = x^+ +x^-$, $x^+\geq 0$, $x^- \leq 0$ и $x^+ \notin ac_0$
	(последнее условие эквивалентно условию $x^- \notin ac_0$).
	Пусть $n = 2k$, $k\in\N$.
	Тогда $x^n \notin ac_0$.
\end{theorem}

\begin{proof}
	Рассмотрим последовательность $y = (x^+)^n \geq 0$.

	Предположим, что $y \in ac_0$.
	Тогда $x^+ = y^{1/n}$ и по теореме~\ref{thm:Zvol_pow_pos} выполнено $x^+\in ac_0$,
	что противоречит условию доказываемой теоремы.
	Значит, $y = (x^+)^n \notin ac_0$ и выполнено неравенство $p\left((x^+)^n\right) > 0$.

	Заметим теперь, что  в силу чётности $n$ выполнено $(x^-)^n \geq 0$, откуда $p\left((x^-)^n\right) \geq 0$.
	(Строго говоря, можно по аналогии с $(x^+)^n$ показать, что $(x^-)^n\notin ac_0$ и, следовательно, $p\left((x^-)^n\right) > 0$.)
	В силу построения $x^+$ и $x^-$ мы имеем $\supp x^+ \cap \supp x^- = \varnothing$,
	откуда
	\begin{equation}
		x^n = (x^+ + x^-)^n = (x^+)^n + (x^-)^n
		.
	\end{equation}
	Очевидно, что для любых ограниченных последовательностей $a\geq0$, $b\geq 0$ выполнено неравенство для верхнего функционала Сачестона $p(a+b) \geq p(a)$.
	Отсюда получаем
	\begin{equation}
		p(x^n) = p\left((x^+)^n + (x^-)^n\right) \geq p\left((x^+)^n\right) > 0
		,
	\end{equation}
	что по теореме Сачестона означает, что $x^n \notin ac_0$.
\end{proof}

Для нечётной степени условие разложение в сумму двух знакопостоянных почти сходящихся последовательностей в теореме~\ref{thm:Zvol_pow_composed} тоже существенно.

\begin{example}
	Напомним,
	%TODO: ссылка!
	что любая периодическая последовательность почти сходится к своему среднему по периоду
	(см. следствие~\ref{thm:period_ac_avg}).
	Пусть
	\begin{equation}
		x = (1;1;-2;\ 1;1;-2;\ 1;1;-2;\ ...) \in ac_0
	\end{equation}
	и пусть $\alpha = 3$.
	Тогда
	\begin{equation}
		x^+ = (1;1;0;\ 1;1;0;\ 1;1;0;\ ...) \notin ac_0, \quad x^+ \in ac_{2/3}
		,
	\end{equation}
	\begin{equation}
		x^- = (0;0;-2;\ 0;0;-2;\ 0;0;-2;\ ...) \notin ac_0, \quad x^- \in ac_{-2/3}
		,
	\end{equation}
	\begin{equation}
		x^\alpha = (1;1;-8;\ 1;1;-8;\ 1;1;-8;\ ...) \notin ac_0, \quad x^\alpha \in ac_{-2}
		.
	\end{equation}
\end{example}

Однако для нечётной степени доказать аналог теоремы~\ref{thm:ac0_pow_even} не удастся "--- это показывает пример~\ref{example:ac0_pow_signum_classic}, в котором $x^+\in ac_1$, $x^-\in ac_{-1}$.

В заключение приведём пример, в котором возведение в нечётную степень выводит не только из пространства $ac_0$,
но и из более широкого пространства $ac$.

\begin{example}
	\label{example:cube_out_of_ac0}
	%Напомним, что $\N_k = \{k, k+1, k+2, k+3,...\}$.
	Пусть
	\begin{equation}
		x_n = \begin{cases}
			 0, & \mbox{~если~} n < 2^{10},
			\\
			 1, & \mbox{~если~} n \ge 2^{10}, 2^k\le n < 2^k+3k \mbox{~и~}  n\neq 2^k + 3m, m\in\N,
			\\
			-2, & \mbox{~если~} n \ge 2^{10}, 2^k\le n < 2^k+3k \mbox{~и~}  n  =  2^k + 3m, m\in\N.
		\end{cases}
	\end{equation}
	Таким образом,
	\begin{multline}
		x = (0,0,...,0,0, \; -2, 1, 1, \; -2, 1, 1, \; -2, 1, 1, ..., -2, 1, 1, \; 0, 0, 0, \\ ..., 0, 0, 0, ..., -2, 1, 1, \; -2, 1, 1, ... )
		.
	\end{multline}
	Легко заметить, что в силу критерия Лоренца $x\in ac_0$.
	С другой стороны,
		\begin{equation}
		(x^3)_n = \begin{cases}
			 0, & \mbox{~если~} n < 2^{10},
			\\
			 1, & \mbox{~если~} n \ge 2^{10}, 2^k\le n < 2^k+3k \mbox{~и~}  n\neq 2^k + 3m, m\in\N,
			\\
			-8, & \mbox{~если~} n \ge 2^{10}, 2^k\le n < 2^k+3k \mbox{~и~}  n  =  2^k + 3m, m\in\N.
		\end{cases}
	\end{equation}
	Получаем $p(x^3) = 0$, $q(x^3) = -2$, откуда $x^3 \notin ac$.
\end{example}

Аналогично строится и пример, когда возведение в нечётную степень, наоборот, вводит в пространство $ac_0$.

\begin{example}
	%Напомним, что $\N_k = \{k, k+1, k+2, k+3,...\}$.
	Пусть
	\begin{equation}
		y_n = \begin{cases}
			 0, & \mbox{~если~} n < 2^{10},
			\\
			 1, & \mbox{~если~} n \ge 2^{10}, 2^k\le n < 2^k+3k \mbox{~и~}  n\neq 2^k + 3m, m\in\N,
			\\
			-\sqrt[3]{2}, & \mbox{~если~} n \ge 2^{10}, 2^k\le n < 2^k+3k \mbox{~и~}  n  =  2^k + 3m, m\in\N.
		\end{cases}
	\end{equation}
	Тогда $p(y) = \frac{2-\sqrt[3]2}{3}$, $q(y) = 0$, и потому $y \notin ac_0$.
	Однако легко заметить, что $y^3 = x$ из примера~\ref{example:cube_out_of_ac0} и потому $y^3\in ac_0$.
\end{example}




Рассуждения данного параграфа подталкивают нас к изучению следующего объекта.

Пусть
\begin{equation}
	ac_0^{(2n+1)} = \{ x\in ac_0 : x^{2n+1} \in ac_0\}, \quad n\in\N
	.
\end{equation}
(Множества $ac_0^{(2n)}$ мы в рассмотрение не вводим, поскольку каждое из них совпадает с $\Iac$ в силу теоремы~\ref{thm:ac0_pow_even}.)

Очевидна следующая
\begin{lemma}
	Пусть $x \approx y$.
	В этом случае $x \in ac_0^{(2n+1)}$ тогда и только тогда, когда $y \in ac_0^{(2n+1)}$.
\end{lemma}

\begin{lemma}
	Пусть $x - y \in \Iac$.
	В этом случае $x \in ac_0^{(2n+1)}$ тогда и только тогда, когда $y \in ac_0^{(2n+1)}$.
\end{lemma}

%TODO! Через бином Ньютона.
%\begin{proof}
%	\begin{equation}
%		x
%	\end{equation}
%\end{proof}

\begin{hypothesis}
	$ac_0^{(2n+1)}$ замкнуто относительно сложения.
\end{hypothesis}

\begin{hypothesis}
	$ac_0^{(2n+1)}$ замкнуто относительно умножения.
\end{hypothesis}

\begin{hypothesis}
	$ac_0^{(2n+1)}$ замкнуто (топологически).
\end{hypothesis}

\begin{hypothesis}
	$ac_0^{(2n+1)} = ac_0^{(2n+3)}$ для любого $n$ (или же, наоборот, там есть собственное включение).
\end{hypothesis}

\begin{hypothesis}
	Если предыдущая гипотеза неверна, то встаёт вопрос об исследовании множеств
	$\bigcup_{n\in\N}ac_0^{(2n+1)}$ и 	$\bigcap_{n\in\N}ac_0^{(2n+1)}$.
\end{hypothesis}

\begin{hypothesis}
	Пусть $x\in ac_0^{(2n+1)}$.
	Тогда $x^{2n+1} \in ac_0^{(2n+1)}$.
\end{hypothesis}

\begin{hypothesis}
	Пусть $x\in ac_0^{(2n+1)}$.
	Тогда $\sigma_k x \in ac_0^{(2n+1)}$ для любого $k\in \N$
\end{hypothesis}
