При изучении инвариантности банаховых пределов относительно различных непрерывных линейных операторов
возникает закономерный вопрос о выделении некоторых классов этих операторов.

\begin{definition}
	Будем говорить, что оператор $H : \ell_\infty \to \ell_\infty$ \emph{эберлейнов},
	если $\B (H) \ne \varnothing$.
\end{definition}

Выбор именования этого класса операторов обусловлен тем, что именно Эберлейн в работе~\cite{Eberlein}
впервые сколь-либо системно изучил инвариантные банаховы пределы
(хотя отдельные шаги в этом направлении были сделаны ещё Эгнью и Морсом~\cite{agnew1938linear,agnew1938extensions}).

В работе~\cite{alekhno2018invariant} вводится следующее

\begin{definition}
	\label{def:B-regular_operator}
	Оператор $H : \ell_\infty \to \ell_\infty$ называется \emph{B-регулярным},
	если $H^*\B \subseteq \B$.
	(Или, что то же самое, $BH\in\B$ для любого $B\in\B$.)
\end{definition}

В той же работе с помощью теоремы Брауэра--Шаудера--Тихонова о неподвижной точке~\cite[Corollary  17.56]{aliprantis2006infinite}
доказывается следующая
\begin{theorem}
	\label{thm:B-regular_is_Eberlein}
	Любой B-регулярный оператор "--- эберлейнов.
\end{theorem}
Там же приводится следующее необходимое и достаточное условие В-регулярности.

\begin{theorem}
	\label{thm:crit_B_regularity}
	Оператор $H:\ell_\infty \to \ell_\infty$ является B-регулярным тогда и только тогда, когда:
	\\(i) $H\one \in ac_1$;
	\\(ii) $q(Hx)\geq 0$ для любого $x\geq 0$;
	\\(iii) $H ac_0 \subseteq ac_0$.
\end{theorem}
Легко заметить, что эти условия являются в целом более слабыми, чем достаточные условия эберлейновости
(теорема~\ref{thm:Semenov_Sukochev_conditions}).

\begin{corollary}
	Пусть $x\in ac_\lambda$ и оператор $H$ является В-регулярным.
	Тогда $Hx \in ac_\lambda$.
\end{corollary}

Кроме того, в той же работе~\cite{alekhno2018invariant}
с помощью теоремы Крейна--Мильмана~\cite[теорема  9.14]{aliprantis2006positive}
доказывается следующая
\begin{lemma}
	Пусть оператор $H:\ell_\infty\to\ell_\infty$ является B-регулярным.
	Тогда множество $\ext\B(H)$ непусто.
\end{lemma}

Возникает закономерный вопрос: существует ли эберлейнов оператор, не являющийся B-регулярным?
Интуитивно кажется, что существует, однако в вопросах, связанных с банаховыми пределами,
многие факты оказываются контринтуитивными.
Пример эберлейнова не В-регулярного оператора строится ниже
в теореме~\ref{thm:Eberlein_but_not_B-regular_exists}.

Подобные рассуждения можно продолжить в обе стороны, а именно "--- следующими двумя определениями:

\begin{definition}
	Будем называть оператор $A:\ell_\infty \to \ell_\infty$ \emph{полуэберлейновым}, если $BA\in\mathfrak B$ для некоторого $B\in\mathfrak B$.
\end{definition}

\begin{definition}
	Будем называть оператор $A:\ell_\infty \to \ell_\infty$ \emph{существенно эберлейновым}, если $BA\in\mathfrak B$ для любого $B\in\mathfrak B$ и $BA\ne B$ для некоторого $B\in\mathfrak B$.
\end{definition}

Эти четыре класса операторов (существенно эберлейновы, В-регулярные, эберлейновы, полуэберлейновы "--- в порядке включения)
получены последовательным ослаблением условий, естественных для <<достаточно хороших>> операторов:
$\sigma_k$, $C$
и образуют иерархию по включению.
Возникает закономерный вопрос о совпадении классов.
Ясно, что оператор сдвига $T$ является В-регулярным, но не является существенно эберлейновым, поскольку относительно него любой банахов предел инвариантен по определению.
Более того, ясно, что операторы растяжения $\sigma_k$, $k\in\N_2$, и оператор Чезаро $C$ являются существенно эберлейновыми,
поскольку для каждого из них существуют как инвариантные, так и неинвариантные банаховы пределы.

Существует ли полуэберлейнов оператор, не являющийся эберлейновым?
Положительный (и конструктивный) ответ на этот вопрос даёт теорема~\ref{thm:amiable_but_not_Eberlein_exists} ниже.



Далее возникает вопрос о свойствах классов операторов.
Из определения В-регулярности незамедлительно следует

\begin{lemma}
	\label{lem:B-regular_superposition_and_addition}
	Множество В-регулярных операторов замкнуто относительно суперпозиции и выпуклой комбинации.
\end{lemma}


%Несколько менее очевидна выпуклость множества полуэберлейновых операторов на единичной сфере соответствующего пространства.
%
%\begin{lemma}
%	Пусть $A, H : \ell_\infty \to \ell_\infty$ ~--- полуэберлейновы операторы и $0\leq \lambda \leq 1$.
%	Тогда $W = \lambda A + (1-\lambda) H$ ~--- также полуэберлейнов оператор.
%\end{lemma}
%\begin{proof}
%	Пусть $B_A\in\B(A)$, $B_H\in\B(H)$.
%	Положим $B_W = \lambda B_A + (1-\lambda) B_H$ и рассмотрим произвольный $x\in\ell_\infty$.
%	\begin{equation}
%
%	\end{equation}
%\end{proof}

%%%% Судя по всему, доказательство неверно!
