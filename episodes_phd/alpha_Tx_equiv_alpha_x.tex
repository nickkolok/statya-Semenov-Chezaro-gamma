В данном пункте обсуждаются некоторые свойства множеств
\begin{equation}
	\label{eq:alpha_T^n_x_equiv_alpha_x}
	\{x \in \ell_\infty : \alpha(T^n x) = \alpha(x) \}, ~n\in\N,
\end{equation}
\begin{equation}
	\label{eq:cap_alpha_T^n_x_equiv_alpha_x}
	\bigcap\limits_{n\in\N}\{x \in \ell_\infty : \alpha(T^n x) = \alpha(x) \}
	,
\end{equation}
\begin{equation}
	\label{eq:cup_alpha_T^n_x_equiv_alpha_x}
	\bigcup_{n\in\N}\{x \in \ell_\infty : \alpha(T^n x) = \alpha(x) \}
	.
\end{equation}

\subsection{Незамкнутость относительно сложения}

\begin{theorem}
	Ни одно из множеств
	\eqref{eq:alpha_T^n_x_equiv_alpha_x}, \eqref{eq:cap_alpha_T^n_x_equiv_alpha_x}, \eqref{eq:cup_alpha_T^n_x_equiv_alpha_x}
	не замкнуто по сложению и, следовательно, не является пространством.
\end{theorem}

\begin{proof}
	Построим два таких элемента, принадлежащих множеству \eqref{eq:alpha_T^n_x_equiv_alpha_x} при любых $n\in\N$,
	сумма которых не принадлежит множеству \eqref{eq:alpha_T^n_x_equiv_alpha_x} ни при каких $n\in\N$.
	Пусть $m\in\N_3$.
	Положим

	\begin{equation}
		x_k = \begin{cases}
			\dfrac{1}{2}(-1)^m,  & \mbox{если } k = 2^m     \\
			1,                   & \mbox{если } k = 2^m + 1 \\
			-1,                  & \mbox{если } k = 2^m + 2 \\
			0                    & \mbox{иначе }
		\end{cases}
	\end{equation}

	и

	\begin{equation}
		y_k = \begin{cases}
			\dfrac{1}{2}(-1)^m,  & \mbox{если } k = 2^m     \\
			-1,                  & \mbox{если } k = 2^m + 1 \\
			1,                   & \mbox{если } k = 2^m + 2 \\
			0                    & \mbox{иначе }
		\end{cases}
	\end{equation}

	Так как
	\begin{equation}
		(T^n x)_{2^m-n+1} - (T^n x)_{2^m-n+2} = 2
	\end{equation}
	и
	\begin{equation}
		(T^n y)_{2^m-n+1} - (T^n y)_{2^m-n+2} = -2
		,
	\end{equation}
	то
	\begin{equation}
		\alpha(x) = \alpha(T^n x) = \alpha(y) = \alpha(T^n y) = 2
		.
	\end{equation}
	С другой стороны,
	\begin{equation}
		(x+y)_k = \begin{cases}
			(-1)^m,  & \mbox{если } k = 2^m     \\
			0        & \mbox{иначе }
		\end{cases}
	\end{equation}
	и
	\begin{equation}
		\alpha(x+y) = 2
		,
	\end{equation}
	но в то же время
	\begin{equation}
		\alpha(T^n(x+y)) = 1
	\end{equation}
	(см. пример \ref{ex:alpha_x_neq_alpha_Tx}),
	следовательно, $x+y$ не принадлежит ни одному из множеств
	\eqref{eq:alpha_T^n_x_equiv_alpha_x}, \eqref{eq:cap_alpha_T^n_x_equiv_alpha_x}, \eqref{eq:cup_alpha_T^n_x_equiv_alpha_x}.
\end{proof}




\subsection{Незамкнутость относительно умножения}

\begin{theorem}
	Ни одно из множеств
	\eqref{eq:alpha_T^n_x_equiv_alpha_x}, \eqref{eq:cap_alpha_T^n_x_equiv_alpha_x}, \eqref{eq:cup_alpha_T^n_x_equiv_alpha_x}
	не замкнуто по умножению.
\end{theorem}

\begin{proof}
	Снова построим два таких элемента, принадлежащих множеству \eqref{eq:alpha_T^n_x_equiv_alpha_x} при любых $n\in\N$,
	произведение которых не принадлежит множеству \eqref{eq:alpha_T^n_x_equiv_alpha_x} ни при каких $n\in\N$.
	Пусть $m\in\N_3$.
	Положим

	\begin{equation}
		x_k = \begin{cases}
			(-1)^m,  & \mbox{если } k = 2^m     \\
			1,                   & \mbox{если } k = 2^m + 1 \\
			0                    & \mbox{иначе }
		\end{cases}
	\end{equation}

	и

	\begin{equation}
		y_k = \begin{cases}
			(-1)^{m+1},  & \mbox{если } k = 2^m     \\
			1,                   & \mbox{если } k = 2^m + 2 \\
			0                    & \mbox{иначе }
		\end{cases}
	\end{equation}

	Так как
	\begin{equation}
		(T^n x)_{2^{2m+1}-n} - (T^n x)_{2^{2m+1}-n+1} = -2
	\end{equation}
	и
	\begin{equation}
		(T^n y)_{2^{2m}-n} - (T^n y)_{2^{2m}-n+2} = -2
		,
	\end{equation}
	то
	\begin{equation}
		\alpha(x) = \alpha(T^n x) = \alpha(y) = \alpha(T^n y) = 2
		.
	\end{equation}
	С другой стороны,
	\begin{equation}
		(x\cdot y)_k = \begin{cases}
			(-1)^m,  & \mbox{если } k = 2^m     \\
			0        & \mbox{иначе }
		\end{cases}
	\end{equation}
	и
	\begin{equation}
		\alpha(x+y) = 2
		,
	\end{equation}
	но в то же время
	\begin{equation}
		\alpha(T^n(x \cdot y)) = 1
	\end{equation}
	(см. пример \ref{ex:alpha_x_neq_alpha_Tx}),
	следовательно, $x \cdot y$ не принадлежит ни одному из множеств
	\eqref{eq:alpha_T^n_x_equiv_alpha_x}, \eqref{eq:cap_alpha_T^n_x_equiv_alpha_x}, \eqref{eq:cup_alpha_T^n_x_equiv_alpha_x}.

\end{proof}
