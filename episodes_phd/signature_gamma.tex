Результаты данного параграфа отражены в~\cite{avdeev2024set_DAN_rus}.


В~\cite{semenov2019mainclasses_rus} была введена следующая функциональная характеристика банахова предела.

Обозначим через $\Gamma$ множество всех неубывающих на $[0, 1]$ функций $f$ таких,
что $f(0) = 0$ и $f(1) = 1$. Каждому $B \in \mathfrak B$
ставится в соответствие следующая функция, определенная на $[0, 1]$:
$$
        \gamma(B, t) = B \chi\Bigg(\bigcup^\infty_{n = 1} [2^n, 2^{n + t})\Bigg)
        .
$$
Функция $\gamma(B,t)$ была введена в статье \cite{semenov2019mainclasses_rus}.

Легко увидеть, что $\gamma (B, t) \in \Gamma$ для всех $B \in \mathfrak B$.
Справедливо и обратное. Для любой $f \in \Gamma$ существует такой
$B \in \mathfrak B$, что $\gamma(B, t) = f(t)$ для всех
$t \in [0, 1]$~\cite[Предложение 2]{semenov2019mainclasses_rus}.
В рамках этого параграфа, говоря о функциональной характеристике банахова предела $B$,
мы будем иметь в виду именно $\gamma(B,t)$.

Другая функциональная характеристика -- аналог функции распределения -- была введена в работе Е.~А.~Алехно~\cite{alekhno2015banach}.

В силу монотонности функциональная характеристика имеет не более чем счётное количество точек разрыва.
В~\cite[Теорема 23]{semenov2019mainclasses_rus} доказано, что
для любых $B \in \mathfrak B(C)$, $t \in [0, 1]$ верно равенство $\gamma (B, t) = t$.

В этом параграфе мы приводим пример класса банаховых пределов,
для которых функциональная характеристика является почти константой.

\begin{lemma}
    Пусть $R$ есть оператор взятия подпоследовательности, определенным следующим образом:
	\begin{multline}
		\label{eq:sprawling_R}
		Rx = (x_2; \; x_4; x_5; \; x_{2^k}; x_{2^k + 1}; ... ; x_{2^k + 2^{k-1} - 1}; \; x_{2^{k+1}}; x_{2^{k+1} + 1};...)
		=\\
		(x_2; \; x_4; x_5; \; x_8; x_9; x_{10}; x_{11}; \; x_{16}; x_{17}; ...)
		.
	\end{multline}
    Тогда $\B(R) \ne\varnothing$ и для каждого $B\in \B(R)$ имеет место
    \begin{equation}
        \gamma(B, t) = \chi(0;1]
        .
    \end{equation}
\end{lemma}


\begin{proof}
	Легко показать, что $R$ является B-регулярным
	в силу леммы~\ref{lem:suff_B_reg} (доказательство полностью аналогично лемме~\ref{lem:AT-TA}),
    следовательно, эберлейновым по теореме~\ref{thm:B-regular_is_Eberlein},
    поэтому $\B(R)\ne \varnothing$.

    Возьмем любой $B \in \mathfrak{B}(R)$ и
    \begin{equation}
        y = \chi \cup_{n=1}^\infty \left[ 2^n; 2^{n-1+\log_2 3}\right) =
        \chi \{ 2; \; 4;5; \; 8;9;10;11; \; 16;17;... \}\quad]
        ,
    \end{equation}
    т тогда $Ry=\one$.

	Следовательно, по определению функциональной характеристики,
	$\gamma(B,t) = 1$ для всех $t \geq \log_2 3 - 1 = \log_2 \frac32$.

	Далее, $R^2 x = (x_4; \; x_8; x_9;\; x_{16}; x_{17}; x_{18}; x_{19}; \; x_{32}; x_{33};...)$.
	В силу вложения $\mathfrak{B}(R)\subset\mathfrak{B}(R^2)$
	имеем $B\in\mathfrak{B}(R^2)$.
	Легко убедиться, что $\gamma(B,t) = 1$ для всех $t \geq \log_2 \frac54$.

	Для произвольного $n\in\N$ $B\in\mathfrak{B}(R^n)$
	и потому
	$\gamma(B,t) = 1$ для всех $t \geq \log_2 (1+2^{-n})$.

	В силу произвольности выбора $n$ мы немедленно получаем, что
	\begin{equation}
		\label{eq:gamma_chi_0_1}
		\gamma(B,t) = \chi{(}0;1{]}
	\end{equation}



\end{proof}
