Результаты данного параграфа отражены в~\cite{avdeev2024set_DAN_rus}.


In~\cite{semenov2019mainclasses_rus}, the following functional characteristic of a Banach limit was introduced.

 Обозначим через $\Gamma$ множество всех неубывающих на $[0, 1]$ функций $f$ таких,
    что $f(0) = 0$ и $f(1) = 1$. Каждому $B \in \mathfrak B$
    ставится в соответствие следующая функция, определенная на $[0, 1]$:
    $$
		\gamma(B, t) = B \chi\Bigg(\bigcup^\infty_{n = 1} [2^n, 2^{n + t})\Bigg)
		.
    $$
    Функция $\gamma(B,t)$ была введена в статье \cite{semenov2019mainclasses_rus}.

    Легко увидеть, что $\gamma (B, t) \in \Gamma$ для всех $B \in \mathfrak B$.
    Справедливо и обратное. Для любой $f \in \Gamma$ существует такой
	$B \in \mathfrak B$, что $\gamma(B, t) = f(t)$ для всех
	$t \in [0, 1]$~\cite[Предложение 2]{semenov2019mainclasses_rus}.
    For the sake of brevity, we will can $\gamma(B,t)$ the \emph{signature}
    of Banach limit $B$.

	Другая функциональная характеристика -- аналог функции распределения -- была введена в работе Е.~А.~Алехно~\cite{Alekhno2:TODO}.

	В силу монотонности сигнатура имеет не более чем счётное количество точек разрыва.
	In~\cite[Теорема 23]{semenov2019mainclasses_rus}, it is proved that
	для любых $B \in \mathfrak B(C)$, $t \in [0, 1]$ верно равенство $\gamma (B, t) = t$.




In this section, we provide the examples of Banach limits for that the signature is almost constant.

\begin{lemma}
	Let R be the subsequence operator
	defined by~\eqref{eq:sprawling_r}.
	Then $\B(R) \ne\varnothing$ and for every $B\in \B(R)$ we have
	\begin{equation}
		\gamma(B, t) = \chi(0;1]
		.
	\end{equation}
\end{lemma}

\begin{proof}
	Due to Lemma~\ref{lem:R_is_sprawling}, $R$ is sprawling.
	Thus, $R$ is B-regular by Theorem~\ref{thm:sprawling_is_B-regular} and thus Eberlein due to Theorem~\ref{thm:B-regular_is_Eberlein},
	so $\B(R)\ne \varnothing$.

	Take any $B \in \mathfrak{B}(R)$ and
	\begin{equation}
		y = \chi \cup_{n=1}^\infty \left[ 2^n; 2^{n-1+\log_2 3}\right) =
		\chi \{ 2; \; 4;5; \; 8;9;10;11; \; 16;17;... \}\quad]
		,
	\end{equation}
	then $Ry=\one$.

	Следовательно, по определению сигнатуры, $\gamma(B,t) = 1$ для всех $t \geq \log_2 3 - 1 = \log_2 \frac32$.

	Далее, $R^2 x = (x_4; \; x_8; x_9;\; x_{16}; x_{17}; x_{18}; x_{19}; \; x_{32}; x_{33};...)$.
	В силу вложения $\mathfrak{B}(R)\subset\mathfrak{B}(R^2)$

	TODO: всегда ли оно строгое?

	имеем $B\in\mathfrak{B}(R^2)$.
	Легко убедиться, что $\gamma(B,t) = 1$ для всех $t \geq \log_2 \frac54$.

	TODO: Дальше через всякие там пределы доказывается, что

	\begin{equation}
		\label{eq:gamma_chi_0_1}
		\gamma(B,t) = \chi{[}0;1)
	\end{equation}


	%Ясно, что $B\in\mathfrak{B}(R)$ ~--- достаточное условие для~\eqref{eq:gamma_chi_0_1}.
	%TODO: является ли оно необходимым?



\end{proof}
