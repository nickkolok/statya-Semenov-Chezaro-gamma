На пространстве $\ell_\infty$ определяется $\alpha$--функция следующим равенством:
\begin{equation}
	\alpha(x) = \varlimsup_{i\to\infty} \max_{i<j\leqslant 2i} |x_i - x_j|
	.
\end{equation}

Легко видеть, что $\alpha$--функция неотрицательна.
Более того, эта функция является полунормой на $\ell_\infty$
и естественным образом возникла в работе~\cite[\S 2]{semenov2020invariant_noncommutative}
при исследовании свойств суперпозиции оператора Чезаро $C$ и банаховых пределов.
Как выяснилось, эта полунорма обладает достаточно интересными свойствами сама по себе.


\begin{property}
	\label{thm:alpha_x_triangle_ineq}
	$\alpha$--функция является однородным выпуклым функционалом:
	\begin{equation}
		\alpha(\lambda x+ \mu y)
		\leq
		 \alpha(\lambda x) + \alpha(\mu y) = \lambda \alpha(x) + \mu \alpha(y)
		.
	\end{equation}
\end{property}
%TODO2: доказывать нужно или очевидно?

\begin{property}
	На пространстве $\ell_\infty$ $\alpha$--функция удовлетворяет условию Липшица:
	\begin{equation}\label{alpha_Lipshitz}
		|\alpha(x) - \alpha(y)| \leq 2 \|x-y\|
		.
	\end{equation}
	(и эта оценка точна).
%	TODO2: доказывать нужно или очевидно?
\end{property}

\begin{property}
	Если $y\in c$, то $\alpha(y) = 0$ и $\alpha(x+y) = \alpha(x)$ для любого $x \in \ell_\infty$.
\end{property}

Кроме того, иногда полезно помнить про следующее очевидное
\begin{property}
	\label{thm:alpha_x_leq_limsup_minus_liminf}
	\begin{equation}
		\alpha(x) \leq \varlimsup_{k\to\infty} x_k - \varliminf_{k\to\infty} x_k
		.
	\end{equation}
\end{property}

Отдельный интерес представляет множество
\begin{equation}
	A_0 = \{x\in\ell_\infty : \alpha(x) = 0\}
	.
\end{equation}

Ниже в этой главе мы покажем, что оно является подпространством $\ell_\infty$ и обладает рядом интересных свойств.
Одно из этих свойств доказывается уже не в настоящей главе, в а теореме~\ref{thm:A_0_c_infty_lin}.

Результаты, излагаемые в данной главе, опубликованы в%
~\cite{
our-mz2019ac0,
our-mz2021linearhulls,
avdeev2021vestnik,
our-ped-2018-alpha-Tx,
our-vzms-2018,
}.
