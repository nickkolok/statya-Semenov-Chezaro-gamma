Результаты этого пункта опубликованы в~\cite{our-ped-2018-inf-dim-ker}.


Конечномерность ядра оператора не является необходимым условием существования
банахова предела, инвариантного относительно данного оператора.

Приведём сначала несложную лемму, а потом соответствующий пример.

\begin{lemma}
	%TODO: а не баян ли?
	%\textit{Скорее всего, это настолько очевидно, что это никто не пишет и не доказывает.}

	Пусть $Q:\ell_\infty \to \ell_\infty$.
	\begin{equation}
		\forall(B\in\mathfrak{B})[QB = B]
	\end{equation}
	тогда и только тогда, когда
	\begin{equation}\label{I-Q_to_ac_0}
		I-Q : \ell_\infty \to ac_0,
	\end{equation}
	где $I$~--- тождественный оператор на $\ell_\infty$.

\end{lemma}

\paragraph{Необходимость.}
\begin{equation}
	B((I-Q)x) =
	B(Ix) - B(Qx) =
	Bx - B(Qx)=
	Bx-Bx
	=
	0
	,
\end{equation}
откуда в силу произвольности выбора $B$ и $x$ вкупе с тем, что $B((I-Q)x)=0$,
имеем $(I-Q)x \in ac_0$ и, следовательно, $I-Q : \ell_\infty \to ac_0$.

\paragraph{Достаточность.}
\begin{equation}
	B(Qx) = B((I-(I-Q))x) =
	B(Ix)-B((I-Q)x) =
	B(x) - 0 = B(x).
\end{equation}


\begin{example}
	\begin{equation}
		(Qx)_k =
		\begin{cases}
			0,~\mbox{если}~ k = 2^n, n \in\N,
			\\
			x_k~\mbox{иначе.}
		\end{cases}
	\end{equation}
\end{example}
Очевидно, что $\dim \ker Q = \infty$.

Покажем, что $Q$ удовлетворяет условию (\ref{I-Q_to_ac_0}).
\begin{equation}
	\sum_{k=m+1}^{m+n} x_k - \sum_{k=m+1}^{m+n} (Qx)_k \leqslant (2 + \log_2 n) \|x\|,
\end{equation}
%(TODO:это очевидно??)\\
откуда немедленно
\begin{equation}
	\frac{1}{n}\sum_{k=m+1}^{m+n} x_k - \frac{1}{n}\sum_{k=m+1}^{m+n} (Qx)_k \leqslant \frac{(2 + \log_2 n) \|x\|}{n} \to 0.
\end{equation}

По предыдущему утверждению отсюда следует, что относительно $Q$ инвариантен любой банахов предел.
