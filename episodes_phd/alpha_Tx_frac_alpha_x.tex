В зависимости от выбора $x$ последовательность $\left\{\frac{\alpha(T^n x)}{\alpha(x)}\right\}$
может монотонно сходиться к любому числу из отрезка $\left[\frac{1}{2}; 1\right]$ с любой скоростью
(в том числе и сколь угодно медленно).
Говоря строже, верна следующая

\begin{theorem}
	\label{thm:alpha_beta_T_seq}
	Пусть $\beta_k$~--- монотонная невозрастающая последовательность,
	$\beta_k \to \beta$, $\beta\in\left[\frac{1}{2}; 1\right]$, $\beta_1 \leq 1$.
	Тогда существует такой $x\in\ell_\infty$, что для любого натурального $n$
	\begin{equation}
		\frac{\alpha(T^n x)}{\alpha(x)} = \beta_n.
	\end{equation}
\end{theorem}

\begin{proof}
	Для удобства обозначим $\beta_0 = 1$.
	Это логично, так как
	\begin{equation}
		\frac{\alpha(T^0 x)}{\alpha(x)} = \frac{\alpha(x)}{\alpha(x)} = 1
		.
	\end{equation}

	Пусть $m\geq 3$, $m\in\N$.
	Положим
	\begin{equation}
		x_k = \begin{cases}
			0,  & \mbox{если } k = 2^{2m}     \\
			\beta_l,  & \mbox{если } 2^{2m} < k = 2^{2m+1}-l, l\in\N\cup\{0\}     \\
			\dfrac{1}{2}                    & \mbox{иначе.}
		\end{cases}
	\end{equation}

	Тогда $\alpha(x) = 1$.
	Действительно, $\left| x_{2^{2m}} - x_{2^{2m+1}} \right| =1$,
	а по свойству \ref{thm:alpha_x_leq_limsup_minus_liminf} $\alpha(x) \leq 1$.

	Пусть
	\begin{equation}
		\alpha_{i,n}(x)= \max_{i< j \leq 2i - n} |x_i - x_j|
		,
		\quad
		i>n
		,
	\end{equation}
	тогда
	\begin{equation}
		\alpha(T^n x) = \varlimsup_{i \to \infty} \alpha_{i,n}(x)
		.
	\end{equation}

	Вычислим теперь все $\alpha_{i,n}(x)$.
	Заметим, что
	\begin{multline}
		\alpha_{2^{2m}, n} (x)
		=
		\max_{2^{2m}< j \leq 2^{2m+1} - n} |0 - x_j|
		=
		\max_{2^{2m}< j \leq 2^{2m+1} - n} x_j
		=
		\\=
		\mbox{(замена: $q = 2^{2m+1} - j$, тогда $j = 2^{2m+1} - q$)}
		=
		\\=
		\max_{2^{2m}< 2^{2m+1} - q \leq 2^{2m+1} - n} x_{2^{2m+1} - q}
		=
		\max_{0< 2^{2m} - q \leq 2^{2m} - n} x_{2^{2m+1} - q}
		=
		\max_{n \leq q < 2^{2m}} x_{2^{2m+1} - q}
		=
		\\=
		\max_{n \leq q < 2^{2m}} \beta_q
		=
		%\\=
		\mbox{(в силу невозрастания $\beta_q$)}
		=
		\beta_n
		\geq
		\frac{1}{2}
		.
	\end{multline}

	Пусть теперь $i$ таково, что $2^{2m+1}<i<2^{2m+2}$,
	тогда $x_i = \frac{1}{2}$.
	Так как
	$\forall(j)\left[x_j\in[0;1]\right]$,
	то
	$|x_i - x_j| \leq \frac{1}{2}$
	и, следовательно,
	$\alpha_{i,n}(x)  \leq \frac{1}{2}$.

	Пусть, наконец, $i$ таково, что $2^{2m}<i \leq 2^{2m+2}$,
	тогда
	$x_i = \beta_k \in [1/2;1]$.
	Для таких $j$, что $i<j\leq 2i-n$ и, более того,
	для любых таких $j$, что $2^{2m}<j<2^{2m+2}$
	выполнено $x_j\in[1/2; 1]$
	и, значит, снова $|x_i - x_j| \leq \frac{1}{2}$.

	Таким образом, получаем, что
	\begin{equation}
		\alpha(T^n x) = \varlimsup_{i \to \infty} \alpha_{i,n}(x) = \beta_n
		.
	\end{equation}
\end{proof}
