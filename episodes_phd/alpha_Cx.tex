Ниже приводится расширенная версия материала, опубликованного в
\cite{our-vzms-2018}.

На пространстве ограниченных последовательностей $\ell_\infty$ определяется оператор Чезаро $C$
равенством
\begin{equation}
	(Cx)_n = {1}/{n} \cdot \sum_{k=1}^n x_k
	.
\end{equation}

Можно доказать, что верна
\begin{theorem}
	\label{thm:alpha_Cx_leq_alpha_x}
	%TODO: ссылка?
	$\alpha(Cx) \leqslant \alpha(x)$.
\end{theorem}
Выясняется, что эта оценка достаточно точна.

\subsection{Вспомогательная сумма специального вида}
\begin{lemma}
	Если $p\geq 2$, то
	\begin{equation}\label{summa_drobey}
		\sum_{i=0}^{p-1} \frac{i \cdot 2^i}{p} = \frac{2^p(p-2) + 2}{p}
	\end{equation}
\end{lemma}

% В Демидовиче этого не нашёл

\paragraph{Доказательство.}
Равенство \eqref{summa_drobey} равносильно равенству
\begin{equation}\label{summa_drobey_multiplied}
	\sum_{i=0}^{p-1} i \cdot 2^i = 2^p(p-2) + 2
	.
\end{equation}
Докажем это равенство методом математической индукции.

\paragraph{База индукции.}
Для $p=2$ имеем
\begin{equation}
	\sum_{i=0}^{2-1} i \cdot 2^i = 0 \cdot 2^0 + 1 \cdot 2^1 = 2
\end{equation}
и
\begin{equation}
	2^2(2-2) + 2 = 2
	.
\end{equation}
Видим, что для $p=2$ соотношение \eqref{summa_drobey_multiplied} выполняется.

\paragraph{Шаг индукции.}
Пусть соотношение \eqref{summa_drobey_multiplied} выполняется для $p=m$, $m\geq 2$, т.е.
\begin{equation}\label{summa_drobey_multiplied_m}
	\sum_{i=0}^{m-1} i \cdot 2^i = 2^m(m-2) + 2
	.
\end{equation}

Покажем, что тогда соотношение \eqref{summa_drobey_multiplied} выполняется и для $p=m+1$.
Действительно,
\begin{multline}
	\sum_{i=0}^{(m+1)-1} i \cdot 2^i
	=
	\sum_{i=0}^{m} i \cdot 2^i
	=
	\sum_{i=0}^{m - 1} i \cdot 2^i + m\cdot 2 ^m
	\mathop{=}^{\eqref{summa_drobey_multiplied_m}}
	2^m(m-2) + 2 + m\cdot 2 ^m
	=
	\\=
	m\cdot2^m-2\cdot2^m  + 2 + m\cdot 2 ^m
	=
	m\cdot2^{m+1}-2^{m+1}  + 2
	=
	\\=
	2^{m+1}(m-1)  + 2
	=
	2^{m+1}((m+1)-1)  + 2
	,
\end{multline}
т.е. соотношение \eqref{summa_drobey_multiplied} выполняется и для $p=m+1$,
что и требовалось доказать.



\subsection{Вспомогательный оператор $S$}
Пусть $y\in \ell_\infty$.
Определим оператор $S:\ell_\infty \to \ell_\infty$ следующим образом:
\begin{equation}\label{operator_S}
	(Sy)_k = y_{i+2}, \mbox{ где } 2^i < k \leq 2^i+1
\end{equation}
Этот оператор вводится исключительно для упрощения изложения конструкции.

\begin{example}
	$$
		S(\{1,2,3,4,5,6, ...\}) = \{1,2,3,3,4,4,4,4,5,5,5,5,5,5,5,5,6...\}
	$$
\end{example}

Теперь нам потребуются некоторые свойства оператора $S$.

\begin{lemma}
	\label{thm:alpha_S}
	\begin{equation}\label{alpha_S}
		\alpha(Sx) = \varlimsup_{k\to\infty} |x_{k+1} - x_{k}|
	\end{equation}
\end{lemma}

\paragraph{Доказательство.}

\begin{equation*}
	\alpha(Sx) =
	\varlimsup_{i\to\infty} \sup_{i < j \leq 2i} | (Sx)_i - (Sx)_j | = ...
\end{equation*}
Положим для каждого $i$ число $m_i$ так,
что $m_i = 2^{k_i}$, $i \leq m_i < 2i$
(очевидно, это всегда можно сделать).
\begin{equation*}
	... =
	\varlimsup_{i\to\infty} \max \left\{
		\max_{i   < j \leq m_i} | (Sx)_i - (Sx)_j |,
		\max_{m_i < j \leq 2i } | (Sx)_i - (Sx)_j |
	\right\} =
	...
\end{equation*}
Но при $2^{k_i - 1} < i < j \leq m_i = 2^{k_i}$
имеем $(Sx)_i = (Sx)_j$, и первый модуль обращается в нуль.

\begin{equation*}
	... =
	\varlimsup_{i\to\infty}
		\max_{m_i < j \leq 2i } | (Sx)_i - (Sx)_j |
	=
	...
\end{equation*}
Но при $2^{k_i - 1} < i \leq m_i = 2^{k_i} < j \leq 2^{k_i+1}$
имеем $(Sx)_i = x_{k_i+1}$, $(Sx)_j = x_{k_i+2}$, откуда
\begin{equation}\label{alpha_S_sosedi}
	... =
	\varlimsup_{k\to\infty}
		| x_{k+1} - x_k |
\end{equation}

Лемма доказана.

\begin{lemma}
	\begin{equation}\label{summa_S_less}
		\sum_{k=2}^{2^p} (Sy)_k =
		\sum_{i=0}^{p-1} 2^i y_{i+2}
	\end{equation}
\end{lemma}

\paragraph{Доказательство.}

\begin{equation*}
	\sum_{k=2}^{2^p} (Sy)_k =
	\sum_{i=0}^{p-1} \sum_{k=2^i+1}^{2^{i+1}} (Sy)_k =
	\sum_{i=0}^{p-1} \sum_{k=2^i+1}^{2^{i+1}} y_{i+2} =
	\sum_{i=0}^{p-1} 2^i y_{i+2}
\end{equation*}

Лемма доказана.


\begin{lemma}
	\begin{equation}\label{summa_S}
		\sum_{k=2^i+1}^{2^{i+j+1}} (Sx)_k =
		2^i\sum_{k=2}^{2^{j+1}} (ST^ix)_k
	\end{equation}

	Здесь и далее $(Tx)_n = x_{n+1}$.
\end{lemma}

\paragraph{Доказательство.}

\begin{multline*}
	\sum_{k=2^i+1}^{2^{i+j+1}} (Sx)_k =
	\sum_{m = i}^{i+j}\sum_{k=2^m+1}^{2^{m+1}} (Sx)_k =
	\sum_{m = i}^{i+j}2^m \cdot x_{m+2} =
	\\=
	2^i \cdot \sum_{n = 0}^{j}2^n \cdot x_{n+2+i} =
	2^i \cdot \sum_{n = 0}^{j}2^n (T^i x)_{n+2} =
	2^i \cdot \sum_{k=2}^{2^{j+1}} (ST^i x)_k
\end{multline*}

Лемма доказана.

\subsection{Вспомогательная функция $k_b$}

Введём функцию
\begin{equation}\label{def_k_b}
	k_b(x) = \frac{1}{2b}\left|
		\sum_{k=1}^{b}x_k - \sum_{k=b+1}^{2b}x_k
	\right|
\end{equation}

\begin{lemma}
	\begin{equation}\label{alpha_greater_k_b}
		\alpha (Cx) \geq \varlimsup_{i\to \infty} k_i(x)
	\end{equation}
\end{lemma}

\paragraph{Доказательство.}

\begin{multline*}
	\alpha (Cx) \mathop{=}\limits^{def}
	\varlimsup_{i\to \infty} \sup_{i<j\leq 2i} |(Cx)_i - (Cx)_j| \geq
	\varlimsup_{i\to \infty} |(Cx)_i - (Cx)_{2i}| =
	\\ =
	\varlimsup_{i\to \infty} \left|\frac{1}{i}\sum_{k=1}^i  - \frac{1}{2i}\sum_{k=1}^{2i} \right| =
	\varlimsup_{i\to \infty} \left|\frac{1}{i}\sum_{k=1}^i  - \frac{1}{2i}\sum_{k=1}^{i}- \frac{1}{2i}\sum_{k=i+1}^{2i}\right| =
	\\=
	\varlimsup_{i\to \infty} \left|\frac{1}{2i}\sum_{k=1}^i - \frac{1}{2i}\sum_{k=i+1}^{2i}\right| =
	\varlimsup_{i\to \infty} k_i(x)
\end{multline*}

\paragraph{Примечание.}
Введение функции $k_b(x)$ позволит нам в дальнейшем перейти от работы с оператором Чезаро
к несложным преобразованиям сумм.



\subsection{Основные построения}

Построим вектор $y\in \ell_\infty$ следующим образом:

\begin{equation}\label{y_construction}
	y = \left\{
		0, 0, \frac{1}{p}, \frac{2}{p}, \frac{3}{p},
		...,
		\frac{p-1}{p}, 1, \frac{p-1}{p},
		...,
		\frac{1}{p},
		~~
		\underbrace{
		\phantom{\frac{1}{1}\!\!\!}
			0, 0, 0, ..., 0,
		}_\text{$\phantom{\frac{1}{1}\!\!\!}$\!\!\!$3p+1$ раз}
		~~
		\frac{1}{p}, ...
	\right\}
\end{equation}
так, что
\begin{equation}\label{T_y}
	T^{5p}y = y
\end{equation}
%(0 повторяется $3p+1$ раз или около того, надо будет ещё очень аккуратно пересчитать).


Положим $x = Sy$.
Тогда с учётом (\ref{alpha_S})
\begin{equation}\label{alpha_x}
	\alpha (x) = \alpha (Sy) = \frac{1}{p}
\end{equation}


Оценим $\alpha(Cx)$:

\begin{multline*}
	\alpha (Cx) \mathop{\geq}^{(\ref{alpha_greater_k_b})}
	\varlimsup_{b\to \infty} k_b(x) =
	\varlimsup_{b\to \infty}\frac{1}{2b}\left|
		\sum_{k=1}^{b}x_k - \sum_{k=b+1}^{2b}x_k
	\right| \geq
	\\ \geq
	\varlimsup_{
		i\to \infty,~
		b=2^i~
	}\frac{1}{2^{i+1}}\left|
		\sum_{k=1}^{2^i}(Sy)_k - \sum_{k=2^i+1}^{2^{i+1}}(Sy)_k
	\right| =
	\\=
	\varlimsup_{i\to \infty}\frac{1}{2^{i+1}}\left|
		\sum_{k=1}^{2^i}(Sy)_k - 2^i y_{i+2}
	\right| =
	\varlimsup_{i\to \infty}\left|
		\frac{1}{2^{i+1}}\sum_{k=1}^{2^i}(Sy)_k - \frac{y_{i+2}}{2}
	\right| \geq
\end{multline*}
\begin{multline*}
	\\ \geq
	\varlimsup_{
		m\to \infty,~
		i=5pm+p~
	}\left|
		\frac{1}{2^{5pm+p+1}}\sum_{k=1}^{2^{5pm+p}}(Sy)_k - \frac{y_{5pm+p+2}}{2}
	\right| =
	\\=
	\varlimsup_{m\to \infty}\left|
		\frac{1}{2^{5pm+p+1}}\sum_{k=1}^{2^{5pm+p}}(Sy)_k - \frac{1}{2}
	\right| =
	\\=
	\varlimsup_{m\to \infty}\left|
		\frac{1}{2^{5pm+p+1}}\sum_{k=1}^{2^{5pm}}(Sy)_k
		+
		\frac{1}{2^{5pm+p+1}}\sum_{k=2^{5pm}+1}^{2^{5pm+p}}(Sy)_k
		- \frac{1}{2}
	\right|
	\mathop{=}^{(\ref{summa_S})}
	\\=
	\varlimsup_{m\to \infty}\left|
		\frac{1}{2^{5pm+p+1}}\sum_{k=1}^{2^{5pm}}(Sy)_k
		+
		\frac{1}{2^{5pm+p+1}} \cdot 2^{5pm} \cdot \sum_{k=2}^{2^p}(ST^{5pm}y)_k
		- \frac{1}{2}
	\right|
	\mathop{=}^{(\ref{T_y})}
	\\=
	\varlimsup_{m\to \infty}\left|
		\frac{1}{2^{5pm+p+1}}\sum_{k=1}^{2^{5pm}}(Sy)_k
		+
		\frac{1}{2^{5pm+p+1}} \cdot 2^{5pm} \cdot \sum_{k=2}^{2^p}(Sy)_k
		- \frac{1}{2}
	\right| =
	\\=
	\varlimsup_{m\to \infty}\left|
		\frac{1}{2^{5pm+p+1}}\sum_{k=1}^{2^{5pm}}(Sy)_k
		+
		\frac{1}{2^{p+1}} \sum_{k=2}^{2^p}(Sy)_k
		- \frac{1}{2}
	\right|
	\mathop{=}^{(\ref{summa_S_less})}
	\\=
	\varlimsup_{m\to \infty}\left|
		\frac{1}{2^{5pm+p+1}}\sum_{k=1}^{2^{5pm}}(Sy)_k
		+
		\frac{1}{2^{p+1}} \sum_{i=0}^{p-1}2^i y_{i+2}
		- \frac{1}{2}
	\right| =
	\\=
	\varlimsup_{m\to \infty}\left|
		\frac{1}{2^{5pm+p+1}}\sum_{k=1}^{2^{5pm}}(Sy)_k
		+
		\frac{1}{2^{p+1}} \sum_{i=0}^{p-1}2^i \cdot \frac{i}{p}
		- \frac{1}{2}
	\right|
	\mathop{=}^{(\ref{summa_drobey})}
	\\=
	\varlimsup_{m\to \infty}\left|
		\frac{1}{2^{5pm+p+1}}\sum_{k=1}^{2^{5pm}}(Sy)_k
		+
		\frac{1}{2^{p+1}} \cdot \frac{2^p(p-2)+2}{p}
		- \frac{1}{2}
	\right| =
	\\=
	\varlimsup_{m\to \infty}\left|
		\frac{1}{2^{5pm+p+1}}\sum_{k=1}^{2^{5pm}}(Sy)_k
		+
		\frac{1}{2} \cdot \frac{p-2}{p} + \frac{1}{p 2^p}
		- \frac{1}{2}
	\right| =
	\\=
	\varlimsup_{m\to \infty}\left|
		\frac{1}{2^{5pm+p+1}}\sum_{k=1}^{2^{5pm}}(Sy)_k
		-
		\frac{1}{p} + \frac{1}{p 2^p}
	\right| =
	\\=
	\varlimsup_{m\to \infty}\left|
		\frac{1}{2^{5pm+p+1}}\sum_{k=1}^{2^{5pm-2p}}(Sy)_k
		+
		\frac{1}{2^{5pm+p+1}}\sum_{k=2^{5pm-2p}+1}^{2^{5pm}}(Sy)_k
		-\frac{1}{p} + \frac{1}{p 2^p}
	\right| =
	\\\mbox{но во второй сумме все $(Sy)_k$ --- нули по построению}
\end{multline*}
\begin{multline*}
	\\=
	\varlimsup_{m\to \infty}\left|
		\frac{1}{2^{5pm+p+1}}\sum_{k=1}^{2^{5pm-2p}}(Sy)_k
		-\frac{1}{p} + \frac{1}{p 2^p}
	\right| = h
\end{multline*}

Но $0 \leq (Sy)_k \leq 1$,
значит,
$$
	\frac{1}{2^{5pm+p+1}}\sum_{k=1}^{2^{5pm-2p}}(Sy)_k
	\leq
	\frac{1}{2^{5pm+p+1}} \cdot 2^{5pm-2p}
	=
	\frac{1}{2^{3p+1}}
$$
Модуль раскрываем со знаком ``-''

\begin{multline*}
	h=
	\varlimsup_{m\to \infty} \left(
		\frac{1}{p} (1-2^{-p})
		- \frac{1}{2^{3p+1}}
	\right) =
	\frac{1}{p} (1-2^{-p})
	- \frac{1}{2^{3p+1}}
	= \\ =
	\frac{1}{p} (1-2^{-p})
	- \frac{1}{2^{2p+1}} \cdot 2^{-p}
	>
	\frac{1}{p} (1-2^{-p})
	- \frac{1}{p} \cdot 2^{-p}
	=
	\frac{1}{p} (1-2^{-p+1})
\end{multline*}


Таким образом,
$$
	\frac{\alpha(Cx)}{\alpha(x)} \geq
	\frac{	\frac{1}{p} (1-2^{-p+1}) }{\frac{1}{p}} =
	1-2^{-p+1}
$$

Рассматривая $x$ как функцию от $p$, имеем:
$$
	\sup_{p\in\N} \frac{\alpha(Cx(p))}{\alpha(x(p))} \geq
	\sup_{p\in\N} (1-2^{-p+1}) =
	1
	.
$$
Таким образом, может быть сформулирована следующая

\begin{theorem}
	\label{thm:alpha_Cx_no_gamma}
	\begin{equation}
		\sup_{x\in\ell_\infty, \alpha(x)\neq 0} \frac{\alpha(Cx)}{\alpha(x)}=1
		.
	\end{equation}
\end{theorem}

\subsection{Некоторые гипотезы}

\begin{hypothesis}
	Пусть $x\in\ell_\infty$ и $0 \leq x_k \leq 1$.
	Тогда для любого $n\in\N$
	\begin{equation}
		\alpha(C^n x) \leq \frac{1}{2^n}
		.
	\end{equation}
\end{hypothesis}

\begin{hypothesis}
	Для любых $x\in\ell_\infty$ и $n\in\N$
	\begin{equation}
		\alpha(C^n x) - \alpha(C^{n+1} x) \leq \alpha(C^{n-1} x) - \alpha(C^{n} x)
		.
	\end{equation}
\end{hypothesis}

\begin{hypothesis}
	Для любого $0<a<1$ существует такой $x\in\ell_\infty$, что
	\begin{equation}
		\lim_{m\to\infty} \frac{\alpha(C^m x)}{a^m} = \infty
		.
	\end{equation}
\end{hypothesis}
