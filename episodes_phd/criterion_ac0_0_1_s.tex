Результаты этого пункта опубликованы в~\cite{our-mz2019ac0}.

\paragraph{Задача о пасьянсе из нулей и единиц.}

Эта задача основана на идеях, изложенных в \cite[\S 5]{Semenov2014geomprops}.

Пусть $n_i$~--- строго возрастающая последовательность натуральных чисел,
\begin{equation}
	\label{eq:definition_M_j}
	M(j) = \liminf_{i\to\infty} n_{i+j} - n_i,
\end{equation}
\begin{equation}
	x_k = \left\{\begin{array}{ll}
		1, & \mbox{~если~} k = n_i
		\\
		0  & \mbox{~иначе~}
	\end{array}\right.
\end{equation}

Так как $M(j)$ есть нижний предел последовательности натуральных чисел,
то он всегда достигается,
т.е. $M(j)\in\N$.

Более того, для любого $j$ существует лишь конечное количество отрезков длины $M(j)$,
содержащих более $j$ единиц,
и бесконечное количество отрезков длины $M(j)$,
содержащих ровно $j$ единиц.

Через $E_j$ будем обозначать конец последнего отрезка длины $M(j)$,
содержащего более $j$ единиц.

\begin{lemma}
	\label{thm:lim_M(j)/j_neobh}
	Если $x \in ac_0$, то
	\begin{equation}\label{lim_M(j)/j}
		\lim_{j \to \infty} \frac{M(j)}{j} = +\infty
		.
	\end{equation}
\end{lemma}

\begin{proof}
Очевидно, если
\begin{equation}
	\liminf_{j \to \infty} \frac{M(j)}{j} = +\infty
	,
\end{equation}
то выполнено \eqref{lim_M(j)/j}.
Предположим противное:
\begin{equation}
	\liminf_{j \to \infty} \frac{M(j)}{j} = s < +\infty
	.
\end{equation}
Очевидно, что в таком случае $s>0$.

По определению нижнего предела найдётся счётное множество
$J\subset\N$ такое, что
\begin{equation}
	\forall(j\in J)\left[s \leq \frac{M(j)}{j} \leq s+1 \right],
\end{equation}
т.е. для любого $j\in J$ существует бесконечно много отрезков длины $j\cdot(s+1)$,
на каждом из которых не менее $j$ единиц.

Т.к. $x\in ac_0$, то
\begin{equation}\label{Lorencz_ac0_epsilon}
	\forall(\varepsilon>0)
	\exists(n_0\in\N)
	\forall(n \geq n_0)
	\forall(m\in\N)
	\left[
		\frac{1}{n} \sum_{k=m+1}^{m+n}x_k < \varepsilon
	\right]
	.
\end{equation}

Положим $\varepsilon = 1/(s+2)$ и отыщем $n_0$.
Положим $n\in J$, $n\geq n_0$.
(Такое $n$ всегда найдётся, т.к. $J$ счётно и $J\subset\N$.)
Выберем $m$ так, чтобы отрезок длины $n\cdot(s+1)$,
содержащий не менее $n$ единиц,
начинался с $m+1$.
Тогда
\begin{equation}
	\frac{1}{n\cdot(c+1)}\sum_{k=m+1}^{m+n\cdot(s+1)}x_k
	\geq
	\frac{1}{n\cdot(s+1)} \cdot n
	=
	\frac{1}{s+1}
	>
	\frac{1}{s+2}
	=
	\varepsilon,
\end{equation}
что противоречит \eqref{Lorencz_ac0_epsilon}.

Полученное противоречие завершает доказательство.
\end{proof}

\begin{lemma}
	\label{thm:lim_M(j)/j_dost}
	Если
	\begin{equation}\label{lim_M(j)/j_dost}
		\lim_{j \to \infty} \frac{M(j)}{j} = +\infty
		,
	\end{equation}
	то $x \in ac_0$.
\end{lemma}

\begin{proof}
	По определению предела \eqref{lim_M(j)/j_dost} означает, что
	\begin{equation}\label{lim_M(j)/j_infty_def}
		\forall(S  \in\N)
		\exists(j_0\in\N)
		\forall(j \geq j_0)
		\left[
			\frac{M(j)}{j}>S
		\right]
		.
	\end{equation}
	Покажем, что выполнен модифицированный критерий Лоренца почти сходимости последовательности к нулю
	\eqref{crit_pos_ac0}, т.е.
	\begin{equation}
		\forall(b  \in\N)
		\exists(n_0\in\N)
		\exists(m_0\in\N)
		\forall(n\geq n_0)
		\forall(m\geq m_0)
		\\
		\left[
			\frac{1}{n}
			\sum_{k=m+1}^{m+n} x_k
			<
			\frac{1}{b}
		\right]
		.
	\end{equation} Действительно, зафиксируем $b$.
	Используя \eqref{lim_M(j)/j_infty_def} и положив $S=2b$,
	отыщем $j_0$ такое, что для любого $j\geq j_0$ выполнено
	$M(j)>2bj$.
	Положим $n_0 = 2bj_0$.
	Выберем
	$$
		m_0 = 2+\max_{1\leq j \leq j_0} E_j
		.
	$$

	Тогда для любых $m\geq m_0$ и $n\geq n_0$ имеем
	\begin{equation}
		\frac{1}{n} \sum_{k=m+1}^{m+n} x_k
		<
		\frac{1}{n} \cdot \left( \frac{n}{M(j_0)} + 1 \right) j_0
		=
		\frac{j_0}{M(j_0)} + \frac{j_0}{n}
		\leq
		\frac{1}{2b} + \frac{j_0}{n}
		\leq
		%\\ \leq
		\frac{1}{2b} + \frac{j_0}{n_0}
		=
		\frac{1}{2b} + \frac{1}{2b}
		=
		\frac{1}{b}
		,
	\end{equation}
	т.е. условие критерия выполнено
	и $x\in ac_0$,
	что и требовалось доказать.
\end{proof}

Последовательность $\{M(j)\}$, как легко выяснить, удовлетворяет некоторым условиям.
