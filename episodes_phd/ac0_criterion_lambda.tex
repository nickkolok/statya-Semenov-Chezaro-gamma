Определим (нелинейный) оператор $\lambda$--порога $A_\lambda$ на пространстве $\ell_\infty$.
Для $x = (x_1, x_2, ...) \in \ell_\infty$ положим
\begin{equation}
	(A_\lambda x)_k = \begin{cases}
		1, & \mbox{~если~} x_k \geq \lambda
		\\
		0  & \mbox{~иначе.~}
	\end{cases}
\end{equation}

\begin{theorem}
	[Пороговый критерий почти сходимости к нулю неотрицательной последовательности]
	Пусть $x\in\ell_\infty$, $x\geq 0$.
	Тогда
	\begin{equation}
		x\in ac_0 \Leftrightarrow
		\forall(\lambda>0)[A_\lambda x \in ac_0]
		.
	\end{equation}
\end{theorem}

\paragraph{Необходимость.}
Пусть $x\in ac_0$.
Зафиксируем $\lambda > 0$.
Пусть $y=A_\lambda x$, тогда
\begin{equation}
	(\lambda y)_k = \begin{cases}
		\lambda, & \mbox{~если~} x_k \geq \lambda
		\\
		0  & \mbox{~иначе~}
	\end{cases}
\end{equation}
Таким образом, $0 \leq \lambda y \leq x$.
Следовательно, если $x \in ac_0$,
то $\lambda y \in ac_0$ и $y \in ac_0$,
что и требовалось доказать.

\paragraph{Достаточность.}
Очевидно, что при $\|x\| = 0$, т.е. $x = (0,0,0,0,...)$,
утверждение теоремы верно.
Пусть $\|x\| \ne 0$.
В силу того, что $ac_0$ является линейным пространством,
\begin{equation}
	x\in ac_0 \Leftrightarrow
	\frac{x}{\|x\|}\in ac_0
	.
\end{equation}
Поэтому, не теряя общности, будем полагать $\|x\|\leq 1$.
Более того,
\begin{equation}
	A_\lambda x \in ac_0 \Leftrightarrow
	(1-\lambda)A_\lambda x \in ac_0
	.
\end{equation}

Предположим противное, т.е. $x\notin ac_0$,
но $\forall(\lambda>0)[(1-\lambda)A_\lambda x \in ac_0]$.
Запишем покванторное отрицание критерия Лоренца:
\begin{equation}\label{ac0_lambda_Lorencz_neg}
	\exists(\varepsilon_0 > 0)
	\forall(n_0 \in \N)
	\exists(n > n_0)
	\exists(m \in \N)
	\left[
		\frac{1}{n}\sum_{k=m+1}^{m+n} x_k \geq \varepsilon_0
	\right]
	.
\end{equation}
Найдём такое $\varepsilon_0$ и положим
\begin{equation}
	\varepsilon = \min\left\{ \frac{\varepsilon_0}{2}, \frac{1}{2} \right\}
	.
\end{equation}
Легко видеть, что
\begin{equation}\label{ac0_lambda_Lorencz_neg_epsilon}
	\forall(n_0 \in \N)
	\exists(n > n_0)
	\exists(m \in \N)
	\left[
		\frac{1}{n}\sum_{k=m+1}^{m+n} x_k > \varepsilon
	\right]
	.
\end{equation}
(знак неравенства сменился на строгий, это будет играть ключевую роль в дальнейших выкладках).

Построим последовательность
\begin{equation}
	y = \left( 1 - \frac{\varepsilon}{2} \right) A_{\varepsilon/2} x
	.
\end{equation}
Заметим, что $y\in ac_0$, т.е. по критерию Лоренца
\begin{equation}\label{ac0_lambda_Lorencz}
	\forall(\varepsilon_1 > 0)
	\exists(n_1 \in \N)
	\forall(n' > n_1)
	\forall(m' \in \N)
	\left[
		\frac{1}{n}\sum_{k=m+1}^{m+n} y_k < \varepsilon_1
	\right]
	.
\end{equation}

Положим в \eqref{ac0_lambda_Lorencz} $\varepsilon_1 = \varepsilon/2$
и отыщем $n_1$ в соответствии с квантором существования.
Положим далее в \eqref{ac0_lambda_Lorencz_neg_epsilon}
$n_0 = n_1$ и отыщем $n$ и $m$.
Положим в \eqref{ac0_lambda_Lorencz} $n' = n$, $m' = m$.
Тогда получим, что по \eqref{ac0_lambda_Lorencz_neg_epsilon}
\begin{equation}
	\frac{1}{n}\sum_{k=m+1}^{m+n} x_k > \varepsilon
	.
\end{equation}
С другой стороны, по \eqref{ac0_lambda_Lorencz}
\begin{equation}
	\frac{1}{n}\sum_{k=m+1}^{m+n} y_k < \varepsilon/2
	.
\end{equation}
Вычитая, получим
\begin{equation}
	\frac{1}{n}\sum_{k=m+1}^{m+n} (x_k - y_k) > \varepsilon/2
	.
\end{equation}
Если среднее арифметическое чисел вида $x_k - y_k$ больше $\varepsilon/2$,
то существует хотя бы один индекс $k$ такой, что $x_k - y_k > \varepsilon/2$.

Предположим, что $k$ таково, что $x_k < \varepsilon/2$.
Тогда $y_k = 0$ и $x_k - y_k < \varepsilon/2$.
Значит, предположение неверно и $x_k \geq \varepsilon/2$.
Тогда $y_k = 1-\varepsilon/2$ и с учётом $\|x\|\leq 1$ имеем
\begin{equation}
	x_k - y_k \leq 1- y_k = 1 - (1-\varepsilon/2) = \varepsilon/2
	.
\end{equation}
Следовательно, требуемого индекса $k$ не существует,
и \eqref{ac0_lambda_Lorencz_neg} не выполнено.

Полученное противоречие завершает доказательство.

Сформулируем теперь этот критерий в терминах функций $M^{(\lambda)}(j)$.

\begin{theorem}
	\label{thm:crit_ac0_Mj_lambda}
	Пусть $x\in\ell_\infty$, $x \geq 0$, $\lambda>0$.
	Обозначим через $n^{(\lambda)}_i$ возрастающую последовательность
	индексов таких элементов $x$, что $x_k \geq \lambda$ тогда и только тогда,
	когда $k=n^{(\lambda)}_i$ для некоторого $i$.
	Обозначим
	\begin{equation}
		M^{(\lambda)}(j) = \liminf_{i\to\infty} n^{(\lambda)}_{i+j} - n^{(\lambda)}_i
		.
	\end{equation}


	Тогда для того, чтобы $x\in ac_0$, необходимо и достаточно, чтобы
	для любого $\lambda>0$ было выполнено
	\begin{equation}
		\lim_{j \to \infty} \frac{M^{(\lambda)}(j)}{j} = +\infty
		.
	\end{equation}
\end{theorem}
