\documentclass[a4paper,openbib]{article}
\usepackage{amsmath}
\usepackage[utf8]{inputenc}
\usepackage[english,russian]{babel}
\usepackage{amsfonts,amssymb}
\usepackage{latexsym}
\usepackage{euscript}
\usepackage{enumerate}
\usepackage{graphics}
\usepackage[dvips]{graphicx}
\usepackage{geometry}
\usepackage{wrapfig}
\usepackage[colorlinks=true,allcolors=black]{hyperref}
\usepackage{enumitem}


\righthyphenmin=2


\geometry{left=2cm}% левое поле
\geometry{right=1cm}% правое поле
\geometry{top=2cm}% верхнее поле
\geometry{bottom=2cm}% нижнее поле



\DeclareMathOperator{\supp}{supp}

\usepackage{cite}


\usepackage{amsthm}
\theoremstyle{definition}
\newtheorem{lemma}{Лемма}
\newtheorem{theorem}[lemma]{Теорема}
\newtheorem{example}[lemma]{Пример}
\newtheorem{property}[lemma]{Свойство}
\newtheorem{remark}[lemma]{Замечание}
\newtheorem{definition}{Определение}
\newtheorem{designation}[lemma]{Обозначение}
\newtheorem{corollary}{Следствие}[lemma]


%Only referenced equations are numbered
\usepackage{mathtools}
\mathtoolsset{showonlyrefs}

%\mathtoolsset{showonlyrefs=false}
% (an equation/multline to be force-numbered)
%\mathtoolsset{showonlyrefs=true}


\begin{document}

УДК  { 517.982.276 % Пространства последовательностей и матриц
     + 511.172    }% Мультипликативная структура кольца целых чисел (включая делимость, наименьшее общее кратное, наибольший общий делитель, сравнения, степенные вычеты, квадратичные вычеты)

MSC~
	46B45,
	11A51,
	11B57



\begin{center}
	Почти сходящиеся последовательности из 0 и 1 и простые числа
	\footnote{
		Работа выполнена в Воронежском госуниверситете при поддержке РНФ, грант 19-11-00197.
	}
\end{center}

\begin{center}
	{Н.~Н.~Авдеев$^{2}$\\[4pt]}
	{\rm\footnotesize{$^2$\,
	Воронежский государственный университет\\
	Россия, 394006, Университетская пл., 1\\
	E-mail: nickkolok@mail.ru, avdeev@math.vsu.ru}}
\end{center}


\begin{flushright}
	\textit{
		Светлой памяти профессора \\
		Александра Дмитриевича Баева
	}
\end{flushright}


\paragraph{Аннотация.}
Ограниченная последовательность $(x_1, x_2, ...)\in \ell_\infty$
называется почти сходящейся, если
$Bx$ принимает постоянное значение
для любого банахова предела $B$.
Работа посвящена изучения почти сходимости последовательностей из 0 и 1.


\paragraph{Ключевые слова:}
	ограниченная последовательность,
	пространство ограниченных последовательностей,
	банахов предел,
	функционал Сачестона,
	почти сходящаяся последовательность,
	последовательности из нулей и единиц,
	разложение на множители,
	подмножеcтва натуральных чисел

\paragraph{Abstract.}
A bounded sequence $(x_1, x_2, ...)\in \ell_\infty$
is called almost convergent if the value of $Bx$
does not depend on the choice of Banach limit $B$.
The paper is devoted to almost convergence of 0-1-sequences.

\paragraph{Keywords:}
	bounded sequence,
	space of bounded sequences,
	Banach limit,
	Sucheston functional,
	0-1-sequences,
	factorization,
	subsets of positive integers



%\section{Введение}
\paragraph{\S 1.}

Рассмотрим пространство ограниченных последовательностей $\ell_\infty$ с обычной нормой
\begin{equation*}
	\|x\| = \sup_{k\in\mathbb{N}} |x_k|
	,
\end{equation*}
где $\mathbb{N}$ "--- множество натуральных чисел,
$x=(x_1, x_2, ..., x_n, ...)$,
и обычной полуупорядоченностью.


\begin{definition}
	Линейный функционал $B\in \ell_\infty^*$ называется банаховым пределом,
	если
	\begin{enumerate}[label=(\roman*)]
		\item
			$B\geqslant0$, т.~е. $Bx \geqslant 0$ для $x \geqslant 0$,
		\item
			$B\mathbb{I}=1$, где $\mathbb{I} =(1,1,\ldots)$,
		\item
			$B(Tx)=Bx$ для всех $x\in \ell_\infty$, где $T$~---
		оператор сдвига, т.~е. $T(x_1,x_2,\ldots)=(x_2,x_3,\ldots)$.
	\end{enumerate}
\end{definition}
Множество всех банаховых пределов обозначим через $\mathfrak{B}$.
Существование банаховых пределов было анонсировано С. Мазуром \cite{Mazur} и позднее доказано в книге С.~Банаха~\cite{banach2001theory_rus}.
%
Лоренц доказал~\cite{lorentz1948contribution}, что существуют почти сходящиеся последовательности ---
такие последовательности из $\ell_\infty$,
на которых все банаховы пределы принимают одинаковые значения.
\begin{theorem}[критерий Лоренца]
	Для заданного $r\in\mathbb{R}$ равенство $Bx=r$ выполнено для всех $B\in\mathfrak{B}$
	тогда и только тогда, когда
	\begin{equation}
		\label{eq:crit_Lorentz}
		\lim_{n\to\infty} \frac{1}{n} \sum_{k=m+1}^{m+n} x_k = r
	\end{equation}
	равномерно по всем $m\in\mathbb{N}$.
\end{theorem}


Пространство почти сходящихся последовательностей будем обозначать через $ac$.
%
Уточняя результат Лоренца, Сачестон установил~\cite{sucheston1967banach}, что
для любых $x\in \ell_\infty$ и $B\in\mathfrak{B}$
\begin{equation}\label{Sucheston}
	q(x) \leqslant Bx \leqslant p(x)
	,
\end{equation}
где
\begin{equation*}
	q(x) = \lim_{n\to\infty} \inf_{m\in\mathbb{N}}  \frac{1}{n} \sum_{k=m+1}^{m+n} x_k
	\mbox{~~~~и~~~~}
	p(x) = \lim_{n\to\infty} \sup_{m\in\mathbb{N}}  \frac{1}{n} \sum_{k=m+1}^{m+n} x_k
\end{equation*}
называют нижним и верхним функционалом Сачестона соответственно.
Заметим, что $p(x) = -q(-x)$.
Неравенства \eqref{Sucheston} точны:
для данного $x$ для любого $r\in[q(x); p(x)]$ найдётся банахов предел
$B\in\mathfrak{B}$ такой, что $Bx = r$.

Легко видеть, что множество таких $x\in\ell_\infty$, что $p(x)=q(x)$,
в точности является пространством почти сходящихся последовательностей $ac$.
Будем говорить, что $x\in ac_0$, если $x\in ac$ и $p(x)=q(x)=0$.
Таким образом, функционалы Сачестона для почти сходимости играют такую же роль,
как верхний и нижний пределы для <<обычной>> сходимости.
Более подробно о свойствах почти сходящихся последовательностей и банаховых пределах см.~\cite{semenov2020geomBL}.


Отдельный интерес представляет множество всех последовательностей из 0 и 1,
которое в дальнейшем мы будем обозначать через $\Omega$.
Понятно, что каждый $x\in \Omega$ можно отождествить с подмножеством множества натуральных чисел
$\supp x \subset \mathbb{N}$.
%
В настоящей статье изучается взаимосвязь значений, которые могут принимать функционалы Сачестона
на последовательностях $x\in \Omega$, и мультипликативных свойств множеств $\supp x$.


%\section{Добавление множителей}
\paragraph{\S 2.}

Пользуясь критерием Лоренца~\eqref{eq:crit_Lorentz},
нетрудно доказать следующую лемму.

\begin{lemma}
	\label{lem:simple_powers_in_ac0}
	Пусть $x_n = m^n$, $m\in\mathbb{N}$, $m\geq 2$.
	Тогда
	$
		\chi_x \in ac_0
		.
	$
\end{lemma}

Утверждение леммы~\ref{lem:simple_powers_in_ac0} может быть обобщено следующим образом.

\begin{lemma}
	Пусть $y = \{y_n\}$ --- строго возрастающая последовательность,
	$\chi_y\in\Omega \cap ac_0$ и
	$m \in \mathbb{N}$.
	Тогда $x\in ac_0$, где последовательность $x=\{x_k\}$ определена соотношением
	\begin{equation}
		x_k = \begin{cases}
			1, &\mbox{~если~} k = y_i \cdot m^j \mbox{~для некоторых~} i,j\in\mathbb{N},
			\\
			0  &\mbox{~иначе}
			.
		\end{cases}
	\end{equation}
\end{lemma}


\begin{corollary}
	\label{cor:ac0_powers_finite_set_of_numbers}
	Пусть $\{p_1, ..., p_k\} \subset \mathbb{N}$,
	\begin{equation}
		x_k = \begin{cases}
			1, &\mbox{~если~} k = p_1^{j_1}\cdot p_2^{j_2}\cdot ... \cdot p_k^{j_k} \mbox{~для некоторых~} j_1,...,j_k\in\mathbb{N},
			\\
			0  &\mbox{~иначе}.
		\end{cases}
	\end{equation}
	Тогда $x\in ac_0$.
\end{corollary}

%\section{Исключение множителей и нижний функционал Сачестона}
\paragraph{\S 3.}


\begin{definition}
	Через $\psi(A)$, где $A\subset\mathbb{N}\setminus\{1\}$,
	будем обозначать последовательность из нулей и единиц,
	определяемую соотношением
	\begin{equation}
		(\psi(A))_k = \begin{cases}
			0, & \mbox{~если~} k = m\cdot a, m \in \mathbb{N}, a\in A,
			\\
			1  & \mbox{~иначе.}
		\end{cases}
	\end{equation}
\end{definition}

Так, например,
\begin{gather*}
	\psi(\{2\}) = \psi(\{2, 4\}) = \psi(\{2,4,8,16,...\}) = (1,0,1,0,1,0,1,0,1,0,...),
\\
	\psi(\{3\}) = \psi(\{3,9,27,...\}) = (1,1,0,\;1,1,0,\;1,1,0,\;1,1,0,\;1,1,0,\;...),
\\
	\psi(\{2,3\}) = \psi(\{2,3,6\}) = (1,0,0,0,1,0,\;1,0,0,0,1,0,\;1,0,0,0,1,0,...).
\end{gather*}


\begin{lemma}
	Для любого непустого $A\subset \mathbb{N} $ выполнено $\psi(A) \notin ac_1$.
\end{lemma}

При дополнительных ограничениях верно и более сильное утверждение.

\begin{theorem}
	\label{lem:ac0_primes_infinity_mutually_prime_subset}
	Пусть $A'$ "--- бесконечное подмножество попарно взаимно простых чисел
	(т.е. для любых двух чисел $a_1, a_2 \in A'$ их наибольший общий делитель равен единице).
	Тогда для любого $A \supset A' $ выполнено $q(\psi(A))=0$.
\end{theorem}

\begin{definition}
	Будем говорить, что множество $A\subset\mathbb{N}$ обладает $P$-свойством,
	если для любого $n\in\mathbb{N}$ найдётся набор попарно взаимно простых чисел
	$
		\{a_{n,1}, a_{n,2}, ..., a_{n,n}  \} \subset A
		.
	$
\end{definition}

В условии теоремы~\ref{lem:ac0_primes_infinity_mutually_prime_subset}
для множества $A$ достаточно потребовать $P$-свойства.
Интересно, что на самом деле $P$-свойство эквивалентно условиям,
наложенным на множество $A$ в теореме~\ref{lem:ac0_primes_infinity_mutually_prime_subset}.

\begin{lemma}
	\label{lem:ac0_primes_infinity_mutually_prime_subset_equiv_to_P_property}
	Пусть множество $A$ обладает $P$-свойством.
	Тогда существует бесконечное подмножество $A'\subset A$ попарно взаимно простых чисел.
\end{lemma}


\begin{lemma}
	\label{lem:ac0_primes_q_psi_A_0_causes_P}
	Пусть для множества $A\subset\mathbb{N}\setminus\{1\}$ выполнено $q(\psi(A))=0$.
	Тогда $A$ обладает $P$-свойством.
\end{lemma}


Таким образом,
из теоремы~\ref{lem:ac0_primes_infinity_mutually_prime_subset}
и лемм~\ref{lem:ac0_primes_infinity_mutually_prime_subset_equiv_to_P_property},~\ref{lem:ac0_primes_q_psi_A_0_causes_P}
незамедлительно следует
\begin{theorem}
	Пусть $A\subset \mathbb{N}\setminus\{1\}$.
	Тогда следующие условия эквивалентны:
	\begin{enumerate}[label=(\roman*)]
		\item
			$A$ обладает $P$-свойством
		\item
			В $A$ существует бесконечное подмножество попарно взаимно простых чисел
		\item
			$q(\psi(A))=0$.
	\end{enumerate}
\end{theorem}

%\section{Исключение множителей и верхний функционал Сачестона}
\paragraph{\S 4.}


Перейдём теперь к изучению верхнего функционала Сачестона $p(\psi(A))$.

\begin{theorem}
	\label{thm:ac0_primes_p_psi_A_prod}
	Пусть $A = \{a_1, a_2, ..., a_n,...\}$ "--- бесконечное множество попарно взаимно простых чисел.
	Тогда
	\begin{equation}
		p(\psi(A)) = \prod_{j=1}^\infty \left(1-\frac{1}{a_j}\right)
		.
	\end{equation}
\end{theorem}


\begin{corollary}
	В случае, когда $\mathbb{N}\setminus A$ конечно, из следствия~\ref{cor:ac0_powers_finite_set_of_numbers} непосредственно вытекает, что $\psi(A)\in ac_0$
	и, соответственно, $p(\psi(A))=0$.
\end{corollary}



Классическая теорема Эйлера~\cite{euler1737variae} говорит о том, что
ряд обратных простых чисел
\begin{equation}
	\frac{1}{2} + \frac{1}{3} + \frac{1}{5} + \frac{1}{7} + ...
	=
	\sum_j \frac{1}{j},
\end{equation}
где $j$ пробегает все простые числа, расходится.

Из любого расходящегося ряда, члены которого положительны и стремятся к нулю,
можно выделить ряд, сходящийся к любому наперёд заданному положительному числу.
Аналогично, из любого расходящегося произведения положительных чисел, стремящихся к единице, можно выделить
подпроизведение, сходящееся к наперёд заданному числу из интервала $(0;1)$.
С учётом этого факта
из теоремы~\ref{thm:ac0_primes_p_psi_A_prod} вытекает
\begin{corollary}
	Пусть $\varepsilon \in {[} 0; 1)$.
	Существует бесконечное множество попарно непересекающихся подмножеств простых чисел
	$A_i$ такое, что $p(\psi(A_i))=\varepsilon$ для любого $i\in\mathbb{N}$.
\end{corollary}

\begin{theorem}
	Пусть $A=\{a_1, a_2, ..., a_n, ...\}\subset\mathbb{N}$ есть бесконечное множество попарно взаимно простых чисел.
	Тогда следующие условия эквивалентны:
	\begin{enumerate}[label=(\roman*)]
		\item
			$\psi(A)\in ac$.
		\item
			$\psi(A)\in ac_0$.
		\item
			$\prod_{j=1}^\infty \left(1-\frac{1}{a_j}\right) = 0$.
	\end{enumerate}
\end{theorem}

\subsection*{Список литературы}

\begin{thebibliography}{99}
%
\bibitem{Mazur}
\emph{Mazur} \emph{S.}. O metodach sumomalności // Ann. Soc. Polon. Math. (Supplement). — 1929. —
с. 102—107.
%
\bibitem{banach2001theory_rus}
\emph{Банах} \emph{С.}. Теория линейных операций. — Регуляр. и хаот. динамика М, 2001.
%
\bibitem{lorentz1948contribution}
\emph{Lorentz} \emph{G. G.}. A contribution to the theory of divergent sequences // Acta Mathematica. —
1948. — т. 80, No 1. — с. 167—190. — ISSN 0001-5962.%
\bibitem{sucheston1967banach}
\emph{Sucheston} \emph{L.}. Banach limits // Amer. Math. Monthly. — 1967. — т. 74. — с. 308—311. —
ISSN 0002-9890.
%
\bibitem{semenov2020geomBL}
\emph{Семёнов} \emph{Е. М.}, \emph{Сукочев} \emph{Ф. А.}, \emph{Усачев} \emph{А. С.}.
Геометрия банаховых пределов и их приложения // Успехи математических наук. — 2020. — т. 75, 4
(454. — с. 153—194.
%
\bibitem{euler1737variae}
\emph{Euler} \emph{L.}. Variae observationes circa series infinitas // Commentarii academiae scientiarum
imperialis Petropolitanae. — 1737. — т. 9, No 1737. — с. 160—188.
\end{thebibliography}

\end{document}

