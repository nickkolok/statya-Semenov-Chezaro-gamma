\documentclass[a4paper,openbib]{article}
\usepackage{amsmath}
\usepackage[utf8]{inputenc}
\usepackage[english,russian]{babel}
\usepackage{amsfonts,amssymb}
\usepackage{enumerate}
\usepackage{geometry}
\usepackage{wrapfig}
\usepackage[colorlinks=true,allcolors=black]{hyperref}
\usepackage{enumitem}
\usepackage{mathrsfs}


\righthyphenmin=2

\usepackage[14pt]{extsizes}

\geometry{left=3cm}% левое поле
\geometry{right=1cm}% правое поле
\geometry{top=2cm}% верхнее поле
\geometry{bottom=2cm}% нижнее поле

\renewcommand{\baselinestretch}{1.3}

\renewcommand{\leq}{\leqslant}
\renewcommand{\geq}{\geqslant} % И делись оно всё нулём!

\DeclareMathOperator{\ext}{ext}
\DeclareMathOperator{\mes}{mes}
\DeclareMathOperator{\supp}{supp}

\newcommand{\longcomment}[1]{}

\usepackage[backend=biber,style=gost-numeric,sorting=none]{biblatex}
\addbibresource{../bib/general_monographies.bib}
\addbibresource{../bib/ext.bib}
\addbibresource{../bib/my.bib}
\addbibresource{../bib/Semenov.bib}
\addbibresource{../bib/Bibliography_from_Usachev.bib}
\addbibresource{../bib/classic.bib}

\input{../bib/ext.hyphens.bib}

\usepackage{amsthm}
\theoremstyle{definition}
\newtheorem{lemma}{Лемма}[section]
\newtheorem{theorem}[lemma]{Теорема}
\newtheorem{example}[lemma]{Пример}
\newtheorem{property}[lemma]{Свойство}
\newtheorem{remark}[lemma]{Замечание}
\newtheorem{definition}[lemma]{Определение}
\newtheorem{designation}[lemma]{Обозначение}
\newtheorem{corollary}[lemma]{Следствие}

%Only referenced equations are numbered
\usepackage{mathtools}
\mathtoolsset{showonlyrefs}

%\mathtoolsset{showonlyrefs=false}
% (an equation/multline to be force-numbered)
%\mathtoolsset{showonlyrefs=true}


\begin{document}

УДК  { 517.982.276 % Пространства последовательностей и матриц
     + 511.172    }% Мультипликативная структура кольца целых чисел (включая делимость, наименьшее общее кратное, наибольший общий делитель, сравнения, степенные вычеты, квадратичные вычеты)

MSC~
	46B45,
	11A51,
	11B57



\begin{center}
	Почти сходящиеся последовательности из 0 и 1 и простые числа
	\footnote{
		Работа выполнена в Воронежском госуниверситете при поддержке РНФ, грант 19-11-00197.
	}
\end{center}

\begin{center}
	{Н.~Н.~Авдеев$^{2}$\\[4pt]}
	{\rm\footnotesize{$^2$\,
	Воронежский государственный университет\\
	Россия, 394006, Университетская пл., 1\\
	E-mail: nickkolok@mail.ru, avdeev@math.vsu.ru}}
\end{center}



\paragraph{Аннотация}
В статье изучаются последовательности из нулей и единиц.
Устанавливается связь между значениями верхнего и нижнего функционалов Сачестона
на такой последовательности и множеством всевозможных делителей
элементов, входящих в носитель такой последовательности.
Если объединение множеств всех простых делителей чисел из носителя некоторой последовательности
из нулей и единиц конечно, то такая последовательность почти сходится к нулю.
Изучаются такие последовательности из нулей и единиц,
носитель которых в точности состоит из чисел,
кратных элементам некоторого заданного множества,
и устанавливаются необходимые и достаточные условия для обращения в единицу верхнего функционала Сачестона
на такой последовательности.
Доказывается, что нижний функционал Сачестона на такой последовательности
может принимать любое значение от нуля до единицы (включительно),
при этом в нуль никогда не обращается.



Ключевые слова:
	пространство ограниченных последовательностей,
	банахов предел,
	функционал Сачестона,
	почти сходящаяся последовательность,
	последовательности из нулей и единиц,
	разложение на множители,
	подмножеcтва натуральных чисел

\newpage

\begin{flushright}
	\textit{
		Светлой памяти
		декана математического факультета ВГУ
		\\
		профессора
		Александра Дмитриевича Баева
	}
\end{flushright}

\section{Введение}

Рассмотрим пространство ограниченных последовательностей $\ell_\infty$ с обычной нормой
\begin{equation*}
	\|x\| = \sup_{k\in\mathbb{N}} |x_k|
	,
\end{equation*}
где $\mathbb{N}$ "--- множество натуральных чисел,
и обычной полуупорядоченностью.


\begin{definition}
	Линейный функционал $B\in \ell_\infty^*$ называется банаховым пределом,
	если
	\begin{enumerate}[label=(\roman*)]
		\item
			$B\geq0$, т.~е. $Bx \geq 0$ для $x \geq 0$,
		\item
			$B\mathbb{I}=1$, где $\mathbb{I} =(1,1,\ldots)$,
		\item
			$B(Tx)=B(x)$ для всех $x\in \ell_\infty$, где $T$~---
		оператор сдвига, т.~е. $T(x_1,x_2,\ldots)=(x_2,x_3,\ldots)$.
	\end{enumerate}
\end{definition}
Множество всех банаховых пределов обозначим через $\mathfrak{B}$.
Существование банаховых пределов было анонсировано С. Мазуром \cite{Mazur} и позднее доказано в книге С.~Банаха~\cite{banach2001theory_rus}.
%
Cуществуют почти сходящиеся последовательности ---
такие последовательности из $\ell_\infty$,
на которых все банаховы пределы принимают одинаковые значения.
\begin{theorem}[критерий Лоренца~\cite{lorentz1948contribution}]
	Для заданного $t\in\mathbb{R}$ равенство $Bx=t$ выполнено для всех $B\in\mathfrak{B}$
	тогда и только тогда, когда
	\begin{equation}
		\label{eq:crit_Lorentz}
		\lim_{n\to\infty} \frac{1}{n} \sum_{k=m+1}^{m+n} x_k = t
	\end{equation}
	равномерно по $m\in\mathbb{N}$.
\end{theorem}


Пространство почти сходящихся последовательностей будем обозначать через $ac$.

Уточняя результат Лоренца, Сачестон установил~\cite{sucheston1967banach}, что
для любых $x\in \ell_\infty$ и $B\in\mathfrak{B}$
\begin{equation}\label{Sucheston}
	q(x) \leqslant Bx \leqslant p(x)
	,
\end{equation}
где
\begin{equation*}
	q(x) = \lim_{n\to\infty} \inf_{m\in\mathbb{N}}  \frac{1}{n} \sum_{k=m+1}^{m+n} x_k
	\mbox{~~~~и~~~~}
	p(x) = \lim_{n\to\infty} \sup_{m\in\mathbb{N}}  \frac{1}{n} \sum_{k=m+1}^{m+n} x_k
\end{equation*}
называют нижним и верхним функционалом Сачестона соответственно.
Заметим, что $p(x) = -q(-x)$.
Неравенства \eqref{Sucheston} точны:
для данного $x$ для любого $t\in[q(x); p(x)]$ найдётся банахов предел
$B\in\mathfrak{B}$ такой, что $Bx = t$.

Легко видеть, что множество таких $x\in\ell_\infty$, что $p(x)=q(x)$,
в точности является пространством почти сходящихся последовательностей $ac$.
Будем говорить, что $x\in ac_t$, если $x\in ac$ и $p(x)=q(x)=t$.
Таким образом, функционалы Сачестона для почти сходимости играют такую же роль,
как верхний и нижний пределы для <<обычной>> сходимости.
Более подробно о свойствах почти сходящихся последовательностей и банаховых пределах см.~\cite{semenov2020geomBL,our-mz2019ac0}.

Дальнейшим ослаблением понятия сходимости является сходимость по Чезаро (сходимость в среднем).
Говорят, что последовательность $\{x_n\}\in\ell_\infty$ сходится по Чезаро к $t$, если
\begin{equation}
	\lim_{n\to\infty}\frac1{n}\sum_{i=1}^n x_i = t
	.
\end{equation}
Легко заметить, что обсуждаемые обобщения верхнего и нижнего пределов удовлетворяют соотношению
\begin{multline}
	\label{eq:generalization_of_limits}
	\liminf_{n\to\infty} x_n \leq q(x) \leq \liminf_{n\to\infty}\frac1{n}\sum_{i=1}^n x_i
	\leq
	\\ \leq
	\limsup_{n\to\infty}\frac1{n}\sum_{i=1}^n x_i
	\leq p(x)
	\leq \limsup_{n\to\infty} x_n
	.
\end{multline}

Отдельный интерес представляет множество всех последовательностей из 0 и 1,
которое в дальнейшем мы будем обозначать через $\Omega$.
%(иногда в литературе~\cite{semenov2020geomBL,Semenov2014geomprops} встречается также обозначение $\{0;1\}^\mathbb{N}$).
Понятно, что каждый $x\in \Omega$ можно отождествить с подмножеством множества натуральных чисел
$\supp x \subset \mathbb{N}$.

Вслед за~\cite{hall1992behrend} будем обозначать через $\mathscr{M}A$ множество всех чисел,
кратных элементам множества $A\subset\mathbb{N}$, т.е.
\begin{equation}
	\mathscr{M}A = \{ka: k\in\mathbb{N}, a\in A\}
	,
\end{equation}
через $\chi A$ "--- характеристическую функцию множества $A$.

Так, например,
\begin{gather}
	\chi \mathscr{M}\!A(\{2\}) = \chi \mathscr{M}\!A(\{2, 4\}) = \chi \mathscr{M}\!A(\{2,4,8,16,...\})
	= (0,1,0,1,0,1,0,1,0,...),
\\
	\chi \mathscr{M}\!A(\{3\}) = \chi \mathscr{M}\!A(\{3,9,27,...\}) = (0,0,1,\;0,0,1,\;0,0,1,\;0,0,1,\;0,0,1,\;...),
\\
	\chi \mathscr{M}\!A(\{2,3\}) = \chi \mathscr{M}\!A(\{2,3,6\}) = (0,1,1,1,0,1,\;0,1,1,1,0,1,\;0,1,1,1,0,1,...).
\end{gather}

Возникает закономерный вопрос о взаимосвязи структуры множества $A$
и значений, которые принимают обобщения верхнего и нижнего пределов~\eqref{eq:generalization_of_limits}
на последовательности $\chi \mathscr{M}\!A$.
Так, в работах~\cite{davenport1936sequences,davenport1951sequences} доказано, что для любого
$A=\{a_1,a_2,...\}\subset\mathbb{N}$
выполнено
\begin{equation}
	\liminf_{n\to\infty}\frac1{n}\sum_{i=1}^n (\chi\mathscr{M}A)_i =
	\lim_{j\to\infty}\lim_{n\to\infty}\frac1{n}\sum_{i=1}^n (\chi\mathscr{M}\{a_1,a_2,...,a_j\})_i
	.
\end{equation}
В работе~\cite[\S 7]{besicovitch1935density} построено такое множество $A\subset\mathbb{N}$, что
\begin{equation}
	\liminf_{n\to\infty}\frac1{n}\sum_{i=1}^n (\chi\mathscr{M}A)_i \neq
	\limsup_{n\to\infty}\frac1{n}\sum_{i=1}^n (\chi\mathscr{M}A)_i
	.
\end{equation}
За более подробной информацией о множествах типа $\mathscr{M}A$ отсылаем читателя к монографии~\cite{hall1996multiples}.

В настоящей статье изучается зависимость значений, которые могут принимать функционалы Сачестона
на последовательностях $\chi\mathscr{M}A$, от свойств множества $A$.





\section{Конечное число множителей}

Пользуясь критерием Лоренца~\eqref{eq:crit_Lorentz},
нетрудно доказать для $x_n = m^n$, $m\in\mathbb{N}$, $m\geq 2$ выполнено $\chi x \in ac_0$.

\begin{lemma}
	Пусть $y = \{y_n\}$ --- строго возрастающая последовательность,
	$\chi_y\in\Omega \cap ac_0$.
	Пусть $m \in \mathbb{N}$
	и последовательность $x=\{x_k\}$ определена соотношением
	\begin{equation}
		x_k = \begin{cases}
			1, &\mbox{~если~} k = y_i \cdot m^j \mbox{~для некоторых~} i,j\in\mathbb{N},
			\\
			0  &\mbox{~иначе}
			.
		\end{cases}
	\end{equation}
	Тогда $x\in ac_0$.
\end{lemma}

\begin{proof}
	Зафиксируем некоторое $K \in \mathbb{N}$ и покажем, что $p(x) < m^{-K}$.
	Действительно, представим $x$ в виде суммы
	\begin{equation}
		\label{eq:ac0_powers_x_as_sum}
		x \leq z_1 + z_2 + \dots + z_K + z'_{K+1}
		,
	\end{equation}
	где каждое слагаемое $z_j$ соответствует умножению индексов на $m^j$:
	\begin{equation}
		(z_j)_k = \begin{cases}
			1, &\mbox{~если~} k = y_i \cdot m^j \mbox{~для некоторого~} i\in\mathbb{N},
			\\
			0  &\mbox{~иначе}
			,
		\end{cases}
	\end{equation}
	а слагаемое $z'_j$ соответствует умножению индексов на $m^{j+1}$, $m^{j+2},...$:
	\begin{equation}
		\label{eq:ac0_powers_residue}
		(z'_j)_k = \begin{cases}
			1, &\mbox{~если~} k = y_i \cdot m^{j'} \mbox{~для некоторых~} i,j'\in\mathbb{N},~~ j' > j
			,%\\ &\mbox{~и ни для какого~} j' \leq j \mbox{~не выполнено~} k = y_i \cdot m^{j'},
			\\
			0  &\mbox{~иначе}
			.
		\end{cases}
	\end{equation}
	(Знак неравенства в~\eqref{eq:ac0_powers_x_as_sum} возникает ввиду того, что возможен случай
	$y_i \cdot m^j = y_{i'} \cdot m^{j'}$ для $j\neq j'$.
	Например, в случае $y_1 = 3$, $y_2 = 6$ и $m=2$ имеем $y_1 \cdot m^2 = y_2 \cdot m^1$.)
	Понятно, что $p(z_j)=0$.
	Таким образом,
	\begin{equation}
		p(x) \leq p(z_1) + p(z_2) + \dots + p(z_K) + p(z'_{K+1}) = p(z'_{K+1})
		.
	\end{equation}
	Заметим, что в силу определения~\eqref{eq:ac0_powers_residue} $z'_j$ каждый отрезок из $m^{j+1}$ элементов
	содержит не более одной единицы, и потому $p(z'_j) \leq m^{-j-1} < m^{-j}$.
	Таким образом, для любого $K\in\mathbb{N}$ выполнена оценка $p(x) < m^{-K}$,
	откуда $p(x) = 0$.
\end{proof}

\begin{corollary}
	\label{cor:ac0_powers_finite_set_of_numbers}
	Пусть $\{p_1, ..., p_k\} \subset \mathbb{N}$,
	\begin{equation}
		x_k = \begin{cases}
			1, &\mbox{~если~} k = p_1^{j_1}\cdot p_2^{j_2}\cdot ... \cdot p_k^{j_k} \mbox{~для некоторых~} j_1,...,j_k\in\mathbb{N},
			\\
			0  &\mbox{~иначе}.
		\end{cases}
	\end{equation}
	Тогда $x\in ac_0$.
\end{corollary}


%\begin{hypothesis}
%	Пусть $y=\{y_n\}$ и $z=\{z_n\}$ --- строго возрастающие последовательности,
%	$\chi_y,\chi_z\in\{0;1\}^\mathbb{N} \cap ac_0$.
%	Тогда почти сходится к нулю последовательность $x=\{x_k\}$, определённая соотношением
%	\begin{equation}
%		x_k = \begin{cases}
%			1, &\mbox{~если~} k = y_i \cdot z_j \mbox{~для некоторых~} i,j\in\mathbb{N},
%			\\
%			0  &\mbox{~иначе}
%			.
%		\end{cases}
%	\end{equation}
%\end{hypothesis}


\section{Бесконечное число множителей и \\  верхний функционал Сачестона}




\begin{lemma}
	Для любого непустого $A\subset \mathbb{N} $ выполнено $\chi\mathscr{M}A \notin ac_0$.
\end{lemma}
\begin{proof}
	Пусть $a_1\in A$.
	Тогда из каждых идущих подряд $a_1$ элементов последовательности $\chi\mathscr{M}A$
	хотя бы один равен единице,
	следовательно,
	\begin{equation}
		q(\chi\mathscr{M}A) \geq \frac{1}{a_1} > 0
		.
	\end{equation}
\end{proof}

При дополнительных ограничениях верно и более сильное утверждение.

\begin{theorem}
	\label{lem:ac0_primes_infinity_mutually_prime_subset}
	Пусть $A'$ "--- бесконечное подмножество попарно взаимно простых чисел
	(т.е. для любых двух чисел $a_1, a_2 \in A'$ их наибольший общий делитель равен единице).
	Тогда для любого $A \supset A' $ выполнено $p(\chi\mathscr{M}A)=1$.
\end{theorem}
\begin{proof}
	Пусть $A' = \{ a_1, a_2, ..., a_j, ... \}$ и
	\begin{equation}
		\label{eq:ac0_primes_A_j_prod_des}
		A_j = \prod_{i=1}^j a_i
		.
	\end{equation}

	Для каждого $k$ найдём такие номера $n_k$, что
	\begin{equation}
		(\chi\mathscr{M}A)_{n_k+1} = (\chi\mathscr{M}A)_{n_k+2} = \dots = (\chi\mathscr{M}A)_{n_k+k} = 1
		.
	\end{equation}
	Тем самым мы докажем, что отрезок из любого наперёд заданного количества единиц подряд
	встречается в последовательности $\chi\mathscr{M}A$ и, следовательно, $p(\chi\mathscr{M}A) = 1$.

	Действительно,
	пусть $n_1 = a_1 - 1$.
	Рассмотрим множество  $F_1 = \{ n_1 + A_1, n_1 + 2A_1, n_1 + 3A_1, \dots, n_1 + a_2A_1 \}$
	и отметим два следующих факта.

	Во-первых, пусть $f \in F_1$,
	тогда
	\begin{equation}
		f \equiv n_1 \mod a_1
		.
	\end{equation}
	Во-вторых, числа $a_2$ и $A_1$ взаимно просты.
	Следовательно, все $a_2$ чисел из множества $F_1$ дают разные остатки при делении на $a_2$.

	В качестве $n_2$ возьмём такое $f\in F_1$, что
	\begin{equation}
		f \equiv a_2 - 2 \mod a_2
		.
	\end{equation}
	Заметим, что тогда
	\begin{equation}
		n_2 + 1 \equiv n_1 + 1 \equiv 0 \mod a_1
	\end{equation}
	и
	\begin{equation}
		n_2 + 2 \equiv 0 \mod a_2
		,
	\end{equation}
	следовательно,
	$(\chi\mathscr{M}A)_{n_2 + 1} = (\chi\mathscr{M}A)_{n_2 + 2} = 1$.

	Полученные рассуждения несложно продолжить по индукции.

	Рассмотрим множество  $F_{j+1} = \{ n_j + A_j, n_j + 2A_j, n_j + 3A_j, \dots, n_j + a_{j+1}A_j \}$
	и отметим два следующих факта.

	Во-первых, пусть $f \in F_{j+1}$,
	тогда
	\begin{equation}
		f \equiv n_j \mod A_j
	\end{equation}
	и, в силу~\eqref{eq:ac0_primes_A_j_prod_des},
	\begin{equation}
		\begin{array}{rl}
		f &\equiv n_j \mod a_1
		\\
		f &\equiv n_j \mod a_2
		\\
		&\dots
		\\
		f &\equiv n_j \mod a_j
		.
		\end{array}
	\end{equation}
	Во-вторых, числа $a_{j+1}$ и $A_j$ взаимно просты, поскольку $a_{j+1}$ взаимно просто с каждым из чисел $a_1,...,a_j$.
	Следовательно, все $a_{j+1}$ чисел из множества $F_j$ дают разные остатки при делении на $a_{j+1}$.

	В качестве $n_{j+1}$ возьмём такое $f\in F_j$, что
	\begin{equation}
		f \equiv a_{j+1} - (j+1) \mod a_{j+1}
		.
	\end{equation}
	Заметим, что тогда
	\begin{equation}
		\begin{array}{l}
			n_{j+1} + 1 \equiv n_j + 1 \equiv n_{j-1} + 1 \equiv \dots \equiv n_2 + 1 \equiv n_1 + 1 \equiv 0 \mod a_1
			\\
			n_{j+1} + 2 \equiv n_j + 2 \equiv n_{j-1} + 2 \equiv \dots \equiv n_2 + 2 \equiv 0 \mod a_2
			\\
			\dots
			\\
			n_{j+1} + j \equiv n_j + j  \equiv 0 \mod a_j
		\end{array}
	\end{equation}
	и
	\begin{equation}
		n_{j+1} + (j+1) \equiv 0 \mod a_{j+1}
		,
	\end{equation}
	следовательно,
	\begin{equation}
		(\chi\mathscr{M}A)_{n_{j+1} + j+1} = (\chi\mathscr{M}A)_{n_{j} + j} =
		\dots = (\chi\mathscr{M}A)_{n_{j} + 2} = (\chi\mathscr{M}A)_{n_{j} + 1} = 1
		.
	\end{equation}


\end{proof}

\begin{remark}
	Понятно, что в качестве примера бесконечного множества
	попарно взаимно простых чисел можно взять любое бесконечное множество простых чисел.
	Однако бывают бесконечные множества попарно взаимно простых чисел,
	не содержащие простых чисел вовсе, например множество
	\begin{equation}
		A = \{ 2\cdot 3,~5 \cdot 7,~11 \cdot 13,~17\cdot 19,~23\cdot29,~31\cdot 37,~...\},
	\end{equation}
	элементами которого являются произведения пар соседних простых чисел.
\end{remark}

\begin{definition}
	Будем говорить, что множество $A\subset\mathbb{N}$ обладает $P$-свойством,
	если для любого $n\in\mathbb{N}$ найдётся набор попарно взаимно простых чисел
	\begin{equation}
		\{a_{n,1}, a_{n,2}, ..., a_{n,n}  \} \subset A
		.
	\end{equation}
\end{definition}

Из доказательства теоремы~\ref{lem:ac0_primes_infinity_mutually_prime_subset} понятно,
что для множества $A$ в условии теоремы достаточно потребовать $P$-свойства.
Интересно, что на самом деле $P$-свойство эквивалентно условиям,
наложенным на множество $A$ в теореме~\ref{lem:ac0_primes_infinity_mutually_prime_subset}.

\begin{lemma}
	\label{lem:ac0_primes_infinity_mutually_prime_subset_equiv_to_P_property}
	Пусть множество $A$ обладает $P$-свойством.
	Тогда существует бесконечное подмножество $A'\subset A$ попарно взаимно простых чисел.
\end{lemma}
\begin{proof}
	Зафиксируем $f_0\in A$, $f_0 \neq 1$ и представим $A$ в виде объединения трёх попарно непересекающихся множеств:
	\begin{equation}
		A = \{f_0\} \cup E \cup F
		,
	\end{equation}
	где
	\begin{alignat*}{4}
		& E &&= \{ a \in A \mid a \mbox{~не~}&&\mbox{взаимно просто с~} f_0 \mbox{~и~} a\neq f_0\}
		,
		\\
		& F &&= \{ a \in A \mid a            &&\mbox{взаимно просто с~} f_0 \}
		.
	\end{alignat*}

	Пусть разложение $f_0$ на простые множители имеет вид
	\begin{equation}
		f_0 = p_1^{j_1} \cdot p_2^{j_2} \cdot ... \cdot p_k^{j_k}
		.
	\end{equation}

	Тогда множество $E$ можно представить в виде объединения (возможно пересекающихся) множеств:
	\begin{equation}
		E = \bigcup_{i=1}^{k} E_i,\quad E_i = \{a\in E \mid a \mbox{~кратно~} p_i\}
		.
	\end{equation}

	Покажем, что множество $F$ обладает $P$-свойством.
	Действительно, зафиксируем $n\in\mathbb{N}$.
	Так как множество $A$ обладает $P$-свойством,
	то в нём найдётся подмножество попарно взаимно простых чисел
	$$G=\{a_1, a_2, ..., a_{n+k-1}, a_{n+k}\}\subset A.$$

	Если $f_0\in G$, то $G\setminus f_0 \subset F$ в силу построения множеств $G$ и $F$, и требуемый набор попарно взаимно простых чисел предъявлен.

	Пусть теперь $f_0\notin G$.
	Заметим, что в каждое из множеств $E_i$ может входит не более одного элемента множества $G$
	в силу того, что при фиксированном $i$ все элементы множества $E_i$ имеют нетривиальный общий делитель.
	Следовательно, как минимум $n$ элементов из $G$ принадлежат множеству $F$,
	и требуемый набор попарно взаимно простых чисел снова предъявлен.

	%Поскольку множество $F$ обладает $P$-свойством, то оно, очевидно, счётно.

	Итак, нам удалось получить число $f_0\in A$ и бесконечное множество $F$, обладающее $P$-свойством
	и состоящее из чисел, взаимно простых с $f_0$.
	Продолжая по индукции, получим требуемое бесконечное множество попарно взаимно простых чисел.
\end{proof}

%Условие теоремы~\ref{lem:ac0_primes_infinity_mutually_prime_subset}
%является не только достаточным, но и необходимым.

\begin{lemma}
	\label{lem:ac0_primes_q_psi_A_0_causes_P}
	Пусть для множества $A\subset\mathbb{N}\setminus\{1\}$ выполнено $p(\chi\mathscr{M}A)=1$.
	Тогда $A$ обладает $P$-свойством.
\end{lemma}

\begin{proof}
	Предположим противное: пусть $A$ не обладает $P$-свойством.
	Тогда существует $n\in\mathbb{N}$ такое, что из множества $A$ можно выбрать $n$
	попарно взаимно простых чисел, но нельзя выбрать $n+1$.

	Пусть $\{a_1, a_2, ..., a_n\}\subset A$ "--- набор попарно взаимно простых чисел.
	Так как $p(\chi\mathscr{M}A)=1$, то в последовательности $\chi\mathscr{M}A$ найдётся отрезок, состоящий сплошь из единиц,
	любой наперёд заданной длины.
	Выберем $k$ таким, что
	\begin{equation}
		(\chi\mathscr{M}A)_{k+1} = (\chi\mathscr{M}A)_{k+2} = ... = (\chi\mathscr{M}A)_{k+a_1a_2\cdots a_n} = 1
		.
	\end{equation}
	Тогда существует такое число $k_0$, $k+1 \leq k_0 \leq k+a_1a_2\cdots a_n$,
	что $k_0$ даёт в остатке $1$ при делении на $a_1a_2\cdots a_n$.
	Так как $(\chi\mathscr{M}A)_{k_0} = 1$, то $k_0 = m\cdot a_0$ для некоторых $m\in\mathbb{N}$ и $a_0\in A$.
	С другой стороны, $k_0$ взаимно просто с каждым из чисел $a_1, a_2, ..., a_n$.
	Следовательно, $a_0$ также взаимно просто с каждым из чисел $a_1, a_2, ..., a_n$,
	и $\{a_0, a_1, a_2, ..., a_n\}\subset A$ "--- набор из $n+1$ попарно взаимно простых чисел.
	Полученное противоречие завершает доказательство.
\end{proof}

Таким образом,
из теоремы~\ref{lem:ac0_primes_infinity_mutually_prime_subset}
и лемм~\ref{lem:ac0_primes_infinity_mutually_prime_subset_equiv_to_P_property},~\ref{lem:ac0_primes_q_psi_A_0_causes_P}
незамедлительно следует
\begin{theorem}
	Пусть $A\subset \mathbb{N}\setminus\{1\}$.
	Тогда следующие условия эквивалентны:
	\begin{enumerate}[label=(\roman*)]
		\item
			$A$ обладает $P$-свойством
		\item
			В $A$ существует бесконечное подмножество попарно взаимно простых чисел
		\item
			$p(\chi\mathscr{M}A)=1$.
	\end{enumerate}
\end{theorem}

\section{Бесконечное количество множителей и \\ нижний функционал Сачестона}


Перейдём теперь к изучению нижнего функционала Сачестона $q(\chi\mathscr{M}A)$.

\begin{theorem}
	\label{thm:ac0_primes_p_psi_A_prod}
	Пусть $A = \{a_1, a_2, ..., a_n,...\}$ "--- бесконечное множество попарно взаимно простых чисел.
	Тогда
	\begin{equation}
		q(\chi\mathscr{M}A) = 1-\prod_{j=1}^\infty \left(1-\frac{1}{a_j}\right)
		.
	\end{equation}
\end{theorem}

\begin{proof}
	Зафиксируем $k\in\mathbb{N}$.

	Заметим, что нижний функционал Сачестона можно представить в виде
	\begin{equation}
		\label{eq:ac0_primes_px_lim_pnx}
		q(x) = \lim_{n\to\infty} q_n(x)
		,
	\end{equation}
	где
	\begin{equation}
		q_n(x) = \inf_{m\in\mathbb{N}}  \frac{1}{n} \sum_{j=m}^{m+n-1} x_j
		.
	\end{equation}

	Поскольку предел~\eqref{eq:ac0_primes_px_lim_pnx} существует, то для его оценки можно использовать предел подпоследовательности
	$q_{n_k}(x)$, где $n_k = a_1\cdot a_2 \cdot ... \cdot a_k$.


	Cреди первых $n_k$ элементов последовательности $\chi\mathscr{M}A$
	ровно $\prod_{j=1}^k (a_j-1)$ нулей, поскольку комбинации остатков от деления
	на взаимно простые числа $a_1, a_2, ..., a_k$ <<не успевают>> повторяться.
	Следовательно,
	\begin{equation}
		q_{n_k}(\chi\mathscr{M}A) =
		\inf_{m\in\mathbb{N}}  \frac{1}{n_k} \sum_{j=m}^{m+n_k-1} (\chi\mathscr{M}A)_j
		\leq
		\frac{1}{n_k} \sum_{j=1}^{n_k} (\chi\mathscr{M}A)_j
		=
		1-\prod_{i=1}^k \frac{a_i-1}{a_i}
		.
	\end{equation}
	Переходя к пределу по $k$, имеем
	\begin{equation}
		\label{eq:ac0_primes_p_psi_A_lower_bound}
		q(\chi\mathscr{M}A) \leq 1-\lim_{k\to \infty} \prod_{i=1}^k \frac{a_i-1}{a_i}
		=
		1-\prod_{i=1}^\infty \frac{a_i-1}{a_i}
		.
	\end{equation}

	Заметим теперь, что в любом отрезке последовательности $\chi\mathscr{M}A$ длины $n_k$
	содержится не более $\prod_{j=1}^k (a_j-1)$ нулей
	(могут попадаться <<дополнительные>> единицы "--- элементы с индексами, кратными $a_j$ для $j>k$).
	Значит,
	\begin{equation}
		q_{n_k}(\chi\mathscr{M}A) =
		\sup_{m\in\mathbb{N}}  \frac{1}{n_k} \sum_{j=m}^{m+n_k-1} (\chi\mathscr{M}A)_j
		\geq
		1-\prod_{i=1}^k \frac{a_i-1}{a_i}
		.
	\end{equation}
	Снова перейдя к пределу по $k$, получим
	\begin{equation}
		\label{eq:ac0_primes_p_psi_A_upper_bound}
		q(\chi\mathscr{M}A) \geq \lim_{k\to \infty} \prod_{i=1}^k \frac{a_i-1}{a_i}
		=
		1-\prod_{i=1}^\infty \frac{a_i-1}{a_i}
		.
	\end{equation}

	Сопоставив~\eqref{eq:ac0_primes_p_psi_A_lower_bound} и~\eqref{eq:ac0_primes_p_psi_A_upper_bound}, получим утверждение теоремы.
\end{proof}

\begin{corollary}
	В случае, когда $\mathbb{N}\setminus A$ конечно,
	из следствия~\ref{cor:ac0_powers_finite_set_of_numbers} непосредственно вытекает, что $\chi\mathscr{M}A\in ac_1$
	и, соответственно, $q(\chi\mathscr{M}A)=1$.
\end{corollary}



Классическая теорема Эйлера~\cite{euler1737variae} говорит о том, что
ряд обратных простых чисел
\begin{equation}
	\frac{1}{2} + \frac{1}{3} + \frac{1}{5} + \frac{1}{7} + ...
	=
	\sum_j \frac{1}{j},
\end{equation}
где $j$ пробегает все простые числа, расходится.

С учётом этого факта
из теоремы~\ref{thm:ac0_primes_p_psi_A_prod} вытекает
\begin{lemma}
	Пусть $\varepsilon \in  (0; 1{]}$.
	Существует бесконечное множество попарно непересекающихся подмножеств простых чисел
	$A_i$ такое, что $q(\chi\mathscr{M}A)=\varepsilon$ для любого $i\in\mathbb{N}$.
\end{lemma}

%\begin{proof}
%	В силу следствия~\ref{cor:ac0_primes_sum_ln_diverges} мы можем выбрать такое множество простых чисел $A$,
%	что
%	\begin{equation}
%		\sum_{j\in A} \ln \frac{j}{j-1} = -\ln \varepsilon > 0
%		.
%	\end{equation}
%	Тогда в силу теоремы~\ref{thm:ac0_primes_p_psi_A_prod}
%	\begin{multline}
%		p(\psi(A)) = \prod_{j\in A} \frac{j - 1}{j}
%		=
%		\exp \sum_{j\in A} \ln \frac{j - 1}{j}
%		=
%		\exp \sum_{j\in A} \left( -\ln \frac{j}{j-1}\right)
%		=
%		\\=
%		\exp \left( -\sum_{j\in A} \ln \frac{j}{j-1}\right)
%		=
%		\exp (\ln \varepsilon)
%		=
%		\varepsilon
%		.
%	\end{multline}
%
%\end{proof}

\begin{theorem}
	Пусть $A=\{a_1, a_2, ..., a_n, ...\}\subset\mathbb{N}$ есть бесконечное множество попарно взаимно простых чисел.
	Тогда следующие условия эквивалентны:
	\begin{enumerate}[label=(\roman*)]
		\item
			$\chi\mathscr{M}A\in ac$.
		\item
			$\chi\mathscr{M}A\in ac_1$.
		\item
			$\prod_{j=1}^\infty \left(1-\frac{1}{a_j}\right) = 0$.
	\end{enumerate}
\end{theorem}

\begin{proof}
	По теореме~\ref{lem:ac0_primes_infinity_mutually_prime_subset} $p(\chi\mathscr{M}A)=1$,
	поэтому условия (i) и (ii) эквивалентны.
	По теореме~\ref{thm:ac0_primes_p_psi_A_prod} условие (iii) эквивалентно тому, что $q(\chi\mathscr{M}A)=1$.
\end{proof}

Авдеев Николай Николаевич

Воронежский государственный университет, кафедра теории функций и геометрии, аспирант

Россия, 394006, г. Воронеж, Университетская пл., 1

E-mail: nickkolok@mail.ru, avdeev@math.vsu.ru


\paragraph{Abstract}

This paper is devoted to 0-1-sequences.
We establish the connection between values that Sucheston functional can take
on 0-1-sequence and multiplicative structure of the support of the sequence.
If the set of all the divisors of support elements is finite,
then the sequence is almost convergent to zero.
Then we consider characteristic sequences of sets of multiples
and establish necessary and sufficient conditions
for the upper Sucheston functional
to be 1 on such sequence.
We prove that the lower Sucheston functional
on a characteristic sequence of set of multiples can take any value from the interval $(0;1]$.

\paragraph{Keywords}
	space of bounded sequences,
	Banach limit,
	Sucheston functional,
	almost convergent sequence,
	0-1-sequence,
	integer factorization,
	sets of multiples


Avdeev Nikolai Nikolaevich

Voronezh State University, Department of Function Theory and Geometry, postgraduate student

Russia, 394006, Voronezh, University sq., 1


\printbibliography{}

\end{document}

