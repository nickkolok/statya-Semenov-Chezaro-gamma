\documentclass[a4paper,openbib]{article}
\usepackage{amsmath}
\usepackage[utf8]{inputenc}
\usepackage[english,russian]{babel}
\usepackage{amsfonts,amssymb}
\usepackage{latexsym}
\usepackage{euscript}
\usepackage{enumerate}
\usepackage{graphics}
\usepackage[dvips]{graphicx}
\usepackage{geometry}
\usepackage{wrapfig}
\usepackage[colorlinks=true,allcolors=black]{hyperref}



\righthyphenmin=2

\usepackage[14pt]{extsizes}

\geometry{left=3cm}% левое поле
\geometry{right=1cm}% правое поле
\geometry{top=2cm}% верхнее поле
\geometry{bottom=2cm}% нижнее поле

\renewcommand{\baselinestretch}{1.3}

\renewcommand{\leq}{\leqslant}
\renewcommand{\geq}{\geqslant} % И делись оно всё нулём!

\DeclareMathOperator{\ext}{ext}
\DeclareMathOperator{\mes}{mes}
\DeclareMathOperator{\supp}{supp}

\newcommand{\longcomment}[1]{}

\usepackage[backend=biber,style=gost-numeric,sorting=none]{biblatex}
\addbibresource{../bib/general_monographies.bib}
\addbibresource{../bib/ext.bib}
\addbibresource{../bib/my.bib}
\addbibresource{../bib/Semenov.bib}
\addbibresource{../bib/Bibliography_from_Usachev.bib}
\addbibresource{../bib/classic.bib}

\input{../bib/ext.hyphens.bib}

\usepackage{amsthm}
\theoremstyle{definition}
\newtheorem{lemma}{Лемма}[section]
\newtheorem{theorem}[lemma]{Теорема}
\newtheorem{example}[lemma]{Пример}
\newtheorem{property}[lemma]{Свойство}
\newtheorem{remark}[lemma]{Замечание}
\newtheorem{definition}[lemma]{Определение}
\newtheorem{designation}[lemma]{Обозначение}
\newtheorem{corollary}{Следствие}[lemma]

\newtheorem{hhypothesis}[lemma]{Гипотеза}


\newcommand\hypotlist{ }
\newcounter{hypcount}

\makeatletter
\usepackage{environ}
\NewEnviron{hypothesis}{%

	\edef\curlabel{hhypothesis\thehypcount}
    \begin{hhypothesis}
		\label{\curlabel}
		\BODY%
    \end{hhypothesis}
	\edef\curref{\noexpand\ref{\curlabel}}

	\expandafter\g@addto@macro\expandafter\hypotlist\expandafter
	{\paragraph{Гипотеза\!\!\!}}


	\expandafter\g@addto@macro\expandafter\hypotlist\expandafter
	{\expandafter\textbf\expandafter{\curref}}

	\expandafter\g@addto@macro\expandafter\hypotlist\expandafter
	{\textbf{.}~~}

	\expandafter\g@addto@macro\expandafter\hypotlist\expandafter
	{\BODY}

	\addtocounter{hypcount}{1}
}
\makeatother

%Only referenced equations are numbered
\usepackage{mathtools}
%\mathtoolsset{showonlyrefs}

%\mathtoolsset{showonlyrefs=false}
% (an equation/multline to be force-numbered)
%\mathtoolsset{showonlyrefs=true}

% https://superuser.com/questions/517025/how-can-i-append-two-pdfs-that-have-links
\usepackage{pdfpages}% http://ctan.org/pkg/pdfpages

%\newcommand{\longcomment}[1]{}


\begin{document}

УДК  { 517.982.276 % Пространства последовательностей и матриц
     + 511.172    }% Мультипликативная структура кольца целых чисел (включая делимость, наименьшее общее кратное, наибольший общий делитель, сравнения, степенные вычеты, квадратичные вычеты)

MSC~
	46B45,
	11A51,
	11B57



\begin{center}
	Почти сходящиеся последовательности из 0 и 1 и простые числа
	\footnote{
		Работа выполнена в Воронежском госуниверситете при поддержке РНФ, грант 19-11-00197.
	}
\end{center}

\begin{center}
	{Н.~Н.~Авдеев$^{2}$\\[4pt]}
	{\rm\footnotesize{$^2$\,
	Воронежский государственный университет\\
	Россия, 394006, Университетская пл., 1\\
	E-mail: nickkolok@mail.ru, avdeev@math.vsu.ru}}
\end{center}


\paragraph{Аннотация} (короткая, для ФАиП)
В статье изучаются последовательности из нулей и единиц.
Устанавливается связь между значениями верхнего и нижнего функционалов Сачестона
на такой последовательности и множеством всевозможных делителей
элементов носителя такой последовательности.


\paragraph{Аннотация} (длинная, для ВМЖ)
В статье изучаются последовательности из нулей и единиц.
Устанавливается связь между значениями верхнего и нижнего функционалов Сачестона
на такой последовательности и множеством всевозможных делителей
элементов носителя такой последовательности.
Доказывается, что если объединение множеств всех простых делителей чисел из носителя некоторой последовательности
из нулей и единиц конечно, то такая последовательность почти сходится к нулю.
Изучаются такие последовательности из нулей и единиц,
что их носитель не содержит чисел,
кратных элементам некоторого заданного множества,
и устанавливаются неоходимые и достаточные условия для обращения в нуль нижнего функционала Сачестона
на такой последовательности.
Доказывается, что верхний функционал Сачестона на такой последовательности
может принимать любое значение от нуля (включительно) до единицы,
при этом в единицу никогда не обращается.



Ключевые слова:
	ограниченная последовательность,
	пространство ограниченных последовательностей,
	банахов предел,
	функционал Сачестона,
	почти сходящаяся последовательность,
	последовательности из нулей и единиц,
	разложение на множители,
	подмножеcтва натуральных чисел


\section{Введение}

Рассмотрим пространство ограниченных последовательностей $\ell_\infty$ с обычной нормой
\begin{equation*}
	\|x\| = \sup_{k\in\mathbb{N}} |x_k|
	.
\end{equation*}
и обычной полуупорядоченностью, где $\mathbb{N}$ "--- множество натуральных чисел.


\begin{definition}
	Линейный функционал $B\in l_\infty^*$ называется банаховым пределом,
	если
	\begin{enumerate}
		\item
			$B\geq0$, т.~е. $Bx \geq 0$ для $x \geq 0$,
		\item
			$B\mathbb{I}=1$, где $\mathbb{I} =(1,1,\ldots)$,
		\item
			$B(Tx)=B(x)$ для всех $x\in l_\infty$, где $T$~---
		оператор сдвига, т.~е. $T(x_1,x_2,\ldots)=(x_2,x_3,\ldots)$.
	\end{enumerate}
\end{definition}
Множество всех банаховых пределов обозначим через $\mathfrak{B}$.
Существование банаховых пределов было анонсировано С. Мазуром \cite{Mazur} и позднее доказано в книге С.~Банаха~\cite{banach1993theorie}.

Лоренц доказал~\cite{lorentz1948contribution}, что существуют почти сходящиеся последовательности ---
такие последовательности из $\ell_\infty$,
на которых все банаховы пределы принимают одинаковые значения.
\begin{theorem}[критерий Лоренца]
	Для заданного $r\in\mathbb{R}$ равенство $Bx=r$ выполнено для всех $B\in\mathfrak{B}$
	тогда и только тогда, когда
	\begin{equation}
		\label{eq:crit_Lorentz}
		\lim_{n\to\infty} \frac{1}{n} \sum_{k=m+1}^{m+n} x_k = r
	\end{equation}
	равномерно по всем $m\in\mathbb{N}$.
\end{theorem}


Пространство почти сходящихся последовательностей будем обозначать через $ac$.

Уточняя результат Лоренца, Сачестон установил~\cite{sucheston1967banach}, что
для любых $x\in l_\infty$ и $B\in\mathfrak{B}$
\begin{equation}\label{Sucheston}
	q(x) \leqslant Bx \leqslant p(x)
	,
\end{equation}
где
\begin{equation*}
	q(x) = \lim_{n\to\infty} \inf_{m\in\mathbb{N}}  \frac{1}{n} \sum_{k=m+1}^{m+n} x_k
	\mbox{~~~~и~~~~}
	p(x) = \lim_{n\to\infty} \sup_{m\in\mathbb{N}}  \frac{1}{n} \sum_{k=m+1}^{m+n} x_k
\end{equation*}
называют нижним и верхним функционалом Сачестона соответственно.
Заметим, что $p(x) = -q(-x)$.
Неравенства \eqref{Sucheston} точны:
для данного $x$ для любого $r\in[q(x); p(x)]$ найдётся банахов предел
$B\in\mathfrak{B}$ такой, что $Bx = r$.

Легко видеть, что множество таких $x\in\ell_\infty$, что $p(x)=q(x)$,
в точности является пространством почти сходящихся последовательностей $ac$.
Будем говорить, что $x\in ac_0$, если $x\in ac$ и $p(x)=q(x)=0$.
Таким образом, функционалы Сачестона для почти сходимости играют такую же роль,
как верхний и нижний пределы для <<обычной>> сходимости.


Отдельный интерес представляет множество всех последовательностей из 0 и 1,
которое в дальнейшем мы будем обозначать через $\Omega$
(иногда в литературе~\cite{Semenov2014geomprops} встречается также обозначение $\{0;1\}^\mathbb{N}$).
Понятно, что каждый $x\in \Omega$ можно отождествить с подмножеством множества натуральных чисел
$\supp x \subset \mathbb{N}$.

В настоящей статье изучается взаимосвязь значений, которые могут принимать функционалы Сачестона
на последовательностях $x\in \Omega$, и мультипликативных свойств множеств $\supp x$.





\section{Добавление множителей}

Пользуясь критерием Лоренца~\eqref{eq:crit_Lorentz},
нетрудно доказать следующую лемму.

\begin{lemma}
	\label{lem:simple_powers_in_ac0}
	Пусть $x_n = m^n$, $m\in\mathbb{N}$, $m\geq 2$.
	Тогда
	\begin{equation}
		\chi_x \in ac_0
		.
	\end{equation}
\end{lemma}

Обобщим теперь утверждение леммы~\ref{lem:simple_powers_in_ac0}.

\begin{lemma}
	Пусть $y = \{y_n\}$ --- строго возрастающая последовательность,
	$\chi_y\in\{0;1\}^\mathbb{N} \cap ac_0$.
	Пусть $m \in \mathbb{N}$.
	Тогда почти сходится к нулю последовательность $x=\{x_k\}$, определённая соотношением
	\begin{equation}
		x_k = \begin{cases}
			1, &\mbox{~если~} k = y_i \cdot m^j \mbox{~для некоторых~} i,j\in\mathbb{N},
			\\
			0  &\mbox{~иначе}
			.
		\end{cases}
	\end{equation}
\end{lemma}

\begin{proof}
	Зафиксируем некоторое $K \in \mathbb{N}$ и покажем, что $p(x) < m^{-K}$.
	Действительно, представим $x$ в виде суммы
	\begin{equation}
		\label{eq:ac0_powers_x_as_sum}
		x \leq z_1 + z_2 + \dots + z_K + z'_{K+1}
		,
	\end{equation}
	где
	\begin{equation}
		(z_j)_k = \begin{cases}
			1, &\mbox{~если~} k = y_i \cdot m^j \mbox{~для некоторого~} i\in\mathbb{N},
			\\
			0  &\mbox{~иначе}
			,
		\end{cases}
	\end{equation}
	\begin{equation}
		\label{eq:ac0_powers_residue}
		(z'_j)_k = \begin{cases}
			1, &\mbox{~если~} k = y_i \cdot m^{j'} \mbox{~для некоторых~} i,j'\in\mathbb{N},~~ j' > j
			\\ &\mbox{~и ни для какого~} j' \leq j \mbox{~не выполнено~} k = y_i \cdot m^{j'},
			\\
			0  &\mbox{~иначе}
			.
		\end{cases}
	\end{equation}
	(Знак неравенства в~\eqref{eq:ac0_powers_x_as_sum} ставится ввиду того, что возможен случай
	$y_i \cdot m^j = y_{i'} \cdot m^{j'}$ для $j\neq j'$.)
	Понятно, что $p(z_j)=0$.
	Таким образом,
	\begin{equation}
		p(x) \leq p(z_1) + p(z_2) + \dots + p(z_K) + p(z'_{K+1}) = p(z'_{K+1})
		.
	\end{equation}
	Заметим, что в силу определения~\eqref{eq:ac0_powers_residue} $z'_j$ каждый отрезок из $m^{j+1}$ элементов
	содержит не более одной единицы, и потому $p(z'_j) \leq m^{-j-1} < m^{-j}$.
	Таким образом, для любого $K\in\mathbb{N}$ выполнена оценка $p(x) < m^{-K}$,
	откуда $p(x) = 0$.
\end{proof}

\begin{corollary}
	\label{cor:ac0_powers_finite_set_of_numbers}
	Пусть $\{p_1, ..., p_k\} \subset \mathbb{N}$,
	\begin{equation}
		x_k = \begin{cases}
			1, &\mbox{~если~} k = p_1^{j_1}\cdot p_2^{j_2}\cdot ... \cdot p_k^{j_k} \mbox{~для некоторых~} j_1,...,j_k\in\mathbb{N},
			\\
			0  &\mbox{~иначе}.
		\end{cases}
	\end{equation}
	Тогда $x\in ac_0$.
\end{corollary}


%\begin{hypothesis}
%	Пусть $y=\{y_n\}$ и $z=\{z_n\}$ --- строго возрастающие последовательности,
%	$\chi_y,\chi_z\in\{0;1\}^\mathbb{N} \cap ac_0$.
%	Тогда почти сходится к нулю последовательность $x=\{x_k\}$, определённая соотношением
%	\begin{equation}
%		x_k = \begin{cases}
%			1, &\mbox{~если~} k = y_i \cdot z_j \mbox{~для некоторых~} i,j\in\mathbb{N},
%			\\
%			0  &\mbox{~иначе}
%			.
%		\end{cases}
%	\end{equation}
%\end{hypothesis}


\section{Исключение множителей и \\  нижний функционал Сачестона}



\begin{definition}
	Через $\psi(A)$, где $A\subset\mathbb{N}\setminus\{1\}$,
	будем обозначать последовательность из нулей и единиц,
	определяемую соотношением
	\begin{equation}
		(\psi(A))_k = \begin{cases}
			0, & \mbox{~если~} k = m\cdot a, m \in \mathbb{N}, a\in A,
			\\
			1  & \mbox{~иначе.}
		\end{cases}
	\end{equation}
\end{definition}

Так, например,
\begin{equation}
	\psi(\{2\}) = \psi(\{2, 4\}) = \psi(\{2,4,8,16,...\}) = (1,0,1,0,1,0,1,0,1,0,...),
\end{equation}

\begin{equation}
	\psi(\{3\}) = \psi(\{3,9,27,...\}) = (1,1,0,\;1,1,0,\;1,1,0,\;1,1,0,\;1,1,0,\;...),
\end{equation}

\begin{equation}
	\psi(\{2,3\}) = \psi(\{2,3,6\}) = (1,0,0,0,1,0,\;1,0,0,0,1,0,\;1,0,0,0,1,0,...),
\end{equation}


\begin{lemma}
	Для любого непустого $A\subset \mathbb{N} $ выполнено $\psi(A) \notin ac_1$.
\end{lemma}
\begin{proof}
	Пусть $a_1\in A$.
	Тогда из каждых идущих подряд $a_1$ элементов последовательности $\psi(A)$
	хотя бы один нулевой,
	следовательно,
	\begin{equation}
		p(\psi(A)) \leq \frac{a_1-1}{a_1} < 1
		.
	\end{equation}
\end{proof}

Оказывается, в более узком случае верно и более сильное утверждение.

\begin{theorem}
	\label{lem:ac0_primes_infinity_mutually_prime_subset}
	Пусть $A'$ "--- бесконечное подмножество попарно взаимно простых чисел
	(т.е. для любых двух чисел $a_1, a_2 \in A'$ их наибольший общий делитель равен единице).
	Тогда для любого $A \supset A' $ выполнено $q(\psi(A))=0$.
\end{theorem}
\begin{proof}
	Пусть $A' = \{ a_1, a_2, ..., a_j, ... \}$ и
	\begin{equation}
		\label{eq:ac0_primes_A_j_prod_des}
		A_j = \prod_{i=1}^j a_i
		.
	\end{equation}

	Предъявим такие номера $n_k$, что для любого $k$
	\begin{equation}
		\psi(A)_{n_k+1} = \psi(A)_{n_k+2} = \dots = \psi(A)_{n_k+k} = 0
		.
	\end{equation}
	Тем самым мы докажем, что отрезок из любого наперёд заданного количества нулей подряд
	встречается в последовательности $\psi(A)$ и, следовательно, $q(\psi(A)) = 0$.

	Действительно,
	пусть $n_1 = a_1 - 1$.
	Рассмотрим множество  $F_1 = \{ n_1 + A_1, n_1 + 2A_1, n_1 + 3A_1, \dots, n_1 + a_2A_1 \}$
	и отметим два следующих факта.

	Во-первых, пусть $f \in F_1$,
	тогда
	\begin{equation}
		f \equiv n_1 \mod a_1
		.
	\end{equation}
	Во-вторых, числа $a_2$ и $A_1$ взаимно просты.
	Следовательно, все $a_2$ чисел из множества $F_1$ дают разные остатки при делении на $a_2$.

	В качестве $n_2$ возьмём такое $f\in F_1$, что
	\begin{equation}
		f \equiv a_2 - 2 \mod a_2
		.
	\end{equation}
	Заметим, что тогда
	\begin{equation}
		n_2 + 1 \equiv n_1 + 1 \equiv 0 \mod a_1
	\end{equation}
	и
	\begin{equation}
		n_2 + 2 \equiv 0 \mod a_2
		,
	\end{equation}
	следовательно,
	$x_{n_2 + 1} = x_{n_2 + 2} = 0$.

	Полученные рассуждения несложно продолжить по индукции.

	Рассмотрим множество  $F_{j+1} = \{ n_j + A_j, n_j + 2A_j, n_j + 3A_j, \dots, n_j + a_{j+1}A_j \}$
	и отметим два следующих факта.

	Во-первых, пусть $f \in F_{j+1}$,
	тогда
	\begin{equation}
		f \equiv n_j \mod A_j
	\end{equation}
	и, в силу~\eqref{eq:ac0_primes_A_j_prod_des},
	\begin{equation}
		\begin{array}{rl}
		f &\equiv n_j \mod a_1
		\\
		f &\equiv n_j \mod a_2
		\\
		&\dots
		\\
		f &\equiv n_j \mod a_j
		.
		\end{array}
	\end{equation}
	Во-вторых, числа $a_{j+1}$ и $A_j$ взаимно просты, поскольку $a_{j+1}$ взаимно просто с каждым из чисел $a_1,...,a_j$.
	Следовательно, все $a_{j+1}$ чисел из множества $F_j$ дают разные остатки при делении на $a_{j+1}$.

	В качестве $n_{j+1}$ возьмём такое $f\in F_j$, что
	\begin{equation}
		f \equiv a_{j+1} - (j+1) \mod a_{j+1}
		.
	\end{equation}
	Заметим, что тогда
	\begin{equation}
		\begin{array}{l}
			n_{j+1} + 1 \equiv n_j + 1 \equiv n_{j-1} + 1 \equiv \dots \equiv n_2 + 1 \equiv n_1 + 1 \equiv 0 \mod a_1
			\\
			n_{j+1} + 2 \equiv n_j + 2 \equiv n_{j-1} + 2 \equiv \dots \equiv n_2 + 2 \equiv 0 \mod a_2
			\\
			\dots
			\\
			n_{j+1} + j \equiv n_j + j  \equiv 0 \mod a_j
		\end{array}
	\end{equation}
	и
	\begin{equation}
		n_{j+1} + (j+1) \equiv 0 \mod a_{j+1}
		,
	\end{equation}
	следовательно,
	$\psi(A)_{n_{j+1} + j+1} = \psi(A)_{n_{j} + j} = \dots = \psi(A)_{n_2 + 1} = \psi(A)_{n_2 + 2} = 0$.


\end{proof}

\begin{remark}
	Понятно, что в качестве примера бесконечного множества
	попарно взаимно простых чисел можно взять любое бесконечное множество простых чисел.
	Однако бывают бесконечные множества попарно взаимно простых чисел,
	не содержащие простых чисел вовсе, например множество
	\begin{equation}
		A = \{ 2\cdot 3,~5 \cdot 7,~11 \cdot 13,~17\cdot 19,~23\cdot29,~31\cdot 37,~...\},
	\end{equation}
	элементами которого яляются произведения пар соседних простых чисел.
\end{remark}

\begin{definition}
	Будем говорить, что множество $A\subset\mathbb{N}$ обладает $prp$-свойством (англ. ``pairwise relatively prime''),
	если для любого $n\in\mathbb{N}$ найдётся набор попарно взаимно простых чисел
	\begin{equation}
		\{a_{n,1}, a_{n,2}, ..., a_{n,n}  \} \subset A
		.
	\end{equation}
\end{definition}

Из доказательства теоремы~\ref{lem:ac0_primes_infinity_mutually_prime_subset} понятно,
что для множества $A$ в условии леммы достаточно потребовать $prp$-свойства.
Интересно, что на самом деле $prp$-свойство эквивалентно условиям,
наложенным на множество $A$ в теореме~\ref{lem:ac0_primes_infinity_mutually_prime_subset}.

\begin{lemma}
	\label{lem:ac0_primes_infinity_mutually_prime_subset_equiv_to_prp_property}
	Пусть множество $A$ обладает $prp$-свойством.
	Тогда существует бесконечное подмножество $A'\subset A$ попарно взаимно простых чисел.
\end{lemma}
\begin{proof}
	Зафиксируем $f_0\in A$, $f_0 \neq 1$ и представим $A$ в виде объединения трёх попарно непересекающихся множеств:
	\begin{equation}
		A = \{f_0\} \cup E \cup F
		,
	\end{equation}
	где
	\begin{equation}
		E = \{ a \in A \mid a \mbox{~не взаимно просто с~} f_0 \}
		,
	\end{equation}
	\begin{equation}
		F = \{ a \in A \mid a \mbox{ ~~~~взаимно просто с~} f_0 \}
		.
		%TODO: нормальное выравнивание!!
	\end{equation}

	Пусть разложение $f_0$ на простые множители имеет вид
	\begin{equation}
		f_0 = p_1^{j_1} \cdot p_2^{j_2} \cdot ... \cdot p_k^{j_k}
		.
	\end{equation}

	Тогда множество $E$ можно представить в виде объединения (возможно пересекающихся) множеств:
	\begin{equation}
		E = \bigcup_{i=1}^{k} E_i,\quad E_i = \{a\in E \mid a \mbox{~кратно~} p_i\}
		.
	\end{equation}

	Покажем, что множество $F$ обладает $prp$-свойством.
	Действительно, зафиксируем $n\in\mathbb{N}$.
	Так как множество $A$ обладает $prp$-свойством,
	то в нём найдётся подмножество попарно взаимно простых чисел
	$$G=\{a_1, a_2, ..., a_{n+k-1}, a_{n+k}\}\subset A.$$

	Если $f_0\in G$, то $G\setminus f_0 \subset F$ в силу построения множеств $G$ и $F$, и требуемый набор попарно взаимно простых чисел предъявлен.

	Пусть теперь $f_0\notin G$.
	Заметим, что в каждое из множеств $E_i$ может входит не более одного элемента множества $G$
	(в силу того, что при фиксированном $i$ все элементы множества $E_i$ имеют нетривиальный общий делитель).
	Следовательно, как минимум $n$ элементов из $G$ принадлежат множеству $F$,
	и требуемый набор попарно взаимно простых чисел снова предъявлен.

	Поскольку множество $F$ обладает $prp$-свойством, то оно, очевидно, счётно.

	Итак, нам удалось получить число $f_0\in A$ и счётное множество $F$, обладающее $prp$-свойством
	и состоящее из чисел, взаимно простых с $f_0$.
	Продолжая по индукции, получим требуемое счётное множество попарно взаимно простых чисел.
\end{proof}

%Условие теоремы~\ref{lem:ac0_primes_infinity_mutually_prime_subset}
%является не только достаточным, но и необходимым.

\begin{lemma}
	\label{lem:ac0_primes_q_psi_A_0_causes_prp}
	Пусть для множества $A\subset\mathbb{N}\setminus\{1\}$ выполнено $q(\psi(A))=0$.
	Тогда $A$ обладает $prp$-свойством.
\end{lemma}

\begin{proof}
	Предположим противное: пусть $A$ не обладает $prp$-свойством.
	Тогда существует $n\in\mathbb{N}$ такое, что из множества $A$ можно выбрать $n$
	попарно взаимно простых чисел, но нельзя выбрать $n+1$.

	Пусть $\{a_1, a_2, ..., a_n\}\subset A$ "--- набор попарно взаимно простых чисел.
	Так как $q(\psi(A))=0$, то в последовательности $\psi(A)$ найдётся отрезок, состоящий сплошь из нулей,
	любой наперёд заданной длины.
	Выберем $k$ таким, что
	\begin{equation}
		(\psi(A))_{k+1} = (\psi(A))_{k+2} = ... = (\psi(A))_{k+a_1a_2\cdots a_n} = 0
		.
	\end{equation}
	Тогда существует такое число $k_0$, $k+1 \leq k_0 \leq k+a_1a_2\cdots a_n$,
	что $k_0$ даёт в остатке $1$ при делении на $a_1a_2\cdots a_n$.
	Так как $(\psi(A))_{k_0} = 0$, что $k_0 = m\cdot a_0$, где $m\in\mathbb{N}$, $a_0\in A$.
	С другой стороны, $k_0$ взаимно просто с каждым из чисел $a_1, a_2, ..., a_n$.
	Следовательно, $a_0$ также взаимно просто с каждым из чисел $a_1, a_2, ..., a_n$,
	и $\{a_0, a_1, a_2, ..., a_n\}\subset A$ "--- набор из $n+1$ попарно взаимно простых чисел.
	Полученное противоречие завершает доказательство.
\end{proof}

Таким образом,
из лемм~\ref{lem:ac0_primes_infinity_mutually_prime_subset},~\ref{lem:ac0_primes_infinity_mutually_prime_subset_equiv_to_prp_property}~и~\ref{lem:ac0_primes_q_psi_A_0_causes_prp}
незамедлительно следует
\begin{theorem}
	Пусть $A\subset \mathbb{N}\setminus\{1\}$.
	Тогда следующие три условия эквивалентны:
	\begin{enumerate}
		\item
			$A$ обладает $prp$-свойством
		\item
			В $A$ существует счётное подмножество попарно взаимно простых чисел
		\item
			$q(\psi(A))=0$.
	\end{enumerate}
\end{theorem}

\section{Исключение множителей и \\ верхний функционал Сачестона}


Перейдём теперь к изучению верхнего функционала Сачестона $p(\psi(A))$.

\begin{theorem}
	\label{thm:ac0_primes_p_psi_A_prod}
	Пусть $A = \{a_1, a_2, ..., a_n,...\}$ "--- счётное множество попарно взаимно простых чисел.
	Тогда
	\begin{equation}
		p(\psi(A)) = \prod_{j=1}^\infty \frac{a_j -1}{a_j}
		.
	\end{equation}
\end{theorem}

\begin{proof}
	Зафиксируем $k\in\mathbb{N}$.

	Заметим, что верхний функционал Сачестона можно представить в виде
	\begin{equation}
		\label{eq:ac0_primes_px_lim_pnx}
		p(x) = \lim_{n\to\infty} p_n(x)
		,
	\end{equation}
	где
	\begin{equation}
		p_n(x) = \sup_{m\in\mathbb{N}}  \frac{1}{n} \sum_{j=m}^{m+n-1} x_j
		.
	\end{equation}

	Поскольку предел~\eqref{eq:ac0_primes_px_lim_pnx} существует, то для его оценки можно использовать предел подпоследовательности
	$p_{n_k}(x)$, где $n_k = a_1\cdot a_2 \cdot ... \cdot a_k$.


	Cреди первых $n_k$ элементов последовательности $\psi(A)$
	ровно $\prod_{j=1}^k \frac{a_j-1}{a_j}$ единиц, поскольку комбинации остатков от деления
	на взаимно простые числа $a_1, a_2, ..., a_k$ <<не успевают>> повторяться.
	Следовательно,
	\begin{equation}
		p_{n_k}(\psi(A)) =
		\sup_{m\in\mathbb{N}}  \frac{1}{n_k} \sum_{j=m}^{m+n_k-1} (\psi(A))_j
		\geq
		\frac{1}{n} \sum_{j=1}^{n_k} (\psi(A))_j
		=
		\prod_{i=1}^k \frac{a_i-1}{a_i}
		.
	\end{equation}
	Переходя к пределу по $k$, имеем
	\begin{equation}
		\label{eq:ac0_primes_p_psi_A_lower_bound}
		p(\psi(A)) \geq \lim_{k\to \infty} \prod_{i=1}^k \frac{a_i-1}{a_i}
		=
		\prod_{i=1}^\infty \frac{a_i-1}{a_i}
		.
	\end{equation}

	Заметим теперь, что в любом отрезке последовательности $\psi(A)$ длины $n_k$
	содержится не более $\prod_{j=1}^k \frac{a_j-1}{a_j}$ единиц
	(могут попадаться <<дополнительные>> нули "--- элементы с индексами, кратными $a_j$ для $j>k$).
	Значит,
	\begin{equation}
		p_{n_k}(\psi(A)) =
		\sup_{m\in\mathbb{N}}  \frac{1}{n_k} \sum_{j=m}^{m+n_k-1} (\psi(A))_j
		\leq
		\prod_{i=1}^k \frac{a_i-1}{a_i}
		.
	\end{equation}
	Снова перейдя к пределу по $k$, получим
	\begin{equation}
		\label{eq:ac0_primes_p_psi_A_upper_bound}
		p(\psi(A)) \leq \lim_{k\to \infty} \prod_{i=1}^k \frac{a_i-1}{a_i}
		=
		\prod_{i=1}^\infty \frac{a_i-1}{a_i}
		.
	\end{equation}

	Сопоставив~\eqref{eq:ac0_primes_p_psi_A_lower_bound} и~\eqref{eq:ac0_primes_p_psi_A_upper_bound}, получим утверждение теоремы.
\end{proof}

\begin{remark}
	В случае, когда $\mathbb{N}\setminus A$ конечно, из следствия~\ref{cor:ac0_powers_finite_set_of_numbers} непосредственно вытекает, что $\psi(A)\in ac_0$
	и, соответственно, $p(\psi(A))=0$.
\end{remark}



В~\cite{euler1737variae} доказана следующая
\begin{theorem}
	\label{thm:Euler_reverse_primes_diverge}
	Ряд обратных простых чисел
	\begin{equation}
		\frac{1}{2} + \frac{1}{3} + \frac{1}{5} + \frac{1}{7} + ...
		=
		\sum_j \frac{1}{j},
	\end{equation}
	где $j$ пробегает все простые числа, расходится.
\end{theorem}

Из теоремы~\ref{thm:Euler_reverse_primes_diverge} выводится следующее утверждение.

\begin{corollary}
	\label{cor:ac0_primes_sum_ln_diverges}
	Ряд
	\begin{equation}
		\sum_j \ln \frac{j+1}{j}
		,
	\end{equation}
	где $j$ пробегает все простые числа, расходится.
\end{corollary}
\begin{proof}
	Из второго замечательного предела:
	\begin{equation}
		\lim_{n\to\infty} \left( 1 + \frac{1}{n} \right)^n = e
	\end{equation}
	получаем, что для достаточно больших $n$
	\begin{equation}
		\label{eq:ac0_primes_ineq_2nd_nice_lim}
		\sqrt{e} < \left( 1 + \frac{1}{n} \right)^n < e^2
		.
	\end{equation}
	Прологарифмировав~\eqref{eq:ac0_primes_ineq_2nd_nice_lim}, получим
	\begin{equation}
		\frac{1}{2} < n \ln  1 + \frac{1}{n} < 2
		,
	\end{equation}
	откуда
	\begin{equation}
		\ln \frac{n+1}{n} > \frac{1}{2n}
		.
	\end{equation}
	Пусть теперь $j$ пробегает все достаточно большие простые числа, тогда
	\begin{equation}
		\sum_j \ln \frac{j+1}{j}  > \sum_j \frac{1}{2j} = \frac{1}{2} \sum_j \frac{1}{j}
		.
	\end{equation}
	Ряд в правой части есть ряд обратных простых чисел, который расходится в силу теоремы~\ref{thm:Euler_reverse_primes_diverge}.
	Про признаку сравнения отсюда следует, что и ряд в левой части расходится.
\end{proof}

Из следствия~\ref{cor:ac0_primes_sum_ln_diverges} и теоремы~\ref{thm:ac0_primes_p_psi_A_prod} незамедлительно вытекает
\begin{corollary}
	Пусть $\varepsilon \in {[} 0; 1)$.
	Тогда существует такое счётное множество простых чисел $A$, что $p(\psi(A))=\varepsilon$.
\end{corollary}


\begin{corollary}
	Пусть $A=\{a_1, a_2, ..., a_n, ...\}\subset\mathbb{N}$ есть множество попарно взаимно простых чисел.
	Тогда следующие три условия эквивалентны:
	\begin{enumerate}
		\item
			$\psi(A)\in ac$.
		\item
			$\psi(A)\in ac_0$.
		\item
			$\prod_{j=1}^\infty \frac{a_j -1}{a_j} = 0$.
	\end{enumerate}
\end{corollary}



\printbibliography{}

\end{document}

