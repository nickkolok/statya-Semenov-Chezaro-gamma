\documentclass[a4paper,11pt]{book}
%\documentclass{book}
%\usepackage{cmap}
\usepackage[cp1251]{inputenc}
\usepackage{latexsym}
\usepackage{amsmath, amsfonts, amssymb}
\usepackage[russian,english]{babel}
\usepackage{epsfig, epic}
%\usepackage{eepic, rotating}
\usepackage{graphicx}
\usepackage{wrapfig}
\usepackage{calrsfs}
\usepackage{bbm}
%\usepackage{bbold}
\usepackage{mathrsfs}
\usepackage{epsfig}
\usepackage{amscd}
\usepackage[all]{xy}

%\ifx\pdfoutput\undefined
%\usepackage{graphicx}
% \DeclareGraphicsExtensions{.eps}
% \else
%\usepackage[pdftex]{graphicx}
% \DeclareGraphicsExtensions{.jpg}
%\fi
% \graphicspath{{pict/}}
%% \DeclareGraphicsExtensions{.eps,.jpg}
%%\DeclareGraphicsExtensions{.jpg}
%%\DeclareGraphicsRule{.bmp}{bmp}{}{}


%\usepackage {curves, eepic, epic}
%\usepackage{amsxtra,  amsthm}
%\input diagxy




%\usepackage{makeidx}\makeindex



\sloppy \textheight = 230mm \textwidth = 155mm \topmargin = -3mm
\oddsidemargin=3mm \evensidemargin=4pt

 \NeedsTeXFormat{LaTeX2e} \ProvidesPackage{bbold}[1994/04/06 Bbold symbol package]
 \newcommand{\bbfamily}{\fontencoding{U}\fontfamily{bbold}\selectfont}
 \newcommand{\textbb}[1]{{\bbfamily#1}}
 \DeclareMathAlphabet{\mathbbt}{U}{bbold}{m}{n}

\begin{document}



\makeatletter%{@}


\renewcommand{\thefootnote}{\fnsymbol{}}

\newcommand{\Author}[2]{{
\begin{center}\textbf{#1}\footnote{$\copyright$\ \Year\ #2}\end{center} \medskip}
    \renewcommand{\@evenhead}
    {\raisebox{0pt}[\headheight][0pt]%
{\vbox{\hbox to\textwidth{\thepage \hfill\strut {\sl
#2}}\hrule}}}}

\newcommand{\shorttitle}[1]{
\renewcommand{\@oddhead}{\raisebox{0pt}[\headheight][0pt]%
{\vbox{\hbox to\textwidth{\strut {\sl #1}%\rightmark
\hfill\thepage}\hrule}}} }

 \headsep=3mm
 \renewcommand{\section}[1]{\medskip\begin{center}\textbf{#1}\end{center}}
 \newcommand{\subsec}[1]{\par\smallskip{\bf #1}}
 \newcommand{\subsubsec}[1]{\qquad\qquad{\bf #1}}
 \def\beginproof{\smallskip\par\text{$\vartriangleleft$}}
 \def\endproof{\text{$\vartriangleright$}}
 \def\Endproof{\text{$\vartriangleright$}\smallskip}

 \makeatother%@


%\newcommand{\beginarticle}{\newpage\thispagestyle{empty}
%\begin{flushright}
%\sl Владикавказский математический журнал\\
%\Year, Том~\Vol, Выпуск~\issu, С.~\thepage--\endpage
%\end{flushright}
%\bigskip}

\newcommand{\beginarticleeng}{\newpage\thispagestyle{empty}
\begin{flushright}
\sl Vladikavkaz Mathematical Journal\\
\Year, Volume~\Vol, Issue~\issu, P.~\thepage--\endpage
\end{flushright}
\bigskip}

\newcommand{\metadanrus}{
\begin{flushright}
\sl Владикавказский математический журнал\\
\Year, Том~\Vol, Выпуск~\issu, С.~\beginpage--\endpage
\end{flushright}
\bigskip}


\newcommand{\metadaneng}{
\begin{flushright}
\sl Vladikavkaz Mathematical Journal\\
\Year, Volume~\Vol, Issue~\issu, P.~\beginpage--\endpage
\end{flushright}
\bigskip}

\newcommand{\UDC}[1]{{\small{{\bf УДК} #1}}\par}
\newcommand{\DOI}[1]{{\small{{\bf DOI} #1}}\par\vspace{4mm}}
\newcommand{\Title}[1]{\begin{center}\uppercase{#1}\end{center} \vspace{0mm}}
\newcommand{\TITLE}[1]{\begin{center}{#1}\end{center} \vspace{0mm}}
\renewcommand{\title}[1]{\begin{center}#1\end{center} \vspace{0mm}}
\newcommand{\Abstract}[1]{\hangindent17pt\hangafter=0\noindent{\footnotesize \bf Аннотация.\ }{\footnotesize#1} \bigskip\par\medskip}
\newcommand{\Abstracteng}[1]{\hangindent17pt\hangafter=0\noindent{\footnotesize \bf Abstract.\ }{\footnotesize#1} \bigskip\par\medskip}


\newcommand{\Lit}{\medskip\begin{center}\textbf{Литература}\end{center}}
\newcommand{\Ref}{\medskip\begin{center}\textbf{References}\end{center}}


%\newcommand{\endLit}{\end{enumerate}
%
\newcommand{\bib}[2]{{\baselineskip=11pt\footnotesize\item{}
\textsl{#1\/}\ #2}}
%
\newcommand{\Address}[1]{\par\bigskip\baselineskip=11pt\hangindent24pt
\hangafter=0\noindent{\footnotesize#1}\par\normalsize}
%\newcommand{\email}[1]{#1}}
\newcommand{\Endarticle}[2]{\bigskip\bigskip{\textit{#1\hfill Статья поступила
#2}} \vfill\eject}
%
\newcommand{\Endproc}{\rm}
\newcommand{\Proclaim}[1]{\smallskip{\textbf{#1\/~}}\sl}
\newcommand{\proclaim}[1]{{\textbf{#1\/~}}\sl}
\newcommand{\Teorema}[1]{\smallskip{\textbf{Теорема#1.\/~}}\sl}
\newcommand{\teorema}[1]{{\textbf{Теорема#1.\/~}}\sl}
\newcommand{\Sledstvie}[1]{\smallskip{\textbf{Следствие#1.\/~}}\sl}
\newcommand{\sledstvie}[1]{{\textbf{Следствие#1.\/~}}\sl}
\newcommand{\Lema}[1]{\smallskip\textbf{{Лемма#1}.\/~}\sl}
\newcommand{\lema}[1]{\textbf{{Лемма#1}.\/~}\sl}
\newcommand{\Predl}[1]{\smallskip\textbf{{Предложение#1}.\/~}\sl}
\newcommand{\predl}[1]{\textbf{{Предложение#1}.\/~}\sl}
\newcommand{\Utv}[1]{\smallskip\textbf{{Утверждение#1}.\/~}\sl}
\newcommand{\utv}[1]{\textbf{{Утверждение#1}.\/~}\sl}
%-----------------
\newcommand{\Def}[1]{\smallskip{\sc Определение#1.}}
\newcommand{\deff}[1]{{\sc Определение#1.}}
\newcommand{\Zam}[1]{\smallskip{\sc Замечание#1.}}
\newcommand{\zam}[1]{{\sc Замечание#1.}}
\newcommand{\Primer}[1]{\smallskip{\sc Пример#1.}}
\newcommand{\primer}[1]{{\sc Пример#1.}}
\newcommand{\Zadacha}[1]{\smallskip{\sc Задача#1.}}
\newcommand{\zadacha}[1]{{\sc Задача#1.}}


%
%%%%%%%%%%%%%%%%%%%%%%%%%%%%%%%%%%%%%%%%%%%%%%%%%%%%%%%%
%%%%%%%%%     English
\newcommand{\Lemma}[1]{\smallskip\textbf{{Lemma#1}.\/~}\sl}
\newcommand{\lemma}[1]{\textbf{{Lemma#1}.\/~}\sl}
\newcommand{\Theorem}[1]{\smallskip\textbf{{Theorem#1}.\/ }\sl}
\newcommand{\theorem}[1]{{\textbf{Theorem#1.\/~}}\sl}
\newcommand{\Corollary}[1]{\smallskip\textbf{{Corollary#1}.\/~}\sl}
\newcommand{\corollary}[1]{\textbf{{Corollary#1}.\/~}\sl}
\newcommand{\Proposition}[1]{\smallskip\textbf{{Proposition#1}.\/~}\sl}
\newcommand{\proposition}[1]{\textbf{{Proposition#1}.\/~}\sl}
\newcommand{\Crit}[1]{\smallskip\textbf{{Criterion#1}.\/ }\sl}
%-------------
\newcommand{\defin}[1]{{\sc Definition#1.}}
\newcommand{\Definition}[1]{\smallskip{\sc Definition#1.}}
\newcommand{\Remark}[1]{\smallskip{\sc Remark#1.}~}
\newcommand{\Ex}[1]{\smallskip{\sc Example#1.}}
\newcommand{\Problem}[1]{{\sc Problem#1:}}
\newcommand{\Note}[1]{{\sc Note#1.}}
%%%%%%%%%%%%%%%%%%%%%%%%%%%%%%%%%%%%%%%%%%%%%%%%%%%%%%%%%%%%%%

%%%%%%%%------------
\newcommand{\slim}{\mathop{\fam0 s\text{-}\!\lim}}
\def\osum{\mathop{o\text{-}\!\sum}}
\def\oleqq{\mathop{\raisebox{1pt}{$\scriptstyle\bigcirc$}\hspace{-7.7pt}\raisebox{1.6pt}{$\scriptscriptstyle\leq$}\hspace{2pt}}}
\def\bbold#1{{\mathbbm #1}}
%\def\BbbO{\bbold O}
%\def\Bbb#1{\fontencoding{}\fontfamily{bbold}\fontseries{m}\fontshape{n}
%\fontsize\selectfont #1}}
\def\Cal#1{\mathcal{#1}}
\def\goth#1{\frak{#1}}

\def\No{\textnumero}
%%%%---------
\newcommand{\pd}[2]{\frac{\partial #1}{\partial #2}}
\newcommand{\pg}[2]{\frac{{\partial}^2 #1}{\partial #2^2}}
\renewcommand{\Re}{\mathop{\fam0 Re}}
\newcommand{\Div}{\mathop{\fam0 div}}
\newcommand{\supp}{\mathop{\fam0 supp}}
\newcommand{\dom}{\mathop{\fam0 dom}}
\newcommand{\const}{\mathop{\fam0 const}}
\newcommand{\grad}{\mathop{\fam0 grad}}
\newcommand{\sgn}{\mathop{\fam0 sgn}}
\newcommand{\sign}{\mathop{\fam0 sign}}
\newcommand{\rot}{\mathop{\fam0 rot}}
\newcommand{\im}{\mathop{\fam0 Im}}
\newcommand{\cl}{\mathop{\fam0 cl}}
\newcommand{\mix}{\mathop{\fam0 mix}}
\newcommand{\ind}{\mathop{\fam0 ind}}
\newcommand{\proj}{\mathop{\fam0 proj}}
\newcommand{\Ker}{\mathop{\fam0 Ker}}
\newcommand{\card}{\mathop{\fam0 card}}
\newcommand{\diag}{\mathop{\fam0 diag}}
\newcommand{\loc}{\mathop{\fam0 loc}}
\newcommand{\conv}{\mathop{\fam0 conv}}
\newcommand{\Int}{\mathop{\fam0 Int}}
\newcommand{\Sig}{\mathop{\fam0 Sig}}
\newcommand{\new}{\mathop{\fam0 new}}
\newcommand{\Inf}{\mathop{\fam0 Inf}}
\newcommand{\Disp}{\mathop{\fam0 Disp}}
\newcommand{\Orth}{\mathop{\fam0 Orth}}

\newcommand{\mC}{\mathbb{C}}
\newcommand{\N}{\mathbb{N}}
\newcommand{\R}{\mathbb{R}}
\newcommand{\E}{\mathbb{E}}



\renewcommand{\leq}{\leqslant}
\renewcommand{\geq}{\geqslant}
\renewcommand{\le}{\leqslant}
\renewcommand{\ge}{\geqslant}
%---------------------
\def\reduce{\mskip-5mu }

%%%жирные нормы

\def\Bnorml{\mathopen{\kern1pt\/ \vrule height6.5pt depth2.2pt width1pt\kern1.5pt}}
\def\Bnormr{\mathclose{\kern1.5pt \vrule height6.5pt depth2.2pt width1pt\kern1pt}}
\def\widevert{\kern1.5pt \vrule height7.5pt depth2.5pt width1pt\kern1pt}

\def\[{\mathopen{\kern1pt\/ \vrule height8pt depth2.2pt width1pt\kern1.5pt}}
\def\]{\mathclose{\kern1.5pt \vrule height8pt depth2.2pt width1pt\kern1pt}}

%%%тройные нормы
\def\tnorm{|\hspace{-1pt}|\hspace{-1pt}|}                        %|||
\def\bigtnorm{\big|\hspace{-1.5pt}\big|\hspace{-1.5pt}\big|}     %\big|||
\def\biggtnorm{\bigg|\hspace{-1.5pt}\bigg|\hspace{-1.5pt}\bigg|} %\bigg|||




\def\sfatop{\mathopen{\kern1pt \vrule height5.5pt depth1.5pt width1pt\kern1pt}}
\def\sfatcl{\mathclose{\kern1pt\vrule height5.5pt depth1.5pt width1pt\kern1pt}}

\def\leftbbr{\mathopen{\kern1pt \vrule height14pt depth8pt width1pt\kern1.5pt}}
\def\leftbbrr{\mathopen{\kern1pt\vrule height16pt depth13pt width1pt\kern1.5pt}}
\def\rightbbrr{\mathclose{\kern1.5pt\vrule height16pt depth13pt width1pt\kern1pt}}
\def\rightbbr{\mathclose{\kern1.5pt\vrule height14pt depth8pt width1pt\kern1pt}}

%\def\reduce{\mskip-5mu }
%\def\Bnorml{\reduce\left\bracevert\reduce\vphantom{X}}
%\def\Bnormr{\vphantom{X}\reduce\right\bracevert\reduce}
%---------------------
\newcommand{\shortpage}{\enlargethispage{-\baselineskip}}
\newcommand{\largepage}{\enlargethispage{+\baselineskip}}


 \def\olim{\mathop{o\text{-}\fam0 lim\,}}
 \def\bolim{\mathop{bo\text{-}\fam0 lim\,}}
 \def\llim{\mathop{l\text{-}\fam0 lim\,}}
 \def\Otimes{\overset{\underline{\phantom{\ \,}}}\otimes}


 \def\to{\rightarrow}
