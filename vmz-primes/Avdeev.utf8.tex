\input vmj.utf8.tex

\Russian

 \def\endpage{??}




 \let\thefootnote\relax\setcounter{section}{0}\setcounter{equation}{0}


 \UDC{ 517.982.276 % Пространства последовательностей и матриц
     + 511.172    }% Мультипликативная структура кольца целых чисел (включая делимость, наименьшее общее кратное, наибольший общий делитель, сравнения, степенные вычеты, квадратичные вычеты)
 \DOI{ } %заполняется редакцией



 \Title{Функционалы Сачестона и разложение чисел на простые множители}

 \shorttitle{Функционалы Сачестона и простые множители}

 \begin{center}
	Работа выполнена в Воронежском госуниверситете при поддержке РНФ, грант 19-11-00197.
 \end{center}


 \begin{center}
 {Н.~Н.~Авдеев$^{1}$\\[4pt]}
 {\rm\footnotesize{$^1$\,
 Воронежский государственный университет\\
 Россия, 394006, Университетская пл., 1\\
 E-mail: nickkolok@mail.ru, avdeev@math.vsu.ru}}
 \end{center}


 \Abstract{
 
 
 } %аннотация не менее 200-250слов

 {\hangindent17pt\hangafter=0\noindent\footnotesize{\bf Ключевые слова:}
	ограниченная последовательность,
	пространство ограниченных последовательностей,
	банахов предел,
	функционалы Сачестона,
	последовательности из нулей и единиц,
	разложение на множители,
	подмножеcтва натуральных чисел
 \par} %ключевые слова

 \smallskip

 {\hangindent17pt\hangafter=0\noindent\footnotesize {\bf Mathematical Subject Classification (2000):}
	46B45,
	11A51,
	11B57
	\par} %математические классификаторы








 %оформление параграфов
 \section{N. Название параграфа}


 %оформление теорем
 \Teorema{ 1 {\rm (название теоремы при наличии)}}
 %текст теоремы
 \Endproc


 %оформление доказательства
 \beginproof
 %текст доказательства
 \endproof


 %оформление леммы
 \Lema{ 1}
 %текст леммы
 \Endproc


 %оформление следствия
 \Sledstvie{}
 %текст следствия
 \Endproc

 %оформление замечания
 \Zam{}
 %текст замечания

 %оформление примера
 \Primer{}
 %текст примера




%%%% Список литературы на рус. яз.

% При наличии DOI обязательно указывать!!!

\Lit

 \begin{enumerate}
 {\footnotesize
 \itemsep=0pt
 \parskip=0pt


 \bib{Канторович Л.~В., Акилов Г.~П.}
 {Функциональный анализ.---М.: Наука,  1984.---750~с.} %ссылка на книгу

 \bib{Рябых~В.~Г.}
 {Приближение неаналитических функций аналитическими~/\!/ Мат. сб.---2006.---Т.~191, \No~2.---С.~87--94.}%ссылка на журнал

 %\bib{Дурдиев Д.~К., Тотиева Ж.~Д.}
 %Задача об определении одномерного ядра электровязкоупругости~/\!/
 %Сиб. мат. журн.---2017.---Т.~58, \No~3.---С.~553--572. DOI: 10.17377/smzh.2017.58.307. %ссылка на журнал

 \bib{Бурчаев Х.~Х., Рябых В.~Г., Рябых Г.~Ю.}
 {Некоторые свойства экстремальных функций линейных функционалов над пространствами Харди и Бергмана~/\!/
 Мат. форум. Т. 9. Исслед. по мат. анализу, дифференц. уравнениям и мат.
 моделированию.---Владикавказ: ЮМИ ВНЦ РАН, 2015.---С.~125--139.---(Итоги науки. Юг России).} %ссылка на сборник трудов

 }
 \end{enumerate}



 \Address{\textit{Статья поступила  чч.мм.гггг.}\\[4pt]  % заполняется редакцией
 \textsc{ФИО полностью}\\
 место работы,\\
 \textsl{должность}\\
 полный адрес места работы\\
 E-mail:
 \par}




 \newpage

 % Заполняем данные на английском языке!


 \bigskip
 \begin{center}
 НАЗВАНИЕ СТАТЬИ НА АНГЛ. ЯЗ. (заглавными буквами)
 \end{center}


 \begin{center}
 Фамилия, И.~О. на англ. яз.$^{1}$\\[4pt]
 {\rm\footnotesize{$^1$\,
 место работы  на англ. яз.\\
 полный адрес места работы на англ. яз.\\
 E-mail: }}
 \end{center}



 {\hangindent0pt\hangafter=0\footnotesize{\textbf{Abstract.} Аннотация на англ. яз.
 \par}

 {\hangindent0pt\hangafter=0\footnotesize{\bf Key words:}~ключевые слова на англ. яз.
 \par}



%%%% Список литературы на англ. яз.

% При наличии DOI обязательно указывать!!!

% Указывать переводную версию статьи, книги, сборника при наличии!


 \Ref

 \begin{enumerate}
 {\footnotesize
 \itemsep=0pt
 \parskip=0pt

 \bib{Kantorovich, L.~V. and Akilov, G.~P.}
 {\it Funktsional'nyy Analliz} [Functional Analysis],
 Moscow,
 Nauka,
 1984,
 750~p.
 (in Russian).


 \bib{Ryabykh, V.~G.}
 Approximation of non-analytic functions by analytic functions.
 {\it Sbornik: Mathematics},
 2006,
 vol. 197,
 no. 2,
 pp. 225--233.


 \bib{Durdiev, D.~K. and Totieva, Zh.~D.} The Problem of Finding the One-Dimensional Kernel of the Thermoviscoelasticity Equation,
 \textit{Mathematical Notes},
 2018,
 vol.~103,
 no.~1--2,
 p.~118--132.
 DOI: 10.1134/S0001434618010145.


 \bib{Burchaev, Kh.~Kh., Ryabykh, V.~G. and Ryabykh, G.~Yu.}
 Some properties of the extremal functions of linear functionals on Hardy and Bergman spaces.
 \textit{Mat. Forum. T. 9. Issledovaniya po Matematicheskomu Analizu, Differentsialnym Uravneniyam i Matematicheskomu Modelirovaniyu $($Itogi Nauki. Yug Rossii$)$}
 [Math. Forum. Vol. 9. Studies in Mathematical Analysis, Differential Equations, and Mathematical Modeling (Review of Science: The South of Russia)],
 Vladikavkaz, SMI VSC RAS,
 2015,
 pp.~125--139
 (in Russian).

}
\end{enumerate}




 \Address{\textit{Received  }\\[4pt] % заполняется редакцией
 \textsc{ФИО полностью на англ. яз.}\\
 место работы  на англ. яз.,\\
 полный адрес места работы на англ. яз.,\\
 \textsl{Должность на англ. яз.}\\
 E-mail:
 \par}


\end{document}
