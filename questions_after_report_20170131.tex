\documentclass[a4paper,14pt]{article} %размер бумаги устанавливаем А4, шрифт 12пунктов
\usepackage[T2A]{fontenc}
\usepackage[utf8]{inputenc}
\usepackage[english,russian]{babel} %используем русский и английский языки с переносами
\usepackage{amssymb,amsfonts,amsmath,mathtext,cite,enumerate,float,amsthm} %подключаем нужные пакеты расширений
\usepackage[unicode,colorlinks=true,citecolor=black,linkcolor=black]{hyperref}
%\usepackage[pdftex,unicode,colorlinks=true,linkcolor=blue]{hyperref}
\usepackage{indentfirst} % включить отступ у первого абзаца
\usepackage[dvips]{graphicx} %хотим вставлять рисунки?
\graphicspath{{illustr/}}%путь к рисункам

\makeatletter
\renewcommand{\@biblabel}[1]{#1.} % Заменяем библиографию с квадратных скобок на точку:
\makeatother %Смысл этих трёх строчек мне непонятен, но поверим "Запискам дебианщика"

\usepackage{geometry} % Меняем поля страницы.
\geometry{left=2cm}% левое поле
\geometry{right=1cm}% правое поле
\geometry{top=2cm}% верхнее поле
\geometry{bottom=2cm}% нижнее поле

\renewcommand{\theenumi}{\arabic{enumi}}% Меняем везде перечисления на цифра.цифра
\renewcommand{\labelenumi}{\arabic{enumi}}% Меняем везде перечисления на цифра.цифра
\renewcommand{\theenumii}{.\arabic{enumii}}% Меняем везде перечисления на цифра.цифра
\renewcommand{\labelenumii}{\arabic{enumi}.\arabic{enumii}.}% Меняем везде перечисления на цифра.цифра
\renewcommand{\theenumiii}{.\arabic{enumiii}}% Меняем везде перечисления на цифра.цифра
\renewcommand{\labelenumiii}{\arabic{enumi}.\arabic{enumii}.\arabic{enumiii}.}% Меняем везде перечисления на цифра.цифра

\sloppy


\renewcommand\normalsize{\fontsize{14}{25.2pt}\selectfont}

\begin{document}
% !!!
% Здесь начинается реальный ТеХ-код
% Всё, что выше - беллетристика

{\center\bf
	Вопросы, возникшие по итогам доклада
}

\begin{enumerate}
	\item
		(У меня)
		\\
		Нужно ли аккуратное доказательство того, что $l_1 \cap \mathfrak{B} = \varnothing$ ?
		Этот вопрос возник при попытке рассказать,
		почему я не могу выписать формулу какого-нибудь банахова предела в явном виде.
	\item
		(У проф. В.А. Костина)
		\\
		Что будет, если вместо оператора Чезаро использовать оператор
		\begin{equation}
			(Kx)_n = \frac{1}{n(n+1)}\sum_{m=1}^n mx_m ~~~ \mbox{?}
		\end{equation}
	\item
		(Кажется, у А.И. Корчагина, Санкт-Петербург --- надо уточнять?)
		\\
		Существуют ли банаховы пределы $B$ такие, что
		\begin{equation}
			B(x_1,x_2,...) \cdot B(y_1,y_2,...) = B(x_1\cdot y_1, x_2\cdot y_2, ...)  ~~~ \mbox{?}
		\end{equation}
		Автор вопроса пояснил, что это было бы естественно с точки зрения $C^*$--алгебр,
		поскольку дистрибутивность относительно умножения верна для <<обычных>> пределов.
		\\
		Сужение задачи от меня:
		Существуют ли банаховы пределы $B$ такие, что для $n\in\mathbb{Z}$ или хотя бы для $n\in\mathbb{N}$
		\begin{equation}
			(B(x_1,x_2,...))^n = B(x_1^n, x_2^n, ...)  ~~~ \mbox{?}
		\end{equation}
		Тут явно намечаются проблемы, вызванные инвариантностью относительно сдвига.
		Имею смелость предположить, что если такие банаховы пределы существуют,
		то они инварианты относительно оператора Чезаро
		(но это пока абсолютно необоснованная гипотеза).
\end{enumerate}


\end{document}
