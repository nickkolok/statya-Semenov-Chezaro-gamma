\documentclass[10pt,pdf,hyperref={unicode},aspectratio=169,color={usenames, dvipsnames}]{beamer}\usepackage{amsmath}
\usepackage[utf8]{inputenc}
\usepackage[english,russian]{babel}
\usepackage{amsfonts}
\usepackage{amsfonts,amssymb}
\usepackage{amssymb}
\usepackage{latexsym}
\usepackage{euscript}
\usepackage{enumerate}
\usepackage{graphics}
\usepackage{graphicx}
\usepackage{geometry}
\usepackage{wrapfig}

\usepackage{bbm}
\usepackage{mathrsfs}


\usepackage{varwidth}

\usepackage{amsthm}
\theoremstyle{definition}
\newtheorem{llemma}{Лемма}
\newtheorem{ttheorem}[llemma]{Теорема}
\newtheorem{eexample}[llemma]{Пример}
\newtheorem{property}[llemma]{Свойство}
\newtheorem{remark}[llemma]{Замечание}
\newtheorem{ccorollary}{Следствие}[llemma]
\newtheorem{hhypothesis}{Гипотеза}[llemma]
\newtheorem{ddefinition}{Определение}[llemma]

\righthyphenmin=2

%\usetheme{Antibes}
\usefonttheme{professionalfonts} % using non standard fonts for beamer
\usefonttheme[onlymath]{serif} % default family is serif
%\usepackage{fontspec}
%\setmainfont{Liberation Serif}
\begin{document}


\title{
	Обобщения предела и мультипликативные свойства носителя последовательности
}
\author{Н.Н.~Авдеев}
\institute{ Воронежский Государственный Университет \\
		{Работа выполнена  при поддержке РНФ, грант 19-11-00197.
	}}
\date{Воронеж 2022}

\maketitle

\setbeamertemplate{footline}[frame number]
\setbeamertemplate{navigation symbols}{\large{}}

\begin{frame}
	\frametitle{Обобщения понятия сходимости в $\ell_\infty$}
	%\hspace{1.468em}
	\begin{varwidth}[t]{\linewidth}
		\centering
		Верхний и нижний
		\\
		пределы:
		\\
		$\displaystyle \limsup_{n\to\infty} x_n$
		,~
		$\displaystyle \liminf_{n\to\infty} x_n$
		\\~\\
		\emph{
			не характеризуют
			\\
			распределение элементов
		}
	\end{varwidth}
	\hfill
	\begin{varwidth}[t]{\linewidth}
		\centering
		Сходимость в среднем
		\\
		(по Чезаро):
		$\displaystyle \lim_{n\to\infty} (Cx)_n$
		\\
		$\displaystyle (Cx)_n = \frac1n\sum_{i=1}^n x_i$
		\\~\\
		\emph{слишком универсальна}
	\end{varwidth}
	\hfill
	\begin{varwidth}[t]{\linewidth}
		\centering
		Банаховы пределы $B\in \mathfrak{B}$:
		\\
		т. Хана--Банаха к $\lim: c\to\mathbb R$
		\\
			$B \geqslant 0$
		\\
			$B\mathbbm{1} = 1$
		\\
			$B=BT$
		\\
		\emph{почти сходимость}
	\end{varwidth}
	\\~\\~\\
	\begin{varwidth}[t]{\linewidth}
		\centering
		\emph{Критерий Лоренца:}
		$\displaystyle
			x\in ac_t
			\Leftrightarrow
			\forall(B\in\mathfrak{B})[Bx = t]
			\Leftrightarrow
			\lim_{n\to\infty}  \sum_{k=m+1}^{m+n} \frac{x_k}n = t
		$
		равном. по $m\in\mathbb{N}$.
	\end{varwidth}
	\\~\\
	\vspace{0.8em}
	\begin{varwidth}[t]{\linewidth}
		$\displaystyle ac = \mathop{\cup}\limits_{t\in\mathbb R} ac_t$ "--- пространство почти сходящихся последовательностей
	\end{varwidth}
\end{frame}


\begin{frame}\frametitle{Пространство $\ell_\infty$}
	$\ell_\infty$~--- пространство всех ограниченных последовательностей
	$x=(x_1, x_2, ..., x_n, ...)$
	с нормой
	$$
		\|x\|_{\ell_\infty} = \sup_{k\in\mathbb{N}} |x_k|
	$$
	{Свойства:}

	\begin{itemize}
		\item
			$\ell_\infty$~--- линейное пространство над полем $\mathbb{R}$
		\item
			$\ell_\infty$  несепарабельно
		\item
			$\ell_1 \subset \ell_2 \subset \dots \subset \ell_\infty$
		\item
			$\ell_1^* = \ell_\infty$
		\item
			$\ell_\infty^* \neq \ell_1$
	\end{itemize}
\end{frame}

\begin{frame}\frametitle{Банаховы пределы}
	Банаховым пределом называется функционал $B\in \ell_\infty^*$ такой, что:
	\begin{enumerate}
		\item
			$B \geqslant 0$
		\item
			$B\mathbbm{1} = 1$
		\item
			$B=BT$
	\end{enumerate}
	Здесь $\mathbbm{1}=(1,1,1,1,1,...)$,
	$T$~--- оператор сдвига: $T(x_1, x_2, x_3, ...) = (x_2, x_3, ...)$.

	Простейшие свойства:
	\begin{itemize}
		\item
			$\|B\|_{\ell_\infty^*} = 1$
		\item
			$Bx = \lim_{n\to\infty} x_n$ для любого $x=(x_1, x_2, ...) \in c$,
			\\
			где $c$~--- множество сходящихся последовательностей.

			Таким образом,
			банахов предел~--- естественное обобщение понятия предела сходящейся последовательности
			на все ограниченные последовательности.
	\end{itemize}
\end{frame}

\begin{frame}\frametitle{Свойства множества банаховых пределов $\mathfrak{B}$}
	\begin{enumerate}
		\item
			$\mathfrak{B}$~--- замкнутое выпуклое подмножество $ S_{\ell_\infty^*}$~---
			единичной сферы пространства $\ell_\infty^*$
		\item
			$d(\mathfrak{B}) = 2$
		\item
			$\mathfrak{B}$ слабо *-компактно
	\end{enumerate}
\end{frame}

\begin{frame}\frametitle{Теорема Лоренца}
	Для заданного $r\in\mathbb{R}$ равенство $Bx=r$ выполнено для всех $B\in\mathfrak{B}$
	тогда и только тогда, когда
	\begin{equation*}
		\lim_{n\to\infty} \frac{1}{n} \sum_{k=m+1}^{m+n} x_k = r
	\end{equation*}
	сходится равномерно по всем $m\in\mathbb{N}$.

	Множество всех таких $x \in \ell_\infty$ обозначается $ac$.
\end{frame}

\begin{frame}\frametitle{Теорема Сачестона}
	является уточнением теоремы Лоренца.
	Пусть
	\begin{equation*}
		q(x) = \lim_{n\to\infty} \inf_{m\in\mathbb{N}}  \frac{1}{n} \sum_{k=m+1}^{m+n} x_k,
		~~~~~~~~
		p(x) = \lim_{n\to\infty} \sup_{m\in\mathbb{N}}  \frac{1}{n} \sum_{k=m+1}^{m+n} x_k.
	\end{equation*}
	Тогда для любых $x\in \ell_\infty$ и $B\in\mathfrak{B}$
	\begin{equation}\label{Sucheston}
		q(x) \leqslant Bx \leqslant p(x)
	\end{equation}
	Неравенства (\ref{Sucheston}) точны:
	для данного $x$ для любого $r\in[q(x); p(x)]$ найдётся банахов предел
	$B\in\mathfrak{B}$ такой, что $Bx = r$.
\end{frame}

\begin{frame}\frametitle{Оператор Чезаро}
	Оператор Чезаро $C:\ell_\infty\to \ell_\infty$ определяется равенством
	\begin{equation*}
		(Cx)_n = \frac{1}{n} \cdot \sum_{k=1}^n x_k
	\end{equation*}
	Свойства:
	\begin{itemize}
		\item
			$\|C\|_{\ell_\infty,\ell_\infty} = 1$
		\item
			Оператор $C$ не является обратимым
		\item
			Существуют $B\in\mathfrak{B}$ такие, что $BC=B$.
			Такие $B$ называются инвариантными относительно $C$,
			их множество обозначается $\mathfrak{B}(C)$.
	\end{itemize}
\end{frame}

\begin{frame}
	\frametitle{Что такое почти сходимость?}
	\begin{varwidth}[t]{\linewidth}
		\hspace{3em}
		Пример $x\in ac_0$, но $\displaystyle0=\liminf_{n\to\infty}x_n < \limsup_{n\to\infty}x_n=1$:~~
		$$\displaystyle
			x_k=\begin{cases}
				1, & k=2^j,~j\in\mathbb N
				\\
				0 & \mbox{для прочих~} k
			\end{cases}
		$$
		$x=(0,1,0,1,\;0,0,0,1,\;\;0,0,0,0,\;0,0,0,1,\;\;0,0,0,0,\;0,0,0,0,\;\;0,0,0,0,\;0,0,0,1,\;\;0,0,0,0,\;...)$
	\end{varwidth}
	\\
	\vspace{2em}
	\begin{varwidth}[t]{\linewidth}
		Пример сходящейся по Чезаро к нулю последовательности $x\notin ac$:~~
		$$\displaystyle
			x_k=\begin{cases}
				1, & 2^j-j < k \leq 2^j,~j\in\mathbb N
				\\
				0 & \mbox{для прочих~} k
			\end{cases}
		$$
		$x=(0,1,1,1,\;0,1,1,1,\;\;0,0,0,0,\;1,1,1,1,$\\
		$\phantom{x=(}0,0,0,0,\;0,0,0,0,\;\;0,0,0,1,\;1,1,1,1,$\\
		$\phantom{x=(}0,0,0,0,\;0,0,0,0,\;\;0,0,0,0,\;0,0,0,0,$\\
		$\phantom{x=(}0,0,0,0,\;0,0,0,0,\;\;0,0,1,1,\;1,1,1,1,...)$
	\end{varwidth}
\end{frame}

\begin{frame}
	\frametitle{Критерий принадлежности $ac_0$~\cite{avdeev2019space}}


	\begin{llemma}
		Пусть $\{n_i\}\subset \mathbb{N}$~--- строго возрастающая последовательность,
		$\displaystyle
			x_k = \left\{\begin{array}{ll}
				1, & \mbox{~если~} k = n_i
				\\
				0  & \mbox{~иначе.~}
			\end{array}\right.
		$

		Для того, чтобы $x\in ac_0$,
		необходимо и достаточно, чтобы
		\begin{equation*}\label{lim_M(j)/j}
			\lim_{j \to \infty} \frac{M(j)}{j} = +\infty
			,
			\quad\mbox{~где~\quad}
			M(j) = \liminf_{i\to\infty} (n_{i+j} - n_i)
			.
		\end{equation*}
	\end{llemma}


	(Нелинейный) оператор $\lambda$--срезки $A_\lambda$
	для $x \in\ell_\infty$:~
	$\displaystyle
		(A_\lambda x)_k = \begin{cases}
			1, & \mbox{~если~} x_k \geq \lambda
			\\
			0  & \mbox{~иначе.~}
		\end{cases}
	$

	\begin{ttheorem}
		\label{thm:lambda_prelim}
		Пусть $x\in\ell_\infty$, $x\geq 0$.
		Тогда
		$
			x\in ac_0
		$
		\\
		если
		и только если
		для любого $\lambda > 0$
		\\
		выполнено
		$
			A_\lambda x \in ac_0
		$
		.
	\end{ttheorem}

\end{frame}

\begin{frame}
	\frametitle{Критерий принадлежности $ac_0$~\cite{avdeev2019space}}



	\begin{ttheorem}
		Пусть $x\in\ell_\infty$, $x \geq 0$,
		\begin{equation*}
			0<\lambda < \limsup_{k\to\infty} x_k
			.
		\end{equation*}
		Пусть $\{k: x_k \geq \lambda \} = \{n(\lambda,1),n(\lambda,2),...\}$.
		Обозначим
		\begin{equation*}
			M_{\lambda}(j) = \liminf_{i\to\infty} n(\lambda,i+j) - n(\lambda,i)
			.
		\end{equation*}
		Для того, чтобы $x\in ac_0$, необходимо и достаточно, чтобы
		для любого $\lambda>0$
		\begin{equation*}
			\lim_{j \to \infty} \frac{M_{\lambda}(j)}{j} = \infty
			.
		\end{equation*}
	\end{ttheorem}

\end{frame}

\begin{frame}
	\frametitle{Последовательности из 0 и 1}

	Обсуждаемые обобщения верхнего и нижнего пределов удовлетворяют соотношению
	\begin{equation}
		\label{eq:generalization_of_limits}
		\liminf_{n\to\infty} x_n \leq q(x) \leq \liminf_{n\to\infty}\frac1{n}\sum_{i=1}^n x_i
		\leq
		\limsup_{n\to\infty}\frac1{n}\sum_{i=1}^n x_i
		\leq p(x)
		\leq \limsup_{n\to\infty} x_n
		.
	\end{equation}


	Каждый $x\in \Omega$ можно отождествить с подмножеством множества натуральных чисел
	$\operatorname{supp} x \subset \mathbb{N}$.
	Будем обозначать (см., напр., Hall, Tenenbaum, 1992) через $\mathscr{M}A$ множество всех чисел,
	кратных элементам множества $A\subset\mathbb{N}$, т.е.
	\begin{equation}
		\mathscr{M}A = \{ka: k\in\mathbb{N}, a\in A\}
		,
	\end{equation}
	через $\chi A$ "--- характеристическую функцию множества $A$.

	Так, например,
	\begin{gather}
		\chi \mathscr{M}\!A(\{2\}) = \chi \mathscr{M}\!A(\{2, 4\}) = \chi \mathscr{M}\!A(\{2,4,8,16,...\})
		= (0,1,0,1,0,1,0,1,0,...),
	\\
		\chi \mathscr{M}\!A(\{3\}) = \chi \mathscr{M}\!A(\{3,9,27,...\}) = (0,0,1,\;0,0,1,\;0,0,1,\;0,0,1,\;0,0,1,\;...),
	\\
		\chi \mathscr{M}\!A(\{2,3\}) = \chi \mathscr{M}\!A(\{2,3,6\}) = (0,1,1,1,0,1,\;0,1,1,1,0,1,\;0,1,1,1,0,1,...).
	\end{gather}
\end{frame}

\begin{frame}
	\frametitle{Последовательности из 0 и 1}


	Возникает закономерный вопрос о взаимосвязи структуры множества $A$
	и значений, которые принимают обобщения верхнего и нижнего пределов~\eqref{eq:generalization_of_limits}
	на последовательности $\chi \mathscr{M}\!A$.
	Так, в работах Дэвенпорта и Эрдёша (1936 г., 1951 г.) доказано, что для любого
	$A=\{a_1,a_2,...\}\subset\mathbb{N}$
	выполнено
	\begin{equation}
		\liminf_{n\to\infty}\frac1{n}\sum_{i=1}^n (\chi\mathscr{M}A)_i =
		\lim_{j\to\infty}\lim_{n\to\infty}\frac1{n}\sum_{i=1}^n (\chi\mathscr{M}\{a_1,a_2,...,a_j\})_i
		.
	\end{equation}
	Построено (Besicovitch, 1935) такое множество $A\subset\mathbb{N}$, что
	\begin{equation}
		\liminf_{n\to\infty}\frac1{n}\sum_{i=1}^n (\chi\mathscr{M}A)_i \neq
		\limsup_{n\to\infty}\frac1{n}\sum_{i=1}^n (\chi\mathscr{M}A)_i
		.
	\end{equation}

\end{frame}


\begin{frame}
	\frametitle{Последовательности из 0 и 1 и функционалы Сачестона}

	\begin{llemma}
		Пусть $y = \{y_n\}$ --- строго возрастающая последовательность,
		$\chi y\in\Omega \cap ac_0$.
		Пусть $m \in \mathbb{N}$
		и последовательность $x=\{x_k\}$ определена соотношением
		\begin{equation}
			x_k = \begin{cases}
				1, &\mbox{~если~} k = y_i \cdot m^j \mbox{~для некоторых~} i,j\in\mathbb{N},
				\\
				0  &\mbox{~иначе}
				.
			\end{cases}
		\end{equation}
		Тогда $x\in ac_0$.
	\end{llemma}

	\begin{ccorollary}
		\label{cor:ac0_powers_finite_set_of_numbers}
		Пусть $\{p_1, ..., p_k\} \subset \mathbb{N}$,
		\begin{equation}
			x_k = \begin{cases}
				1, &\mbox{~если~} k = p_1^{j_1}\cdot p_2^{j_2}\cdot ... \cdot p_k^{j_k} \mbox{~для некоторых~} j_1,...,j_k\in\mathbb{N},
				\\
				0  &\mbox{~иначе}.
			\end{cases}
		\end{equation}
		Тогда $x\in ac_0$.
	\end{ccorollary}

\end{frame}


\begin{frame}
	\frametitle{Последовательности из 0 и 1 и функционалы Сачестона}

	\begin{hhypothesis}
		Пусть $y=\{y_n\}$ и $z=\{z_n\}$ --- строго возрастающие последовательности,
		$\chi_y,\chi_z\in\{0;1\}^\mathbb{N} \cap ac_0$.
		Тогда почти сходится к нулю последовательность $x=\{x_k\}$, определённая соотношением
		\begin{equation}
			x_k = \begin{cases}
				1, &\mbox{~если~} k = y_i \cdot z_j \mbox{~для некоторых~} i,j\in\mathbb{N},
				\\
				0  &\mbox{~иначе}
				.
			\end{cases}
		\end{equation}
	\end{hhypothesis}
\end{frame}

\begin{frame}
	\frametitle{Последовательности из 0 и 1 и функционалы Сачестона}

	\begin{llemma}
		Для любого непустого $A\subset \mathbb{N} $ выполнено $\chi\mathscr{M}A \notin ac_0$.
	\end{llemma}

	\begin{ttheorem}
		\label{lem:ac0_primes_infinity_mutually_prime_subset}
		Пусть $A'$ "--- бесконечное подмножество попарно взаимно простых чисел
		(т.е. для любых двух чисел $a_1, a_2 \in A'$ их наибольший общий делитель равен единице).
		Тогда для любого $A \supset A' $ выполнено $p(\chi\mathscr{M}A)=1$.
	\end{ttheorem}
	\begin{ddefinition}
		Будем говорить, что множество $A\subset\mathbb{N}$ обладает $P$-свойством,
		если для любого $n\in\mathbb{N}$ найдётся набор попарно взаимно простых чисел
		\begin{equation}
			\{a_{n,1}, a_{n,2}, ..., a_{n,n}  \} \subset A
			.
		\end{equation}
	\end{ddefinition}
\end{frame}

\begin{frame}
	\frametitle{Последовательности из 0 и 1 и функционалы Сачестона}

	\begin{llemma}
		\label{lem:ac0_primes_infinity_mutually_prime_subset_equiv_to_P_property}
		Пусть множество $A$ обладает $P$-свойством.
		Тогда существует бесконечное подмножество $A'\subset A$ попарно взаимно простых чисел.
	\end{llemma}
	\begin{llemma}
		\label{lem:ac0_primes_q_psi_A_0_causes_P}
		Пусть для множества $A\subset\mathbb{N}\setminus\{1\}$ выполнено $p(\chi\mathscr{M}A)=1$.
		Тогда $A$ обладает $P$-свойством.
	\end{llemma}
	\begin{ttheorem}
		Пусть $A\subset \mathbb{N}\setminus\{1\}$.
		Тогда следующие условия эквивалентны:
		\begin{enumerate}
			\item
				$A$ обладает $P$-свойством
			\item
				В $A$ существует бесконечное подмножество попарно взаимно простых чисел
			\item
				$p(\chi\mathscr{M}A)=1$.
		\end{enumerate}
	\end{ttheorem}
\end{frame}

\begin{frame}
	\frametitle{Последовательности из 0 и 1 и функционалы Сачестона}

	\begin{ttheorem}
		\label{thm:ac0_primes_p_psi_A_prod}
		Пусть $A = \{a_1, a_2, ...\}$ "--- бесконечное множество попарно взаимно простых чисел.
		Тогда
		\begin{equation}
			q(\chi\mathscr{M}A) \geq 1-\prod_{j=1}^\infty \left(1-\frac{1}{a_j}\right)
			.
		\end{equation}
	\end{ttheorem}
	\begin{ccorollary}
		\label{cor:ac0_primes_p_psi_A_prod}
		Пусть $A = \{a_1, a_2, ...\}$ "--- бесконечное множество попарно взаимно простых чисел
		и $a_{n+1}>a_1\cdot...\cdot a_n$.
		Тогда
		\begin{equation}
			q(\chi\mathscr{M}A) = 1-\prod_{j=1}^\infty \left(1-\frac{1}{a_j}\right)
			.
		\end{equation}
	\end{ccorollary}
\end{frame}

\begin{frame}
	\frametitle{Последовательности из 0 и 1 и функционалы Сачестона}
	\begin{ttheorem}
		Пусть $A=\{a_1, a_2, ... \}\subset\mathbb{N}$ "--- бесконечное множество попарно взаимно простых чисел
		и
		\begin{equation}
			\label{eq:prod_causes_q}
			\prod_{j=1}^\infty \left(1-\frac{1}{a_j}\right) = 0
		\end{equation}
		Тогда $\chi\mathscr{M}A\in ac$ и, более того, $\chi\mathscr{M}A\in ac_1$.
	\end{ttheorem}

\end{frame}


\def\rinc{\color[rgb]{0.5,0,0}{(РИНЦ)~}}
\def\vak{\color[rgb]{0.5,0.5,0}{(ВАК)~}}
\def\mzm{\color[rgb]{0,0.5,0}{(WoS)~(Scopus)~}}

\setbeamertemplate{bibliography item}{\insertbiblabel\small}

\begin{frame}
	\frametitle{Основные публикации с участием докладчика}
	\vspace{-1.2em}
	\hfill
	\begin{varwidth}[t]{0.95\linewidth}
	\setlength\itemsep{0pt}
	\begin{thebibliography}{99}
		\addtobeamertemplate{block begin}{\vspace*{-30pt}}{}
		\addtobeamertemplate{block end}{}{\vspace*{-30pt}}
		\setlength{\parskip}{0pt}%
		\setlength{\itemsep}{0pt}
		{}
		\bibitem{avdeev2021subsets}\small
			\emph{Avdeev N.} On Subsets of the Space of Bounded Sequences // Mathematical Notes. — 2021. — т. 109, No 1. — с. 150—154.
			\mzm
		{}
		\vspace{-0.5em}
		\bibitem{avdeev2019banach}\small
			\emph{Avdeev} \emph{N.}, \emph{Semenov} \emph{E.}, \emph{Usachev} \emph{A.} Banach Limits and a Measure on the Set of 0-1-Sequences //
			Mathematical Notes. — 2019. — т. 106, No 5/6. — с. 833—836.
			\mzm
		{}
		\vspace{-0.5em}
		\bibitem{avdeev2019space}\small
			\emph{Avdeev} \emph{N.} On the Space of Almost Convergent Sequences // Mathematical Notes. — 2019. — т. 105, No 3/4. — с. 464—468.
			%— DOI: \href
			%{https://doi.org/10.1134/S0001434619030179} {\nolinkurl {10.1134/S0001434619030179}}. — URL: \url
			%{https://www.scopus.com/record/display.uri?origin=inward&eid=2-s2.0-85065492318}.
			\mzm
		{}
		\vspace{-0.5em}
		\bibitem{avdeed2021AandA}\small
			\emph{Авдеев} \emph{Н. Н.}, \emph{Семёнов} \emph{Е. М.}, \emph{Усачев} \emph{А. С.}
			Банаховы пределы: экстремальные свойства,
			инвариантность и теорема Фубини // Алгебра и анализ. — 2021. — т. 33, No 4. — с. 32—48.
			\vak
		{}
		\vspace{-0.5em}
		\bibitem{our-ped-2018-alpha-Tx}\footnotesize
			\emph{Авдеев} \emph{Н. Н.} О суперпозиции оператора сдвига и одной функции на пространстве ограниченных последовательностей //
			Некоторые вопросы анализа, алгебры, геометрии и математического образования. — 2018. — с. 20—21.
			\rinc
		{}
		\vspace{-0.6em}
		\bibitem{our-vzms-2018}\footnotesize
			\emph{Авдеев} \emph{Н. Н.}, \emph{Семенов} \emph{Е. М.}
			Об асимптотических свойствах оператора Чезаро \\// Материалы Воронежской Зимней
			Математической\\школы С.Г. Крейна – 2018.  Под ред.\\В.А. Костина. — 2018. — с. 107—109.
			\rinc
	\end{thebibliography}
	\end{varwidth}
	\vspace{10em}
\end{frame}

\setbeamertemplate{navigation symbols}{}

\begin{frame}
	{
		\huge\centering
		~\\~\\~\\~\\
		Спасибо за внимание
	}
	~\\
	\vspace{6.28em}
	nickkolok@mail.ru, avdeev@math.vsu.ru
	\\
	github.com/nickkolok
	\\
	arxiv.org/a/avdeev\_n\_1.html
\end{frame}

\end{document}
