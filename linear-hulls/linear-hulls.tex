\documentclass[a4paper,14pt]{article} %размер бумаги устанавливаем А4, шрифт 12пунктов
\usepackage[T2A]{fontenc}
\usepackage[utf8]{inputenc}
\usepackage[croatian,german,french,english,russian]{babel}
\usepackage{amssymb,amsfonts,amsmath,mathtext,enumerate,float,amsthm} %подключаем нужные пакеты расширений
\usepackage[unicode,colorlinks=true,citecolor=black,linkcolor=black]{hyperref}
%\usepackage[pdftex,unicode,colorlinks=true,linkcolor=blue]{hyperref}
\usepackage{indentfirst} % включить отступ у первого абзаца
\usepackage[dvips]{graphicx} %хотим вставлять рисунки?
\graphicspath{{illustr/}}%путь к рисункам

\makeatletter
\renewcommand{\@biblabel}[1]{#1.} % Заменяем библиографию с квадратных скобок на точку:
\makeatother %Смысл этих трёх строчек мне непонятен, но поверим "Запискам дебианщика"

\usepackage{geometry} % Меняем поля страницы.
\geometry{left=2cm}% левое поле
\geometry{right=1cm}% правое поле
\geometry{top=2cm}% верхнее поле
\geometry{bottom=2cm}% нижнее поле

\renewcommand{\theenumi}{\arabic{enumi}.}% Меняем везде перечисления на цифра.цифра
\renewcommand{\labelenumi}{\arabic{enumi}.}% Меняем везде перечисления на цифра.цифра
\renewcommand{\theenumii}{\arabic{enumii}}% Меняем везде перечисления на цифра.цифра
\renewcommand{\labelenumii}{\arabic{enumi}.\arabic{enumii}.}% Меняем везде перечисления на цифра.цифра
\renewcommand{\theenumiii}{\arabic{enumiii}}% Меняем везде перечисления на цифра.цифра
\renewcommand{\labelenumiii}{\arabic{enumi}.\arabic{enumii}.\arabic{enumiii}.}% Меняем везде перечисления на цифра.цифра

\sloppy




\renewcommand\normalsize{\fontsize{14}{25.2pt}\selectfont}

\usepackage[backend=biber,language=autobib,autolang=other,clearlang=true,style=gost-numeric,sorting=none]{biblatex}
\addbibresource{../bib/general_monographies.bib}
\addbibresource{../bib/ext.bib}
\addbibresource{../bib/my.bib}
\addbibresource{../bib/Semenov.bib}
\addbibresource{../bib/Bibliography_from_Usachev.bib}
\addbibresource{../bib/classic.bib}

\input{../bib/ext.hyphens.bib}


\theoremstyle{plain}
\newtheorem{lemma}{Лемма}[section]
\newtheorem{theorem}[lemma]{Теорема}
\newtheorem{example}[lemma]{Пример}
\newtheorem{property}[lemma]{Свойство}
\newtheorem{remark}[lemma]{Замечание}
\newtheorem{corollary}{Следствие}[lemma]
\newtheorem{definition}[lemma]{Определение}
\newtheorem{proposition}[lemma]{Утверждение}
\newtheorem{hypothesis}[lemma]{Гипотеза}

\usepackage{bbm}


%Only referenced equations are numbered
\let\etoolboxforlistloop\forlistloop % save the good meaning of \forlistloop
\usepackage{mathtools}
\mathtoolsset{showonlyrefs}
\let\forlistloop\etoolboxforlistloop % restore the good meaning of \forlistloop

% https://tex.stackexchange.com/questions/220268/biblatex-and-autonum-dont-work-together

%\mathtoolsset{showonlyrefs=false}
% (write an equation/multline to be force-numbered here)
%\mathtoolsset{showonlyrefs=true}




\begin{document}

%\renewcommand{\bibname}{Список цитированной литературы}
%\renewcommand\refname{\bibname}
% !!!
% The text starts here

\title{
	О разделяющих множествах меры нуль и функционалах Сачестона
	\footnote{
		Работа выполнена в Воронежском университете при поддержке РНФ, грант 19-11-00197.
	}
}

\author{
	Н.Н. Авдеев
	\footnote{nickkolok@mail.ru, avdeev@math.vsu.ru}
}

\maketitle

УДК 517.982.276 % Пространства последовательностей и матриц

\paragraph{Аннотация}
Известно, что существуют нетривиальные подмножества ограниченных последовательностей,
являющиеся разделяющими для банаховых пределов
(т.е. любые два различных банахова предела принимают различные значения
хотя бы на одном элементе разделяющего множества).
В статье приводится пример множества последовательностей из нулей и единиц,
имеющего меру нуль и являющегося разделяющим для банаховых пределов.
При построении такого множества используется тот факт, что множество последовательностей
из нулей и единиц содержится в линейной оболочке своего подмножества,
определямого значениями функционалов Сачестона.
Далее доказывается, что аналогичным свойством обладает всё пространство ограниченных последовательностей
и его подпространство $A_0$, определяемое асимпотическими свойствами.
Для подпространства $A_0$ иследуются свойства инвариантности относительно классических линейных операторов.

\paragraph{Ключевые слова}
	пространство ограниченных последовательностей,
	банаховы пределы,
	разделяющее множество,
	функционалы Сачестона,
	мера множества


On separating sets of measure zero and Sucheston functionals

\paragraph{Abstract}
It is well known that the space of bounded sequences contains non-trivial subsets
that separate Banach limits,
that is
for every pair of distinct Banach limits a separating set contains an element
on which the values of the Banach limits do not equal.)
In the present paper we provide an example of a set of 0-1-sequences that has measure zero
but separates Banach limits.
In order to construct the separating set, we prove that
the set of all 0-1-sequences belongs to the linear hull of its subset
defined by fixed values of Sucheston functionals.
Then we prove that the property holds for the whole space of the bounded sequences
and a subspace $A_0$ defined by a special function.
We prove that the space $A_0$ is invariant under some classical linear operators.




\paragraph{Keywords}
	space of bounded sequences,
	Banach limits,
	separating set,
	Sucheston functional,
	measure of a set



\section{Введение}

Рассмотрим пространство ограниченных последовательностей $\ell_\infty$ с обычной нормой
\begin{equation*}
	\|x\| = \sup_{k\in\mathbb{N}} |x_k|
	.
\end{equation*}
и обычной полуупорядоченностью, где $\mathbb{N}$ "--- множество натуральных чисел.
Через $c$ будем обозначать пространство сходящихся последовательностей.


\begin{definition}
	Линейный функционал $B\in l_\infty^*$ называется банаховым пределом,
	если
	\begin{enumerate}
		\item
			$B\geq0$, т.~е. $Bx \geq 0$ для $x \geq 0$,
		\item
			$B\mathbbm{1}=1$, где $\mathbbm{1} =(1,1,\ldots)$,
		\item
			$B(Tx)=B(x)$ для всех $x\in l_\infty$, где $T$~---
		оператор сдвига, т.~е. $T(x_1,x_2,\ldots)=(x_2,x_3,\ldots)$.
	\end{enumerate}
\end{definition}
Множество всех банаховых пределов обозначим через $\mathfrak{B}$.
Существование банаховых пределов было анонсировано С. Мазуром \cite{Mazur} и позднее доказано в книге С.~Банаха~\cite{B}.


Сачестон~\cite{sucheston1967banach} установил, что
для любых $x\in l_\infty$ и $B\in\mathfrak{B}$
\begin{equation}\label{Sucheston}
	q(x) \leqslant Bx \leqslant p(x)
	,
\end{equation}
где
\begin{equation*}
	q(x) = \lim_{n\to\infty} \inf_{m\in\mathbb{N}}  \frac{1}{n} \sum_{k=m+1}^{m+n} x_k
	~~~~\mbox{и}~~~~
	p(x) = \lim_{n\to\infty} \sup_{m\in\mathbb{N}}  \frac{1}{n} \sum_{k=m+1}^{m+n} x_k
	.
\end{equation*}
называют нижним и верхним функционалом Сачестона соотвественно.
Заметим, что $p(x) = -q(-x)$.
Неравенства \eqref{Sucheston} точны:
для данного $x$ для любого $r\in[q(x); p(x)]$ найдётся банахов предел
$B\in\mathfrak{B}$ такой, что $Bx = r$.

Множество таких $x\in\ell_\infty$, что $p(x)=q(x)$,
образует подпространство почти сходящихся последовательностей $ac$~~\cite{lorentz1948contribution}.
На почти сходящейся последовательности все банаховы пределы принимают одинаковые значения.

При исследовании банаховых пределов особый интерес представляют разделяющие множества~\cite[\S 3]{Semenov2014geomprops}.
Множество $Q\in\ell_\infty$ называют разделяющим, если
для любых неравных $B_1, B_2\in\mathfrak{B}$ существует такая последовательность $x\in Q$,
что $B_1 x \neq B_2 x$.
В частности, разделяющим является~\cite{semenov2010characteristic} множество всех последовательностей из 0 и 1,
которое в дальнейшем мы будем обозначать через $\Omega$
(иногда в литературе встречается также обозначение $\{0;1\}^\mathbb{N}$).

Каждой последовательности $(x_1, x_2, \dots)\in \Omega$ можно поставить в соответствие число
\begin{equation}\label{eq:bijection_omega_0_1}
	\sum_{k=1}^\infty 2^{-k} x_k \in [0,1]
	.
\end{equation}
С точностью до счётного множества это соответствие взаимно однозначно и определяет на множестве $\Omega$ меру,
которую мы будем отождествлять с мерой Лебега на $[0,1]$.

Оказывается, что из $\Omega$ можно выделить некоторые подмножества, которые также будут разделяющими,
например \cite[\S 3, Теорема 11]{Semenov2014geomprops},
\begin{equation}
	U = \{ x\in\Omega: q(x) = 0, p(x) = 1 \}
	.
\end{equation}

Однако множество $U$ имеет меру 1~\cite{semenov2010characteristic}.

В настоящей статье строится пример разделяющего множества,
являющегося подмножеством $\Omega$ и имеющего меру нуль.
Для построения такого множества используется следующий факт.

\begin{lemma}[{\cite[\S 3, замечание 6]{Semenov2014geomprops}}]
	Пусть $X$~--- разделяющее множество и $X \subset \operatorname{Lin} Y$,
	где $\operatorname{Lin} Y$ обозначает линейную оболочку $Y$.
	Тогда $Y$ также является разделяющим множеством.
\end{lemma}

Затем в статье обсуждаются свойства линейных оболочек множеств, определённых с помощью функционалов Сачестона.
В частности, доказывается,
что наряду с использованным при построении разделяющего множества меры нуль включением
\begin{equation}
	\Omega \subset \operatorname{Lin}\{x\in\Omega : p(x) = a,~ q(x) = b\}
\end{equation}
для любых $0\leq b < a \leq 1$,
имеет место равенство
\begin{equation}
	\ell_\infty = \operatorname{Lin}\{x\in\ell_\infty : p(x) = a,~ q(x) = b\}
\end{equation}
для любых $a>b$.

Возникает закономерный вопрос: для каких ещё подмножеств пространства $\ell_\infty$
верны аналогичные соотношения?

Оказывается, что аналогичным свойством обладает и ещё одно подмножество пространства $\ell_\infty$: подпространство
$A_0 = \{ x \in \ell_\infty : \alpha(x) =0 \}$,
где~\cite{our-vzms-2018}
\begin{equation*}
	\alpha(x) = \varlimsup_{i\to\infty} \max_{i<j\leqslant 2i} |x_i - x_j|
	.
\end{equation*}

Пространство $A_0c$ обладает рядом интересных свойств.


\begin{theorem}[{\cite[следствие 2]{our-mz2019ac0}}]
	\label{thm:alpha_c_ac_c}
	Пусть $x\in ac$, т.е. $p(x) = q(x)$.
	Тогда $x\in c$ если и только если $\alpha(x) = 0$.
\end{theorem}
Таким образом, $c = ac \cap A_0$.
Включение $c\subset A_0$ собственно.

В~\cite{our-ped-2018-alpha-Tx} показано, что, хотя сама функция $\alpha(x)$ не инвариантна относительно оператора сдвига $T$,
подпространство $A_0$ такой инвариантностью обладает;
из доказанного в~\cite{SSUZ2} соотношения
\begin{equation}
	\alpha(Cx) \leq \alpha(x)
	,
\end{equation}
где $Cx$ есть оператор Чезаро
\begin{equation}
	(Cx)_n = \frac{1}{n} \sum_{k=1}^n x_k
	,
\end{equation}
следует инварианнтность пространства $A_0$ относительно оператора Чезаро.

В настоящей статье доказывается,
что пространство $A_0$ инвариантно относительно операторов растяжения $\sigma_n$
и усредняющего сжатия $\sigma_{1/n}$.

%, а также покоординатного умножения (суперпозиции).

Некоторые результаты данной статьи были анонсированы в~\cite{our-mz2021linearhulls}.

\section{Вспомогательные построения}

В данном параграфе вводятся некоторые вспомогательные объекты,
которые потребуются далее при доказательстве теоремы~\ref{thm:Lin_Omega_Sucheston}.

\subsection{Двоичные приближения}

\begin{definition}
	$k$-м двоичным приближением к произвольному числу $d\in[0;1]$
	называется такое число $d_{(k)}\in\mathbb{N}\cup\{0\}$,
	что
	\begin{equation}
		\label{eq:binary_approximations_for_number}
		\frac{d_{(k)}}{2^k} < d \leq \frac{d_{(k)}+1}{2^k}
		.
	\end{equation}
\end{definition}

\begin{remark}
	Очевидно, что $d_{(k+1)}\in\{2d_{(k)},2d_{(k)}+1\}$.
\end{remark}

\subsection{Последовательности-<<блоки>>}

Введём последовательности-<<блоки>> "---
стабилизирующиеся на нуле последовательности,
которые затем будут использованы для формирования последовательностей,
обладающих некоторыми интересными свойствами.

Пусть $n$ зафиксировано.
Пусть
\begin{equation}
	K = \{k\in\mathbb{N} : k \geq n\} = \{n, n+1, n+2, ...\}
	.
\end{equation}


Определим функцию $\operatorname{Br}:K\times [0;1] \to \ell_\infty$,
генерирующую <<блоки>> из нулей и единиц,
соответствующие приближению $d_{(k)}$ к числу $d\in[0;1]$ для $k \geq n$.

Определение $\operatorname{Br}$ построим рекурсивно.
Сначала определим $\operatorname{Br}(k,d)$ для $k=n$ по следующему правилу:
\begin{equation}
	(\operatorname{Br}(n,d))_j = \begin{cases}
		1, & \mbox{~если~} 2^n - d_{(n)} < j \leq 2^n,
		\\
		0  & \mbox{~для остальных~} j
		.
	\end{cases}
\end{equation}
Заметим, что все элементы $\operatorname{Br}(n,d)$, начиная с $(2^n+1)$-го, равны нулю;
кроме того, в $\operatorname{Br}(n,d)$ ровно $d_{(n)}$ единиц.

Для каждого $k \geq n$ положим
\begin{equation}
	\label{eq:Br(k+1,d)}
	\operatorname{Br}(k+1,d) = \operatorname{Br}(k,d) + T^{2^k}\operatorname{Br}(k,d) + (d_{(k+1)}-2d_{(k)})e_{2^k+2^n-d_{(n)}}
	,
\end{equation}
где через $e_j$ обозначен $j$-й орт.


\begin{proposition}
	\label{prop:Br_k_c_0_1}
	Последовательность $\operatorname{Br}(k,d)$ состоит из нулей и единиц.
\end{proposition}
\begin{proof}
	Легко доказать по индукции, что все элементы $\operatorname{Br}(k,d)$, начиная с $(2^k+1)$-го, равны нулю.
	Следовательно, носители первых двух слагаемых в~\eqref{eq:Br(k+1,d)} не пересекаются.
	Далее заметим, что третье слагаемое отлично от нуля тогда и только тогда,
	когда переход между приближениями $d_{(k)} / 2^k$ и $d_{(k+1)}/2^{k+1}$
	приводит к улучшению приближения.
	Более того,
	\begin{multline}
		\left(\operatorname{Br}(k,d) + T^{2^k}\operatorname{Br}(k,d)\right)_{2^k+2^n-d_{(n)}}
		=
		(\operatorname{Br}(k,d))_{2^n-d_{(n)}}
		=
		(\operatorname{Br}(k-1,d))_{2^n-d_{(n)}}
		=
		\\=
		...
		=
		(\operatorname{Br}(n,d))_{2^n-d_{(n)}}
		=
		0
		,
	\end{multline}
	т.е. выражение~\eqref{eq:Br(k+1,d)} действительно задаёт последовательность из нулей и единиц.
\end{proof}

\begin{remark}
	Из доказательства утверждения~\ref{prop:Br_k_c_0_1} следует, что в $k$-м блоке ровно $d_{(k)}$ единиц.
\end{remark}


\begin{remark}
	Выполнено включение $\operatorname{supp}\operatorname{Br}(k,d) \subset \operatorname{supp}\operatorname{Br}(k+1,d)$
	и, более того, справедливо соотношение
	\begin{equation}
		(\operatorname{Br}(k,d))_j = \begin{cases}
			(\operatorname{Br}(k+1,d))_j, & \mbox{~если~}  j \leq 2^k,
			\\
			0  & \mbox{~для остальных~} j
			.
		\end{cases}
	\end{equation}
\end{remark}

\begin{lemma}
	\label{lem:sum_Br_k_c}
	Для любых таких $m$, $k$ и $i$, что $n \leq m \leq k$ и  $ i + 2^m - 1 \leq 2^k$,
	выполнено
	\begin{equation}
		d_{(m)} \leq \sum_{j=i}^{i+2^m-1} (\operatorname{Br}(k,d))_j \leq d_{(m)}+1
		.
	\end{equation}
\end{lemma}
\begin{proof}
	Представление~\eqref{eq:Br(k+1,d)} может быть переписано в виде:
	\begin{equation}
		\operatorname{Br}(m+1,d) = \sum_{j=0}^{1} T^{j2^m} \operatorname{Br}(m,d) + \sum_{j=0}^{1} \gamma_{j} e_{j 2^m+2^n-d_{(n)}}
		,
	\end{equation}
	где $\gamma_{j} \in \{0,1\}$.
	Продолжая по индукции, получаем
	\begin{equation}
		\operatorname{Br}(k,d) = \sum_{j=0}^{2^{k-m}-1} T^{j2^m} \operatorname{Br}(m,d) +
		\sum_{j=0}^{2^{k-m}-1} T^{j2^m} \gamma_{j} e_{j2^m+2^n-d_{(n)}}
		.
	\end{equation}
	Тогда
	\begin{equation}
		\sum_{j=i}^{i+2^m-1} (\operatorname{Br}(k,d))_j
		=
		\sum_{j=1}^{2^m-1} (\operatorname{Br}(m,d))_j
		+ \gamma_h
		=
		d_{(m)} + \gamma_h
		\in \{d_{(m)}, d_{(m)}+1\}
		.
	\end{equation}
\end{proof}

Из~\eqref{eq:binary_approximations_for_number} непосредственно вытекает следующий факт.
\begin{proposition}
	\label{prop:Br_has_nulls}
	Пусть $d<1-3/2^n$.
	Тогда
	\begin{equation}
		(\operatorname{Br}(d,k))_j = 0 ~~\mbox{для}~~j = m\cdot 2^n + 1, m\in\mathbb{N} \cup\{0\}
		.
	\end{equation}
\end{proposition}


\begin{example}
	Для $n=2$ и $d=1/3$ имеем:
	\begin{equation*}
		\begin{array}{lll}
			d_{(2)} = 1/4, &
			\operatorname{Br}(2,1/3) = (&0,0,0,1, \ 0,0,...)
			\\
			d_{(3)} = 2/8, &
			\operatorname{Br}(3,1/3) = (&0,0,0,1, \ 0,0,0,1, \ \   0,0,...)
			\\
			d_{(4)} = 5/16, &
			\operatorname{Br}(4,1/3) = (&0,0,0,1, \ 0,0,0,1, \ \   0,0,1,1, \ 0,0,0,1, \ \ \  0,0,...)
			\\
			d_{(5)} = 10/32, &
			\operatorname{Br}(5,1/3) = (&
			                             0,0,0,1, \ 0,0,0,1, \ \   0,0,1,1, \ 0,0,0,1,\\&
			                           & 0,0,0,1, \ 0,0,0,1, \ \   0,0,1,1, \ 0,0,0,1,
			\ \ \ 0,0,...)
			\\
			d_{(6)} = 21/64, &
			\operatorname{Br}(6,1/3) = (&
			                             0,0,0,1, \ 0,0,0,1, \ \   0,0,1,1, \ 0,0,0,1,\\&
			                           & 0,0,0,1, \ 0,0,0,1, \ \   0,0,1,1, \ 0,0,0,1,\\&
			                           & 0,0,1,1, \ 0,0,0,1, \ \   0,0,1,1, \ 0,0,0,1,\\&
			                           & 0,0,0,1, \ 0,0,0,1, \ \   0,0,1,1, \ 0,0,0,1,
			\ \ \ 0,0,...)
		\end{array}
	\end{equation*}
\end{example}

\subsection{Частичный предел в функционале Сачестона}

\begin{proposition}
	\label{prop:Sucheston_partial_limit}
	Предел в функционале Сачестона можно заменить частичным пределом, а именно
	\begin{equation*}
		p(x) = \lim_{n\to\infty} \sup_{m\in\mathbb{N}}  \frac{1}{n} \sum_{k=m+1}^{m+n} x_k
		= \lim_{n\to\infty} \sup_{m\in\mathbb{N}}  \frac{1}{2^n} \sum_{k=m+1}^{m+2^n} x_k
		.
	\end{equation*}
\end{proposition}
Аналогичное соотношение выполнено и для функционала $q(x)$.

\section{Разделяющее множество нулевой меры}

\begin{theorem}
	\label{thm:Lin_Omega_Sucheston}
	Пусть
	$1 \geq a > b \geq 0$ и
	$\Omega^a_b = \{x\in\Omega : p(x) = a, q(x) = b\}$,
	где $p(x)$ и $q(x)$~--- верхний и нижний функционалы Сачестона~\cite{sucheston1967banach} соответственно.
	Тогда $\Omega \subset \operatorname{Lin} \Omega^a_b$.
\end{theorem}

\begin{proof}
	Выберем $n\in\mathbb{N}$ таким образом, что
	\begin{equation}
		\label{eq:Omega_a_b_gap}
		a - b > \frac{3}{2^n}
	\end{equation}
	и $n$ чётно.

	Очевидно, что существует разложение
	\begin{equation}
		x = \sum_{i=0}^{k-1} T^i x_i, \quad x_i \in \Omega
		,
	\end{equation}
	где $k\in\mathbb{N}$ и все элементы последовательностей $x_i$,
	кроме имеющих индексы $km+1$, $m\in\mathbb{N}_0$, являются нулевыми.
	Пусть $k=2^n$; зафиксируем $i$ и в дальнейшем для удобства записи положим $w=x_i$.
	Наша задача~--- построить конечную линейную комбинацию элементов из $\Omega^a_b$, равную $w$.
	%
	Положим
	%\begin{equation}
	%	v = T^{2^n} \operatorname{Br}(2^ n   ,a) + T^{2^{n+1}} \operatorname{Br}(2^{n+1},b) +
	%	T^{2^{n+2}} \operatorname{Br}(2^{n+2},a) + T^{2^{n+3}} \operatorname{Br}(2^{n+3},b) + ...
	%\end{equation}
	%Эта сумма является формальной, т.е. не сходится в смысле ряда по норме,
	%однако носители слагаемых попарно не пересекаются.
	%Иначе говоря,
	\begin{equation}
		v_j = \begin{cases}
			0,  & \mbox{~если~} j \leq 2^n,
			\\
			(\operatorname{Br}(2^{2k  },a))_{j-2^{2k  }},  & \mbox{~если~} 2^{2k  } < j \leq 2^{2k+1}, 2k   \geq n,
			\\
			(\operatorname{Br}(2^{2k+1},b))_{j-2^{2k+1}},  & \mbox{~если~} 2^{2k+1} < j \leq 2^{2k+2}, 2k+1 \geq n
			.
		\end{cases}
	\end{equation}
	Иначе говоря, сначала <<резервируется>> $2^n$ нулевых элементов
	(большей частью для удобства записи, поскольку конечное количество членов в начале последовательности
	не влияет на функционалы Сачестона),
	а затем по очереди приписываются блоки "--- от первого элемента (нулевого) до последнего ненулевого элемента
	(конца носителя).
	Положим далее
	\begin{equation}
		u_j = \begin{cases}
			v_j + w_j,  & \mbox{~если~} j \leq 2^n
			\\
			            & \mbox{~или~} 2^{4k+3} < j \leq 2^{4k+4} \mbox{~и~} 4k + 3 \geq n,
			\\
			v_j         & \mbox{~для остальных~} j
			.
		\end{cases}
	\end{equation}

	В силу утверждения~\ref{prop:Br_has_nulls} все элементы, к которым прибавляются ненулевые элементы $w_j$, равны нулю.
	Кроме того, с учётом леммы~\ref{lem:sum_Br_k_c} и утверждения~\ref{prop:Sucheston_partial_limit}
	имеем $p(u)=p(w)=a$ и $q(u)=q(w)=b$.
	(На <<возмущённом>> блоке $u$ среднее, соответствующее функционалу $q$,
	увеличивается не более чем на $2^{-n}$ и не влияет на значение функционала $p$,
	в силу условия~\eqref{eq:Omega_a_b_gap}.)
	Следовательно, $u,v\in\Omega^a_b$.
	Заметим теперь, что
	\begin{equation}
		(u-v)_j = \begin{cases}
			w_j,  & \mbox{~если~} j \leq 2^n,
			\\
			0,  & \mbox{~если~} 2^{2k  } < j \leq 2^{2k+1}, 2k    \geq n,
			\\
			0,  & \mbox{~если~} 2^{4k+1} < j \leq 2^{4k+2}, 4k + 1 \geq n,
			\\
			w_j,  & \mbox{~если~} 2^{4k+3} < j \leq 2^{4k+4}, 4k + 3 \geq n
			.
		\end{cases}
	\end{equation}

	Аналогично строятся пары элементов, разность которых равна $w_j$ на $2^{4k+i} < j \leq 2^{4k+i+1}, 4k + i \geq n$ для $i=0,1,2$
	(требуется только обнулить первые $2^n$ элементов).
	Складывая полученные таким образом $4\cdot 2^n$ разностей элементов из $\Omega^a_b$, получаем требуемый элемент $x$.

\end{proof}

\begin{corollary}
	Множество $\Omega^a_b$ является разделяющим.
	Т.к. при $a\neq 1$ или $b\neq 0$ множество $\Omega^a_b$ имеет меру нуль~\cite{semenov2010characteristic,connor1990almost},
	то оно является разделяющим множеством нулевой меры.
\end{corollary}



\section{Линейные оболочки множеств, определяемых функционалами Сачестона}

Итак, $\Omega \subset \operatorname{Lin}\{x\in\Omega : p(x) = a,~ q(x) = b\}$, где $1\geq a>b\geq 0$.


Пусть $Y^a_b = \{x\in A_0 : p(x) = a, q(x) = b\}$, где $a>b$.
Подготовим сначала вспомогательные леммы о константе.

\begin{lemma}
	\label{lem:const_Lin_alpha_0}
	Пусть $a\neq -b$.
	Тогда справедливо включение
	$\mathbbm{1}\in \operatorname{Lin} Y^a_b$.
\end{lemma}

\begin{proof}
	Не теряя общности, будем полагать, что $a>0$.

	Определим оператор $S:\ell_\infty \to \ell_\infty$ следующим образом:
	\begin{equation}\label{operator_S}
		(Sy)_k = y_{i+2}, \mbox{ где } 2^i < k \leq 2^i+1
		.
	\end{equation}
	\begin{lemma}[{\cite{our-vzms-2018}}]
		Для любого $x\in \ell_\infty$ выполнено равенство
		\begin{equation}\label{alpha_S}
			\alpha(Sx) = \varlimsup_{k\to\infty} |x_{k+1} - x_{k}|
			.
		\end{equation}
	\end{lemma}
	Положим
	\begin{equation}
		\label{eq:y_for_s_alpha}
		y = \left(0,1,0,\frac{1}{2},1,\frac{1}{2},0,\frac{1}{3},\frac{2}{3},1,\frac{2}{3},\frac{1}{3},0,...\right)
		,
	\end{equation}
	тогда $Sy\in A_0$ и $\mathbbm{1}-Sy = S(\mathbbm{1}-y)\in A_0$.

	Пусть $x=(a-b)Sy+b\mathbbm{1}$, $z=(a-b)(\mathbbm{1}-Sy)+b\mathbbm{1}$.
	Тогда $p(x)=p(z)=a$, $q(x)=q(z)=b$ и, следовательно, $x,z\in Y^a_b$.
	Кроме того, заметим, что
	\begin{equation}
		x+z = (a-b)Sy+b\mathbbm{1} + (a-b)(\mathbbm{1}-Sy)+b\mathbbm{1}
		=
		(a-b)(Sy-Sy+\mathbbm{1}) + 2 b\mathbbm{1} = (a+b)\mathbbm{1}
		,
	\end{equation}
	откуда и следует, что $\mathbbm{1}\in Y^a_b$.
\end{proof}


\begin{lemma}
	\label{lem:const_Lin_alpha_0_a_eq_-b}
	Справедливо включение
	$\mathbbm{1}\in \operatorname{Lin} Y^a_{-a}$.
\end{lemma}

\begin{proof}
	Определим линейный оператор $M:\ell_\infty \to \ell_\infty$ следующим образом:
	\begin{multline}
		M\omega=\left(
			0, 1\omega_1,
			0, \frac{1}{2}\omega_2, 1\omega_2, \frac{1}{2}\omega_2,
			0, \frac{1}{3}\omega_3, \frac{2}{3}\omega_3, 1\omega_3, \frac{2}{3}\omega_3, \frac{1}{3}\omega_3,
			0, ...,
		\right. \\ \left.
			0, \frac{1}{p}\omega_p, \frac{2}{p}\omega_p, ..., \frac{p-1}{p}\omega_p, 1\omega_p,
				\frac{p-1}{p}\omega_p, ..., \frac{2}{p}\omega_p, \frac{1}{p}\omega_p,
			0, \frac{1}{p+1}\omega_{p+1}, ...
		\right)
		.
	\end{multline}
	Тогда $SM: \ell_\infty \to A_0$.

	Положим $x=aS(2M(\mathbbm{1})-\mathbbm{1})$,
	$y=-aS(2M(1,0,1,0,1,0,1,0,...)-\mathbbm{1})$,
	$z=-aS(2M(0,1,0,1,0,1,0,1,...)-\mathbbm{1})$.

	Тогда, очевидно, каждая из последовательностей $x,y,z$ содержит отрезки сколь угодно большой длины,
	состоящие из $a$ (равно как и из $-a$), при этом $-a \leq x,y,z \leq a$.
	Следовательно, $p(x)=p(y)=p(z) = a$ и $q(x)=q(y)=q(z) = -a$,
	откуда $x,y,z \in Y^a_{-a}$.

	Заметим теперь, что
	\begin{multline}
		x + y + z
		=
		\\=
		aS(2M(\mathbbm{1})-\mathbbm{1}) - aS(2M(1,0,1,0,1,0,...)-\mathbbm{1}) - aS(2M(0,1,0,1,0,1,...)-\mathbbm{1})
		=
		\\=
		aS(2M(\mathbbm{1})-\mathbbm{1}  -    2M(1,0,1,0,1,0,...)+\mathbbm{1}  -    2M(0,1,0,1,0,1,...)+\mathbbm{1})
		=
		\\=
		aS(2M(\mathbbm{1}) - 2M(1,0,1,0,1,0,...) - 2M(0,1,0,1,0,1,...)+\mathbbm{1}+\mathbbm{1}-\mathbbm{1})
		=
		\\=
		aS(2M(\mathbbm{1}) - 2M(1,0,1,0,1,0,...) - 2M(0,1,0,1,0,1,...)+\mathbbm{1})
		=
		aS\mathbbm{1}
		=
		a\mathbbm{1}
		,
	\end{multline}
	откуда $\mathbbm{1} \in \operatorname{Lin} Y^a_{-a}$.
\end{proof}


\begin{lemma}
	\label{lem:c_0_Lin_alpha_0}
	Справедливо включение $c_0 \subset Y^a_b$.
\end{lemma}

\begin{proof}
	Зафиксируем $z\in c_0$.
	Выберем произвольный $x \in Y^a_b$.
	Тогда $x+z\in Y^a_b$ и, очевидно, $z=(x+z)-x$.
\end{proof}

\begin{theorem}
	\label{thm:A_0_c_infty_lin}
	Пусть $a\neq -b$.
	Тогда справедливо равенство $\operatorname{Lin} Y^a_b = A_0$.
\end{theorem}

\begin{proof}
	Зафиксируем $x \in A_0$.

	Пусть сначала $p(x) = q(x)$.
	Тогда, согласно теореме~\ref{thm:alpha_c_ac_c}, $x\in c$
	и $x$ может быть представлен в виде суммы константы и последовательности из $c_0$.
	Утверждение теоремы следует из лемм~\ref{lem:const_Lin_alpha_0}, ~\ref{lem:const_Lin_alpha_0_a_eq_-b} и~\ref{lem:c_0_Lin_alpha_0}.

	Пусть теперь $p(x) > q(x)$.
	Положим $y=k\cdot x + C\cdot\mathbbm{1}$,
	где $k=\frac{a-b}{p(x)-q(x)}$, $C=\frac{bp(x)-aq(x)}{p(x)-q(x)}$.
	Тогда, очевидно,
	\begin{equation}
		\label{eq:x_representation}
		x=(y-C\cdot\mathbbm{1})/k
		.
	\end{equation}
	Представление~\eqref{eq:x_representation} искомое.
	Действительно, в силу лемм~\ref{lem:const_Lin_alpha_0}~и~\ref{lem:const_Lin_alpha_0_a_eq_-b} выполнено
	$C\cdot\mathbbm{1}\in Y^a_b$; кроме того,
	\begin{equation}
		p(y) = k\cdot p(x) + C
		=
		%\\=
		\frac{ap(x)-bp(x)+bp(x)-aq(x)}{p(x)-q(x)}
		=
		a
		,
	\end{equation}
	\begin{equation}
		q(y) = k\cdot q(x) + C
		=
		%\\=
		\frac{aq(x)-bq(x)+bp(x)-aq(x)}{p(x)-q(x)}
		=
		b
		.
	\end{equation}


\end{proof}

Факт, аналогичный теоремам~\ref{thm:Lin_Omega_Sucheston} и~\ref{thm:A_0_c_infty_lin}, верен и для
всего пространство $\ell_\infty$:
$\ell_\infty\subset \operatorname{Lin} X^a_b$, где
$X^a_b = \{x\in\ell_\infty : p(x) = a,~ q(x) = b\}$, $a>b$.


\begin{lemma}
	\label{lem:const_Lin_ell_infty}
	Справедливо включение
	$\mathbbm{1}\in \operatorname{Lin} X^a_b$.
\end{lemma}

\begin{proof}
	В самом деле,
	$Y^a_b \subset X^a_b$
	и, следовательно,
	\begin{equation}
		\mathbbm{1} \in \operatorname{Lin} Y^a_b \subset \operatorname{Lin} X^a_b
		.
	\end{equation}
\end{proof}

\begin{theorem}
	\label{thm:Lin_ell_infty}
	Справедливо равенство $\operatorname{Lin} X^a_b = \ell_\infty$.
\end{theorem}

\begin{proof}
	Зафиксируем $x \in \ell_\infty$ и представим его в виде линейной комбинации последовательностей из $X^a_b$.

	Не теряя общности, положим $x\geq 0$
	(иначе представим сначала $x$ в виде $x = y - z$, где $y \geq 0$, $z \geq 0$.
	и найдём представления для $y$ и $z$).

	Если $p(x) = q(x)$, то возьмём некоторый $y\in\ell_\infty$,
	такой, что $p(y) > p(x) = q(x)  \geq q(y) \geq 0$.
	Тогда в силу выпуклости функционала $p$ имеем
	\begin{equation}
		p(x+y) \geq p(y) > p(x) = q(x)
		,
	\end{equation}


	\begin{equation}
		q(x+y) = -p(-x-y) \leq -p(-x) -p(-y) = q(x) + q(y) \leq q(x) < p(x+y)
		,
	\end{equation}
	и задача сведена к отысканию представлений для $y$ и $x+y$.
	Таким образом, случай $p(x) = q(x)$ можно исключить,
	и, не теряя общности, рассматривать только такие $x$, что $p(x) > q(x)$.

	Снова, как и в доказательстве теоремы~\ref{thm:A_0_c_infty_lin},
	положим $y=k\cdot x + C\cdot\mathbbm{1}$,
	где $k=\frac{a-b}{p(x)-q(x)}$, $C=\frac{bp(x)-aq(x)}{p(x)-q(x)}$.
	Дальнейшее доказательство переносится дословно.
\end{proof}

%TODO: теорема с необходимыми (но не достаточными) условиями для того, чтобы множество было неразделяющим?
%TODO: а разделяющее ли множество Кантора? Нулевой/ненулевой меры?


\section{Свойства инвариантности пространства $A_0$}

\begin{theorem}
	\label{thm:alpha_sigma_n}
	Для любого $x\in\ell_\infty$ и для любого натурального $n$ верно равенство
	\begin{equation}
		\alpha(\sigma_n x) = \alpha(x)
		.
	\end{equation}
\end{theorem}

\begin{proof}
	По определению
	\begin{equation}
		\alpha(x) = \varlimsup_{i\to\infty} \max_{i<j\leqslant 2i} |x_i - x_j|
	\end{equation}

	Положим
	\begin{equation}
		\alpha_i(x) =
		\max_{i<j\leqslant 2i} |x_i - x_j| =
		\max_{i\leqslant j\leqslant 2i} |x_i - x_j|
	\end{equation}

	Тогда
	\begin{equation}
		\alpha(x) = \varlimsup_{i\to\infty} \alpha_i(x)
	\end{equation}

	Пусть $y = \sigma_n x$.
	Тогда для $k=1, ..., n-1$, $a\in\mathbb{N}$ имеем
	\begin{multline}
		\alpha_{an-k}(y) =
		\max_{an-k \leqslant j \leqslant 2an-2k} |y_{an-k} - y_j| =
		\\=
		(\mbox{т.к.}~y_{an-(n-1)}=y_{an-(n-2)}=...=y_{an-k}=...=y_{an-1}=y_{an})=
		\\=
		\max_{an \leqslant j \leqslant 2an-2k} |y_{an} - y_j| \leqslant
		\\ \leqslant
		(\mbox{переходим к максимуму по большему множеству}) \leqslant
		\\ \leqslant
		\max_{an \leqslant j \leqslant 2an} |y_{an} - y_j| =
		\alpha_{an}(y)
	\end{multline}

	С другой стороны,
	\begin{multline}
		\alpha_{an}(y) =
		\max_{an \leqslant j \leqslant 2an} |y_{an} - y_j| =
		\\ =
		(\mbox{т.к.}~y=\sigma_n x,~\mbox{можем рассматривать только}~j=kn)=
		\\ =
		\max_{an \leqslant kn \leqslant 2an} |y_{an} - y_{kn}| =
		\max_{a \leqslant k \leqslant 2a} |y_{an} - y_{kn}| =
		\max_{a \leqslant k \leqslant 2a} |x_a - x_k| =
		\alpha_a(x)
	\end{multline}

	Таким образом, для $k=1, ..., n-1$, $a\in\mathbb{N}$ имеем соотношения:
	\begin{gather}
		\alpha_{an}(y) = \alpha_a(x),
	\\
		\alpha_{an-k}(y) \leqslant \alpha_a(x),
	\end{gather}
	откуда немедленно следует, что
	\begin{equation}
		\varlimsup_{i\to\infty} \alpha_i(y) =
		\varlimsup_{i\to\infty} \alpha_i(x),
	\end{equation}
	т.е.
	\begin{equation}
		\alpha(\sigma_n x) = \alpha(x)
		.
	\end{equation}
\end{proof}

\begin{corollary}
	Пространство $A_0$ инвариантно относительно оператора $\sigma_n$,
	$n\in\mathbb{N}$,
	т.е. для любых $x\in A_0$ и $n\in\mathbb{N}$ выполнено $\sigma_n x\in A_0$.
\end{corollary}

Завершая обсуждение операторов растяжения $\sigma_n$, сделаем ещё одно замечание.
В статье \cite[lemma 16]{Semenov2010invariant} доказывается, что
\begin{equation}
	\sigma_2 C - C \sigma_2 : \ell_\infty \to c_0
	.
\end{equation}

\begin{corollary}
	$$
		\alpha(C\sigma_2 x) =
		\alpha(\sigma_2 Cx) =
		\alpha(Cx)
	$$
\end{corollary}

\begin{hypothesis}
	Для любого $n\in\mathbb{N}$
	$$
		\alpha(C\sigma_n x) =
		\alpha(\sigma_n Cx) =
		\alpha(Cx)
	$$
\end{hypothesis}

Перейдём теперь к операторам усредняющего сжатия.

Следуя~\cite[p. 131, prop. 2.b.2]{lindenstrauss1979classical},
будем рассматривать на $\ell_\infty$ оператор
\begin{equation}
	\sigma_{1/n} x = n^{-1}
	\left(
		\sum_{i=1}^{n} x_i,
		\sum_{i=n+1}^{2n} x_i,
		\sum_{i=2n+1}^{3n} x_i,
		...
	\right).
\end{equation}

Понятно, что если последовательность $x$~--- периодическая с периодом $n$,
то $\alpha(\sigma_{1/n}x)=0$.
Значит, оценить $\alpha(\sigma_{1/n}x)$ снизу через $\alpha(x)$ не удастся.

Для построения верхней оценки нам потребуется следующая

\begin{lemma}
	\label{thm:distance_from_average}
	Пусть $a=\frac{a_1+...+a_n}{n}$, $a_1 \leq ... \leq a_n$.
	Тогда $a_n - a \leq \frac{n-1}{n} (a_n - a_1)$.
\end{lemma}

\begin{proof}
	\begin{multline}
		a_n - a = a_n - \frac{a_1+...+a_n}{n}
		=
		\frac{n-1}{n}a_n - \frac{a_1+...+a_{n-1}}{n}
		\leq
		\\\leq
		\frac{n-1}{n}a_n - \frac{(n-1)a_1}{n}
		=
		\frac{n-1}{n}(a_n - a_1)
		.
	\end{multline}
\end{proof}

\begin{theorem}
	\label{thm:alpha_sigma_1_n}
	Для любого $n\in\mathbb{N}$ и любого $x\in\ell_\infty$ выполнено
	\begin{equation}
		\alpha(\sigma_{1/n} x) \leq \left( 2- \frac{1}{n} \right) \alpha(x)
		.
	\end{equation}
\end{theorem}

\begin{proof}
	Положим
	\begin{equation}
		\alpha_i(x) =
		\max_{i<j\leqslant 2i} |x_i - x_j| =
		\max_{i\leqslant j\leqslant 2i} |x_i - x_j|
		.
	\end{equation}
	Тогда
	\begin{equation}
		\alpha(x) = \varlimsup_{i\to\infty} \alpha_i(x)
		.
	\end{equation}
	Пусть $y=\sigma_n \sigma_{1/n} x$.
	Из теоремы~\ref{thm:alpha_sigma_n} следует, что $\alpha(\sigma_{1/n} x)=\alpha(y)$.
	Сосредоточим наши усилия на оценке $\alpha(y)$.

	Пусть $1\leq j \leq n$.
	Заметим, что
	\begin{multline}
		\alpha_{kn+j}(y)
		=
		\max_{kn+j \leq i \leq 2kn+2j } |y_{kn+j} - y_i|
		=
		\\=
		\mbox{(т.к. $y_{kn+j}=y_{kn+1}=y_{kn+n} = (\sigma_{1/n}x)_k$)}
		=
		\\=
		\max_{kn+n \leq i \leq 2kn+2j } |y_{kn+n} - y_i|
		\leq
		\\\leq
		\mbox{(переходим к максимуму по не меньшему множеству)}
		\leq
		\\\leq
		\max_{kn+n \leq i \leq 2kn+2n } |y_{kn+n} - y_i|
		=
		\alpha_{kn+n}(y)
		.
	\end{multline}

	Итак, $\alpha_{kn+j}(y) \leq \alpha_{kn+n}(y)$,
	значит,
	\begin{equation}
		\label{eq:alpha_sigma_1_n_subseq_limsup}
		\alpha(\sigma_{1/n}x) = \alpha(y) = \varlimsup_{i\to\infty} \alpha_i(y)
		=
		\varlimsup_{k\to\infty} \alpha_{kn+n}(y)
		.
	\end{equation}

	По лемме~\ref{thm:distance_from_average} имеем
	\begin{multline}
		\label{eq:alpha_sigma_1_n_distance}
		|x_{kn+n}-y_{kn+n}|
		\leq
		\frac{n-1}{n}\max_{1\leq i<j \leq n}|x_{kn+i}-x_{kn+j}|
		\leq
		\\\leq
		\frac{n-1}{n} \max_{1\leq i \leq n} \alpha_{kn+i}(x)
		=
		\frac{n-1}{n}\alpha_{kn+i_k}(x)
		.
	\end{multline}

	Из того, что $y_j = \frac{1}{n}(x_{kn+1}+...+x_{kn+n})$,
	следует, что
	\begin{equation}
		\label{eq:alpha_sigma_1_n_alpha_x}
		\max_{kn+n \leq i \leq 2kn+2n } |x_{kn+n} - y_i|
		\leq
		\max_{kn+n \leq i \leq 2kn+2n } |x_{kn+n} - x_i|
		=
		\alpha_{kn+n}(x)
		.
	\end{equation}

	Оценим:
	\begin{multline}
		\alpha_{kn+n}(y)
		=
		\max_{kn+n \leq i \leq 2kn+2n } |y_{kn+n} - y_i|
		=
		\\=
		\max_{kn+n \leq i \leq 2kn+2n } |y_{kn+n} - x_{kn+n} + x_{kn+n} - y_i|
		\leq
		\\\leq
		|y_{kn+n} - x_{kn+n}| + \max_{kn+n \leq i \leq 2kn+2n } |x_{kn+n} - y_i|
		\mathop{\leq}^{\eqref{eq:alpha_sigma_1_n_distance}}
		\\\leq
		\frac{n-1}{n} \alpha_{kn+i_k}(x) + \max_{kn+n \leq i \leq 2kn+2n } |x_{kn+n} - y_i|
		\mathop{\leq}^{\eqref{eq:alpha_sigma_1_n_alpha_x}}
		\frac{n-1}{n} \alpha_{kn+i_k}(x)+\alpha_{kn+n}(x)
		.
	\end{multline}

	С учётом~\eqref{eq:alpha_sigma_1_n_subseq_limsup} имеем
	\begin{multline}
		\alpha(\sigma_{1/n}x)
		=
		\varlimsup_{k\to\infty} \alpha_{kn+n}(y)
		\leq
		\\\leq
		\varlimsup_{k\to\infty} \left( \frac{n-1}{n} \alpha_{kn+i_k}(x)+\alpha_{kn+n}(x) \right)
		=
		\left(2-\frac{1}{n}\right)\alpha(x)
		.
	\end{multline}
\end{proof}

\begin{corollary}
	Пространство $A_0$ инвариантно относительно оператора $\sigma_{1/n}$,
	$n\in\mathbb{N}$,
	т.е. для любых $x\in A_0$ и $n\in\mathbb{N}$ выполнено $\sigma_{1/n} x\in A_0$.
\end{corollary}

Точность теоремы~\ref{thm:alpha_sigma_1_n} для $n=1$ очевидна.
Для $n=2$ её показывает
\begin{example}
	Положим для всех $p\in\mathbb{N}$:
	\begin{equation}
		x_k=\begin{cases}
			0, & k \leq 2^3, \\
			0, & k = 2^{3p}+1, \\
			1, & k = 2^{3p}+2, \\
			1, & 2^{3p}+3 \leq k \leq 2^{3p+1}+2, \\
			2, & 2^{3p+1}+3 \leq k \leq 2^{3p+1}+4, \\
			1, & 2^{3p+1}+5 \leq k \leq 2^{3(p+1)},
		\end{cases}
	\end{equation}
	тогда
	\begin{equation}
		(\sigma_{1/2}x)_k=\begin{cases}
			0, & k \leq 2^3, \\
			1/2, & k = 2^{3p}+1, \\
			1/2, & k = 2^{3p}+2, \\
			1, & 2^{3p}+3 \leq k \leq 2^{3p+1}+2, \\
			2, & 2^{3p+1}+3 \leq k \leq 2^{3p+1}+4, \\
			1, & 2^{3p+1}+5 \leq k \leq 2^{3(p+1)}.
		\end{cases}
	\end{equation}
	%\begin{table}
	%	\begin{tabular}{c||c|c|c|c|c|c|}
	%		\hline
	%		$k$     & $0..2^3$ & $2^{3p}+1$ & $2^{3p}+2$ & $2^{3p}+3 .. 2^{3p+1}+2$ & $2^{3p+1}+3 .. 2^{3p+1}+4 $ & $2^{3p+1}+5 .. 2^{3(p+1)}$ \\
	%	\end{tabular}
	%\end{table}
	Очевидно, что $\alpha(x)=1$, но $\alpha(\sigma_{1/2}x)=3/2$
	(достигается на $i=2^{3p}+2$, $j=2^{3p+1}+4$).
\end{example}

\begin{hypothesis}
	Оценка теоремы~\ref{thm:alpha_sigma_1_n} точна для любого $n\in\mathbb{N}$.
\end{hypothesis}



\section{Заключительные замечания}
Подмножество $\Omega$ нулевой меры, не являющееся разделяющим, сконструировать очень легко
(например, можно взять конечное множество или множество ортов).
Однако пока неясно, существует ли измеримое подмножество $\Omega$ ненулевой меры,
не являющееся разделяющим множеством.
Остаётся открытым и вопрос о том, какие множества (кроме $\Omega$, $A_0$ и самого $\ell_\infty$)
обладают свойством, аналогичным установленному в теоремах~\ref{thm:Lin_Omega_Sucheston},~\ref{thm:Lin_ell_infty}~и~\ref{thm:A_0_c_infty_lin}.

Автор выражает сердечную благодарность проф. Е.М. Семёнову за ценные cоветы и плодотворное обсуждение.

\printbibliography{}


Авдеев Николай Николаевич

Воронежский государственный университет, кафедра теории функций и геометрии, аспирант

Россия, 394006, г. Воронеж, Университетская пл., 1

E-mail: nickkolok@mail.ru, avdeev@math.vsu.ru


Avdeev Nikolai Nikolaevich

Voronezh State University, Department of Function Theory and Geometry, postgraduate student

Russia, 394006, Voronezh, University sq., 1

\end{document}
