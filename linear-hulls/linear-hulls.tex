\documentclass[a4paper,14pt]{article} %размер бумаги устанавливаем А4, шрифт 12пунктов
\usepackage[T2A]{fontenc}
\usepackage[utf8]{inputenc}
\usepackage[english,russian]{babel} %используем русский и английский языки с переносами
\usepackage{amssymb,amsfonts,amsmath,mathtext,enumerate,float,amsthm} %подключаем нужные пакеты расширений
\usepackage[unicode,colorlinks=true,citecolor=black,linkcolor=black]{hyperref}
%\usepackage[pdftex,unicode,colorlinks=true,linkcolor=blue]{hyperref}
\usepackage{indentfirst} % включить отступ у первого абзаца
\usepackage[dvips]{graphicx} %хотим вставлять рисунки?
\graphicspath{{illustr/}}%путь к рисункам

\makeatletter
\renewcommand{\@biblabel}[1]{#1.} % Заменяем библиографию с квадратных скобок на точку:
\makeatother %Смысл этих трёх строчек мне непонятен, но поверим "Запискам дебианщика"

\usepackage{geometry} % Меняем поля страницы.
\geometry{left=2cm}% левое поле
\geometry{right=1cm}% правое поле
\geometry{top=2cm}% верхнее поле
\geometry{bottom=2cm}% нижнее поле

\renewcommand{\theenumi}{\arabic{enumi}.}% Меняем везде перечисления на цифра.цифра
\renewcommand{\labelenumi}{\arabic{enumi}.}% Меняем везде перечисления на цифра.цифра
\renewcommand{\theenumii}{\arabic{enumii}}% Меняем везде перечисления на цифра.цифра
\renewcommand{\labelenumii}{\arabic{enumi}.\arabic{enumii}.}% Меняем везде перечисления на цифра.цифра
\renewcommand{\theenumiii}{\arabic{enumiii}}% Меняем везде перечисления на цифра.цифра
\renewcommand{\labelenumiii}{\arabic{enumi}.\arabic{enumii}.\arabic{enumiii}.}% Меняем везде перечисления на цифра.цифра

\sloppy




\renewcommand\normalsize{\fontsize{14}{25.2pt}\selectfont}

\usepackage[backend=biber,style=gost-numeric,sorting=none]{biblatex}
\addbibresource{../bib/ext.bib}
\addbibresource{../bib/my.bib}
\addbibresource{../bib/Semenov.bib}
\input{../bib/ext.hyphens.bib}


\theoremstyle{plain}
\newtheorem{lemma}{Лемма}[section]
\newtheorem{theorem}[lemma]{Теорема}
\newtheorem{example}[lemma]{Пример}
\newtheorem{property}[lemma]{Свойство}
\newtheorem{remark}[lemma]{Замечание}
\newtheorem{corollary}{Следствие}[lemma]

\usepackage{bbm}


\begin{document}

%\renewcommand{\bibname}{Список цитированной литературы}
%\renewcommand\refname{\bibname}
% !!!
% The text starts here

\title{
	О разделяющих множествах меры нуль и линейных оболочках
	\footnote{
		Работа выполнена в Воронежском университете при поддержке РНФ, грант 19-11-00197.
	}
}

\author{
	Н.Н. Авдеев
	\footnote{nickkolok@mail.ru, avdeev@math.vsu.ru}
}

\maketitle

\section{Введение}

Рассмотрим пространство ограниченных последовательностей с обычной нормой
\begin{equation*}
	\|x\| = \sup_{k} |x_k|
	.
\end{equation*}
Естественным обобщением предела с простанства сходящихся последовательностей $c$ на пространство $\ell_\infty$
является понятие банахова предела.
Банаховым пределом называется функционал $B\in \ell_\infty^*$ такой, что:
\begin{enumerate}
	\item
		$B \geqslant 0$
	\item
		$B\mathbbm{1} = 1$
	\item
		$B=BT$
\end{enumerate}
Здесь $\mathbbm{1}=(1,1,1,1,1,...)$,
а $T$~--- оператор сдвига: $T(x_1, x_2, x_3, ...) = (x_2, x_3, ...)$.
Множество всех банаховых пределов обозначим через $\mathfrak{B}$.

Сачестон~\cite{sucheston1967banach} установил, что
для любых $x\in l_\infty$ и $B\in\mathfrak{B}$
\begin{equation}\label{Sucheston}
	q(x) \leqslant Bx \leqslant p(x)
	,
\end{equation}
где
\begin{equation*}
	q(x) = \lim_{n\to\infty} \inf_{m\in\mathbb{N}}  \frac{1}{n} \sum_{k=m+1}^{m+n} x_k
	~~~~\mbox{и}~~~~
	p(x) = \lim_{n\to\infty} \sup_{m\in\mathbb{N}}  \frac{1}{n} \sum_{k=m+1}^{m+n} x_k
	.
\end{equation*}
называют нижним и верхним функционалом Сачестона соотвественно.
Неравенства \eqref{Sucheston} точны:
для данного $x$ для любого $r\in[q(x); p(x)]$ найдётся банахов предел
$B\in\mathfrak{B}$ такой, что $Bx = r$.

Множество таких $x\in\ell_\infty$, что $p(x)=q(x)$,
образует подпространство почти сходящихся последовательностей $ac$~~\cite{lorentz1948contribution}.
На почти сходящейся последовательности все банаховы пределы принимают одинаковые значения.

При исследовании банаховых пределов особый интерес представляют разделяющие множества~\cite[\S 3]{Semenov2014geomprops}.
Множество $Q\in\ell_\infty$ называют разделяющим, если
для любых неравных $B_1, B_2\in\mathfrak{B}$ существует такая последовательность $x\in Q$,
что $B_1 x \neq B_2 x$.
В частности, разделяющим является множество всех последовательностей из 0 и 1, т.е.
$\Omega=\{0;1\}^\mathbb{N}$~\cite{semenov2010characteristic}.
На множестве $\Omega$ естественным образом определяется мера (см., напр.,~\cite{connor1990almost}).
Оказывается, что из $\Omega$ можно выделить некоторые подмножества, которые также будут разделяющими,
например \cite[\S 3, Теорема 11]{Semenov2014geomprops},
\begin{equation}
	U = \{ x\in\Omega: q(x) = 0, p(x) = 1 \}
	.
\end{equation}

Однако множество $U$ имеет меру 1~\cite{semenov2010characteristic}.

В настоящей статье строится пример разделяющего множества,
являющегося подмножеством $\Omega$ и имеющего меру нуль.
Для построения таких множества используется следующий факт.

\begin{lemma}[{\cite[\S 3, замечание 6]{Semenov2014geomprops}}]
	Пусть $X$~--- разделяющее множество и $X \subset \operatorname{Lin} Y$,
	где $\operatorname{Lin} Y$ обозначает линейную оболочку $Y$.
	Тогда $Y$ также является разделяющим множеством.
\end{lemma}
Затем в статье обсуждаются свойства линейных оболочек множеств, определённых с помощью функционалов Сачестона.

\section{Разделяющее множество нулевой меры}

\begin{theorem}
	\label{thm:Lin_Omega_Sucheston}
	Пусть $\Omega = \{0;1\}^\mathbb{N}$,
	$1 \geq a > b \geq 0$ и
	$\Omega^a_b = \{x\in\Omega : p(x) = a, q(x) = b\}$,
	где $p(x)$ и $q(x)$~--- верхний и нижний функционалы Сачестона~\cite{sucheston1967banach} соответственно.
	Тогда $\Omega \subset \operatorname{Lin} \Omega^a_b$.
\end{theorem}

\begin{proof}
	Выберем $n\in\mathbb{N}$ таким образом, что
	\begin{equation}
		\label{eq:Omega_a_b_gap}
		a - b > \frac{3}{2^n}
	\end{equation}
	и $n$ чётно.
	Определим теперь двоичные приближения к числам $a$ и $b$:
	\begin{equation}
		\label{eq:binary_approximations_for_number}
		\frac{a_k}{2^k} \leq a < \frac{a_k+1}{2^k}
		,
		~~~a_k>0
		,
		~~~a_k\in\mathbb{Z}
		.
	\end{equation}
	(и аналогично для $b$).


	Теперь определим <<блоки>> из нулей и единиц, соответствующие числу $c\in[0;1]$ для $k \geq n$.
	Пусть определены $c_k$ аналогично~\eqref{eq:binary_approximations_for_number}.
	Тогда $\operatorname{Br}(k,c) \in \ell_\infty$.
	Дальнейшее определение построим рекурсивно.
	\begin{equation}
		(\operatorname{Br}(n,c))_j = \begin{cases}
			1, & \mbox{~если~} 2^n - c_n < j \leq 2^n,
			\\
			0  & \mbox{~иначе}
			.
		\end{cases}
	\end{equation}
	Заметим, что все элементы $\operatorname{Br}(n,c)$, начиная с $(2^n+1)$-го, равны нулю.

	Для каждого $k \geq n$ положим
	\begin{equation}
		\label{eq:Br(k+1,c)}
		\operatorname{Br}(k+1,c) = \operatorname{Br}(k,c) + T^{2^k}\operatorname{Br}(k,c) + (c_{k+1}-2c_k)e_{2^k+2^n-c_n}
		,
	\end{equation}
	где через $e_j$ обозначен $j$-й орт.
	Легко заметить (по индукции), что все элементы $\operatorname{Br}(k,c)$, начиная с $(2^k+1)$-го, равны нулю.
	Следовательно, носители первых двух слагаемых в~\eqref{eq:Br(k+1,c)} не пересекаются.
	Далее заметим, что третье слагаемое отлично от нуля тогда и только тогда,
	когда переход между приближениями $с_k / 2^k$ и $c_{k+1}/2^{k+1}$
	приводит к улучшению прилижения.
	Более того,
	\begin{equation}
		(\operatorname{Br}(k,c) + T^{2^k}\operatorname{Br}(k,c))_{2^k+2^n-c_n}
		=
		(\operatorname{Br}(k,c))_{2^n-c_n}
		=
		0
		,
	\end{equation}
	т.е. выражение~\eqref{eq:Br(k+1,c)} действительно задаёт последовательность из нулей и единиц.
	Таким образом, в $k$-м блоке ровно $c_k$ единиц.


	Нетрудно заметить, что для любых таких $m$, $k$ и $i$, что $n \leq m \leq k$ и  $ i + 2^m - 1 \leq 2^k$,
	выполнено
	\begin{equation}
		c_m \leq \sum_{j=i}^{i+2^m-1} (\operatorname{Br}(k,c))_j \leq c_{m}+1
		.
	\end{equation}


	Так, например, для $n=2$ и $c=1/3$ имеем:
	\begin{equation*}
		\begin{array}{ll}
			\operatorname{Br}(2,1/3) = (&0,0,0,1,0,0,...)
			\\
			\operatorname{Br}(3,1/3) = (&0,0,0,1,0,0,0,1,0,0,...)
			\\
			\operatorname{Br}(4,1/3) = (&0,0,0,1, 0,0,0,1,   0,0,1,1, 0,0,0,1,  0,0,...)
			\\
			\operatorname{Br}(5,1/3) = (&
			                             0,0,0,1, 0,0,0,1,   0,0,1,1, 0,0,0,1,\\
			                           & 0,0,0,1, 0,0,0,1,   0,0,1,1, 0,0,0,1,
			0,0,...)
			\\
			\operatorname{Br}(6,1/3) = (&
			                             0,0,0,1, 0,0,0,1,   0,0,1,1, 0,0,0,1,\\
			                           & 0,0,0,1, 0,0,0,1,   0,0,1,1, 0,0,0,1,\\
			                           & 0,0,1,1, 0,0,0,1,   0,0,1,1, 0,0,0,1,\\
			                           & 0,0,0,1, 0,0,0,1,   0,0,1,1, 0,0,0,1,
			0,0,...)
		\end{array}
	\end{equation*}

	Однако напомним, что нам нужно представление для произвольного $x\in\Omega$ в виде линейной комбинации.
	Очевидно, что существует разложение
	\begin{equation}
		x = \sum_{i=0}^{k-1} T^i x_i
		,
	\end{equation}
	где $k\in\mathbb{N}$~--- любое, $T$~--- оператор сдвига, а все элементы последовательностей $x_i$,
	кроме имеющих индексы $km+1$, $m\in\mathbb{N}_0$, являются нулевыми.
	Пусть $k=2^n$; зафиксируем $i$ и в дальнейшем для удобства записи положим $w=x_i$.
	Наша задача~--- построить конечную линейную комбинацию элементов из $\Omega^a_b$, равную $w$.

	Положим
	\begin{equation}
		v = T^{2^n} \operatorname{Br}(2^ n   ,a) + T^{2^{n+1}} \operatorname{Br}(2^{n+1},b) +
		T^{2^{n+2}} \operatorname{Br}(2^{n+2},a) + T^{2^{n+3}} \operatorname{Br}(2^{n+3},b) + ...
	\end{equation}
	Эта сумма является формальной, т.е. не сходится в смысле ряда по норме,
	однако носители слагаемых попарно не пересекаются.
	Иначе говоря,
	\begin{equation}
		v_j = \begin{cases}
			0,  & \mbox{~если~} j \leq 2^n,
			\\
			(\operatorname{Br}(2^{2k  },a))_{j-2^{2k  }},  & \mbox{~если~} 2^{2k  } < j \leq 2^{2k+1}, 2k   \geq n,
			\\
			(\operatorname{Br}(2^{2k+1},b))_{j-2^{2k+1}},  & \mbox{~если~} 2^{2k+1} < j \leq 2^{2k+2}, 2k+1 \geq n
			.
		\end{cases}
	\end{equation}
	Положим теперь
	\begin{equation}
		u_j = \begin{cases}
			v_j + w_j,  & \mbox{~если~} j \leq 2^n
			\\
			            & \mbox{~или~} 2^{4k+3} < j \leq 2^{4k+4} \mbox{~и~} 4k + 3 \geq n,
			\\
			v_j         & \mbox{~иначе.~}
		\end{cases}
	\end{equation}

	Заметим, что все элементы, к которым прибавляются ненулевые элементы $w_j$, равны нулю.
	Кроме того, $p(u)=p(w)=a$ и $q(u)=q(w)=b$.
	(На <<возмущённом>> блоке $u$ сумма, соответствующая функционалу $q$,
	увеличивается достаточно мало и не влияет на значение функционала $p$,
	в силу условия~\eqref{eq:Omega_a_b_gap}.)
	Следовательно, $u,v\in\Omega^a_b$.
	Заметим теперь, что
	\begin{equation}
		(u-v)_j = \begin{cases}
			w_j,  & \mbox{~если~} j \leq 2^n,
			\\
			0,  & \mbox{~если~} 2^{2k  } < j \leq 2^{2k+1}, 2k    \geq n,
			\\
			0,  & \mbox{~если~} 2^{4k+1} < j \leq 2^{4k+2}, 4k + 1 \geq n,
			\\
			w_j,  & \mbox{~если~} 2^{4k+3} < j \leq 2^{4k+4}, 4k + 3 \geq n
			.
		\end{cases}
	\end{equation}

	Аналогично строятся пары элементов, разность которых равна $w_j$ на $2^{4k+i} < j \leq 2^{4k+i+1}, 4k + i \geq n$ для $i=0,1,2$
	(требуется только обнулить первые $2^n$ элементов).
	Складывая полученные таким образом $4\cdot 2^n$ разностей элементов из $\Omega^a_b$, получаем требуемый элемент $x$.

\end{proof}

\begin{corollary}
	Множество $\Omega^a_b$ является разделяющим.
	Т.к. при $a\neq 1$ или $b\neq 0$ множество $\Omega^a_b$ имеет меру нуль~\cite{semenov2010characteristic},
	то оно является разделяющим множеством нулевой меры.
\end{corollary}

\begin{remark}
	Понятно, что неизмеримое разделяющее множество легко сконструировать,
	объединив разделяющее множество нулевой меры и произвольное неизмеримое множество.
	Для полноты изложения, однако, приведём другую конструкцию, являющуюся близким аналогом классического множества Лузина.
	%[TODO: ссылка?]
	% Множество не Лузина, а Витали?

	Две точки отрезка $x,y \in [0;1]$ назовём эквивалентными, если
	$x - y = a/2^k$, где $k\in\mathbb{N}$, $a\in\mathbb{Z}$.
	Множество $L$ составим, взяв из каждого класса эквивалентности по одному представителю.
	Тогда $L$, очевидно, неизмеримо (доказательство дословно повторяет доказательство неизмеримости множества Лузина),
	однако является разделяющим, т.к. для каждой последовательности из $\omega\in\Omega$
	найдётся элемент $\hat\omega\in L$, такой, что $\omega$ и $\hat\omega$ отличаются лишь конечным числом членов.
\end{remark}


\section{Линейные оболочки множеств, определяемых функционалами Сачестона}

Итак, $\Omega \subset \operatorname{Lin}\{x\in\Omega : p(x) = a,~ q(x) = b\}$, где $a>b$.
Оказывается, аналогичным свойством обладает и всё пространство $\ell_\infty$:
$\ell_\infty\subset \operatorname{Lin} X^a_b$, где
$X^a_b = \{x\in\ell_\infty : p(x) = a,~ q(x) = b\}$, $a>b$.

\begin{lemma}
	\label{lem:const_Lin_ell_infty}
	Справедливо включение
	$\mathbbm{1}\in \operatorname{Lin} X^a_b$.
\end{lemma}

\begin{proof}
	В самом деле,
	$\Omega^a_b \subset X^a_b$
	и, следовательно,
	\begin{equation}
		\mathbbm{1} \in \Omega \subset \operatorname{Lin} \Omega^a_b \subset \operatorname{Lin} X^a_b
		.
	\end{equation}
\end{proof}

\begin{theorem}
	\label{thm:Lin_ell_infty}
	Справедливо равенство $\operatorname{Lin} X^a_b = \ell_\infty$.
\end{theorem}

\begin{proof}
	Зафиксируем $x \in \ell_\infty$ и представим его в виде линейной комбинации последовательностей из $X^a_b$.

	Не теряя общности, положим $x\geq 0$
	(иначе представим сначала $x$ в виде $x = y - z$, где $y \geq 0$, $z \geq 0$.
	и найдём представления для $y$ и $z$).

	Если $p(x) = q(x)$, то возьмём некоторый $y\in\ell_\infty$,
	такой, что $p(y) > p(x) = q(x)  \geq q(y) \geq 0$.
	Тогда в силу выпуклости функционала $p$ имеем
	\begin{equation}
		p(x+y) \geq p(y) > p(x) = q(x)
		,
	\end{equation}


	\begin{equation}
		q(x+y) = -p(-x-y) \leq -p(-x) -p(-y) = q(x) + q(y) \leq q(x) < p(x+y)
		,
	\end{equation}
	и задача сведена к отысканию представлений для $y$ и $x+y$.
	Таким образом, случай $p(x) = q(x)$ можно исключить,
	и, не теряя общности, рассматривать только такие $x$, что $p(x) > q(x)$.

	Положим $y=k\cdot x + C\cdot\mathbbm{1}$,
	где $k=\frac{a-b}{p(x)-q(x)}$, $C=\frac{bp(x)-aq(x)}{p(x)-q(x)}$.
	Тогда, очевидно,
	\begin{equation}
		\label{eq:x_representation}
		x=(y-C\cdot\mathbbm{1})/k
		.
	\end{equation}
	Представление~\eqref{eq:x_representation} искомое.
	Действительно, в силу леммы~\ref{lem:const_Lin_ell_infty} выполнено
	$C\cdot\mathbbm{1}\in X^a_b$; кроме того,
	\begin{equation}
		p(y) = k\cdot p(x) + C
		=
		%\\=
		\frac{ap(x)-bp(x)+bp(x)-aq(x)}{p(x)-q(x)}
		=
		a
		,
	\end{equation}
	\begin{equation}
		q(y) = k\cdot q(x) + C
		=
		%\\=
		\frac{aq(x)-bq(x)+bp(x)-aq(x)}{p(x)-q(x)}
		=
		b
		.
	\end{equation}
\end{proof}

Факт, аналогичный теоремам~\ref{thm:Lin_Omega_Sucheston} и~\ref{thm:Lin_ell_infty}, верен и для подпространства
$A_0 c = \{ x \in \ell_\infty : \alpha(x) =0 \}$,
где~\cite{our-vzms-2018}
\begin{equation*}
	\alpha(x) = \varlimsup_{i\to\infty} \max_{i<j\leqslant 2i} |x_i - x_j|
	.
\end{equation*}
В~\cite{our-ped-2018-alpha-Tx} показано, что, хотя сама функция $\alpha(x)$ не инвариантна относительно оператора сдвига $T$,
пространство $A_0 c$ такой инвариантностью обладает.

Пусть $Y^a_b = \{x\in A_0 c : p(x) = a, q(x) = b\}$, где $a>b$.
Следуя той же схеме доказательства, что и для $\ell_\infty$,
подготовим сначала вспомогательные леммы о константе.

\begin{lemma}
	\label{lem:const_Lin_alpha_0}
	Пусть $a\neq -b$.
	Тогда справедливо включение
	$\mathbbm{1}\in \operatorname{Lin} Y^a_b$.
\end{lemma}

%TODO: доказательство не проходит при $a=-b$.
%Составить отдельное (есть в блокноте).

\begin{proof}
	Не теряя общности, будем полагать, что $b\geq 0$, тогда $a>0$.

	Определим оператор $S:\ell_\infty \to \ell_\infty$ следующим образом:
	\begin{equation}\label{operator_S}
		(Sy)_k = y_{i+2}, \mbox{ где } 2^i < k \leq 2^i+1
		.
	\end{equation}
	\begin{lemma}[{\cite{our-vzms-2018}}]
		Для любого $x\in \ell_\infty$ выполнено равенство
		\begin{equation}\label{alpha_S}
			\alpha(Sx) = \varlimsup_{k\to\infty} |x_{k+1} - x_{k}|
			.
		\end{equation}
	\end{lemma}
	Положим
	\begin{equation}
		\label{eq:y_for_s_alpha}
		y = \left(0,1,0,\frac{1}{2},1,\frac{1}{2},0,\frac{1}{3},\frac{2}{3},1,\frac{2}{3},\frac{1}{3},0,...\right)
		,
	\end{equation}
	тогда $Sy\in A_0 c$ и $\mathbbm{1}-Sy = S(\mathbbm{1}-y)\in A_0 c$.

	Пусть $x=(a-b)Sy+b\mathbbm{1}$, $z=(a-b)(\mathbbm{1}-Sy)+b\mathbbm{1}$.
	Тогда $p(x)=p(z)=a$, $q(x)=q(z)=b$ и, следовательно, $x,z\in Y^a_b$.
	Кроме того, заметим, что
	\begin{equation}
		x+z = (a-b)Sy+b\mathbbm{1} + (a-b)(\mathbbm{1}-Sy)+b\mathbbm{1}
		=
		(a-b)(Sy-Sy+\mathbbm{1}) + 2 b\mathbbm{1} = (a+b)\mathbbm{1}
		,
	\end{equation}
	откуда и следует, что $\mathbbm{1}\in Y^a_b$.
\end{proof}


\begin{lemma}
	\label{lem:const_Lin_alpha_0_a_eq_-b}
	Справедливо включение
	$\mathbbm{1}\in \operatorname{Lin} Y^a_{-a}$.
\end{lemma}

\begin{proof}
	Определим линейный оператор $M:\ell_\infty \to \ell_\infty$ следующим образом:
	\begin{multline}
		M\omega=\left(
			0, 1\omega_1,
			0, \frac{1}{2}\omega_2, 1\omega_2, \frac{1}{2}\omega_2,
			0, \frac{1}{3}\omega_3, \frac{2}{3}\omega_3, 1\omega_3, \frac{2}{3}\omega_3, \frac{1}{3}\omega_3,
			0, ...,
		\right. \\ \left.
			0, \frac{1}{p}\omega_p, \frac{2}{p}\omega_p, ..., \frac{p-1}{p}\omega_p, 1\omega_p,
				\frac{p-1}{p}\omega_p, ..., \frac{2}{p}\omega_p, \frac{1}{p}\omega_p,
			0, \frac{1}{p+1}\omega_{p+1}, ...
		\right)
		.
	\end{multline}
	Тогда $SM: \ell_\infty \to A_0 c$.

	Положим $x=aS(2M(\mathbbm{1})-\mathbbm{1})$,
	$y=-aS(2M(1,0,1,0,1,0,1,0,...)-\mathbbm{1})$,
	$z=-aS(2M(0,1,0,1,0,1,0,1,...)-\mathbbm{1})$.

	TODO: расписать первые несколько координат?

	Тогда, очевидно, каждая из последовательностей $x,y,z$ содержит отрезки сколь угодно большой длины,
	состоящие из $a$ (равно как и из $-a$), при этом $-a \leq x,y,z \leq a$.
	Следовательно, $p(x)=p(y)=p(z) = a$ и $q(x)=q(y)=q(z) = -a$,
	откуда $x,y,z \in Y^a_{-a}$.

	Заметим теперь, что
	\begin{multline}
		x + y + z
		=
		\\=
		aS(2M(\mathbbm{1})-\mathbbm{1}) - aS(2M(1,0,1,0,1,0,...)-\mathbbm{1}) - aS(2M(0,1,0,1,0,1,...)-\mathbbm{1})
		=
		\\=
		aS(2M(\mathbbm{1})-\mathbbm{1}  -    2M(1,0,1,0,1,0,...)+\mathbbm{1}  -    2M(0,1,0,1,0,1,...)+\mathbbm{1})
		=
		\\=
		aS(2M(\mathbbm{1}) - 2M(1,0,1,0,1,0,...) - 2M(0,1,0,1,0,1,...)+\mathbbm{1}+\mathbbm{1}-\mathbbm{1})
		=
		\\=
		aS(2M(\mathbbm{1}) - 2M(1,0,1,0,1,0,...) - 2M(0,1,0,1,0,1,...)+\mathbbm{1})
		=
		aS\mathbbm{1}
		=
		a\mathbbm{1}
		,
	\end{multline}
	откуда $\mathbbm{1} \in \operatorname{Lin} Y^a_{-a}$.
\end{proof}


\begin{lemma}
	\label{lem:c_0_Lin_alpha_0}
	Справедливо включение $c_0 \subset Y^a_b$.
\end{lemma}

\begin{proof}
	Зафиксируем $z\in c_0$.
	Определим $y$ согласно~\eqref{eq:y_for_s_alpha} и положим
	$x=(a-b)Sy+b\mathbbm{1}$.
	Тогда $x, x+z\in Y^a_b$ и, очевидно, $z=(x+z)-x$.
\end{proof}

\begin{theorem}[{\cite[следствие 2]{our-mz2019ac0}}]
	\label{thm:alpha_c_ac_c}
	Пусть $x\in ac$, т.е. $p(x) = q(x)$.
	Тогда $x\in c$ если и только если $\alpha(x) = 0$.
\end{theorem}

\begin{theorem}
	\label{thm:A_0_c_infty_lin}
	Пусть $a\neq -b$.
	Тогда справедливо равенство $\operatorname{Lin} Y^a_b = A_0 c$.
\end{theorem}

\begin{proof}
	Зафиксируем $x \in A_0 c$.

	Пусть сначала $p(x) = q(x)$.
	Тогда, согласно теореме~\ref{thm:alpha_c_ac_c}, $x\in c$
	и может быть представлен в виде суммы константы и последовательности из $c_0$.
	Утверждение теоремы следует из лемм~\ref{lem:const_Lin_alpha_0}, ~\ref{lem:const_Lin_alpha_0_a_eq_-b} и~\ref{lem:c_0_Lin_alpha_0}.

	Пусть теперь $p(x) > q(x)$.
	Снова, как и в доказательстве теоремы~\ref{thm:Lin_ell_infty},
	положим $y=k\cdot x + C\cdot\mathbbm{1}$,
	где $k=\frac{a-b}{p(x)-q(x)}$, $C=\frac{bp(x)-aq(x)}{p(x)-q(x)}$.
	Дальнейшее доказательство переносится дословно.
\end{proof}

%TODO: теорема с необходимыми (но не достаточными) условиями для того, чтобы множество было неразделяющим?
%TODO: а разделяющее ли множество Кантора? Нулевой/ненулевой меры?

\section{Заключительные замечания}
Подмножество $\Omega$ нулевой меры, не являющееся разделяющим, сконструировать очень легко
(например, можно взять конечное множество или множество ортов).
Однако пока неясно, существует ли измеримое подмножество $\Omega$ ненулевой меры,
не являющееся разделяющим множеством.
Остаётся открытым и вопрос о том, какие множества (кроме $\Omega$, $A_0 c$ и самого $\ell_\infty$)
обладают свойством, аналогичным установленному в теоремах~\ref{thm:Lin_Omega_Sucheston},~\ref{thm:Lin_ell_infty}~и~\ref{thm:A_0_c_infty_lin}.


\printbibliography

\end{document}
