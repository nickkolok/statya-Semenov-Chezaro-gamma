Введём, вслед за~\cite[p. 131, prop. 2.b.2]{lindenstrauss1979classical},
на $\ell_\infty$ оператор
\begin{equation}
	\sigma_{1/n} x = n^{-1}
	\left(
		\sum_{i=1}^{n} x_i,
		\sum_{i=n+1}^{2n} x_i,
		\sum_{i=2n+1}^{3n} x_i,
		...
	\right).
\end{equation}

Из критерия Лоренца немедленно следует
\begin{lemma}
	$\sigma_n \sigma_{1/n} - I : \ell_\infty \to ac_0$.
\end{lemma}

\begin{theorem}
	Для любого $n\in\mathbb{N}$ выполнено $\mathfrak{B}(\sigma_n) = \mathfrak{B}(\sigma_{1/n})$.
\end{theorem}

\begin{proof}
	Пусть сначала $B \in \mathfrak{B}(\sigma_{1/n})$.
	Тогда
	\begin{equation}
		B(x) = B(Ix) = B(\sigma_{1/n} \sigma_n x) = B(\sigma_n x)
		.
	\end{equation}
	Пусть теперь $B \in \mathfrak{B}(\sigma_n)$.
	Тогда
	\begin{equation}
		B(x) = B(Ix) = B((I+(\sigma_n \sigma_{1/n} - I)) x) = B(\sigma_n \sigma_{1/n} x) = B(\sigma_{1/n} x)
		.
	\end{equation}
\end{proof}

\begin{remark}
	Мы видим, что множество банаховых пределов, инвариантных относительно суперпозиции операторов,
	может быть шире, чем объединение множеств банаховых пределов,
	инвариантных относительно каждого из операторов:
	\begin{equation}
		\mathfrak{B}(\sigma_n) \cup \mathfrak{B}(\sigma_{1/n}) = \mathfrak{B}(\sigma_n) \subsetneq \mathfrak{B}(\sigma_{1/n}\sigma_n) = \mathfrak{B}(I)
		.
	\end{equation}
\end{remark}
